\section{Proofs For Rank-1 Case}\label{sec_proof_rank1}

In this section, we present the missing proofs from Section \ref{sec:rank1}.

\begin{proof}[Proof Sketch of Proposition~\ref{prop:rank1_Et}]
Using the update rule~\eqref{eqn:update-et} for $E_t$, we have that 
\begin{align}
	\normFro{E_{t+1}}^2
	&= \normFro{E_t}^2 - 2\eta \cdot \innerProduct{E_t}{\Pustar M_t U_t}
	+ \eta^2 \normFro{\Pustar M_t U_t}^2 \label{eq:rank1_Et_fro}
\end{align}
	When $\eta$ is sufficiently small and $\normFro{M_t}, \norm{U_t}$ are bounded from above, the third term on the RHS is negligible compared to the second term. Therefore, we focus on the second term that is linear in $\eta$. 
\begin{align}
	&\quad~ \innerProduct{E_t}{\Pustar M_t U_t} \nonumber \\
	&= \frac 1 m \sum_{i=1}^m \innerProduct{A_i}{U_t U_t^{\top} - \bXg} \innerProduct{ A_i}{\Pustar E_t U_t^{\top}} \label{eqn:12}
\end{align}
	where in the last line we rearrange the terms and use the fact that $\Pustar$ is symmetric.
	Now we use Lemma \ref{lem:RIP3} to show that equation~\eqref{eqn:12} is close to 
	$\innerProduct{U_t U_t^{\top} - \bXg}{\Pustar E_t U_t^{\top}}$, which is its expectation w.r.t the randomness of $A_i$'s if $A_i$'s were chosen from spherical Gaussian distribution.
	If $U_t U_t^{\top} - \bXg$ was a rank-1 matrix, then this would follow from Lemma \ref{lem:RIP3} directly. However, $U_tU_t^\top$ is approximately low-rank. Thus, we decompose it into a low-rank part and an error part with small trace norm as in equation~\eqref{eqn:decompose}. Since $U_t U_t^{\top} - \bXg - E_t E_t^{\top}$ has rank at most 4,  we can apply Lemma~\ref{lem:RIP3} to control the effect of $A_i$'s, 
	\begin{align}
& \frac 1 m \sum_{i=1}^m \innerProduct{A_i}{(U_t U_t^{\top} - \bXg - E_t E_t^{\top})} \innerProduct{A_i}{E_t U_t^{\top}} \nonumber\\
& \ge \inner{U_t U_t^{\top} - \bXg - E_t E_t^{\top}, E_tU_t^\top} - 	\delta \norm{U_t U_t^{\top} - \bXg - E_t E_t^{\top}}_F \norm{E_t U_t^{\top}}\nonumber \\
& \ge \inner{U_t U_t^{\top} - \bXg - E_t E_t^{\top}, E_tU_t^\top} - 	1.5\delta \norm{E_t U_t^{\top}}\label{eqn:100}
	\end{align}
	where the last inequality uses that 	$\normFro{U_t U_t^{\top} - \bXg - E_t E_t^{\top}}^2 = (1 - \norm{r_t}^2)^2 + 2\norm{E_t r_t}^2 \le 1 + \norm{E_t}^2 \norm{r_t}^2 \le 11/8$.
	
	For the $E_tE_t^\top$ term in the decomposition~\eqref{eqn:decompose}, by Lemma \ref{lem:RIP4}, we have that 
	\begin{align}
 	\frac 1 m \sum_{i=1}^m \innerProduct{A_i}{E_t E_t^{\top}} \innerProduct{A_i}{E_t U_t^{\top}} &\ge \innerProduct{E_t E_t^{\top}}{E_t U_t^{\top}} -\delta \norm{E_t E_t^{\top}}_\star\norm{E_tU_t^\top}	\nonumber \\
& 	\ge \innerProduct{E_t E_t^{\top}}{E_t U_t^{\top}} - 0.5\delta \norm{E_tU_t^\top}	\label{eqn:101}
 	\end{align}
 	Combining equation~\eqref{eqn:12}, ~\eqref{eqn:100} and~\eqref{eqn:101}, we conclude that 
 	\begin{align} \innerProduct{E_t}{\Pustar M_t U_t} 
	&\ge \innerProduct{U_t U_t^{\top} - \bXg}{E_t U_t^{\top}} - 2\delta \norm{E_t U_t^{\top}}\label{eqn:103}
	\end{align}
	Note that ${u^\star}^\top E_t = 0$, which implies that $\bXg E_t = 0$ and $U_t^\top E_t = E_t^\top E_t$. Therefore, 
	\begin{align*}
	\innerProduct{U_t U_t^{\top} - \bXg}{E_t U_t^{\top}} &  = \innerProduct{U_t U_t^{\top} }{E_t U_t^{\top}} \\
	& = \innerProduct{U_t^{\top} }{U_T^\top E_t U_t^{\top}} = \innerProduct{U_t^{\top} }{E_T^\top E_t U_t^{\top}} \\
	& = \innerProduct{E_tU_t^{\top} }{ E_t U_t^{\top}} = \norm{E_t U_t^{\top}}_F^2 \ge \norm{E_t U_t^{\top}}^2
	\end{align*}

We can also control the third term in RHS of equation~\eqref{eq:rank1_Et_fro} by 
$
\eta^2 \normFro{\Pustar M_t U_t}^2 \le 9\eta^2
$.
Since the bound here is less important (because one can always choose small enough $\eta$ to make this term dominated by the first order term), we left the details to the reader. Combining the equation above with~\eqref{eqn:103} and ~\eqref{eq:rank1_Et_fro}, we conclude the proof of equation~\eqref{eqn:etfro}. 
%
	Towards bounding the spectral norm of $E_{t+1}$, we use a similar technique to control the difference between $\Pustar M_t U_t$ and $\Pustar (U_t U_t^{\top} - \bXg) U_t$ in spectral norm. By decomposing $U_t U_t^{\top} - \bXg$ as in equation~\eqref{eqn:decompose} and applying Lemma \ref{lem:property_1} 
	and Lemma \ref{lem:property_2} respectively, we obtain that 
	\begin{align*}
		&\norm{\Pustar M_t U_t - \Pustar (U_t U_t^{\top} - \bXg) U_t} \nonumber\\
		& \le 
		4\delta\left(\normFro{U_t U_t^{\top} - \bXg - E_t E_t^{\top}} + \norm{E_tE_t^\top}_\star \right) \norm{U_t^{\top}}\nonumber\\
		& \le 8 \delta \norm{U_t} \le 8\delta (\norm{r_t} + \norm{E_t}) \tag{by the assumptions that $\norm{E_t}_F\le 1/2, \norm{r_t}\le 3/2$}
	\end{align*}
	Observing that $\Pustar (U_t U_t^{\top} - \bXg) U_t = E_tU_t^\top U_t$. Plugging the equation above into equation~\eqref{eqn:update-et}, we conclude that
	\begin{align*}
		\norm{E_{t+1}} &\le \norm{E_t (\Id - \eta U_t^{\top}U_t)} + 2\eta\delta (\norm{r_t} + \norm{E_t}) \\
		&\le \norm{E_t} + 2\eta\delta (\norm{r_t} + \norm{E_t})
	\end{align*}
\end{proof}


\begin{proof}[Proof of Proposition \ref{prop:rank1_Rt}]
By the update rule~\eqref{eqn:def-Ut}, we have that 
	\begin{align*}
		r_{t+1} &= U_{t+1}^{\top} \vecgt = U_t^{\top} (\Id -\eta M_t^{\top}) \vecgt \nonumber\\
			&= r_t - \eta U_t^{\top} M_t^{\top} \vecgt.
				\end{align*}
	By decomposing $M_t = \frac 1 m \sum_{i=1}^m \innerProduct{A_i}{U_t U_t^{\top} - \bXg} A_i$ as in equation \eqref{eqn:decompose},
	and then Lemma \ref{lem:property_1} and \ref{lem:property_2}, we obtain that
	\begin{align*}
		\bignorm{r_{t+1} - (r_t - \eta U_t^{\top} (U_t U_t^{\top} - \bXg) \vecgt)}
		\le \delta \left(\bignormFro{U_t U_t^{\top} - \bXg - E_t E_t^{\top}} + \normNuclear{E_t E_t^{\top}} \right) \norm{U_t}.
	\end{align*}
	Observe that $U_t^{\top} (U_t U_t^{\top} - \bXg) \vecgt = U_t^{\top} U_t r_t - r_t = (r_t r_t^{\top} + E_t^{\top} E_t) r_t - r_t
	= (\norm{r_t}^2 - 1) r_t - E_t^{\top} E_t r_t$.
	Also note that
	$\normFro{U_t U_t^{\top} - \bXg - E_t E_t^{\top}}^2 \le 11/8$ and $\normFro{E_t}^2 \le 1/4$, we have that
	\begin{align*}
		\norm{r_{t+1} - (1 + \eta(1 - \norm{r_t}^2)) r_t}
		\le \eta \norm{E_t^{\top}E_t r_t} +
		2\eta\delta \norm{U_t}
	\end{align*}
	Since $\norm{U_t} \le \norm{r_t} + \norm{E_t}$,
	we obtain the conclusion.
\end{proof}




\begin{proof}[Proof Sketch of Proposition \ref{prop:induction}]
	We will analyze the dynamics of $r_t$ and $E_t$ in two stages.
	The first stage consists of all the steps such that $\Norm{r_t}\le 1/2$,
	and the second stage consists of the rest of steps.  We will show that
	\begin{enumerate}
		\item[a)] Stage 1 has at most $O(\log(\frac 1 {\alpha})/\eta)$ steps.
		Throughout Stage 1, we have
		\begin{align}
			&	\|E_t\|\le 9\delta \label{eqn:er} \\
			&	\norm{r_{t+1}}\ge (1+\eta/3)\norm{r_t} \label{eqn:rr}
		\end{align}
		\item[b)] In Stage 2, we have that 
		\begin{align*}
			& \Norm{E_t}^2_F \lesssim \delta^2\log{\frac 1{\delta}} \\
			& \Norm{r_t}\le 1+O(\delta).
		\end{align*}
		And after at most $O(\log(\frac 1 {\delta})/\eta)$ steps in Stage 2, we have $\Norm{r_t}\ge 1-O(\delta)$. 
	\end{enumerate}

	We use induction with Proposition~\ref{prop:rank1_Et} and ~\ref{prop:rank1_Rt}. For $t=0$, we have that $\|E_0\| = \norm{r_0} = \alpha$ because $U_0 = \alpha B$, where $B$ is an orthonormal matrix. Suppose equation~\eqref{eqn:er} is true for some $t$, then we can prove equation~\eqref{eqn:rr} holds by Proposition~\ref{prop:rank1_Rt}:
	\begin{align*}
		\bignorm{r_{t+1}} & \ge   (1 + \eta (1 - \norm{r_t}^2 - 2\delta - O(\delta^2)))\norm{ r_t} \\
		& \ge (1+\eta/3)\norm{r_t}	\tag{by $\delta \le 0.01$ and $\Norm{r_t}\le 1/2$}
	\end{align*}
	Suppose equation~\eqref{eqn:rr} holds, we can prove equation~\eqref{eqn:er} holds by Proposition~\ref{prop:rank1_Et}.
	We first observe that $t \le O(\log{\frac 1 {\alpha}} / \eta)$,
	since $r_t$ grows by a rate of $1 + \frac{\eta} 3$.
	Denote by $\lambda = 1 + 2\eta\delta$. We have
	\begin{align*}
		& \norm{E_{t+1}} \le \lambda \norm{E_t} + 2\eta\delta \norm{r_t}
		\Rightarrow \frac {\norm{E_{t+1}}} {{\lambda}^{t+1}} \le \frac{\norm{E_t}}{{\lambda}^t} + 2\eta\delta \times \left(\frac{\norm{r_t}} {{\lambda}^{t+1}} \right) \\
		\Rightarrow & \norm{E_{t+1}} \le {\lambda}^{t+1} \times \alpha +
		2\eta\delta \times \sum_{i=0}^t \frac{\lambda^i \norm{r_t}}{(1+\frac{\eta}3)^i} \tag{by $\norm{r_{t+1}} \ge \norm{r_t} (1 + \frac {\eta} 3)$.}
		\le 9\delta,
	\end{align*}
	where the last inequality uses that
	\begin{align*}
		& \lambda^{t+1} \alpha \le \alpha \times \exp(2\eta\delta \times O(\log(\frac 1 {\alpha})) / \eta) = \alpha^{1 - O(\delta)} \le o(\delta), \mbox{ and} \\
		& \sum_{i=1}^t \frac{\lambda^i \norm{r_t}}{(1+\frac{\eta}3)^i} \le \frac{1 + \frac{\eta}3}{2 (\frac{\eta}3 - 2\eta\delta)}
	\end{align*}

For claim b), we first apply the bound obtained in claim a) on $\norm{E_t}$ and the fact that $\norm{U_t} \le 3/2$
(this follows trivially from our induction, so we omit the details).
We have already proved that $\norm{E_t} \le 9\delta$ in the first stage.
For the second stage, as long as the number of iterations is less than $O(\log(\frac 1 {\delta}) / \eta)$,
we can still obtain via Proposition \ref{prop:rank1_Et} that:
\[ \norm{E_t} \le 9\delta + 4\eta\delta \times O(\log(\frac 1 {\delta} / \eta))
= O(\delta \log(\frac 1 {\delta})) \tag{since $\norm{E_t} \le 1/2$ and $\norm{r_t} \le 3/2$} \]
To summarize, we bound $\normFro{E_t}$ as follows:
	\begin{align*}
		\normFro{E_{t}}^2 & \le \normFro{E_0}^2 + 9\eta^2 t + O(\eta t \delta^2 \log(\frac 1 {\delta})) \lesssim \delta^2\log(\frac 1{\delta}),
	\end{align*}
	because $\normFro{E_0}^2 = \alpha^2 d \le \delta^2 \log{\frac 1 {\delta}}$,
	$\eta t \le O(1)$,
	and	$\eta^2 t \le O(\eta) \le O(\delta^2 \log(\frac 1{\delta}))$.
		For the bound on $\norm{r_t}$, we note that since $\norm{E_t} \le O(\delta \log(\frac 1 {\delta}))$, we can simplify Proposition~\ref{prop:rank1_Rt} by 
\begin{align*}
	\bignorm{r_{t+1} -  (1 + \eta (1 - \norm{r_t}^2))r_t}
		\le 2\eta\delta\norm{r_t} + O(\eta\delta^2\log^2(\frac 1 {\delta})).
\end{align*}
	The proof that $\norm{r_t} = 1 \pm O(\delta)$ after at most $O(\log(\frac 1{\delta}) / \eta)$ steps follows similarly by induction.
	The details are left to the readers.
\end{proof}

