% \documentclass[anon,12pt]{colt2018} % Anonymized submission
\documentclass[final]{colt2018} % Include author names

% The following packages will be automatically loaded:
% amsmath, amssymb, natbib, graphicx, url, algorithm2e

\usepackage[utf8]{inputenc} % allow utf-8 input
\usepackage[T1]{fontenc}    % use 8-bit T1 fonts
\usepackage{hyperref}       % hyperlinks
\usepackage{url}            % simple URL typesetting
\usepackage{booktabs}       % professional-quality tables
\usepackage{amsfonts}       % blackboard math symbols
\usepackage{nicefrac}       % compact symbols for 1/2, etc.
\usepackage{microtype}      % microtypography
\usepackage{transparent}

% \usepackage{amsmath}
%\usepackage{amsthm}
% \usepackage{amssymb}
\usepackage{mathtools}
%\usepackage[usenames]{xcolor}

%\usepackage[ruled,linesnumbered]{algorithm2e}

\usepackage{enumitem}
\setitemize{noitemsep,topsep=0pt,parsep=0pt,partopsep=0pt}

\definecolor{darkblue}{rgb}{0.0,0.0,0.55}
\hypersetup{
	colorlinks = true,
	citecolor  = darkblue,
	linkcolor  = darkblue,
	citecolor  = darkblue,
	filecolor  = darkblue,
	urlcolor   = darkblue,
}


\usepackage{enumitem}
\usepackage{xspace}

\usepackage{wrapfig}

%\newtheorem{lemma}{Lemma}
%\newtheorem{theorem}{Theorem}
%\newtheorem{corollary}{Corollary}
%\newtheorem{proposition}{Proposition}
%\newtheorem{remark}{Remark}
%\newtheorem{definition}{Definition}
\newtheorem{claim}{Claim}
\newtheorem{assumption}{Assumption}

\newcommand{\Exp}{\mathrm{Exp}}
\newcommand{\arccosh}{\mathrm{arccosh}}
\newcommand\numberthis{\addtocounter{equation}{1}\tag{\theequation}}
\newcommand{\ragd}{\textsc{Ragd}}
\newcommand{\inj}{\mathrm{inj}}
\renewcommand{\nabla}{\mathrm{grad}}

\title[An Estimate Sequence for Geodesically Convex Optimization]{An Estimate Sequence for Geodesically Convex Optimization}
\usepackage{times}
 % Use \Name{Author Name} to specify the name.
 % If the surname contains spaces, enclose the surname
 % in braces, e.g. \Name{John {Smith Jones}} similarly
 % if the name has a "von" part, e.g \Name{Jane {de Winter}}.
 % If the first letter in the forenames is a diacritic
 % enclose the diacritic in braces, e.g. \Name{{\'E}louise Smith}

 % Two authors with the same address
  % \coltauthor{\Name{Author Name1} \Email{abc@sample.com}\and
  %  \Name{Author Name2} \Email{xyz@sample.com}\\
  %  \addr Address}

 % Three or more authors with the same address:
 % \coltauthor{\Name{Author Name1} \Email{an1@sample.com}\\
 %  \Name{Author Name2} \Email{an2@sample.com}\\
 %  \Name{Author Name3} \Email{an3@sample.com}\\
 %  \addr Address}


 % Authors with different addresses:
 \coltauthor{\Name{Hongyi Zhang} \Email{hongyiz@mit.edu}\\
 \addr BCS and LIDS, Massachusetts Institute of Technology
 \AND
 \Name{Suvrit Sra} \Email{suvrit@mit.edu}\\
 \addr EECS and LIDS, Massachusetts Institute of Technology
 }

\begin{document}

\maketitle

\begin{abstract}
We propose a Riemannian version of Nesterov's Accelerated Gradient algorithm (\ragd), and show that for \emph{geodesically} smooth and strongly convex problems, within a neighborhood of the minimizer whose radius depends on the condition number as well as the sectional curvature of the manifold, \ragd{} converges to the minimizer with acceleration. Unlike the algorithm in \citep{liu2017accelerated} that requires the exact solution to a nonlinear equation which in turn may be intractable, our algorithm is constructive and computationally tractable\footnote{ as long as Riemannian gradient, exponential map and its inverse are computationally tractable, which is the case for many matrix manifolds \citep{absil2009optimization}.}. Our proof exploits a new estimate sequence and a novel bound on the nonlinear metric distortion, both ideas may be of independent interest.
\end{abstract}

\begin{keywords}
	Riemannian optimization; geodesically convex optimization; Nesterov's accelerated gradient method; nonlinear optimization
\end{keywords}

Online learning algorithms are a key tool in web search and content optimization, adaptively learning what users want to see. In a typical application, each time a user arrives, the algorithm chooses among various content presentation options (\eg news articles to display), the chosen content is presented to the user, and an outcome (\eg a click) is observed. Such algorithms must balance \emph{exploration} (making potentially suboptimal decisions now for the sake of acquiring information that will improve decisions in the future) and \emph{exploitation} (using information collected in the past to make better decisions now). Exploration could degrade the experience of a current user, but improves user experience in the long run. This exploration-exploitation tradeoff is commonly studied in the online learning framework of \emph{multi-armed bandits}~\citep{Bubeck-survey12}.

Concerns have been raised about whether exploration in such scenarios could be unfair, in the sense that some individuals or groups may experience too much of the downside of exploration without sufficient upside \citep{bird2016exploring}. We formally study these concerns in the \emph{linear contextual bandits} model~\citep{Langford-www10,chu2011contextual}, a standard variant of multi-armed bandits appropriate for content personalization scenarios.  We focus on \emph{externalities} arising due to exploration, that is, undesirable side effects that the presence of one party may impose on another.


We first examine the effects of exploration at a group level.  We introduce the notion of a \emph{group externality} in an online learning system, quantifying how much the presence of one population (which we dub the majority) impacts the rewards of another (the minority). We show that this impact can be negative, and that, in a particular precise sense, no algorithm can avoid it. This cannot be explained by the absence of suitably good policies since our adoption of the linear contextual bandits framework implies the existence of a feasible policy that is simultaneously optimal for everyone. Instead, the problem is inherent to the process of exploration. We come to a surprising conclusion that more data can sometimes lead to worse outcomes for the users of an explore-exploit-based system. \looseness=-1

We next turn to the effect of exploration at an individual level. We interpret exploration as a potential externality imposed on the current user by future users of the system. Indeed, it is only for the sake of the future users that the algorithm would forego the action that currently looks optimal. To avoid this externality, one may use the greedy algorithm that always chooses the action that appears optimal according to current estimates of the problem parameters. While this greedy algorithm performs poorly in the worst case,
it tends to work well in many applications and experiments.\footnote{Both positive and negative findings are folklore. One way to precisely state the negative result is that the greedy algorithm incurs constant per-round regret with constant probability; while results of this form have likely been known for decades,
\citet[Corollary A.2(b)]{competingBandits-itcs16}
proved this for a wide variety of scenarios. Very recently, the good empirical performance has been confirmed by state-of-art experiments in \citet{practicalCB-arxiv18}.}

In a new line of work, \citet{bastani2017exploiting} and \citet{kannan2018smoothed}
analyzed conditions under which inherent diversity in the data makes explicit exploration unnecessary.
\citet{kannan2018smoothed} proved that the greedy algorithm achieves a regret rate of
$\tilde{O}(\sqrt{T})$ in expectation over small perturbations of the context vectors (which ensure sufficient data diversity). This is the best rate that can be achieved in the worst case (\ie for all problem instances, without data diversity assumptions), but it leaves open the possibilities that (i) another algorithm may perform much better than the greedy algorithm on some problem instances, or (ii) the greedy algorithm may perform much better than worst case under the diversity conditions. We expand on this line of work. We prove that under the same diversity conditions, the greedy algorithm almost matches the best possible Bayesian regret rate of \emph{any} algorithm \emph{on the same problem instance}. This could be as low as $\polylog(T)$ for some instances, and, as we prove, at most $\tilde{O}(T^{1/3})$ whenever the diversity conditions hold.


Returning to group-level effects, we show that under the same diversity conditions, the negative group externalities imposed by the majority essentially vanish if one runs the greedy algorithm. Together, our results illustrate a sharp contrast between the high individual and group externalities that exist in the worst case, and the ability to remove all externalities if the data is sufficiently diverse.   \looseness=-1

\xhdr{Additional motivation.} Whether and when explicit exploration is necessary is an important concern in the study of the exploration-exploitation tradeoff. Fairness considerations aside, explicit exploration is expensive. It is wasteful and risky in the short term, it adds a layer of complexity to algorithm design \citep{Langford-nips07,monster-icml14}, and its adoption at scale tends to require substantial systems support and buy-in from management \citep{MWT-WhitePaper-2016,DS-arxiv}. A system based on the greedy algorithm would typically be cheaper to design and deploy.

Further, explicit exploration can run into incentive issues in applications such as recommender systems. Essentially, when it is up to the users which products or experiences to choose and the algorithm can only issue recommendations and ratings, an explore-exploit algorithm needs to provide incentives to explore for the sake of the future users \citep{Kremer-JPE14,Frazier-ec14,Che-13,ICexploration-ec15,Bimpikis-exploration-ms17}. Such incentive guarantees tend to come at the cost of decreased performance, and rely on assumptions about human behavior. The greedy algorithm avoids this problem as it is inherently consistent with the users' incentives.



\xhdr{Additional related work.}
Our research draws inspiration from the growing body of work on fairness in machine learning~\cite[e.g.,][]{dwork2012fairness,hardt2016equality,kleinberg2017inherent,chouldechova2017fair}.  Several other authors have studied fairness in the context of the contextual bandits framework.  Our work differs from the line of research on meritocratic fairness in online learning \citep{kearns2017meritocratic,liu2017calibrated,joseph2016fairness}, which considers the allocation of limited resources such as bank loans and requires that nobody should be passed over in favor of a less qualified applicant. We study a fundamentally different scenario in which there are no allocation constraints and we would like to serve each user the best content possible.  Our work also differs from that of \citet{celis2017fair}, who studied an alternative notion of fairness in the context of news recommendations. According to this notion, all users should have approximately the same probability of seeing a particular type of content (e.g., Republican-leaning articles), regardless of their individual preferences, in order to mitigate the possibility of discriminatory personalization.

The data diversity conditions in \citet{kannan2018smoothed} and this paper are inspired by the smoothed analysis framework of \citet{SmoothedAnalysis-jacm04}, who proved that the expected running time of the simplex algorithm is polynomial for perturbations of any initial problem instance (whereas the worst-case running time has long been known to be exponential). Such disparity implies that very bad problem instances are brittle. 
We find a similar disparity for the greedy algorithm in our setting.



\xhdr{Our results on group externalities.}  A typical goal in online learning is to minimize \emph{regret}, the (expected) difference between the cumulative reward that would have been obtained had the optimal policy been followed at every round and the cumulative reward obtained by the algorithm.  We define a corresponding notion of \emph{minority regret}, the portion of the regret experienced by the minority.  Since online learning algorithms update their behavior based on the history of their observations, minority regret is influenced by the entire population on which an algorithm is run.  If the minority regret is much higher when a particular algorithm is run on the full population than it is when the same algorithm is run on the minority alone, we can view the majority as imposing a negative externality on the minority; the minority population would achieve a higher cumulative reward if the majority were not present. Asking whether this can ever happen
amounts to asking whether access to more data points can ever lead an explore-exploit algorithm to make inferior decisions. One might think that more data should always lead to better decisions and therefore better outcomes for the users.
Surprisingly, we show that this is not the case, even with a standard algorithm.

Consider LinUCB~\citep{Langford-www10,chu2011contextual,abbasi2011improved}, a standard algorithm for linear contextual bandits that is based on the principle of ``optimism under uncertainty.''  We provide a specific problem instance on which, after observing $T$ users, LinUCB would have a minority regret of $\Omega(\sqrt T)$ if run on the full population, but only constant minority regret if run on the minority alone. While stylized, this example is motivated by the problem of providing driving directions to different populations of users, about which fairness concerns have been raised~\citep{bird2016exploring}. Further, the situation is reversed on a slight variation of this example: LinUCB obtains constant minority regret when run on the full population and $\Omega(\sqrt T)$ on the minority alone.  That is, group externalities can be large and positive in some cases, and large and negative in others.

Although these regret rates are specific to LinUCB, we show that this phenomenon is, in some sense, unavoidable. Consider the minority regret of LinUCB when run on the full population and the minority regret that LinUCB would incur if run on the minority alone. We know that one may be much smaller or larger than the other. We ask whether any algorithm could  achieve the minimum of the two on every problem instance. Using a variation of the same problem instance, we prove that this is impossible; in fact, no algorithm could simultaneously approximate both up to any $o(\sqrt{T})$ factor. In other words, an externality-free algorithm would sometimes ``leave money on the table."


In terms of techniques, we rely on the special structure of our example, which can be viewed as an instance of the sleeping bandits problem~\citep{SleepingBandits-ml10}. This simplifies the behavior and analysis of LinUCB, allowing us to obtain the $O(1)$ upper bounds.  The lower bounds are obtained using KL-divergence techniques to show that the two variants of our example are essentially indistinguishable, and an algorithm that performs well on one must obtain $\Omega(\sqrt{T})$ regret on the other. \looseness=-1


\xhdr{Our results on the greedy algorithm.} We consider a version of linear contextual bandits in which the latent weight vector $\theta$ is drawn from a known prior. In each round, an algorithm is presented several actions to choose from, each represented by a \emph{context vector}. The expected reward of an action is a linear product of $\theta$ and the corresponding context vector. The tuple of context vectors is drawn independently from a fixed distribution. In the spirit of smoothed analysis, we assume that this distribution has a small amount of jitter. Formally, the tuple of context vectors is drawn from some fixed distribution, and then a small \emph{perturbation}---small-variance Gaussian noise---is added independently to each coordinate of each context vector. This ensures arriving contexts are diverse. We are interested in Bayesian regret, i.e., regret in expectation over the Bayesian prior. Following the literature, we are primarily interested in the dependence on the time horizon $T$. \looseness=-1

We focus on a batched version of the greedy algorithm, in which new data arrives to the algorithm's optimization routine in small batches, rather than every round. This is well-motivated from a practical perspective---in high-volume applications data usually arrives to the ``learner" only after a substantial delay \citep{MWT-WhitePaper-2016,DS-arxiv}---and is essential for our analysis.

Our main result is that the greedy algorithm matches the Bayesian regret of any algorithm up to polylogarithmic factors, for each problem instance, fixing the Bayesian prior and the context distribution. We also prove that LinUCB achieves regret $\tilde{O}(T^{1/3})$ for each realization of $\theta$. This implies a worst-case Bayesian regret of $\tilde{O}(T^{1/3})$ for the greedy algorithm under the perturbation assumption. \looseness=-1

Our results hold for both natural versions of the batched greedy algorithm, Bayesian and frequentist, henceforth called \BayesGreedy and \FreqGreedy. In \BayesGreedy, the chosen action maximizes expected reward according to the Bayesian posterior. \FreqGreedy estimates $\theta$ using ordinary least squares regression and chooses the best action according to this estimate. The results for \FreqGreedy come with additive polylogarithmic factors, but are stronger in that the algorithm does not need to know the prior. Further, the $\tilde{O}(T^{1/3})$ regret bound for \FreqGreedy is approximately prior-independent, in the sense that it applies even to very concentrated priors such as independent Gaussians with standard deviation on the order of $T^{-2/3}$.

The key insight in our analysis of \BayesGreedy is that any (perturbed) data can be used to simulate any other data, with some discount factor. The analysis of \FreqGreedy requires an additional layer of complexity. We consider a hypothetical algorithm that receives the same data as \FreqGreedy, but chooses actions based on the Bayesian-greedy selection rule. We analyze this hypothetical algorithm using the same technique as \BayesGreedy, and then upper bound the difference in Bayesian regret between the hypothetical algorithm and \FreqGreedy.

Our analyses extend to group externalities and (Bayesian) minority regret. In particular, we circumvent the impossibility result mentioned above. We prove that both \BayesGreedy and \FreqGreedy match the Bayesian minority regret of any algorithm run on either the full population or the minority alone, up to polylogarithmic factors

\xhdr{Detailed comparison with prior work.} We substantially improve over the $\tilde{O}(\sqrt{T})$ worst-case regret bound from \citet{kannan2018smoothed}, at the cost of some additional assumptions. First, we consider Bayesian regret, whereas their regret bound is for each realization of $\theta$.%
\footnote{Equivalently, they allow point priors, whereas our priors must have variance $T^{-O(1)}$.} Second, they allow the context vectors to be chosen by an adversary before the perturbation is applied. Third, they extend their analysis to a somewhat more general model, in which there is a separate latent weight vector for every action (which amounts to a more restrictive model of perturbations). However, this extension relies on the greedy algorithm being initialized with a substantial amount of data. The results of \citet{kannan2018smoothed} do not appear to have implications on group externalities.

\citet{bastani2017exploiting} show that the greedy algorithm achieves logarithmic regret in an alternative linear contextual bandits setting that is incomparable to ours in several important ways.
They consider two-action instances where the actions share a common context vector in each round, but are parameterized by different latent vectors. They ensure data diversity via a strong assumption on the context distribution. This assumption does not follow from our perturbation conditions; among other things, it implies that each action is the best action in a constant fraction of rounds. Further, they assume a version of Tsybakov's \emph{margin condition}, which is known to substantially reduce regret rates in bandit problems \citep[\eg see][]{Zeevi-colt10}.


\section{Background}


We briefly review concepts in Riemannian geometry that are related to our analysis; for a thorough introduction one standard text is~\citep[e.g.][]{jost2011riemannian}. A \emph{Riemannian manifold} $(\mathcal{M}, \mathfrak{g})$ is a real smooth manifold $\mathcal{M}$ equipped with a Riemannain metric $\mathfrak{g}$. The metric $\mathfrak{g}$ induces an inner product structure on each tangent space $T_x\mathcal{M}$ associated with every $x\in\mathcal{M}$.  We denote the inner product of $u,v\in T_x\mathcal{M}$ as $\langle u, v \rangle \triangleq \mathfrak{g}_x(u,v)$; and the norm of $u\in T_x\mathcal{M}$ is defined as $\|u\|_x \triangleq \sqrt{\mathfrak{g}_x(u,u)}$; we omit the index $x$ for brevity wherever it is obvious from the context. The angle between $u,v$ is defined as $\arccos\frac{\langle u, v \rangle}{\|u\|\|v\|}$. A geodesic is a constant speed curve $\gamma: [0,1]\to\mathcal{M}$ that is locally distance minimizing. An exponential map $\Exp_x:T_x\mathcal{M}\to\mathcal{M}$ maps $v$ in $T_x\mathcal{M}$ to $y$ on $\mathcal{M}$, such that there is a geodesic $\gamma$ with $\gamma(0) = x, \gamma(1) = y$ and $\dot{\gamma}(0) \triangleq \frac{d}{dt}\gamma(0) = v$.  If between any two points in $\mathcal{X}\subset\mathcal{M}$ there is a unique geodesic, the exponential map has an inverse $\Exp_x^{-1}:\mathcal{X}\to T_x\mathcal{M}$ and the geodesic is the unique shortest path with $\|\Exp_x^{-1}(y)\| = \|\Exp_y^{-1}(x)\|$ the geodesic distance between $x,y\in\mathcal{X}$. Parallel transport is the Riemannian analogy of vector translation, induced by the Riemannian metric.

Let $u,v\in T_x\mathcal{M}$ be linearly independent, so that they span a two dimensional subspace of $T_x\mathcal{M}$. Under the exponential map, this subspace is mapped to a two dimensional submanifold of $\mathcal{U}\subset\mathcal{M}$. The sectional curvature $\kappa(x,\mathcal{U})$ is defined as the Gauss curvature of $\mathcal{U}$ at $x$, and is a critical concept in the comparison theorems involving geodesic triangles \citep{burago2001course}.

The notion of geodesically convex sets, geodesically (strongly) convex functions and geodesically smooth functions are defined as straightforward generalizations of the corresponding vector space objects to Riemannian manifolds. In particular,
\begin{itemize}
	\item A set $\mathcal{X}$ is called \emph{geodesically convex} if for any $x,y\in\mathcal{X}$, there is a geodesic $\gamma$ with $\gamma(0) = x, \gamma(1) = y$ and $\gamma(t)\in\mathcal{X}$ for $t\in [0,1]$.
	\item We call a function $f:\mathcal{X}\to\mathbb{R}$ \emph{geodesically convex} (g-convex) if for any $x,y\in\mathcal{X}$ and any geodesic $\gamma$ such that $\gamma(0)=x$, $\gamma(1)=y$ and $\gamma(t)\in\mathcal{X}$ for all $t\in [0,1]$, it holds that
	\[ f(\gamma(t)) \le (1-t)f(x) + tf(y). \]
	It can be shown that if the inverse exponential map is well-defined, an equivalent definition is that for any $x,y\in\mathcal{X}$, $f(y) \ge f(x) + \langle g_x, \Exp_x^{-1}(y) \rangle$,
	where $g_x$ is the gradient of $f$ at $x$ (in this work we assume $f$ is differentiable). A function $f:\mathcal{X}\to\mathbb{R}$ is called \emph{geodesically $\mu$-strongly convex} ($\mu$-strongly g-convex) if for any $x,y\in\mathcal{X}$ and gradient $g_x$, it holds that
	\[ f(y) \ge f(x) + \langle g_x, \Exp_x^{-1}(y) \rangle + \tfrac{\mu}{2}\|\Exp_x^{-1}(y)\|^2.\]
	\item We call a vector field $g :\mathcal{X}\to\mathbb{R}^d$ \emph{geodesically $L$-Lipschitz} ($L$-g-Lipschitz) if for any $x,y\in\mathcal{X}$,
	\[ \|g(x) - \Gamma_y^x g(y)\| \le L \|\Exp_x^{-1}(y)\|, \]
	where $\Gamma_y^x$ is the parallel transport from $y$ to $x$. We call a differentiable function $f:\mathcal{X}\to\mathbb{R}$ \emph{geodesically $L$-smooth} ($L$-g-smooth) if its gradient is $L$-g-Lipschitz, in which case we have
	\[ f(y) \le f(x) + \langle g_x, \Exp_x^{-1}(y) \rangle + \tfrac{L}{2}\|\Exp_x^{-1}(y)\|^2. \]
\end{itemize}
Throughout our analysis, for simplicity, we make the following standing assumptions:
\begin{assumption} \label{assumption:1}
	$\mathcal{X}\subset\mathcal{M}$ is a geodesically convex set where the exponential map $\Exp$ and its inverse $\Exp^{-1}$ are well defined.
\end{assumption} \vspace{-18pt}
\begin{assumption} \label{assumption:2}
	The sectional curvature in $\mathcal{X}$ is bounded, i.e. $|\kappa(x,\cdot)|\le K, \forall x\in\mathcal{X}$.
\end{assumption} \vspace{-18pt}
\begin{assumption} \label{assumption:3}
	$f$ is geodesically $L$-smooth, $\mu$-strongly convex, and assumes its minimum inside $\mathcal{X}$.
\end{assumption} \vspace{-18pt}
\begin{assumption} \label{assumption:4}
	All the iterates remain in $\mathcal{X}$.
\end{assumption}
With these assumptions, the problem being solved can be stated formally as $\min_{x\in\mathcal{X}\subset\mathcal{M}} ~ f(x)$.

%Classic results in Riemannian geometry, esp. Jacobi field estimate, Rauch comparison theorem and its corollary on bi-Lipschitzness of the exponential map \citep[e.g.][Cor. 5.6.1]{jost2011riemannian}. This line of work is closely related to our key lemma, but does not apply to our problem, as we will explain in section 3.


%%% Local Variables:
%%% mode: latex
%%% TeX-master: "colt2018"
%%% End:

\section{Proposed algorithm: \ragd}
\begin{algorithm}[hbtp]
	\caption{Riemannian-Nesterov($x_0, \gamma_0, \{h_k\}_{k=0}^{T-1}, \{\beta_k\}_{k=0}^{T-1}$)} \label{alg:riemannian-ag}
	\SetAlgoLined
	\SetKwInput{KwData}{Parameters}
	\KwData{initial point $x_0\in\mathcal{X}$, $\gamma_0>0$, step sizes $\{h_k\le\frac{1}{L}\}$, shrinkage parameters $\{\beta_k>0\}$}
	initialize $v_0 = x_0$\\
	\For{$k=0,1,\dots,T-1$}{
		Compute $\alpha_k\in(0,1)$ from the equation
		$\alpha_k^2 = h_k\cdot\left((1-\alpha_k)\gamma_k + \alpha_k\mu\right)$\\
		Set $\overline{\gamma}_{k+1} = (1-\alpha_k)\gamma_k + \alpha_k\mu$\\
\nl	\label{ln:y_k}	Choose $y_k = \Exp_{x_k}\left(\frac{\alpha_k\gamma_k}{\gamma_k+\alpha_k\mu}\Exp_{x_k}^{-1}(v_k)\right)$\\
		Compute $f(y_k)$ and $\nabla f(y_k)$\\
\nl	\label{ln:x_k+1}	Set $x_{k+1} = \Exp_{y_k}\left(-h_k\nabla f(y_k)\right)$\label{eq:x-k+1} \\ 
\nl	\label{ln:v_k+1}	Set $v_{k+1} = \Exp_{y_k}\left(\frac{(1-\alpha_k)\gamma_k}{\overline{\gamma}_{k+1}} \Exp_{y_k}^{-1}(v_k) - \frac{\alpha_k}{\overline{\gamma}_{k+1}} \nabla f(y_k)\right)$\label{eq:v-k+1}\\ 
		Set $\gamma_{k+1} = \frac{1}{1+\beta_k}\overline{\gamma}_{k+1}$
	}
	{\bf Output:} $x_T$
\end{algorithm}

\begin{figure}[hbt]
	\centering \def\svgwidth{200pt}
	\input{figures/alg1.pdf_tex} \def\svgwidth{200pt} 
	\input{figures/alg1-2.pdf_tex}
	\caption{Illustration of the geometric quantities in Algorithm \ref{alg:riemannian-ag}. \textbf{Left:} iterates and minimizer $x^*$ with $y_{k}$'s tangent space shown schematically. \textbf{Right:} the inverse exponential maps of relevant iterates in $y_{k}$'s tangent space. Note that $y_k$ is on the geodesic from $x_k$ to $v_k$ (Algorithm \ref{alg:riemannian-ag}, Line \ref{ln:y_k}); $\Exp_{y_k}^{-1}(x_{k+1})$ is in the opposite direction of $\mathrm{grad} f(y_k)$ (Algorithm \ref{alg:riemannian-ag}, Line \ref{ln:x_k+1}); also note how $\Exp_{y_k}^{-1}(v_{k+1})$ is constructed (Algorithm \ref{alg:riemannian-ag}, Line \ref{ln:v_k+1}).}
\end{figure}

Our proposed optimization procedure is shown in Algorithm \ref{alg:riemannian-ag}. We assume the algorithm is granted access to oracles that can efficiently compute the exponential map and its inverse, as well as the Riemannian gradient of function $f$. In comparison with Nesterov's accelerated gradient method in vector space \citep[p.76]{nesterov2004introductory}, we note two important differences: first, instead of linearly combining vectors, the update for iterates is computed via exponential maps; second, we introduce a paired sequence of parameters $\{(\gamma_k, \overline{\gamma}_k)\}_{k=0}^{T-1}$, for reasons that will become clear when we analyze the convergence of the algorithm. 

Algorithm \ref{alg:riemannian-ag} provides a general scheme for Nesterov-style algorithms on Riemannian manifolds, leaving the choice of many parameters to users' preference. To further simplify the parameter choice as well as the analysis, we note that the following specific choice of parameters
\[ \gamma_0\equiv\gamma = \frac{\sqrt{\beta^2+4(1+\beta)\mu h}-\beta}{\sqrt{\beta^2+4(1+\beta)\mu h}+\beta}\cdot \mu, \qquad h_k\equiv h, \forall k\ge 0, \qquad \beta_k\equiv \beta > 0, \forall k\ge 0, \]
which leads to Algorithm \ref{alg:constant-step}, a constant step instantiation of the general scheme. We leave the proof of this claim as a lemma in the Appendix.

\begin{algorithm}[hbtp]
	\caption{Constant Step Riemannian-Nesterov($x_0, h, \beta$)}  \label{alg:constant-step}
	\SetAlgoLined
	\SetKwInput{KwData}{Parameters}
	\KwData{initial point $x_0\in\mathcal{X}$, step size $h\le\frac{1}{L}$, shrinkage parameter $\beta > 0$}
	initialize $v_0 = x_0$\\
	set $\alpha = \frac{\sqrt{\beta^2+4(1+\beta)\mu h}-\beta}{2}$,~ $\gamma = \frac{\sqrt{\beta^2+4(1+\beta)\mu h}-\beta}{\sqrt{\beta^2+4(1+\beta)\mu h}+\beta}\cdot \mu$,~ $\overline{\gamma} = (1+\beta)\gamma$\\
	\For{$k=0,1,\dots,T-1$}{
		Choose $y_k = \Exp_{x_k}\left(\frac{\alpha\gamma}{\gamma+\alpha\mu}\Exp_{x_k}^{-1}(v_k)\right)$\\
		Set $x_{k+1} = \Exp_{y_k}\left(-h\nabla f(y_k)\right)$ \\ 
		Set $v_{k+1} = \Exp_{y_k}\left(\frac{(1-\alpha)\gamma}{\overline{\gamma}} \Exp_{y_k}^{-1}(v_k) - \frac{~\alpha~}{~\overline{\gamma}~} \nabla f(y_k)\right)$
	}
	{\bf Output:} $x_T$
\end{algorithm}

We move forward to analyzing the convergence properties of these two algorithms in the following two sections. In Section \ref{sec:general-analysis}, we first provide a novel generalization of Nesterov's estimate sequence to Riemannian manifolds, then show that if a specific tangent space distance comparison inequality (\ref{eq:base-change-assumption}) always holds, then Algorithm \ref{alg:riemannian-ag} converges similarly as its vector space counterpart. In Section \ref{sec:constant-step-analysis}, we establish sufficient conditions for this tangent space distance comparison inequality to hold, specifically for Algorithm \ref{alg:constant-step}, and show that under these conditions Algorithm \ref{alg:constant-step} converges in $O\left(\sqrt{\frac{L}{\mu}}\log(1/\epsilon)\right)$ iterations, a faster rate than the $O\left(\frac{L}{\mu}\log(1/\epsilon)\right)$ complexity of Riemannian gradient descent.


%%% Local Variables:
%%% mode: latex
%%% TeX-master: "colt2018"
%%% End:

\section{Analysis of a new estimate sequence} \label{sec:general-analysis}
First introduced in \citep{nesterov1983method}, estimate sequences are central tools in establishing the acceleration of Nesterov's method. We first note a weaker notion of estimate sequences for functions whose domain is not necessarily a vector space.
%\begin{definition}
%	A pair of sequences $\{\Phi_{k}(x):\mathcal{X}\to\mathbb{R}\}_{k=0}^{\infty}$ and $\{\lambda_k\}_{k=0}^{\infty}$ is called an estimate sequence of function $f(x):\mathcal{X}\to\mathbb{R}$ if $\lambda_k\to 0$ and for any $x\in\mathcal{X}$ and all $k\ge 0$ we have:
%	\begin{equation} \label{eq:estimate-sequence-definition}
%	\Phi_k(x) \le (1-\lambda_k)f(x) + \lambda_k\Phi_0(x).
%	\end{equation}
%\end{definition}
\begin{definition} \label{def:weak-estimate-sequence}
	A pair of sequences $\{\Phi_{k}(x):\mathcal{X}\to\mathbb{R}\}_{k=0}^{\infty}$ and $\{\lambda_k\}_{k=0}^{\infty}$ is called a (weak) estimate sequence of a function $f(x):\mathcal{X}\to\mathbb{R}$, if $\lambda_k\to 0$ and for all $k\ge 0$ we have:
	\begin{equation} \label{eq:weak-estimate-sequence-definition}
	\Phi_k(x^*) \le (1-\lambda_k)f(x^*) + \lambda_k\Phi_0(x^*).
	\end{equation}
\end{definition}
This definition relaxes the original definition proposed by \citet[def. 2.2.1]{nesterov2004introductory}, in that the latter requires $\Phi_k(x) \le (1-\lambda_k)f(x) + \lambda_k\Phi_0(x)$ to hold for all $x\in\mathcal{X}$, whereas our definition only assumes it holds at the minimizer $x^*$. We note that similar observations have been made, e.g., in \citep{carmon2017convex}. This relaxation is essential for sparing us from fiddling with the global geometry of Riemannian manifolds.
%
%In the sequel, we will instead use this weak notion of estimate sequence in our analysis.


%Following the above definition, the main theme of this section is to introduce a new type of estimate sequence that generalizes Nesterov's original construction to Riemannian manifolds. 
However, there is one major obstacle in the analysis -- Nesterov's construction of quadratic function sequence critically relies on the linear metric and does not generalize to nonlinear space. An example is given in Figure \ref{fig:change-base}, where we illustrate the distortion of distance (hence quadratic functions) in tangent spaces. The key novelty in our construction is inequality (\ref{eq:phi-less-overline-phi}) which allows a broader family of estimate sequences, as well as inequality (\ref{eq:base-change-assumption}) which handles nonlinear metric distortion and fulfills inequality (\ref{eq:phi-less-overline-phi}). Before delving into the analysis of our specific construction, we recall how to construct estimate sequences and note their use in the following two lemmas.

\begin{lemma} \label{thm:estimate-sequence-construction}
	Let us assume that:
	\begin{enumerate}
          \setlength{\itemsep}{1pt}
		\item $f$ is geodesically $L$-smooth and $\mu$-strongly geodesically convex on domain $\mathcal{X}$.
		\item $\Phi_0(x)$ is an arbitrary function on $\mathcal{X}$.
		\item $\{y_k\}_{k=0}^{\infty}$ is an arbitrary sequence in $\mathcal{X}$.
		\item $\{\alpha_k\}_{k=0}^{\infty}$: $\alpha_k\in(0,1)$,  $\sum_{k=0}^{\infty} \alpha_k = \infty$. \label{eq:alpha-k-not-summable}
		\item $\lambda_0 = 1$.
	\end{enumerate}
	Then the pair of sequences $\{\Phi_k(x)\}_{k=0}^{\infty}$, $\{\lambda_k\}_{k=0}^{\infty}$ which satisfy the following recursive rules:
	\begin{align}
	\lambda_{k+1} = &~ (1-\alpha_k)\lambda_k, \\
	\overline{\Phi}_{k+1}(x) = &~ (1-\alpha_k)\Phi_k(x) + \alpha_k \left[f(y_k) + \langle \nabla f(y_k), \Exp_{y_k}^{-1}(x)\rangle + \frac{\mu}{2}\|\Exp_{y_k}^{-1}(x)\|^2\right], \label{eq:phi-recursion}\\
	\Phi_{k+1}(x^*) \le &~ \overline{\Phi}_{k+1}(x^*), \label{eq:phi-less-overline-phi}
	\end{align}
	is a (weak) estimate sequence.
\end{lemma}
	The proof is similar to~\citep[Lemma 2.2.2]{nesterov2004introductory} which we include in Appendix \ref{prf:estimate-sequence-construction}.

\begin{lemma} \label{thm:estimate-sequence-implication}
	If for a (weak) estimate sequence $\{\Phi_{k}(x):\mathcal{X}\to\mathbb{R}\}_{k=0}^{\infty}$ and $\{\lambda_k\}_{k=0}^{\infty}$ we can find a sequence of iterates $\{x_k\}$, such that
	\[ f(x_k) \le \Phi_k^* \equiv \min_{x\in\mathcal{X}}\Phi_k(x), \]
	then $f(x_k)-f(x^*) \le \lambda_k(\Phi_0(x^*)-f(x^*)) \to 0$.
\end{lemma}
\begin{proof} By Definition \ref{def:weak-estimate-sequence} we have
	$f(x_k)\le \Phi_k^* \le \Phi_k(x^*) \le (1-\lambda_k)f(x^*) + \lambda_k\Phi_0(x^*)$.
	Hence $f(x_k) - f(x^*) \le \lambda_k(\Phi_0(x^*) - f(x^*)) \to 0$.
\end{proof}
Lemma \ref{thm:estimate-sequence-implication} immediately suggest the use of (weak) estimate sequences in establishing the convergence and analyzing the convergence rate of certain iterative algorithms. The following lemma shows that a weak estimate sequence exists for Algorithm \ref{alg:riemannian-ag}. Later in Lemma \ref{thm:x-k-bound}, we prove that the sequence $\{x_k\}$ in Algorithm \ref{alg:riemannian-ag} satisfies the requirements in Lemma \ref{thm:estimate-sequence-implication} for our estimate sequence.

\begin{lemma} \label{thm:estimate-sequence-lemma}
	Let $\Phi_0(x) = \Phi_0^* + \frac{\gamma_0}{2}\|\Exp_{y_0}^{-1}(x)\|^2$.
	Assume for all $k\ge 0$, the sequences $\{\gamma_k\}$, $\{\overline{\gamma}_k\}$, $\{v_k\}$, $\{\Phi_k^*\}$ and $\{\alpha_k\}$ satisfy
	\begin{align}
	\overline{\gamma}_{k+1} = &~ (1-\alpha_k)\gamma_k + \alpha_k\mu, \label{eq:overline-gamma-k+1}\\
	v_{k+1} = &~ \Exp_{y_k}\left(\frac{(1-\alpha_k)\gamma_k}{\overline{\gamma}_{k+1}} \Exp_{y_k}^{-1}(v_k) - \frac{\alpha_k}{\overline{\gamma}_{k+1}} \nabla f(y_k)\right)  \\
	\nonumber \Phi_{k+1}^* = &~ \left(1 - \alpha_k\right) \Phi_k^* + \alpha_k f(y_k) - \frac{\alpha_k^2}{2\overline{\gamma}_{k+1}}\|\nabla f(y_k)\|^2 \\
	&~ + \frac{\alpha_k(1-\alpha_k)\gamma_k}{\overline{\gamma}_{k+1}}\left(\frac{\mu}{2}\|\Exp_{y_k}^{-1}(v_k)\|^2 + \langle \nabla f(y_k), \Exp_{y_k}^{-1}(v_k)\rangle \right), \label{eq:phi-k+1-star} \\
	\gamma_{k+1} \| \Exp&_{y_{k+1}}^{-1}(x^*) -\Exp_{y_{k+1}}^{-1}(v_{k+1})\|^2 \le \overline{\gamma}_{k+1}\|\Exp_{y_k}^{-1}(x^*)-\Exp_{y_k}^{-1}(v_{k+1})\|^2, \label{eq:base-change-assumption} \\
	\alpha_k\in &~ (0,1), \quad \sum_{k=0}^{\infty} \alpha_k = \infty,
	\end{align}
	then the pair of sequence $\{\Phi_k(x)\}_{k=0}^{\infty}$ and $\{\lambda_k\}_{k=0}^{\infty}$, defined by
	\begin{align}
	\Phi_{k+1}(x) = &~ \Phi_{k+1}^* + \frac{\gamma_{k+1}}{2}\|\Exp_{y_{k+1}}^{-1}(x)-\Exp_{y_{k+1}}^{-1}(v_{k+1})\|^2, \\
	\lambda_0 = 1,  \quad &~ \lambda_{k+1} = (1-\alpha_k)\lambda_k.
	\end{align}
	is a (weak) estimate sequence.
\end{lemma}
\begin{proof} Recall the definition of $\overline{\Phi}_{k+1}(x)$ in Equation (\ref{eq:phi-recursion}). We claim that if $\Phi_k(x) = \Phi_k^* + \frac{\gamma_k}{2}\|\Exp_{y_k}^{-1}(x)-\Exp_{y_k}^{-1}(v_k)\|^2$, then we have $\overline{\Phi}_{k+1}(x) \equiv \Phi_{k+1}^* + \frac{\overline{\gamma}_{k+1}}{2}\|\Exp_{y_k}^{-1}(x)-\Exp_{y_k}^{-1}(v_{k+1})\|^2$. The proof of this claim requires a simple algebraic manipulation as is noted as Lemma \ref{thm:complete-square}. Now using the assumption (\ref{eq:base-change-assumption}) we immediately get $\Phi_{k+1}(x^*)\le\overline{\Phi}_{k+1}(x^*)$. By Lemma \ref{thm:estimate-sequence-construction} the proof is complete.
\end{proof}
We verify the specific form of $\overline{\Phi}_{k+1}(x)$ in Lemma~\ref{thm:complete-square}, whose proof can be found in the Appendix \ref{prf:complete-square}.
\begin{lemma} \label{thm:complete-square}
	For all $k\ge 0$, if $\Phi_k(x) = \Phi_k^* + \frac{\gamma_k}{2}\|\Exp_{y_k}^{-1}(x)-\Exp_{y_k}^{-1}(v_k)\|^2$, then with $\overline{\Phi}_{k+1}$ defined as in (\ref{eq:phi-recursion}), $\overline{\gamma}_{k+1}$ as in (\ref{eq:overline-gamma-k+1}), $v_{k+1}$ as in Algorithm \ref{alg:riemannian-ag} and $\Phi_{k+1}^*$ as in ($\ref{eq:phi-k+1-star}$) we have $\overline{\Phi}_{k+1}(x) \equiv \Phi_{k+1}^* + \frac{\overline{\gamma}_{k+1}}{2}\|\Exp_{y_k}^{-1}(x)-\Exp_{y_k}^{-1}(v_{k+1})\|^2$.
\end{lemma}
The next lemma asserts that the iterates $\{x_k\}$ of Algorithm \ref{alg:riemannian-ag} satisfy the requirement that the function values $f(x_k)$ are upper bounded by $\Phi_k^*$ defined in our estimate sequence.
\begin{lemma} \label{thm:x-k-bound}
	Assume $\Phi_0^*=f(x_0)$, and $\{\Phi_k^*\}$ be defined as in (\ref{eq:phi-k+1-star}) with $\{x_k\}$ and other terms defined as in Algorithm \ref{alg:riemannian-ag}. Then we have $\Phi_{k}^*\ge f(x_k)$ for all $k\ge 0$.
\end{lemma}
The proof is standard. We include it in Appendix \ref{prf:x-k-bound} for completeness.
Finally, we are ready to state the following theorem on the convergence rate of Algorithm \ref{alg:riemannian-ag}.

\begin{theorem}[Convergence of Algorithm \ref{alg:riemannian-ag}] \label{thm:main-theorem-general-scheme}
	For any given $T\ge 0$, assume (\ref{eq:base-change-assumption}) is satisfied for all $0\le k\le T$, then Algorithm \ref{alg:riemannian-ag} generates a sequence $\{x_k\}_{k=0}^{\infty}$ such that
	\begin{equation} \label{eq:convergence-algorithm-1}
	 f(x_T) - f(x^*) \le \lambda_T\left(f(x_0) - f(x^*) + \frac{\gamma_0}{2}\|\Exp_{x_0}^{-1}(x^*)\|^2 \right)
	 \end{equation}
	where $\lambda_0 = 1$ and $\lambda_k = \prod_{i=0}^{k-1}(1-\alpha_i)$.
\end{theorem}
\begin{proof}
	The proof is similar to \citep[Theorem 2.2.1]{nesterov2004introductory}. We choose $\Phi_0(x) = f(x_0) + \frac{\gamma_0}{2}\|\Exp_{y_0}^{-1}(x)\|^2$, hence $\Phi_0^* = f(x_0)$. By Lemma \ref{thm:estimate-sequence-lemma} and Lemma \ref{thm:x-k-bound}, the assumptions in Lemma \ref{thm:estimate-sequence-implication} hold. It remains to use Lemma \ref{thm:estimate-sequence-implication}.
\end{proof}

\section{Local fast rate with a constant step scheme} \label{sec:constant-step-analysis}
% Towards this goal, we first make a few specification to our general scheme in Algorithm \ref{alg:riemannian-ag} to simplify the choice of parameters.
%\begin{lemma}[Bounding $\|\nabla f(y_k)\|$]
%	\begin{equation}
%	\|\nabla f(y_k)\|^2 \le 2L (f(y_k)-f(x^*))
%	\end{equation}
%\end{lemma}
%\begin{proof}
%	Use $L$-smooth assumption and note that $x^*$ is the minimizer, we have
%	\[ f(x^*)\le f\left(\Exp_{y_k}\left(-\nabla f(y_k)/L\right)\right) \le f(y_k) - \frac{1}{2L}\|\nabla f(y_k)\|^2. \]
%	Rearrange the terms and the proof is complete.
%\end{proof}

By now we see that almost all the analysis of Nesterov's generalizes, except that the assumption in (\ref{eq:base-change-assumption}) is not necessarily satisfied. In vector space, the two expressions both reduce to $x^* - v_{k+1}$ and hence (\ref{eq:base-change-assumption}) trivially holds with $\gamma = \overline{\gamma}$. On Riemannian manifolds, however, due to the nonlinear Riemannian metric and the associated exponential maps,  $\| \Exp_{y_{k+1}}^{-1}(x^*) -\Exp_{y_{k+1}}^{-1}(v_{k+1})\|$ and $\|\Exp_{y_k}^{-1}(x^*)-\Exp_{y_k}^{-1}(v_{k+1})\|$ in general do not equal (illustrated in Figure \ref{fig:change-base}). Bounding the difference between these two quantities points the way forward for our analysis, which is also our main contribution in this section. We start with two lemmas comparing a geodesic triangle and the triangle formed by the preimage of its vertices in the tangent space, in two constant curvature spaces: hyperbolic space and the hypersphere.
\begin{figure}[hbt]
	\centering \hspace{30pt} \def\svgwidth{220pt}
	\input{figures/thm10-2.pdf_tex} \hspace{-80pt} \def\svgwidth{190pt} 
	\input{figures/thm10.pdf_tex} \hspace{-30pt}
	\caption{A schematic illustration of the geometric quantities in Theorem \ref{thm:squared-distance-ratio-bound}. Tangent spaces of $y_{k}$ and $y_{k+1}$ are shown in separate figures to reduce cluttering. Note that even on a sphere (which has constant positive sectional curvature), $d(x^*, v_{k+1}), \|\Exp_{y_{k}}^{-1}(x^*)-\Exp_{y_{k}}^{-1}(v_{k+1})\|$ and $ \|\Exp_{y_{k+1}}^{-1}(x^*)-\Exp_{y_{k+1}}^{-1}(v_{k+1})\|$ generally do not equal.} \label{fig:change-base}
\end{figure}

\begin{lemma}[bi-Lipschitzness of the exponential map in hyperbolic space] \label{thm:hyperbolic-squared-distance-distortion}
	Let $a,b,c$ be the side lengths of a geodesic triangle in a hyperbolic space with constant sectional curvature $-1$, and $A$ is the angle between sides $b$ and $c$. Furthermore, assume $b\le\frac{1}{4},c\ge 0$. Let $\triangle\bar{a}\bar{b}\bar{c}$ be the comparison triangle in Euclidean space, with $\bar{b}=b,\bar{c}=c,\bar{A}=A$, then
	\begin{equation}
	\bar{a}^2 \le a^2\le (1+2b^2)\bar{a}^2.
	\end{equation}
\end{lemma}
\begin{proof}
	The proof of this lemma contains technical details that deviate from our main focus; so we defer them to the appendix. The first inequality is well known. To show the second inequality, we have Lemma \ref{thm:large-c-hyperbolic} and Lemma \ref{thm:small-c-hyperbolic} (in Appendix) which in combination complete the proof.
\end{proof}
We also state without proof that by the same techniques one can show the following result holds.
\begin{lemma}[bi-Lipschitzness of the exponential map on hypersphere] \label{thm:hypersphere-squared-distance-distortion}
	Let $a,b,c$ be the side lengths of a geodesic triangle in a hypersphere with constant sectional curvature $1$, and $A$ is the angle between sides $b$ and $c$. Furthermore, assume $b\le\frac{1}{4},c\in[0,\frac{\pi}{2}]$. Let $\triangle\bar{a}\bar{b}\bar{c}$ be the comparison triangle in Euclidean space, with $\bar{b}=b,\bar{c}=c,\bar{A}=A$, then
	\begin{equation}
	a^2\le \bar{a}^2\le (1+2b^2)a^2.
	\end{equation}
\end{lemma}
Albeit very much simplified, spaces of constant curvature are important objects to study, because often their properties can be generalized to general Riemannian manifolds with bounded curvature, specifically via the use of powerful comparison theorems in metric geometry \citep{burago2001course}. In our case, we use these two lemmas to derive a tangent space distance comparison theorem for Riemannian manifolds with bounded sectional curvature.
\begin{theorem}[Multiplicative distortion of squared distance on Riemannian manifold] \label{thm:squared-distance-ratio-bound}\\
	Let $x^*$, $v_{k+1}$, $y_k$, $y_{k+1}\in\mathcal{X}$ be four points in a g-convex, uniquely geodesic set $\mathcal{X}$ where the sectional curvature is bounded within $[-K, K]$, for some nonnegative number $K$.
%	 where $\kappa_{\max}>0>\kappa_{\min}$ and $K = \max\{-\kappa_{\min},\kappa_{\max}\}$, 
	Define $b_{k+1}=\max\left\{\|\Exp_{y_k}^{-1}(x^*)\|,\|\Exp_{y_{k+1}}^{-1}(x^*)\|\right\}$. Assume $b_{k+1}\le\frac{1}{4\sqrt{K}}$ for $K>0$ (otherwise $b_{k+1} < \infty$), then we have 
	\begin{equation}
	\|\Exp_{y_{k+1}}^{-1}(x^*)-\Exp_{y_{k+1}}^{-1}(v_{k+1})\|^2 \le (1+5K b_{k+1}^2) \|\Exp_{y_k}^{-1}(x^*)-\Exp_{y_k}^{-1}(v_{k+1})\|^2.
	\end{equation}
\end{theorem}

\begin{proof}
	The high level idea is to think of the tangent space distance distortion on Riemannian manifolds of bounded curvature as a consequence of bi-Lipschitzness of the exponential map. Specifically, note that $\triangle y_k x^* v_{k+1}$ and $\triangle y_{k+1} x^* v_{k+1}$ are two geodesic triangles in $\mathcal{X}$, whereas $\|\Exp_{y_k}^{-1}(x^*)-\Exp_{y_k}^{-1}(v_{k+1})\|$ and $\|\Exp_{y_{k+1}}^{-1}(x^*)-\Exp_{y_{k+1}}^{-1}(v_{k+1})\|$ are side lengths of two comparison triangles in vector space. Since $\mathcal{X}$ is of bounded sectional curvature, we can apply comparison theorems.
	
	First, we consider bound on the distortion of squared distance in a Riemannian manifold with constant curvature $-K$. Note that in this case, the hyperbolic law of cosines becomes
	\[ \cosh(\sqrt{K}a) = \cosh(\sqrt{K}b)\cosh(\sqrt{K}c) - \sinh(\sqrt{K}b)\sinh(\sqrt{K}c)\cos(A), \]
	which corresponds to the geodesic triangle in hyperbolic space with side lengths $\sqrt{K}a, \sqrt{K}b,\sqrt{K}c$, with the corresponding comparison triangle in Euclidean space having lengths $\sqrt{K}\bar{a}, \sqrt{K}\bar{b}, \sqrt{K}\bar{c}$. Apply Lemma \ref{thm:hyperbolic-squared-distance-distortion} we have $(\sqrt{K}a)^2 \le (1+2(\sqrt{K}b)^2)(\sqrt{K}\bar{a})^2$, i.e. $a^2 \le (1+2K b^2)\bar{a}^2$.
	%	and the squared distance ratio becomes
	%	\begin{align}
	%	\nonumber &~ \frac{\left(\arccosh\left(\cosh(\sqrt{K}b)\cosh(\sqrt{K}c) - \sinh(\sqrt{K}b)\sinh(\sqrt{K}c)\cos(A)\right)/\sqrt{K}\right)^2}{b^2 + c^2 - 2bc\cos(A)} \\
	%	\nonumber = &~ \frac{\left(\arccosh\left(\cosh(\sqrt{K}b)\cosh(\sqrt{K}c) - \sinh(\sqrt{K}b)\sinh(\sqrt{K}c)\cos(A)\right)\right)^2}{(\sqrt{K}b)^2 + (\sqrt{K}c)^2 - 2(\sqrt{K}b)(\sqrt{K}c)\cos(A)} \\
	%	\le &~ 1 + 2(\sqrt{K}b)^2 = 1 + 2K b^2 \label{eq:hyperbolic-squared-distance-ratio}
	%	\end{align}
	%	assuming $\sqrt{K}b\le\frac{1}{4}$, where the inequality is by using lengths $\sqrt{K}b, \sqrt{K}c$ in Lemma \ref{thm:hyperbolic-squared-distance-distortion}.
	Now consider the geodesic triangle $\triangle y_k x^* v_{k+1}$. Let $\tilde{a}=\|\Exp_{v_{k+1}}^{-1}(x^*)\|, b=\|\Exp_{y_k}^{-1}(v_{k+1})\|\le b_{k+1}, c=\|\Exp_{y_k}^{-1}(x^*)\|, A=\angle x^* y_k v_{k+1}$, so that $\|\Exp_{y_k}^{-1}(x^*)-\Exp_{y_k}^{-1}(v_{k+1})\|^2 = b^2+c^2-2bc\cos(A)$. By Toponogov's comparison theorem \citep{burago2001course}, we have $\tilde{a}\le a$
	hence
	\begin{equation} \label{eq:y-k-squared-distance-ratio}
	\|\Exp_{v_{k+1}}^{-1}(x^*)\|^2 \le \left(1+2K b_{k+1}^2\right)\|\Exp_{y_k}^{-1}(x^*)-\Exp_{y_k}^{-1}(v_{k+1})\|^2.
	\end{equation}
	Similarly, using the spherical law of cosines for a space of constant curvature $K$ 
	\[ \cos(\sqrt{K}a) = \cos(\sqrt{K}b)\cos(\sqrt{K}c) + \sin(\sqrt{K}b)\sin(\sqrt{K}c)\cos(A) \]
	and Lemma \ref{thm:hypersphere-squared-distance-distortion} we can show $\bar{a}^2 \le (1+2K b^2)a^2$, where $\bar{a}$ is the  side length in Euclidean space corresponding to $a$. 
	%	corresponds to the geodesic triangle in hypersphere with side lengths $\sqrt{K}a, \sqrt{K}b,\sqrt{K}c$. For 
	%	and lengths $\sqrt{K}b, \sqrt{K}c$ in Lemma \ref{thm:hypersphere-squared-distance-distortion}, we have that when comparing geodesic triangles $(\{b,c\},A)$ in Euclidean space and a Riemannian manifold with constant curvature $K$, assuming $\sqrt{K}b\le\frac{1}{4}, \sqrt{K}c\le\frac{\pi}{2}$, the squared distance ratio is bounded by
	%	\begin{align}
	%	\nonumber &~ \frac{b^2 + c^2 - 2bc\cos(A)}{\left(\arccos\left(\cos(\sqrt{K}b)\cos(\sqrt{K}c) + \sin(\sqrt{K}b)\sin(\sqrt{K}c)\cos(A)\right)/\sqrt{K}\right)^2} \\
	%	\nonumber = &~ \frac{(\sqrt{K}b)^2 + (\sqrt{K}c)^2 - 2(\sqrt{K}b)(\sqrt{K}c)\cos(A)}{\left(\arccos\left(\cos(\sqrt{K}b)\cos(\sqrt{K}c) + \sin(\sqrt{K}b)\sin(\sqrt{K}c)\cos(A)\right)\right)^2} \\
	%	\le &~ 1 + 2(\sqrt{K}b)^2 = 1 + 2K b^2 \label{eq:hypersphere-squared-distance-ratio}
	%	\end{align}
	%	
	%	Similarly, for the geodesic triangle $\triangle y_{k+1} x v_{k+1}$. We now let $a=\|\Exp_{v_{k+1}}^{-1}(x)\|, b=\|\Exp_{y_{k+1}}^{-1}(v_{k+1})\|, c=\|\Exp_{y_{k+1}}^{-1}(x)\|, A=\angle x y_{k+1} v_{k+1}$, so that $\|\Exp_{y_{k+1}}^{-1}(x)-\Exp_{y_{k+1}}^{-1}(v_{k+1})\|^2 = b^2+c^2-2bc\cos(A)$. Using a corollary of the Rauch comparison theorem (TODO: note this is a local result; assumptions required), we have that
	Hence by our uniquely geodesic assumption and \citep[Theorem 2.2, Remark 7]{meyer1989toponogov}, with similar reasoning for the geodesic triangle $\triangle y_{k+1} x^* v_{k+1}$, we have $a \le \|\Exp_{v_{k+1}}^{-1}(x^*)\|$, so that
	\begin{equation} \label{eq:y-k+1-squared-distance-ratio}
	\|\Exp_{y_{k+1}}^{-1}(x^*)-\Exp_{y_{k+1}}^{-1}(v_{k+1})\|^2 \le \left(1+2K b_{k+1}^2\right)a^2 \le \left(1+2K b_{k+1}^2\right)\|\Exp_{v_{k+1}}^{-1}(x^*)\|^2.
	\end{equation}
	Finally, combining inequalities (\ref{eq:y-k-squared-distance-ratio}) and (\ref{eq:y-k+1-squared-distance-ratio}), and noting that $(1+2K b_{k+1}^2)^2 = 1+4K b_{k+1}^2 + (4K b_{k+1}^2)K b^2 \le 1 + 5K b_{k+1}^2$, the proof is complete.
\end{proof}
Theorem \ref{thm:squared-distance-ratio-bound} suggests that if $b_{k+1}\le\frac{1}{4\sqrt{K}}$, we could choose $\beta\ge 5K b_{k+1}^2$ and $\gamma \le\frac{1}{1+\beta}\overline{\gamma}$ to guarantee $\Phi_{k+1}(x^*)\le \overline{\Phi}_{k+1}(x^*)$. It then follows that the analysis holds for $k$-th step. Still, it is unknown that under what conditions can we guarantee $\Phi_{k+1}(x^*)\le \overline{\Phi}_{k+1}(x^*)$ hold for all $k\ge 0$, which would lead to a convergence proof. We resolve this question in the next theorem. 

\begin{theorem}[Local fast convergence] \label{thm:convergence-induction}
	With Assumptions \ref{assumption:1}, \ref{assumption:2}, \ref{assumption:3}, \ref{assumption:4}, denote $D = \frac{1}{20\sqrt{K}}\left(\frac{\mu}{L}\right)^{\frac{3}{4}}$ and assume $\mathcal{B}_{x^*, D}:=\{x\in\mathcal{M}: d(x,x^*)\le D\} \subseteq\mathcal{X}$.
	 If we set $h=\frac{1}{L}, \beta=\frac{1}{5}\sqrt{\frac{\mu}{L}}$ and $x_0 \in \mathcal{B}_{x^*, D}$,
	then Algorithm \ref{alg:constant-step} converges; moreover, we have
	\begin{equation} \label{eq:convergence-rate}
	f(x_k)-f(x^*)\le \left(1-\frac{9}{10}\sqrt{\frac{\mu}{L}}\right)^k\left(f(x_0)-f(x^*)+\frac{\mu}{2}\|\Exp_{x_0}^{-1}(x^*)\|^2\right).
	\end{equation}
\end{theorem}

\begin{proof}{\bf sketch.}
Recall that in Theorem 		\ref{thm:main-theorem-general-scheme} we already establish that if the tangent space distance comparison inequality (\ref{eq:base-change-assumption}) holds, then the general Riemannian Nesterov iteration (Algorithm \ref{alg:riemannian-ag}) and hence its constant step size special case (Algorithm \ref{alg:constant-step}) converge with a guaranteed rate. By the tangent space distance comparison theorem (Theorem \ref{thm:squared-distance-ratio-bound}), the comparison inequality should hold if $y_k$ and $x^*$ are close enough. Indeed, 
we use induction to assert that with a good initialization, (\ref{eq:base-change-assumption}) holds for each step. Specifically, for every $k>0$, if $y_k$ is close to $x^*$ and the comparison inequality holds until the $(k-1)$-th step, then $y_{k+1}$ is also close to $x^*$ and the comparison inequality holds until the $k$-th step. We postpone the complete proof until Appendix \ref{prf:convergence-induction}.
%	\emph{The base case.} First we verify that $y_0, y_1$ is sufficiently close to $x^*$ so that the comparison inequality (\ref{eq:base-change-assumption}) holds at step $k=0$. In fact, since $y_0=x_0$ by construction, we have 
%	\begin{equation} \label{eq:xstar-y0}
%	\|\Exp_{y_0}^{-1}(x^*)\| =\|\Exp_{x_0}^{-1}(x^*)\| \le  \frac{1}{4\sqrt{K}}, \qquad 5K\|\Exp_{y_0}^{-1}(x^*)\|^2 \le \frac{1}{80}\left(\frac{\mu}{L}\right)^{\frac{3}{2}} \le  \beta.
%	\end{equation}
%	To bound $\|\Exp_{y_1}^{-1}(x^*)\|$, observe that $y_1$ is on the geodesic between $x_1$ and $v_1$. So first we bound $\|\Exp_{x_1}^{-1}(x^*)\|$ and $\|\Exp_{v_1}^{-1}(x^*)\|$. Bound on $\|\Exp_{x_1}^{-1}(x^*)\|$ comes from strong g-convexity:
%	\begin{align*}
%	\|\Exp_{x_1}^{-1}(x^*)\|^2\le &~ \frac{2}{\mu}(f(x_1)-f(x^*))\le \frac{2}{\mu}(f(x_0)-f(x^*))+\frac{\gamma}{\mu}\|\Exp_{x_0}^{-1}(x^*)\|^2 \\
%	\le &~ \frac{L+\gamma}{\mu}\|\Exp_{x_0}^{-1}(x^*)\|^2, 
%	\end{align*}
%	whereas bound on $\|\Exp_{v_1}^{-1}(x^*)\|$ utilizes the tangent space distance comparison theorem. First, from the definition of $\overline{\Phi}_1$ we have
%	$$\|\Exp_{y_0}^{-1}(x^*)-\Exp_{y_0}^{-1}(v_1)\|^2 = \frac{2}{\gamma}(\overline{\Phi}_1(x^*)-\Phi_1^*)\le \frac{2}{\gamma}(\Phi_0(x^*)-f(x^*))\le \frac{L+\gamma}{\gamma}\|\Exp_{x_0}^{-1}(x^*)\|^2.$$
%	Then note that (\ref{eq:xstar-y0}) implies that the assumption in Theorem \ref{thm:squared-distance-ratio-bound} is satisfied when $k=0$, whereby
%    $$\|\Exp_{v_1}^{-1}(x^*)\|^2\le  (1+\beta)\|\Exp_{y_0}^{-1}(x^*)-\Exp_{y_0}^{-1}(v_1)\|^2\le \frac{2(L+\gamma)}{\gamma}\|\Exp_{x_0}^{-1}(x^*)\|^2.$$
%    Together we have the following sequence of inequalities
%    \begin{align}
%    	    \nonumber\|\Exp_{y_1}^{-1}(x^*)\|\le ~& \|\Exp_{x_1}^{-1}(x^*)\| + \frac{\alpha\gamma}{\gamma+\alpha\mu}\|\Exp_{x_1}^{-1}(v_1)\|\\  \nonumber\le ~& \|\Exp_{x_1}^{-1}(x^*)\| + \frac{\alpha\gamma}{\gamma+\alpha\mu}\left(\|\Exp_{x_1}^{-1}(x^*)\| + \|\Exp_{v_1}^{-1}(x^*)\|\right) \\
%    	    \nonumber\le ~& \sqrt{\frac{L+\gamma}{\mu}}\|\Exp_{x_0}^{-1}(x^*)\| + \frac{\alpha\gamma}{\gamma+\alpha\mu}\left(\sqrt{\frac{L+\gamma}{\mu}}+\sqrt{\frac{2(L+\gamma)}{\mu}}\right)\|\Exp_{x_0}^{-1}(x^*)\| \\
%    	    \nonumber\le ~& \left(1 + \frac{1+\sqrt{2}}{2}\right)\sqrt{\frac{L+\gamma}{\mu}}\|\Exp_{x_0}^{-1}(x^*)\| \\
%    	    \le ~& \frac{1}{10\sqrt{K}}\left(\frac{\mu}{L}\right)^{\frac{1}{4}} 
%    	    \le \frac{1}{4\sqrt{K}}, \label{eq:base-xstar-y}
%    \end{align}
%    which also imply the bound
%    \begin{equation}
%		5K\|\Exp_{y_1}^{-1}(x^*)\|^2 \le \frac{1}{20}\sqrt{\frac{\mu}{L}} \le \beta. \label{eq:base-beta}
%	\end{equation}
%     By inequalities (\ref{eq:base-xstar-y}), (\ref{eq:base-beta}) and Theorem \ref{thm:squared-distance-ratio-bound} it is therefore guaranteed that 
%    $$\gamma \| \Exp_{y_1}^{-1}(x^*) -\Exp_{y_1}^{-1}(v_1)\|^2 \le \overline{\gamma} \|\Exp_{y_0}^{-1}(x^*)-\Exp_{y_0}^{-1}(v_1)\|^2.$$
%	\emph{The inductive step.}
%	Assume that for $i=0,\dots,k-1$, (\ref{eq:base-change-assumption}) hold, i.e.:
%	$$\gamma \| \Exp_{y_{i+1}}^{-1}(x^*) -\Exp_{y_{i+1}}^{-1}(v_{i+1})\|^2 \le \overline{\gamma}\|\Exp_{y_i}^{-1}(x^*)-\Exp_{y_i}^{-1}(v_{i+1})\|^2, \forall i=0,\dots,k-1$$
%	and also that $\|\Exp_{y_k}^{-1}(x^*)\|\le \frac{1}{10\sqrt{K}}\left(\frac{\mu}{L}\right)^{\frac{1}{4}}$.
%	Note that due to the sequential nature of the algorithm, statements about any step only depend on its previous steps, but not any step afterwards. 
%	Since (\ref{eq:base-change-assumption}) hold for steps $i=0,\dots,k-1$, the analysis in the previous section already applies for steps $i=0,\dots,k-1$. Therefore, by Theorem  \ref{thm:main-theorem-general-scheme} and the proof of Lemma \ref{thm:x-k-bound} we know that 
%	\begin{align*}
%	f(x^*)\le &~ f(x_{k+1})\le\Phi_{k+1}^*\le\Phi_{k+1}(x^*)
%	\le f(x^*)+(1-\alpha)^{k+1}(\Phi_0(x^*)-f(x^*)) \\
%	\le &~ \Phi_0(x^*) = f(x_0)+\frac{\gamma}{2}\|\Exp_{x_0}^{-1}(x^*)\|^2.
%	\end{align*}
%	Following a similar route as the analysis for the base case\footnote{A complete proof is present in Appendix \ref{sec:complete-proof}.}, we can show $\|\Exp_{y_{k+1}}^{-1}(x^*)\|\le \frac{1}{10\sqrt{K}}\left(\frac{\mu}{L}\right)^{\frac{1}{4}}$ and also the desired comparison inequality
%	$$\gamma \| \Exp_{y_{k+1}}^{-1}(x^*) -\Exp_{y_{k+1}}^{-1}(v_{k+1})\|^2 \le \overline{\gamma} \|\Exp_{y_k}^{-1}(x^*)-\Exp_{y_k}^{-1}(v_{k+1})\|^2.$$
%	In other words, inequality (\ref{eq:base-change-assumption}) holds for $i=0,\dots,k$. This concludes the inductive step. \\
%	By induction, (\ref{eq:base-change-assumption}) holds for all $k\ge 0$. Hence by Theorem \ref{thm:main-theorem-general-scheme}, Algorithm \ref{alg:constant-step} converges, with
%	$$\alpha_i\equiv \alpha=\frac{\sqrt{\beta^2+4(1+\beta)\mu h}-\beta}{2} = \frac{\sqrt{\mu h}}{2}\left(\sqrt{\frac{1}{25}+4\left(1+\frac{\sqrt{\mu h}}{5}\right)} - \frac{1}{5}\right)\ge \frac{9}{10}\sqrt{\frac{\mu}{L}}.$$
%	Hence plugging $\lambda_k = \prod_{i=0}^{k-1} (1-\alpha_i) = \left(1-\frac{9}{10}\sqrt{\frac{\mu}{L}}\right)^k$ in (\ref{eq:convergence-algorithm-1}) completes the proof.
\end{proof}


%%% Local Variables:
%%% mode: latex
%%% TeX-master: "colt2018"
%%% End:

\section{Discussion}
In this work, we proposed a Riemannian generalization of the accelerated gradient algorithm and developed its convergence and complexity analysis. For the first time (to the best of our knowledge), we show gradient based algorithms on Riemannian manifolds can be accelerated, at least in a neighborhood of the minimizer. Central to our analysis are the two main technical contributions of our work: a new estimate sequence (Lemma \ref{thm:estimate-sequence-lemma}), which relaxes the assumption of Nesterov's original construction and handles metric distortion on Riemannian manifolds; a tangent space distance comparison theorem (Theorem \ref{thm:squared-distance-ratio-bound}), which provides sufficient conditions for bounding the metric distortion and could be of interest for a broader range of problems on Riemannian manifolds.

Despite not matching the standard convex results, our result exposes the key difficulty of analyzing Nesterov-style algorithms on Riemannian manifolds, an aspect missing in previous work. Critically, the convergence analysis relies on bounding a new distortion term per each step. Furthermore, we observe that the side length sequence $d(y_k, v_{k+1})$ can grow much greater than $d(y_k, x^*)$, even if we reduce the ``step size'' $h_k$ in Algorithm 1, defeating any attempt to control the distortion globally by modifying the algorithm parameters. This is a benign feature in vector space analysis, since (\ref{eq:base-change-assumption}) trivially holds nonetheless; however it poses a great difficulty for analysis in nonlinear space. Note the stark contrast to (stochastic) gradient descent, where the step length can be effectively controlled by reducing the step size, hence bounding the distortion terms globally \citep{zhang2016first}.

%A surprising observation we make is that step size only controls the speed of growth of $b_{k+1}$, a key quantity we would like to bound, but not its limit (see Lemma \ref{thm:b-bound} and the discussion afterwards). Consequentially our result has a small radius of convergence. We hypothesize that this fact is fundamentally connected to the update rule of $v_{k+1}$ we use in Line \ref{eq:v-k+1} of Algorithm \ref{alg:riemannian-ag}. One could of course try to add damping terms in Line \ref{eq:v-k+1}. However, as the expression in Line \ref{eq:v-k+1} is a direct result of Lemma \ref{thm:complete-square}, any such modification may inevitably require a completely new kind of estimate sequence. Developing and analyzing algorithms with greater convergence radius, most likely via better control of $b_{k+1}$, is an important topic for future research.

A topic of future interest is to study whether assumption (\ref{eq:base-change-assumption}) can be further relaxed, while maintaining that overall the algorithm still converges. By bounding the squared distance distortion in every step, our analysis provides guarantee for the worst-case scenario, which seems unlikely to happen in practice. It would be interesting to conduct experiments to see how often (\ref{eq:base-change-assumption}) is violated versus how often it is loose. It would also be interesting to construct some adversarial problem case (if any) and study the complexity lower bound of gradient based Riemannian optimization, to see if geodesically convex optimization is strictly more difficult than convex optimization. Generalizing the current analysis to non-strongly g-convex functions is another interesting direction.


\acks{The authors thank the anonymous reviewers for helpful feedback. This work was supported in part by NSF-IIS-1409802 and the DARPA Lagrange grant.}

\bibliographystyle{abbrvnat}

{\small
	\bibliography{main_agd17}
}

\newpage

%\section*{Acknowledgement}

%Y. Fei and Y. Chen were partially supported by the National Science
%Foundation CRII award 1657420 and grant 1704828.


\appendix
%\appendixpage

\section{Additional notations}

We define the shorthand $\error\coloneqq\norm[\Yhat-\Ystar]1$. For
a matrix $\M$, we write $\norm[\M]{\infty}\coloneqq\max_{i,j}\left|M_{ij}\right|$
as its entry-wise $\ell_{\infty}$ norm, and $\opnorm{\M}$ as its
spectral norm (maximum singular value).  We let $\I$ and $\OneMat$
be the $\num\times\num$ identity matrix and all-one matrix, respectively.
For a real number $x$, $\left\lceil x\right\rceil $ denotes its
ceiling. We denote by $\clustset a\coloneqq\left\{ i\in\left[\num\right]:\labelstar(i)=a\right\} $
the set of indices of points in cluster $a$, and we define $\size\coloneqq\left|\clustset a\right|=\frac{\num}{\numclust}$. 

\section{Proof of Theorem \ref{thm:ip_sdp}\label{sec:proof_ip_sdp}}

\subsection{Initial steps}

We assume $\error>0$ since otherwise we are done. We can write $\Adj=\C+\C\t-2\H\H\t$,
where $\H$ is a matrix whose $i$-th row is the point $\h_{i}$ and
$\C$ is a matrix where the entries in the $i$-th row are identical
and equal to $\norm[\h_{i}]2^{2}$. Since the row-sum constraint in
the program (\ref{eq:SDP1}) ensures that the matrix $\Yhat-\Ystar$
has zero row sum, we have $\left\langle \Yhat-\Ystar,\C\right\rangle =\left\langle \Yhat-\Ystar,\C\t\right\rangle =0$
which implies $\left\langle \Yhat-\Ystar,\C+\C\t\right\rangle =0$.

Let $\G\coloneqq\H-\E\H$ be a matrix of entries in $\H$ with their
means removed. We can compute
\begin{align*}
\H\H\t & =\left(\G+\E\H\right)\left(\G+\E\H\right)\t\\
 & =\G\G\t+\G\left(\E\H\right)\t+\left(\E\H\right)\G\t+\left(\E\H\right)\left(\E\H\right)\t
\end{align*}
and 
\[
\E\H\H^{\top}=\E\G\G\t+\left(\E\H\right)\left(\E\H\right)\t.
\]
Therefore 
\[
\H\H\t-\E\H\H^{\top}=\left(\G\G\t-\E\G\G\t\right)+\G\left(\E\H\right)\t+\left(\E\H\right)\G\t.
\]
Let $\U\in\real^{\num\times\numclust}$ be the matrix of the left
singular vectors of $\Ystar$. For any $\M\in\real^{\num\times\num}$,
define the projection $\PT\left(\M\right)\coloneqq\U\U\t\M+\M\U\U\t-\U\U\t\M\U\U\t$
and its orthogonal complement $\PTperp\left(\M\right)\coloneqq\M-\PT\left(\M\right)$.
The fact that $\Yhat$ is optimal and $\Ystar$ is feasible to the
program (\ref{eq:SDP1}) implies 
\begin{align*}
0 & \leq-\frac{1}{2}\left\langle \Yhat-\Ystar,\Adj\right\rangle \\
 & =\left\langle \Yhat-\Ystar,\H\H\t-\E\H\H^{\top}\right\rangle +\left\langle \Yhat-\Ystar,\E\H\H^{\top}\right\rangle \\
 & =\left\langle \Yhat-\Ystar,\G\G\t-\E\G\G\t+\G\left(\E\H\right)\t+\left(\E\H\right)\G\t\right\rangle +\left\langle \Yhat-\Ystar,\E\H\H^{\top}\right\rangle \\
 & =\left\langle \Yhat-\Ystar,\PT\left(\G\G\t-\E\G\G\t\right)\right\rangle +\left\langle \Yhat-\Ystar,\PTperp\left(\G\G\t-\E\G\G\t\right)\right\rangle \\
 & \quad+2\left\langle \Yhat-\Ystar,\G\left(\E\H\right)\t\right\rangle +\left\langle \Yhat-\Ystar,\E\H\H^{\top}\right\rangle \\
 & \eqqcolon S_{1}+S_{2}+2S_{3}+S_{4}.
\end{align*}
We may control $S_{1}$, $S_{2}$ and $S_{4}$ using the following. 
\begin{prop}
\label{prop:S1} If $\snr^{2}\geq C\left(\sqrt{\frac{\numclust\vecdim}{\num}\log\left(\num\numclust\right)}+\sqrt{\frac{\numclust}{\num}}\log\left(\num\numclust\right)\right)$
for some universal constant $C>0$, then $S_{1}\leq\frac{1}{100}\minsep^{2}\error$
with probability at least $1-\left(2\num\right)^{-2}$.
\end{prop}

\begin{prop}
\label{prop:S2} If $\snr^{2}\geq C\numclust\left(\sqrt{\frac{\vecdim}{\num}}+1\right)$
for some universal constant $C>0$, then $S_{2}\leq\frac{1}{100}\minsep^{2}\error$
with probability at least $1-2e^{-\num}$.
\end{prop}

\begin{prop}
\label{prop:S4} We have $S_{4}=-\frac{1}{2}\sum_{a\ne b}T_{ab}\minsep_{ab}^{2}\le-\frac{1}{4}\minsep^{2}\error$
where $T_{ab}\coloneqq\sum_{i\in\clustset a,j\in\clustset b}\left(\Yhat-\Ystar\right)_{ij}$. 
\end{prop}
The proofs are given in Sections \ref{sec:proof_S1}, \ref{sec:proof_S2}
and \ref{sec:proof_S4}, respectively. Combining the above propositions,
we have $S_{1}+S_{2}\le-\frac{1}{2}S_{4}$ and therefore 
\begin{equation}
0\leq S_{3}+\frac{1}{4}S_{4}\eqqcolon S_{0}\label{eq:error_S3_bound}
\end{equation}
with probability at least $1-\left(2\num\right)^{-C'}-2e^{-\num}$
for some universal constant $C'>0$.

Let $\B\coloneqq\Yhat-\Ystar$. We have 
\begin{align*}
S_{3} & =\sum_{j}\sum_{a}\sum_{i\in C_{a}}B_{ji}\left\langle \Mean_{a},\g_{j}\right\rangle \\
 & =\size\sum_{j}\sum_{a}\left\langle \Mean_{a},\g_{j}\right\rangle \left(\frac{1}{\size}\sum_{i\in\clustset a}B_{ji}\right)\\
 & =\size\sum_{j}\sum_{a\ne\labelstar(j)}\left\langle \Mean_{a}-\Mean_{\labelstar(j)},\g_{j}\right\rangle \left(\frac{1}{\size}\sum_{i\in\clustset a}B_{ji}\right)
\end{align*}
where the last step holds since $\sum_{a\ne\labelstar(j)}\left(\sum_{i\in\clustset a}B_{ji}\right)=-\sum_{i\in\clustset a:a=\labelstar(j)}B_{ji}$
for each $j\in\left[\num\right]$ which follows from the row-sum constraint
of program (\ref{eq:SDP1}). By Proposition \ref{prop:S4}, we have
\begin{align*}
S_{4} & =-\size\sum_{j}\sum_{a\ne\labelstar(j)}\frac{1}{2}\minsep_{\labelstar(j),a}^{2}\left(\frac{1}{\size}\sum_{i\in\clustset a}B_{ji}\right).
\end{align*}
Therefore, we have 
\[
S_{0}=\size\sum_{j}\sum_{a\ne\labelstar(j)}\left(\left\langle \Mean_{a}-\Mean_{\labelstar(j)},\g_{j}\right\rangle -c\minsep_{\labelstar(j),a}^{2}\right)\left(\frac{1}{\size}\sum_{i\in\clustset a}B_{ji}\right)
\]
where $c=\frac{1}{8}$.

To control $S_{0}$, we define $\beta_{ja}\coloneqq\left\langle \Mean_{a}-\Mean_{\labelstar(j)},\g_{j}\right\rangle -c\minsep_{\labelstar(j),a}^{2}$
and consider the program 
\begin{align}
\max_{\X}\  & \sum_{j}\sum_{a\ne\labelstar(j)}\beta_{ja}X_{ja}\nonumber \\
\text{s.t.}\  & 0\leq X_{ja}\leq1,\qquad\forall a\ne\labelstar(j),j\in\left[\num\right]\nonumber \\
 & \sum_{a\ne\labelstar(j)}X_{ja}\leq1,\qquad\forall j\in\left[\num\right]\label{eq: int_opt}\\
 & \sum_{j}\sum_{a\ne\labelstar(j)}X_{ja}=R,\nonumber 
\end{align}
where $R\in(0,\num]$. Let us denote by $V(R)$ the optimal value
of the above program and we let $V(R)=-\infty$ if the program is
infeasible. The constraints of program (\ref{eq:SDP1}) implies that
$\frac{\error}{2\size}\in(0,\num]$ and 
\[
\sum_{j\in\left[\num\right]}\sum_{a\ne\labelstar(j)}\left(\sum_{i\in\clustset a}B_{ji}\right)=\frac{\error}{2}.
\]
Hence, by Equation (\ref{eq:error_S3_bound}), we have 
\begin{equation}
0\leq S_{0}\leq\size\cdot V\left(\frac{\error}{2\size}\right).\label{eq:basic_ineq_upper_bound_V}
\end{equation}


\subsection{Controlling $\protect\error$ by LP}

We show that $\error$ is upper bounded by the objective value of
an LP that is related to program (\ref{eq: int_opt}). If $\error=0$
then the conclusion of Theorem \ref{thm:ip_sdp} holds trivially.
For $\error>0$, we have the following cases:
\begin{enumerate}
\item If $\frac{\error}{2\size}\in(0,1]$, it follows from Equation (\ref{eq:basic_ineq_upper_bound_V})
that the error $\error$ must satisfy 
\[
0\le V\left(\frac{\error}{2\size}\right)=\beta^{*}\frac{\error}{2\size}\le\beta^{*}\left\lceil \frac{\error}{2\size}\right\rceil =V\left(\left\lceil \frac{\error}{2\size}\right\rceil \right)
\]
where $\beta^{*}\coloneqq\max_{j\in\left[\num\right],a\ne\labelstar(j)}\beta_{ja}$.
This implies 
\begin{align*}
\frac{\error}{2\size}\le\left\lceil \frac{\error}{2\size}\right\rceil  & \le\max\left\{ R\in\{0,1,.\ldots\}:V(R)\ge0\right\} .
\end{align*}
\item If $\frac{\error}{2\size}>1$, it follows from Equation (\ref{eq:basic_ineq_upper_bound_V})
that the error $\error$ must satisfy 
\[
0\le V\left(\frac{\error}{2\size}\right)\le\max\left\{ V\left(\left\lceil \frac{\error}{2\size}\right\rceil \right),V\left(\left\lfloor \frac{\error}{2\size}\right\rfloor \right)\right\} =\max\left\{ V\left(\left\lceil \frac{\error}{2\size}\right\rceil \right),V\left(\left\lceil \frac{\error}{2\size}\right\rceil -1\right)\right\} .
\]
In other words, we have
\begin{align*}
\frac{\error}{2\size}\le\left\lceil \frac{\error}{2\size}\right\rceil  & \le\max\left\{ R\in\{0,1,.\ldots\}:V(R)\vee V(R-1)\ge0\right\} \\
 & =1+\max\left\{ R\in\{0,1,.\ldots\}:V(R)\ge0\right\} .
\end{align*}
Note that $\left\lceil \frac{\error}{2\size}\right\rceil \ge2$, and
therefore we must have $1\le\max\left\{ R\in\{0,1,.\ldots\}:V(R)\ge0\right\} $.
This implies 
\[
\frac{\error}{2\size}\le2\max\left\{ R\in\{0,1,.\ldots\}:V(R)\ge0\right\} .
\]
\end{enumerate}
Consequently, we have 
\[
\frac{\error}{2\size}\le2\max\left\{ R\in\{0,1,.\ldots\}:V(R)\ge0\right\} .
\]


\subsection{Converting LP to IP}

We are now ready to formally establish a connection between the error
of the SDP (\ref{eq:SDP1}) and that of the Oracle IP (\ref{eq:oracleIP}),
by relating $\max\left\{ R\in\{0,1,.\ldots\}:V(R)\ge0\right\} $ to
the quantity (\ref{eq:IPerror}). Note that if $R\ge0$ is an integer,
then there exists an optimal solution $\left\{ w_{ja}\right\} $ of
program (\ref{eq: int_opt}) such that $w_{ja}\in\{0,1\}$ for all
$j\in[\num],a\in[\numclust]$. Therefore, if $R\in\{0,1,\ldots\}$
is an integer, then 
\begin{equation}
V(R)=\IP_{1}(R)\coloneqq\left\{ \begin{aligned}\max_{\X}\  & \sum_{j}\sum_{a\ne\labelstar(j)}\beta_{ja}X_{ja}\\
\text{s.t.}\  & X_{ja}\in\{0,1\},\qquad\forall a\ne\labelstar(j),j\in\left[\num\right]\\
 & \sum_{a\ne\labelstar(j)}X_{ja}\leq1,\qquad\forall j\in\left[\num\right]\\
 & \sum_{j}\sum_{a\ne\labelstar(j)}X_{ja}=R
\end{aligned}
\right\} .\label{eq:IP1}
\end{equation}
Combining the last two display equations we obtain that
\begin{align}
\frac{\error}{2\size} & \le2\max\left\{ R\in\{0,1,.\ldots\}:\IP_{1}(R)\ge0\right\} \nonumber \\
 & \overset{}{=}2\cdot\left\{ \begin{aligned}\max_{R,\X}\; & R\\
\text{s.t.}\; & R\in\{0,1,\ldots\}\\
 & \sum_{j}\sum_{a\ne\labelstar(j)}\beta_{ja}X_{ja}\ge0\\
 & X_{ja}\in\{0,1\},\qquad\forall a\ne\labelstar(j),j\in\left[\num\right]\\
 & \sum_{a\ne\labelstar(j)}X_{ja}\leq1,\qquad\forall j\in\left[\num\right]\\
 & \sum_{j}\sum_{a\ne\labelstar(j)}X_{ja}=R,
\end{aligned}
\right\} \nonumber \\
 & =2\cdot\IP_{2}\coloneqq2\cdot\left\{ \begin{aligned}\max_{\X}\; & \sum_{j}\sum_{a\ne\labelstar(j)}X_{ja}\\
\text{s.t.}\; & \sum_{j}\sum_{a\ne\labelstar(j)}\beta_{ja}X_{ja}\ge0\\
 & X_{ja}\in\{0,1\},\qquad\forall a\ne\labelstar(j),j\in\left[\num\right]\\
 & \sum_{a\ne\labelstar(j)}X_{ja}\leq1,\qquad\forall j\in\left[\num\right]
\end{aligned}
\right\} .\label{eq:error_bound2}
\end{align}

Let us reparameterize the integer program $\IP_{2}$ by a change of
variable. Recall that 
\[
\mathcal{F}\coloneqq\left\{ \F\in\{0,1\}^{\num\times\numclust}:\F\one_{\numclust}=\one_{\num}\right\} 
\]
is the set of all possible assignment matrices and $\F^{*}\in\mathcal{F}$
is the true assignment matrix; that is, $F_{ja}^{*}=\indic\left\{ a=\labelstar(j)\right\} $
for all $j\in[\num],a\in[\numclust]$. Consider any feasible solution
$\X$ of $\IP_{2}$; here for each $j\in[\num]$, we may fix $X_{j,\labelstar(j)}=-\sum_{a\neq\labelstar(j)}X_{ja}$
\textemdash{} doing so does not affect the feasibility and objective
value of $\X$ w.r.t. $\IP_{2}$. Define the new variable $\F\coloneqq\F^{*}+\X\in\mathcal{F}$.
The objective value and constraints of the old variable $\X$ can
be mapped to those of $\F$; in particular, we have 
\begin{align*}
\sum_{j}\sum_{a\ne\labelstar(j)}X_{ja} & =\sum_{j}\sum_{a\ne\labelstar(j)}(F_{ja}-F_{ja}^{*})=\frac{1}{2}\norm[\F-\F^{*}]1
\end{align*}
and
\begin{align*}
\left.\begin{array}{c}
X_{ja}\in\{0,1\},\forall a\ne\labelstar(j),j\in\left[\num\right]\\
\sum_{a\ne\labelstar(j)}X_{ja}\leq1,\forall j\in\left[\num\right]\\
X_{j,\labelstar(j)}=-\sum_{a\neq\labelstar(j)}X_{ja},\forall j\in[\num]
\end{array}\right\}  & \Longleftrightarrow\F\in\mathcal{F}
\end{align*}
and
\[
\sum_{j}\sum_{a\ne\labelstar(j)}\beta_{ja}X_{ja}\overset{(i)}{=}\sum_{j}\sum_{a}\beta_{ja}X_{ja}=\sum_{j}\sum_{a}\beta_{ja}F_{ja}-\sum_{j}\sum_{a}\beta_{ja}F_{ja}^{*}\overset{(ii)}{=}\sum_{j}\sum_{a}\beta_{ja}F_{ja},
\]
where steps $(i)$ and $(ii)$ both follow from the fact that $\beta_{j,\labelstar(j)}=0,\forall j.$
It follows that $\IP_{2}$ has the same optimal value as a corresponding
integer program in terms of $\X$; in particular, we have
\[
\IP_{2}=\IP_{3}\coloneqq\left\{ \begin{aligned}\max_{\F}\; & \frac{1}{2}\norm[\F-\F^{*}]1\\
\text{s.t.}\; & \sum_{j}\sum_{a}\beta_{ja}F_{ja}\ge0\\
 & \F\in\mathcal{F}
\end{aligned}
\right\} .
\]
Combining with equation (\ref{eq:error_bound2}), we see that the
error $\error$ satisfies
\begin{equation}
\frac{\error}{2\size}\le2\cdot\IP_{3}.\label{eq:error_bound3}
\end{equation}

We further simplify the first constraint in $\IP_{3}$. Recall that
$\bar{\h}_{i}\coloneqq\Mean_{\labelstar(i)}+(2c)^{-1}\g_{i}$ for
each $i\in[\num]$. Note that $\left\{ \bar{\h}_{i}\right\} $ can
be viewed as data points generated from the Sub-Gaussian Mixture Model
but with $(2c)^{-1}$ times the standard deviation. By definition
of $\beta_{ja}$, we have 
\begin{align*}
\beta_{ja} & =\left\langle \Mean_{a}-\Mean_{\labelstar(j)},\g_{j}\right\rangle -c\minsep_{\labelstar(j),a}^{2}\\
 & =c\left(2\left\langle \Mean_{a}-\Mean_{\labelstar(j)},(2c)^{-1}\g_{j}\right\rangle -\minsep_{\labelstar(j),a}^{2}\right)\\
 & =c\left(2\left\langle \Mean_{a}-\Mean_{\labelstar(j)},(2c)^{-1}\g_{j}\right\rangle -\norm[\Mean_{a}-\Mean_{\labelstar(j)}]2^{2}\right)\\
 & =c\left(2\left\langle \Mean_{a}-\Mean_{\labelstar(j)},(2c)^{-1}\g_{j}\right\rangle -\norm[\Mean_{a}-\Mean_{\labelstar(j)}]2^{2}-\norm[(2c)^{-1}\g_{j}]2^{2}+\norm[(2c)^{-1}\g_{j}]2^{2}\right)\\
 & =c\left(-\norm[\Mean_{\labelstar(j)}-\Mean_{a}+(2c)^{-1}\g_{j}]2^{2}+\norm[(2c)^{-1}\g_{j}]2^{2}\right)\\
 & =c\left(-\norm[\bar{\h}_{j}-\Mean_{a}]2^{2}+\norm[(2c)^{-1}\g_{j}]2^{2}\right).
\end{align*}
For any $\F\in\mathcal{F}$, we then have
\begin{align*}
\sum_{j}\sum_{a}\beta_{ja}F_{ja} & =c\sum_{j}\sum_{a}\left(-\norm[\bar{\h}_{j}-\Mean_{a}]2^{2}+\norm[(2c)^{-1}\g_{j}]2^{2}\right)F_{ja}\\
 & =c\left(-\sum_{j}\sum_{a}\norm[\bar{\h}_{j}-\Mean_{a}]2^{2}F_{ja}+\sum_{j}\norm[(2c)^{-1}\g_{j}]2^{2}\sum_{a}F_{ja}\right)\\
 & \overset{(i)}{=}c\left(-\sum_{j}\sum_{a}\norm[\bar{\h}_{j}-\Mean_{a}]2^{2}F_{ja}+\sum_{j}\norm[(2c)^{-1}\g_{j}]2^{2}\sum_{a}F_{ja}^{*}\right)\\
 & =c\left(-\sum_{j}\sum_{a}\norm[\bar{\h}_{j}-\Mean_{a}]2^{2}F_{ja}+\sum_{j}\sum_{a}\norm[(2c)^{-1}\g_{j}]2^{2}F_{ja}^{*}\right)\\
 & =c\left(-\sum_{j}\sum_{a}\norm[\bar{\h}_{j}-\Mean_{a}]2^{2}F_{ja}+\sum_{j}\sum_{a}\norm[\bar{\h}_{j}-\Mean_{\labelstar(j)}]2^{2}F_{ja}^{*}\right)\\
 & \overset{(ii)}{=}c\left(-\sum_{j}\sum_{a}\norm[\bar{\h}_{j}-\Mean_{a}]2^{2}F_{ja}+\sum_{j}\sum_{a}\norm[\bar{\h}_{j}-\Mean_{a}]2^{2}F_{ja}^{*}\right),
\end{align*}
where step $(i)$ holds because $\sum_{a}F_{ja}=1=\sum_{a}F_{ja}^{*},\forall j$,
and step $(ii)$ holds because $F_{ja}^{*}=1$ only if $a=\labelstar(j)$.
Again recall the shorthand
\[
\eta(\F)\coloneqq\sum_{j}\sum_{a}\norm[\bar{\h}_{j}-\Mean_{a}]2^{2}F_{ja}.
\]
We have the more compact expression
\begin{equation}
\sum_{j}\sum_{a}\beta_{ja}F_{ja}=c\left(\eta(\F^{*})-\eta(\F)\right)\label{eq:eta_func_equivalence}
\end{equation}
It follows that for any $\F\in\mathcal{F}$, the first constraint
in $\IP_{3}$ is satisfied if and only if 
\[
\eta(\F)\le\eta(\F^{*}).
\]
Combining with the (\ref{eq:error_bound3}), we obtain that
\[
\frac{\error}{2\size}\le2\cdot\IP_{3}=2\cdot\left\{ \begin{aligned}\max_{\F}\; & \frac{1}{2}\norm[\F-\F^{*}]1\\
\text{s.t.}\; & \eta(\F)\le\eta(\F^{*})\\
 & \F\in\mathcal{F}
\end{aligned}
\right\} .
\]
Rearranging terms, we have the bound
\begin{equation}
\error\le2\size\cdot\max\left\{ \norm[\F-\F^{*}]1:\eta(\F)\le\eta(\F^{*}),\F\in\mathcal{F}\right\} .\label{eq:error_bound3a}
\end{equation}
The result follows from the fact that $\norm[\Ystar]1=\num\size$
and $\norm[\F^{*}]1=\num$. 

\subsection{Proof of Proposition \ref{prop:S1} \label{sec:proof_S1}}

In this section we control $S_{1}$. We can further decompose $S_{1}$
as 
\begin{align*}
S_{1} & =\left\langle \Yhat-\Ystar,\U\U\t\left(\G\G\t-\E\G\G\t\right)\right\rangle +\left\langle \Yhat-\Ystar,\left(\G\G\t-\E\G\G\t\right)\U\U\t\right\rangle \\
 & \qquad-\left\langle \Yhat-\Ystar,\U\U\t\left(\G\G\t-\E\G\G\t\right)\U\U\t\right\rangle \\
 & \leq2\left|\left\langle \Yhat-\Ystar,\U\U\t\left(\G\G\t-\E\G\G\t\right)\right\rangle \right|+\left|\left\langle \Yhat-\Ystar,\U\U\t\left(\G\G\t-\E\G\G\t\right)\U\U\t\right\rangle \right|\\
 & \eqqcolon2T_{1}+T_{2}
\end{align*}
By the generalized Holder's inequality, we have 
\begin{align*}
T_{1} & \leq\error\cdot\norm[\U\U\t\left(\G\G\t-\E\G\G\t\right)]{\infty}
\end{align*}
and 
\begin{align*}
T_{2} & =\left|\left\langle \Yhat-\Ystar,\U\U\t\left(\G\G\t-\E\G\G\t\right)\U\U\t\right\rangle \right|\\
 & =\left|\left\langle \left(\Yhat-\Ystar\right)\U\U\t,\U\U\t\left(\G\G\t-\E\G\G\t\right)\right\rangle \right|\\
 & \leq\error\cdot\norm[\U\U\t\left(\G\G\t-\E\G\G\t\right)]{\infty}
\end{align*}
where the last inequality holds since 
\[
\norm[\left(\Yhat-\Ystar\right)\U\U\t]1\leq\norm[\Yhat-\Ystar]1=\error.
\]
Combining the above, we have 
\[
S_{1}\leq3\error\cdot\norm[\U\U\t\left(\G\G\t-\E\G\G\t\right)]{\infty}.
\]

Note that there are $m=\num\numclust$ distinct random variables in
$\U\U\t\left(\G\G\t-\E\G\G\t\right)$ and let us call them $X_{1},\ldots,X_{m}$.
For each $i$, we can see that $X_{i}$ is the average of $\size$
entries in $\G\G\t-\E\G\G\t$ and we let $\B_{i}$ be an $\num\times\num$
matrix with $\size$ entries equal to 1 and the others equal to 0
such that $\size X_{i}=\left\langle \B_{i},\G\G\t-\E\G\G\t\right\rangle $.
To proceed, we need the Hanson-Wright inequality (an extension of
Exercise 6.2.7 on pp.$\ $140 in \citet{vershynin2017high}).
\begin{lem}[Higher-dimensional Hanson-Wright inequality]
\emph{ \label{lem:hanson-wright} }Let $\x_{1},\ldots,\x_{N}$ be
independent, mean zero, sub-Gaussian random vectors in $\real^{M}$.
Let $\B$ be an $N\times N$ matrix. For every $t\geq0$ and some
universal constant $c>0$, we have 
\[
\P\left[\left|\sum_{i,j}B_{ij}\left\langle \x_{i},\x_{j}\right\rangle -\E\sum_{i,j}B_{ij}\left\langle \x_{i},\x_{j}\right\rangle \right|\geq t\right]\leq4\exp\left[-c\min\left(\frac{t^{2}}{K^{4}M\norm[\B]F^{2}},\frac{t}{K^{2}\norm[\B]{}}\right)\right]
\]
where $K\coloneqq\max_{i}\norm[\x_{i}]{\psi_{2}}.$ 
\end{lem}
The proof is given in Section \ref{sec:proof_hanson_wright}. Using
Lemma \ref{lem:hanson-wright}, we see that for any $t\ge0$ 
\[
\P\left\{ \size X_{i}\geq t\right\} =\P\left\{ \left\langle \B_{i},\G\G\t-\E\G\G\t\right\rangle \geq t\right\} \leq4\exp\left[-c\min\left(\frac{t^{2}}{K^{4}\vecdim\size},\frac{t}{K^{2}\sqrt{\size}}\right)\right].
\]
We can choose $t^{*}=DK^{2}\sqrt{\size}\left(\sqrt{\vecdim\log m}+\log m\right)$
with $K=\sgnorm$ and $D>0$ a universal constant. Apply the union
bound, we have 
\[
S_{1}\leq3\error\cdot\frac{1}{\size}\cdot t^{*}
\]
with probability at least $1-m\cdot\P\left\{ \size X\geq t\right\} \geq1-\exp\left(-C'\log m\right)=1-m^{-C'}$
where $C'>0$ is a universal constant. The result follows from the
condition of the proposition.

\subsection{Proof of Proposition \ref{prop:S2} \label{sec:proof_S2}}

In this section we control $S_{2}$. We have 
\begin{align*}
S_{2} & =\left\langle \PTperp\left(\Yhat-\Ystar\right),\G\G\t-\E\G\G\t\right\rangle \\
 & \leq\Tr\left[\PTperp\left(\Yhat-\Ystar\right)\right]\cdot\opnorm{\G\G\t-\E\G\G\t}\\
 & \le\frac{\error}{\size}\cdot\opnorm{\G\G\t-\E\G\G\t}.
\end{align*}
Let $\Var\left(g_{ij}\right)=\std^{2}$. We record a fact about the
sub-Gaussian property of columns of $\G$. 
\begin{fact}
\label{fact:satisfy_cond_gauss_choas_operator_norm_bound} Let $\x\in\real^{\num}$
be an arbitrary column of $\G$. We have 
\[
\norm[\left\langle \x,\w\right\rangle ]{\psi_{2}}\leq C\frac{\sgnorm}{\std}\sqrt{\E\left\langle \x,\w\right\rangle ^{2}}\qquad\text{for any }\w\in\real^{\num},
\]
where $C>0$ is a universal constant and $C\frac{\sgnorm}{\std}\ge1$.
\end{fact}
The proof is given in Section \ref{sec:proof_satisfy_cond}. Applying
Lemma \ref{lem:subg_cov_mat_bound} with $\rho_{0}=\frac{\sgnorm}{\std}$,
we have 
\[
\opnorm{\frac{1}{d}\G\G\t-\frac{1}{d}\E\G\G\t}\leq C_{1}\rho_{0}^{2}\left(\sqrt{\frac{2\num}{\vecdim}}+\frac{2\num}{\vecdim}\right)\opnorm{\frac{1}{d}\E\G\G\t}
\]
with probability at least $1-2e^{-\num}$. Here we let $m=\vecdim,u=\num$
and define $\x_{i}$ to be the $i$-th column of $\G$ and $\x$ to
be a vector independent of but identically distributed as each column of $\G$ (note that columns of $\G$ are identically
distributed). We also use the fact that $\E\x\x\t=\frac{1}{\vecdim}\E\G\G\t=\std^{2}\I$.
Multiplying $\vecdim$ on both sides of the above equation yields
\[
\opnorm{\G\G\t-\E\G\G\t}\leq C_{1}\left(\sqrt{\frac{2\num}{\vecdim}}+\frac{2\num}{\vecdim}\right)\vecdim\sgnorm^{2}.
\]
Hence, we have 
\[
S_{2}\le\frac{\error}{\size}\cdot C_{1}\left(\sqrt{\frac{2\num}{\vecdim}}+\frac{2\num}{\vecdim}\right)\vecdim\sgnorm^{2}=2C_{1}\error\numclust\left(\sqrt{\frac{\vecdim}{\num}}+1\right)\frac{\minsep^{2}}{\snr^{2}}
\]
The result follows from the condition of the proposition.

\subsection{Proof of Proposition \ref{prop:S4} \label{sec:proof_S4}}

We can compute 
\[
\left(\E\H\H\t\right)_{ij}=\begin{cases}
\vecdim\std^{2}+\norm[\Mean_{\labelstar(i)}]2^{2} & \text{if }i=j\\
\norm[\Mean_{\labelstar(i)}]2^{2} & \text{if }i\ne j\text{ and }\labelstar(i)=\labelstar(j)\\
\left\langle \Mean_{\labelstar(i)},\Mean_{\labelstar(j)}\right\rangle  & \text{otherwise}.
\end{cases}
\]
We partition the matrix $\Yhat-\Ystar$ into $\numclust^{2}$ of $\size\times\size$
blocks, and note that $T_{ab}$ denotes the sum of entries within
the $(a,b)$-th block. The constraints of program (\ref{eq:SDP1})
implies that 
\begin{enumerate}
\item $T_{aa}\leq0$ for each $a\in\left[\numclust\right]$ and $T_{ab}\geq0$
for each $a\ne b\in\left[\numclust\right]$;
\item $T_{ab}=T_{ba}$ for each $a,b\in\left[\numclust\right]$;
\item $-T_{aa}=\sum_{b\in\left[\numclust\right]:b\ne a}T_{ab}$ for each
$a\in\left[\numclust\right]$;
\item $-\sum_{a\in\left[\numclust\right]}T_{aa}+\sum_{a,b\in\left[\numclust\right]:a\ne b}T_{ab}=\error$
and thus $-\sum_{a\in\left[\numclust\right]}T_{aa}=\sum_{a,b\in\left[\numclust\right]:a\ne b}T_{ab}=\frac{\error}{2}$.
\end{enumerate}
Since $\Yhat-\Ystar$ has zero diagonal, we can write 
\begin{align*}
S_{4} & =\sum_{a\in\left[\numclust\right]}T_{aa}\norm[\Mean_{a}]2^{2}+2\sum_{a,b\in\left[\numclust\right]:a<b}T_{ab}\left\langle \Mean_{a},\Mean_{b}\right\rangle \\
 & =-\sum_{a,b\in\left[\numclust\right]:a<b}T_{ab}\minsep_{ab}^{2}\\
 & =-\frac{1}{2}\sum_{a,b\in\left[\numclust\right]:a\ne b}T_{ab}\minsep_{ab}^{2}\\
 & \leq-\frac{1}{2}\sum_{a,b\in\left[\numclust\right]:a\ne b}T_{ab}\minsep^{2}\\
 & =-\frac{1}{4}\minsep^{2}\error.
\end{align*}


\section{Proof of Theorem \ref{thm:ip_exp_rate}\label{sec:proof_ip_exp_rate}}

We define the shorthand 
\[
\iperror\coloneqq\max\left\{ \frac{1}{2}\norm[\F-\F^{*}]1:\eta(\F)\le\eta(\F^{*}),\F\in\mathcal{F}\right\} .
\]
It is not hard to see that $\iperror$ takes integer values in $[0,\num]$.
If $\iperror=0$ then we are done. We therefore focus on the case
$\iperror\in\left[\num\right]$.

Suppose $\iperror>3\num\numclust e^{-\snr^{2}/C_{0}^{2}}$ for a fixed
$C_{0}>D/c$. Note that 
\[
3\num\numclust e^{-\snr^{2}/C_{0}^{2}}\overset{(i)}{\le}\num\numclust\cdot\frac{1}{\numclust}\cdot e^{-\snr^{2}/\left(2C_{0}^{2}\right)}\le\num e^{-\snr^{2}/\left(2C_{0}^{2}\right)}<\num
\]
where step $(i)$ holds since we have assumed $\snr^{2}\ge\consts\numclust$
for some universal constant $\consts>0$. We record an important result
for our proof.
\begin{lem}
\label{lm:order_stats} Let $m\ge4$ and $g\ge1$ be integers. Let
$\X\in\real^{m\times g}$ be a matrix such that each $X_{ja}$ is
a sub-Gaussian random variable with its mean equal to $\lambda_{ja}$
and its sub-Gaussian norm no larger than $\rho_{ja}$, and each pair
$X_{ja}$ and $X_{ib}$ are independent for $j\ne i$ and $a,b\in\left[g\right]$.
Then for some universal constant $D>0$ and for any $\beta\in(0,m]$,
we have 
\begin{align*}
\sum_{j,a}X_{ja}M_{ja} & \le D\sqrt{\left\lceil \beta\right\rceil \left(\sum_{j,a}\rho_{ja}^{2}M_{ja}\right)\log\left(3mg/\beta\right)}+\sum_{j,a}\lambda_{ja}M_{ja},\\
 & \qquad\quad\forall\M\in\left\{ 0,1\right\} ^{m\times g}:\M\one_{g}\le\one_{m},\norm[\M]1=\left\lceil \beta\right\rceil ,
\end{align*}
with probability at least $1-\frac{1.5}{m}$.
\end{lem}
The proof is given in Section \ref{sec:proof_lm_order_stat}. Define
the set 
\[
\calM\coloneqq\left\{ \M\in\left\{ 0,1\right\} ^{\num\times\numclust}:\M\one_{\numclust}\le\one_{\num},\norm[\M]1=\iperror,M_{j,\labelstar(j)}=0\ \forall j\in\left[\num\right]\right\} .
\]
For any $\F$ feasible to $\IP_{3}$, we have 
\begin{align*}
0 & \le\frac{1}{c}\left(\eta(\F^{*})-\eta(\F)\right)\\
 & \overset{(i)}{=}\sum_{j\in[\num]}\sum_{a\in[\numclust]}\beta_{ja}F_{ja}\\
 & =\sum_{\left(j,a\right):F_{ja}=1,a\ne\labelstar(j)}\beta_{ja}\\
 & \le\max_{\M\in\calM}\sum_{j}\sum_{a\ne\labelstar(j)}\beta_{ja}M_{ja}\\
 & \overset{(ii)}{\le}\max_{\M\in\calM}\left[D\sqrt{\iperror\sgnorm^{2}\left(\sum_{j}\sum_{a\ne\labelstar(j)}\minsep_{\labelstar(j),a}^{2}M_{ja}\right)\log\left(3\num\left(\numclust-1\right)/\iperror\right)}-c\sum_{j}\sum_{a\ne\labelstar(j)}\minsep_{\labelstar(j),a}^{2}M_{ja}\right]\\
 & \le\max_{\M\in\calM}\left[D\sqrt{\iperror\sgnorm^{2}\left(\sum_{j}\sum_{a\ne\labelstar(j)}\minsep_{\labelstar(j),a}^{2}M_{ja}\right)\frac{\snr^{2}}{C_{0}^{2}}}-c\sum_{j}\sum_{a\ne\labelstar(j)}\minsep_{\labelstar(j),a}^{2}M_{ja}\right]\\
 & \le\left(\frac{D}{C_{0}}-c\right)\cdot\max_{\M\in\calM}\sum_{j}\sum_{a\ne\labelstar(j)}\minsep_{\labelstar(j),a}^{2}M_{ja}
\end{align*}
where step $(i)$ holds by Equation (\ref{eq:eta_func_equivalence}),
step $(ii)$ holds by Lemma \ref{lm:order_stats} with $g=\numclust-1$
since only $\numclust-1$ entries of $\left\{ \beta_{ja}\right\} $
are considered for each $j$ in the sum above $(ii)$, and the last
step holds since $\iperror\minsep^{2}\le\sum_{j}\sum_{a\ne\labelstar(j)}\minsep_{\labelstar(j),a}^{2}M_{ja}$.
Since $C_{0}>D/c$ and $\sum_{j}\sum_{a\ne\labelstar(j)}\minsep_{\labelstar(j),a}^{2}M_{ja}>0$,
the RHS above is negative, which is a contradiction. Hence, we must
have $\iperror\le3\num\numclust e^{-\snr^{2}/C_{0}^{2}}\le\num e^{-\snr^{2}/\left(2C_{0}^{2}\right)}$
and the result follows from the fact that $\norm[\F^{*}]1=\num$.

\section{Proof of technical results}

In this section we provide the proofs of the technical results used
in the proofs of our main theorems.

\subsection{Proof of Lemma \ref{lem:hanson-wright}\label{sec:proof_hanson_wright}}

We record the following lemma (Exercise 6.2.7 on pp.$\ $140 in \citet{vershynin2017high}).
\begin{lem}[Higher-dimensional Hanson-Wright inequality]
\emph{} \label{lem:hanson_wright_hdp} Let $\x_{1},\ldots,\x_{N}$
be independent, mean zero, sub-Gaussian random vectors in $\real^{M}$.
Let $\B=\left\{ B_{ij}\right\} $ be an $N\times N$ matrix. There
exists some universal constant $c>0$ such that for every $t\geq0$
\[
\P\left[\left|\sum_{i,j:i\ne j}^{N}B_{ij}\left\langle \x_{i},\x_{j}\right\rangle \right|\geq t\right]\leq2\exp\left[-c\min\left(\frac{t^{2}}{K^{4}M\norm[\B]F^{2}},\frac{t}{K^{2}\opnorm{\B}}\right)\right]
\]
where $K\coloneqq\max_{i}\norm[\x_{i}]{\psi_{2}}.$ 
\end{lem}
With this result, we only need to prove the same tail bound for $\P\left[\left|\sum_{i=1}^{N}B_{ii}\left(\norm[\x_{i}]2^{2}-\E\norm[\x_{i}]2^{2}\right)\right|\geq t\right]$.
To prove that, we cite another useful lemma (Theorem 2.8.2 on pp.$\ $36
in \citet{vershynin2017high}).
\begin{lem}[Bernstein's inequality for sub-exponential random variables]
\emph{}\label{lem:bernstein-subexp} Let $X_{1},\ldots,X_{N}$ be
independent, mean zero, sub-exponential random variables, and $\a\in\real^{N}$.
Then for every $t\geq0$, we have 
\[
\P\left[\left|\sum_{i=1}^{N}a_{i}X_{i}\right|\geq t\right]\leq2\exp\left[-c\min\left(\frac{t^{2}}{K_{1}^{2}\norm[\a]2^{2}},\frac{t}{K_{1}\norm[\a]{\infty}}\right)\right]
\]
where $K_{1}\coloneqq\max_{i}\norm[X_{i}]{\psi_{1}}$.
\end{lem}
Here, $\norm[\cdot]{\psi_{1}}$ denotes the sub-exponential norm;
see \citet{vershynin2017high} for more details. We work under the
premise of Lemma \ref{lem:hanson_wright_hdp}. Since $\x_{i}$ are
independent sub-Gaussian random vectors, each $\norm[\x_{i}]2^{2}-\E\norm[\x_{i}]2^{2}$
is the sum of $M$ independent, mean zero, sub-exponential random
variables with sub-exponential norm equal to $K^{2}$. Then Lemma
\ref{lem:bernstein-subexp} implies 
\[
\P\left[\left|\sum_{i=1}^{N}B_{ii}\left(\norm[\x_{i}]2^{2}-\E\norm[\x_{i}]2^{2}\right)\right|\geq t\right]\leq2\exp\left[-c\min\left(\frac{t^{2}}{K^{4}M\norm[\B]F^{2}},\frac{t}{K^{2}\opnorm{\B}}\right)\right]
\]
as required. 

\subsection{Proof of Fact \ref{fact:satisfy_cond_gauss_choas_operator_norm_bound}
\label{sec:proof_satisfy_cond}}

We prove the following equivalent statement 
\[
\norm[\left\langle \x,\w\right\rangle ]{\psi_{2}}^{2}\leq C\frac{\sgnorm^{2}}{\std^{2}}\E\left\langle \x,\w\right\rangle ^{2}\qquad\text{for any }\w\in\real^{\num},
\]
where $C>0$ is a universal constant and $C\frac{\sgnorm^{2}}{\std}\ge1.$
We first establish a relationship between $\sgnorm^{2}$ and $\Var\left(x_{1}\right)$:
Proposition 2.5.2 on pp. 24 of \citet{vershynin2017high} implies
that $\frac{C'\sgnorm^{2}}{\std^{2}}\ge\frac{1}{2}$ for some universal
constant $C'>0$. Hence, we have 
\begin{align*}
\norm[\left\langle \x,\w\right\rangle ]{\psi_{2}}^{2} & \overset{(i)}{\le}2C'\sum_{i\in\left[\num\right]}w_{i}^{2}\norm[x_{i}]{\psi_{2}}^{2}\\
 & =2C'\frac{\sgnorm^{2}}{\std^{2}}\sum_{i\in\left[\num\right]}w_{i}^{2}\std^{2}\\
 & \overset{(ii)}{=}2C'\frac{\sgnorm^{2}}{\std^{2}}\E\left\langle \x,\w\right\rangle ^{2},
\end{align*}
where $(i)$ holds according to Proposition 2.6.1 on pp. 28 of \citet{vershynin2017high},
and $(ii)$ holds since $x_{i}$ are i.i.d.$\ $and $\E x_{i}=0$.
Letting $C=2C'$ completes the proof.

\subsection{Proof of Lemma \ref{lm:order_stats} \label{sec:proof_lm_order_stat}}

We define 
\begin{align*}
L_{\M} & \coloneqq\sum_{j,a}\left(X_{ja}-\lambda_{ja}\right)M_{ja},\\
R_{\beta,\M} & \coloneqq D\sqrt{\left\lceil \beta\right\rceil \left(\sum_{j,a}\rho_{ja}^{2}M_{ja}\right)\log\left(3mg/\beta\right)},\\
\calM_{\beta} & \coloneqq\left\{ \M\in\left\{ 0,1\right\} ^{m\times g}:\M\one_{g}\le\one_{m},\norm[\M]1=\left\lceil \beta\right\rceil \right\} .
\end{align*}
To establish a uniform bound in $\beta$, we apply a discretization
argument to the possible values of $\beta$. Define the shorthand
$E\coloneqq(0,m]$. We can cover $E$ by the sub-intervals $E_{t}\coloneqq(t-1,t]$
for $t\in[m]$. For each $t\in[m]$ we define the probability

\begin{align*}
\alpha_{t} & \coloneqq\P\left\{ \exists\beta\in E_{t},\exists\M\in\calM_{\beta}:L_{\M}>R_{\beta,\M}\right\} .
\end{align*}
We bound each of these probabilities: 

\begin{align}
\alpha_{t} & \overset{(i)}{\le}\P\left\{ \exists\M\in\calM_{t}:L_{\M}>R_{t,\M}\right\} \nonumber \\
 & \leq\P\left\{ \bigcup_{\M\in\calM_{t}}\left\{ L_{\M}>R_{t,\M}\right\} \right\} \nonumber \\
 & \leq\sum_{\M\in\calM_{t}}\P\left\{ L_{\M}>R_{t,\M}\right\} ,\label{eq:double union bd on normal-1-1-2}
\end{align}
where step $(i)$ holds since $\beta\in E_{t}$ implies $\beta\le\left\lceil \beta\right\rceil =t$. 

Note that each $X_{ja}-\lambda_{ja}$ is an independent zero-mean
sub-Gaussian random variable and the squared sub-Gaussian norm of
$L_{\M}$ is at most $C_{\psi_{2}}\sum_{j,a}\rho_{ja}^{2}M_{ja}$
where $C_{\psi_{2}}>0$ is a universal constant. We apply Hoeffding
inequality (Lemma \ref{lem:hoeffding}) to bound the probability on
the RHS of (\ref{eq:double union bd on normal-1-1-2}): 
\begin{align*}
\P\left\{ L_{\M}>R_{t,\M}\right\}  & \leq\exp\left\{ -\frac{cD^{2}t\left(\sum_{j,a}\rho_{ja}^{2}M_{ja}\right)\log(3mg/t)}{C_{\psi_{2}}\sum_{j,a}\rho_{ja}^{2}M_{ja}}\right\} \\
 & \leq\exp\left\{ -4t\log(3mg/t)\right\} 
\end{align*}
where $c>0$ is a universal constant. Plugging this back to (\ref{eq:double union bd on normal-1-1-2}),
we have for each $t\in\left[m\right]$, 
\begin{align}
\alpha_{t} & \leq\sum_{\M\in\calM_{t}}\exp\left\{ -4t\log(3mg/t)\right\} \nonumber \\
 & =\binom{m}{t}g^{t}\exp\left\{ -4t\log(3mg/t)\right\} \nonumber \\
 & \leq\left(\frac{me}{t}\right)^{t}g^{t}\exp\left\{ -4t\log(3mg/t)\right\} \nonumber \\
 & \leq\exp\left\{ t\log(3mg/t)+t-4t\log(3mg/t)\right\} \nonumber \\
 & \leq\exp\left\{ -t\log(3mg/t)\right\} =\left(\frac{t}{3mg}\right)^{t},\label{eq:binom coeff bound-1-2}
\end{align}
where the last inequality follows from $t\leq t\log(3mg/t)$ for $t\in\left[m\right]$.
It follows that 
\begin{align*}
 & \quad\P\left\{ \exists\beta\in E,\exists\M\in\calM_{\beta}:L_{\M}>R_{\beta,\M}\right\} \\
 & \leq\P\left\{ \bigcup_{t=1}^{m}\left\{ \exists\beta\in E_{t},\exists\M\in\calM_{\beta}:L_{\M}>R_{\beta,\M}\right\} \right\} \\
 & \le\sum_{t=1}^{m}\alpha_{t}\\
 & \leq\sum_{t=1}^{m}\left(\frac{t}{3mg}\right)^{t}\eqqcolon P_{1}(m).
\end{align*}

It remains to show that $P_{1}(m)\leq\frac{1.5}{m}$. Since 
\begin{align*}
P_{1}(m) & \leq\sum_{t=1}^{m}\left(\frac{t}{3m}\right)^{t}\\
 & \le\frac{1}{3m}+\sum_{t=2}^{m}\left(\frac{t}{3m}\right)^{t}\\
 & \le\frac{1}{3m}+m\cdot\max_{t=2,3,\ldots,m}\left(\frac{t}{3m}\right)^{t},
\end{align*}
the proof is completed if for each integer $t=2,3,\ldots,m$, we can
show the bound $\left(\frac{t}{3m}\right)^{t}\leq\frac{1}{m^{2}}$,
or equivalently $f(t)\coloneqq t(\log3m-\log t)\geq2\log m.$ Since
$t\le m$, $f(t)$ has derivative 
\[
f'(t)=\log3m-\log t-1\ge\log3m-\log\left(\frac{3m}{3}\right)-1=\log3-1\ge0.
\]
Therefore, $f(t)$ is non-decreasing for $2\le t\le m$ and therefore
$f(t)\ge f(2)=2\log3m-2\log2\ge2\log m.$ Hence, $P_{1}(m)\le\frac{1.5}{m}$. 

\section{Proof of Theorem \ref{thm:cluster_error_rate}\label{sec:proof_cluster_error_rate}}

We only need to prove the first part of the theorem. The second part
follows immediately from the first part and Theorem \ref{cor:SDP_exp_rate}.

The proof follows similar lines as those of Theorem 17 and Lemma 18
in \citet{makarychev2016learning}. In the rest of the section, we
work under the context of Algorithms \ref{alg:apx_clustering} and
\ref{alg:est_clustering}. Recall that $\numclust'=\left|\left\{ B_{t}\right\} _{t\ge1}\right|$
and we let $\epsilon\coloneqq\norm[\Yhat-\Ystar]1/\norm[\Ystar]1$.
We have the following lemma.
\begin{lem}
\label{lem:apx_clustering} There exists a partial matching $\perm'$
between $\left[\numclust\right]$ and $\left[\numclust'\right]$ and
a universal constant $C>0$ such that 
\[
\left|\bigcup_{t=\perm'(a)}\clustset a\cap B_{t}\right|\ge\left(1-C\epsilon\right)\num.
\]
\end{lem}
The proof is given in Section \ref{sec:proof_apx_clustering}. The
next lemma concerns the quality of clustering by Algorithm \ref{alg:clustering}.
\begin{lem}
\label{lem:clustering} There exists a permutation $\perm$ on $\left[\numclust\right]$
and a universal constant $C>0$ such that 
\[
\left|\bigcup_{t=\perm(a)}\clustset a\cap U_{t}\right|\ge\left(1-C\epsilon\right)\num.
\]
\end{lem}
The proof is given in Section \ref{sec:proof_clustering}. The result
follows from combining the above lemmas and the fact that 
\[
\misrate(\LabelHat,\LabelStar)=1-\frac{1}{\num}\max_{\perm\in S_{\numclust}}\left|\bigcup_{t=\perm(a)}\clustset a\cap U_{t}\right|.
\]


\subsection{Proof of Lemma \ref{lem:apx_clustering}\label{sec:proof_apx_clustering}}

We define $\y_{a}$ to be an arbitrary row of $\Ystar$ whose index
is in $\clustset a$.
\begin{align*}
G_{a} & \coloneqq\left\{ i\in\clustset a:\norm[\Yhat_{i\bullet}-\y_{a}]1\leq\frac{\size}{8}\right\} ,\qquad\forall a\in\left[\numclust\right]\\
G & \coloneqq\bigcup_{a\in\left[\numclust\right]}G_{a},\\
H & \coloneqq\vertexset\backslash G.
\end{align*}

We construct a partial matching $\perm'$ between sets $\clustset a$
and $B_{t}$ by matching every cluster $\clustset a$ with the first
$B_{t}$ that intersects $G_{a}$, and we let $\perm'(a)=t$. Since
each $i\in\left[\num\right]$ belongs to some $B_{t}$, we are able
to match every $\clustset a$ with some $B_{t}$. The fact that we
cannot match two distinct clusters $\clustset a$ and $\clustset b$
with the same $B_{t}$ as well as the rest of the proof are given
by the following fact.
\begin{fact}
\label{fact:apx_clustering} We have 
\begin{enumerate}
\item For each $a\in\left[\numclust\right]$ and $t\in\left[\numclust'\right]$
such that $t=\perm'(a)$, we have $B_{t}\cap G_{b}=\emptyset$ for
any $b\in\left[\numclust\right]\backslash\left\{ a\right\} $ and
$B_{t}\subset G_{a}\cup H$;
\item For each $a\in\left[\numclust\right]$ and $t\in\left[\numclust'\right]$
such that $t=\perm'(a)$, we have 
\[
\left|B_{t}\cap\clustset a\right|\geq\left|G_{a}\right|-\left|B_{t}\cap H\right|.
\]
\item We have 
\[
\sum_{t=\perm'(a)}\left|B_{t}\cap\clustset a\right|\ge\left|\vertexset\right|-2\left|H\right|.
\]
\item There exists a universal constant $C>0$ such that $\left|H\right|\leq C\epsilon\num$.
\end{enumerate}
\end{fact}
The proof is given below.

\subsubsection{Proof of Fact \ref{fact:apx_clustering}\label{sec:proof_fact_apx_clustering}}
\begin{enumerate}
\item Suppose that there exist $B_{t}$ and $b\in\left[\numclust\right]$
such that $b\ne a$ and $B_{t}\cap G_{b}\ne\emptyset$. Let $u\in B_{t}\cap G_{a}$
and $v\in B_{t}\cap G_{b}$. Since $G_{a}$ and $G_{b}$ are disjoint,
we know that $u\ne v$. Let $w\in B_{t}$. Then we have 
\begin{align*}
\norm[\Yhat_{u\bullet}-\Yhat_{w\bullet}]1 & \leq\frac{\size}{4}\\
\norm[\Yhat_{v\bullet}-\Yhat_{w\bullet}]1 & \leq\frac{\size}{4}.
\end{align*}
Therefore 
\[
\norm[\Yhat_{u\bullet}-\Yhat_{v\bullet}]1\leq\norm[\Yhat_{u\bullet}-\Yhat_{w\bullet}]1+\norm[\Yhat_{v\bullet}-\Yhat_{w\bullet}]1\le\frac{\size}{2}.
\]
This implies 
\begin{align*}
\norm[\y_{a}-\y_{b}]1 & \leq\norm[\y_{a}-\Yhat_{u\bullet}]1+\norm[\Yhat_{u\bullet}-\Yhat_{v\bullet}]1+\norm[\y_{b}-\Yhat_{v\bullet}]1\\
 & \leq\frac{\size}{8}+\frac{\size}{2}+\frac{\size}{8}<\size,
\end{align*}
which is a contradiction to the fact that $\norm[\y_{a}-\y_{b}]1=2\size$.
To complete the proof, we note that for any $i\in B_{t}$ we have
either $i\in G_{a}$ or $i\in H$.
\item Fix $i\in G_{a}$ for some $a\in\left[\numclust\right]$. For any
$j\in G_{a}$ we have $j\in B(i)$ since 
\[
\norm[\Yhat_{i\bullet}-\Yhat_{j\bullet}]1\le\norm[\y_{a}-\Yhat_{i\bullet}]1+\norm[\y_{a}-\Yhat_{j\bullet}]1\le\frac{\size}{4}.
\]
Therefore, by definition 
\[
\left|B_{t}\right|\ge\left|B(i)\right|\ge\left|G_{a}\right|.
\]
We have 
\begin{align*}
\left|B_{t}\cap\clustset a\right| & \overset{(i)}{\ge}\left|B_{t}\cap G_{a}\right|\\
 & =\left|B_{t}\right|-\left|B_{t}\backslash G_{a}\right|\\
 & \overset{(ii)}{=}\left|B_{t}\right|-\left|B_{t}\cap H\right|\\
 & \ge\left|G_{a}\right|-\left|B_{t}\cap H\right|,
\end{align*}
where step $(i)$ holds since $G_{a}\subset\clustset a$ and step
$(ii)$ holds since $B_{t}\subset G_{a}\cup H$.
\item Summing the LHS of the above equation over $t=\perm'(a)$ gives 
\begin{align*}
\sum_{t=\perm'(a)}\left|B_{t}\cap\clustset a\right| & =\sum_{a\in\left[\numclust\right]}\left|G_{a}\right|-\sum_{t=\perm'(a)}\left|B_{t}\cap H\right|\\
 & \ge\sum_{a\in\left[\numclust\right]}\left|G_{a}\right|-\sum_{t\ge1}\left|B_{t}\cap H\right|\\
 & \overset{(i)}{=}\left|G\right|-\left|\vertexset\cap H\right|\\
 & =\left|\vertexset\right|-2\left|H\right|,
\end{align*}
where step $(i)$ holds since $B_{t}\cap H$ are disjoint and $\bigcup_{t\geq1}B_{t}=\vertexset$.
\item We have 
\[
\left|H\right|\cdot\frac{\size}{8}\leq\sum_{i\in H}\norm[\Yhat_{i\bullet}-\y_{\labelstar(i)}]1\leq\norm[\Yhat-\Ystar]1\leq\epsilon\norm[\Ystar]1=\epsilon\cdot\num\size
\]
where the last step follows from the fact that $\norm[\Ystar]1=\num\size$.
The result follows.
\end{enumerate}

\subsection{Proof of Lemma \ref{lem:clustering}\label{sec:proof_clustering}}

Let $\perm'$ be the partial matching between $\clustset a$ and $B_{t}$
from Lemma \ref{lem:apx_clustering}. Define $\perm(a)=\perm'(a)$
for $\perm'(a)\le\numclust$. If the resulting $\perm$ is a partial
permutation, we extend $\perm$ to a permutation defined on $\left[\numclust\right]$
in an arbitrary way. We may assume that $\left\{ U_{t}\right\} _{t\in\left[\numclust\right]}$
are $\left\{ B_{t}\right\} _{t\in\left[\numclust\right]}$ WLOG, and
that $U_{t}$ consists of $B_{t}$ and some elements from sets $B_{u}$
with $u>\numclust$. We have 
\begin{align*}
\left|\bigcup_{t=\perm(a)}\clustset a\cap U_{t}\right| & \ge\left|\bigcup_{t=\perm'(a)\le\numclust}\clustset a\cap B_{t}\right|\\
 & =\left|\bigcup_{t=\perm'(a)}\clustset a\cap B_{t}\right|-\left|\bigcup_{t=\perm'(a)>\numclust}\clustset a\cap B_{t}\right|\\
 & \ge\left(1-C'\epsilon\right)\num-\left|\bigcup_{t=\perm'(a)>\numclust}\clustset a\cap B_{t}\right|
\end{align*}
where $C'>0$ is a universal constant. Define 
\begin{align*}
T_{1} & \coloneqq\left\{ t>\numclust:t=\perm'(a)\text{ for some }a\in\left[\numclust\right]\right\} ,\\
T_{2} & \coloneqq\left\{ t\in\left[\numclust\right]:t\ne\perm'(a)\text{ for any }a\in\left[\numclust\right]\right\} .
\end{align*}
Note that $\left|T_{1}\right|=\left|T_{2}\right|$ and for any $t_{1}\in T_{1}$
and $t_{2}\in T_{2}$ we have $\left|B_{t_{1}}\right|\le\left|B_{t_{2}}\right|$.
Therefore, 
\begin{align*}
\left|\bigcup_{t=\perm'(a)>\numclust}\clustset a\cap B_{t}\right| & \le\left|\bigcup_{t\in T_{1}}B_{t}\right|\\
 & \le\left|\bigcup_{t\in T_{2}}B_{t}\right|\\
 & \le\left|\vertexset\right|-\left|\bigcup_{t=\perm'(a)}\clustset a\cap B_{t}\right|\\
 & =C'\epsilon\num.
\end{align*}
The result follows by setting $C\coloneqq2C'$.

\section{Proof of Theorem \ref{thm:mean_estimation_error}\label{sec:proof_mean_estimation_error}}

Let $\Var\left(g_{ij}\right)=\std^{2}$. For $a\in\left[\numclust\right]$,
define $\clustest a\coloneqq\left\{ i\in\left[\num\right]:\labelhat_{i}=a\right\} $
the estimated clusters encoded in $\LabelHat$, and recall that our
cluster center estimators are defined by $\Meanhat_{a}\coloneqq\size^{-1}\sum_{i\in\clustest a}\h_{i}$.
We assume $\left\{ \clustest a\right\} $ achieves the lowest clustering
error as given in Theorem \ref{thm:cluster_error_rate} WLOG. For
each $a\in\left[\numclust\right]$, we have 
\begin{align*}
\norm[\Meanhat_{a}-\Mean_{a}]2 & \le\norm[\frac{1}{\size}\sum_{i\in\clustest a}\h_{i}-\frac{1}{\size}\sum_{j\in\clustset a}\h_{j}]2+\norm[\frac{1}{\size}\sum_{j\in\clustset a}\h_{j}-\Mean_{a}]2\\
 & \eqqcolon Q_{1}+Q_{2}.
\end{align*}


\subsection{Controlling $Q_{1}$}

Define $\epsilon\coloneqq\misrate(\LabelHat,\LabelStar)$. We work
on the event that the result Theorem \ref{thm:cluster_error_rate}
is true. We have 
\[
Q_{1}=\frac{1}{\size}\norm[\sum_{i\in\clustest a\backslash\clustset a}\h_{i}-\sum_{j\in\clustset a\backslash\clustest a}\h_{j}]2
\]
Note that $\left|\clustest a\backslash\clustset a\right|=\left|\clustset a\backslash\clustest a\right|$
so we can pair each point in $\clustest a\backslash\clustset a$ with
a point in $\clustset a\backslash\clustest a$. Let us pair $i$th
point in $\clustest a\backslash\clustset a$ with $j(i)$th point
in $\clustset a\backslash\clustest a$, and define $\calM\coloneqq\left\{ \left(i,j(i)\right)\right\} $.
We have $\left|\calM\right|\le\num\epsilon$ and we can write 
\begin{align*}
Q_{1} & =\frac{1}{\size}\norm[\sum_{(i,j(i))\in\calM}\left(\h_{i}-\h_{j(i)}\right)]2\\
 & \le\frac{1}{\size}\sum_{(i,j(i))\in\calM}\norm[\h_{i}-\h_{j(i)}]2\\
 & \le\frac{1}{\size}\sum_{(i,j(i))\in\calM}\left(\minsep_{\labelstar(i),\labelstar(j(i))}+\norm[\g_{i}-\g_{j(i)}]2\right)\\
 & \le\frac{1}{\size}\sum_{(i,j(i))\in\calM}\left(C_{q}\minsep+\norm[\g_{i}-\g_{j(i)}]2\right),
\end{align*}
where the last step holds for some universal constant $C_{q}>0$ given
that $\max_{a,b\in\left[\numclust\right]}\minsep_{ab}\le C_{q}\minsep$.
By Theorem 3.1.1 on pp.$\ $41 of \citet{vershynin2017high}, $\frac{1}{\sqrt{2}\std}\norm[\g_{i}-\g_{j(i)}]2-\sqrt{\vecdim}$
is a sub-Gaussian random variable with sub-Gaussian norm at most $C_{\psi_{2}}\frac{\sgnorm^{2}}{\std^{2}}$
where $C_{\psi_{2}}>0$ is a universal constant. Then Lemma \ref{lem:hoeffding}
implies that 
\[
\P\left[\frac{1}{\sqrt{2}\std}\norm[\g_{i}-\g_{j(i)}]2-\sqrt{\vecdim}\ge C\frac{\sgnorm^{2}}{\std^{2}}\sqrt{\log\num}\right]\le\num^{-C'}
\]
for some universal constants $C,C'>2$. By the union bound and the
facts that $\left|\calM\right|\le\num$ and $\std\lesssim\sgnorm$,
we have 
\[
\max_{(i,j)\in\calM}\norm[\g_{i}-\g_{j(i)}]2\le C_{g}\left(\sgnorm\sqrt{2\vecdim}+C\sgnorm\sqrt{2\log\num}\right)
\]
with probability at least $1-n^{-C_{1}}$ where $C_{g},C_{1}>0$ are
universal constants. 

Therefore, we have 
\begin{align*}
Q_{1} & \le C_{0}\left(\minsep+\sgnorm\sqrt{\vecdim}+\sgnorm\sqrt{\log\num}\right)\cdot\numclust\exp\left[-\frac{\snr^{2}}{\conste}\right]\\
 & \le C_{0}\left(\minsep+\sgnorm\sqrt{\vecdim}+\sgnorm\sqrt{\log\num}\right)\cdot\exp\left[-\frac{\snr^{2}}{2\conste}\right]
\end{align*}
for some universal constant $C_{0},\conste>0$ with probability at
least $1-n^{-C_{1}}$, where the last step holds since $\snr^{2}\ge\numclust$.
The fact that $e^{x}\ge1+x>x$ for any $x$ implies 
\[
\exp\left[-\frac{\snr^{2}}{4\conste}\right]\le\frac{4\conste}{\snr^{2}}=\frac{\sgnorm}{\minsep}\cdot\frac{4\conste}{\snr}\le4\conste\frac{\sgnorm}{\minsep}
\]
where the last step holds since we have $\snr\ge1$ by the conditions
of Theorem \ref{thm:cluster_error_rate}. Hence, we have 
\begin{align*}
Q_{1} & \le C_{0}\sgnorm\left(4\conste+\sqrt{\vecdim}+\sqrt{\log\num}\right)\cdot\exp\left[-\frac{\snr^{2}}{4\conste}\right]\\
 & \leq C_{1}\sgnorm\left(1+\sqrt{\vecdim}+\sqrt{\log\num}\right)\cdot\exp\left[-\frac{\snr^{2}}{4\conste}\right]\\
 & \leq2C_{1}\sgnorm\left(\sqrt{\vecdim}+\sqrt{\log\num}\right)\cdot\exp\left[-\frac{\snr^{2}}{4\conste}\right]
\end{align*}
where $C_{1}>0$ is a universal constant.

\subsection{Controlling $Q_{2}$}

We have 
\[
Q_{2}=\norm[\frac{1}{\size}\sum_{j\in\clustset a}\g_{j}]2.
\]
We see that $\frac{1}{\size}\sum_{j\in\clustset a}g_{ji}$ has variance
$\frac{1}{\size}\std^{2}$. By Proposition 2.6.1 on pp.$\ $28 and
Theorem 3.1.1 on pp. 41 of \citet{vershynin2017high}, $\frac{\sqrt{\size}}{\std}\norm[\frac{1}{\size}\sum_{j\in\clustset a}\g_{j}]2-\sqrt{\vecdim}$
is a sub-Gaussian random variable with sub-Gaussian norm at most $C_{\psi_{2}}\frac{\sgnorm^{2}}{\std^{2}}$
where $C_{\psi_{2}}>0$ is a universal constant. Then Lemma \ref{lem:hoeffding}
implies that 
\[
\P\left[\frac{\sqrt{\size}}{\std}\norm[\frac{1}{\size}\sum_{j\in\clustset a}\g_{j}]2-\sqrt{\vecdim}\ge C\frac{\sgnorm^{2}}{\std^{2}}\sqrt{\log\num}\right]\le\num^{-C'}
\]
for some universal constants $C,C'>0$. Since $\std\lesssim\sgnorm$,
there exists a universal constant $C_{0}>0$ such that 
\[
Q_{2}\leq C_{0}\sgnorm\left(\sqrt{\frac{\numclust\vecdim}{\num}}+\sqrt{\frac{\numclust\log\num}{\num}}\right)
\]
with probability at least $1-\num^{-C'}$. 

\section{Technical lemmas}

The following lemma is Theorem 2.6.2 on pp.$\ $28 in \citet{vershynin2017high}.
\begin{lem}[General Hoeffding's inequality]
\emph{ \label{lem:hoeffding} }Let $X_{1},\ldots,X_{N}$ be independent,
mean zero, sub-Gaussian random variables. Then, for every $t\geq0$
we have 
\[
\P\left[\left|\sum_{i=1}^{N}X_{i}\right|\geq t\right]\leq2\exp\left[-\frac{ct^{2}}{\sum_{i=1}^{N}\norm[X_{i}]{\psi_{2}}^{2}}\right],
\]
where $c>0$ is a universal constant.
\end{lem}
The following lemma is Exercise 4.7.3 in \citet{vershynin2017high}.
\begin{lem}[Tail bound of covariance matrix of sub-Gaussians]
\emph{ }\label{lem:subg_cov_mat_bound} Let $\x$ be a sub-Gaussian
vector and let $\x_{1},\ldots,\x_{m}$ be independent samples of $\x$.
Let $m$ be a positive integer and define 
\begin{align*}
\boldsymbol{\Sigma} & \coloneqq\E\x\x\t,\\
\boldsymbol{\Sigma}_{m} & \coloneqq\frac{1}{m}\sum_{i=1}^{m}\x_{i}\x_{i}\t.
\end{align*}
Let $\rho_{0}\ge1$ be such that 
\[
\norm[\left\langle \x,\w\right\rangle ]{\psi_{2}}\leq\rho_{0}\sqrt{\E\left\langle \x,\w\right\rangle ^{2}}\qquad\text{for any }\w\in\real^{N}.
\]
For any $u\geq0$, we have for a universal constant $C>0$, 
\[
\opnorm{\boldsymbol{\Sigma}_{m}-\boldsymbol{\Sigma}}\leq C\rho_{0}^{2}\left(\sqrt{\frac{N+u}{m}}+\frac{N+u}{m}\right)\opnorm{\boldsymbol{\Sigma}}
\]
with probability at least $1-2e^{-u}$.
\end{lem}



\end{document}
