% !TEX root = stoc-paper.tex
\newcommand{\realm}{N}
\section{An SDP Relaxation for Norm Amplification on the Positive Orthant}
\label{c-sec:sdp}

For a matrix $M\in \mathbb{C}^{A\times X}$, 
$\displaystyle\tau_M(\delta)=\sup_{\substack{\mathbb{P}\in \Delta_X\\ \|\mathbb{P}\|_2\le 1/|X|^{1-\delta/2}}} \log_{|A|} (\|M\cdot \mathbb{P}\|_2)$.\\[+1ex]
That is, $\tau_M(\delta)=\frac12 \log_{|A|} OPT_{M,\delta}$ where $OPT_{M,\delta}$ is the
optimum of the following quadratic program:
\begin{equation}
\begin{aligned}\label{c-original}
\text{Maximize}  &\quad \|M\cdot \mathbb{P}\|_2^2=\langle M\cdot \mathbb{P},M\cdot \mathbb{P}\rangle,& \\
\text{subject to:}&\quad \sum_{i\in X} \mathbb{P}_i=1,&\\
&\quad \sum_{i\in X} \mathbb{P}_i^2\le |X|^{\delta-1},&\\
&\quad \mathbb{P}_i\geq 0 &\quad \text{for all }i\in X.
\end{aligned}
\end{equation}
Instead of attempting to solve \eqref{c-original}, presumably a difficult
quadratic program, we consider the following semidefinite program (SDP):
\begin{equation} \label{c-SDP-relaxation}
\begin{aligned}
\text{Maximize}&\quad \langle M^{*} M,U\rangle\cdot |X|^2/|A| &\\
\text{subject to:} &&\\
[V]&\quad U\succeq 0,&\\
[w]&\quad \sum_{i,j\in X} U_{ij}=1,  &\\
[z]&\quad \sum_{i\in X} U_{ii}\le |X|^{\delta-1},&\\
&\quad U_{ij}\in \mathbb{R},U_{ij}\geq 0 &\quad\text{for all } i,j\in X.
\end{aligned}
\end{equation}
Recall that $M^{*}$ is the conjugate transpose of $M$.
Note that for any $\mathbb{P}\in \Delta_X$ achieving the optimum value
of \eqref{c-original} the positive semidefinite matrix
$U=\mathbb{P}\cdot \mathbb{P}^T$ 
has the same value in \eqref{c-SDP-relaxation} (where the $|X|^2/|A|$ factor
accounts for the difference in scaling factors based on the dimensions for the
two expectation inner products), and hence \eqref{c-SDP-relaxation}
is an SDP relaxation of \eqref{c-original}.

But this is not a standard SDP,
since $M$ is over $\mathbb{C}$ and $M^{*}M$ might contain complex entries.
In order to apply techniques on real matrices,
we define $\realm:X\times X\rightarrow \mathbb{R}$ as $\realm(x,x')=Re(M^{*}M(x,x'))$,
that is,
$\realm$ is the real part of $M^{*}M$.
Then we obtain a (real) definite program that is identical to \eqref{c-SDP-relaxation}, except that $M^* M$ is replaced by $\realm$ and the condition $U_{ij}\in \mathbb{R}$ is superfluous.
%\begin{equation} \label{c-SDP-relaxation-real}
%\begin{aligned}
%\text{Maximize}&\quad \langle \realm,U\rangle\cdot |X|^2/|A| &\\
%\text{subject to:} &&\\
%[V]&\quad U\succeq 0,&\\
%[w]&\quad \sum_{i,j\in X} U_{ij}=1,  &\\
%[z]&\quad \sum_{i\in X} U_{ii}\le |X|^{\delta-1},&\\
%&\quad U_{ij}\geq 0 &\quad\text{for all } i,j\in X.
%\end{aligned}
%\end{equation} 
%
The key observation is that \eqref{c-SDP-relaxation} and this real program (not shown) have the same optimal value.
This is because for any $U\in \mathbb{R}^{X\times X}$,\\
\centerline{$\displaystyle |X|^2\langle M^{*} M,U\rangle=\sum_{i,j}(M^{*}M)_{ij}\cdot U_{ij}=\sum_{ij}Re((M^{*}M)_{ij})\cdot U_{ij}+i\cdot\sum_{x,x'}Im((M^{*}M)_{ij})\cdot U_{ij}$}
Since $M^{*}M$ is a Hermitian matrix,
we have $(M^{*}M)_{ij}=\overline{(M^{*}M)_{ji}}$.
But $U$ is real symmetric,
so we have $\sum_{i,j}Im((M^{*}M)_{ij})\cdot U_{ij}=0$,
namely
\[
|X|^2\langle M^{*} M,U\rangle=\sum_{i,j}Re((M^{*}M)_{ij})\cdot U_{ij}=|X|^2\langle \realm,U\rangle
\]
and we only need to consider the real parts.

In order to upper bound the value of \eqref{c-SDP-relaxation},
we consider the dual program to this real SDP which can be written as:
%\begin{equation} \label{c-dual-SDP}
%\begin{aligned}
%\text{Minimize}  &\quad w+z\cdot |X|^{\delta-1} &\\
%\text{subject to:}&&\\
%[U]&\quad V\succeq 0,&\\
%[U_{ii}] &\quad w+z\geq V_{ii}+\realm_{ii}/|A|,&\quad\text{for all }i\in X\\
%[U_{ij}] &\quad w\geq V_{ij}+\realm_{ij}/|A|,&\quad\text{for all }i\ne j \in X\\
%&\quad z\ge 0
%\end{aligned}
%\end{equation}
%or equivalently,
\begin{equation} \label{c-dual-SDP1}
\begin{aligned}
\text{Minimize}  &\quad w+z\cdot |X|^{\delta}\cdot |X|^{-1} &\\
\text{subject to:}&\quad V\succeq 0,&\\
&\quad z I + w J\geq V+ \realm/|A|,\\
&\quad z\ge 0.
\end{aligned}
\end{equation}
where $I$ is the identity matrix and $J$ is the all 1's matrix over $X\times X$.

Any dual solution of \eqref{c-dual-SDP1} yields an upper bound on the optimum
of \eqref{c-SDP-relaxation} and hence $OPT_{M,\delta}$ and $\tau_M(\delta)$.
%In order to balance the two terms of the objective function,
%for $\delta<1$ we can afford a large $z$ to obtain a smaller $w$.
To simplify the complexity of analysis we restrict ourselves to considering
semidefinite matrices $V$ that are suitably chosen Laplacian matrices.
For any set $S$ in $X\times X$ and any $\alpha:S\rightarrow \mathbb{R}_+$
the Laplacian matrix associated with $S$ and $\alpha$
is defined by $L_{(S,\alpha)}:=\sum_{(i,j)\in S}\alpha(i,j) L_{ij}$ where
$L_{ij}=(e_i-e_j)(e_i-e_j)^T$ for the standard basis $\{e_i\}_{i\in X}$ .
Intuitively, in the dual SDP \eqref{c-dual-SDP1}, by adding matrix $V=L_{S,\alpha}$ for
suitable $S$ and $\alpha$ depending on $M$ we can shift weight from the
off-diagonal
entries of $N$ to the diagonal where they can be covered by the
$z+w$ entries on the diagonal rather than being covered by the $w$ values in
the off-diagonal entries.  This will be advantageous for us since the
objective function has much smaller coefficient for $z$ which helps cover
the diagonal entries than coefficient for $w$, which is all that covers the
off-diagonal entries.

\begin{defn}\label{d-W}
Suppose that $N\in \mathbb{R}^{X\times X}$ is a symmetric matrix.
For $\kappa\in \mathbb{R}_+$, define\\
\centerline{$\displaystyle W_\kappa(N)=\max_{i\in X} \sum_{j\in X:\ N_{i,j}> \kappa} (N_{i,j}-\kappa)$.}
\end{defn}


The following lemma is the basis for our bounds on $\tau_M(\delta)$.

\begin{lemma}
\label{c-SDP-lemma}
Let $\kappa\in \mathbb{R}_+$.  Then
%\begin{equation*}
$OPT_{M,\delta}\le (\kappa+W_\kappa(\realm)\cdot |X|^{\delta-1})/|A|$.
%\end{equation*}
\end{lemma}

\begin{sloppypar}
\begin{proof}
For each off-diagonal entry of $\realm$ with $\realm(i,j)>\kappa$,
include matrix $L_{ij}$ with coefficient $(\realm(i,j)-\kappa)/|A|$ in the sum for
the Laplacian $V$.
By construction, the matrix $V+\realm/|A|$ has off-diagonal entries
at most $\kappa/|A|$ and diagonal entries at most
$(\kappa+W_\kappa(\realm))/|A|$.  The solution to \eqref{c-dual-SDP1}
with $w=\kappa/|A|$ and $z=W_\kappa(\realm)/|A|$ is therefore feasible,
which yields the bound as required.
\end{proof}
\end{sloppypar}


It may not be easy to bound $W_\kappa(N)$ directly,
since the real part of $M^{*}M$ may not have good structure. 
Fortunately,
we have the following measure:
\begin{defn}
Let $M\in \mathbb{C}^{A\times X}$ be a complex matrix.
For $\kappa\in \mathbb{R}_+$, define\\
\centerline{$\displaystyle\tilde W_\kappa(M)=\max_{i\in X} \sum_{j\in X:\ |(M^{*}M)_{i,j}|> \kappa} (|(M^{*}M)_{i,j}|-\kappa)$.}
\end{defn}

\begin{proposition}
Let $\kappa\in \mathbb{R}_+$.  Then
$W_\kappa(\realm)\leq \tilde W_\kappa(M)$ 
\end{proposition}
\begin{proof}
Whenever $N_{i,j}> \kappa$,
we have $|(M^{*}M)_{i,j}|\geq N_{i,j}> \kappa$.
Moreover,
this gives $|(M^{*}M)_{i,j}|-\kappa\geq N_{i,j}-\kappa$.
Then the statement follows the two definitions.
\end{proof}


For specific matrices $M$, we obtain the required bounds on
$\tau_M(\delta)<0$ for some $0<\delta<1$ by showing that we can
set $\kappa=|A|^\gamma$ for some $\gamma<1$ and obtain that
$W_\kappa(N)$ or $\tilde W_\kappa(M)$ is at most
$\kappa\cdot |X|^{\gamma'}$ for some $\gamma'<1$.

