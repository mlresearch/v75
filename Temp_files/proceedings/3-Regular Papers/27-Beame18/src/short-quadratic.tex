% !TEX root = stoc-paper.tex
\newcommand{\val}{\textrm{val}}
\newcommand{\monomial}[3]{\mathcal{M}_{#1}(#2,#3)}
\newcommand{\mjs}{M^{(j^*)}}
\newcommand{\biasj}{\mathrm{bias}_j}
\newcommand{\biasjs}{\mathrm{bias}_{j^*}}




\subsection{The Bias of $\mathbb{F}_p$ Polynomials for Odd Prime $p$}
\label{sec:complex-quadratic}

Let $d\ge 1$ be an integer and $p$ be an odd prime.
For any integer $m\ge d$, consider the learning
problem for $\mathbb{F}_p$ polynomials in $m$ variables of degree at most $d$.
Unlike the case over $\mathbb{F}_2$, the monomials are not necessarily
multilinear but can have degree at most $p-1$ in each variable.
Let $\monomial{p}{d}{m}$ be the set of monomials in $m$ variables of
total degree at most $d$ and degree at most $p-1$ in each
variable.
That is $A=\mathbb{F}_p^m$ and, expressing polynomials by their coefficients,
we have $X=\mathbb{F}_p^n$ where $n=|\monomial{p}{d}{m}|$ is the number of
monomials of total degree at most $d$ and degree at most $p-1$ in each
variable.
As in the case of $\mathbb{F}_2$, $n$ is the dimension of a Reed-Muller code
$RM_p(d,m)$ over $\mathbb{F}_p$, 
and for $a\in A$ and $x\in X$,
$\fun (a,x)=x(a)\in \mathbb{F}_p$.
For $d\ge p$ there is no convenient closed form known for $|\monomial{p}{d}{m}|$
but the following is known:

\begin{proposition}
For $d<p$, $|\monomial{p}{d}{m}|=\binom{m+d}{d}$ and
for $2<p\le d\le m$,
$\sum_{i=0}^d \binom{m}{d}\le |\monomial{p}{d}{m}|\le \binom{m+d}{d}$.
\end{proposition}

Since $p>2$, the learning problem for $\mathbb{F}_p$ polynomials is governed
by $p-1$ complex matrices $M^{(1)},\ldots,M^{(p-1)}$ where
$\mj(a,x)=\omega^{j\cdot x(a)}$ and $\omega=e^{2\pi i/p}$.   
We need to bound the norm amplification curves of all these matrices.
We will relate these curves to the values of $\biasj(x)$ for $j\in \coltset{1,\ldots,p-1}$ and $x\in X$, where
$$\biasj(x)=\E_{a\in_R A} \omega^{j\cdot x(a)}.$$

\begin{sloppypar}
Fix an arbitrary $j^*\in \coltset{1,\ldots,p-1}$,
For $N=(\mjs)^*\cdot \mjs$, the $(x,y)$ entry of $N$ is $p^m \langle \mjs_x,\mjs_y\rangle$
where $\langle \cdot,\cdot\rangle$ is the complex inner product.
\end{sloppypar}

\begin{proposition}
\label{c-prop:equalrows}
Let $\mathbf{0}=0^n$.
Then for $x,y\in X$,
$\langle \mjs_x,\mjs_y\rangle=\langle \mjs_{\mathbf{0}},\mjs_{y-x}\rangle$.
\end{proposition}

\begin{proof}
\begin{align*}
\langle \mjs_x,\mjs_y\rangle&=\E_{a\in \mathbb{F}_p^m} \overline {\mjs_x(a)}\mjs_y(a)
=\E_{a\in \mathbb{F}_p^m} \omega^{-j^*\cdot x(a)}\omega^{j^*\cdot y(a)}
=\E_{a\in \mathbb{F}_p^m} \omega^{-j^*\cdot x(a)+j^*\cdot y(a)}\\
&=\E_{a\in \mathbb{F}_p^m} \omega^{j^*(y-x)(a)}
=\E_{a\in \mathbb{F}_p^m} \mjs_{\mathbf{0}}(a)\mjs_{y-x}(a)
=\langle \mjs_{\mathbf{0}},\mjs_{y-x}\rangle
\end{align*}
\end{proof}

Since the mapping $y\mapsto y-x$ for $x\in \mathbb{F}_p^n$ is
1-1 on $\mathbb{F}_p^n$, 
every row of $N_x$ for $x\in X$ contains the same multi-set of values.
Therefore, in order to analyze the function $\tilde W_\kappa(N)$,
we only need to examine the fixed row $N_\mathbf{0}$ of $N$.
where each entry
$$N_{\mathbf{0}x}=\sum_{a\in \mathbb{F}_p^m}\omega^{j^*\cdot x(a)}=p^m\cdot \biasjs(x).$$
Therefore we have shown the following:

\begin{lemma}
\label{lem:matrix-bias}
Let $j^*\in \coltset{1,\ldots,p-1}$.  For every $v\in \mathbb{C}$, the number of
entries in each row of $N=(\mjs)^*\cdot \mjs$ equal to $v$ is precisely
the number of polynomials $x\in X$ such that $p^m\cdot \biasjs(x)=v$. 
\end{lemma}

Therefore, to bound $\tilde W_\kappa(N)$ it suffices to bound the numbers of
polynomials $x\in X$ such that $|\biasjs(x)|$ is large.

\paragraph{Affine Functions over $\mathbb{F}_p$}

For $d=1$, an $x\in X=\mathbb{F}_p^{m+1}$ yields the function
$x(a)=x_0+\sum_{i=1}^m x_i a_i$.  Unless $x_1=\cdots=x_m=0$, for every 
$k\in \mathbb{F}_p$ we have exactly $p^{m-1}$ values $a\in \mathbb{F}_p^m$
for which $x(a)=k$ and hence $\biasjs(x)=0$.
For each of the remaining $p$ inputs with $x_1=\cdots=x_m=0$
and different values for $x_0$, we get $\biasjs(x)=\omega^{j^*\cdot x_0}$
and hence $|\biasjs(x)|=1$.
In this case we choose $\kappa=0$ and observe that $\tilde W_0(N)=p^{m+1}$.
Therefore for any $\delta$ with $0\le \delta\le 1$, we have
$$OPT_{\mjs,\delta}\le p^{m+1} |X|^{\delta-1}/|A|=p^{1+(\delta-1)(m+1)}=(p^m)^{-(1-\delta)+\delta/m},$$
so $\tau_{\mjs}(\delta)=\frac12 \log_{|A|} OPT_{\mjs,\delta}\le -\frac{1-\delta}2+\frac{\delta}{2m}$.
(Note that if we only took linear functions instead of affine functions
all non-zero $x$ would be balanced and the term $\frac{\delta}{2m}$ would not
appear. This is the analog of the parity learning bound for higher moduli.)

\paragraph{Quadratic Polynomials over $\mathbb{F}_p$}


\begin{lemma}\label{main_counting_quadratic_complex}
Let $p$ be an odd prime and $m\ge 2$ be an integer.
Let $X$ be the set of quadratic polynomials over $A=\mathbb{F}_p^m$.
Then for $j^*\in \coltset{1,\ldots,p-1}$,
\begin{enumerate}
\item For any $x\in X$,  $\biasjs(x)=0$ or $|\biasjs(x)|\in \coltset{p^{-m/2},p^{(m-1)/2},\cdots, p^{-1/2},1}$.
\item For $0\le k\le m$ the number of $x\in X$ such that
$|\biasjs(x)|=p^{-k/2}$ is less than $p^{km+2k+1}$.
\end{enumerate}
\end{lemma}

To prove Lemma~\ref{main_counting_quadratic_complex} we start with the
following structure lemma for quadratic polynomials
over fields of odd characteristic.  This lemma is an easier analog of
Dickson's Lemma for characteristic 2~\cite{dickson:book} and is well
known but we include a proof for completeness.

\begin{lemma}\label{lem:structure-quadratic}
Let $p$ be an odd prime and integer $t\ge 1$.
For every quadratic polynomial $q$ over $\mathbb{F}_{p^t}$ 
in variables $z=(z_1,\ldots, z_m)$,
there is an invertible affine transformation $T$ over $\mathbb{F}_{p^t}$ such
that for $z'=T(z)$,
there is a unique $k\le m$, and $(c_1,\ldots,c_k)\in \coltset{1,\cdots,p-1}^k$,
and an affine form $\ell$ over $\mathbb{F}_{p^t}$ 
in $m-k$ variables such that:
\begin{align*}
q(z)&=\sum_{i=1}^k c_i{z'_i}^2+\ell(z'_{k+1},\cdots,z'_m)
\end{align*}
\end{lemma}

\begin{proof}
We show this by induction on $m$. The statement is clearly true when $m=0$.
Assume that this is true for any polynomial in $m-1$ variables.
We have several cases when $q$ has $m$ variables:\\
{\sc Case 1:} $q$ is affine: Then the statement is true with $k=0$.\\
{\sc Case 2:} $q$ contains some square term $b_i\cdot z_i^2$:
In this case we can write $q$ as
$b_i\cdot z_i^2+ \ell_i\cdot z_i+q'$,
where $\ell_i$ is affine, $q'$ is a quadratic polynomial,
and neither of them involves $z_i$.
Then we can define
$$z_i'=z_i+2^{-1}b_i^{-1}\cdot \ell_i$$
since $b_i^{-1}$ and $2^{-1}$ are defined in
field $\mathbb{F}_{p^t}$ because $b_i\ne 0$ and the characteristic $p$ is odd.
Also define $q''=q'-2^{-2}b_i^{-1}\ell_i^2$. 
Thus 
\begin{align*}
&b_i(z_i')^2+q''\\
&=b_i(z_i')^2+q'-2^{-2}b_i^{-1}\ell_i^2\\
&=b_i(z_i +2^{-1}b_i^{-1}\cdot \ell_i)^2+q'-2^{-2} b_i^{-1}\ell_i^2\\
&=b_i(z_i^2 + b_i^{-1}\ell_i \cdot z_i + 2^{-2}b_i^{-2}\ell_i^2)
+q'-2^{-2} b_i^{-1}\ell_i^2\\
&=b_i\cdot z_i^2 + \ell_i \cdot z_i +q' =q.
\end{align*}
Define $T_i$ to be the map which sets $z_j'=z_j$ for
$j\ne i$ and replaces $z_i$ with $z_i'$ according to the above
formula.
Clearly by the definition of $z_i'$, $T_i$ is an affine map; moreover,
it is invertible, with $T_i^{-1}$ setting $z_i=z_i' -2^{-1}b_i^{-1}\cdot \ell_i$
and leaving all other $z_j$ for $j\ne i$ unchanged.
By definition, $q''$ is a quadratic form defined only on the $m-1$ variables
$z_1,\ldots,z_{i-1},z_{i+1},\ldots,z_m$, a property inherited from $q'$ and
$\ell_i$.  Let $P_{im}$ be the permutation that swaps positions $i$ and
$m$ and leaves the rest alone and define $q'''=P_{im}(q'')$.

We now can apply the inductive hypothesis to $q'''$ and derive that there
is an invertible affine mapping $T'$ on the $m-1$ variables (excluding $z_i$)
and some $k'$ together with constants $a'_{1},\ldots,a'_{k'}\in \mathbb{F}_p^*$,yielding variables $z''_1,\ldots,z''_{m-1}$ as affine functions of the
previous values such that 
$$q'''=\sum_{j=1}^{k'} c_i{z_{j}''}^2+\ell''(z''_{k'+1},\ldots,z''_{m-1}).$$
We can extend $T'$ to an affine transformation $T''$ on $m$ variables 
by keeping the $m$-th variable unchanged.  

\begin{sloppypar}
Finally, define $k=k'+1$, $c_k=b_i$ and
the invertible affine transformation,
$T=P_{mk}\circ T'' \circ P_{im}\circ T_i$ where
$P_{mk}$ is the permutation that swaps positions $k$ and $m$.  
Then $T(z)=(z''_1,\ldots,z''_{k-1},z'_i,z''_{k+1},\ldots,z''_{m-1},z''_{k})$.
$$T(q)=\sum_{j=1}^{k-1} c_i{z_{j}''}^2+c_k(z_i')^2+\ell''(z''_{k+1},\ldots,z''_{m-1},z''_k)$$
which is of the required form.
\end{sloppypar}

{\sc Case 3:} $q$ has no squared terms and is not affine.
Then $q$ must contain some cross term $b_{ij}\cdot z_iz_j$ for $i\ne j$.
Here we can use the identity
\[
z_iz_j=2^{-2}\cdot ((z_i+z_j)^2-(z_i-z_j)^2)
\]
and let $S_{ij}$ be the affine mapping that leaves all other variables unchanged
and assigns $z_i'=2^{-1}(z_i+z_j)$ and
$z_j'=2^{-1}(z_i-z_j)$ which exists since $2$ is invertible over $\mathbb{F}_{p^t}$.
$S_{ij}$ is clearly invertible since $z_i=z'_i+z'_j$ and $z_j=z'_i-z'_j$.
Hency, for $z'=S_{ij}(z)$, we have $q(z)=q_{ij}(z')$ for some quadratic
$q_{ij}$ that
has two squared terms $(z'_i)^2$ and $(z'_j)^2$ and hence is covered by Case 2
above.
Let $T_2$ be the resulting affine transformation derived for $q_{ij}$.
It follows that $T=T_2\circ S_{ij}$ is the required transformation for $q$.
\end{proof}

Lemma~\ref{lem:structure-quadratic} provides a clean way of studying the bias
of quadratic polynomials.
For any invertible affine mapping $T$ on $\mathbb{F}_p^m$, for $y\in X$
and $x(z)=y(T(z))$, we have $x\in X$ and
\begin{align*}
\biasjs(x)=\E_{a\in \mathbb{F}_p^{m}} \omega^{j^*\cdot x(a)}=\E_{a\in \mathbb{F}_p^{m}} \omega^{j^*\cdot y(T(a))}=\E_{b\in \mathbb{F}_p^m}\omega^{j^* y(b)}=\biasjs(y)
\end{align*}
since $T$ is a bijection on $\mathbb{F}_p^m$.

We therefore first analyze the polynomials of the normal form in
Lemma~\ref{lem:structure-quadratic}.
Let $y(z)=\sum_{i=1}^k c_iz_i^2+\ell(z_{k+1},\ldots, z_{m})$ where
each $c_1,\ldots,c_k \ne 0$.
Write $\ell=c_0+\sum_{i=k+1}^m c_iz_i$.
If there is any $j$ with $k+1\le j\le m$ such that $c_j\ne 0$ then
then $\biasjs(y)=0$ just as in the affine case.
Therefore it remains to consider 
\begin{equation}
y(z)=\sum_{i=1}^k c_iz_i^2+c_0\quad\mbox{ for }c_1,\ldots,c_k\in \F_p^*,\ c_0\in \F_p. \label{y-form}
\end{equation}
Observe that the number of such $y$ is $(p-1)^k p< p^{k+1}$.
Furthermore,
\begin{equation*}
p^m\cdot |\biasjs(y)|=|\sum_{a\in \F_p^m}\omega^{\,j^*\cdot \sum_{i=1}^k c_ia_i^2+c_0}|
=p^{m-k}\cdot\prod_{i=1}^k|\sum_{a_i=0}^{p-1}\omega^{j^*\cdot c_ia_i^2}|
\end{equation*}
The term $\sum_{a_i=0}^{p-1}\omega^{j^*\cdot c_ia_i^2}$ in the product
is called a \emph{quadratic Gauss sum} and has been studied previously.
For our purpose, we need the following result:

\begin{proposition}[Proposition 6.3.2 in \cite{ireland2013classical}]
Let $p$ be an odd prime.
For $c\in \{1,\cdots,p-1\}$,
\[
|\sum_{j=0}^{p-1}\omega^{\,cj^2}|=\sqrt{p}.
\]
\end{proposition}

Therefore setting $c=c_i\cdot j^*$ for the $i$-th term, we have $|\biasjs(y)|=p^{-k/2}$.  We now put things together to
prove Lemma~\ref{main_counting_quadratic_complex}.

\begin{proof}[Proof of Lemma~\ref{main_counting_quadratic_complex}]
By Lemma~\ref{lem:structure-quadratic}, since $\biasjs$ is preserved under
invertible linear transformations $T$ of the inputs, 
it follows that every polynomial $x$ such that $\biasjs(x)\ne 0$ must have
$|\biasjs(x)|=p^{-k/2}$ for some non-negative integer $k\le m$.  
Moreover, the number of polynomials $x$ whose normal form $y$ of the form
(\ref{y-form})
is at most
the number of affine transformations that define $z'_1,\ldots,z'_k$ in
terms of $z_1,\ldots,z_k$ which is $(p^{m+1})^k$ since there are precisely
$p^{m+1}$ affine functions on $\mathbb{F}_p^m$. 
Therefore the total number of $x$ such that $\biasjs(x)=p^{-k/2}$ is less than 
$p^{(m+1))k}\cdot p^{k+1}=p^{mk+2k+1}$.
\end{proof}

Now we can use Proposition \ref{main_counting_quadratic_complex} to prove Theorem \ref{thm:quadcurve-complex}.

\begin{proof}[Proof of Theorem \ref{thm:quadcurve-complex}]
Proposition \ref{main_counting_quadratic_complex} implies that
for $N=(\mjs)*\cdot \mjs$,
\[
W_{p^{m-k/2}}(N)\leq \tilde W_{p^{m-k/2}}(N)\leq \sum_{t=0}^{k-1}p^{m-t/2}\cdot p^{tm+2t+1}=p^{m+1}\cdot \frac{p^{km+3k/2}-1}{p^{m+3/2}-1}\leq p^{km+3k/2}
\]
Therefore if we set $k=\lfloor \frac {1-\delta} 2 m\rfloor\geq \frac {1-\delta} 2 m-1$,
then by Lemma~\ref{c-SDP-lemma}, since $|X|=p^{\binom{m+2}2}$
we have
\[
OPT_{\mjs,\delta}\leq p^{-k}+p^{-m}\cdot p^{km+3k/2+(\delta-1) \binom{m+2}2}\leq 2p^{-k}\leq p^{-\frac {1-\delta} 2 m+2}
\]
Therefore, 
\[
\tau_{\mjs}(\delta)=\frac 1 2\log_{p^m}OPT_{\mjs,\delta}\leq \frac {1-\delta}4+\frac 1 m
\]
Since $j^*$ was an arbitrary fixed element of $\coltset{1,\ldots,p-1}$, the 
theorem follows.
\end{proof}
