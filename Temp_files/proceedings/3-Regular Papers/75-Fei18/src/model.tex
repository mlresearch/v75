
\section{Models and algorithms\label{sec:setup}}

In this section, we formally set up the clustering problem under SGMMs
and describe our SDP relaxation approach.

\subsection{Notations}

We first introduce some notations. Vectors and matrices are denoted
by bold letters such as $\mathbf{\u}$ and $\M$. For a vector $\u$,
we denote by $u_{i}$ its $i$-th entry. For a matrix $\M$, $\Tr(\M)$
denotes its trace, $M_{ij}$ its $(i,j)$-th entry, $\mbox{diag}\left(\M\right)$
the vector of its diagonal entries, $\norm[\M]1\coloneqq\sum_{i,j}M_{ij}$
its entry-wise $\ell_{1}$ norm, $\M_{i\bullet}$ its $i$-th row
and $\M_{\bullet j}$ its $j$-th column. We write $\M\succeq0$ if
$\M$ is symmetric positive semidefinite. The trace inner product
between two matrices $\M$ and $\Q$ of the same dimension is denoted
by $\left\langle \M,\Q\right\rangle \coloneqq\Tr(\M^{\top}\Q)$. For
a number $a$, $\M\geq a$ means $M_{ij}\geq a,\forall i,j$. We denote
by $\onevec_{m}$ the all-one column vector of dimension $m$. For
a positive integer $i$, let $[i]\coloneqq\{1,2,\ldots,i\}$.  For
two non-negative sequences $\{a_{\num}\}$ and $\{b_{\num}\}$, we
write $a_{\num}\lesssim b_{\num}$ if there exists a universal constant
$C>0$ such that $a_{\num}\le Cb_{\num}$ for all $\num$, and write
$a_{\num}\asymp b_{\num}$ if $a_{\num}\lesssim b_{\num}$ and $b_{\num}\lesssim a_{\num}$.
Finally, $\norm[X]{\psi_{2}}\coloneqq\inf\left\{ t>0:\E\exp\left(X^{2}/t^{2}\right)\le2\right\} $
denotes the sub-Gaussian norm of a random variable $X$, and $X$
is called sub-Gaussian if $\norm[X]{\psi_{2}}<\infty$. Note that
Normal and bounded random variables are sub-Gaussian. 

\subsection{Sub-Gaussian Mixture Model \label{sec:setup_model}}

We focus on Sub-Gaussian Mixture Models (SGMMs) with balanced clusters
and isotropic components.

\begin{mdl}[Sub-Gaussian Mixture Model]\label{mdl:SGMM} Let $\Mean_{1},\ldots,\Mean_{\numclust}\in\real^{\vecdim}$
be $\numclust$ unknown cluster centers. We observe $\num$ random
points in $\real^{\vecdim}$ of the form 
\[
\h_{i}\coloneqq\Mean_{\labelstar(i)}+\g_{i},\qquad i\in\left[\num\right]
\]
where $\labelstar(i)\in[\numclust]$ is the unknown cluster label
of the $i$-th point, and $\left\{ \g_{i}\right\} $ are i.i.d.~zero-mean random
vectors such that each $g_{ij}$ are i.i.d.~with $\norm[g_{ij}]{\psi_{2}}=\sgnorm$.
We assume that the ground-truth clusters have equal sizes, that is,
$\left|\left\{ i\in\left[\num\right]:\labelstar(i)=a\right\} \right|=\frac{\num}{\numclust}$
for each $a\in\left[\numclust\right]$.

\end{mdl}

Throughout the paper we assume $\num\ge4$ and $\numclust\ge2$ to
avoid degeneracy. Let $\LabelStar\in[\numclust]^{\num}$ be the vector
of the true cluster labels, that is, its $i$-th coordinate is $\labelstar_{i}\equiv\labelstar(i)$
(we use them interchangeably throughout the paper.) This unknown true
underlying clustering can be encoded by \emph{cluster matrix} $\Ystar\in\{0,1\}^{\num\times\num}$
such that for each $i,j\in\left[\num\right]$, 
\[
\ystar_{ij}=\begin{cases}
\ensuremath{1} & \text{if \ensuremath{\labelstar(i)=\labelstar(j)}, i.e., points \ensuremath{i} and \ensuremath{j} are in the same cluster},\\
\ensuremath{0} & \text{if \ensuremath{\labelstar(i)=\labelstar(j)}, i.e., points \ensuremath{i} and \ensuremath{j} are in different clusters, }
\end{cases}
\]
with the convention $\ystar_{ii}=1,\forall i\in[\num]$. The task
is to estimate the underlying clustering $\Ystar$ given the observed
data $\{\h_{i}:i\in[\num]\}$. From the data one may compute the pairwise
squared distance matrix $\Adj\in\real^{\num\times\num}$, defined
as 
\[
\adj_{ij}=\norm[\h_{i}-\h_{j}]2^{2},\qquad(i,j)\in[\num]\times[\num].
\]
The separation of the centers of clusters $a$ and $b$ is denoted
by $\minsep_{ab}\coloneqq\norm[\Mean_{a}-\Mean_{b}]2$, and $\minsep\coloneqq\min_{a\ne b\in\left[\numclust\right]}\norm[\Mean_{a}-\Mean_{b}]2$
is the minimum separation of the centers. Playing a crucial role in
our results is the quantity 
\begin{equation}
\snr\coloneqq\frac{\minsep}{\sgnorm},\label{eq:snr}
\end{equation}
which is a measure of the SNR of an SGMM. 

\subsection{Semidefinite programming relaxation\label{sec:setup_sdp}}

We now describe our SDP relaxation for clustering SGMMs. To begin,
note that any candidate clustering of $\num$ points into $\numclust$
clusters can be represented using an \emph{assignment matrix $\F\in\{0,1\}^{\num\times\numclust}$
}where 
\[
F_{ia}=\begin{cases}
1 & \text{if point \ensuremath{i} is assigned to cluster \ensuremath{a}}\\
0 & \text{otherwise.}
\end{cases}
\]
Let $\mathcal{F}\coloneqq\left\{ \F\in\{0,1\}^{\num\times\numclust}:\F\one_{\numclust}=\one_{\num}\right\} $
be the set of all possible assignment matrices. Given the points $\{\h_{i}\}$
to be clustered, a natural approach is to find a assignment $\F$
that minimizes the total within-cluster pairwise distance. This objective
can be expressed as
\[
\sum_{i,j}\adj_{ij}\indic\{\text{\ensuremath{i} and \ensuremath{j} are assigned to the same cluster}\}=\sum_{i,j}\adj_{ij}(\F\F^{\top})_{ij}=\left\langle \F\F^{\top},\Adj\right\rangle .
\]
Therefore, the approach described above is equivalent to solving the
integer program (\ref{eq:kmeans1}) below:
\begin{center}
\begin{tabular}{cc}
\begin{minipage}[t]{0.3\textwidth}%
\begin{equation}
\begin{aligned}\min_{\F}\; & \left\langle \F\F^{\top},\Adj\right\rangle \\
\text{s.t.}\; & \F\in\mathcal{F}\\
 & \one_{\num}^{\top}\F=\frac{\num}{\numclust}\one_{\numclust}^{\top}
\end{aligned}
\label{eq:kmeans1}
\end{equation}
%
\end{minipage} & %
\begin{minipage}[t]{0.5\textwidth}%
\begin{equation}
\begin{aligned}\min_{\mathbf{Y}}\; & \left\langle \Y,\Adj\right\rangle \\
\text{s.t.}\; & \Y\one_{\num}=\frac{\num}{\numclust}\one_{\num}\\
 & \Y\succeq0\\
 & \mbox{diag}\left(\Y\right)=\one_{\num}\\
 & \Y\in\left\{ 0,1\right\} ^{\num\times\num};\rank{\Y}=\numclust.
\end{aligned}
\label{eq:kmeans2}
\end{equation}
%
\end{minipage}\tabularnewline
\end{tabular}
\par\end{center}

In (\ref{eq:kmeans1}) the additional constraint $\one_{\num}^{\top}\F=\frac{\num}{\numclust}\one_{\numclust}^{\top}$
enforces that all $\numclust$ clusters have the same size $\frac{\num}{\numclust}$,
as we are working with an SGMM whose true clusters are balanced. Under
this balanced model, it is not hard to see that the program (\ref{eq:kmeans1})
is equivalent to the classical k-means formulation. With a change
of variable $\Y=\F\F^{\top}$, we may lift the program (\ref{eq:kmeans1})
to the space of $\num\times\num$ matrices and obtain the equivalent
formulation (\ref{eq:kmeans2}). Both programs (\ref{eq:kmeans1})
and (\ref{eq:kmeans2}) involve non-convex combinatorial constraints
and are computationally hard to solve. To obtain a tractable formulation,
we drop the non-convex rank constraint in (\ref{eq:kmeans2}) and
replace the integer constraint with a linear constrain $0\le\Y\le1$
(the constraint $\Y\le1$ is redundant). This leads to the following
SDP relaxation:
\begin{equation}
\begin{aligned}\Yhat\in\argmin_{\mathbf{Y}\in\mathbb{R}^{\num\times\num}}\; & \left\langle \Y,\Adj\right\rangle \\
\mbox{s.t.}\; & \Y\one_{\num}=\frac{\num}{\numclust}\one_{\num}\\
 & \Y\succeq0\\
 & \mbox{diag}\left(\Y\right)=\one_{\num}\\
 & \Y\ge0.
\end{aligned}
\label{eq:SDP1}
\end{equation}
It is not hard to see that the true cluster matrix $\Ystar$ is feasible
to program (\ref{eq:SDP1}). We view any optimal solution $\Yhat$
to (\ref{eq:SDP1}) as an estimate of the true clustering $\Ystar$.
Our goal is to characterize the cluster recovery/estimation error
$\norm[\Yhat-\Ystar]1$ in terms of the number of points $\num$,
number of clusters $\numclust$, data dimension $\vecdim$ and SNR
$\snr$ defined above. Note that here we measure the error of $\Yhat$
in $\ell_{1}$ metric; as we shall see later, this metric is directly
related to the clustering error (i.e., the fraction of misclassified
points).

We remark that the SDP (\ref{eq:SDP1}) is somewhat different from
the more classical and well-known SDP relaxation of k-means proposed
by \citet{peng2007approximating}. This SDP (\ref{eq:SDP1}) is closely
related to the one considered by \citet{amini2014semidefinite} in
the context of the Stochastic Block Model, though it seems to be much
less studied under SGMMs with the notable exception of \citet{li2017kmeans}.

\subsection{Explicit clustering \label{sec:setup_explict_clustering}}

Our main results directly concern the SDP solution $\Yhat$, which
is not integral in general and hence does not directly correspond
to an explicit clustering. In case an explicit clustering is desired,
we may easily extract cluster memberships from the solution $\Yhat$
using a simple procedure.  

The procedure consists of two steps given as Algorithms \ref{alg:apx_clustering}
and \ref{alg:est_clustering}, respectively. In the first step, we
treat the rows of $\Yhat$ as elements of $\real^{\num}$, and consider
the $\ell_{1}$ balls centered at each row with a certain radius.
The ball that contains the most rows is identified, and the indices
of the rows in this ball are output and removed. The process continues
iteratively with the remaining rows of $\Yhat$. This step outputs
a number of sets whose sizes are no larger than $\frac{\num}{\numclust}$
but may not equal to each other. 
\begin{algorithm}[H]
\caption{First step \label{alg:apx_clustering}}

Input: data matrix $\Yhat\in\real^{\num\times\num}$, size of each
cluster $\frac{\num}{\numclust}$.
\begin{enumerate}
\item $B_{0}\leftarrow\emptyset$, $t\leftarrow0$, $\vertexset\leftarrow\left[\num\right]$
\item While $\vertexset\backslash\bigcup_{i=0}^{t}B_{i}\ne\emptyset$:
\begin{enumerate}
\item $t\leftarrow t+1$
\item $\vertexset_{t}\leftarrow\vertexset\backslash\bigcup_{i=0}^{t-1}B_{i}$
\item For each $u\in\vertexset_{t}$: $B(u)\leftarrow\left\{ w\in V_{t}:\norm[\Yhat_{u\bullet}-\Yhat_{w\bullet}]1\leq\frac{\num}{4\numclust}\right\} $.
\item $B_{t}\leftarrow\argmax_{B(u):u\in\vertexset_{t}}\left|B(u)\right|$
\item If $\left|B_{t}\right|>\frac{\num}{\numclust}$, then remove arbitrary
elements in $B_{t}$ so that $\left|B_{t}\right|=\frac{\num}{\numclust}$
\end{enumerate}
\end{enumerate}
Output: sets $\left\{ B_{t}\right\} _{t\ge1}$.
\end{algorithm}
In the second step, we convert the sets output by Algorithm \ref{alg:apx_clustering}
above into $\numclust$ equal-size clusters. This is done by picking
the $\numclust$ largest sets among them, and distributing points
in the remaining sets across the chosen $\numclust$ sets so that
each of the $\numclust$ sets contains exactly $\frac{\num}{\numclust}$
points. 
\begin{algorithm}[H]
\caption{Second step \label{alg:est_clustering}}

Input: approximate clustering sets $\left\{ B_{t}\right\} _{t\ge1}$,
number of points $\num$, number of clusters to extract $\numclust$.
\begin{enumerate}
\item $\numclust'\leftarrow\left|\left\{ B_{t}\right\} _{t\ge1}\right|$
\item Choose $\numclust$ largest sets among $\left\{ B_{t}\right\} _{t\ge1}$
and rename the chosen sets as $\left\{ U_{t}\right\} _{t\in\left[\numclust\right]}$
\item Arbitrarily distribute elements of $\left\{ B_{t}\right\} _{t\ge1}\backslash\left\{ U_{t}\right\} _{t\in\left[\numclust\right]}$
among $\left\{ U_{t}\right\} _{t\in\left[\numclust\right]}$ so that
each $U_{t}$ has exactly $\frac{\num}{\numclust}$ elements
\item For each $i\in\left[\num\right]$: $\labelhat_{i}\leftarrow t$, where
$t$ is the unique index in $[\numclust]$ such that $i\in U_{t}$
\end{enumerate}
Output: clustering assignment vector $\LabelHat\in[\numclust]^{\num}$.
\end{algorithm}
Our final clustering algorithm, $\clustering$, is a combination of
the above two algorithms. 
\begin{algorithm}[H]
\caption{$\protect\clustering$ \label{alg:clustering}}

Input: data matrix $\Yhat\in\real^{\num\times\num}$, number of points
$\num$, number of clusters to extract $\numclust$.
\begin{enumerate}
\item Run Algorithm \ref{alg:apx_clustering} with $\Yhat$ and $\frac{\num}{\numclust}$
as input and get $\left\{ B_{t}\right\} _{t\ge1}$
\item Run Algorithm \ref{alg:est_clustering} with $\left\{ B_{t}\right\} _{t\ge1}$,
$\num$ and $\numclust$ as input and get $\LabelHat$
\end{enumerate}
Output: clustering assignment $\LabelHat\in[\numclust]^{\num}$.
\end{algorithm}
The output 
\[
\LabelHat\coloneqq\clustering(\Yhat,\num,\numclust)
\]
is a vector in $[\numclust]^{\num}$ such that point $i$ is assigned
to the $\labelhat_{i}$-th cluster. We are interested in controlling
the clustering error of $\LabelHat$ relative to the ground-truth
clustering $\LabelStar$. Let $\calS_{\numclust}$ denote the symmetric
group consisting of all permutations of $[\numclust]$. The clustering
error is defined by
\begin{equation}
\misrate(\LabelHat,\LabelStar)\coloneqq\min_{\perm\in\calS_{\numclust}}\frac{1}{n}\left|\left\{ i\in[\num]:\labelhat_{i}\neq\pi(\labelstar_{i})\right\} \right|,\label{eq:mis_rate_def}
\end{equation}
which is the proportion of points that are misclassified, modulo permutations
of the cluster labels. 

Variants of the above $\clustering$ procedure have been considered
before by \citet{makarychev2016learning} and \citet{mixon2017clustering}.
In our main results, we show that the clustering error $\misrate(\LabelHat,\LabelStar)$
is always upper bounded by the $\ell_{1}$ error $\norm[\Yhat-\Ystar]1$
of the SDP solution $\Yhat$. 

 
