%% LyX 2.2.2 created this file.  For more info, see http://www.lyx.org/.
%% Do not edit unless you really know what you are doing.
%\documentclass[11pt,english]{article}

\documentclass[final,12pt]{colt2018} % Anonymized submission
% \documentclass{colt2017} % Include author names

% The following packages will be automatically loaded:
% amsmath, amssymb, natbib, graphicx, url, algorithm2e

%\title{Hidden Integrality and Exponential Rates of SDP Relaxations for Sub-Gaussian Mixture Models}
\title{Hidden Integrality of SDP Relaxations for Sub-Gaussian Mixture Models}


 % Two authors with the same address
 % \coltauthor{\Name{Author Name1} \Email{abc@sample.com}\and
 %  \Name{Author Name2} \Email{xyz@sample.com}\\
 %  \addr Address}
 
% Two authors with the same address (different default)
\coltauthor{\Name{Yingjie Fei} \Email{yf275@cornell.edu}\\
	\Name{Yudong Chen} \Email{yudong.chen@cornell.edu}\\
	\addr Cornell University}
\usepackage{times}


%\usepackage[T1]{fontenc}
%\usepackage[latin9]{inputenc}
%\usepackage{geometry}
%\geometry{verbose,tmargin=1.5in,bmargin=1.5in,lmargin=1.3in,rmargin=1.3in}
%\synctex=-1
\usepackage{color}
\usepackage[english]{babel}
\usepackage{array}
\usepackage{float}
\usepackage{mathtools}
\usepackage{multirow}
%\usepackage{amsmath}
%\usepackage{amsthm}
%\usepackage{amssymb}
%\usepackage[authoryear]{natbib}
\usepackage{xargs}[2008/03/08]
%\usepackage[pdfusetitle,
% bookmarks=true,bookmarksnumbered=false,bookmarksopen=false,
% breaklinks=false,backref=false,colorlinks=true]
% {hyperref}
 
%\usepackage[pdfusetitle,
%bookmarks=true,bookmarksnumbered=false,bookmarksopen=false,
%breaklinks=false,backref=false,colorlinks=true]
%{hyperref}

%\hypersetup{
% citecolor=blue, linkcolor=red}

%\makeatletter

%%%%%%%%%%%%%%%%%%%%%%%%%%%%%% LyX specific LaTeX commands.
%% Because html converters don't know tabularnewline
\providecommand{\tabularnewline}{\\}
\floatstyle{ruled}
\newfloat{algorithm}{tbp}{loa}
\providecommand{\algorithmname}{Algorithm}
\floatname{algorithm}{\protect\algorithmname}

%%%%%%%%%%%%%%%%%%%%%%%%%%%%%% Textclass specific LaTeX commands.
%\theoremstyle{plain}
\newtheorem{thm}{\protect\theoremname}
%  \theoremstyle{plain}
  \newtheorem{cor}{\protect\corollaryname}
%  \theoremstyle{plain}
  \newtheorem{prop}{\protect\propositionname}
%  \theoremstyle{plain}
  \newtheorem{lem}{\protect\lemmaname}
%  \theoremstyle{plain}
  \newtheorem{fact}{\protect\factname}

%%%%%%%%%%%%%%%%%%%%%%%%%%%%%% User specified LaTeX commands.
%\usepackage{fancyheadings}
\usepackage[mathscr]{euscript}
\usepackage{amsfonts}
%\usepackage{dsfont}
%\usepackage{times}
%\usepackage{amsmath}
%\pdfmapfile{=mtpro2.map}
%\usepackage[lite]{mtpro2}

%\pdfmapfile{+txfonts.map}
%\usepackage{txfonts}
%\usepackage{appendix}
\allowdisplaybreaks

%\theoremstyle{definition}
\newtheorem{mdl}{Model}

%\usepackage{backref}
%\newcommand*\diff{\mathop{}\!\mathrm{d}}
%\newcommand*\Diff[1]{\mathop{}\!\mathrm{d^#1}}
%\usepackage{pagebackref}
%\AtBeginDocument{%
%\let\ref\autoref
%\renewcommand\equationautorefname{\@gobble}
%}

%\usepackage[lite,subscriptcorrection,slantedGreek,nofontinfo]{mtpro2}
%\usepackage[lite,subscriptcorrection,nofontinfo]{mtpro2}
%\usepackage[utopia]{mathdesign}
%\renewcommand\thesection{\Alph{section}}


%\pagestyle{fancy}
%\lhead{}
%\chead{}
%\rhead{Yingjie (Tom) Fei}



%\DeclareMathOperator*{\argmin}{\arg\!\min}
%\DeclareMathOperator*{\argmax}{\arg\!\max}

%\newcommand{\E}{\mathbb{E}}
%\newcommand{\Prob}{\mathbb{P}}
%\newcommand{\Var}{\text{Var}}
%\newcommand{\Cov}{\text{Cov}}
%\DeclareMathOperator{\Tr}{Tr}

%\makeatother

  \providecommand{\factname}{Fact}
  \providecommand{\lemmaname}{Lemma}
  \providecommand{\propositionname}{Proposition}
\providecommand{\corollaryname}{Corollary}
\providecommand{\theoremname}{Theorem}

 
\begin{document}
\global\long\def\E{\mathbb{E}}
\global\long\def\P{\mathbb{P}}
\global\long\def\Var{\operatorname*{Var}}
\global\long\def\Cov{\operatorname*{Cov}}
\global\long\def\Tr{\operatorname*{Tr}}
\global\long\def\diag{\operatorname*{diag}}
\newcommandx\norm[2][usedefault, addprefix=\global, 1=\#1]{\left\Vert #1\right\Vert {}_{#2}}
\renewcommandx\norm[2][usedefault, addprefix=\global, 1=\#1]{\Vert#1\|_{#2}}
\global\long\def\opnorm#1{\norm[#1]{\textup{op}}}
\global\long\def\rank#1{\operatorname*{rank}(#1)}
\global\long\def\md#1{\operatorname*{maxdiag}(#1)}
\global\long\def\indic{\operatorname*{\mathbb{I}}}
\global\long\def\diff{\operatorname{d}\!}
\global\long\def\argmax{\operatorname*{arg\,max}}
\global\long\def\argmin{\operatorname*{arg\,min}}
\global\long\def\Bern{\operatorname{Bern}}
\global\long\def\pos{\operatorname{pos}}
\global\long\def\round{\operatorname{round}}

\global\long\def\mtx#1{\bm{#1}}
\global\long\def\vct#1{\bm{#1}}
\global\long\def\real{\mathbb{R}}

\global\long\def\A{\mathbf{A}}
\global\long\def\Atilde{\widetilde{\A}}
\global\long\def\B{\mathbf{B}}
\global\long\def\C{\mathbf{C}}
\global\long\def\D{\mathbf{D}}
\global\long\def\F{\mathbf{F}}
\global\long\def\G{\mathbf{G}}
\global\long\def\H{\mathbf{H}}
\global\long\def\I{\mathbf{I}}
\global\long\def\J{\mathbf{J}}
\global\long\def\K{\mathbf{K}}
\global\long\def\L{\mathbf{L}}
\global\long\def\M{\mathbf{M}}
\global\long\def\P{\mathbb{P}}
\global\long\def\O{\mathbf{O}}
\global\long\def\Q{\mathbf{Q}}
\global\long\def\R{\mathbf{R}}
\global\long\def\U{\mathbf{U}}
\global\long\def\Uperp{\mathbf{U}_{\perp}}
\global\long\def\V{\mathbf{V}}
\global\long\def\Vtilde{\widetilde{\mathbf{V}}}
\global\long\def\Vprime{\mathbf{V}'}
\global\long\def\W{\mathbf{W}}
\global\long\def\Wtilde{\widetilde{\W}}
\global\long\def\X{\mathbf{X}}
\global\long\def\Y{\mathbf{Y}}
\global\long\def\Z{\mathbf{Z}}
\global\long\def\one{\mathbf{1}}
\global\long\def\calV{\mathcal{V}}
\global\long\def\calR{\mathcal{R}}
\global\long\def\calVrow{\mathcal{V}_{\mbox{row}}}
\global\long\def\calVcol{\mathcal{V}_{\mbox{col}}}
\global\long\def\x{\mathbf{x}}
\global\long\def\y{\mathbf{y}}
\global\long\def\a{\mathbf{a}}

\global\long\def\Yhat{\widehat{\Y}}
\global\long\def\Ystar{\Y^{*}}
\global\long\def\Adj{\A}
\global\long\def\ObsAdj{\B}
\global\long\def\Noise{\W}
\global\long\def\HalfAdj{\boldsymbol{\Lambda}}
\global\long\def\HalfNoise{\boldsymbol{\Psi}}
\global\long\def\Yhp{\widehat{\Y}_{\perp}}
\global\long\def\Yround{\widehat{\Y}^{\text{R}}}
\global\long\def\CensorMat{\Z}
\global\long\def\bOmega{\boldsymbol{\Omega}}
\global\long\def\Meanhat{\hat{\boldsymbol{\mu}}}
\global\long\def\Mean{\boldsymbol{\mu}}
\global\long\def\mean{\mu}
\global\long\def\g{\mathbf{g}}
\global\long\def\h{\mathbf{h}}
\global\long\def\w{\mathbf{w}}
\global\long\def\u{\mathbf{u}}
\global\long\def\v{\mathbf{v}}

\global\long\def\yhat{\widehat{Y}}
\global\long\def\ystar{Y^{*}}
\global\long\def\adj{A}
\global\long\def\obsadj{B}
\global\long\def\noise{W}
\global\long\def\halfadj{\Lambda}
\global\long\def\halfnoise{\Psi}
\global\long\def\yround{\widehat{Y}^{\text{R}}}
\global\long\def\censormat{Z}

\global\long\def\OneMat{\J}
\global\long\def\onemat{J}
\global\long\def\onevec{\mathbf{1}}

\global\long\def\num{n}
\global\long\def\size{\ell}
\global\long\def\numclust{k}
\global\long\def\snr{s}
\global\long\def\inprob{p}
\global\long\def\outprob{q}
\global\long\def\flipprob{\epsilon}
\global\long\def\obsprob{\alpha}
\global\long\def\apxconst{\rho}
\global\long\def\error{\gamma}
\global\long\def\LabelStar{\boldsymbol{\sigma}^{*}}
\global\long\def\labelstar{\sigma^{*}}
\global\long\def\LabelHat{\widehat{\boldsymbol{\sigma}}}
\global\long\def\labelhat{\widehat{\sigma}}
\global\long\def\z{\mathbf{z}}
\global\long\def\vecdim{d}
\global\long\def\misrate{\operatorname*{\texttt{err}}}
\global\long\def\clustering{\texttt{cluster}}
\global\long\def\minsep{\Delta}
\global\long\def\sepratio{\lambda}
\global\long\def\std{\nu}
\global\long\def\sgnorm{\tau}

\global\long\def\PT{\mathcal{P}_{T}}
\global\long\def\PTperp{\mathcal{P}_{T^{\perp}}}
\global\long\def\positify{{\cal T}}
\global\long\def\clustset#1{C_{#1}^{*}}
\global\long\def\clustest#1{\widehat{C}_{#1}}
\global\long\def\vertexset{V}
\global\long\def\perm{\pi}
\global\long\def\poly{\text{poly}}

\global\long\def\constsumthreeterms{72}
\global\long\def\constoperatornormW{182}
\global\long\def\t{{\displaystyle ^{\top}}}
\global\long\def\Abar{\bar{A}}
\global\long\def\bbar{\bar{b}}
\global\long\def\ybar{\bar{y}}
\global\long\def\xbar{\bar{x}}
\global\long\def\xtilde{\tilde{x}}
\global\long\def\constp{C_{\inprob}}
\global\long\def\consts{C_{\snr}}
\global\long\def\conste{C_{e}}
\global\long\def\constu{c_{u}}
\global\long\def\constdense{c_{d}}
\global\long\def\constgamma{C_{g}}
\global\long\def\constmis{C_{m}}
\global\long\def\constb{C_{b}}

\global\long\def\pairset{\mathcal{L}}
 \global\long\def\calM{\mathcal{M}}
 \global\long\def\calG{\mathcal{G}}
 \global\long\def\calS{\mathcal{S}}
 \global\long\def\calA{\mathcal{A}}
 

\global\long\def\IP{\textup{\texttt{IP}}}
\global\long\def\iperror{\gamma_{\IP}}




%\title{Hidden integrality of SDP relaxation for Sub-Gaussian Mixture Models}
%
%\author{}
%
%\date{}
\maketitle
\begin{abstract}
We consider the problem of finding discrete clustering structures
under Sub-Gaussian Mixture Models. We establish a hidden integrality
property of a semidefinite programming (SDP) relaxation for this problem:
while the optimal solutions to the SDP are not integer-valued in general,
their estimation errors can be upper bounded by the error of an idealized
integer program. The error of the integer program, and hence that
of the SDP, are further shown to decay exponentially in the signal-to-noise
ratio. To the best of our knowledge, this is the first exponentially
decaying error bound for convex relaxations of mixture models. A special
case of this result shows that in certain regimes the SDP solutions
are in fact integral and exact, improving on existing exact recovery
results for convex relaxations. More generally, our result establishes
sufficient conditions for the SDP to correctly recover the cluster
memberships of at least $(1-\delta)$ fraction of the points for any
$\delta\in(0,1)$. Error bounds for estimating cluster centers can
also be derived directly from our results.
\end{abstract}

\begin{keywords}
Sub-Gaussian Mixture Models, semidefinite programming, integer programming.
\end{keywords}
Online learning algorithms are a key tool in web search and content optimization, adaptively learning what users want to see. In a typical application, each time a user arrives, the algorithm chooses among various content presentation options (\eg news articles to display), the chosen content is presented to the user, and an outcome (\eg a click) is observed. Such algorithms must balance \emph{exploration} (making potentially suboptimal decisions now for the sake of acquiring information that will improve decisions in the future) and \emph{exploitation} (using information collected in the past to make better decisions now). Exploration could degrade the experience of a current user, but improves user experience in the long run. This exploration-exploitation tradeoff is commonly studied in the online learning framework of \emph{multi-armed bandits}~\citep{Bubeck-survey12}.

Concerns have been raised about whether exploration in such scenarios could be unfair, in the sense that some individuals or groups may experience too much of the downside of exploration without sufficient upside \citep{bird2016exploring}. We formally study these concerns in the \emph{linear contextual bandits} model~\citep{Langford-www10,chu2011contextual}, a standard variant of multi-armed bandits appropriate for content personalization scenarios.  We focus on \emph{externalities} arising due to exploration, that is, undesirable side effects that the presence of one party may impose on another.


We first examine the effects of exploration at a group level.  We introduce the notion of a \emph{group externality} in an online learning system, quantifying how much the presence of one population (which we dub the majority) impacts the rewards of another (the minority). We show that this impact can be negative, and that, in a particular precise sense, no algorithm can avoid it. This cannot be explained by the absence of suitably good policies since our adoption of the linear contextual bandits framework implies the existence of a feasible policy that is simultaneously optimal for everyone. Instead, the problem is inherent to the process of exploration. We come to a surprising conclusion that more data can sometimes lead to worse outcomes for the users of an explore-exploit-based system. \looseness=-1

We next turn to the effect of exploration at an individual level. We interpret exploration as a potential externality imposed on the current user by future users of the system. Indeed, it is only for the sake of the future users that the algorithm would forego the action that currently looks optimal. To avoid this externality, one may use the greedy algorithm that always chooses the action that appears optimal according to current estimates of the problem parameters. While this greedy algorithm performs poorly in the worst case,
it tends to work well in many applications and experiments.\footnote{Both positive and negative findings are folklore. One way to precisely state the negative result is that the greedy algorithm incurs constant per-round regret with constant probability; while results of this form have likely been known for decades,
\citet[Corollary A.2(b)]{competingBandits-itcs16}
proved this for a wide variety of scenarios. Very recently, the good empirical performance has been confirmed by state-of-art experiments in \citet{practicalCB-arxiv18}.}

In a new line of work, \citet{bastani2017exploiting} and \citet{kannan2018smoothed}
analyzed conditions under which inherent diversity in the data makes explicit exploration unnecessary.
\citet{kannan2018smoothed} proved that the greedy algorithm achieves a regret rate of
$\tilde{O}(\sqrt{T})$ in expectation over small perturbations of the context vectors (which ensure sufficient data diversity). This is the best rate that can be achieved in the worst case (\ie for all problem instances, without data diversity assumptions), but it leaves open the possibilities that (i) another algorithm may perform much better than the greedy algorithm on some problem instances, or (ii) the greedy algorithm may perform much better than worst case under the diversity conditions. We expand on this line of work. We prove that under the same diversity conditions, the greedy algorithm almost matches the best possible Bayesian regret rate of \emph{any} algorithm \emph{on the same problem instance}. This could be as low as $\polylog(T)$ for some instances, and, as we prove, at most $\tilde{O}(T^{1/3})$ whenever the diversity conditions hold.


Returning to group-level effects, we show that under the same diversity conditions, the negative group externalities imposed by the majority essentially vanish if one runs the greedy algorithm. Together, our results illustrate a sharp contrast between the high individual and group externalities that exist in the worst case, and the ability to remove all externalities if the data is sufficiently diverse.   \looseness=-1

\xhdr{Additional motivation.} Whether and when explicit exploration is necessary is an important concern in the study of the exploration-exploitation tradeoff. Fairness considerations aside, explicit exploration is expensive. It is wasteful and risky in the short term, it adds a layer of complexity to algorithm design \citep{Langford-nips07,monster-icml14}, and its adoption at scale tends to require substantial systems support and buy-in from management \citep{MWT-WhitePaper-2016,DS-arxiv}. A system based on the greedy algorithm would typically be cheaper to design and deploy.

Further, explicit exploration can run into incentive issues in applications such as recommender systems. Essentially, when it is up to the users which products or experiences to choose and the algorithm can only issue recommendations and ratings, an explore-exploit algorithm needs to provide incentives to explore for the sake of the future users \citep{Kremer-JPE14,Frazier-ec14,Che-13,ICexploration-ec15,Bimpikis-exploration-ms17}. Such incentive guarantees tend to come at the cost of decreased performance, and rely on assumptions about human behavior. The greedy algorithm avoids this problem as it is inherently consistent with the users' incentives.



\xhdr{Additional related work.}
Our research draws inspiration from the growing body of work on fairness in machine learning~\cite[e.g.,][]{dwork2012fairness,hardt2016equality,kleinberg2017inherent,chouldechova2017fair}.  Several other authors have studied fairness in the context of the contextual bandits framework.  Our work differs from the line of research on meritocratic fairness in online learning \citep{kearns2017meritocratic,liu2017calibrated,joseph2016fairness}, which considers the allocation of limited resources such as bank loans and requires that nobody should be passed over in favor of a less qualified applicant. We study a fundamentally different scenario in which there are no allocation constraints and we would like to serve each user the best content possible.  Our work also differs from that of \citet{celis2017fair}, who studied an alternative notion of fairness in the context of news recommendations. According to this notion, all users should have approximately the same probability of seeing a particular type of content (e.g., Republican-leaning articles), regardless of their individual preferences, in order to mitigate the possibility of discriminatory personalization.

The data diversity conditions in \citet{kannan2018smoothed} and this paper are inspired by the smoothed analysis framework of \citet{SmoothedAnalysis-jacm04}, who proved that the expected running time of the simplex algorithm is polynomial for perturbations of any initial problem instance (whereas the worst-case running time has long been known to be exponential). Such disparity implies that very bad problem instances are brittle. 
We find a similar disparity for the greedy algorithm in our setting.



\xhdr{Our results on group externalities.}  A typical goal in online learning is to minimize \emph{regret}, the (expected) difference between the cumulative reward that would have been obtained had the optimal policy been followed at every round and the cumulative reward obtained by the algorithm.  We define a corresponding notion of \emph{minority regret}, the portion of the regret experienced by the minority.  Since online learning algorithms update their behavior based on the history of their observations, minority regret is influenced by the entire population on which an algorithm is run.  If the minority regret is much higher when a particular algorithm is run on the full population than it is when the same algorithm is run on the minority alone, we can view the majority as imposing a negative externality on the minority; the minority population would achieve a higher cumulative reward if the majority were not present. Asking whether this can ever happen
amounts to asking whether access to more data points can ever lead an explore-exploit algorithm to make inferior decisions. One might think that more data should always lead to better decisions and therefore better outcomes for the users.
Surprisingly, we show that this is not the case, even with a standard algorithm.

Consider LinUCB~\citep{Langford-www10,chu2011contextual,abbasi2011improved}, a standard algorithm for linear contextual bandits that is based on the principle of ``optimism under uncertainty.''  We provide a specific problem instance on which, after observing $T$ users, LinUCB would have a minority regret of $\Omega(\sqrt T)$ if run on the full population, but only constant minority regret if run on the minority alone. While stylized, this example is motivated by the problem of providing driving directions to different populations of users, about which fairness concerns have been raised~\citep{bird2016exploring}. Further, the situation is reversed on a slight variation of this example: LinUCB obtains constant minority regret when run on the full population and $\Omega(\sqrt T)$ on the minority alone.  That is, group externalities can be large and positive in some cases, and large and negative in others.

Although these regret rates are specific to LinUCB, we show that this phenomenon is, in some sense, unavoidable. Consider the minority regret of LinUCB when run on the full population and the minority regret that LinUCB would incur if run on the minority alone. We know that one may be much smaller or larger than the other. We ask whether any algorithm could  achieve the minimum of the two on every problem instance. Using a variation of the same problem instance, we prove that this is impossible; in fact, no algorithm could simultaneously approximate both up to any $o(\sqrt{T})$ factor. In other words, an externality-free algorithm would sometimes ``leave money on the table."


In terms of techniques, we rely on the special structure of our example, which can be viewed as an instance of the sleeping bandits problem~\citep{SleepingBandits-ml10}. This simplifies the behavior and analysis of LinUCB, allowing us to obtain the $O(1)$ upper bounds.  The lower bounds are obtained using KL-divergence techniques to show that the two variants of our example are essentially indistinguishable, and an algorithm that performs well on one must obtain $\Omega(\sqrt{T})$ regret on the other. \looseness=-1


\xhdr{Our results on the greedy algorithm.} We consider a version of linear contextual bandits in which the latent weight vector $\theta$ is drawn from a known prior. In each round, an algorithm is presented several actions to choose from, each represented by a \emph{context vector}. The expected reward of an action is a linear product of $\theta$ and the corresponding context vector. The tuple of context vectors is drawn independently from a fixed distribution. In the spirit of smoothed analysis, we assume that this distribution has a small amount of jitter. Formally, the tuple of context vectors is drawn from some fixed distribution, and then a small \emph{perturbation}---small-variance Gaussian noise---is added independently to each coordinate of each context vector. This ensures arriving contexts are diverse. We are interested in Bayesian regret, i.e., regret in expectation over the Bayesian prior. Following the literature, we are primarily interested in the dependence on the time horizon $T$. \looseness=-1

We focus on a batched version of the greedy algorithm, in which new data arrives to the algorithm's optimization routine in small batches, rather than every round. This is well-motivated from a practical perspective---in high-volume applications data usually arrives to the ``learner" only after a substantial delay \citep{MWT-WhitePaper-2016,DS-arxiv}---and is essential for our analysis.

Our main result is that the greedy algorithm matches the Bayesian regret of any algorithm up to polylogarithmic factors, for each problem instance, fixing the Bayesian prior and the context distribution. We also prove that LinUCB achieves regret $\tilde{O}(T^{1/3})$ for each realization of $\theta$. This implies a worst-case Bayesian regret of $\tilde{O}(T^{1/3})$ for the greedy algorithm under the perturbation assumption. \looseness=-1

Our results hold for both natural versions of the batched greedy algorithm, Bayesian and frequentist, henceforth called \BayesGreedy and \FreqGreedy. In \BayesGreedy, the chosen action maximizes expected reward according to the Bayesian posterior. \FreqGreedy estimates $\theta$ using ordinary least squares regression and chooses the best action according to this estimate. The results for \FreqGreedy come with additive polylogarithmic factors, but are stronger in that the algorithm does not need to know the prior. Further, the $\tilde{O}(T^{1/3})$ regret bound for \FreqGreedy is approximately prior-independent, in the sense that it applies even to very concentrated priors such as independent Gaussians with standard deviation on the order of $T^{-2/3}$.

The key insight in our analysis of \BayesGreedy is that any (perturbed) data can be used to simulate any other data, with some discount factor. The analysis of \FreqGreedy requires an additional layer of complexity. We consider a hypothetical algorithm that receives the same data as \FreqGreedy, but chooses actions based on the Bayesian-greedy selection rule. We analyze this hypothetical algorithm using the same technique as \BayesGreedy, and then upper bound the difference in Bayesian regret between the hypothetical algorithm and \FreqGreedy.

Our analyses extend to group externalities and (Bayesian) minority regret. In particular, we circumvent the impossibility result mentioned above. We prove that both \BayesGreedy and \FreqGreedy match the Bayesian minority regret of any algorithm run on either the full population or the minority alone, up to polylogarithmic factors

\xhdr{Detailed comparison with prior work.} We substantially improve over the $\tilde{O}(\sqrt{T})$ worst-case regret bound from \citet{kannan2018smoothed}, at the cost of some additional assumptions. First, we consider Bayesian regret, whereas their regret bound is for each realization of $\theta$.%
\footnote{Equivalently, they allow point priors, whereas our priors must have variance $T^{-O(1)}$.} Second, they allow the context vectors to be chosen by an adversary before the perturbation is applied. Third, they extend their analysis to a somewhat more general model, in which there is a separate latent weight vector for every action (which amounts to a more restrictive model of perturbations). However, this extension relies on the greedy algorithm being initialized with a substantial amount of data. The results of \citet{kannan2018smoothed} do not appear to have implications on group externalities.

\citet{bastani2017exploiting} show that the greedy algorithm achieves logarithmic regret in an alternative linear contextual bandits setting that is incomparable to ours in several important ways.
They consider two-action instances where the actions share a common context vector in each round, but are parameterized by different latent vectors. They ensure data diversity via a strong assumption on the context distribution. This assumption does not follow from our perturbation conditions; among other things, it implies that each action is the best action in a constant fraction of rounds. Further, they assume a version of Tsybakov's \emph{margin condition}, which is known to substantially reduce regret rates in bandit problems \citep[\eg see][]{Zeevi-colt10}.



\section{Property Testing Background and Models}
\label{sec:model}
\subsection{Query Testing (Standard Property Testing)}
Given functions $f$ and $g$ over domain $X$, we define the distance between $f$ and $g$ with respect to distribution $\calD$ over $X$ to be
\begin{equation}
\dist_\calD(f,g)=\Pr_{x\sim\calD}[f(x)\neq g(x)].
\end{equation}
Given a class $\calC$ of functions over domain $X$ and a margin $\epsilon$, a \emph{property tester} distinguishes the case that the input function $f$ is in the class $\calC$ from the case that $f$ is $\epsilon$-far from $\calC$:
\begin{enumerate}
\item if $f\in\calC$, the tester accepts $f$ with probability at least $\frac{2}{3}$;
\item if $\forall g\in\calC,\dist_\calD(f,g)>\epsilon$, the tester rejects $f$ with probability at least $\frac{2}{3}$.
\end{enumerate}
\citet{RS96} first studied the property testing model assuming $X$ is finite and $\calD$ is uniform. We call the testing model of \citet{RS96} as \emph{query testing}, because the tester makes queries to access $f$, i.e., the tester asks for the value of $f(x)$ for some $x\in X$ for each query it makes.

\citet{PRR06} first studied the \emph{tolerant} version of property testing: given an additional parameter, the threshold $\alpha$, to distinguish a function $\alpha$-close to the class from a function $(\alpha+\epsilon)$-far from the class. In other words, 
\begin{enumerate}
\item if $\exists g\in \calC,\dist_\calD(f,g)\leq\alpha$, the tester accepts $f$ with probability at least $\frac{2}{3}$;
\item if $\forall g\in\calC,\dist_\calD(f,g)>\alpha+\epsilon$, the tester rejects $f$ with probability at least $\frac{2}{3}$.
\end{enumerate}
They showed tolerant testers for clustering and for monotonicity in the query testing model. \citet{FF05} showed the existence of classes of binary functions that are efficiently query-testable in the non-tolerant case but are not efficiently query-testable in the tolerant case.

\citet{PRR06} also considered a similar task called distance approximation: estimating the distance from the function to the class so that with probability at least $\frac 23$ the output is within $\pm\epsilon$ to the true distance. Note that distance approximation with additive error $\epsilon$ implies tolerant testing with margin $2\epsilon$ with the same query complexity. Based on this observation, all the tolerant testers we design in this paper actually perform distance approximation (so we don't need the parameter $\alpha$) because distance approximation is a slightly more convenient model for our presentation.
\subsection{Passive Testing (Sample-Based Testing)}
\label{subsec:passive}
\citet{GGR98} first studied testers with the ability to obtain a random sample in addition to making queries so that the tester can potentially work on arbitrary distributions (see Section \ref{subsec:distributionfree} for distribution-free testing), although their algorithmic results remained in the query testing framework over the uniform distribution.
\citet{KR98} developed the first \emph{passive} testers, testers that don't make queries and only rely on the random i.i.d.\ sample to access the input function $f$, for a variety of classes with sub-learning sample complexity. \citet{GR13} advanced the study of passive testers by providing several general positive results as well as by revealing relations with other testing models.

\emph{Proper} learning implies testing, simply by testing using the output hypothesis, but passive testing can be substantially harder than \emph{improper} learning. \citet{GGR98} pointed out that the class of $k$-term-DNF is NP-hard for non-tolerant passive testing while it is efficiently PAC learnable via $k$-CNF \citep{PV88}, if we require testing and learning on an arbitrary distribution.

The general hardness of \emph{tolerant} passive testing based on hardness of improper \emph{agnostic} learning can be implied from the recent work by \citet{KL18}. They considered the task of refutation: for any fixed distribution $\calD$ over domain $X$, given a sample of example-label pairs $\{(x_i,y_i)\}$ and margin $\epsilon>0$, to distinguish the following two cases:
\begin{enumerate}
\item accept when every $(x_i,y_i)$ is i.i.d.\ from some distribution $\calD'$ over $X\times\{0,1\}$ with marginal on $X$ being $\calD$ and $\exists f\in\calC,\Pr_{(x,y)\sim\calD'}[f(x)\neq y]\leq\frac 12-\epsilon$;
\item reject when every $x_i$ is i.i.d.\ from $\calD$ and every $y_i$ is i.i.d.\ from the uniform distribution over $\{0,1\}$.
\end{enumerate}
They showed that a refutation algorithm for distribution $\calD$ with margin $\epsilon$ and sample complexity $s$ implies an improper agnostic learning algorithm for the same distribution with error $3\epsilon$ and sample complexity $O(\frac{s^3}{\epsilon^2})$. We show in Appendix \ref{sec:refutation} that the refutation algorithm can be reduced to a tolerant passive tester for arbitrary unknown distributions with threshold $\alpha=\frac 12-\frac{3\epsilon}4$, margin $\frac{\epsilon}{2}$, and sample complexity $\Omega(s)$, implying that tolerant passive testing for arbitrary unknown distributions can't be substantially more sample-efficient than improper agnostic learning for any distribution $\calD$ (with some reasonable assumptions about the distribution $\calD$). %The reduction also has a uniform-distribution version (Lemma \ref{}), implying the hardness of tolerant passive testing over the uniform distribution.

%\citet{Vad17} showed that if one can perform refutation, i.e., distinguishing a function in the class from a random function, using a sample of size $s$, then one can also do improper PAC learning with sample complexity $O(\poly(s))$.  \citet{KL18} showed similar results for the tolerant (agnostic) case. They used a different definition of refutation: distinguishing a function having distance $\frac 12-\epsilon$ to the class from a random function, and showed that if one can perform refutation using a sample of size $s(\epsilon)$, then one can also do improper agnostic learning with sample complexity $O(\frac{\left(s(\frac\epsilon 2)\right)^3}{\epsilon^2})$.\footnote{One difference between the two results is that \citet{Vad17} assumes that the distribution is arbitrary while the result by \citet{KL18} is distribution-specific. } Note that passive testing implies refutation with the same sample complexity both in the non-tolerant and the tolerant case, assuming the concept class has a finite VC-dimension and the distribution has no massive points so that a random function has distance $\frac 12$ to it with probability 1. Therefore these results imply that passive testing can't be substantially more sample-efficient than improper learning.

\subsection{Active Testing}
Both query testing and passive testing have shortcomings. The assumption of query testing that the tester can make queries to arbitrary points in the domain is usually impractical, while passive testing is too restrictive: for the tolerant case, passive testing can't be substantially more sample-efficient than agnostic learning (recall Section \ref{subsec:passive}).

To avoid both shortcomings, \citet{BBBY12} proposed the active testing model where the tester first receives an unlabeled random i.i.d.\ sample and then makes queries to points in the sample. While the size of the unlabeled sample might be comparable to the labeled sample complexity for learning, the number of queries the tester makes should be substantially smaller.
%large (but is still required to be polynomial in the VC-dimension of the class), the number of queries the tester makes should be substantially smaller than the sample complexity for learning. 
They showed (non-tolerant) active testers for unions of $d$ intervals and for linear separators.
\subsection{Distribution-Free Testing}
\label{subsec:distributionfree}
Distribution-free testing \citep{GGR98} considers testers that work on arbitrary unknown distributions with the ability to obtain random i.i.d.\ sample in addition to making queries. \citet{HK03} designed distribution-free testers for low-degree multivariate polynomials, monotone functions, and several other classes.

The difference between distribution-free testing and passive testing (over arbitrary unknown distributions) is that distribution-free testers have the ability to make queries while passive testers don't. However, the query ability is helpful only when we do \emph{non-tolerant} testing where the tester is only required to accept functions in the class, rather than functions having distance 0 to the class with respect to the unknown distribution. For \emph{tolerantly} testing binary functions, we show that distribution-free testing implies passive testing with the same sample complexity (see Section \ref{sec:relationship} Lemmas \ref{lm:distributionfreenontolerant} and \ref{lm:distributionfreetolerant}) and thus the hardness for tolerant passive testing extends automatically to tolerant distribution-free testing.



%\subsection{Property Testing, Tolerant Testing and Distance Approximation}
%Suppose we have a ground set $X$ and a distribution $\calD$ over $X$. For any two binary functions $f,g\in\{0,1\}^X$, we define their distance to be $\dist_{\calD}(f,g)=\Pr_{x\sim \calD}[f(x)\neq g(x)]$. 

%Suppose we also have a concept class $\calC\subseteq\{0,1\}^X$. Given a function $f\in\{0,1\}^X$ and a margin $\epsilon$ as input, the task of property testing $\pt_{\calD}(f,\epsilon)$ \citep{RS96} is to distinguish the case that $f$ belongs to class $\calC$ from the case that $f$ is $\epsilon$-far from $\calC$. In other words, $\forall f$,
%\begin{enumerate}
%\item if $f\in\calC$, the algorithm outputs ``YES'' with probability at least $\frac{2}{3}$;
%\item if $\forall g\in\calC,\dist_{\calD}(f,g)>\epsilon$, the algorithm outputs ``NO'' with probability at least $\frac{2}{3}$.
%\end{enumerate}
%A property testing algorithm may be randomized, and in this case, the goal is to output the correct answer with probability at least $\frac{2}{3}$. The success probability can be boosted to $1-\delta$ by repeating the algorithm for $O(\log\frac{1}{\delta})$ times and taking the majority.

%The function $f$ can be given to the algorithm in many different ways. In the \emph{query testing} framework \citep{RS96}, the algorithm can query the value of $f(x)$ for any $x\in X$. In this framework, we say the algorithm has \emph{query access} to $f$, or has access to $f\qu$. The query complexity of the algorithm, as a function of $\frac{1}{\epsilon}$, is measured by the maximum number of queries needed by the algorithm.

%\citet{BBBY12} argued that the query testing framework is not realistic for machine learning practice. They proposed the \emph{active testing} framework, in which the algorithm first requests $N$ unlabeled examples $x_1,x_2,\cdots,x_{N}\in X$ sampled independently according to $\calD$ and can only choose to query $f(x_i)$ for $1\leq i\leq N$. In this framework, we say the algorithm has \emph{active access} to $f$, or has access to $f\ac$. The maximum value of $N$, as a function of $\frac{1}{\epsilon}$, is called the \emph{unlabeled sample complexity}. The query complexity is still defined as a function of $\frac{1}{\epsilon}$ measuring the maximum number of queries needed by the algorithm.

%\citet{GGR98} and \citet{KR98} studied an even more strict way of accessing $f$, called \emph{passive access}, in which the algorithm is given the label of an example chosen independently at random from $\calD$ for each query the algorithm makes. 

%Tolerant testing $\tot_{\calD}(f,\alpha,\epsilon)$ \citep{PRR06} is a similar task to property testing. The only difference is that besides the margin $\epsilon$, we are given another parameter $\alpha$ as input, and we are asked to distinguish the case that $f$ is $\alpha$-close to $\calC$ from the case that $f$ is $(\alpha+\epsilon)$-far from $\calC$. %following two cases:
%\begin{enumerate}
%\item $\exists g\in\calC,\dist_{\calD}(f,g)\leq \alpha$;
%\item $\forall g\in\calC,\dist_{\calD}(f,g)>\alpha+\epsilon$.
%\end{enumerate} 
%The query complexity of tolerant testing is still measured as a function of $\frac{1}{\epsilon}$ \citep{PRR06}. 

%A natural generalization of tolerant testing is distance approximation, in which we are only given the function $f$ and the margin $\epsilon$ as input and required to output $\hat\alpha$ as an approximation of the distance from $f$ to $\calC$ up to an additive error $\epsilon$. More specifically, the goal of $\da_{\calD}(f,\epsilon)$ is to output $\hat\alpha$ such that $\forall f$,
%\begin{enumerate}
%\item $\forall \alpha$ such that $\exists g\in\calC,\dist_{\calD}(f,g)\leq \alpha$, it holds with probability at least $\frac{2}{3}$ that $\hat\alpha\leq \alpha+\epsilon$;
%\item $\forall \alpha$ such that $\forall g\in\calC,\dist_{\calD}(f,g)>\alpha$, it holds with probability at least $\frac{2}{3}$ that $\hat\alpha> \alpha-\epsilon$.
%\end{enumerate}

%The success probability of a distance approximation algorithm can be boosted to $1-\delta$ by repeating it $O(\log\frac{1}{\delta})$ times and taking the median. 

%Because for any $\calD$ and $\epsilon$, it's clear that a $\da_{\calD}(f,\frac{\epsilon}{2})$ algorithm implies a $\tot_{\calD}(f,\alpha,\epsilon)$ algorithm with the same query and unlabeled sample complexity \citep{PRR06}, we focus on distance approximation rather than tolerant testing throughout the paper.

%For all of the above examples, the distribution $\calD$ appears in the subscript meaning that the distribution is fixed and known to the algorithm. In this paper, we will consider a more general case, for example $\pt(f\qu,\epsilon,\calD)$, in which the distribution $\calD$ is arbitrarily given to the algorithm as input. When the algorithm has active or passive access to $f$, it's possible to consider an even more general case, for example $\pt(f\ac,\epsilon)$, in which the distribution $\calD$ is unknown to the algorithm (but implicitly given to the algorithm when accessing $f$).




%We use $q^{\text{query}}(\epsilon)$ and $q^{\text{active}}(\epsilon)$ to denote the maximum number of queries needed in the query testing model and the active testing model, respectively. \citet{BBBY12} showed that $q^{\text{query}}(\epsilon)\leq q^{\text{active}}(\epsilon)$ always holds for non-tolerant property testing, which can be easily generalized to tolerant testing and distance approximation. They also showed for non-tolerant property testing that when $\calC$ is the class of dictator functions and $\calD$ is the uniform distribution over $\{0,1\}^p$, there is a testing algorithm in the query testing framework whose query complexity $q^{\text{query}}(p,\epsilon)$ does not grow with respect to $p$, while every testing algorithm in the active testing framework requires $q^{\text{active}}(p,\epsilon)=\Omega(\log p)$ queries for any fixed $0<\epsilon<\frac{1}{2}$.

%In this paper, however, we will show in Theorem \ref{thm:unlabeled} a reversed inequality that $q^{\text{active}}(\epsilon)=O(q^{\text{query}}(\frac{\epsilon}{2}))$ when we are required to do testing on \emph{arbitrary} distribution $\calD$ and the queries are required to lie in the support of $\calD$ (when we have query access)\footnote{We can assume without loss of generality that the queries all lie in the support of $\calD$ when we do tolerant testing and distance approximation, but we cannot assume this for non-tolerant property testing. The reason is that $f\in\calC$ is stronger than $\exists g\in\calC,\dist_{\calD}(f,g)=0$.}, uncovering the relationship between query access and active access.


%Similar to Theorem \ref{thm:unlabeled}, we have the following theorem in tolerant testing. Its proof is a simple modification of the proof to Theorem \ref{thm:unlabeled}.






% \documentclass[anon,12pt]{colt2018} % Anonymized submission
\documentclass[final,12pt]{colt2018} % Include author names
\usepackage{mathtools}
\usepackage{enumitem}
\usepackage{physics}
\usepackage[ruled]{algorithm2e}
\usepackage{tikz}
\usepackage{breqn}
\usepackage{soul}
\usepackage{mwe}
\usepackage{caption}
\usepackage{cancel}
\usepackage{algpseudocode}
\usepackage{thmtools}
\usepackage{thm-restate}
\usepackage{cleveref}
\usepackage{mathtools}
\usepackage{bbm}
\usepackage{enumitem}
\usepackage{amsmath}
\declaretheorem[name=Theorem]{thm}
\declaretheorem[name=Lemma]{lem}

\usepackage{lipsum}

\renewcommand\footnotemark{}

\DeclareMathOperator*{\argmin}{arg\,min}
\DeclareMathOperator*{\argmax}{arg\,max}
\DeclareMathOperator*{\mix}{mix}
% \newlength{\commentWidth}
% \setlength{\commentWidth}{4cm}
% \newcommand{\atcp}[1]{\tcp*[r]{\makebox[\commentWidth]{#1\hfill}}}
% \makeatletter
% \renewcommand{\@algocf@capt@plain}{above}% formerly {bottom}
% \makeatother
\newcommand{\nosemic}{\SetEndCharOfAlgoLine{\relax}}
\newtheorem{assumption}{Assumption}
\newcommand{\at}[2][]{#1|_{#2}}
\newcommand{\ghatthetat}{ \bar{g}(\theta_t) }
\newcommand{\normghatthetat}{ \norm{\bar{g}(\theta_t)} }
\newcommand{\Vt}{V_{\theta_t}}
\newcommand{\Phitrans}{\Phi^\top}
\newcommand{\PhiD}{\Phi^\top D}
\newcommand{\Vstar}{V_{\theta^*}}
\newcommand{\PiP}{\Pi_{\perp}}
\newcommand{\Vdiff}{V_{\theta_t} - V_{\theta^*}}
\newcommand{\Dpi}{D}
\newcommand{\indep}{\rotatebox[origin=c]{90}{$\models$}}
\newcommand{\E}{\mathbb{E}}


\newcommand{\Xc}{\mathcal{X}}
\newcommand{\Yc}{\mathcal{Y}}
\newcommand{\Zc}{\mathcal{Z}}
\newcommand{\Pc}{\mathcal{P}}
\newcommand{\Qc}{\mathcal{Q}}
\newcommand{\Oc}{\mathcal{O}}
\newcommand{\Fc}{\mathcal{F}}
\newcommand{\Gc}{\mathcal{G}}
\newcommand{\Rc}{\mathcal{R}}
\newcommand{\Sc}{\mathcal{S}}
\newcommand{\Ac}{\mathcal{A}}
\newcommand{\Mc}{\mathcal{M}}
\newcommand{\Tc}{\mathcal{T}}
\newcommand{\Ic}{\mathcal{I}}
\newcommand{\Vc}{\mathcal{V}}
\newcommand{\Dc}{\mathcal{D}}
\newcommand{\Bc}{\mathcal{B}}
\newcommand{\Hc}{\mathcal{H}}
\newcommand{\Prob}{\mathbb{P}}

% The following packages will be automatically loaded:
% amsmath, amssymb, natbib, graphicx, url, algorithm2e

\title[Finite Time Analysis of TD]{A Finite Time Analysis of Temporal Difference Learning With Linear Function Approximation}
\usepackage{times}
 % Use \Name{Author Name} to specify the name.
 % If the surname contains spaces, enclose the surname
 % in braces, e.g. \Name{John {Smith Jones}} similarly
 % if the name has a "von" part, e.g \Name{Jane {de Winter}}.
 % If the first letter in the forenames is a diacritic
 % enclose the diacritic in braces, e.g. \Name{{\'\mathbb{E}}louise Smith}

 % Two authors with the same address
  % \coltauthor{\Name{Author Name1} \\mathbb{E}mail{abc@sample.com}\and
  %  \Name{Author Name2} \\mathbb{E}mail{xyz@sample.com}\\
  %  \addr Address}

%  Three or more authors with the same address:
%  \coltauthor{\Name{Jalaj Bhandari} \\{jb3618@columbia.edu}\\
%  \\
%   \Name{Daniel Russo} \\{dan.joseph.russo@gmail.com}\\
%   \\
%   \Name{Raghav Singal} \\{rs3566@columbia.edu}\\
%   \addr Columbia University, NYC}

%  \coltauthor{\Name{Jalaj Bhandari} \\
%   \Name{Daniel Russo} \\
%   \Name{Raghav Singal} \\
%   \addr Columbia University, NYC}
  
 % Authors with different addresses:
 \coltauthor{\Name{Jalaj Bhandari} \Email{jb3618@columbia.edu}\\
 \addr Industrial Engineering and Operations Research, Columbia University
 \AND
 \Name{Daniel Russo} \Email{dan.joseph.russo@gmail.com}\\
 \addr Decision Risk and Operations, Columbia Business School
 \AND
 \Name{Raghav Singal} \Email{rs3566@columbia.edu}\\
 \addr Industrial Engineering and Operations Research, Columbia University
 }
 
 

\def\dr#1{\textbf{\color{green}#1}}
\def\jb#1{\textbf{\color{red}#1}}
\def\rs#1{\textbf{\color{blue}#1}}

\begin{document}
\maketitle
\footnote{Extended abstract. Full version available on arXiv with the same name.}
\begin{abstract}
Temporal difference learning (TD) is a simple iterative algorithm used to estimate the value function corresponding to a given policy in a Markov decision process. Although TD is one of the most widely used algorithms in reinforcement learning, its theoretical analysis has proved challenging and few guarantees on its statistical efficiency are available. In this work, we provide a \emph{simple and explicit finite time analysis} of temporal difference learning with linear function approximation. Except for a few key insights, our analysis mirrors standard techniques for  analyzing stochastic gradient descent algorithms, and therefore inherits the simplicity and elegance of that literature. A final section of the paper shows that all of our main results extend to the study of a variant of Q-learning applied to optimal stopping problems.
\end{abstract}

\begin{keywords}
Reinforcement learning, temporal difference learning, finite sample bounds, stochastic gradient descent. 
\end{keywords}



\section{Introduction} \label{sec:introduction}

Reinforcement learning (RL) offers a general paradigm for learning effective policies for stochastic control problems. At the core of RL is the task of value prediction: the problem of learning to predict cumulative discounted future reward as a function of the current state of the system. Usually, this is framed formally as the problem of estimating the value function corresponding to a given policy in a Markov decision process (MDP).  Temporal difference learning (TD),  first introduced by \cite{sutton1988learning}, is the most widely used algorithm for this task. The method approximates the value function by a member of some parametric class of functions. The parameters of this approximation are then updated online in a simple iterative fashion as data is gathered. 

While easy to implement,  theoretical analysis of TD is quite subtle. A central challenge is that TD’s incremental updates, which are cosmetically similar to stochastic gradient updates, are not true gradient steps with respect to any fixed loss function. This makes it difficult to show that the algorithm makes consistent, quantifiable, progress toward any particular goal. Reinforcement learning researchers in the 1990s gathered both limited convergence guarantees for TD and examples of divergence. Many issues were clarified in the work of \cite{tsitsiklis1997analysis}, who established precise conditions for the asymptotic convergence of TD with linear function approximation, and provided counterexamples when these conditions are violated.  With guarantees of asymptotic convergence in place, a natural next step is to understand the algorithm's statistical efficiency. How much data is required to reach a given level of accuracy? Can one give uniform bounds on this, or could data-requirements explode depending on the problem instance? Twenty years after the work of \cite{tsitsiklis1997analysis}, such questions remain largely unsettled. 

In this work, we take a step toward correcting this by providing \emph{a simple and explicit finite time analysis of temporal difference learning.}  We draw inspiration from the analysis of projected stochastic gradient descent. These analyses are simple--enough so that they are frequently taught in machine learning courses--and the explicit bounds they produce provide clear assurance of the robustness of SGD. Unfortunately, there are critical differences between TD and SGD, and as such these simple analyses do not apply to TD. Instead, past work on TD has needed to invoke powerful results from the theory of stochastic approximation. In this work, we uncover an approach to analyzing TD which, except for a few crucial steps, leverages the standard tools for finite time analysis of SGD. In addition to the several novel guarantees we derive in the paper, we feel the analysis offers insight into the dynamics of TD, and we hope our approach helps future researchers derive stronger bounds and principled improvements to the algorithm. 


%-----------------------------------------------------
\bibliography{bibfile}
%-----------------------------------------------------

\end{document}


\acks{Y. Fei and Y. Chen were partially supported by the National Science
	  Foundation CRII award 1657420 and grant 1704828.}
	
\bibliography{references}

%\section*{Acknowledgement}

%Y. Fei and Y. Chen were partially supported by the National Science
%Foundation CRII award 1657420 and grant 1704828.


\appendix
%\appendixpage

\section{Additional notations}

We define the shorthand $\error\coloneqq\norm[\Yhat-\Ystar]1$. For
a matrix $\M$, we write $\norm[\M]{\infty}\coloneqq\max_{i,j}\left|M_{ij}\right|$
as its entry-wise $\ell_{\infty}$ norm, and $\opnorm{\M}$ as its
spectral norm (maximum singular value).  We let $\I$ and $\OneMat$
be the $\num\times\num$ identity matrix and all-one matrix, respectively.
For a real number $x$, $\left\lceil x\right\rceil $ denotes its
ceiling. We denote by $\clustset a\coloneqq\left\{ i\in\left[\num\right]:\labelstar(i)=a\right\} $
the set of indices of points in cluster $a$, and we define $\size\coloneqq\left|\clustset a\right|=\frac{\num}{\numclust}$. 

\section{Proof of Theorem \ref{thm:ip_sdp}\label{sec:proof_ip_sdp}}

\subsection{Initial steps}

We assume $\error>0$ since otherwise we are done. We can write $\Adj=\C+\C\t-2\H\H\t$,
where $\H$ is a matrix whose $i$-th row is the point $\h_{i}$ and
$\C$ is a matrix where the entries in the $i$-th row are identical
and equal to $\norm[\h_{i}]2^{2}$. Since the row-sum constraint in
the program (\ref{eq:SDP1}) ensures that the matrix $\Yhat-\Ystar$
has zero row sum, we have $\left\langle \Yhat-\Ystar,\C\right\rangle =\left\langle \Yhat-\Ystar,\C\t\right\rangle =0$
which implies $\left\langle \Yhat-\Ystar,\C+\C\t\right\rangle =0$.

Let $\G\coloneqq\H-\E\H$ be a matrix of entries in $\H$ with their
means removed. We can compute
\begin{align*}
\H\H\t & =\left(\G+\E\H\right)\left(\G+\E\H\right)\t\\
 & =\G\G\t+\G\left(\E\H\right)\t+\left(\E\H\right)\G\t+\left(\E\H\right)\left(\E\H\right)\t
\end{align*}
and 
\[
\E\H\H^{\top}=\E\G\G\t+\left(\E\H\right)\left(\E\H\right)\t.
\]
Therefore 
\[
\H\H\t-\E\H\H^{\top}=\left(\G\G\t-\E\G\G\t\right)+\G\left(\E\H\right)\t+\left(\E\H\right)\G\t.
\]
Let $\U\in\real^{\num\times\numclust}$ be the matrix of the left
singular vectors of $\Ystar$. For any $\M\in\real^{\num\times\num}$,
define the projection $\PT\left(\M\right)\coloneqq\U\U\t\M+\M\U\U\t-\U\U\t\M\U\U\t$
and its orthogonal complement $\PTperp\left(\M\right)\coloneqq\M-\PT\left(\M\right)$.
The fact that $\Yhat$ is optimal and $\Ystar$ is feasible to the
program (\ref{eq:SDP1}) implies 
\begin{align*}
0 & \leq-\frac{1}{2}\left\langle \Yhat-\Ystar,\Adj\right\rangle \\
 & =\left\langle \Yhat-\Ystar,\H\H\t-\E\H\H^{\top}\right\rangle +\left\langle \Yhat-\Ystar,\E\H\H^{\top}\right\rangle \\
 & =\left\langle \Yhat-\Ystar,\G\G\t-\E\G\G\t+\G\left(\E\H\right)\t+\left(\E\H\right)\G\t\right\rangle +\left\langle \Yhat-\Ystar,\E\H\H^{\top}\right\rangle \\
 & =\left\langle \Yhat-\Ystar,\PT\left(\G\G\t-\E\G\G\t\right)\right\rangle +\left\langle \Yhat-\Ystar,\PTperp\left(\G\G\t-\E\G\G\t\right)\right\rangle \\
 & \quad+2\left\langle \Yhat-\Ystar,\G\left(\E\H\right)\t\right\rangle +\left\langle \Yhat-\Ystar,\E\H\H^{\top}\right\rangle \\
 & \eqqcolon S_{1}+S_{2}+2S_{3}+S_{4}.
\end{align*}
We may control $S_{1}$, $S_{2}$ and $S_{4}$ using the following. 
\begin{prop}
\label{prop:S1} If $\snr^{2}\geq C\left(\sqrt{\frac{\numclust\vecdim}{\num}\log\left(\num\numclust\right)}+\sqrt{\frac{\numclust}{\num}}\log\left(\num\numclust\right)\right)$
for some universal constant $C>0$, then $S_{1}\leq\frac{1}{100}\minsep^{2}\error$
with probability at least $1-\left(2\num\right)^{-2}$.
\end{prop}

\begin{prop}
\label{prop:S2} If $\snr^{2}\geq C\numclust\left(\sqrt{\frac{\vecdim}{\num}}+1\right)$
for some universal constant $C>0$, then $S_{2}\leq\frac{1}{100}\minsep^{2}\error$
with probability at least $1-2e^{-\num}$.
\end{prop}

\begin{prop}
\label{prop:S4} We have $S_{4}=-\frac{1}{2}\sum_{a\ne b}T_{ab}\minsep_{ab}^{2}\le-\frac{1}{4}\minsep^{2}\error$
where $T_{ab}\coloneqq\sum_{i\in\clustset a,j\in\clustset b}\left(\Yhat-\Ystar\right)_{ij}$. 
\end{prop}
The proofs are given in Sections \ref{sec:proof_S1}, \ref{sec:proof_S2}
and \ref{sec:proof_S4}, respectively. Combining the above propositions,
we have $S_{1}+S_{2}\le-\frac{1}{2}S_{4}$ and therefore 
\begin{equation}
0\leq S_{3}+\frac{1}{4}S_{4}\eqqcolon S_{0}\label{eq:error_S3_bound}
\end{equation}
with probability at least $1-\left(2\num\right)^{-C'}-2e^{-\num}$
for some universal constant $C'>0$.

Let $\B\coloneqq\Yhat-\Ystar$. We have 
\begin{align*}
S_{3} & =\sum_{j}\sum_{a}\sum_{i\in C_{a}}B_{ji}\left\langle \Mean_{a},\g_{j}\right\rangle \\
 & =\size\sum_{j}\sum_{a}\left\langle \Mean_{a},\g_{j}\right\rangle \left(\frac{1}{\size}\sum_{i\in\clustset a}B_{ji}\right)\\
 & =\size\sum_{j}\sum_{a\ne\labelstar(j)}\left\langle \Mean_{a}-\Mean_{\labelstar(j)},\g_{j}\right\rangle \left(\frac{1}{\size}\sum_{i\in\clustset a}B_{ji}\right)
\end{align*}
where the last step holds since $\sum_{a\ne\labelstar(j)}\left(\sum_{i\in\clustset a}B_{ji}\right)=-\sum_{i\in\clustset a:a=\labelstar(j)}B_{ji}$
for each $j\in\left[\num\right]$ which follows from the row-sum constraint
of program (\ref{eq:SDP1}). By Proposition \ref{prop:S4}, we have
\begin{align*}
S_{4} & =-\size\sum_{j}\sum_{a\ne\labelstar(j)}\frac{1}{2}\minsep_{\labelstar(j),a}^{2}\left(\frac{1}{\size}\sum_{i\in\clustset a}B_{ji}\right).
\end{align*}
Therefore, we have 
\[
S_{0}=\size\sum_{j}\sum_{a\ne\labelstar(j)}\left(\left\langle \Mean_{a}-\Mean_{\labelstar(j)},\g_{j}\right\rangle -c\minsep_{\labelstar(j),a}^{2}\right)\left(\frac{1}{\size}\sum_{i\in\clustset a}B_{ji}\right)
\]
where $c=\frac{1}{8}$.

To control $S_{0}$, we define $\beta_{ja}\coloneqq\left\langle \Mean_{a}-\Mean_{\labelstar(j)},\g_{j}\right\rangle -c\minsep_{\labelstar(j),a}^{2}$
and consider the program 
\begin{align}
\max_{\X}\  & \sum_{j}\sum_{a\ne\labelstar(j)}\beta_{ja}X_{ja}\nonumber \\
\text{s.t.}\  & 0\leq X_{ja}\leq1,\qquad\forall a\ne\labelstar(j),j\in\left[\num\right]\nonumber \\
 & \sum_{a\ne\labelstar(j)}X_{ja}\leq1,\qquad\forall j\in\left[\num\right]\label{eq: int_opt}\\
 & \sum_{j}\sum_{a\ne\labelstar(j)}X_{ja}=R,\nonumber 
\end{align}
where $R\in(0,\num]$. Let us denote by $V(R)$ the optimal value
of the above program and we let $V(R)=-\infty$ if the program is
infeasible. The constraints of program (\ref{eq:SDP1}) implies that
$\frac{\error}{2\size}\in(0,\num]$ and 
\[
\sum_{j\in\left[\num\right]}\sum_{a\ne\labelstar(j)}\left(\sum_{i\in\clustset a}B_{ji}\right)=\frac{\error}{2}.
\]
Hence, by Equation (\ref{eq:error_S3_bound}), we have 
\begin{equation}
0\leq S_{0}\leq\size\cdot V\left(\frac{\error}{2\size}\right).\label{eq:basic_ineq_upper_bound_V}
\end{equation}


\subsection{Controlling $\protect\error$ by LP}

We show that $\error$ is upper bounded by the objective value of
an LP that is related to program (\ref{eq: int_opt}). If $\error=0$
then the conclusion of Theorem \ref{thm:ip_sdp} holds trivially.
For $\error>0$, we have the following cases:
\begin{enumerate}
\item If $\frac{\error}{2\size}\in(0,1]$, it follows from Equation (\ref{eq:basic_ineq_upper_bound_V})
that the error $\error$ must satisfy 
\[
0\le V\left(\frac{\error}{2\size}\right)=\beta^{*}\frac{\error}{2\size}\le\beta^{*}\left\lceil \frac{\error}{2\size}\right\rceil =V\left(\left\lceil \frac{\error}{2\size}\right\rceil \right)
\]
where $\beta^{*}\coloneqq\max_{j\in\left[\num\right],a\ne\labelstar(j)}\beta_{ja}$.
This implies 
\begin{align*}
\frac{\error}{2\size}\le\left\lceil \frac{\error}{2\size}\right\rceil  & \le\max\left\{ R\in\{0,1,.\ldots\}:V(R)\ge0\right\} .
\end{align*}
\item If $\frac{\error}{2\size}>1$, it follows from Equation (\ref{eq:basic_ineq_upper_bound_V})
that the error $\error$ must satisfy 
\[
0\le V\left(\frac{\error}{2\size}\right)\le\max\left\{ V\left(\left\lceil \frac{\error}{2\size}\right\rceil \right),V\left(\left\lfloor \frac{\error}{2\size}\right\rfloor \right)\right\} =\max\left\{ V\left(\left\lceil \frac{\error}{2\size}\right\rceil \right),V\left(\left\lceil \frac{\error}{2\size}\right\rceil -1\right)\right\} .
\]
In other words, we have
\begin{align*}
\frac{\error}{2\size}\le\left\lceil \frac{\error}{2\size}\right\rceil  & \le\max\left\{ R\in\{0,1,.\ldots\}:V(R)\vee V(R-1)\ge0\right\} \\
 & =1+\max\left\{ R\in\{0,1,.\ldots\}:V(R)\ge0\right\} .
\end{align*}
Note that $\left\lceil \frac{\error}{2\size}\right\rceil \ge2$, and
therefore we must have $1\le\max\left\{ R\in\{0,1,.\ldots\}:V(R)\ge0\right\} $.
This implies 
\[
\frac{\error}{2\size}\le2\max\left\{ R\in\{0,1,.\ldots\}:V(R)\ge0\right\} .
\]
\end{enumerate}
Consequently, we have 
\[
\frac{\error}{2\size}\le2\max\left\{ R\in\{0,1,.\ldots\}:V(R)\ge0\right\} .
\]


\subsection{Converting LP to IP}

We are now ready to formally establish a connection between the error
of the SDP (\ref{eq:SDP1}) and that of the Oracle IP (\ref{eq:oracleIP}),
by relating $\max\left\{ R\in\{0,1,.\ldots\}:V(R)\ge0\right\} $ to
the quantity (\ref{eq:IPerror}). Note that if $R\ge0$ is an integer,
then there exists an optimal solution $\left\{ w_{ja}\right\} $ of
program (\ref{eq: int_opt}) such that $w_{ja}\in\{0,1\}$ for all
$j\in[\num],a\in[\numclust]$. Therefore, if $R\in\{0,1,\ldots\}$
is an integer, then 
\begin{equation}
V(R)=\IP_{1}(R)\coloneqq\left\{ \begin{aligned}\max_{\X}\  & \sum_{j}\sum_{a\ne\labelstar(j)}\beta_{ja}X_{ja}\\
\text{s.t.}\  & X_{ja}\in\{0,1\},\qquad\forall a\ne\labelstar(j),j\in\left[\num\right]\\
 & \sum_{a\ne\labelstar(j)}X_{ja}\leq1,\qquad\forall j\in\left[\num\right]\\
 & \sum_{j}\sum_{a\ne\labelstar(j)}X_{ja}=R
\end{aligned}
\right\} .\label{eq:IP1}
\end{equation}
Combining the last two display equations we obtain that
\begin{align}
\frac{\error}{2\size} & \le2\max\left\{ R\in\{0,1,.\ldots\}:\IP_{1}(R)\ge0\right\} \nonumber \\
 & \overset{}{=}2\cdot\left\{ \begin{aligned}\max_{R,\X}\; & R\\
\text{s.t.}\; & R\in\{0,1,\ldots\}\\
 & \sum_{j}\sum_{a\ne\labelstar(j)}\beta_{ja}X_{ja}\ge0\\
 & X_{ja}\in\{0,1\},\qquad\forall a\ne\labelstar(j),j\in\left[\num\right]\\
 & \sum_{a\ne\labelstar(j)}X_{ja}\leq1,\qquad\forall j\in\left[\num\right]\\
 & \sum_{j}\sum_{a\ne\labelstar(j)}X_{ja}=R,
\end{aligned}
\right\} \nonumber \\
 & =2\cdot\IP_{2}\coloneqq2\cdot\left\{ \begin{aligned}\max_{\X}\; & \sum_{j}\sum_{a\ne\labelstar(j)}X_{ja}\\
\text{s.t.}\; & \sum_{j}\sum_{a\ne\labelstar(j)}\beta_{ja}X_{ja}\ge0\\
 & X_{ja}\in\{0,1\},\qquad\forall a\ne\labelstar(j),j\in\left[\num\right]\\
 & \sum_{a\ne\labelstar(j)}X_{ja}\leq1,\qquad\forall j\in\left[\num\right]
\end{aligned}
\right\} .\label{eq:error_bound2}
\end{align}

Let us reparameterize the integer program $\IP_{2}$ by a change of
variable. Recall that 
\[
\mathcal{F}\coloneqq\left\{ \F\in\{0,1\}^{\num\times\numclust}:\F\one_{\numclust}=\one_{\num}\right\} 
\]
is the set of all possible assignment matrices and $\F^{*}\in\mathcal{F}$
is the true assignment matrix; that is, $F_{ja}^{*}=\indic\left\{ a=\labelstar(j)\right\} $
for all $j\in[\num],a\in[\numclust]$. Consider any feasible solution
$\X$ of $\IP_{2}$; here for each $j\in[\num]$, we may fix $X_{j,\labelstar(j)}=-\sum_{a\neq\labelstar(j)}X_{ja}$
\textemdash{} doing so does not affect the feasibility and objective
value of $\X$ w.r.t. $\IP_{2}$. Define the new variable $\F\coloneqq\F^{*}+\X\in\mathcal{F}$.
The objective value and constraints of the old variable $\X$ can
be mapped to those of $\F$; in particular, we have 
\begin{align*}
\sum_{j}\sum_{a\ne\labelstar(j)}X_{ja} & =\sum_{j}\sum_{a\ne\labelstar(j)}(F_{ja}-F_{ja}^{*})=\frac{1}{2}\norm[\F-\F^{*}]1
\end{align*}
and
\begin{align*}
\left.\begin{array}{c}
X_{ja}\in\{0,1\},\forall a\ne\labelstar(j),j\in\left[\num\right]\\
\sum_{a\ne\labelstar(j)}X_{ja}\leq1,\forall j\in\left[\num\right]\\
X_{j,\labelstar(j)}=-\sum_{a\neq\labelstar(j)}X_{ja},\forall j\in[\num]
\end{array}\right\}  & \Longleftrightarrow\F\in\mathcal{F}
\end{align*}
and
\[
\sum_{j}\sum_{a\ne\labelstar(j)}\beta_{ja}X_{ja}\overset{(i)}{=}\sum_{j}\sum_{a}\beta_{ja}X_{ja}=\sum_{j}\sum_{a}\beta_{ja}F_{ja}-\sum_{j}\sum_{a}\beta_{ja}F_{ja}^{*}\overset{(ii)}{=}\sum_{j}\sum_{a}\beta_{ja}F_{ja},
\]
where steps $(i)$ and $(ii)$ both follow from the fact that $\beta_{j,\labelstar(j)}=0,\forall j.$
It follows that $\IP_{2}$ has the same optimal value as a corresponding
integer program in terms of $\X$; in particular, we have
\[
\IP_{2}=\IP_{3}\coloneqq\left\{ \begin{aligned}\max_{\F}\; & \frac{1}{2}\norm[\F-\F^{*}]1\\
\text{s.t.}\; & \sum_{j}\sum_{a}\beta_{ja}F_{ja}\ge0\\
 & \F\in\mathcal{F}
\end{aligned}
\right\} .
\]
Combining with equation (\ref{eq:error_bound2}), we see that the
error $\error$ satisfies
\begin{equation}
\frac{\error}{2\size}\le2\cdot\IP_{3}.\label{eq:error_bound3}
\end{equation}

We further simplify the first constraint in $\IP_{3}$. Recall that
$\bar{\h}_{i}\coloneqq\Mean_{\labelstar(i)}+(2c)^{-1}\g_{i}$ for
each $i\in[\num]$. Note that $\left\{ \bar{\h}_{i}\right\} $ can
be viewed as data points generated from the Sub-Gaussian Mixture Model
but with $(2c)^{-1}$ times the standard deviation. By definition
of $\beta_{ja}$, we have 
\begin{align*}
\beta_{ja} & =\left\langle \Mean_{a}-\Mean_{\labelstar(j)},\g_{j}\right\rangle -c\minsep_{\labelstar(j),a}^{2}\\
 & =c\left(2\left\langle \Mean_{a}-\Mean_{\labelstar(j)},(2c)^{-1}\g_{j}\right\rangle -\minsep_{\labelstar(j),a}^{2}\right)\\
 & =c\left(2\left\langle \Mean_{a}-\Mean_{\labelstar(j)},(2c)^{-1}\g_{j}\right\rangle -\norm[\Mean_{a}-\Mean_{\labelstar(j)}]2^{2}\right)\\
 & =c\left(2\left\langle \Mean_{a}-\Mean_{\labelstar(j)},(2c)^{-1}\g_{j}\right\rangle -\norm[\Mean_{a}-\Mean_{\labelstar(j)}]2^{2}-\norm[(2c)^{-1}\g_{j}]2^{2}+\norm[(2c)^{-1}\g_{j}]2^{2}\right)\\
 & =c\left(-\norm[\Mean_{\labelstar(j)}-\Mean_{a}+(2c)^{-1}\g_{j}]2^{2}+\norm[(2c)^{-1}\g_{j}]2^{2}\right)\\
 & =c\left(-\norm[\bar{\h}_{j}-\Mean_{a}]2^{2}+\norm[(2c)^{-1}\g_{j}]2^{2}\right).
\end{align*}
For any $\F\in\mathcal{F}$, we then have
\begin{align*}
\sum_{j}\sum_{a}\beta_{ja}F_{ja} & =c\sum_{j}\sum_{a}\left(-\norm[\bar{\h}_{j}-\Mean_{a}]2^{2}+\norm[(2c)^{-1}\g_{j}]2^{2}\right)F_{ja}\\
 & =c\left(-\sum_{j}\sum_{a}\norm[\bar{\h}_{j}-\Mean_{a}]2^{2}F_{ja}+\sum_{j}\norm[(2c)^{-1}\g_{j}]2^{2}\sum_{a}F_{ja}\right)\\
 & \overset{(i)}{=}c\left(-\sum_{j}\sum_{a}\norm[\bar{\h}_{j}-\Mean_{a}]2^{2}F_{ja}+\sum_{j}\norm[(2c)^{-1}\g_{j}]2^{2}\sum_{a}F_{ja}^{*}\right)\\
 & =c\left(-\sum_{j}\sum_{a}\norm[\bar{\h}_{j}-\Mean_{a}]2^{2}F_{ja}+\sum_{j}\sum_{a}\norm[(2c)^{-1}\g_{j}]2^{2}F_{ja}^{*}\right)\\
 & =c\left(-\sum_{j}\sum_{a}\norm[\bar{\h}_{j}-\Mean_{a}]2^{2}F_{ja}+\sum_{j}\sum_{a}\norm[\bar{\h}_{j}-\Mean_{\labelstar(j)}]2^{2}F_{ja}^{*}\right)\\
 & \overset{(ii)}{=}c\left(-\sum_{j}\sum_{a}\norm[\bar{\h}_{j}-\Mean_{a}]2^{2}F_{ja}+\sum_{j}\sum_{a}\norm[\bar{\h}_{j}-\Mean_{a}]2^{2}F_{ja}^{*}\right),
\end{align*}
where step $(i)$ holds because $\sum_{a}F_{ja}=1=\sum_{a}F_{ja}^{*},\forall j$,
and step $(ii)$ holds because $F_{ja}^{*}=1$ only if $a=\labelstar(j)$.
Again recall the shorthand
\[
\eta(\F)\coloneqq\sum_{j}\sum_{a}\norm[\bar{\h}_{j}-\Mean_{a}]2^{2}F_{ja}.
\]
We have the more compact expression
\begin{equation}
\sum_{j}\sum_{a}\beta_{ja}F_{ja}=c\left(\eta(\F^{*})-\eta(\F)\right)\label{eq:eta_func_equivalence}
\end{equation}
It follows that for any $\F\in\mathcal{F}$, the first constraint
in $\IP_{3}$ is satisfied if and only if 
\[
\eta(\F)\le\eta(\F^{*}).
\]
Combining with the (\ref{eq:error_bound3}), we obtain that
\[
\frac{\error}{2\size}\le2\cdot\IP_{3}=2\cdot\left\{ \begin{aligned}\max_{\F}\; & \frac{1}{2}\norm[\F-\F^{*}]1\\
\text{s.t.}\; & \eta(\F)\le\eta(\F^{*})\\
 & \F\in\mathcal{F}
\end{aligned}
\right\} .
\]
Rearranging terms, we have the bound
\begin{equation}
\error\le2\size\cdot\max\left\{ \norm[\F-\F^{*}]1:\eta(\F)\le\eta(\F^{*}),\F\in\mathcal{F}\right\} .\label{eq:error_bound3a}
\end{equation}
The result follows from the fact that $\norm[\Ystar]1=\num\size$
and $\norm[\F^{*}]1=\num$. 

\subsection{Proof of Proposition \ref{prop:S1} \label{sec:proof_S1}}

In this section we control $S_{1}$. We can further decompose $S_{1}$
as 
\begin{align*}
S_{1} & =\left\langle \Yhat-\Ystar,\U\U\t\left(\G\G\t-\E\G\G\t\right)\right\rangle +\left\langle \Yhat-\Ystar,\left(\G\G\t-\E\G\G\t\right)\U\U\t\right\rangle \\
 & \qquad-\left\langle \Yhat-\Ystar,\U\U\t\left(\G\G\t-\E\G\G\t\right)\U\U\t\right\rangle \\
 & \leq2\left|\left\langle \Yhat-\Ystar,\U\U\t\left(\G\G\t-\E\G\G\t\right)\right\rangle \right|+\left|\left\langle \Yhat-\Ystar,\U\U\t\left(\G\G\t-\E\G\G\t\right)\U\U\t\right\rangle \right|\\
 & \eqqcolon2T_{1}+T_{2}
\end{align*}
By the generalized Holder's inequality, we have 
\begin{align*}
T_{1} & \leq\error\cdot\norm[\U\U\t\left(\G\G\t-\E\G\G\t\right)]{\infty}
\end{align*}
and 
\begin{align*}
T_{2} & =\left|\left\langle \Yhat-\Ystar,\U\U\t\left(\G\G\t-\E\G\G\t\right)\U\U\t\right\rangle \right|\\
 & =\left|\left\langle \left(\Yhat-\Ystar\right)\U\U\t,\U\U\t\left(\G\G\t-\E\G\G\t\right)\right\rangle \right|\\
 & \leq\error\cdot\norm[\U\U\t\left(\G\G\t-\E\G\G\t\right)]{\infty}
\end{align*}
where the last inequality holds since 
\[
\norm[\left(\Yhat-\Ystar\right)\U\U\t]1\leq\norm[\Yhat-\Ystar]1=\error.
\]
Combining the above, we have 
\[
S_{1}\leq3\error\cdot\norm[\U\U\t\left(\G\G\t-\E\G\G\t\right)]{\infty}.
\]

Note that there are $m=\num\numclust$ distinct random variables in
$\U\U\t\left(\G\G\t-\E\G\G\t\right)$ and let us call them $X_{1},\ldots,X_{m}$.
For each $i$, we can see that $X_{i}$ is the average of $\size$
entries in $\G\G\t-\E\G\G\t$ and we let $\B_{i}$ be an $\num\times\num$
matrix with $\size$ entries equal to 1 and the others equal to 0
such that $\size X_{i}=\left\langle \B_{i},\G\G\t-\E\G\G\t\right\rangle $.
To proceed, we need the Hanson-Wright inequality (an extension of
Exercise 6.2.7 on pp.$\ $140 in \citet{vershynin2017high}).
\begin{lem}[Higher-dimensional Hanson-Wright inequality]
\emph{ \label{lem:hanson-wright} }Let $\x_{1},\ldots,\x_{N}$ be
independent, mean zero, sub-Gaussian random vectors in $\real^{M}$.
Let $\B$ be an $N\times N$ matrix. For every $t\geq0$ and some
universal constant $c>0$, we have 
\[
\P\left[\left|\sum_{i,j}B_{ij}\left\langle \x_{i},\x_{j}\right\rangle -\E\sum_{i,j}B_{ij}\left\langle \x_{i},\x_{j}\right\rangle \right|\geq t\right]\leq4\exp\left[-c\min\left(\frac{t^{2}}{K^{4}M\norm[\B]F^{2}},\frac{t}{K^{2}\norm[\B]{}}\right)\right]
\]
where $K\coloneqq\max_{i}\norm[\x_{i}]{\psi_{2}}.$ 
\end{lem}
The proof is given in Section \ref{sec:proof_hanson_wright}. Using
Lemma \ref{lem:hanson-wright}, we see that for any $t\ge0$ 
\[
\P\left\{ \size X_{i}\geq t\right\} =\P\left\{ \left\langle \B_{i},\G\G\t-\E\G\G\t\right\rangle \geq t\right\} \leq4\exp\left[-c\min\left(\frac{t^{2}}{K^{4}\vecdim\size},\frac{t}{K^{2}\sqrt{\size}}\right)\right].
\]
We can choose $t^{*}=DK^{2}\sqrt{\size}\left(\sqrt{\vecdim\log m}+\log m\right)$
with $K=\sgnorm$ and $D>0$ a universal constant. Apply the union
bound, we have 
\[
S_{1}\leq3\error\cdot\frac{1}{\size}\cdot t^{*}
\]
with probability at least $1-m\cdot\P\left\{ \size X\geq t\right\} \geq1-\exp\left(-C'\log m\right)=1-m^{-C'}$
where $C'>0$ is a universal constant. The result follows from the
condition of the proposition.

\subsection{Proof of Proposition \ref{prop:S2} \label{sec:proof_S2}}

In this section we control $S_{2}$. We have 
\begin{align*}
S_{2} & =\left\langle \PTperp\left(\Yhat-\Ystar\right),\G\G\t-\E\G\G\t\right\rangle \\
 & \leq\Tr\left[\PTperp\left(\Yhat-\Ystar\right)\right]\cdot\opnorm{\G\G\t-\E\G\G\t}\\
 & \le\frac{\error}{\size}\cdot\opnorm{\G\G\t-\E\G\G\t}.
\end{align*}
Let $\Var\left(g_{ij}\right)=\std^{2}$. We record a fact about the
sub-Gaussian property of columns of $\G$. 
\begin{fact}
\label{fact:satisfy_cond_gauss_choas_operator_norm_bound} Let $\x\in\real^{\num}$
be an arbitrary column of $\G$. We have 
\[
\norm[\left\langle \x,\w\right\rangle ]{\psi_{2}}\leq C\frac{\sgnorm}{\std}\sqrt{\E\left\langle \x,\w\right\rangle ^{2}}\qquad\text{for any }\w\in\real^{\num},
\]
where $C>0$ is a universal constant and $C\frac{\sgnorm}{\std}\ge1$.
\end{fact}
The proof is given in Section \ref{sec:proof_satisfy_cond}. Applying
Lemma \ref{lem:subg_cov_mat_bound} with $\rho_{0}=\frac{\sgnorm}{\std}$,
we have 
\[
\opnorm{\frac{1}{d}\G\G\t-\frac{1}{d}\E\G\G\t}\leq C_{1}\rho_{0}^{2}\left(\sqrt{\frac{2\num}{\vecdim}}+\frac{2\num}{\vecdim}\right)\opnorm{\frac{1}{d}\E\G\G\t}
\]
with probability at least $1-2e^{-\num}$. Here we let $m=\vecdim,u=\num$
and define $\x_{i}$ to be the $i$-th column of $\G$ and $\x$ to
be a vector independent of but identically distributed as each column of $\G$ (note that columns of $\G$ are identically
distributed). We also use the fact that $\E\x\x\t=\frac{1}{\vecdim}\E\G\G\t=\std^{2}\I$.
Multiplying $\vecdim$ on both sides of the above equation yields
\[
\opnorm{\G\G\t-\E\G\G\t}\leq C_{1}\left(\sqrt{\frac{2\num}{\vecdim}}+\frac{2\num}{\vecdim}\right)\vecdim\sgnorm^{2}.
\]
Hence, we have 
\[
S_{2}\le\frac{\error}{\size}\cdot C_{1}\left(\sqrt{\frac{2\num}{\vecdim}}+\frac{2\num}{\vecdim}\right)\vecdim\sgnorm^{2}=2C_{1}\error\numclust\left(\sqrt{\frac{\vecdim}{\num}}+1\right)\frac{\minsep^{2}}{\snr^{2}}
\]
The result follows from the condition of the proposition.

\subsection{Proof of Proposition \ref{prop:S4} \label{sec:proof_S4}}

We can compute 
\[
\left(\E\H\H\t\right)_{ij}=\begin{cases}
\vecdim\std^{2}+\norm[\Mean_{\labelstar(i)}]2^{2} & \text{if }i=j\\
\norm[\Mean_{\labelstar(i)}]2^{2} & \text{if }i\ne j\text{ and }\labelstar(i)=\labelstar(j)\\
\left\langle \Mean_{\labelstar(i)},\Mean_{\labelstar(j)}\right\rangle  & \text{otherwise}.
\end{cases}
\]
We partition the matrix $\Yhat-\Ystar$ into $\numclust^{2}$ of $\size\times\size$
blocks, and note that $T_{ab}$ denotes the sum of entries within
the $(a,b)$-th block. The constraints of program (\ref{eq:SDP1})
implies that 
\begin{enumerate}
\item $T_{aa}\leq0$ for each $a\in\left[\numclust\right]$ and $T_{ab}\geq0$
for each $a\ne b\in\left[\numclust\right]$;
\item $T_{ab}=T_{ba}$ for each $a,b\in\left[\numclust\right]$;
\item $-T_{aa}=\sum_{b\in\left[\numclust\right]:b\ne a}T_{ab}$ for each
$a\in\left[\numclust\right]$;
\item $-\sum_{a\in\left[\numclust\right]}T_{aa}+\sum_{a,b\in\left[\numclust\right]:a\ne b}T_{ab}=\error$
and thus $-\sum_{a\in\left[\numclust\right]}T_{aa}=\sum_{a,b\in\left[\numclust\right]:a\ne b}T_{ab}=\frac{\error}{2}$.
\end{enumerate}
Since $\Yhat-\Ystar$ has zero diagonal, we can write 
\begin{align*}
S_{4} & =\sum_{a\in\left[\numclust\right]}T_{aa}\norm[\Mean_{a}]2^{2}+2\sum_{a,b\in\left[\numclust\right]:a<b}T_{ab}\left\langle \Mean_{a},\Mean_{b}\right\rangle \\
 & =-\sum_{a,b\in\left[\numclust\right]:a<b}T_{ab}\minsep_{ab}^{2}\\
 & =-\frac{1}{2}\sum_{a,b\in\left[\numclust\right]:a\ne b}T_{ab}\minsep_{ab}^{2}\\
 & \leq-\frac{1}{2}\sum_{a,b\in\left[\numclust\right]:a\ne b}T_{ab}\minsep^{2}\\
 & =-\frac{1}{4}\minsep^{2}\error.
\end{align*}


\section{Proof of Theorem \ref{thm:ip_exp_rate}\label{sec:proof_ip_exp_rate}}

We define the shorthand 
\[
\iperror\coloneqq\max\left\{ \frac{1}{2}\norm[\F-\F^{*}]1:\eta(\F)\le\eta(\F^{*}),\F\in\mathcal{F}\right\} .
\]
It is not hard to see that $\iperror$ takes integer values in $[0,\num]$.
If $\iperror=0$ then we are done. We therefore focus on the case
$\iperror\in\left[\num\right]$.

Suppose $\iperror>3\num\numclust e^{-\snr^{2}/C_{0}^{2}}$ for a fixed
$C_{0}>D/c$. Note that 
\[
3\num\numclust e^{-\snr^{2}/C_{0}^{2}}\overset{(i)}{\le}\num\numclust\cdot\frac{1}{\numclust}\cdot e^{-\snr^{2}/\left(2C_{0}^{2}\right)}\le\num e^{-\snr^{2}/\left(2C_{0}^{2}\right)}<\num
\]
where step $(i)$ holds since we have assumed $\snr^{2}\ge\consts\numclust$
for some universal constant $\consts>0$. We record an important result
for our proof.
\begin{lem}
\label{lm:order_stats} Let $m\ge4$ and $g\ge1$ be integers. Let
$\X\in\real^{m\times g}$ be a matrix such that each $X_{ja}$ is
a sub-Gaussian random variable with its mean equal to $\lambda_{ja}$
and its sub-Gaussian norm no larger than $\rho_{ja}$, and each pair
$X_{ja}$ and $X_{ib}$ are independent for $j\ne i$ and $a,b\in\left[g\right]$.
Then for some universal constant $D>0$ and for any $\beta\in(0,m]$,
we have 
\begin{align*}
\sum_{j,a}X_{ja}M_{ja} & \le D\sqrt{\left\lceil \beta\right\rceil \left(\sum_{j,a}\rho_{ja}^{2}M_{ja}\right)\log\left(3mg/\beta\right)}+\sum_{j,a}\lambda_{ja}M_{ja},\\
 & \qquad\quad\forall\M\in\left\{ 0,1\right\} ^{m\times g}:\M\one_{g}\le\one_{m},\norm[\M]1=\left\lceil \beta\right\rceil ,
\end{align*}
with probability at least $1-\frac{1.5}{m}$.
\end{lem}
The proof is given in Section \ref{sec:proof_lm_order_stat}. Define
the set 
\[
\calM\coloneqq\left\{ \M\in\left\{ 0,1\right\} ^{\num\times\numclust}:\M\one_{\numclust}\le\one_{\num},\norm[\M]1=\iperror,M_{j,\labelstar(j)}=0\ \forall j\in\left[\num\right]\right\} .
\]
For any $\F$ feasible to $\IP_{3}$, we have 
\begin{align*}
0 & \le\frac{1}{c}\left(\eta(\F^{*})-\eta(\F)\right)\\
 & \overset{(i)}{=}\sum_{j\in[\num]}\sum_{a\in[\numclust]}\beta_{ja}F_{ja}\\
 & =\sum_{\left(j,a\right):F_{ja}=1,a\ne\labelstar(j)}\beta_{ja}\\
 & \le\max_{\M\in\calM}\sum_{j}\sum_{a\ne\labelstar(j)}\beta_{ja}M_{ja}\\
 & \overset{(ii)}{\le}\max_{\M\in\calM}\left[D\sqrt{\iperror\sgnorm^{2}\left(\sum_{j}\sum_{a\ne\labelstar(j)}\minsep_{\labelstar(j),a}^{2}M_{ja}\right)\log\left(3\num\left(\numclust-1\right)/\iperror\right)}-c\sum_{j}\sum_{a\ne\labelstar(j)}\minsep_{\labelstar(j),a}^{2}M_{ja}\right]\\
 & \le\max_{\M\in\calM}\left[D\sqrt{\iperror\sgnorm^{2}\left(\sum_{j}\sum_{a\ne\labelstar(j)}\minsep_{\labelstar(j),a}^{2}M_{ja}\right)\frac{\snr^{2}}{C_{0}^{2}}}-c\sum_{j}\sum_{a\ne\labelstar(j)}\minsep_{\labelstar(j),a}^{2}M_{ja}\right]\\
 & \le\left(\frac{D}{C_{0}}-c\right)\cdot\max_{\M\in\calM}\sum_{j}\sum_{a\ne\labelstar(j)}\minsep_{\labelstar(j),a}^{2}M_{ja}
\end{align*}
where step $(i)$ holds by Equation (\ref{eq:eta_func_equivalence}),
step $(ii)$ holds by Lemma \ref{lm:order_stats} with $g=\numclust-1$
since only $\numclust-1$ entries of $\left\{ \beta_{ja}\right\} $
are considered for each $j$ in the sum above $(ii)$, and the last
step holds since $\iperror\minsep^{2}\le\sum_{j}\sum_{a\ne\labelstar(j)}\minsep_{\labelstar(j),a}^{2}M_{ja}$.
Since $C_{0}>D/c$ and $\sum_{j}\sum_{a\ne\labelstar(j)}\minsep_{\labelstar(j),a}^{2}M_{ja}>0$,
the RHS above is negative, which is a contradiction. Hence, we must
have $\iperror\le3\num\numclust e^{-\snr^{2}/C_{0}^{2}}\le\num e^{-\snr^{2}/\left(2C_{0}^{2}\right)}$
and the result follows from the fact that $\norm[\F^{*}]1=\num$.

\section{Proof of technical results}

In this section we provide the proofs of the technical results used
in the proofs of our main theorems.

\subsection{Proof of Lemma \ref{lem:hanson-wright}\label{sec:proof_hanson_wright}}

We record the following lemma (Exercise 6.2.7 on pp.$\ $140 in \citet{vershynin2017high}).
\begin{lem}[Higher-dimensional Hanson-Wright inequality]
\emph{} \label{lem:hanson_wright_hdp} Let $\x_{1},\ldots,\x_{N}$
be independent, mean zero, sub-Gaussian random vectors in $\real^{M}$.
Let $\B=\left\{ B_{ij}\right\} $ be an $N\times N$ matrix. There
exists some universal constant $c>0$ such that for every $t\geq0$
\[
\P\left[\left|\sum_{i,j:i\ne j}^{N}B_{ij}\left\langle \x_{i},\x_{j}\right\rangle \right|\geq t\right]\leq2\exp\left[-c\min\left(\frac{t^{2}}{K^{4}M\norm[\B]F^{2}},\frac{t}{K^{2}\opnorm{\B}}\right)\right]
\]
where $K\coloneqq\max_{i}\norm[\x_{i}]{\psi_{2}}.$ 
\end{lem}
With this result, we only need to prove the same tail bound for $\P\left[\left|\sum_{i=1}^{N}B_{ii}\left(\norm[\x_{i}]2^{2}-\E\norm[\x_{i}]2^{2}\right)\right|\geq t\right]$.
To prove that, we cite another useful lemma (Theorem 2.8.2 on pp.$\ $36
in \citet{vershynin2017high}).
\begin{lem}[Bernstein's inequality for sub-exponential random variables]
\emph{}\label{lem:bernstein-subexp} Let $X_{1},\ldots,X_{N}$ be
independent, mean zero, sub-exponential random variables, and $\a\in\real^{N}$.
Then for every $t\geq0$, we have 
\[
\P\left[\left|\sum_{i=1}^{N}a_{i}X_{i}\right|\geq t\right]\leq2\exp\left[-c\min\left(\frac{t^{2}}{K_{1}^{2}\norm[\a]2^{2}},\frac{t}{K_{1}\norm[\a]{\infty}}\right)\right]
\]
where $K_{1}\coloneqq\max_{i}\norm[X_{i}]{\psi_{1}}$.
\end{lem}
Here, $\norm[\cdot]{\psi_{1}}$ denotes the sub-exponential norm;
see \citet{vershynin2017high} for more details. We work under the
premise of Lemma \ref{lem:hanson_wright_hdp}. Since $\x_{i}$ are
independent sub-Gaussian random vectors, each $\norm[\x_{i}]2^{2}-\E\norm[\x_{i}]2^{2}$
is the sum of $M$ independent, mean zero, sub-exponential random
variables with sub-exponential norm equal to $K^{2}$. Then Lemma
\ref{lem:bernstein-subexp} implies 
\[
\P\left[\left|\sum_{i=1}^{N}B_{ii}\left(\norm[\x_{i}]2^{2}-\E\norm[\x_{i}]2^{2}\right)\right|\geq t\right]\leq2\exp\left[-c\min\left(\frac{t^{2}}{K^{4}M\norm[\B]F^{2}},\frac{t}{K^{2}\opnorm{\B}}\right)\right]
\]
as required. 

\subsection{Proof of Fact \ref{fact:satisfy_cond_gauss_choas_operator_norm_bound}
\label{sec:proof_satisfy_cond}}

We prove the following equivalent statement 
\[
\norm[\left\langle \x,\w\right\rangle ]{\psi_{2}}^{2}\leq C\frac{\sgnorm^{2}}{\std^{2}}\E\left\langle \x,\w\right\rangle ^{2}\qquad\text{for any }\w\in\real^{\num},
\]
where $C>0$ is a universal constant and $C\frac{\sgnorm^{2}}{\std}\ge1.$
We first establish a relationship between $\sgnorm^{2}$ and $\Var\left(x_{1}\right)$:
Proposition 2.5.2 on pp. 24 of \citet{vershynin2017high} implies
that $\frac{C'\sgnorm^{2}}{\std^{2}}\ge\frac{1}{2}$ for some universal
constant $C'>0$. Hence, we have 
\begin{align*}
\norm[\left\langle \x,\w\right\rangle ]{\psi_{2}}^{2} & \overset{(i)}{\le}2C'\sum_{i\in\left[\num\right]}w_{i}^{2}\norm[x_{i}]{\psi_{2}}^{2}\\
 & =2C'\frac{\sgnorm^{2}}{\std^{2}}\sum_{i\in\left[\num\right]}w_{i}^{2}\std^{2}\\
 & \overset{(ii)}{=}2C'\frac{\sgnorm^{2}}{\std^{2}}\E\left\langle \x,\w\right\rangle ^{2},
\end{align*}
where $(i)$ holds according to Proposition 2.6.1 on pp. 28 of \citet{vershynin2017high},
and $(ii)$ holds since $x_{i}$ are i.i.d.$\ $and $\E x_{i}=0$.
Letting $C=2C'$ completes the proof.

\subsection{Proof of Lemma \ref{lm:order_stats} \label{sec:proof_lm_order_stat}}

We define 
\begin{align*}
L_{\M} & \coloneqq\sum_{j,a}\left(X_{ja}-\lambda_{ja}\right)M_{ja},\\
R_{\beta,\M} & \coloneqq D\sqrt{\left\lceil \beta\right\rceil \left(\sum_{j,a}\rho_{ja}^{2}M_{ja}\right)\log\left(3mg/\beta\right)},\\
\calM_{\beta} & \coloneqq\left\{ \M\in\left\{ 0,1\right\} ^{m\times g}:\M\one_{g}\le\one_{m},\norm[\M]1=\left\lceil \beta\right\rceil \right\} .
\end{align*}
To establish a uniform bound in $\beta$, we apply a discretization
argument to the possible values of $\beta$. Define the shorthand
$E\coloneqq(0,m]$. We can cover $E$ by the sub-intervals $E_{t}\coloneqq(t-1,t]$
for $t\in[m]$. For each $t\in[m]$ we define the probability

\begin{align*}
\alpha_{t} & \coloneqq\P\left\{ \exists\beta\in E_{t},\exists\M\in\calM_{\beta}:L_{\M}>R_{\beta,\M}\right\} .
\end{align*}
We bound each of these probabilities: 

\begin{align}
\alpha_{t} & \overset{(i)}{\le}\P\left\{ \exists\M\in\calM_{t}:L_{\M}>R_{t,\M}\right\} \nonumber \\
 & \leq\P\left\{ \bigcup_{\M\in\calM_{t}}\left\{ L_{\M}>R_{t,\M}\right\} \right\} \nonumber \\
 & \leq\sum_{\M\in\calM_{t}}\P\left\{ L_{\M}>R_{t,\M}\right\} ,\label{eq:double union bd on normal-1-1-2}
\end{align}
where step $(i)$ holds since $\beta\in E_{t}$ implies $\beta\le\left\lceil \beta\right\rceil =t$. 

Note that each $X_{ja}-\lambda_{ja}$ is an independent zero-mean
sub-Gaussian random variable and the squared sub-Gaussian norm of
$L_{\M}$ is at most $C_{\psi_{2}}\sum_{j,a}\rho_{ja}^{2}M_{ja}$
where $C_{\psi_{2}}>0$ is a universal constant. We apply Hoeffding
inequality (Lemma \ref{lem:hoeffding}) to bound the probability on
the RHS of (\ref{eq:double union bd on normal-1-1-2}): 
\begin{align*}
\P\left\{ L_{\M}>R_{t,\M}\right\}  & \leq\exp\left\{ -\frac{cD^{2}t\left(\sum_{j,a}\rho_{ja}^{2}M_{ja}\right)\log(3mg/t)}{C_{\psi_{2}}\sum_{j,a}\rho_{ja}^{2}M_{ja}}\right\} \\
 & \leq\exp\left\{ -4t\log(3mg/t)\right\} 
\end{align*}
where $c>0$ is a universal constant. Plugging this back to (\ref{eq:double union bd on normal-1-1-2}),
we have for each $t\in\left[m\right]$, 
\begin{align}
\alpha_{t} & \leq\sum_{\M\in\calM_{t}}\exp\left\{ -4t\log(3mg/t)\right\} \nonumber \\
 & =\binom{m}{t}g^{t}\exp\left\{ -4t\log(3mg/t)\right\} \nonumber \\
 & \leq\left(\frac{me}{t}\right)^{t}g^{t}\exp\left\{ -4t\log(3mg/t)\right\} \nonumber \\
 & \leq\exp\left\{ t\log(3mg/t)+t-4t\log(3mg/t)\right\} \nonumber \\
 & \leq\exp\left\{ -t\log(3mg/t)\right\} =\left(\frac{t}{3mg}\right)^{t},\label{eq:binom coeff bound-1-2}
\end{align}
where the last inequality follows from $t\leq t\log(3mg/t)$ for $t\in\left[m\right]$.
It follows that 
\begin{align*}
 & \quad\P\left\{ \exists\beta\in E,\exists\M\in\calM_{\beta}:L_{\M}>R_{\beta,\M}\right\} \\
 & \leq\P\left\{ \bigcup_{t=1}^{m}\left\{ \exists\beta\in E_{t},\exists\M\in\calM_{\beta}:L_{\M}>R_{\beta,\M}\right\} \right\} \\
 & \le\sum_{t=1}^{m}\alpha_{t}\\
 & \leq\sum_{t=1}^{m}\left(\frac{t}{3mg}\right)^{t}\eqqcolon P_{1}(m).
\end{align*}

It remains to show that $P_{1}(m)\leq\frac{1.5}{m}$. Since 
\begin{align*}
P_{1}(m) & \leq\sum_{t=1}^{m}\left(\frac{t}{3m}\right)^{t}\\
 & \le\frac{1}{3m}+\sum_{t=2}^{m}\left(\frac{t}{3m}\right)^{t}\\
 & \le\frac{1}{3m}+m\cdot\max_{t=2,3,\ldots,m}\left(\frac{t}{3m}\right)^{t},
\end{align*}
the proof is completed if for each integer $t=2,3,\ldots,m$, we can
show the bound $\left(\frac{t}{3m}\right)^{t}\leq\frac{1}{m^{2}}$,
or equivalently $f(t)\coloneqq t(\log3m-\log t)\geq2\log m.$ Since
$t\le m$, $f(t)$ has derivative 
\[
f'(t)=\log3m-\log t-1\ge\log3m-\log\left(\frac{3m}{3}\right)-1=\log3-1\ge0.
\]
Therefore, $f(t)$ is non-decreasing for $2\le t\le m$ and therefore
$f(t)\ge f(2)=2\log3m-2\log2\ge2\log m.$ Hence, $P_{1}(m)\le\frac{1.5}{m}$. 

\section{Proof of Theorem \ref{thm:cluster_error_rate}\label{sec:proof_cluster_error_rate}}

We only need to prove the first part of the theorem. The second part
follows immediately from the first part and Theorem \ref{cor:SDP_exp_rate}.

The proof follows similar lines as those of Theorem 17 and Lemma 18
in \citet{makarychev2016learning}. In the rest of the section, we
work under the context of Algorithms \ref{alg:apx_clustering} and
\ref{alg:est_clustering}. Recall that $\numclust'=\left|\left\{ B_{t}\right\} _{t\ge1}\right|$
and we let $\epsilon\coloneqq\norm[\Yhat-\Ystar]1/\norm[\Ystar]1$.
We have the following lemma.
\begin{lem}
\label{lem:apx_clustering} There exists a partial matching $\perm'$
between $\left[\numclust\right]$ and $\left[\numclust'\right]$ and
a universal constant $C>0$ such that 
\[
\left|\bigcup_{t=\perm'(a)}\clustset a\cap B_{t}\right|\ge\left(1-C\epsilon\right)\num.
\]
\end{lem}
The proof is given in Section \ref{sec:proof_apx_clustering}. The
next lemma concerns the quality of clustering by Algorithm \ref{alg:clustering}.
\begin{lem}
\label{lem:clustering} There exists a permutation $\perm$ on $\left[\numclust\right]$
and a universal constant $C>0$ such that 
\[
\left|\bigcup_{t=\perm(a)}\clustset a\cap U_{t}\right|\ge\left(1-C\epsilon\right)\num.
\]
\end{lem}
The proof is given in Section \ref{sec:proof_clustering}. The result
follows from combining the above lemmas and the fact that 
\[
\misrate(\LabelHat,\LabelStar)=1-\frac{1}{\num}\max_{\perm\in S_{\numclust}}\left|\bigcup_{t=\perm(a)}\clustset a\cap U_{t}\right|.
\]


\subsection{Proof of Lemma \ref{lem:apx_clustering}\label{sec:proof_apx_clustering}}

We define $\y_{a}$ to be an arbitrary row of $\Ystar$ whose index
is in $\clustset a$.
\begin{align*}
G_{a} & \coloneqq\left\{ i\in\clustset a:\norm[\Yhat_{i\bullet}-\y_{a}]1\leq\frac{\size}{8}\right\} ,\qquad\forall a\in\left[\numclust\right]\\
G & \coloneqq\bigcup_{a\in\left[\numclust\right]}G_{a},\\
H & \coloneqq\vertexset\backslash G.
\end{align*}

We construct a partial matching $\perm'$ between sets $\clustset a$
and $B_{t}$ by matching every cluster $\clustset a$ with the first
$B_{t}$ that intersects $G_{a}$, and we let $\perm'(a)=t$. Since
each $i\in\left[\num\right]$ belongs to some $B_{t}$, we are able
to match every $\clustset a$ with some $B_{t}$. The fact that we
cannot match two distinct clusters $\clustset a$ and $\clustset b$
with the same $B_{t}$ as well as the rest of the proof are given
by the following fact.
\begin{fact}
\label{fact:apx_clustering} We have 
\begin{enumerate}
\item For each $a\in\left[\numclust\right]$ and $t\in\left[\numclust'\right]$
such that $t=\perm'(a)$, we have $B_{t}\cap G_{b}=\emptyset$ for
any $b\in\left[\numclust\right]\backslash\left\{ a\right\} $ and
$B_{t}\subset G_{a}\cup H$;
\item For each $a\in\left[\numclust\right]$ and $t\in\left[\numclust'\right]$
such that $t=\perm'(a)$, we have 
\[
\left|B_{t}\cap\clustset a\right|\geq\left|G_{a}\right|-\left|B_{t}\cap H\right|.
\]
\item We have 
\[
\sum_{t=\perm'(a)}\left|B_{t}\cap\clustset a\right|\ge\left|\vertexset\right|-2\left|H\right|.
\]
\item There exists a universal constant $C>0$ such that $\left|H\right|\leq C\epsilon\num$.
\end{enumerate}
\end{fact}
The proof is given below.

\subsubsection{Proof of Fact \ref{fact:apx_clustering}\label{sec:proof_fact_apx_clustering}}
\begin{enumerate}
\item Suppose that there exist $B_{t}$ and $b\in\left[\numclust\right]$
such that $b\ne a$ and $B_{t}\cap G_{b}\ne\emptyset$. Let $u\in B_{t}\cap G_{a}$
and $v\in B_{t}\cap G_{b}$. Since $G_{a}$ and $G_{b}$ are disjoint,
we know that $u\ne v$. Let $w\in B_{t}$. Then we have 
\begin{align*}
\norm[\Yhat_{u\bullet}-\Yhat_{w\bullet}]1 & \leq\frac{\size}{4}\\
\norm[\Yhat_{v\bullet}-\Yhat_{w\bullet}]1 & \leq\frac{\size}{4}.
\end{align*}
Therefore 
\[
\norm[\Yhat_{u\bullet}-\Yhat_{v\bullet}]1\leq\norm[\Yhat_{u\bullet}-\Yhat_{w\bullet}]1+\norm[\Yhat_{v\bullet}-\Yhat_{w\bullet}]1\le\frac{\size}{2}.
\]
This implies 
\begin{align*}
\norm[\y_{a}-\y_{b}]1 & \leq\norm[\y_{a}-\Yhat_{u\bullet}]1+\norm[\Yhat_{u\bullet}-\Yhat_{v\bullet}]1+\norm[\y_{b}-\Yhat_{v\bullet}]1\\
 & \leq\frac{\size}{8}+\frac{\size}{2}+\frac{\size}{8}<\size,
\end{align*}
which is a contradiction to the fact that $\norm[\y_{a}-\y_{b}]1=2\size$.
To complete the proof, we note that for any $i\in B_{t}$ we have
either $i\in G_{a}$ or $i\in H$.
\item Fix $i\in G_{a}$ for some $a\in\left[\numclust\right]$. For any
$j\in G_{a}$ we have $j\in B(i)$ since 
\[
\norm[\Yhat_{i\bullet}-\Yhat_{j\bullet}]1\le\norm[\y_{a}-\Yhat_{i\bullet}]1+\norm[\y_{a}-\Yhat_{j\bullet}]1\le\frac{\size}{4}.
\]
Therefore, by definition 
\[
\left|B_{t}\right|\ge\left|B(i)\right|\ge\left|G_{a}\right|.
\]
We have 
\begin{align*}
\left|B_{t}\cap\clustset a\right| & \overset{(i)}{\ge}\left|B_{t}\cap G_{a}\right|\\
 & =\left|B_{t}\right|-\left|B_{t}\backslash G_{a}\right|\\
 & \overset{(ii)}{=}\left|B_{t}\right|-\left|B_{t}\cap H\right|\\
 & \ge\left|G_{a}\right|-\left|B_{t}\cap H\right|,
\end{align*}
where step $(i)$ holds since $G_{a}\subset\clustset a$ and step
$(ii)$ holds since $B_{t}\subset G_{a}\cup H$.
\item Summing the LHS of the above equation over $t=\perm'(a)$ gives 
\begin{align*}
\sum_{t=\perm'(a)}\left|B_{t}\cap\clustset a\right| & =\sum_{a\in\left[\numclust\right]}\left|G_{a}\right|-\sum_{t=\perm'(a)}\left|B_{t}\cap H\right|\\
 & \ge\sum_{a\in\left[\numclust\right]}\left|G_{a}\right|-\sum_{t\ge1}\left|B_{t}\cap H\right|\\
 & \overset{(i)}{=}\left|G\right|-\left|\vertexset\cap H\right|\\
 & =\left|\vertexset\right|-2\left|H\right|,
\end{align*}
where step $(i)$ holds since $B_{t}\cap H$ are disjoint and $\bigcup_{t\geq1}B_{t}=\vertexset$.
\item We have 
\[
\left|H\right|\cdot\frac{\size}{8}\leq\sum_{i\in H}\norm[\Yhat_{i\bullet}-\y_{\labelstar(i)}]1\leq\norm[\Yhat-\Ystar]1\leq\epsilon\norm[\Ystar]1=\epsilon\cdot\num\size
\]
where the last step follows from the fact that $\norm[\Ystar]1=\num\size$.
The result follows.
\end{enumerate}

\subsection{Proof of Lemma \ref{lem:clustering}\label{sec:proof_clustering}}

Let $\perm'$ be the partial matching between $\clustset a$ and $B_{t}$
from Lemma \ref{lem:apx_clustering}. Define $\perm(a)=\perm'(a)$
for $\perm'(a)\le\numclust$. If the resulting $\perm$ is a partial
permutation, we extend $\perm$ to a permutation defined on $\left[\numclust\right]$
in an arbitrary way. We may assume that $\left\{ U_{t}\right\} _{t\in\left[\numclust\right]}$
are $\left\{ B_{t}\right\} _{t\in\left[\numclust\right]}$ WLOG, and
that $U_{t}$ consists of $B_{t}$ and some elements from sets $B_{u}$
with $u>\numclust$. We have 
\begin{align*}
\left|\bigcup_{t=\perm(a)}\clustset a\cap U_{t}\right| & \ge\left|\bigcup_{t=\perm'(a)\le\numclust}\clustset a\cap B_{t}\right|\\
 & =\left|\bigcup_{t=\perm'(a)}\clustset a\cap B_{t}\right|-\left|\bigcup_{t=\perm'(a)>\numclust}\clustset a\cap B_{t}\right|\\
 & \ge\left(1-C'\epsilon\right)\num-\left|\bigcup_{t=\perm'(a)>\numclust}\clustset a\cap B_{t}\right|
\end{align*}
where $C'>0$ is a universal constant. Define 
\begin{align*}
T_{1} & \coloneqq\left\{ t>\numclust:t=\perm'(a)\text{ for some }a\in\left[\numclust\right]\right\} ,\\
T_{2} & \coloneqq\left\{ t\in\left[\numclust\right]:t\ne\perm'(a)\text{ for any }a\in\left[\numclust\right]\right\} .
\end{align*}
Note that $\left|T_{1}\right|=\left|T_{2}\right|$ and for any $t_{1}\in T_{1}$
and $t_{2}\in T_{2}$ we have $\left|B_{t_{1}}\right|\le\left|B_{t_{2}}\right|$.
Therefore, 
\begin{align*}
\left|\bigcup_{t=\perm'(a)>\numclust}\clustset a\cap B_{t}\right| & \le\left|\bigcup_{t\in T_{1}}B_{t}\right|\\
 & \le\left|\bigcup_{t\in T_{2}}B_{t}\right|\\
 & \le\left|\vertexset\right|-\left|\bigcup_{t=\perm'(a)}\clustset a\cap B_{t}\right|\\
 & =C'\epsilon\num.
\end{align*}
The result follows by setting $C\coloneqq2C'$.

\section{Proof of Theorem \ref{thm:mean_estimation_error}\label{sec:proof_mean_estimation_error}}

Let $\Var\left(g_{ij}\right)=\std^{2}$. For $a\in\left[\numclust\right]$,
define $\clustest a\coloneqq\left\{ i\in\left[\num\right]:\labelhat_{i}=a\right\} $
the estimated clusters encoded in $\LabelHat$, and recall that our
cluster center estimators are defined by $\Meanhat_{a}\coloneqq\size^{-1}\sum_{i\in\clustest a}\h_{i}$.
We assume $\left\{ \clustest a\right\} $ achieves the lowest clustering
error as given in Theorem \ref{thm:cluster_error_rate} WLOG. For
each $a\in\left[\numclust\right]$, we have 
\begin{align*}
\norm[\Meanhat_{a}-\Mean_{a}]2 & \le\norm[\frac{1}{\size}\sum_{i\in\clustest a}\h_{i}-\frac{1}{\size}\sum_{j\in\clustset a}\h_{j}]2+\norm[\frac{1}{\size}\sum_{j\in\clustset a}\h_{j}-\Mean_{a}]2\\
 & \eqqcolon Q_{1}+Q_{2}.
\end{align*}


\subsection{Controlling $Q_{1}$}

Define $\epsilon\coloneqq\misrate(\LabelHat,\LabelStar)$. We work
on the event that the result Theorem \ref{thm:cluster_error_rate}
is true. We have 
\[
Q_{1}=\frac{1}{\size}\norm[\sum_{i\in\clustest a\backslash\clustset a}\h_{i}-\sum_{j\in\clustset a\backslash\clustest a}\h_{j}]2
\]
Note that $\left|\clustest a\backslash\clustset a\right|=\left|\clustset a\backslash\clustest a\right|$
so we can pair each point in $\clustest a\backslash\clustset a$ with
a point in $\clustset a\backslash\clustest a$. Let us pair $i$th
point in $\clustest a\backslash\clustset a$ with $j(i)$th point
in $\clustset a\backslash\clustest a$, and define $\calM\coloneqq\left\{ \left(i,j(i)\right)\right\} $.
We have $\left|\calM\right|\le\num\epsilon$ and we can write 
\begin{align*}
Q_{1} & =\frac{1}{\size}\norm[\sum_{(i,j(i))\in\calM}\left(\h_{i}-\h_{j(i)}\right)]2\\
 & \le\frac{1}{\size}\sum_{(i,j(i))\in\calM}\norm[\h_{i}-\h_{j(i)}]2\\
 & \le\frac{1}{\size}\sum_{(i,j(i))\in\calM}\left(\minsep_{\labelstar(i),\labelstar(j(i))}+\norm[\g_{i}-\g_{j(i)}]2\right)\\
 & \le\frac{1}{\size}\sum_{(i,j(i))\in\calM}\left(C_{q}\minsep+\norm[\g_{i}-\g_{j(i)}]2\right),
\end{align*}
where the last step holds for some universal constant $C_{q}>0$ given
that $\max_{a,b\in\left[\numclust\right]}\minsep_{ab}\le C_{q}\minsep$.
By Theorem 3.1.1 on pp.$\ $41 of \citet{vershynin2017high}, $\frac{1}{\sqrt{2}\std}\norm[\g_{i}-\g_{j(i)}]2-\sqrt{\vecdim}$
is a sub-Gaussian random variable with sub-Gaussian norm at most $C_{\psi_{2}}\frac{\sgnorm^{2}}{\std^{2}}$
where $C_{\psi_{2}}>0$ is a universal constant. Then Lemma \ref{lem:hoeffding}
implies that 
\[
\P\left[\frac{1}{\sqrt{2}\std}\norm[\g_{i}-\g_{j(i)}]2-\sqrt{\vecdim}\ge C\frac{\sgnorm^{2}}{\std^{2}}\sqrt{\log\num}\right]\le\num^{-C'}
\]
for some universal constants $C,C'>2$. By the union bound and the
facts that $\left|\calM\right|\le\num$ and $\std\lesssim\sgnorm$,
we have 
\[
\max_{(i,j)\in\calM}\norm[\g_{i}-\g_{j(i)}]2\le C_{g}\left(\sgnorm\sqrt{2\vecdim}+C\sgnorm\sqrt{2\log\num}\right)
\]
with probability at least $1-n^{-C_{1}}$ where $C_{g},C_{1}>0$ are
universal constants. 

Therefore, we have 
\begin{align*}
Q_{1} & \le C_{0}\left(\minsep+\sgnorm\sqrt{\vecdim}+\sgnorm\sqrt{\log\num}\right)\cdot\numclust\exp\left[-\frac{\snr^{2}}{\conste}\right]\\
 & \le C_{0}\left(\minsep+\sgnorm\sqrt{\vecdim}+\sgnorm\sqrt{\log\num}\right)\cdot\exp\left[-\frac{\snr^{2}}{2\conste}\right]
\end{align*}
for some universal constant $C_{0},\conste>0$ with probability at
least $1-n^{-C_{1}}$, where the last step holds since $\snr^{2}\ge\numclust$.
The fact that $e^{x}\ge1+x>x$ for any $x$ implies 
\[
\exp\left[-\frac{\snr^{2}}{4\conste}\right]\le\frac{4\conste}{\snr^{2}}=\frac{\sgnorm}{\minsep}\cdot\frac{4\conste}{\snr}\le4\conste\frac{\sgnorm}{\minsep}
\]
where the last step holds since we have $\snr\ge1$ by the conditions
of Theorem \ref{thm:cluster_error_rate}. Hence, we have 
\begin{align*}
Q_{1} & \le C_{0}\sgnorm\left(4\conste+\sqrt{\vecdim}+\sqrt{\log\num}\right)\cdot\exp\left[-\frac{\snr^{2}}{4\conste}\right]\\
 & \leq C_{1}\sgnorm\left(1+\sqrt{\vecdim}+\sqrt{\log\num}\right)\cdot\exp\left[-\frac{\snr^{2}}{4\conste}\right]\\
 & \leq2C_{1}\sgnorm\left(\sqrt{\vecdim}+\sqrt{\log\num}\right)\cdot\exp\left[-\frac{\snr^{2}}{4\conste}\right]
\end{align*}
where $C_{1}>0$ is a universal constant.

\subsection{Controlling $Q_{2}$}

We have 
\[
Q_{2}=\norm[\frac{1}{\size}\sum_{j\in\clustset a}\g_{j}]2.
\]
We see that $\frac{1}{\size}\sum_{j\in\clustset a}g_{ji}$ has variance
$\frac{1}{\size}\std^{2}$. By Proposition 2.6.1 on pp.$\ $28 and
Theorem 3.1.1 on pp. 41 of \citet{vershynin2017high}, $\frac{\sqrt{\size}}{\std}\norm[\frac{1}{\size}\sum_{j\in\clustset a}\g_{j}]2-\sqrt{\vecdim}$
is a sub-Gaussian random variable with sub-Gaussian norm at most $C_{\psi_{2}}\frac{\sgnorm^{2}}{\std^{2}}$
where $C_{\psi_{2}}>0$ is a universal constant. Then Lemma \ref{lem:hoeffding}
implies that 
\[
\P\left[\frac{\sqrt{\size}}{\std}\norm[\frac{1}{\size}\sum_{j\in\clustset a}\g_{j}]2-\sqrt{\vecdim}\ge C\frac{\sgnorm^{2}}{\std^{2}}\sqrt{\log\num}\right]\le\num^{-C'}
\]
for some universal constants $C,C'>0$. Since $\std\lesssim\sgnorm$,
there exists a universal constant $C_{0}>0$ such that 
\[
Q_{2}\leq C_{0}\sgnorm\left(\sqrt{\frac{\numclust\vecdim}{\num}}+\sqrt{\frac{\numclust\log\num}{\num}}\right)
\]
with probability at least $1-\num^{-C'}$. 

\section{Technical lemmas}

The following lemma is Theorem 2.6.2 on pp.$\ $28 in \citet{vershynin2017high}.
\begin{lem}[General Hoeffding's inequality]
\emph{ \label{lem:hoeffding} }Let $X_{1},\ldots,X_{N}$ be independent,
mean zero, sub-Gaussian random variables. Then, for every $t\geq0$
we have 
\[
\P\left[\left|\sum_{i=1}^{N}X_{i}\right|\geq t\right]\leq2\exp\left[-\frac{ct^{2}}{\sum_{i=1}^{N}\norm[X_{i}]{\psi_{2}}^{2}}\right],
\]
where $c>0$ is a universal constant.
\end{lem}
The following lemma is Exercise 4.7.3 in \citet{vershynin2017high}.
\begin{lem}[Tail bound of covariance matrix of sub-Gaussians]
\emph{ }\label{lem:subg_cov_mat_bound} Let $\x$ be a sub-Gaussian
vector and let $\x_{1},\ldots,\x_{m}$ be independent samples of $\x$.
Let $m$ be a positive integer and define 
\begin{align*}
\boldsymbol{\Sigma} & \coloneqq\E\x\x\t,\\
\boldsymbol{\Sigma}_{m} & \coloneqq\frac{1}{m}\sum_{i=1}^{m}\x_{i}\x_{i}\t.
\end{align*}
Let $\rho_{0}\ge1$ be such that 
\[
\norm[\left\langle \x,\w\right\rangle ]{\psi_{2}}\leq\rho_{0}\sqrt{\E\left\langle \x,\w\right\rangle ^{2}}\qquad\text{for any }\w\in\real^{N}.
\]
For any $u\geq0$, we have for a universal constant $C>0$, 
\[
\opnorm{\boldsymbol{\Sigma}_{m}-\boldsymbol{\Sigma}}\leq C\rho_{0}^{2}\left(\sqrt{\frac{N+u}{m}}+\frac{N+u}{m}\right)\opnorm{\boldsymbol{\Sigma}}
\]
with probability at least $1-2e^{-u}$.
\end{lem}



\end{document}
