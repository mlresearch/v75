
\section{Introduction\label{sec:intro}}

We consider the Sub-Gaussian Mixture Models (SGMMs), where one is
given $\num$ random points drawn from a mixture of $\numclust$ sub-Gaussian
distributions with different means. SGMMs, particularly its special
case Gaussian Mixture Models (GMMs), are widely used in a broad range
of applications including speaker identification, background modeling
and online recommendations systems. In these applications, one is
typically interested in two inference problems under SGMMs:
\begin{itemize}
\item \textbf{Clustering:} (approximately) identify the cluster membership
of each point, that is, which of the $\numclust$ mixture components
generates a given point;
\item \textbf{Center estimation:} estimate the $\numclust$ centers of an
mixture, that is, the means of the $\numclust$ components.
\end{itemize}
Standard approaches to these problems, such as $\numclust$-means
clustering, typically lead to integer programming problems that are
non-convex and NP-hard to optimize~\citep{aloise2009hard,jain2002facility,mahajan2009planar}.
Consequently, much work has been done in developing computationally
tractable algorithms for SGMMs, including expectation maximization
\citep{dempster1977em}, Lloyd's algorithm \citep{lloyd1982least},
spectral methods \citep{vempala2004spectral}, the method of moments
\citep{pearson1936method}, and many more. Among them, convex relaxation,
including those based on linear programming (LP) and semidefinite
programming (SDP), have emerged as an important approach for clustering
SGMMs. This approach has several attractive properties: (a) it is
solvable in polynomial time, and does not require a good initial solution
to be provided; (b) it has the flexibility to incorporate different
quality metrics and additional constraints; (c) it is not restricted
to specific forms of SGMMs (such as Gaussian distributions), and is
robust against model misspecification \citep{peng2005new,peng2007approximating,nellore2015recovery}.

Theoretical performance guarantees for convex relaxation methods have
been studied in a body of classical and recent work. As will be discussed
in the related work section (Section \ref{sec:related}), these existing
results often have one of the two forms:
\begin{enumerate}
\item How well the (rounded) solution of a relaxation optimizes a particular
objective function (e.g., the k-means or k-medians objective) compared
to the original integer program, as captured by an \emph{approximation
factor} \citep{charikar1999constant,kanungo2004local,peng2007approximating,li2016approximating};
\item When the solution of a relaxation corresponds exactly to the ground-truth
clustering, a phenomenon known as \emph{exact recovery}, which is
considered in a more recent line of work \citep{nellore2015recovery,awasthi2015relax,mixon2017clustering,iguchi2017certify,li2017kmeans}.
\end{enumerate}
In many practical scenarios, optimizing a particular objective function,
and designing approximation algorithms for doing so, is often only
a means to the true goal of the problem, namely learning the true
underlying model that generates the observed data. Establishing exact
recovery guarantees is more directly relevant to this goal. However,
such results often require very stringent conditions on the separation
or signal-to-noise ratio (SNR) of the model. In practice, convex relaxation
solutions are rarely exact, even when the data are generated from
the assumed model. On the other hand, it is observed that the solutions,
while not exact or integer-valued, are often a good approximation
to the desired solution that represents the ground truth. Such a phenomenon
is not captured by results on exact recovery.\\

In this paper, we aim to significantly strengthen our understanding
of convex relaxation approaches to SGMMs. In particular, we study
the regime where their solutions are not integral in general, and
seek to directly characterize the \emph{estimation errors} of the
solutions\textemdash namely, their distance to desired integer solution
corresponding to the true underlying model. For a specific class of
SDP relaxations for SGMMs, our results reveal a perhaps surprising
property of them: While the SDP solutions are not integer-valued in
general, their errors can be controlled by that of the solutions of
an idealized integer program (IP), in which one tries to estimate
cluster memberships when an oracle reveals the \emph{true centers}
of the SGMM. In particular, we show that, in a precise sense to be
formalized later, the estimation errors of the SDP and IP satisfy
the following relationship (Theorem~\ref{thm:ip_sdp}):
\[
\text{error(SDP)}\lesssim\text{error(IP)}.
\]
We refer to this property as \emph{hidden integrality} of the SDP
relaxations; its proof in fact involves showing that the optimal solutions
of certain intermediate linear optimization problems are integral.
We then further upper bound the error of the IP and show that it decays
\emph{exponentially }in terms of the SNR (Theorem~\ref{thm:ip_exp_rate}):
\[
\text{error(IP)}\lesssim\exp\left[-\Omega(\text{SNR}^{2})\right],
\]
where the SNR is defined as the ratio of the separation and standard
deviation of the sub-Gaussian components. Combining these two results
immediately leads to explicit bounds on the error of the SDP solution
(Corollary~\ref{cor:SDP_exp_rate}). 

When the SNR is sufficiently large, the above results imply that the
SDP solution is integral and exact up to numerical errors, hence recovering
(sometimes improving) existing results on exact recovery as a special
case. Moreover, when the SNR is lower and the SDP solution is fractional,
an explicit clustering can be obtained from the SDP solution via a
simple, optimization-free rounding procedure. We show that the error
of this explicit clustering (in terms of the fraction of points misclassified)
also decays exponentially in the SNR (Theorem~\ref{thm:cluster_error_rate}).
As a consequence, we obtain sufficient conditions for misclassifying
at most $\delta$ fraction of the points for any given $\delta\in[0,1]$.
Finally, we show that the SDP solutions also lead to an efficient
estimator of the cluster \emph{centers}, for which estimation error
bounds are established (Theorem~\ref{thm:mean_estimation_error}).
Significantly, our results often match and sometimes improve upon
state-of-the-art performance guarantees in settings for which known
results exist, and lead to new guarantees in other less studied settings
of SGMMs. Detailed discussion of these implications of our results
and comparison with existing ones will be provided after we state
our main theorems.

\paragraph{Paper Organization}

The remainder of the paper is organized as follows. In Section~\ref{sec:related},
we discuss related work on SGMMs and its special cases. In Section
\ref{sec:setup}, we describe the problem setup for SGMMs and provide
a summary of our clustering algorithms. In Section \ref{sec:main},
we present our main results, discuss some of their consequences and
compare them with existing results.

\section{Related work\label{sec:related}}

The study of SGMMs has a long history and is still an active area
of research. Here we review the most relevant results with theoretical
guarantees, with a focus on SDP relaxation methods. 

\citet{dasgupta1999learning} is among the first to obtain performance
guarantees for GMMs. Subsequent work has obtained improved guarantees,
achieved by various algorithms including spectral methods. These results
often establish sufficient conditions, in terms of the separation
between the cluster centers (or equivalently the SNR), for achieving
(near)-exact recovery of the cluster memberships. \citet{vempala2004spectral}
obtain one of the best results and require $\text{SNR}\gtrsim\left(\numclust\ln\num\right)^{1/4}$,
which is later improved and extended by \citet{achlioptas2005spectral,kumar2010clustering,awasthi2012improved}.
We compare these results with ours in Section \ref{sec:main}.

Expectation-Maximization (EM) and Lloyd's algorithms are among the
most popular methods for GMMs. Despite their empirical effectiveness,
non-asymptotic statistical guarantees are established only recently.
In particular, convergence and center estimation error bounds for
EM under GMMs with two components are derived in \citet{balakrishnan2017statistical,klusowski2016statistical},
with extension to multiple components given in \citet{yan2017em}.
The work of \citet{lu2016lloyd} provides a general convergence analysis
for Lloyd's algorithm, which implies clustering and center estimation
guarantees for random models including SGMMs. All these results assume
that one has access to a sufficiently good initial solution, typically
obtained by spectral methods. Recent breakthrough has been made by
\citet{daskalakis2016ten,xu2016em}, who establish global convergence
of randomly-initialized EM for GMMs with two symmetric components.
Complementarily, \citet{jin2016local} show that EM may fail to converge
under GMMs with $\numclust\ge3$ components due to the existence of
bad local minima. 

Most relevant to us are work on convex relaxation methods for GMMs
and k-means/median problems, with SDP relaxations first considered
in \citet{peng2005new,peng2007approximating}. Thanks to convexity,
these methods do not suffer from the issues of bad local minima faced
by EM and Lloyd's, though it is far from trivial to round their (typically
fractional) solutions into valid clustering solutions with provable
and quality guarantees. In this direction, \citet{awasthi2015relax,li2017kmeans}
establish conditions for LP/SDP relaxations to achieve exact recovery.
The work of \citet{mixon2017clustering} consider SDP relaxations
as a denoising method, and prove error bounds for a form of approximate
recovery. Robustness of SDP relaxations under a semi-random GMM is
studied in \citet{awasthi2017clustering}. Most of these results are
directly comparable to ours, and we discuss them in more details in
Section \ref{sec:main} after presenting our main theorems.

Clustering problems under Stochastic Block Models (SBMs) have also
witnessed fruitful progress on convex relaxation methods; see \citet{abbe2016recent}
for a survey. Much work has been done on exact recovery guarantees
for SDP relaxations of SBMs \citep{krivelevich2006coloring,oymak2011finding,amini2014semidefinite,ames2014convex,chen2012sparseclustering}.
A more recent line of work establishes approximate recovery guarantees
of the SDPs \citep{Guedon2015,MontanariSen16}. Particularly relevant
to us is the work by \citet{fei2017exponential}, who also establish
exponentially decaying error bounds. Despite the apparent similarity
in the forms of the error bounds, our results require very different
analytical techniques, due to the fundamental difference between the
geometric and probabilistic structures of SBMs and SGMMs; moreover,
our results reveal the more subtle hidden integrality property of
SDP relaxations, which we believe holds more broadly beyond specific
models like SBMs and SGMMs.


