\section{Additional Related Work}
\label{sec:preli}


\subsection{Testing Unions of $d$ Intervals}
\label{subsec:relatedinterval}
We use $\interval(d)\subseteq\{0,1\}^{\mathbb{R}}$ to denote the class of functions $f\in\{0,1\}^{\mathbb{R}}$ satisfying that $f^{-1}(1)$ can be written as a union of at most $d$ intervals. Note that for $d\in\mathbb{N}$, the VC-dimension of $\interval(d)$ is $2d$.

We use $\interval_{\calD}(d,\alpha)$ to denote the class of functions that are $\alpha$-close to $\interval(d)$, i.e. $\interval_{\calD}(d,\alpha)=\{f\in\{0,1\}^{\mathbb{R}}:\exists g\in\interval(d),\dist_{\calD}(f,g)\leq\alpha\}$. Using this notation, property testing for unions of $d$ intervals is to distinguish $f\in\interval(d)$ and $f\notin\interval_{\calD}(d,\epsilon)$. %In the rest of the paper, when $\mathcal{D}$ is the uniform distribution on unit interval $[0,1]$, we omit it for short.

In previous work, \citet{KR98} showed a $(\epsilon,\frac 1\epsilon)$-bi-criteria tester for the class of unions of $d$ intervals in the passive testing model, over the uniform distribution $\calU$ on $[0,1]$. The tester distinguishes $f\in\dintervals$ and $f\notin\mathcal{I}_{\calU}(\frac{d}{\epsilon},\epsilon)$ using $O(\frac{\sqrt d}{\epsilon^{1.5}})$ samples. Their tester also works in the standard query testing framework, using $O(\frac{1}{\epsilon})$ queries. %Their algorithm can be easily generalized to a bi-criteria tolerant testing algorithm that distinguishes $f\in\mathcal{I}(d,\alpha_1)$ and $f\notin\mathcal{I}(\frac{d}{O(\alpha_2-\alpha_2^2-\alpha_1)},\alpha_2)$ using $O(\frac{1}{(\alpha_2-\alpha_2^2-\alpha_1)^2})$ queries. Although their algorithm lies in the query testing framework, the queries they use are randomly chosen and can be generated in the active testing framework using $O(\frac{\sqrt{d}}{\epsilon^{1.5}})$ (or $O(\frac{\sqrt{d}}{(\alpha_2-\alpha_2^2-\alpha_1)^{2.5}})$ with tolerance) unlabeled samples. 
\citet{BBBY12} improved this work by showing that in the active testing framework, there is a uni-criterion testing algorithm that can distinguish $f\in\interval(d)$ and $f\notin\mathcal{I}_\calU(d,\epsilon)$ using $O(\frac{1}{\epsilon^4})$ queries on $O(\frac{\sqrt{d}}{\epsilon^5})$ unlabeled examples. \citet{KNOW13} slightly improved the query complexity to $O(\frac{1}{\epsilon^{3.5}})$ as the one-dimensional special case when studying the more general problem of testing surface area. Though they were considering the query testing framework, the tester can be easily implemented in the active testing model using the same number of label queries. The tester of  \citet{KNOW13} is similar to that of \citet{BBBY12}, using a ``Buffon's Needle''-type algorithm to estimate the ``noise sensitivity'' of the function being tested. \citet{KNOW13} provided a stronger analysis than \citet{BBBY12}, roughly allowing the length of the ``needle'' to be longer by a factor of $\epsilon^{-0.5}$, leading to the improvement of the query complexity. The tester can be generalized to a testing algorithm that distinguishes $f\in\mathcal{I}_\calU(d,\epsilon_1)$ and $f\notin\mathcal{I}_\calU(d,\epsilon)$ when $\epsilon_1=O(\epsilon^{2.5})$ using the same number of queries and unlabeled examples, but can't be directly adapted to the $\epsilon_1=\frac\epsilon 2$ case, or general tolerant testing. Our tolerant tester for the class of unions of intervals uses a completely different technique.

%When the distribution is uniform on $[0,1]$, their algorithm can be viewed as an algorithm in the \emph{query testing} framework implemented in the \emph{active testing} framework by continuous sampling until obtaining the data that the algorithm wants to query. 
As pointed out by \citet{BBBY12}, the tester can be generalized from the uniform distribution on $[0,1]$ to any unknown distribution by taking the advantage of unlabeled examples to approximate the CDF of the distribution to enough accuracy using $O(\frac{d^2}{\epsilon^6})$ unlabeled examples. This unlabeled sample complexity is improved to $O(\frac{d}{\epsilon}\log\frac{1}{\epsilon})$ in our paper, by revealing a general relationship between active testing and query testing (see Section \ref{sec:relationship}).



\subsection {Testing Surface Area}
\label{subsec:surfacearea}
\citet{KNOW13} first studied the problem of property testing for the class $\calC(A)$ of functions $f:(0,1)^n\rightarrow\{0,1\}$ satisfying $f^{-1}(1)$ has surface area at most $A$ over the uniform distribution, which is the high-dimensional version of testing the class of unions of $d$ intervals. They showed a tester that can distinguish $f\in\calC(A)$ and $f$ is $\epsilon$-far from $\calC(\kappa A)$ with any $\kappa>\frac{4}{\pi}\approx 1.27$ using $O(\frac 1\epsilon)$ queries. \citet{Nee14} improved the analysis, showing that an essentially identical tester works for any $\kappa>1$. They both showed similar results for Gaussian surface area. 

The testers of \citet{KNOW13} and \citet{Nee14} (together with the tester of \citet{BBBY12} for unions of intervals) are all based on estimating the ``noise sensitivity'' of the function and accepting (rejecting) if the ``noise sensitivity'' is small (large). They showed that a function in the class has a relatively small ``noise sensitivity'' and a function $\epsilon$-far from the class (with a bi-criteria approximation factor) has a relatively large ``noise sensitivity''. This argument can't be extended to the \emph{tolerant} case for $(\epsilon,\kappa)$-bi-criteria testing for arbitrary $\kappa>1,\epsilon>0$: while it is true that a function $\alpha$-far from the class (with a bi-criteria approximation factor of $\kappa$) has a large ``noise sensitivity'', a function $(\alpha-\epsilon)$-close to the class doesn't necessarily have a smaller ``noise sensitivity'' (the function has to be $O((\kappa-1)^2\alpha)$-close to the class in order to have a smaller ``noise sensitivity'' based on the analysis of \citet{Nee14}).
%Previous works \citep{KR98,BBBY12} on testing union of $d$ intervals reveal close relationship between the query testing framework and the active testing framework, which we will discuss in Section \ref{sec:relation}. 


%there is a tolerant testing algorithm that can distinguish $f\in\interval_{\calD}(d,\alpha)$ and $f\notin\interval_{\calD}(d,\alpha+\epsilon)$ using $O(\frac{1}{\epsilon^6}\log\frac{1}{\epsilon})$ queries on $O(\frac{d}{\epsilon^2}\log\frac{1}{\epsilon})$ unlabeled samples, even when the distribution $\calD$ is unknown to the algorithm.



