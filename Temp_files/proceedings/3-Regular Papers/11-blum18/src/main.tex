\documentclass[final,12pt]{colt2018} % Anonymized submission
% \documentclass{colt2017} % Include author names

% The following packages will be automatically loaded:
% amsmath, amssymb, natbib, graphicx, url, algorithm2e

\title{Active Tolerant Testing}
\usepackage{times}
 % Use \Name{Author Name} to specify the name.
 % If the surname contains spaces, enclose the surname
 % in braces, e.g. \Name{John {Smith Jones}} similarly
 % if the name has a "von" part, e.g \Name{Jane {de Winter}}.
 % If the first letter in the forenames is a diacritic
 % enclose the diacritic in braces, e.g. \Name{{\'E}louise Smith}

 % Two authors with the same address
  % \coltauthor{\Name{Author Name1} \Email{abc@sample.com}\and
  %  \Name{Author Name2} \Email{xyz@sample.com}\\
  %  \addr Address}

 % Three or more authors with the same address:
 % \coltauthor{\Name{Author Name1} \Email{an1@sample.com}\\
 %  \Name{Author Name2} \Email{an2@sample.com}\\
 %  \Name{Author Name3} \Email{an3@sample.com}\\
 %  \addr Address}


 % Authors with different addresses:
 \coltauthor{\Name{Avrim Blum} \Email{avrim@ttic.edu}\\
 \addr {Toyota Technological Institute at Chicago, Chicago, IL, USA}
 \AND
 \Name{Lunjia Hu} \Email{hulj14@mails.tsinghua.edu.cn}\\
 \addr {Institute for Interdisciplinary Information Sciences, Tsinghua University, Beijing, China}
 }


\newcommand{\dintervals}{{\mathcal{I}(d)}}
\newcommand{\dist}{{\mathrm{dist}}}
\newcommand{\interval}{{\mathcal{I}}}
\newcommand{\standardinterval}{{\widetilde{\mathcal{I}}}}
\newcommand{\calC}{{\mathcal{C}}}
\newcommand{\calD}{{\mathcal{D}}}
\newcommand{\calS}{{\mathcal{S}}}
\newcommand{\property}{{\mathcal{P}}}
\newcommand{\D}{{\mathrm{d}}}
\newcommand{\knn}{{\text{$k$-NN}}}
\newcommand{\mv}{{\text{$k$-NN}^{\mathrm{hard}}}}
\newcommand{\thard}{\mathcal{E}^{\mathrm{hard}}}
\newcommand{\tsoft}{\mathcal{E}^{\mathrm{soft}}}
\newcommand{\knnsoft}{{\text{$k$-NN}^{\mathrm{soft}}}}
\newcommand{\loss}{{\mathrm{loss}}}
\newcommand{\calE}{{\mathcal E}}
\newcommand{\udi}{{\mathrm{Int}}}
\newcommand{\pt}{{\mathrm{PT}}}
\newcommand{\tot}{{\mathrm{TT}}}
\newcommand{\da}{{\mathrm{DA}}}
\newcommand{\qu}{_{\mathrm{query}}}
\newcommand{\ac}{_{\mathrm{active}}}
\newcommand{\pa}{_{\mathrm{passive}}}
\newcommand{\ap}{{\mathrm{Comp}}}
\newcommand{\aga}{{\mathrm{AGA}}}
\newcommand{\cga}{{\mathrm{CGA}}}
\newcommand{\calU}{{\mathcal{U}}}
\newcommand{\best}{{\mathrm{Best}}}
\newcommand{\calA}{{\mathcal{A}}}
\newcommand{\error}{{\mathrm{error}}}
\newcommand{\calT}{{\mathcal{T}}}
\newcommand{\err}{{\mathrm{err}_1}}
\newcommand{\calB}{{\mathcal B}}
\newcommand{\poly}{{\mathrm{poly}}}


\newtheorem{claim}[theorem]{Claim}


\begin{document}

\maketitle

\begin{abstract}
In this work, we show that for a nontrivial hypothesis class $\calC$, we can estimate the distance of a target function $f$ to $\calC$ (estimate the error rate of the best $h\in \calC$) using substantially fewer labeled examples than would be needed to actually {\em learn} a good $h \in \calC$.   Specifically, we show that for the class $\calC$ of unions of $d$ intervals on the line, in the active learning setting in which we have access to a pool of unlabeled examples drawn from an arbitrary underlying distribution $\calD$, we can estimate the error rate of the best $h \in \calC$ to an additive error $\epsilon$ with a number of label requests that is {\em independent of $d$} and depends only on $\epsilon$.  In particular, we make $O(\frac{1}{\epsilon^6}\log \frac{1}{\epsilon})$ label queries to an unlabeled pool of size $O(\frac{d}{\epsilon^2}\log \frac{1}{\epsilon})$.  This task of estimating the distance of an unknown $f$ to a given class $\calC$  is called {\em tolerant testing} or {\em distance estimation} in the testing literature, usually studied in a membership query model and with respect to the uniform distribution.  Our work extends that of  \citet{BBBY12} who solved the {\em non}-tolerant testing problem for this class (distinguishing the zero-error case from the case that the best hypothesis in the class has error greater than $\epsilon$).  
%In this work, we give the first algorithms for tolerant testing of nontrivial classes in the active model: estimating the distance of a target function to a hypothesis class $\calC$ with respect to some arbitrary distribution $\calD$, using only a small number of label queries to a polynomial-sized pool of unlabeled examples drawn from $\calD$.   Specifically, we show that for the class $\calC$ of unions of $d$ intervals on the line, we can estimate the error rate of the best hypothesis in the class to an additive error $\epsilon$ from only $O(\frac{1}{\epsilon^6}\log \frac{1}{\epsilon})$ label queries to an unlabeled pool of size $O(\frac{d}{\epsilon^2}\log \frac{1}{\epsilon})$.  The key point here is the number of labels needed is independent of the VC-dimension of the class. This extends the work of  \citet{BBBY12} who solved the {\em non}-tolerant testing problem for this class (distinguishing the zero-error case from the case that the best hypothesis in the class has error greater than $\epsilon$).  

We also consider the related problem of estimating the performance of a given learning algorithm $\calA$ in this setting.  That is, given a large pool of unlabeled examples drawn from distribution $\calD$, can we, from only a few label queries, estimate how well $\calA$ would perform if the entire dataset were labeled and given as training data to $\calA$?   We focus on $k$-Nearest Neighbor style algorithms, and also show how our results can be applied to the problem of hyperparameter tuning (selecting the best value of $k$ for the given learning problem).
\end{abstract}

\begin{keywords}
property testing, agnostic learning, algorithm estimation
\end{keywords}

%\section{Introduction}
%
%This is where the content of your paper goes.  Remember to:
%\begin{itemize}
%\item Limit the main text (without references and appendices) to 12 PMLR-formatted pages (i.e., using this template).
%\item Include, either in the main text or the appendices, all details, proofs
%  and derivations required to substantiate the results.
%\item Include {\em in the main text} enough details, including proof
%  details, to convince the reviewers of the contribution, novelty and significance of the submissions.
%\item Not include author names (this is done automatically), and to
%  the extent possible, avoid directly identifying the authors.  You
%  should still include all relevant references, including your own,
%  and any other relevant discussion, even if this might allow a
%  reviewer to infer the author identities.
%  \end{itemize}

% !TEX root = onlinevarinancebandits.tex

\section{Introduction}
% Structure:
% \begin{itemize}
% \item The approach of (Regularized) ERM and its importance in Machine Learning.
% Solving such problems with sequential optimization algorithms such as SGD/SVRG/Online K-means.
% Maybe focus on SGD as a running example.
% \item  Mentioned the alternative option is uniform sampling. Describe/illustrate how importance sampling can be used to improve the performance. Give references. 
% \item Describe how the variance of the estimates is a natural measure of performance in this setting. Mention that low variance translates to better performance, e.g. for SGD. 
% \textbf{Online Problem:}
% Similarly to  Duchi/EPFL  we formulate importance sampling an online convex optimization problem.
% Describe the approaches of Duchi/EPFL say very nice things about them give them credit and discuss the limitations of their results/approaches.
% \item State our result. State our contributions+ discuss the improvements over previous work: \\
% (i) tighter regret guarantees with respect to the simplex.\\
% (ii) Showing that regret minimization makes sense in this setting.\\
% (iii) (Hopefully) complementary lower bounds. \\
% (iii) Efficient experimental implementation showing the benefits of the proposed method
% \kl{This is for COLT, for ArXiv put the experiments in (ii) place}

% Discuss the technical challenges of our work, specifically the fact that the costs are unbounded + the bandit feedback.
% Discuss the new regularization that we introduce, its benefits (closed form formula for the FTRL) + the challenge.
% Mention other settings with unbounded losses, e.g. log loss in portfolio selection.
 
%  \textbf{Related work}
%  Who should we cite? Look at Jaggi/Duchi/EPFL for references.

% \kl{Mention (where?) that we can use our approach for coordinate descent.}

% \end{itemize}
%Among the most important paradigms in machine learning is Empirical Risk Minimization (ERM) , which is often the strategy of choice due to its generality and statistical efficiency.
Empirical risk minimization (ERM) is among the most important paradigms in machine learning, and  is often the strategy of choice due to its generality and statistical efficiency.
In ERM, we draw a set of  samples $\D=\{x_1,\ldots,x_n\}\subset \X$ from the underlying data distribution, and we aim to find a solution $w\in\W$ that minimizes the empirical risk,  %The empirical risk serves as a proxy to the expected loss which is often. 
%the objective is to  find a solution $w\in\W$ that minimizes the empirical risk based on a collection of $n$ samples $\D=\{x_1,\ldots,x_n\}\subset \X $:
\begin{equation} \label{eq:ERM}
  \min_{w\in\W }L(w) := \frac{1}{n}  \sum_{i=1}^n \ell (x_i, w),
\end{equation}
where $\ell: \mathcal{X} \times \W \rightarrow \reals$ is a given loss function, and $\W\subseteq \reals^d$ is usually a compact domain.

In this work we are interested in sequential procedures for minimizing the ERM objective, and relate to such methods as \emph{ERM solvers}.
More concretely, we focus on the regime where the number of samples $n$ is very large,  and it is therefore desirable to employ ERM solvers that only require  few passes over the dataset. There exists a rich arsenal of such efficient solvers which have been investigated throughout the years, with the canonical example from this category being  Stochastic Gradient Descent (SGD).


% among are SVRG \citep{johnson2013accelerating} and SAGA \citep{defazio2014saga},
%
%
% such efficient sequential solvers have been developed throughout the years, with the canonical example from this category being  Stochastic Gradient Descent (SGD).

Typically, such methods  require an unbiased estimate of the loss function at each round, which is usually  generated   by sampling a few points uniformly at random from the dataset.
However, by employing uniform sampling, these methods are insensitive to the intrinsic structure of the data. In case of SGD, for example, some data points might produce large gradients, but they are nevertheless assigned the same probability of being sampled as any other point. This ignorance often results in high-variance estimates, which is likely to degrade the performance.

The above issue can be mended by employing non-uniform importance sampling.
And indeed, we have recently witnessed several  techniques to do so: %techniques.
%In recent years several approaches have been developed in order to address this issue.
\citet{zhao2015stochastic} and similarly \citet{needell2014stochastic}, suggest using prior knowledge on the gradients of each data point in order to devise predefined importance sampling distributions.  \citet{NIPS2017_7025} devise adaptive sampling techniques guided by a robust optimization approach. These are only a few examples of a larger body of work 
 \citep{bouchard2015online, alain2015variance, csiba2016importance}.

Interestingly, the recent works of \cite{pmlr-v70-namkoong17a} and \cite{salehi2017} formulate the task of devising importance sampling distributions as an online learning problem with bandit feedback. In this context, they  think of the algorithm, which adaptively chooses the distribution, as a player that competes against the ERM solver. The goal of the player is to minimize the cumulative variance of the resulting (gradient) estimates.  Curiously, both methods rely on some form of the ``linearization trick''\footnote{ By ``linearization trick'' we mean that these methods update according to a first order approximation  of the costs rather than the costs themselves.} 
%\footnote{If $g_t$ is a subgradient of the convex function $f:S\rightarrow \mathbb{R}$ at $w_t$, then $f(w_t) - f(u) \leq (w_t-u)^\intercal g_t$,  $\forall u \in S$.} 
 to resort to the analysis of the EXP3  \citep{auer2002nonstochastic}.

On the other hand, the theoretical guarantees of the above methods are somewhat limited. Strictly speaking, none of them provides regret guarantees with respect to the best fixed distribution in hindsight:  \citet{pmlr-v70-namkoong17a} only compete with the best distribution among a \emph{subset} of the simplex (around the uniform distribution).  Conversely, \cite{salehi2017} compete against a solution which might perform worse than the best in hindsight up to a multiplicative factor of $3$.

In this work, we adopt the above mentioned online learning formulation, and design novel importance sampling techniques. 
Our adaptive sampling procedure is simple and efficient, and 
in contrast to previous work, we are able to provide regret guarantees with respect to the best fixed point among the simplex.
As our contribution, we
\vspace{-1.5mm}
\begin{itemize}
\setlength\itemsep{0.05em}
\item motivate theoretically why regret minimization is meaningful in this setting, 
\item propose a novel bandit algorithm for variance reduction ensuring regret  of~$\tO(n^{1/3}T^{2/3})$,
\item empirically validate our method, and provide an efficient implementation\footnote{The source code is available at  \url{https://github.com/zalanborsos/online-variance-reduction}}.
\end{itemize}
On the technical side, we do not rely on a ``linearization trick'' but rather directly employ a scheme based on the classical 
 Follow-the-Regularized-Leader approach. 
Our analysis entails several technical challenges, most notably handling  unbounded cost functions while only receiving partial (bandit) feedback. Our design and analysis draws inspiration from the seminal works of  \citet{auer2002nonstochastic}  and 
\cite{Abernethy08}. 
Although we present our method for choosing \emph{data points}, it naturally applies to choosing \emph{coordinates in coordinate descent} or even \emph{blocks} of thereof \citep{allen2016even,perekrestenko2017faster, nesterov2012efficiency, necoara2011random}.
More broadly, the proposed algorithm can be incorporated in \emph{any sequential algorithm} that relies on an unbiased estimation of the loss. A prominent  application of our method is variance reduction for SGD, which can be achieved by considering  gradient norms as  losses, i.e., replacing $\ell(w,x_i) \leftrightarrow \|\nabla \ell(w,x_i)\|$. With this modification, our method is minimizing the cumulative variance of the gradients throughout the optimization process.

%defining the bandit feedback as the norm of the gradient estimate. 
The paper is organized as follows. In Section \ref{sec:Motivation}, we formalize the online learning setup of variance reduction, and motivate why regret is a suitable performance measure. As the first step of our analysis, we investigate the full information setting in Section \ref{sec:full-info}, which serves as a mean for studying the bandit setting in Section \ref{sec:bandit}. Finally, we validate our method empirically, and provide the detailed discussion of the results in Appendix  \ref{sec:experiments}. 



%We achieve this by relying on classical results from Follow-the-Regularized-Leader framework, where the regularizer is chosen to suit the setting of variance reduction with importance sampling.

%\newpage
%
%
% rely on solving a a ro
%
%There exists
%
%In 
%Addressing the regime where the number of samples $n$ is very large, efficient \emph{sequential} procedures have been developed, that perform only a few passes over the dataset. These methods usually require an unbiased estimate of the loss function at each round, and they generate the estimate by sampling a few points uniformly at random from the dataset. The canonical example from this category is Stochastic Gradient Descent (SGD).
%
%However, by employing uniform sampling, these methods are agnostic to the intrinsic structure of the data. In case of SGD, for example, some data points might produce large gradients, but they are nevertheless assigned the same probability of being sampled as any other point. This ignorance often results in high-variance estimates.
%
%References:
%\begin{itemize}
%\item \textbf{Variance reduction with uniform sampling}: SVRG \citep{johnson2013accelerating} and SAGA \citep{defazio2014saga}, \cite{xiao2014proximal}
%\item \textbf{Variance reduction with importance sampling (points): } \citep{needell2014stochastic, zhao2015stochastic, bouchard2015online, csiba2016importance,alain2015variance, NIPS2017_7025}
%\item \textbf{Importance sampling, but without direct interpretation as variance reduction (points): } \cite{strohmer2009randomized}, 
%\item \textbf{Coordinate descent, with non-uniform sampling:} \cite{allen2016even,perekrestenko2017faster, nesterov2012efficiency, necoara2011random}
%\item \textbf{Bandits, both points and coordinates:} \cite{pmlr-v70-namkoong17a}, \cite{salehi2017}, \cite{salehi2017stochastic}
%\end{itemize}
%
%
%This issue is addressed by several variance reduction techniques, some prominent examples being  SVRG \citep{johnson2013accelerating} and SAGA \citep{defazio2014saga}. In case of (strongly) convex loss functions, the reduced variance directly translates to improved convergence bounds. \textcolor{red}{An important class of methods that allow variance reduction interpretation}  rely on the technique of importance sampling \citep{needell2014stochastic, zhao2015stochastic, bouchard2015online, csiba2016importance,alain2015variance%, allen2016even, perekrestenko2017faster%
%, NIPS2017_7025}, where the sampling distribution is either fixed or adaptive over the iterations. Since competing against \textcolor{red}{the optimal per-round} sampling distributions is usually computationally infeasible, the methods are often compared to the optimal sampling distribution \emph{in hindsight}.
%
%An interesting idea of the recent works of \cite{pmlr-v70-namkoong17a} and \cite{salehi2017} is to formulate the task of finding a competitive sampling distribution under importance sampling for variance reduction as a \emph{bandit problem}. In this setting, no-regret assures that the chosen distribution performs close to the optimal stationary distribution in hindsight. Both methods rely on some form of the ``linearization trick'' \citep{shalev2012online} to resort to the analysis of the EXP3  \citep{auer2002nonstochastic} and obtain similar algorithms. These methods show convincing convergence guarantees for stochastic optimization for convex ERM both theoretically and empirically.
%
%On the other hand, the two methods come with limitations. While the latter is guaranteed to approximate the variance under the optimal sampling distribution in hindsight within a factor of 3, the former incorporates a KL-projection step in order to ensure that the sampling probabilities are larger than some threshold $\pmin$ --- an additional hyperparameter that affects the convergence bounds.
%
%In this work, we pursue the same idea of employing bandit optimization for variance reduction and we design an algorithm that suffers from none of the limitations mentioned above. We achieve this by relying on classical results from Follow-the-Regularized-Leader framework, where the regularizer is chosen to suit the setting of variance reduction with importance sampling. As our contribution, we
%\vspace{-1.5mm}
%\begin{itemize}
%\setlength\itemsep{0.05em}
%\item motivate theoretically why regret minimization is meaningful in this setting,
%\item propose and analyze a novel bandit algorithm for variance reduction,
%\item empirically validate our method and provide an efficient implementation\footnote{The source code is available at  \url{https://github.com/zalanborsos/online-variance-reduction}}.
%\kl{Dont forget to remove this footnote in the anonimized COLT submission!}
%\end{itemize}
%The analysis entails several technical challenges, most notably handling  unbounded cost functions and unbounded regularizers. Although we present our method for choosing \emph{datapoints} in an optimization problem, it naturally applies to choosing \emph{coordinates in coordinate descent} or even \emph{blocks} of thereof. More broadly, the proposed algorithm can be incorporated in \emph{any sequential algorithm} that relies on unbiased estimation of the loss.
%\kl{add citation about coordinate descent, see Cevher and Jaggi}
\section{Property Testing Background and Models}
\label{sec:model}
\subsection{Query Testing (Standard Property Testing)}
Given functions $f$ and $g$ over domain $X$, we define the distance between $f$ and $g$ with respect to distribution $\calD$ over $X$ to be
\begin{equation}
\dist_\calD(f,g)=\Pr_{x\sim\calD}[f(x)\neq g(x)].
\end{equation}
Given a class $\calC$ of functions over domain $X$ and a margin $\epsilon$, a \emph{property tester} distinguishes the case that the input function $f$ is in the class $\calC$ from the case that $f$ is $\epsilon$-far from $\calC$:
\begin{enumerate}
\item if $f\in\calC$, the tester accepts $f$ with probability at least $\frac{2}{3}$;
\item if $\forall g\in\calC,\dist_\calD(f,g)>\epsilon$, the tester rejects $f$ with probability at least $\frac{2}{3}$.
\end{enumerate}
\citet{RS96} first studied the property testing model assuming $X$ is finite and $\calD$ is uniform. We call the testing model of \citet{RS96} as \emph{query testing}, because the tester makes queries to access $f$, i.e., the tester asks for the value of $f(x)$ for some $x\in X$ for each query it makes.

\citet{PRR06} first studied the \emph{tolerant} version of property testing: given an additional parameter, the threshold $\alpha$, to distinguish a function $\alpha$-close to the class from a function $(\alpha+\epsilon)$-far from the class. In other words, 
\begin{enumerate}
\item if $\exists g\in \calC,\dist_\calD(f,g)\leq\alpha$, the tester accepts $f$ with probability at least $\frac{2}{3}$;
\item if $\forall g\in\calC,\dist_\calD(f,g)>\alpha+\epsilon$, the tester rejects $f$ with probability at least $\frac{2}{3}$.
\end{enumerate}
They showed tolerant testers for clustering and for monotonicity in the query testing model. \citet{FF05} showed the existence of classes of binary functions that are efficiently query-testable in the non-tolerant case but are not efficiently query-testable in the tolerant case.

\citet{PRR06} also considered a similar task called distance approximation: estimating the distance from the function to the class so that with probability at least $\frac 23$ the output is within $\pm\epsilon$ to the true distance. Note that distance approximation with additive error $\epsilon$ implies tolerant testing with margin $2\epsilon$ with the same query complexity. Based on this observation, all the tolerant testers we design in this paper actually perform distance approximation (so we don't need the parameter $\alpha$) because distance approximation is a slightly more convenient model for our presentation.
\subsection{Passive Testing (Sample-Based Testing)}
\label{subsec:passive}
\citet{GGR98} first studied testers with the ability to obtain a random sample in addition to making queries so that the tester can potentially work on arbitrary distributions (see Section \ref{subsec:distributionfree} for distribution-free testing), although their algorithmic results remained in the query testing framework over the uniform distribution.
\citet{KR98} developed the first \emph{passive} testers, testers that don't make queries and only rely on the random i.i.d.\ sample to access the input function $f$, for a variety of classes with sub-learning sample complexity. \citet{GR13} advanced the study of passive testers by providing several general positive results as well as by revealing relations with other testing models.

\emph{Proper} learning implies testing, simply by testing using the output hypothesis, but passive testing can be substantially harder than \emph{improper} learning. \citet{GGR98} pointed out that the class of $k$-term-DNF is NP-hard for non-tolerant passive testing while it is efficiently PAC learnable via $k$-CNF \citep{PV88}, if we require testing and learning on an arbitrary distribution.

The general hardness of \emph{tolerant} passive testing based on hardness of improper \emph{agnostic} learning can be implied from the recent work by \citet{KL18}. They considered the task of refutation: for any fixed distribution $\calD$ over domain $X$, given a sample of example-label pairs $\{(x_i,y_i)\}$ and margin $\epsilon>0$, to distinguish the following two cases:
\begin{enumerate}
\item accept when every $(x_i,y_i)$ is i.i.d.\ from some distribution $\calD'$ over $X\times\{0,1\}$ with marginal on $X$ being $\calD$ and $\exists f\in\calC,\Pr_{(x,y)\sim\calD'}[f(x)\neq y]\leq\frac 12-\epsilon$;
\item reject when every $x_i$ is i.i.d.\ from $\calD$ and every $y_i$ is i.i.d.\ from the uniform distribution over $\{0,1\}$.
\end{enumerate}
They showed that a refutation algorithm for distribution $\calD$ with margin $\epsilon$ and sample complexity $s$ implies an improper agnostic learning algorithm for the same distribution with error $3\epsilon$ and sample complexity $O(\frac{s^3}{\epsilon^2})$. We show in Appendix \ref{sec:refutation} that the refutation algorithm can be reduced to a tolerant passive tester for arbitrary unknown distributions with threshold $\alpha=\frac 12-\frac{3\epsilon}4$, margin $\frac{\epsilon}{2}$, and sample complexity $\Omega(s)$, implying that tolerant passive testing for arbitrary unknown distributions can't be substantially more sample-efficient than improper agnostic learning for any distribution $\calD$ (with some reasonable assumptions about the distribution $\calD$). %The reduction also has a uniform-distribution version (Lemma \ref{}), implying the hardness of tolerant passive testing over the uniform distribution.

%\citet{Vad17} showed that if one can perform refutation, i.e., distinguishing a function in the class from a random function, using a sample of size $s$, then one can also do improper PAC learning with sample complexity $O(\poly(s))$.  \citet{KL18} showed similar results for the tolerant (agnostic) case. They used a different definition of refutation: distinguishing a function having distance $\frac 12-\epsilon$ to the class from a random function, and showed that if one can perform refutation using a sample of size $s(\epsilon)$, then one can also do improper agnostic learning with sample complexity $O(\frac{\left(s(\frac\epsilon 2)\right)^3}{\epsilon^2})$.\footnote{One difference between the two results is that \citet{Vad17} assumes that the distribution is arbitrary while the result by \citet{KL18} is distribution-specific. } Note that passive testing implies refutation with the same sample complexity both in the non-tolerant and the tolerant case, assuming the concept class has a finite VC-dimension and the distribution has no massive points so that a random function has distance $\frac 12$ to it with probability 1. Therefore these results imply that passive testing can't be substantially more sample-efficient than improper learning.

\subsection{Active Testing}
Both query testing and passive testing have shortcomings. The assumption of query testing that the tester can make queries to arbitrary points in the domain is usually impractical, while passive testing is too restrictive: for the tolerant case, passive testing can't be substantially more sample-efficient than agnostic learning (recall Section \ref{subsec:passive}).

To avoid both shortcomings, \citet{BBBY12} proposed the active testing model where the tester first receives an unlabeled random i.i.d.\ sample and then makes queries to points in the sample. While the size of the unlabeled sample might be comparable to the labeled sample complexity for learning, the number of queries the tester makes should be substantially smaller.
%large (but is still required to be polynomial in the VC-dimension of the class), the number of queries the tester makes should be substantially smaller than the sample complexity for learning. 
They showed (non-tolerant) active testers for unions of $d$ intervals and for linear separators.
\subsection{Distribution-Free Testing}
\label{subsec:distributionfree}
Distribution-free testing \citep{GGR98} considers testers that work on arbitrary unknown distributions with the ability to obtain random i.i.d.\ sample in addition to making queries. \citet{HK03} designed distribution-free testers for low-degree multivariate polynomials, monotone functions, and several other classes.

The difference between distribution-free testing and passive testing (over arbitrary unknown distributions) is that distribution-free testers have the ability to make queries while passive testers don't. However, the query ability is helpful only when we do \emph{non-tolerant} testing where the tester is only required to accept functions in the class, rather than functions having distance 0 to the class with respect to the unknown distribution. For \emph{tolerantly} testing binary functions, we show that distribution-free testing implies passive testing with the same sample complexity (see Section \ref{sec:relationship} Lemmas \ref{lm:distributionfreenontolerant} and \ref{lm:distributionfreetolerant}) and thus the hardness for tolerant passive testing extends automatically to tolerant distribution-free testing.



%\subsection{Property Testing, Tolerant Testing and Distance Approximation}
%Suppose we have a ground set $X$ and a distribution $\calD$ over $X$. For any two binary functions $f,g\in\{0,1\}^X$, we define their distance to be $\dist_{\calD}(f,g)=\Pr_{x\sim \calD}[f(x)\neq g(x)]$. 

%Suppose we also have a concept class $\calC\subseteq\{0,1\}^X$. Given a function $f\in\{0,1\}^X$ and a margin $\epsilon$ as input, the task of property testing $\pt_{\calD}(f,\epsilon)$ \citep{RS96} is to distinguish the case that $f$ belongs to class $\calC$ from the case that $f$ is $\epsilon$-far from $\calC$. In other words, $\forall f$,
%\begin{enumerate}
%\item if $f\in\calC$, the algorithm outputs ``YES'' with probability at least $\frac{2}{3}$;
%\item if $\forall g\in\calC,\dist_{\calD}(f,g)>\epsilon$, the algorithm outputs ``NO'' with probability at least $\frac{2}{3}$.
%\end{enumerate}
%A property testing algorithm may be randomized, and in this case, the goal is to output the correct answer with probability at least $\frac{2}{3}$. The success probability can be boosted to $1-\delta$ by repeating the algorithm for $O(\log\frac{1}{\delta})$ times and taking the majority.

%The function $f$ can be given to the algorithm in many different ways. In the \emph{query testing} framework \citep{RS96}, the algorithm can query the value of $f(x)$ for any $x\in X$. In this framework, we say the algorithm has \emph{query access} to $f$, or has access to $f\qu$. The query complexity of the algorithm, as a function of $\frac{1}{\epsilon}$, is measured by the maximum number of queries needed by the algorithm.

%\citet{BBBY12} argued that the query testing framework is not realistic for machine learning practice. They proposed the \emph{active testing} framework, in which the algorithm first requests $N$ unlabeled examples $x_1,x_2,\cdots,x_{N}\in X$ sampled independently according to $\calD$ and can only choose to query $f(x_i)$ for $1\leq i\leq N$. In this framework, we say the algorithm has \emph{active access} to $f$, or has access to $f\ac$. The maximum value of $N$, as a function of $\frac{1}{\epsilon}$, is called the \emph{unlabeled sample complexity}. The query complexity is still defined as a function of $\frac{1}{\epsilon}$ measuring the maximum number of queries needed by the algorithm.

%\citet{GGR98} and \citet{KR98} studied an even more strict way of accessing $f$, called \emph{passive access}, in which the algorithm is given the label of an example chosen independently at random from $\calD$ for each query the algorithm makes. 

%Tolerant testing $\tot_{\calD}(f,\alpha,\epsilon)$ \citep{PRR06} is a similar task to property testing. The only difference is that besides the margin $\epsilon$, we are given another parameter $\alpha$ as input, and we are asked to distinguish the case that $f$ is $\alpha$-close to $\calC$ from the case that $f$ is $(\alpha+\epsilon)$-far from $\calC$. %following two cases:
%\begin{enumerate}
%\item $\exists g\in\calC,\dist_{\calD}(f,g)\leq \alpha$;
%\item $\forall g\in\calC,\dist_{\calD}(f,g)>\alpha+\epsilon$.
%\end{enumerate} 
%The query complexity of tolerant testing is still measured as a function of $\frac{1}{\epsilon}$ \citep{PRR06}. 

%A natural generalization of tolerant testing is distance approximation, in which we are only given the function $f$ and the margin $\epsilon$ as input and required to output $\hat\alpha$ as an approximation of the distance from $f$ to $\calC$ up to an additive error $\epsilon$. More specifically, the goal of $\da_{\calD}(f,\epsilon)$ is to output $\hat\alpha$ such that $\forall f$,
%\begin{enumerate}
%\item $\forall \alpha$ such that $\exists g\in\calC,\dist_{\calD}(f,g)\leq \alpha$, it holds with probability at least $\frac{2}{3}$ that $\hat\alpha\leq \alpha+\epsilon$;
%\item $\forall \alpha$ such that $\forall g\in\calC,\dist_{\calD}(f,g)>\alpha$, it holds with probability at least $\frac{2}{3}$ that $\hat\alpha> \alpha-\epsilon$.
%\end{enumerate}

%The success probability of a distance approximation algorithm can be boosted to $1-\delta$ by repeating it $O(\log\frac{1}{\delta})$ times and taking the median. 

%Because for any $\calD$ and $\epsilon$, it's clear that a $\da_{\calD}(f,\frac{\epsilon}{2})$ algorithm implies a $\tot_{\calD}(f,\alpha,\epsilon)$ algorithm with the same query and unlabeled sample complexity \citep{PRR06}, we focus on distance approximation rather than tolerant testing throughout the paper.

%For all of the above examples, the distribution $\calD$ appears in the subscript meaning that the distribution is fixed and known to the algorithm. In this paper, we will consider a more general case, for example $\pt(f\qu,\epsilon,\calD)$, in which the distribution $\calD$ is arbitrarily given to the algorithm as input. When the algorithm has active or passive access to $f$, it's possible to consider an even more general case, for example $\pt(f\ac,\epsilon)$, in which the distribution $\calD$ is unknown to the algorithm (but implicitly given to the algorithm when accessing $f$).




%We use $q^{\text{query}}(\epsilon)$ and $q^{\text{active}}(\epsilon)$ to denote the maximum number of queries needed in the query testing model and the active testing model, respectively. \citet{BBBY12} showed that $q^{\text{query}}(\epsilon)\leq q^{\text{active}}(\epsilon)$ always holds for non-tolerant property testing, which can be easily generalized to tolerant testing and distance approximation. They also showed for non-tolerant property testing that when $\calC$ is the class of dictator functions and $\calD$ is the uniform distribution over $\{0,1\}^p$, there is a testing algorithm in the query testing framework whose query complexity $q^{\text{query}}(p,\epsilon)$ does not grow with respect to $p$, while every testing algorithm in the active testing framework requires $q^{\text{active}}(p,\epsilon)=\Omega(\log p)$ queries for any fixed $0<\epsilon<\frac{1}{2}$.

%In this paper, however, we will show in Theorem \ref{thm:unlabeled} a reversed inequality that $q^{\text{active}}(\epsilon)=O(q^{\text{query}}(\frac{\epsilon}{2}))$ when we are required to do testing on \emph{arbitrary} distribution $\calD$ and the queries are required to lie in the support of $\calD$ (when we have query access)\footnote{We can assume without loss of generality that the queries all lie in the support of $\calD$ when we do tolerant testing and distance approximation, but we cannot assume this for non-tolerant property testing. The reason is that $f\in\calC$ is stronger than $\exists g\in\calC,\dist_{\calD}(f,g)=0$.}, uncovering the relationship between query access and active access.


%Similar to Theorem \ref{thm:unlabeled}, we have the following theorem in tolerant testing. Its proof is a simple modification of the proof to Theorem \ref{thm:unlabeled}.





%!TEX root = main.tex
\vspace{-1.6pt}
\section{Results} \label{sec:results}
\vspace{-1.22pt}
We consider the optimization of a function $f$ over a compact, connected subset $\X \subset \M$,
   \vspace*{-4pt} \[
\min_{x \in \X \subset \M} f(x),\vspace*{-6pt}
\] with access to a (noisy) first-order oracle $\{ \nabla f_n(x) \}_{n \geq 1}$. Given a sequence of iterates $\{x_n\}_{n\geq0}$  in~$\mathcal{M}$ produced from the first-order optimization of $f$,
\begin{align}
 x_{n} = R_{x_{n-1}} \left(-\gamma_n \nabla f_{n} \left(x_{n-1} \right)\right), \label{eq:grad_desc}\vspace*{-6pt}
\end{align}
that are converging to a \emph{strict} local minimum of $f$, denoted by $x_\star$, we consider (and analyze the convergence of) a streaming average of iterates:
\begin{align}
 \tilde{x}_{n} = R_{\tilde{x}_{n-1}} \left(\frac{1}{n} R_{\tilde{x}_{n-1}}^{-1}\left(x_{n}\right)\right). \label{eq:ave_grad_desc}\vspace*{-6pt}
\end{align}
Here we use $R_x$ to denote a retraction mapping (defined formally in \mysec{background}), which provides a natural means of moving along a vector (such as the gradient) while restricting movement to the manifold.
As an example, when $\M = \mathbb{R}^d$  we can take $R_x$ as vector addition by $x$. In this setting, \eq{grad_desc} reduces to the standard gradient update $x_n = x_{n-1} - \gamma_n \nabla f_n(x_{n-1})$ and \eq{ave_grad_desc} reduces to the ordinary average $\tilde{x}_n = \tilde{x}_{n-1} + \frac{1}{n}(x_{n} - \tilde{x}_{n-1})$.
In the update in \eq{grad_desc}, we will always consider step-size sequences of the form $\gamma_n \!\!=\! \frac{C}{n^\alpha}$ for $C\!>\!0$ and $\alpha \in \left(\frac{1}{2}, 1\right)$, which satisfy the usual stochastic approximation step-size rules $\sum_{i=1}^\infty \!\gamma_i\!=\!\infty$ and $\sum_{i=1}^\infty \!\gamma_i^2\!<\!\!\infty$ \citep[see, e.g.,][]{BenPriMet90}.

Intuitively, our main result
states that if the iterates $x_n$ converge to $x_\star$
at a slow $O(\gamma_n)$ rate, their streaming Riemannian average will converge to $x_\star$ at the the optimal $O(\frac{1}{n})$ rate. This result requires  several technical assumptions, which are standard generalizations of those appearing in the Riemannian optimization and stochastic approximation literatures (detailed in \mysec{assumptions}). The critical assumption we make is that all iterates remain bounded in $\X$---where the manifold behaves well and the algorithm is well-defined (Assumption~\ref{assump:manifold}).
The notion of slow convergence to an optimum is formalized in the following assumption:
\vspace*{-6pt}
\begin{assumption}\label{assump:slowrate}
  If $\Delta_n = R_{x_\star}^{-1}(x_n)$ for a sequence of iterates evolving in \eq{grad_desc}, then
  \[ \E[\Vert \Delta_n \Vert^2] = O(\gamma_n). \]
\end{assumption}
\vspace*{-4pt}
Assumption~\ref{assump:slowrate} can be verified in a variety of optimization problems, and
we provide such examples in \mysec{application}. As $x_\star$ is unknown, $\Delta_n$ is not computable but is primarily a tool for our analysis. Importantly, $\Delta_n$ is a tangent vector in $T_{x_\star} \M$. Note also that the norm $\Vert \Delta_n \Vert$  is locally equivalent to the geodesic distance $d(x_n,x_\star)$ on $\M$ (see \mysec{background}). 


We use $\Sigma$ to denote the covariance of the noisy gradients at the optima $x_\star$.
Formally, our main convergence result regarding Polyak-Ruppert averaging in the manifold setting is as follows (where Assumptions~\ref{assump:manifold} through \ref{assump:noiseLip} will be presented later):
\begin{theorem} \label{thm:main}
  Let Assumptions \ref{assump:slowrate}, \ref{assump:manifold}, \ref{assump:strongconvpoint},
  \ref{assump:HessianLip}, \ref{assump:noiseunbiased},
  and  \ref{assump:noiseLip}
  hold for the iterates evolving according to \eq{grad_desc} and \eq{ave_grad_desc}.
    Then $\tilde{\Delta}_n = R_{x_{\star}}^{-1}(\tilde{x}_n)$ satisfies:
  \begin{align}
   \sqrt{n} \tilde{\Delta}_n \overset{D}{\to} \mathcal{N}(0, \Hess f(x_\star)^{-1} \Sigma \Hess f(x_\star)^{-1}). \notag
  \end{align}
  If we additionally assume a bound on the fourth moment of the iterates---of the form $\mathbb{E}[\Vert \Delta_n \Vert^4] = O(\gamma_n^2)$---then a
  non-asymptotic result holds:
  \begin{align}
    \mathbb{E}[\Vert \tilde{\Delta}_n \Vert^2] \leq \frac{1}{n} \tr[\Hess f(x_\star)^{-1} \Sigma \Hess f(x_\star)^{-1}] + O(n^{-2\alpha}) + O(n^{\alpha-2}). \notag
  \end{align}
  \end{theorem}
We make several remarks regarding this theorem:
\begin{itemize}
\vspace*{-6pt}
  \item The asymptotic result in Theorem \ref{thm:main} is a generalization of the classical asymptotic result of \citet{polyak1992acceleration}. In particular, the leading term has variance $O(\frac{1}{n})$ \textit{independently} of the step-size choice $\gamma_n$. In the presence of strong convexity, SGD can achieve the $O(\frac{1}{n})$ rate with a carefully chosen step size, $\gamma_n = \frac{C}{\mu n}$ (for $C=1$). However, the result is fragile: too small a value of $C$ can lead to an arbitrarily slow convergence rate, while too large a $C$ can lead to an ``exploding,'' non-convergent sequence \citep{NemJudLan08}. In practice determining $\mu$ is often as difficult as the problem itself.
 \vspace*{-6pt}
 \item Theorem \ref{thm:main} implies that the distance  (measured in $T_{x_\star} \M$) of the streaming average $\tilde{x}_n$ to the optimum,
  asymptotically saturates the Cramer-Rao bound on the manifold $\M$ \citep{smi05, Bou13}---asymptotically achieving the statistically optimal covariance\footnote{Note the estimator $\tilde{\Delta}_n$ is only asymptotically unbiased, and hence the Cramer-Rao bound is only meaningful in the asymptotic limit. However, this result can also be understood as saturating the H\`{a}jek-Le Cam local asymptotic minimax lower bound \citep[Ch. 8]{van1998asymptotic}.}. SGD, even with the carefully calibrated step-size choice of $\gamma_n = \frac{1}{\mu n}$, does not achieve this optimal asymptotic variance \citep{NevHas73}.
  \vspace*{-6pt}
\end{itemize}
We exhibit two applications of this general result in \mysec{application}. Next, we introduce the relevant background and assumptions that are necessary to prove our theorem.

%!TEX root = ../submission.tex

\section{Sum of Squares proofs and Sum of Squares Optimization}
\label{sec:sospreliminaries}
% The \emph{symmetrized Hausdorff distance} between two sets $S_0,S_1 \subseteq \R^n$ is defined as the $ \max_{b \in \{0,1\}} \min_{s \in S_b} \max_{t \in S_{1-b}} \frac{\langle s,t \rangle^2}{\|s\|^2\|t\|^2} \geq 1-\epsilon$. We say that two matrices $A, A'$ are $\epsilon$-close in symmetrized Hausdorff distance if the set of columns of $A, A'$ are $\epsilon$-close in symmetrized Hausdorff distance.


% We write $\Perm{[n]}{r}$ for the set of all $r$-tuples formed by taking $r$ \emph{distinct} elements from $[n]$. The Kronecker product $T$ of $v_1 \otimes v_2 \otimes \ldots v_k$ for vectors $v_i \in \R^{n_i}$ is a vector in $\R^{n_1 \times n_2 \times \cdots n_k}$ with entries indexed by tuples $(i_1, i_2, \ldots, i_k)$ where any $i_j \in [n_j]$ defined by $T(i_1, i_2, \ldots, i_k) = \Pi_{j \leq k} v_j(i_j)$. $\|\cdot \|$ for matrices will denote the spectral norm and for vectors, the Euclidean norm.

In this section, we define pseudo-distributions and sum-of-squares proofs.
See the lecture notes~\citep{BarakS16} for more details and the appendix in~\citet{DBLP:journals/corr/MaSS16} for proofs of the propositions appearing here.

Let $x = (x_1, x_2, \ldots, x_n)$ be a tuple of $n$ indeterminates and let $\R[x]$ be the set of polynomials with real coefficients and indeterminates $x_1,\ldots,x_n$.
We say that a polynomial $p\in \R[x]$ is a \emph{sum-of-squares (sos)} if there are polynomials $q_1,\ldots,q_r$ such that $p=q_1^2 + \cdots + q_r^2$.



\ignore{
\begin{theorem}[\cite{BM:2002}] \label{generalizationbound}
	%%
	Let $\mathcal{D}$ be a distribution over $\mathcal{X} \times \mathcal{Y}$ and let $\mathcal{L} : \mathcal{Y}^\prime
	\times \mathcal{Y}$ (where $\mathcal{Y} \subseteq \mathcal{Y}^\prime \subseteq \mathbb{R}$) be a
	$b$-bounded loss function that is $L$-Lipschitz in its first argument.  Let
	$\mathcal{F} \subseteq (\mathcal{Y}^\prime)^\mathcal{X}$ and for any $f \in \mathcal{F}$, let $\mathcal{L}(f; \mathcal{D}) := \E_{(\textbf{x}, y)
	\sim \mathcal{D}}[\mathcal{L}(f(\textbf{x}), y)]$ and $\hat{\mathcal{L}}(f; S) := \frac{1}{n} \sum_{i = 1}^n
	\mathcal{L}(f(\textbf{x}_\textbf{i}), y_i)$, where $S = ((\textbf{x}_\textbf{1}, y_1), \ldots,  (\textbf{x}_\textbf{n}, y_n))  \sim
	\mathcal{D}^n$. Then for any $\delta > 0$, with probability at least $1 - \delta$
	(over the random sample draw for $S$), simultaneously for all $f \in
	\mathcal{F}$, the following is true:
	%%
	\[
		|\mathcal{L}(f; \mathcal{D}) - \hat{\mathcal{L}}(f; S)| \leq 4 \cdot L \cdot \mathcal{R}_\textbf{n}(\mathcal{F})
		+ 2\cdot b \cdot \sqrt{\frac{\log (1/\delta)}{2n}}
	\]
	where $\mathcal{R}_\textbf{n}(\mathcal{F})$ is the Rademacher complexity of the function class $\mathcal{F}$. 
\end{theorem}

For a linear concept class, the Rademacher complexity can be bounded as follows.

\begin{theorem}[\cite{KST:2008}] \label{rademachercomplexity}
	Let $\mathcal{X}$ be a subset of a Hilbert space equipped with inner product $\langle
	\cdot, \cdot \rangle$ such that for each $\textbf{x} \in \mathcal{X}$, $\langle \textbf{x}, \textbf{x}
	\rangle \leq X^2$, and let $\mathcal{W} = \{ \textbf{x} \mapsto \langle \textbf{x} , \textbf{w} \rangle
	~|~ \langle \textbf{w}, \textbf{w} \rangle \leq W^2 \}$ be a class of linear functions.
	Then it holds that
	%%
	\[
		\mathcal{R}_\textbf{n}(\mathcal{W}) \leq X \cdot W \cdot \sqrt{\frac{1}{n}}.
	\]
\end{theorem}

The following result is useful for bounding the Rademacher complexity of a smooth function of a concept class.

\begin{theorem}[\cite{BM:2002, LT:1991}]
\label{rademachercomplexity2}
	Let $\phi : \mathbb{R} \rightarrow \mathbb{R}$ be  $L_{\phi}$-Lipschitz
	and suppose that $\phi(0) = 0$. Let $\mathcal{Y} \subseteq \mathbb{R}$, and for a function $f \in \mathcal{Y}^{\mathcal{X}}$. 
	Finally, for $\mathcal{F} \subseteq \mathcal{Y}^{\mathcal{X}}$, let $\phi \circ \mathcal{F} = \{\phi \circ f \colon f \in \mathcal{F}\}$.
	It holds that $\mathcal{R}_\textbf{n}(\phi
	\circ \mathcal{F}) \leq 2 \cdot L_{\phi} \cdot \mathcal{R}_\textbf{n}(\mathcal{F})$.
\end{theorem}
}
\subsection{Pseudo-distributions}

Pseudo-distributions are generalizations of probability distributions.
We can represent a discrete (i.e., finitely supported) probability distribution over $\R^n$ by its probability mass function $D\from \R^n \to \R$ such that $D \geq 0$ and $\sum_{x \in \mathrm{supp}(D)} D(x) = 1$.
Similarly, we can describe a pseudo-distribution by its mass function.
Here, we relax the constraint $D\ge 0$ and only require that $D$ passes certain low-degree non-negativity tests.

Concretely, a \emph{level-$\ell$ pseudo-distribution} is a finitely-supported function $D:\R^n \rightarrow \R$ such that $\sum_{x} D(x) = 1$ and $\sum_{x} D(x) f(x)^2 \geq 0$ for every polynomial $f$ of degree at most $\ell/2$.
(Here, the summations are over the support of $D$.)
A straightforward polynomial-interpolation argument shows that every level-$\infty$-pseudo distribution satisfies $D\ge 0$ and is thus an actual probability distribution.
We define the \emph{pseudo-expectation} of a function $f$ on $\R^d$ with respect to a pseudo-distribution $D$, denoted $\pE_{D(x)} f(x)$, as
\begin{equation}
  \pE_{D(x)} f(x) = \sum_{x} D(x) f(x) \,\mper
\end{equation}
The degree-$\ell$ moment tensor of a pseudo-distribution $D$ is the tensor $\E_{D(x)} (1,x_1, x_2,\ldots, x_n)^{\otimes \ell}$.
In particular, the moment tensor has an entry corresponding to the pseudo-expectation of all monomials of degree at most $\ell$ in $x$.
The set of all degree-$\ell$ moment tensors of probability distribution is a convex set.
Similarly, the set of all degree-$\ell$ moment tensors of degree $d$ pseudo-distributions is also convex.
Key to the algorithmic utility of pseudo-distributions is the fact that while there can be no efficient separation oracle for the convex set of all degree-$\ell$ moment tensors of an actual probability distribution, there's a separation oracle running in time $n^{O(\ell)}$ for the convex set of the degree-$\ell$ moment tensors of all level-$\ell$ pseudodistributions.

\begin{fact}[\citep{MR939596-Shor87,parrilo2000structured,MR1748764-Nesterov00,MR1846160-Lasserre01}]
  \label{fact:sos-separation-efficient}
  For any $n,\ell \in \N$, the following set has a $n^{O(\ell)}$-time weak separation oracle (as defined in ~\citet{MR625550-Grotschel81}):
  \begin{equation}
    \Set{ \pE_{D(x)} (1,x_1, x_2, \ldots, x_n)^{\otimes d} \mid \text{ degree-d pseudo-distribution $D$ over $\R^n$}}\,\mper
  \end{equation}
\end{fact}
This fact, together with the equivalence of weak separation and optimization~\citep{MR625550-Grotschel81} allows us to efficiently optimize over pseudo-distributions (approximately)---this algorithm is referred to as the sum-of-squares algorithm.

The \emph{level-$\ell$ sum-of-squares algorithm} optimizes over the space of all level-$\ell$ pseudo-distributions that satisfy a given set of polynomial constraints---we formally define this next.

\begin{definition}[Constrained pseudo-distributions]
  Let $D$ be a level-$\ell$ pseudo-distribution over $\R^n$.
  Let $\cA = \{f_1\ge 0, f_2\ge 0, \ldots, f_m\ge 0\}$ be a system of $m$ polynomial inequality constraints.
  We say that \emph{$D$ satisfies the system of constraints $\cA$ at
    degree $r$}, denoted $D \sdtstile{r}{} \cA$, if for every
  $S\subseteq[m]$ and every sum-of-squares polynomial $h$ with $\deg h
  + \sum_{i\in S} \max\set{\deg f_i,r} \leq \ell$,
  \begin{displaymath}
    \pE_{D} h \cdot \prod _{i\in S}f_i  \ge 0\,.
  \end{displaymath}
  We write $D \sdtstile{}{} \cA$ (without specifying the degree) if $D \sdtstile{0}{} \cA$ holds.
  Furthermore, we say that $D\sdtstile{r}{}\cA$ holds \emph{approximately} if the above inequalities are satisfied up to an error of $2^{-n^\ell}\cdot \norm{h}\cdot\prod_{i\in S}\norm{f_i}$, where $\norm{\cdot}$ denotes the Euclidean norm\footnote{The choice of norm is not important here because the factor $2^{-n^\ell}$ swamps the effects of choosing another norm.} of the cofficients of a polynomial in the monomial basis.
\end{definition}

We remark that if $D$ is an actual (discrete) probability distribution, then we have  $D\sdtstile{}{}\cA$ if and only if $D$ is supported on solutions to the constraints $\cA$.

We say that a system $\cA$ of polynomial constraints is \emph{explicitly bounded} if it contains a constraint of the form $\{ \|x\|^2 \leq M\}$.
The following fact is a consequence of Fact~\ref{fact:sos-separation-efficient} and~\citet{MR625550-Grotschel81},

\begin{fact}[Efficient Optimization over Pseudo-distributions]
There exists an $(n+ m)^{O(\ell)} $-time algorithm that, given any explicitly bounded and satisfiable system\footnote{Here, we assume that the bitcomplexity of the constraints in $\cA$ is $(n+m)^{O(1)}$.} $\cA$ of $m$ polynomial constraints in $n$ variables, outputs a level-$\ell$ pseudo-distribution that satisfies $\cA$ approximately. 
\end{fact}

A property of pseudo-distributions that we will use frequently is the following:
\begin{fact}[H\"older's inequality] \label{fact:pseudo-Holders}
Let $f,g$ be SoS polynomials. 
Let $p,q$ be positive integers so that $1/p + 1/q = 1$. 
Then, for any pseudo-distribution $\tmu$ of degree $r \geq pq \cdot deg(f) \cdot deg(g)$, we have:
\[
(\pE_{\tmu} [f \cdot g])^{pq} \leq \pE[f^{p}]^{q} \cdot \pE[g^{q}]^{p}
\]
In particular, for all even integers $k \geq 2$, and polynomial $f$ with $deg(f) \cdot k \leq r$, 
$$(\pE_{\tmu}[ f])^k \leq \pE_{\tmu}[f^k].$$
\end{fact}
%\Pnote{we also need the more general one in our SoS proof for certifiability where we apply it f = $\err$ and $g$, a sparse indicator}

\subsection{Sum-of-squares proofs}

Let $f_1, f_2, \ldots, f_r$ and $g$ be multivariate polynomials in $x$.
A \emph{sum-of-squares proof} that the constraints $\{f_1 \geq 0,
\ldots, f_m \geq 0\}$ imply the constraint $\{g \geq 0\}$ consists of
(sum-of-squares) polynomials $(p_S)_{S \subseteq [m]}$ such that
\begin{equation}
g = \sum_{S \subseteq [m]} p_S \cdot \Pi_{i \in S} f_i
\mper
\end{equation}
We say that this proof has \emph{degree $\ell$} if for every set $S \subseteq [m]$, the polynomial $p_S \Pi_{i \in S} f_i$ has degree at most $\ell$.
If there is a degree $\ell$ SoS proof that $\{f_i \geq 0 \mid i \leq r\}$ implies $\{g \geq 0\}$, we write:
\begin{equation}
  \{f_i \geq 0 \mid i \leq r\} \sststile{\ell}{}\{g \geq 0\}
  \mper
\end{equation}


Sum-of-squares proofs satisfy the following inference rules.
For all polynomials $f,g\colon\R^n \to \R$ and for all functions $F\colon \R^n \to \R^m$, $G\colon \R^n \to \R^k$, $H\colon \R^{p} \to \R^n$ such that each of the coordinates of the outputs are polynomials of the inputs, we have:

\begin{align}
&\frac{\cA \sststile{\ell}{} \{f \geq 0, g \geq 0 \} } {\cA \sststile{\ell}{} \{f + g \geq 0\}}, \frac{\cA \sststile{\ell}{} \{f \geq 0\}, \cA \sststile{\ell'}{} \{g \geq 0\}} {\cA \sststile{\ell+\ell'}{} \{f \cdot g \geq 0\}} \tag{addition and multiplication}\\
&\frac{\cA \sststile{\ell}{} \cB, \cB \sststile{\ell'}{} C}{\cA \sststile{\ell \cdot \ell'}{} C}  \tag{transitivity}\\
&\frac{\{F \geq 0\} \sststile{\ell}{} \{G \geq 0\}}{\{F(H) \geq 0\} \sststile{\ell \cdot \deg(H)} {} \{G(H) \geq 0\}} \tag{substitution}\mper
\end{align}

Low-degree sum-of-squares proofs are sound and complete if we take low-level pseudo-distributions as models.

Concretely, sum-of-squares proofs allow us to deduce properties of pseudo-distributions that satisfy some constraints.

\begin{fact}[Soundness]
  \label{fact:sos-soundness}
  If $D \sdtstile{r}{} \cA$ for a level-$\ell$ pseudo-distribution $D$ and there exists a sum-of-squares proof $\cA \sststile{r'}{} \cB$, then $D \sdtstile{r\cdot r'+r'}{} \cB$.
\end{fact}

If the pseudo-distribution $D$ satisfies $\cA$ only approximately, soundness continues to hold if we require an upper bound on the bit-complexity of the sum-of-squares $\cA \sststile{r'}{} B$  (number of bits required to write down the proof).

In our applications, the bit complexity of all sum of squares proofs will be $n^{O(\ell)}$ (assuming that all numbers in the input have bit complexity $n^{O(1)}$).
This bound suffices in order to argue about pseudo-distributions that satisfy polynomial constraints approximately.

The following fact shows that every property of low-level pseudo-distributions can be derived by low-degree sum-of-squares proofs.

\begin{fact}[Completeness]
  \label{fact:sos-completeness}
  Suppose $d \geq r' \geq r$ and $\cA$ is a collection of polynomial constraints with degree at most $r$, and $\cA \vdash \{ \sum_{i = 1}^n x_i^2 \leq B\}$ for some finite $B$.

  Let $\{g \geq 0 \}$ be a polynomial constraint.
  If every degree-$d$ pseudo-distribution that satisfies $D \sdtstile{r}{} \cA$ also satisfies $D \sdtstile{r'}{} \{g \geq 0 \}$, then for every $\epsilon > 0$, there is a sum-of-squares proof $\cA \sststile{d}{} \{g \geq - \epsilon \}$.
\end{fact}

We will use the following standard sum-of-squares inequalities:

\begin{fact}[SoS H\"older's Inequality]
Let $f_1, f_2, \ldots, f_n$ and $g_1, g_2, \ldots, g_n$ be SoS polynomials over $\R^d$. Let $p, q$ be integers such that $1/p + 1/q = 1$. Then, 
\[
\sststile{pq}{f_1, \ldots, f_n,g_1, \ldots, g_n} \Set{ \Paren{\frac{1}{n} \sum_{i} f_i g_i }^{pq} \leq \Paren{\frac{1}{n} \sum_{i=1}^n f_i^p}^q \Paren{\frac{1}{n} \sum_{i=1}^n g_i^q}^p }
\]

\end{fact}
\begin{fact}
For any $a_1, a_2,\ldots,a_n$, 
\[
\sststile{k}{a_1, a_2, \ldots, a_n} \Set{ (\sum_i a_i)^k \leq n^k \Paren{\sum_i a_i^k} } 
\]
\end{fact}
%\begin{fact}[Basis SoS Proofs]  \label{fact:sos-hypercontractivity}
% Let $\mu$ be a distribution on $\R^n$ such that: 
% \begin{enumerate}
% \item $\E[x(i)^\tau] = 1$ for every even $\tau \leq d$.
% \item $\E[ x^{\alpha}] =0$ whenever $\alpha$ has some odd individual degree, $|\alpha|\leq d$.
% \item $|\E[ x^{\alpha}]| \leq \beta$ for every $\alpha$ that is even, $|\alpha| \leq d$.
% \end{enumerate}
% Then, $\E[\langle x, u \rangle^{2\tau}] \leq \beta \cdot \tau^{2\tau} \E[\langle x, u \rangle^2] ^{\tau}$ for any $\tau \leq d/2$.
% \end{fact}




% \paragraph{Notation}
% For polynomial $p, q$ of degree at most $d$ in $v = (v_1, v_2, \ldots, v_n)$, we say that $p \preceq q$ if $q(v)-p(v)$ is a sum-of-squares polynomials in $v$.

% \subsection{Sum-of-Squares Norms}
% The convex set of moment tensors of degree $d$ pseudo-distributions naturally defines a norm on $d$-tensors over $\R^{n}$. 

% \begin{definition}[$sos_k$-norm] \label{def:SoS-norm}
% Given a tensor $T \in (\R^{n})^{\otimes d}$, we define the $sos_k$ ($k \geq d$) norm of $T$, denoted by $\|T\|_{\sos_k}$  and defined by
% \[
% \|T\|^2_{\sos_k} = \sup_{D: D \vDash_k \{\|u\|^2  = 1\}} \pE_{D(u)} [T(u)^2].
% \]
% \end{definition}

% It is instructive to compare the $\sos_k$ norm with the injective tensor norm of $T$:

% \begin{definition}[Symmetric Injective Tensor Norm]
% Let $T \in (\R^{n})^{\otimes d}$ be a $d$-dimension tensor on $\R^n$. The injective tensor norm of $T$ is defined as 
% \[
% \|T\|^2_{inj} = \sup_{x \in \bbS^{n-1}} T(x)^2.
% \]
% That is, $\|T\|^2_{inj}$ is the supremum of the square of the homogenous degree $d$ polynomial whose coefficients are the entries of $T$.  \footnote{Without the symmetry constraints, the injective tensor norm of $T$ is defined as the supremum over the polynomial $\langle T, \otimes_{i \leq d} x_i \rangle$ where $x_i$ are $n$ dimensional unit vectors which is also a well-studied notion. We omit further discussion here but direct the reader to \cite{}.}
% \end{definition}
% The (symmetric) injective tensor norm of $T$ is NP-hard to compute in general and can be thought of as the ``optimization'' version of the problem of constructing a separation oracle for the convex set of all moment tensors of an actual probability distribution. 

% It is simple to verify that the above definition is equivalent to saying 
% \[
% \|T\|^2_{inj} = \sup_{D: D\vDash_{\infty} \{\|x\|^2 \geq 0\}}  \E_{D}[ T(x)^2] 
% \]

% Thus, the $\sos_k$ norm of $T$ is a relaxation of the injective tensor and can be efficiently computed following the discussion from the previous section.



% \subsection{Concentration Results}
% We will need some standard concentration results. 

% The following gives concentration bounds in spectral norm for sums of independent rank 1 matrices.
% \begin{fact}[See Vershynin] \label{fact:vershynin-conc}
% Let $z_1, z_2, \ldots, z_m$ be independent random variables in $\R^n$ with identical covariance $\E[z_iz_i^{\top}] = \Sigma$. Let $\|z_i\| \leq \sqrt{R}$ almost surely for all $i$. Then, there's a constant $c > 0$ such that for every $t \geq 0$, with probability at least $1-n \exp(-ct^2),$
% \[
% \| \frac{1}{m} \sum_{i = 1}^m z_i z_i^{\top} - \Sigma  \| \leq \max (\|\Sigma\|^{1/2} \gamma, \gamma^2),
% \]
% where $\gamma = t \sqrt{R/m}$.
% \end{fact}

% The following stronger concentration result holds when individual rank 1 terms are from a subgaussian distribution. 
% \begin{fact}[\cite{}] \label{fact:subgaussian-conc}
% Let $x \in \R^n$ have a subgaussian distribution with constant subgaussian norm and covariance $\Sigma$. Let $x_1, x_2, \ldots, x_m$ be independent samples from $x$.

% Then, with probability at least $1- 2e^{-nt^2}$ and $m = \Omega( (t/\epsilon)^2 n)$, $\| \frac{1}{m} \sum_{i = 1}^m x_i x_i^{\top}  - \Sigma\| \leq \epsilon$. 
% \end{fact}

% We will need the following concentration results for polynomials on the gaussian distribution and uniform distribution on $\on^n$. 

% \begin{fact}[Hypercontractivity\cite{}] \label{fact:hypercontractivity}
% Let $f(z))$ be a degree $d$ polynomial. Then,
% $$\Pr_{z}[ |f(z) - \E[f(z)]| \geq t] \leq e^2 \cdot \exp(- (\frac{t^2}{RVar[f]})^{1/d}),$$ where $Var[f] = \E[f^2]$ and $R$ is some absolute constant. Here, $z$ is a random vector with each coordinate $z(i)$ being distributed independently as $\cN(0,1)$ or uniform over $\on$.
% \end{fact}

% % \begin{fact}[Concentration of Sum of Independent Subgaussian Rank 1 Terms] \label{fact:subgaussian-conc}

% % Let $z_1,z_2, \ldots, z_m$ be independent random vectors in $\R^n$ satisfying: $\E[ z_i z_i^{\top}] = 0$ and $\E[ exp(u^{\top} x_i )] \leq e^{\|u\|_2^2 \gamma/2}$ for every $u \in \R^n$ for every $1 \leq i \leq n$ almost surely. 
% % % For all $\epsilon_0 \in (0,1/2)$ and $\delta \in (0,1)$,
% % % $\Pr[ \| \frac{1}{m} \sum_{i = 1}^m z_i z_i^{\top} \| > \frac{1}{1-2\epsilon_0} \zeta] \leq \delta$ where $$\zeta = \gamma \cdot \left(     \sqrt{\frac{32 n \log{(1+2/\epsilon_0)} + 32 \log{(2/\delta)}}{m}} + \frac{2 n\log{(1+2/\epsilon_0)} + \log{(2/\delta)} }{m} \right)$$.
% % For every $\delta \in (0,1)$,
% % % In particular, for $\epsilon_0  = \frac{1}{4}$, the above bound says:
% % $\Pr[ \| \frac{1}{m} \sum_{i = 1}^m z_i z_i^{\top} \| > 2 \zeta] \leq \delta$ where $$\zeta \leq 10\gamma \cdot \left(   \sqrt{ \frac{n + \log{(2/\delta)}}{m}} + \frac{2 n + \log{(2/\delta)} }{m} \right)$.$

% % \end{fact}


\section{Relationship between Active Testing and Query Testing}
\label{sec:relationship}
The following Lemma from VC theory shows that when doing non-tolerant testing, the distribution can be assumed to have a finite support with size bounded by a function of the VC-dimension of the concept class.
\begin{lemma}
\label{lm:D'}
There exists an absolute constant $c$ satisfying the following property. Let $\calC$ be a concept class over  domain $X$ with VC-dimension $d$. Let $f$ be any function that is $\epsilon$-far from class $\calC$ with respect to distribution $\calD$ over $X$. Let $\calD'$ be the uniform distribution over a random iid sample from $\calD$ of size at least $\lceil\frac{cd}{\epsilon}\log\frac{1}{\epsilon}\rceil$. Then it holds that $f$ is $\frac{\epsilon}{2}$-far from class $\calC$ with respect to distribution $\calD'$ with probability at least $\frac{9}{10}$ over the random choice of the sample.
\end{lemma}

Therefore, when we perform non-tolerant testing in the active model, we can first sample $\lceil\frac{cd}{\epsilon}\log\frac{1}{\epsilon}\rceil$ unlabeled examples and choose $\calD'$ to be the uniform distribution over these examples. The \emph{active} testing task over $\calD'$ is almost the same as \emph{query} testing, because the active tester can query arbitrary points in the support of $\calD'$, leading to the following Lemma.

\begin{lemma}
\label{thm:unlabeled}
Let $\calC$ be a concept class on ground set $X$ with VC-dimension $d$. Suppose $\epsilon\in(0,\frac{1}{2})$. Suppose there is a non-tolerant query tester $\calA$ with margin $\frac{\epsilon}{2}$ using at most $q$ queries on an \emph{arbitrarily} given distribution with finite support. Suppose all the queries tester $\calA$ makes lie in the support of the distribution.
%\footnote{For $\tot$ and $\da$, we can assume without loss of generality that the algorithm never queries examples outside the support of the distribution, but this is not without loss of generality for $\pt$, because $f\in\calC$ is stronger than $\exists g\in\calC,\dist_{\calD}(f,g)=0$.} 
Then, there is a non-tolerant active tester $\calB$ with margin $\epsilon$ using at most $O(q)$ queries on $O(\frac{d}{\epsilon}\log\frac{1}{\epsilon})$ unlabeled examples for an arbitrary distribution \emph{unknown} to tester $\calB$.
\end{lemma}

%Obviously, as pointed out by \citet{BBBY12},  a $\pt_{\calD}(f\ac,\epsilon)$ algorithm implies a $\pt_{\calD}(f\qu,\epsilon)$ algorithm with the same query complexity when $\calD$ is known to the algorithm, since it can always then create unlabeled data on its own; this also holds for $\tot$ and $\da$.  Here, we show a theorem in the reverse direction for bounds that are worst-case over distributions $\calD$.  
%that has the reverse direction of the previous statement when $\calD$ is arbitrary rather than fixed. 
%Specifically, we show in Theorem \ref{thm:unlabeled} that a $\pt_{\calD}(f\qu,\frac{\epsilon}{2})$ algorithm can induce a $\pt_{\calD}(f\ac,\epsilon)$ algorithm with (except for a constant factor) the same query complexity and reasonable unlabeled sample complexity, under the assumption that the $\pt_{\calD}(f\qu,\frac{\epsilon}{2})$ algorithm never queries examples outside the support of $\calD$, which holds in all normal cases. We extend the theorem to $\da$ in Theorem \ref{thm:reductiontoquery}.

%As \citet{BBBY12} have pointed out, in the task of testing unions of $d$ intervals in the query testing framework, any known distribution can be reduced to uniform distribution on $[0,1]$ by its CDF. Our following theorem shows that once we can deal with arbitrary distributions for query testing, we can automatically deal with unknown distributions for active testing, improving a previous upper bound on unlabeled sample complexity in \citep{BBBY12}.

Since \citet{BBBY12} have an algorithm in the query testing framework that can distinguish $f\in\interval(d)$ and $f\notin\interval_\calD(d,\epsilon)$ for arbitrarily given distribution $\calD$ using $O(\frac{1}{\epsilon^4})$ queries and the tester only makes queries in the support of $\calD$, there is an algorithm in the active testing framework that can distinguish $f\in\interval(d)$ and $f\notin\interval_{\calD}(d,\epsilon)$ using $O(\frac{1}{\epsilon^4})$ queries on $O(\frac{d}{\epsilon}\log\frac{1}{\epsilon})$ unlabeled examples, even when the distribution $\calD$ is unknown, according to Lemma \ref{thm:unlabeled}. Here, the unlabeled sample complexity is $O(\frac{d}{\epsilon}\log\frac{1}{\epsilon})$, an improvement from $O(\frac{d^2}{\epsilon^6})$ (implicit) in their original paper.

The query tester $\calA$ in Lemma \ref{thm:unlabeled} is required to query only points in the support of the distribution. This requirement can be removed if $\calA$ accepts $f$ when $f$ has distance 0 to $\calC$ with respect to the distribution, because in this case the values of $f$ for points outside the support contain no information useful for the tester. The following Lemma shows that such a tester in the distribution-free model implies a passive tester over arbitrary unknown distributions.
\begin{lemma}
\label{lm:distributionfreenontolerant}
Suppose we have a non-tolerant distribution-free tester with margin $\frac{\epsilon}{2}$ and sample complexity $s$ for class $\calC$ with VC-dimension $d$, and the tester accepts $f$ when $f$ has distance 0 to the class $\calC$. Then there is a non-tolerant passive tester with margin $\epsilon$ and sample complexity $O(s)$ for arbitrary unknown distribution $\calD$ with no massive points\footnote{We say $x_0$ is a massive point if $\Pr_{x\sim \calD}[x=x_0]>0$.}.
\end{lemma}
\begin{proof}
Imagine we are performing non-tolerant testing in the passive model. The tester first obtains a sample $S$ of size $s$. When $s<\lceil\frac{cd}{\epsilon}\log\frac{1}{\epsilon}\rceil$, where $c$ is defined in Lemma \ref{lm:D'}, we enlarge $S$ to a bigger sample $S'$ of size $\lceil\frac{cd}{\epsilon}\log\frac{1}{\epsilon}\rceil$, though only $S$ is revealed to the tester. When $s\geq \lceil\frac{cd}{\epsilon}\log\frac{1}{\epsilon}\rceil$, we simply define $S'=S$. The testing task is then transformed to a testing task over distribution $\calD'$ uniform over $S'$ with margin $\frac{\epsilon}{2}$ and success probability at least $(\frac 23)/(\frac 9{10})=\frac{20}{27}$, according to Lemma \ref{lm:D'}. We then perform distribution-free testing with sample $S$. Note that the distribution $\calD$ has no massive points, so with probability 1 no queries made by the distribution-free tester lie in $S'\backslash S$ and thus the queries provide no information useful for the distribution-free tester. Therefore, we can assume the tester gets value 0 for all the queries it makes outside $S$.
\end{proof}

Lemmas \ref{lm:D'}, \ref{thm:unlabeled}  and \ref{lm:distributionfreenontolerant} can be naturally generalized to the tolerant case as follows.
%\begin{proof}
%Algorithm $\calB$ first draws $N=O(\frac{d}{\epsilon}\log\frac{1}{\epsilon})$ unlabeled examples: $x_1,x_2,\cdots,x_N$. We use $\calS$ to denote the uniform distribution over these unlabeled examples. By VC Theory, we know if $\forall g\in\calC,\dist_{\calD}(f,g)>\epsilon$, then with probability at least $\frac{5}{6}$, $\forall g\in\calC,\dist_{\calS}(f,g)>\frac{\epsilon}{2}$. So algorithm $\calB$ only needs to call algorithm $\calA$ to distinguish $f\in\calC$ and $\forall g\in\calC,\dist_{\calS}(f,g)>\frac{\epsilon}{2}$ with probability at least $\frac{5}{6}$. By the Union Bound, algorithm $\calB$ succeeds with probability at least $\frac{2}{3}$.
%\end{proof}

\begin{lemma}
There exists an absolute constant $c$ satisfying the following property. Let $\calC$ be a concept class over  domain $X$ with VC-dimension $d$. Let $f$ be any function that has distance $\alpha$ to class $\calC$ with respect to distribution $\calD$ over $X$. Let $\calD'$ be the uniform distribution over a random iid sample from $\calD$ of size at least $\lceil\frac{cd}{\epsilon^2}\log\frac{1}{\epsilon}\rceil$. Then it holds that $f$ has distance within $\alpha\pm\epsilon$ to class $\calC$ with respect to distribution $\calD'$ with probability at least $\frac{9}{10}$ over the random choice of the sample.
\end{lemma}

\begin{lemma}
\label{thm:reductiontoquery}
Let $\calC$ be a concept class on ground set $X$ with VC-dimension $d$. Suppose $\epsilon\in (0,\frac{1}{2})$. Suppose there is a tolerant query tester $\calA$ with additive error $\frac \epsilon 2$ using at most $q$ queries on an \emph{arbitrarily} given distribution with finite support. Then, there is a tolerant active tester $\calB$ with additive error $\epsilon$ using at most $O(q)$ queries on $O(\frac{d}{\epsilon^2}\log\frac{1}{\epsilon})$ unlabeled examples for an arbitrary distribution \emph{unknown} to tester $\calB$.
\end{lemma}

\begin{lemma}
\label{lm:distributionfreetolerant}
Suppose we have a tolerant distribution-free tester with additive error $\frac{\epsilon}{2}$ and sample complexity $s$ for class $\calC$ with VC-dimension $d$. Then there is a tolerant passive tester with additive error $\epsilon$ and sample complexity $O(s)$ for arbitrary unknown distribution $\calD$ with no massive points.
\end{lemma}
%The unlabeled sample complexity of the algorithm is improved to $O(\frac{d}{\epsilon}\log\frac{1}{\epsilon})$ by our Theorem \ref{thm:unlabeled}.



%In the active tolerant testing model, given a distribution $\mathcal{D}$, to test the concept class $\dintervals$ is to distinguish $f\in\interval_\mathcal{D}(d,\alpha_1)$ and $f\notin\interval_\mathcal{D}(d,\alpha_2)$ for any given parameters $\alpha_1<\alpha_2$, where $\interval_\mathcal{D}(d,\alpha)$ is the class of functions $f$ satisfying $\exists g\in \mathcal{I}(d)$ s.t. $\dist_{\mathcal{D}}(f,g)\leq \alpha$. When $\calD$ is the uniform distribution on unit interval $[0,1]$, we omit it for short.


\section{The Composition Lemma}
\label{sec:composition}
\citet{BBBY12} showed that disjoint unions of testable properties are testable in the non-tolerant, active model. We extend their result to tolerant testing in Appendix \ref{sec:union}. Here, we propose a more general notion of a certain concept class formed by composing smaller concept classes on disjoint ground sets.

Suppose we have $m$ disjoint ground sets $X_1,X_2,\cdots,X_m$ and on each $X_i$, we have a sequence of concept classes $\calC_i^0, \calC_i^1, \calC_i^2, \cdots\subseteq\{0,1\}^X$. Suppose $\calC_i^0\neq\emptyset$ for all $i$. We use $X$ to denote the disjoint union $\bigcup\limits_{i=1}^m X_i$. For any $d\geq 0$, we define a concept class $\property(d)$ on $X$ to be the class of functions $f\in\{0,1\}^X$ satisfying that $\exists k_1,k_2,\cdots,k_m\in\mathbb{N}$ s.t.
\begin{enumerate}
\item $\sum\limits_{i=1}^mk_i\leq d$;
\item $\forall 1\leq i\leq m, f|_{X_i}\in\calC_i^{k_i}$.
\end{enumerate}

We call $\property$ a \emph{composition of $m$ additive properties}. Note that $\property(0)=\{f\in\{0,1\}^X:\forall 1\leq i\leq m,f|_{X_i}\in\calC_i^0\}$, matching the definition of a disjoint union of properties in \citep{BBBY12}. Also note that $\property(0)\neq \emptyset$ because of the assumption that $\calC_i^0\neq \emptyset$ for all $i$. 

For a given $t\geq 0$, we define a composition $\property^t$ in the same way as $\property$ except that we further require every $k_i$ to be at most $t$, or, $\property^t$ is a composition of $m$ additive properties \emph{truncated by $t$}. %If $\property(0)=\emptyset$, i.e., for some $i$, it holds that $\calC_i^0=\emptyset$, we say the composition is \emph{degenerated}. In this case, if we define the new $\calC_i^{k}$ to be the old $\calC_i^{k+1}$ for $k=0,1,\cdots$, then the new $\property(d)$ becomes the old $\property(d+1)$ for all $d$. By repeating this process, as long as there exists some $d$ such that $\property (d)$ is non-empty, we can finally make $\property(0)$ to be non-empty, i.e., we can make the composition to be \emph{non-degenerated}.

For any distribution $\calD$ over $X$, we use $\property_{\calD}(d,\alpha)$ to denote functions that are $\alpha$-close to $\property(d)$ with respect to $\calD$, i.e. $\property_{\calD}(d,\alpha)=\{f\in\{0,1\}^X:\exists g\in\property(d),\dist_{\calD}(f,g)\leq\alpha\}$. Similarly, we define $\property^t_{\calD}(d,\alpha)=\{f\in\{0,1\}^X:\exists g\in\property^t(d),\dist_{\calD}(f,g)\leq\alpha\}$. We say $\calD$ is \emph{semi-uniform} if $\forall 1\leq i\leq m,\Pr_{x\sim \calD}[x\in X_i]=\frac{1}{m}$.


%\subsection{Composition with Truncation}
%\label{subsec:truncation}
%Suppose $m,X_i,\calC_i^k,X$ are defined as in Section \ref{subsec:composition}. 
%Suppose some $t\geq 0$ is chosen as the \emph{threshold of truncation}. For any $d\geq 0$, we define a concept class $\property^t(d)$ on $X$ to be the class of functions $f\in\{0,1\}^X$ satisfying that $\exists k_1,k_2,\cdots,k_m\in\mathbb{N}$ s.t.
%\begin{enumerate}
%\item $\forall 1\leq i\leq m, k_i\leq t$;
%\item $\sum\limits_{i=1}^mk_i\leq d$;
%\item $\forall 1\leq i\leq m, f|_{X_i}\in\calC_i^{k_i}$.
%\end{enumerate}
%We call $\property^t$ a composition of $m$ additive properties \emph{truncated by $t$}. For any distribution $\calD$ over $X$, we use $\property_{\calD}^t(d,\alpha)$ to denote $\{f\in\{0,1\}^X:\exists g\in\property^t(d)\text{ s.t. }\dist_{\calD}(f,g)\leq\alpha\}$.
An $(\epsilon,\mu)$-bi-criteria distance approximation algorithm $\ap_{\calD}(f,(\epsilon,\mu),d)$ for composition $\property$ of additive properties, is an algorithm that takes $f,\epsilon,\mu$ and $d$ as input and outputs $\hat\alpha$ such that $\forall f$
\begin{enumerate}
\item $\forall \alpha$ s.t. $f\in\property_{\calD}(d,\alpha)$, it holds with probability at least $\frac{2}{3}$ that $\hat\alpha\leq\alpha+\epsilon$;
\item $\forall \alpha$ s.t. $f\notin\property_{\calD}((1+\mu)d,\alpha)$, it holds with probability at least $\frac{2}{3}$ that $\hat\alpha>\alpha-\epsilon$.
\end{enumerate}
%\subsection{Distance Approximation Oracle}
%Suppose $m,X_i,\calC_i^k,X,t,\widehat\property(d)$ are defined as in Section \ref{subsec:composition} and Section \ref{subsec:truncation}. Suppose we have a sequence of indices $1\leq i_1<i_2<\cdots i_l\leq m$. We use $\mathbf{i}$ to denote $(i_1,i_2,\cdots,i_l)$ for short. For any $d\geq 0$, we define a concept class $\widehat\property^{\mathbf i}(d)$ on $X$ to be the class of functions $f\in\{0,1\}^X$ satisfying that $\exists k_1,k_2,\cdots,k_l\in\mathbb{N}$ s.t.
%\begin{enumerate}
%\item $\forall 1\leq j\leq l,k_j\leq t$;
%\item $\sum\limits_{j=1}^lk_j\leq d$;
%\item $\forall 1\leq j\leq l,f|_{X_{i_j}}\in\calC_{i_j}^{k_j}$.
%\end{enumerate}
%Here, $\widehat\property^{\mathbf i}$ is called the \emph{sub-composition of $l$ additive properties truncated by $t$}. When $l=m$, we have $\widehat\property^{\mathbf i}(d)=\widehat\property(d)$.

%For any distribution $\calD$ over $X$, we use use $\widehat\property_{\calD}^{\mathbf i}(d,\alpha)$ to denote $\{f\in\{0,1\}^X:\exists g\in\widehat\property^{\mathbf i}(d)\text{ s.t. }\dist_{\calD}(f,g)\leq \alpha\}$.

Suppose we have a sequence of indices $1\leq i_1<i_2<\cdots<i_l\leq m$ denoted by $\mathbf i$ for short. Let $\calD_{\mathbf i}$ denote the conditional distribution of $\calD$ on $\bigcup\limits_{j=1}^lX_{i_j}$. A \emph{$(d,l,t,\epsilon)$ distance approximation oracle} is an algorithm taking a length-$l$ sequence $\mathbf i$ of indices and $f\in\{0,1\}^X$ as input, and performing $\ap_{\calD_{\mathbf i}}(f\ac,(\epsilon,0),d)$ on composition $\property^t$. In other words, this algorithm performs distance approximation on any given $l$-sub-union ($l$ is typically small) of the $m$ ground sets. For convenience of use, we require the success probability of the oracle to be at least $\frac{11}{12}$. The proof of the following lemma can be found in Appendix \ref{sec:compositionproof}.

%such that $\forall \alpha$,:
%\begin{enumerate}
%\item $\forall f\in\widehat\property^{\mathbf i}_{\calD_{\mathbf i}}(d,\alpha)$, with probability at least $\frac{2}{3}$, it holds that $\hat\alpha\leq\alpha+\epsilon$;
%\item $\forall f\notin\widehat\property^{\mathbf i}_{\calD_{\mathbf i}}(d,\alpha_2)$, with probability at least $$.
%\end{enumerate}





%\subsection{Bi-Criteria Distance Approximation}
\begin{lemma}[Composition Lemma]
\label{thm:additive}
Suppose $\property$ is the composition of $m$ additive properties defined above. Let $\calD$ be a semi-uniform distribution. For parameters $\lambda>0,\alpha\in[0,1]$ and $\mu,\epsilon\in(0,1)$ taken as input, there exists $l=O(\frac{1}{\epsilon\mu^2}+\frac{1}{\epsilon^2})$ such that we have an algorithm that performs $\ap_{\calD}(f\ac,(\epsilon,\mu),\lambda m)$ by calling once a $((1+\frac{\mu}{2})\lambda l,l,\frac{4\lambda}{\epsilon},\frac{\epsilon}{2})$ distance approximation oracle. Suppose the query complexity and the unlabeled sample complexity of the oracle are $q$ and $N$, respectively. Then the query complexity and the unlabeled sample complexity of the algorithm are $q$ and $O(\frac{Nm}{l})$, respectively.





%$((1+\frac{\mu}{2})\lambda l,l,t,\alpha+\frac{\epsilon}{3},\alpha+\frac{\epsilon}{2})$-partial testing using at most $q(l,\lambda,\alpha,\mu,\epsilon)$ queries on at most $U(l,\lambda,\alpha,\mu,\epsilon)$ unlabeled samples with truncation $t=\frac{6\lambda}{\epsilon}$. Then, there is an active testing algorithm that distinguishes $f\in\property_{\calD}(\lambda m,\alpha)$ and $f\notin\property_{\calD}((1+\mu)\lambda m,\alpha+\epsilon)$ using at most $O(q(s,\lambda,\alpha,\mu,\epsilon))$ queries on at most $O(U(s,\lambda,\alpha,\mu,\epsilon)\cdot \frac{m}{s})$ unlabeled samples for $s=O(\frac{1}{\epsilon\mu^2}+\frac{1}{\epsilon^2})$.
\end{lemma}


\section{Interval Regret}\label{sec:interval_regret}

In this section we present several algorithms with interval regret
guarantees.  As a starter and a baseline, we first point out that a
generalization of the Exp3.S algorithm~\citep{AuerCeFrSc02} and
Fixed-Share~\citep{HerbsterWa98} to the contextual bandit setting,
which we call Exp4.S, already provides a strong interval regret
guarantee as shown by the following theorem. We include the algorithm
and the proof in Appendix~\ref{app:Exp4.S}.  Crucially, Exp4.S
requires maintaining weights for each policy and is thus not oralce-efficient.

\begin{theorem}\label{thm:Exp4.S}
Exp4.S with parameter $L$ ensures that for any time interval $\calI$
such that $|\calI| \leq L$, we have $\E\left[
  \sum_{t\in\calI}r_t(\pi(x_t)) - r_t(a_t) \right] \leq
\order(\sqrt{LK\ln(NL)})$ for any $\pi \in \Pi$, where the expectation
is with respect to the randomness of both the algorithm and the
environment.
\end{theorem}

Note that in bandit settings, it is impossible to achieve regret
$\order(\sqrt{|\calI|})$ for all interval $\calI$ simultaneously~\citep{DanielyGoSh15}.
When $|\calI|$ is unknown, a safe choice is to pick $L = T$
(this is how we obtain the results in the ``param-free'' column of Table~\ref{tab:results} for interval regret).
Next we prove statements similar to Theorem~\ref{thm:Exp4.S} but with
oracle-efficient algorithms. 

\paragraph{A general approach.}
In the full information setting, a general approach to convert an algorithm with classic regret guarantee
to another with interval regret is to combine different copies of the algorithm with an expert algorithm,
each of which starts at a different time step to learn over different time intervals.
This works well in the full information setting where one has correct feedback to update all the base algorithms,
but becomes challenging in the bandit setting.
We show in Appendix~\ref{app:BISTRO+} how to leverage recent results by~\citet{AgarwalLuNeSc17}
and~\citet{wei2018more} to deal with such challenges.
As an example we use the \bistro algorithm~\citep{SyrgkanisLuKrSc16, rakhlin2016bistro} as the base algorithm
since it is oracle-efficient and allows adversarial rewards.

\begin{theorem}\label{thm:corralling_BISTRO+}
In the transductive setting, Algorithm~\ref{alg:corralling_BISTRO+} in Appendix~\ref{app:BISTRO+} guarantees
that for any time interval $\calI$ such that $|\calI| \leq L$ and any policy $\pi \in \Pi$, we have $\E\left[
  \sum_{t\in\calI}r_t(\pi(x_t)) - r_t(a_t) \right] \leq
\otil(T^{\frac{1}{4}}(LK)^{\frac{1}{2}}(\ln N)^{\frac{1}{4}})$.
\end{theorem}

Unlike the full information setting, this general approach results in worse regret rates and is unsatisfying
(\bistro achieves $\otil(T^{2/3})$ for the classic regret and here we only obtain $\otil(T^{3/4})$).
In the following subsections, we turn to different approaches. 

%%%%%%%%%%%%%%%%%%%%%%%%%%%%%%%%%%
%     New Algorithms
%%%%%%%%%%%%%%%%%%%%%%%%%%%%%%%%%%
\subsection{\AdaEG}

The simplest oracle-efficient contextual bandit algorithm is the \EG
method~\citep{LangfordZh08} which assumes i.i.d. data. 
In this section, we extend the related \EPG
algorithm to enjoy a small interval regret on any interval with a
small variation.

\EPG plays uniformly at random with a small probability and
otherwise follows the empirically best policy $\pi_t=\argmax_{\pi\in\Pi}
\avgR_{[1,t-1]}(\pi)$. The number of oracle calls can be greatly reduced if the learner updates the best policy only at $t=1, 2, 4, 8, \ldots$ (that is, $\pi_t=\argmax_{\pi\in\Pi}
\avgR_{[1,2^{\floor{\log_2{t}}}-1]}(\pi)$). 
\AdaEG, described in Algorithm~\ref{alg:AdaGreedy2}, behaves similarly to this version of \EPG. 
The difference is that at each round, an additional \textit{non-stationarity test} is executed. 
The test monitors whether there is a policy performing significantly better on recent samples (collected in a doubling manner), 
compared to the policy that the algorithm is using. 
Intuitively, if such a policy exists, there should have been a significant shift in the distribution.
In this case, the algorithm restarts from scratch. 

In addition, the algorithm also resets every $L$ rounds for some parameter $L$ (Line~\ref*{line:trigger_rerun}). 
This prevents the risk of slow detection of a distribution change, but at the same time also causes some extra penalty when the environment is stationary.
The parameter $L$ trades these two kinds of costs, and can be selected based on prior knowledge about the environment. 

We call the rounds between resets an {\it epoch} (so epoch $i$ is the interval $[T_i+1, T_{i+1}]$), 
and the rounds between updates of the empirically best policy a {\it block} (so block $j$ of epoch $i$ is the interval $[T_i+2^{j-1}, T_i+2^{j}-1]$).

%\paragraph{Oracle-Efficiency.} 
Note that there are only two places where we need to invoke the oracle: computing $\hat{\pi}_{(i,j)}$ and $\hat{\pi}_{A}$ 
($\hat{\pi}_{B}$ is simply equal to $\hat{\pi}_{(i,j)}$),
and it is thus clear that at most $\order(\ln L) = \order(\ln T)$ oracle calls are used per round.
%Since $t-T_i\leq L$ and we only check for $\ell=1, 2, 4, 8\ldots$, there will be at most $\order(\ln L)$ oracle calls per round. %In fact, the number of oracle calls per round can be reduced to $\order(1)$ without hurting the regret bounds with a more economical but sophisticated test length scheduling. We state this version of non-stationarity test in Appendix XXXXX.

%For convenience, we call an interval with a specific $i$ in the algorithm an \textit{epoch}. In an epoch, we call an interval with a specific $j$ a \textit{block}. 
%Also, we use $B(i,j)$ to denote the time interval $[T_i+1, T_i+2^{j-1}-1]$ (sometimes when $i, j$ are clear, we only use $B$). In \AdaEG, $\pi_t$ is set to be $\hat{\pi}_{B(i,j)}$ for any $t$ in epoch $i$ and block $j$.

We prove the following result for \AdaEG, stating a regret bound for all intervals with length smaller than $L$
and variation smaller than another parameter $v$ of the algorithm.
\begin{theorem}\label{thm:AdaEG2}
With probability at least $1 - \delta$, for all time intervals $\calI$
such that $|\calI| \leq L$ and $\var_\calI \leq v$, \AdaEG with
parameters $L$, $v$ and $\delta$ guarantees for any $\pi \in
\Pi$,%
\footnote{We use notation $\otil$ to suppress dependence on
  logarithmic factors in $L, T, K$ and $\ln(N/\delta)$. } 
\[
\sum_{t \in \calI} r_t(\pi(x_t)) - r_t(a_t) \leq \otil\left(\abs{\calI}v +
L^\frac{1}{6}\sqrt{K\abs{\calI}\ln(N/\delta)} + K\ln(N/\delta)\right).
\]
\end{theorem}

\begin{algorithm}[t]
\SetAlgoLined
\setcounter{AlgoLine}{0}
\SetAlgoVlined
\DontPrintSemicolon
\caption{\AdaEG}\label{alg:AdaGreedy2}
\nl {\bf Input}: largest allowed interval length $L$ and variation $v$, allowed failure probability $\delta$  \\ 
\nl {\bf Define}:
$\mu=\min\Big\{ 
\frac{1}{K}, 
L^{-\frac{1}{3}} \sqrt{\frac{\ln(N/\delta)}{K}} 
\Big\}, 
\beta_\calI= 
2\sqrt{\frac{ \ln(4T^2N/\delta) }{ \mu\abs{\calI}} } 
+ \frac{ \ln(4T^2N/\delta) }{ \mu\abs{\calI} }$\\
\nl {\bf Initialize}: $i=1$, $T_1=0$. \Comment{$i$ indexes an epoch}\label{line:rerun_beginning2}\\
\nl\label{line:adagreedy restart}\For(\Comment{$j$ indexes a block}){$j=1, 2, \ldots$}{ 
%\nl Define $B(i,j)=[T_i+1, T_i+2^{j-1}-1]$ \\
\nl Compute $\hat{\pi}_{(i,j)} = \argmax_{\pi\in\Pi} \avgR_{[T_i+1, T_i+2^{j-1}-1]}(\pi)$ 
\Comment{or arbitrary if $j=1$}\\
\nl \For{$t = T_i+2^{j-1}, \ldots, T_i+2^{j}-1$}{
%\nl Let $\pi_t=\hat{\pi}_{B(i,j)}$\\
\nl Set $p_t(a)=\mu + (1-K\mu) \one \{ a=\hat{\pi}_{(i,j)}(x_t) \}, 
\forall a\in [K]$ \\%, where $\pi_t=\hat{\pi}_{(i,j)}$\\
\nl Play $a_t\sim p_t$ and receive $r_t(a_t)$ \\
\nl\label{line:trigger_rerun}\If{$(t\geq T_i+L) \text{ or } (j>1 \text{ and } \test(t)=\true)$}{
\nl                  $T_{i+1}\leftarrow t$, $i\leftarrow i+1$ \\ 
\nl                  \textbf{goto} Line~\ref{line:adagreedy restart}
              }   
          }
      }
\ \\
\textbf{Procedure\ }$\test(t)$\\
\nl $\ell=1$ \\
\nl \While{$\ell\leq t-T_i$}{
\nl    Let $A\triangleq [t-\ell+1, t]$ and $B\triangleq [T_i+1, T_i+2^{j-1}-1]$ \\ %\Comment{$\hat{\pi}_B = \hat{\pi}_{(i,j)}$ is computed already}
\nl\label{eqn:ada_compare1}\lIf{$ \avgR_{A}(\hat{\pi}_{A}) > \avgR_{A}(\hat{\pi}_B) + 2(\beta_{A} + \beta_{B}+ 2v)$}{
       \textbf{return} \true 
    }
\nl    $\ell \leftarrow 2\ell$
}
\nl \textbf{return} \false
\end{algorithm}

%\footnotetext{The game should proceed for only $T$ rounds. However for conciseness we do not specify this ending condition in the algorithm.}
%\stepcounter{footnote}
%\footnotetext{In \test(t) we only utilize samples up to $t-1$. This is only to make the analysis simpler.}

Note that whenever $v =O(L^{-\frac{1}{3}})$, the rate of the regret above is of order $\otil(L^{2/3})$ (since $|\calI| \leq L$), 
which matches the ordinary regret bound of \EG ($\otil(T^{2/3})$). 
While a condition on both the interval length and variation is seemingly strong
and the bound seems to be meaningful only for very small $v$,
we emphasize that 1) sublinear regret via oracle-efficient algorithms is impossible under a fully adversarial setting 
even for the classic regret~\citep{Hazan2016} and 2) based on Theorem~\ref{thm:AdaEG2} we can in fact derive strong dynamic regret bounds that hold without 
any assumption on the distribution sequence (see Section~\ref{sec:implications}).



%%%%%%%%%%%%%%%%%%%%%%%%%%%%%%%%%%
%    Ada ILTCB
%%%%%%%%%%%%%%%%%%%%%%%%%%%%%%%%%%

\subsection{\AdaILTCB}
\begin{algorithm}[t]
\SetAlgoLined
\setcounter{AlgoLine}{0}
\SetAlgoVlined
\DontPrintSemicolon
\caption{\AdaILTCB}\label{alg:AdaILTCB2}
\nl {\bf Input}: largest allowed interval length $L$ and variation $v$, allowed failure probability $\delta$ \\ 
\nl {\bf Define}:
$\mu=\min\Big\{ \frac{1}{2K}, L^{-\frac{1}{2}}\sqrt{\frac{\ln(8T^2N^2/\delta)\ln(L)}{K}} \Big\}, C_1=4, C_2=10^6, C_3=1.1\times 10^3, C_4=41, C_5=1200, C_6=6.4$ \\
\nl {\bf Initialize}: $i=1$, $T_1=0$ \Comment{$i$ indexes an epoch}\label{line:rerun_beginning2}\\
\nl\label{line:adaILTCB restart}\For(\Comment{$j$ indexes a block}){$j=1,2,\ldots$} {
%\nl       Define $B(i,j)=[T_i+1, T_i+2^{j-1}-1]$\\
\nl       Let $Q_{(i,j)}$ be a solution to (OP) with parameter $\mu$ and data from $[T_i+1, T_i+2^{j-1}-1]$ \\
%        \Comment{or arbitrary if $j=1$} \\ 
\nl       \For{$t = T_i+2^{j-1}, \ldots, T_i+2^{j}-1$}{
%\nl           Let $Q_t=Q^*$\\
\nl          Set $p_t(a)=Q_t^{\mu}(a|x_t), \forall a\in [K]$ where  $Q_t = Q_{(i,j)}$\\
\nl          Play $a_t\sim p_t$ and receive $r_t(a_t)$ \\
\nl \label{line:trigger_rerun}  \If{$(t\geq T_i+L) \text{ or } (\test(t)=\true)$}{
\nl                  $T_{i+1}\leftarrow t$, $i\leftarrow i+1$ \\ 
\nl                  \textbf{goto} Line~\ref{line:adaILTCB restart}
              }   
          }
      }
\ \\
\textbf{Procedure\ }$\test(t)$\\
\nl $\ell=1$ \\
\nl \While{$\ell\leq t-T_i-1$}{
\nl   Let $A\triangleq [t-\ell,t-1]$ and $B\triangleq [T_i+1, T_i+2^{j-1}-1]$ \\
\nl\label{line:ILTCB:check_reg}\lIf{$\max_{\pi\in\Pi}\left\{\whReg_{B}(\pi)-C_1\whReg_{A}(\pi)\right\} > \frac{C_2LK\mu}{\ell} + C_3v$}{
       \textbf{return} \true
    }
\nl\label{line:ILTCB:check_reg_another}\lIf{$\max_{\pi\in\Pi}\left\{\whReg_{A}(\pi)-C_1\whReg_{B}(\pi)\right\} > \frac{C_2LK\mu}{\ell} + C_3v$}{
       \textbf{return} \true
    }
\nl\label{line:ILTCB:check_var}\lIf{$\max_{\pi\in\Pi}\left\{ \whV_{A}(Q_t,\pi) - C_4\whV_{B}(Q_t,\pi) \right\} > \frac{C_5LK}{\ell} + \frac{C_6 v}{\mu}$} {
       \textbf{return} \true
   }
   
\nl    $\ell \leftarrow 2\ell$
}
\nl \textbf{return} \false
\end{algorithm}
%\footnotetext{Here we only check for samples up to time $t-1$ (\textit{cf.} \AdaEG). This is only to simplify the analysis. }

Although being fairly simple, \AdaEG is suboptimal just as \EG is
suboptimal for stationary environments.  In this section we propose
\AdaILTCB, a variant of \minimonster~\citep{AgarwalHsKaLaLiSc14},
which achieves the optimal regret rate while also being
oracle-efficient.  The idea is similar to \AdaEG, but the statistical
checks are more involved.  

For a policy $\pi$ and an interval $\calI$, we denote the expected and
empirical regret of $\pi$ by $\Reg_\calI(\pi) = \max_{\pi' \in
  \Pi}\calR_\calI(\pi') - \calR_\calI(\pi)$ and $\whReg_\calI(\pi) =
\max_{\pi' \in \Pi}\avgR_\calI(\pi')- \avgR_\calI(\pi)$ respectively.
For a context $x$ and a distribution over the policies $Q \in
\Delta^\Pi \coloneqq \{Q \in \fR^N_+: \sum_{\pi \in \Pi} Q(\pi) =
1\}$, the projected distribution over the actions is denoted by
$Q(\cdot|x)$ such that $Q(a|x) = \sum_{\pi: \pi(x)=a} Q(\pi), \;
\forall a \in [K]$. The smoothed projected distribution with a minimum
probability $\mu$ is defined as $Q^\mu(\cdot|x) = \mu\one +
(1-K\mu)Q(\cdot|x)$ where $\one$ is the all-one
vector. Like~\citep{AgarwalHsKaLaLiSc14}, we keep track of a bound
on the variance of the reward estimates and define for a policy $\pi$, an
interval $\calI$ and a distribution $Q \in \Delta^\Pi$
\[
\whV_\calI (Q, \pi) = \frac{1}{|\calI|} \sum_{t \in \calI} \left[ \frac{1}{Q^{\mu}(\pi(x_t) | x_t)} \right], 
\quad V_\calI(Q, \pi) = \frac{1}{|\calI|} \sum_{t \in \calI} \E_{x \sim D_t^\calX} \left[ \frac{1}{Q^{\mu}(\pi(x) | x)} \right].
\]

Similar to \AdaEG, the proposed algorithm \AdaILTCB (Algorithm~\ref{alg:AdaILTCB2}) proceeds like the base algorithm (\sloppy \minimonster in this case) with additional tests to detect the non-stationarity of the environment. We define an epoch and a block similar to those of \AdaEG. The algorithm solves the optimization (OP) defined in~\citep{AgarwalHsKaLaLiSc14} (and included in
Appendix~\ref{app:AdaILTCB2}) at the beginning of each block using the data collected in that epoch so far. 
The solution of (OP) is denoted by $Q_t$, a \textit{sparse} distribution over $\Pi$, and the learner samples actions based on $Q_t^{\mu}(\cdot|x_t)$. 

At each round, the \test checks whether the empirical regret or the variance of reward estimates of any policy has changed significantly in a recent interval (i.e., $[t-\ell,t-1]$), compared to the interval from which we compute $Q_t$ (i.e., $[T_i+1, T_i+2^{j-1}-1]$). If so, the algorithm restarts with a new epoch. Note that detecting the change of regret is similar to detecting the change of reward; but different from \AdaEG, here we also check the change of reward estimate variance. This inherits from the tighter variance control in \minimonster, the key to obtaining better regret compared to \EPG. 

\paragraph{Oracle-Efficiency.}
Note that Lines~\ref*{line:ILTCB:check_reg}, \ref*{line:ILTCB:check_reg_another} and \ref*{line:ILTCB:check_var}
can all be implemented by one call of
the AMO oracle each, after using two extra oracle calls to compute $\max_{\pi' \in \Pi}\avgR_{B}(\pi')$ and $\max_{\pi' \in \Pi}\avgR_{A}(\pi')$ in advance. % for each new $\ell$. 
Specifically, let $\calS = \{(x_\tau,
\frac{-1}{2^{j-1}-1}\ips_\tau)\}_{\tau\in B} \cup \{(x_\tau,
\frac{C_1}{\ell} \ips_\tau)\}_{\tau\in A}$, then the left hand
side of the inequality in Line~\ref*{line:ILTCB:check_reg} can be rewritten
as $\max_\pi \sum_{(x,r) \in \calS} r(\pi(x)) + \max_{\pi' \in
  \Pi}\avgR_{B}(\pi') - C_1\max_{\pi' \in
  \Pi}\avgR_{A}(\pi')$, where clearly the first term can be
computed by one oracle call and the rests are precomputed already.
Similarly, Line~\ref*{line:ILTCB:check_var} can be computed by feeding
the oracle with examples $ \{(x_\tau,
\frac{1}{\ell}\frac{1}{Q_\tau^{\mu}(\cdot| x_\tau)})\}_{\tau\in A} \cup
\{(x_\tau, \frac{-C_4}{2^{j-1}-1} \frac{1}{Q_\tau^{\mu}(\cdot |
  x_\tau)})\}_{\tau\in B}$. 
  
\citet{AgarwalHsKaLaLiSc14} showed that the optimization problem (OP) can be
solved by $\otil(1/\mu)$ oracle calls and the solution
has only $\otil(1/\mu)$ non-zero coordinates. Note that we only solve (OP) at the beginning of each block. Since there are $\order(\ln L)$ blocks in an epoch, the total oracle calls in an epoch is bounded by $\otil\big(\ln(L)/\mu \big) = \otil(\sqrt{LK})$, which amortizes to $\otil(S^\p\sqrt{LK}/T)$ per round if there are $S^\p$ epochs
(in Section~\ref{sec:implications} we relate $S'$ to $S$ or $\bvar$). 
%Summing up two cases, the required number of oracle calls is $\otil(S^\p \sqrt{LK}/{T}+1)$ per round in average.

We next present the interval regret guarantee of \AdaILTCB,
which improves from $\otil(L^\frac{2}{3})$ to $\otil(\sqrt{L})$ compared to \AdaEG
(see Appendix~\ref{app:AdaILTCB2} for the proof),
except that it holds for interval with total variation $\bvar_\calI$ (instead of reward variation $\var_\calI$) bounded by $v$
due to the fact that variation in the context is important for the variance control (Line \ref*{line:ILTCB:check_var}).

\begin{theorem}\label{thm:AdaILTCB2}
With probability at least $1 - \delta$, for any interval $\calI$ such
that $|\calI| \leq L$ and $\bar{\var}_\calI\leq v$, 
\AdaILTCB with parameters $L$, $v$, and $\delta$
guarantees for any $\pi \in \Pi$,
\[
\sum_{t \in \calI} r_t(\pi(x_t)) - r_t(a_t) \leq
\otil\left(\abs{\calI}v + \sqrt{LK\ln(N/\delta)}\right). 
\]
\end{theorem}

%%%%%%%%%%%%%%%%%%%%%%%%%%%%%%%%%
% later part (iffalse)
%%%%%%%%%%%%%%%%%%%%%%%%%%%%%%%%%
\iffalse
\subsection{\textsc{Ada-Greedy2}}
\textsc{Ada-Greedy2} proceeds in \textit{blocks}, which are the intervals indexed by $j$ in the algorithm. In block $j$, the learner follows the leader of blocks $1$ to $j-1$. Note that by the block length scheduling, $\sigma(t)$ is the first round of the block in which $t$ lies. Therefore, the learner always follows the leader in $[1,\sigma(t)-1]$, and we always have $\sigma(t)\geq \floor{\frac{t}{2}}+1$. We define an \textit{epoch} to be an interval between reruns. 
\begin{lemma}
\label{lemma:adagreedy2:no rerun}
If there is no distribution switch from time $1$ to $t$, and $t<L$, then with high probability, \textsc{Ada-Greedy2} will not rerun at time $t$. 
\end{lemma}
\begin{proof}
If the distribution does not change in the interval $[1,t]$, then $\bigabs{\avgR_{[1,\sigma(t)-1]}(\pi)-\avgR_{[\tau,t]}(\pi)}\leq \bigabs{\avgR_{[1,\sigma(t)-1]}(\pi)-\calR_{[1,\sigma(t)-1]}(\pi)}+\bigabs{\calR_{[1,\sigma(t)-1]}(\pi)-\calR_{[\tau,t]}(\pi)}+\bigabs{\calR_{[\tau,t]}(\pi)-\avgR_{[\tau,t]}(\pi)}\leq \beta_{[1,\sigma(t)-1]}+0+\beta_{[\tau,t]}$ for all $\tau$ with high probability by Freedman's inequality. Therefore, \test will not cause a rerun with high probability. 
\end{proof}
\begin{theorem}
With high probability, \textsc{Ada-Greedy2} achieves switching regret of order $\tilde{\mathcal{O}}( S^{\frac{1}{3}}T^{\frac{2}{3}} )$ if $L$ is set to $T/S$; $\tilde{\mathcal{O}}( S^{\frac{1}{2}}T^{\frac{2}{3}} )$ if $L$ is set to $T$. 
\end{theorem}
\begin{proof}
By Freedman's inequality, we have $\bigabs{\avgR_\calI(\pi)-\calR_\calI(\pi)}\leq \beta_{\calI}$ for all $\pi$ and $\calI$ with high probability. For simplicity, we assume that it indeed holds in the following analysis. 

Let $[s,t]$ be a stationary interval, and assume that at time $t$, rerun is not triggered. Then for all $\pi$, 
\begin{align*}
\calR_{t}(\pi)&=\calR_{[s,t]}(\pi) \\
&\leq \avgR_{[s,t]}(\pi) + \beta_{[s,t]} \tag{Freedman's inequality} \\
&\leq \avgR_{[1,\sigma(t)-1]}(\pi)+\beta_{[1,\sigma(t)-1]}+2\beta_{[s,t]} \tag{\textsc{StationaryTest}}\\
&\leq \avgR_{[1,\sigma(t)-1]}(\pi_t)+\beta_{[1,\sigma(t)-1]}+2\beta_{[s,t]} \tag{the optimality of $\pi_t$}\\
&\leq \avgR_{[s,t]}(\pi_t)+2\beta_{[1,\sigma(t)-1]}+3\beta_{[s,t]} \tag{\textsc{StationaryTest}}\\
&\leq \calR_{[s,t]}(\pi_t)+2\beta_{[1,\sigma(t)-1]}+4\beta_{[s,t]} \tag{Freedman's inequality}\\
&= \calR_{t}(\pi_t)+2\beta_{[1,\sigma(t)-1]}+4\beta_{[s,t]}.
\end{align*}
Therefore, in a stationary interval $[s,e]$ which is not interrupted by rerun and totally lies in an epoch $\calE$, the regret is bounded by
\begin{align*}
\sum_{t=s}^e \mathbb{E}_t\left[r_t(\pi(x_t))-r_t(a_t)\right]
&\leq \tilde{\mathcal{O}}\left( \sum_{t=s}^e \beta_{[1,\sigma(t)-1]}+\beta_{[s,t]}+K\mu_t \right)\\
&\leq \tilde{\mathcal{O}}\left( \sum_{t=s}^e \left(\beta_{[1,\sigma(t)-1]}+K\mu_t \right)+\abs{\calE}^{\frac{1}{6}}\sqrt{K(e-s+1)} + \sqrt{K}\abs{\calE}^{\frac{1}{3}} \right).\end{align*}
Now consider the regret in an epoch $\calE$. Let there be $S_\calE$ stationary intervals in it. Then the regret in $\calE$ is the sum of interval regret (we used Cauchy's inequality): 
\begin{align*}
\sum_{t\in\calE} \mathbb{E}_t\left[r_t(\pi_t^*(x_t))-r_t(a_t)\right] 
&\leq 1+\tilde{\mathcal{O}}\left( \sum_{t=1}^{\abs{\calE}-1} \left( \beta_{[1,\sigma(t)-1]}+K\mu_t \right) +\abs{\calE}^{\frac{1}{6}}\sqrt{KS_\calE \abs{\calE}} + \sqrt{K}S_\calE\abs{\calE}^{\frac{1}{3}} \right)\\
&\leq \tilde{\mathcal{O}}\left( \sqrt{K} \left(S_\calE^{\frac{1}{2}} \abs{\calE}^{\frac{2}{3}} + S_\calE \abs{\calE}^{\frac{1}{3}} \right) \right).
\end{align*}
Finally, we sum up the regret in all epochs. Note that the number of epochs $S^\p$ is less or equal to $S+\frac{T}{L}$ because of Lemma~\ref{lemma:adagreedy2:no rerun}. Let the epochs be $\calE_1, \calE_2, \ldots, \calE_{S^\p}$, and in epoch $\calE_i$, there are $S_i$ stationary distributions. Then we have $S_1+\ldots+S_{S^\p}\leq S+S^\p\leq 2S+\frac{T}{L}$. Therefore the total regret is bounded by 
\begin{align*}
\tilde{\mathcal{O}}\left( \sqrt{K}\sum_{i=1}^{S^\p} \left(S_i^{\frac{1}{2}}\abs{\calE_i}^{\frac{2}{3}} + S_i\abs{\calE_i}^{\frac{1}{3}}\right) \right)\leq \tilde{\mathcal{O}}\left( \sqrt{K}L^{\frac{1}{6}}\left(S+\frac{T}{L}\right)^{\frac{1}{2}}T^{\frac{1}{2}} +\sqrt{K}L^{\frac{1}{3}}\left(S+\frac{T}{L}\right) \right). 
\end{align*}
We can get the corresponding bounds by making $L=T/S$ or $L=T$. 
\end{proof}

\subsection{\textsc{Ada-Greedy3} under switching distributions}
\begin{algorithm}[t]
\DontPrintSemicolon
\caption{\textsc{Ada-Greedy3}}\label{alg:AdaGreedy3}
{\bf Input}: allowed failure probability $\delta$. \\ 

{\bf Define}:
$\gamma=\frac{1}{2}$, 
$\theta=\frac{1}{2}$,  
$\mu_t=\min\Big\{\frac{1}{K}, t^{-\frac{1}{3}}\sqrt{\frac{\ln(N/\delta)}{K}}\Big\}, 
\beta_\calI= 2\sqrt{\frac{\ln(4T^2N/\delta)}{\mu_\calI \abs{\calI}}}+\frac{\ln(4T^2N/\delta)}{\mu_\calI \abs{\calI}}, 
\text{ where } \mu_{\calI}\triangleq \min_{t\in\calI} \mu_t, 
\sigma(t)=2^{\floor{\log_2 t} }, 
\alpha_\calI=2\sqrt{\frac{K\ln(4T^2N/\delta)}{\abs{\calI}}}+\frac{K\ln(4T^2N/\delta)}{\abs{\calI}}$.\\
{\bf Initialize}: $t=1$. \label{line:rerun_beginning3}\\ 
\For{$j=1, 2, \ldots$}{
       $E\leftarrow 2^{j-1}$.\\
       In the following $E$ rounds, play in bins, each of length $E^\gamma$. Let $b$ denote the bin index. \\
       \For{$b=1, 2, \ldots, E^{1-\gamma}$}{
            Make bin $b$ an \textit{exploration bin} with probability $b^{-\theta}$; otherwise an \textit{exploitation bin}. \\
             \For{$\tau=1, \ldots, E^\gamma$}{
                     Let $\pi_t=\argmax_{\pi\in\Pi}\avgR_{[1,\sigma(t)-1]}(\pi)$. \\
                     Set $p_t(a)=\begin{cases}
                             \frac{1}{K}, &\text{if bin $b$ is an exploration bin}, \\
                             \mu_t+(1-K\mu_t)\one \{ a=\pi_t(x_t) \},    &\text{if bin $b$ is an exploitation bin}.
                     \end{cases}
                     $\\
                     Play $a_t\sim p_t$. \\
                     Run \textbf{\textsc{StationaryTest}}$(t)$ if bin $b$ is a exploration bin. \\
                     %Run \textbf{\textsc{StationaryTest2}}$(t)$ if bin $b$ is a exploitation bin. \\
                     Goto Line \ref{line:rerun_beginning3} if it fails.   
                     \\$t\leftarrow t+1$.   
            }
            
       }
}

$\textbf{\textsc{StationaryTest}}(t)$\\
Check if there exist $\tau\leq t$ with $[\tau, t]$ \textit{totally lies in the current bin}, and $\pi\in\Pi$ such that
\begin{equation*}
\Big\lvert \avgR_{[1, \sigma(t)-1]}(\pi)-\avgR_{[\tau,t]}(\pi)\Big\rvert > \beta_{[1, \sigma(t)-1]}+2\alpha_{[\tau,t]}, 
\end{equation*}
\\Return \textit{fail} if such $\tau$ and $\pi$ exist. \\
(To reduce complexity, one can only check for $\tau=t-1, t-2, t-4, \ldots$) 

%$\textbf{\textsc{StationaryTest2}}(t)$\\
%Check if there exist $\tau\leq t$ and $\pi\in\Pi$ such that
%\begin{equation*}
%\Big\lvert \avgR_{[1, \sigma(t)-1]}(\pi)-\avgR_{[\tau,t]}(\pi)\Big\rvert > \beta_{[1, %\sigma(t)-1]}+\beta_{[\tau,t]}, 
%\end{equation*}
%\\Return \textit{fail} if such $\tau$ and $\pi$ exist. 
\end{algorithm}
\fi
%%%%%%%%%%%%%%%%%%%%%%%%%%%%%%%%%%%%%%
%%%% fi 
%%%%%%%%%%%%%%%%%%%%%%%%%%%%%%%%%%%%%%%

\section{Estimating the Performance of $k$-Nearest Neighbor Algorithms}
\label{sec:neighbor}
%\subsection{The $k$-Nearest Neighbor Algorithm}
In this section, we develop estimators for estimating the performance of $k$-Nearest Neighbor ($\knn$) algorithms \citep{FH51,FH89,CH67}. 

Let $\calD$ be a distribution on a ground set $X$. Suppose every point $x\in X$ has a (true) label $f(x)\in\{0,1\}$. In addition, we have a distance metric $\D :X\times X\rightarrow \mathbb{R}_{\geq 0}$ that is symmetric, nonnegative and satisfies the triangle inequality. The \emph{$k$-Nearest Neighbor algorithm with soft predictions} ($\knnsoft$) is given a pool $S$ of unlabeled examples, sampled iid from $\calD$, and for any input $x\in X$, finds its $k$ nearest examples $x_1,x_2,\cdots, x_k\in S$ with respect to the distance metric $\D$ and outputs $\hat{f}(x)=\frac{1}{k}\sum\limits_{i=1}^kf(x_i)$ as an approximation of $f(x)$. In this paper, we assume the $k$ nearest examples are calculated by an oracle $M$, i.e., when given $x$ and $S$, $M$ calculates the $k$ nearest examples to $x$ in $S$. There may be ties when distances to $x$ are compared and we assume $M$ breaks ties according to some (probably random) mechanism.

The \emph{$k$-Nearest Neighbor algorithm with hard predictions} ($\mv$) does the same thing as $\knnsoft$, except that $\hat{f}(x)$ is chosen as the majority vote $I[\frac{1}{k}\sum\limits_{i=1}^kf(x_i)>0.5]$.\footnote{$I[\cdot]$ is the indicator function of a statement, which takes value 1 if the statement is true and value 0 if the statement is false.}

For both algorithms, we use $\err(x)=|\hat f(x)-f(x)|$ to denote the $L^1$ error on point $x\in X$.  For soft prediction, we will penalize the algorithm by taking the $p$th power of the $L^1$ error for positive integer $p$.




\subsection{Estimating the Performance of $\knnsoft$}
\label{subsec:knnsoft}
Given a loss function $\loss(\cdot)$, we can measure the performance of $\knnsoft$ by its expected loss $\mathbb E_{x}[\loss(\err(x))]$. The expectation is over the random draw of $x$ with respect to distribution $\calD$ and the randomness of the oracle $M$ when ties occur. In this paper, we focus on the $p$th-power loss $\mathbb E_{x}[(\err(x))^p]$ for positive integer $p$. Let $\tsoft_{\calD} (f,\epsilon,S,k)$ denote the task of estimating the expected loss of a $\knnsoft$ algorithm up to an additive error $\epsilon$ with success probability at least $\frac{2}{3}$. We consider the estimation task in the active model, in which the estimator is only allowed to query labels of examples in an unlabeled pool sampled iid from $\calD$. In addition to the given unlabeled pool $S$ from which $\knnsoft$ would learn, we allow the $\tsoft_{\calD} (f\ac,\epsilon,S,k)$ estimator to sample fresh unlabeled examples and query their labels. We assume the estimator has access to the oracle $M$.%The task of $\knn(f,\epsilon,N)$ is to approximate the expected loss of the $\knn$ algorithm when the true labels are $f$ and the training set has size $N$ within additive error $\epsilon$. When showing upper bound results, we assume the tie breaking mechanism $M$ is given as input and when showing lower bound results, we assume the $\knn(f,\epsilon,N)$ algorithm can assume any $M$ it likes.
\begin{theorem}
\label{thm:pth}
Suppose we consider the $p$th-power loss for $p\in\mathbb N^*$. There is an estimator $\tsoft_{\calD}(f\ac,\epsilon,S,k)$ using $O(\frac{p}{\epsilon^2})$ queries on $N+O(\frac{1}{\epsilon^2})$ unlabeled examples when the unlabeled pool $S$ has size $N$. The underlying distribution $\calD$ is assumed unknown to the estimator. Moreover, the estimator has success probability at least $\frac{2}{3}$ for \emph{any} unlabeled pool $S$.
\end{theorem}
The proof of Theorem \ref{thm:pth} is in Appendix \ref{sec:pthproof}. We will show (Theorem \ref{thm:pthpowerlower} in Appendix \ref{sec:lower}) that the $O(\frac{p}{\epsilon^2})$ query complexity is optimal.

\subsection{Finding an Approximately-Best Choice of $k$}
\label{subsec:bestk}
Based on the result in Section \ref{subsec:knnsoft}, we are able to construct an algorithm that approximately optimizes the choice of $k$ in the $\knnsoft$ algorithm.

Suppose we have active access to the true label $f$ with respect to distribution $\calD$ over ground set $X$ with distance metric $\D$. Suppose the size of the unlabeled pool $S$ is fixed to be $N$. We use $\mathrm{loss}_k$ to denote the expected loss of the $\knnsoft$ algorithm and consider how the $\knnsoft$ algorithm performs with different values of $k$. We assume the oracle $M$ uses the same tie-breaking mechanism for different values of $k$. Specifically, given $x$ and $S$, $M$ arranges the examples in $S$ as $x_1,x_2,\cdots,x_N$ so that $\forall i,\D(x_i,x)\leq \D(x_{i+1},x)$. $x_1,x_2,\cdots,x_k$ are taken by $\knnsoft$ as the $k$ nearest neighbors of $x$ for any $k\in\{1,2,\cdots,N\}$.


We say $k$ is $\epsilon$-approximately-best, if $\forall k'\in\{1,2,\cdots, N\},\mathrm{loss}_{k'}\geq \mathrm{loss}_{k}-\epsilon$. The following theorem states that we can find an $\epsilon$-approximately-best $k$ using a small number of queries. The proof of the theorem is in Appendix \ref{sec:bestkproof}.
\begin{theorem}
\label{thm:bestk}
Suppose $\knnsoft$ algorithms with an unlabeled pool $S$ of size $N$ are measured by $p$th-power loss for $p\in\mathbb N^*$. Suppose $\epsilon\in(0,\frac{1}{2})$. There is an algorithm that finds an $\epsilon$-approximately-best $k$ w.p.\ at least $\frac{2}{3}$ using $O(\frac{p^2\log N}{\epsilon^3}(\log\log N+\log p+\log\frac{1}{\epsilon}))$ queries on $N+O(\frac{p\log N}{\epsilon^3}(\log\log N+\log p+\log\frac{1}{\epsilon}))$ unlabeled examples.
\end{theorem}

\subsection{Estimating the Performance of $\mv$}
\label{subsec:mv}
The performance of $\mv$ is naturally measured by its error rate $\mathbb E_{x}[\err(x)]$ and we use $\thard_{\calD}(f,\epsilon,S,k)$ to denote the corresponding estimation task of estimating the error rate of $\mv$ up to an additive error $\epsilon$ with success probability at least $\frac{2}{3}$.

A trivial estimator achieving this goal using $O(\frac{k}{\epsilon^2})$ queries on $N+O(\frac{1}{\epsilon^2})$ unlabeled examples is to use the empirical mean of $\err(x)$ as an estimator of $\mathbb E_{x}[\err(x)]$. %The algorithm runs $O(\frac{1}{\epsilon^2})$ iterations and in each iteration, the algorithm samples a fresh unlabeled example $x$ from $\calD$ and then uses $k+1$ queries to determine whether $\mv$ makes the correct prediction on $x$. The algorithm outputs the empirical mean of the errors of the $O(\frac{1}{\epsilon^2})$ iterations as an estimate of the error rate of $\mv$. 
This estimator is not satisfactory because its query complexity grows with respect to $k$. In Appendix \ref{sec:lower}, we show (Theorem \ref{thm:mv}) that this linear growth with respect to $k$ can't be eliminated. 
%Also, we will show (Theorem \ref{thm:reductionfrombandit}) that the $O(\frac{k}{\epsilon^2})$ query complexity is optimal if we assume a natural algorithm for \emph{approximating the fraction of good arms} ($\aga$) in the stochastic multi-armed bandit setting has the optimal query complexity. Before we show our lower bound results, we first define the problems of \emph{counting and approximating the number of good arms}.



% Acknowledgments---Will not appear in anonymized version
\acks{This work was supported in part by the National Science Foundation under grants CCF-1525971 and CCF-1800317.}


\bibliography{references}


\appendix
\section{Tolerant Passive Testing Implies Refutation}
\label{sec:refutation}
In this section, we consider a class $\calC$ over domain $X$ with VC-dimension $d$. We are going to build a refutation algorithm (see Section \ref{subsec:passive} for definition) with margin $\epsilon\in(0,\frac 12)$ for a distribution $\calD$ by calling a tolerant passive tester over arbitrary unknown distributions with sample complexity $s$ as oracle.

\begin{lemma}
There exist universal positive constants $c_1,c_2$ satisfying the following property. Assume $\calD$ satisfies that with probability at least $\frac {11}{12}$ no point appears twice in an $s'$-sized i.i.d.\ sample from $\calD$ for $s'=\lceil\max\{\frac{{c_1}d}{\epsilon^2}\log\frac1\epsilon,c_2s\}\rceil$. Suppose there exists a tolerant passive tester $\calA$ for $\calC$ over arbitrary unknown distribution with threshold $\alpha=\frac 12-\frac{3\epsilon}4$, margin $\frac\epsilon 2$ and sample complexity $s$. Then there exists a refutation algorithm $\calB$ for $\calC$ over $\calD$ with margin $\epsilon$ and sample complexity $c_2s$.
\end{lemma}
\begin{proof}
Algorithm $\calB$ first obtains a $c_2s$-sized sample $S$ of example label pairs $\{(x_i,y_i)\}$ and declares failure if there exists $y_i\neq y_j$ for $x_i=x_j$. If the algorithm does not declare a failure, it then treat $y_i=f(x_i)$ for some function $f$ and calls $\calA$ to distinguish whether $f$ is $\alpha$-close to $\calC$ or $f$ is $(\alpha+\frac\epsilon 2)$-far from $\calC$ using sample $S$. Here, the success probability of $\calA$ is boosted to at least $\frac{11}{12}$ by repeating $c_2$ times. $\calB$ accepts if $\calA$ accepts and $\calB$ rejects if $\calA$ rejects.

%To show the correctness of the algorithm, we first enlarge $S$ to a $\lceil\frac{{c_1}d}{\epsilon^2}\log\frac1\epsilon\rceil$-sized sample $S'$. With probability at least $\frac 56$, all the $x_i$ are distinct. 

Now, we show the correctness of the algorithm. We first consider the case that every $(x_i,y_i)$ is i.i.d.\ from a distribution $\calD'$ with marginal on $X$ being $\calD$ and $\exists g\in\calC,\Pr_{(x',y')\sim\calD}[g(x')\neq y']\leq\frac 12-\epsilon$. We enlarge $S$ to an $s'$-sized sample $S'$. Equivalently, we can imagine $S'$ is chosen i.i.d.\ from $\calD'$ first and $S$ is an i.i.d.\ sample from the uniform distribution $\calU$ over $S'$. With probability at least $\frac {11}{12}$, all the $x_i$ in $S'$ are distinct. For every $x\in X$, we define $\error(x)=\Pr_{(x',y')\sim\calD'}[g(x')\neq y'|x'=x]$ and we have $\mathbb E_{x\sim\calD}[\error(x)]\leq\frac 12-\epsilon$. According to the Chernoff Bound, with probability at least $\frac{11}{12}$, $\mathbb E_{x\sim \calU}[\error(x)]\leq\frac 12-\frac {7\epsilon}8$. By the Union Bound, with probability at least $\frac 56$, all $x_i$ in $S'$ are distinct \emph{and} $\mathbb E_{x\sim \calU}[\error(x)]\leq\frac 12-\frac {7\epsilon}8$. Conditioned on that, every $y_i$ is independent from others and thus by the Chernoff Bound, with probability at least $\frac {11}{12}$ we have $\Pr_{(x,y)\sim \calU}[g(x)\neq y]\leq\frac 12-\frac {3\epsilon}4$. If we unwrap the conditional probability of at least $\frac 56$, we know with probability at least $\frac 56\cdot\frac{11}{12}$ that 
\begin{enumerate}
\item all $x_i$ in $S'$ are distinct;
\item $\Pr_{(x,y)\sim \calU}[g(x)\neq y]\leq\alpha$.
\end{enumerate}
Now, if we sample $S$ i.i.d.\ from $\calU$ and feed it to $\calA$, we know $\calA$ accepts with probability at least $\frac{11}{12}$. Therefore, $\calB$ accepts with probability at least $\frac{5}{6}\cdot\frac{11}{12}\cdot\frac{11}{12}\geq \frac 23$.

Next, we consider the case that every $y_i$ is i.i.d.\ uniformly chosen from $\{0,1\}$. Again, we enlarge $S$ to an $s'$-sized sample $S'$ and imagine $S'$ is chosen i.i.d.\ from $\calD'$ first and $S$ is an i.i.d.\ sample from the uniform distribution $\calU$ over $S'$. With probability at least $\frac {11}{12}$, all the $x_i$ in $S'$ are distinct. Conditioned on that, by the Chernoff Bound, for any function $g$, we have $\Pr_{(x,y)\sim\calU}[g(x)\neq y]\geq \frac 12-\frac\epsilon 4=\alpha+\frac\epsilon 2$ with probability at least $1-e^{-\epsilon^2s'/8}= 1-\epsilon^{c'd/8}$ for some $c'\geq c_1$ satisfying $s'=\frac{c'd}{\epsilon^2}\log\frac 1\epsilon$. By Sauer's Lemma, the number of different $g$ over the chosen $x_i$ in $S'$ is at most $(\frac{es'}{d})^d=(\frac{c'e}{\epsilon^2}\log\frac{1}{\epsilon})^d$. Note that when $c_1$ is sufficiently large, we always have $\epsilon^{c'd/8}\cdot (\frac{c'e}{\epsilon^2}\log\frac{1}{\epsilon})^d\leq \frac 16$. Therefore, by the Union Bound, with probability at least $\frac 56$, we have $\forall g\in\calC,\Pr_{(x,y)\sim\calU}[g(x)\neq y]\geq\alpha+\frac\epsilon 2$. If we unwrap the conditional probability of at least $\frac{11}{12}$, we know with probability at least $\frac{11}{12}\cdot \frac{5}{6}$ that
\begin{enumerate}
\item all $x_i$ in $S'$ are distinct;
\item $\forall g\in\calC,\Pr_{(x,y)\sim\calU}[g(x)\neq y]\geq\alpha+\frac\epsilon 2$.
\end{enumerate}
Again, if we sample $S$ i.i.d.\ from $\calU$ and feed it to $\calA$, we know $\calA$ rejects with probability at least $\frac{11}{12}$. Therefore, $\calB$ rejects with probability at least $\frac{5}{6}\cdot\frac{11}{12}\cdot\frac{11}{12}\geq \frac 23$.
\end{proof}
\section{Distance Approximation for Disjoint Unions of Properties}
\label{sec:union}
In this section, we extend the theorem of \citet{BBBY12} that disjoint unions of testable properties are testable from non-tolerant testing to tolerant testing. 
%\subsection{Active Distance Approximation}
%Distance approximation is a natural generalization of tolerant testing. Given a distribution $\calD$ and a concept class $\calC$ on ground set $X$, the distance of a function $f\in\{0,1\}^X$ to $\calC$ is defined to be $\dist_{\calD}(f,\calC)=\inf\limits_{g\in\calC}\dist_{\calD}(f,g)$. A distance approximation algorithm takes a function $f\in\{0,1\}^X$ and a parameter $\epsilon$ as input, and outputs an estimation $\widehat\dist_{\calD}(f,\calC)$ which is in $[\dist_{\calD}(f,\calC)-\epsilon,\dist_{\calD}(f,\calC)+\epsilon]$ with probability at least $\frac{2}{3}$. The parameter $\epsilon$ is called the \emph{additive error}.

%For an active distance approximation algorithm, similar to an active tolerant testing algorithm, the input function $f$ is given through active access. The success probability can be boosted to $1-\delta$ for any $\delta$ by repeating the algorithm $O(\log\frac{1}{\delta})$ times and choosing the median.


We first introduce the definition of disjoint unions of properties in \citep{BBBY12}. Suppose the ground set $X$ is partitioned as a disjoint union $\bigcup\limits_{i=1}^m X_i$. On every $X_i$, there is a property (concept class) $\calC_i\neq\emptyset$. The disjoint union of these properties is defined to be $\calC=\{f\in\{0,1\}^X:\forall 1\leq i\leq m,f|_{X_i}\in\calC_i\}$.

Let $\calD$ be a distribution over $X$. Suppose the conditional distribution of $\calD$ on $X_i$ is denoted by $\calD_i$ and the probability $\Pr_{x\sim\calD}[x\in X_i]$ is denoted by $p_i$.

\begin{theorem}
Suppose $\epsilon\in(0,\frac{1}{2})$. Suppose for every $1\leq i\leq m$, there is an active tolerant tester $\calA$ for $\calC_i$ over $\calD_i$ with additive error $\frac\epsilon 2$ using at most $q$ queries on $N$ unlabeled examples. Then, there is an active tolerant tester $\calB$ for $\calC$ over $\calD$ with additive error $\epsilon$ using at most $O(\frac{q}{\epsilon^2}\log\frac{1}{\epsilon})$ queries on $O(\frac{mN}{\epsilon}\log\frac{1}{\epsilon})$ unlabeled examples. If tester $\calA$ can perform on unknown distributions, then tester $\calB$ can also perform on unknown distributions, though we need extra $O(\frac{1}{\epsilon^2})$ unlabeled examples.


%Suppose there is an $\frac{\epsilon}{2}$-distance approximation oracle using at most $q$ queries on at most $U$ unlabeled samples. Then, the distance $\dist_{\mathcal{D}}(f,\mathcal{C})$ can be approximated with success probability at least $\frac{2}{3}$ within additive error $\epsilon+\delta$ using $O(\frac{q}{\delta^2}\log\frac{1}{\delta})$ queries on $O(\frac{mU}{\delta}\log\frac{1}{\delta})$ unlabeled samples. Furthermore, if $\forall 1\leq i\leq m,p_i=\frac{1}{m}$, then the unlabeled sample complexity can be reduced to $O(mU\log\frac{1}{\delta})$.
\end{theorem}



\begin{proof}
Tester $\calB$ is constructed as follows. The tester chooses $s=O(\frac{1}{\epsilon^2})$, receives an unlabeled pool of size $O(\frac{mN}{\epsilon}\log s)$ and independently chooses $s$ indices $i_1,i_2,\cdots,i_s$ from $\{1,2,\cdots,m\}$ according to distribution $\{p_i\}_{1\leq i\leq m}$. This can be achieved by looking at on which $X_i$'s the extra $s$ unlabeled examples are, when the distribution $\calD$ is unknown. Then for each $1\leq j\leq s$, if there are enough $(O(N\log s))$ unlabeled examples lying in $X_{i_j}$, the tester repeats $\calA$ for $O(\log s)$ times to calculate an estimator $\widehat\dist_{i_j}$ of the distance from $f$ to $\calC$ on $\calD_{i_j}$ up to an additive error $\frac{\epsilon}{2}$ with success probability at least $1-\frac{1}{9s}$;\footnote{Repeat tester $\calA$ $O(\log s)$ times and take the median to boost its success probability to at least $1-\frac{1}{9s}$.} otherwise, define $\widehat\dist_{i_j}=0$. The final output of tester $\calB$ is $\frac{1}{s}\cdot\sum\limits_{j=1}^s\widehat\dist_{i_j}$.

To prove the correctness of the above tester, we first define $\dist_i:=\inf\limits_{g\in\calC}\dist_{\calD_i}(f,g)$ and $\dist:=\inf\limits_{g\in\calC}\dist_{\calD}(f,g)$. Note that $\dist=\sum\limits_{i=1}^mp_i\dist_{i}$. 

For every $1\leq i\leq m$, we further define $\dist_i'=\left\{\begin{array}{ll}\dist_i,&\text{if $p_i\geq\frac{\epsilon}{4m}$}\\0,&\text{if $p_i<\frac{\epsilon}{4m}$}\end{array}\right.$ and $\dist_i''=\left\{\begin{array}{ll}\dist_i,&\text{if $p_i\geq\frac{\epsilon}{4m}$}\\1,&\text{if $p_i<\frac{\epsilon}{4m}$}\end{array}\right.$. Then $\dist-\frac{\epsilon}{4}\leq \sum\limits_{i=1}^mp_i\dist'\leq\sum\limits_{i=1}^mp_i\dist''\leq\dist+\frac{\epsilon}{4}$. By the Chernoff Bound, $s=O(\frac{1}{\epsilon^2})$ is enough to make sure with probability at least $1-\frac{1}{9}$ that $\dist-\frac{\epsilon}{2}<\frac{1}{s}\sum\limits_{j=1}^s\dist'_{i_j}\leq\frac{1}{s}\sum\limits_{j=1}^s\dist''_{i_j}<\dist+\frac{\epsilon}{2}$.

Note that the unlabeled pool has size $O(\frac{mN}{\epsilon}\log s)$, which is enough to make sure that with probability at least $1-\frac{1}{9}$, for every $i_j$ with $p_{i_j}\geq\frac{\epsilon}{4m}$, there are enough ($O(N\log s)$) unlabeled examples lying in $X_{i_j}$. Therefore, with probability at least $(1-\frac{1}{9})(1-s\cdot\frac{1}{9s})\geq 1-\frac{2}{9}$, for all $i_j$ such that $p_{i_j}\geq \frac{p}{4m}$, it holds that $|\widehat \dist_{i_j}-\dist_{i_j}|\leq\frac{\epsilon}{2}$.

Finally, by the Union Bound, we know with probability at least $1-\frac{1}{3}$, it holds that $\dist-\epsilon<\frac{1}{s}\sum\limits_{j=1}^s\dist'_{i_j}-\frac{\epsilon}{2}\leq \frac{1}{s}\sum\limits_{j=1}^s\widehat\dist_{i_j}\leq\frac{1}{s}\sum\limits_{j=1}^s\dist''_{i_j}+\frac{\epsilon}{2}<\dist+\epsilon$.
\end{proof}
%Then, with probability at least $1-\frac{1}{6}$, we have $|\frac{1}{s}\sum\limits_{j=1}^s\dist_{i_j}'-\dist|<\frac{\epsilon}{2}$.
 %If $p_i<\frac{\delta}{2m}$ for some $i$, we conceptually re-define $\dist_i=0$. This will result in an additive change of $\dist_{\mathcal{D}}(f,\mathcal{C})$ by at most $\frac{\delta}{2}$. The algorithm first independently chooses $s=O(\frac{1}{\delta^2})$ indices $i_1,i_2,\cdots,i_s$ from $\{1,2,\cdots,m\}$ according to distribution $\{p_i\}_{1\leq i\leq m}$. By Chernoff Bound, with probability at least $\frac{5}{6}$, $|\dist_{\calD}(f,\mathcal{C})-\frac{1}{s}\sum\limits_{j=1}^s\dist_{i_j}|<\frac{\delta}{2}$. The algorithm then requests for $O(\frac{mU}{\delta}\log\frac{1}{\delta})$ ($O(mU\log\frac{1}{\delta})$ when all $p_i$'s are equal) unlabeled samples. By Chernoff Bound and Union Bound, with probability at least $\frac{5}{6}$, $\forall 1\leq j\leq s$, if $p_{i_j}\geq\frac{\delta}{2m}$, then there are $O(U\log\frac{1}{\delta})$ unlabeled samples in cluster $i_j$. Then, for every such $j$, with error probability at most $\frac{1}{6s}$, the algorithm can approximate $\dist_{i_j}$ by $\widehat{\dist}_{i_j}$within additive error $\epsilon$ using $O(q\log\frac{1}{\delta})$ queries on these $O(U\log\frac{1}{\delta})$ unlabeled samples. Finally, the algorithm outputs $\widehat{\dist}_{\calD}(f,\mathcal{C}):=\frac{1}{s}\sum\limits_{j=1}^s\widehat{\dist}_{i_j}$ as an approximation of $\dist_{\calD}(f,\mathcal{C})$. By Union Bound, the failure probability of the algorithm is bounded by $\frac{1}{6}+\frac{1}{6s}\cdot s=\frac{1}{3}$ and the total additive error is bounded by $\frac{\delta}{2}+\frac{\delta}{2}+\epsilon=\epsilon+\delta$.

\section{Proof of Lemma \ref{thm:additive}}
\label{sec:compositionproof}
Before proving Lemma \ref{thm:additive}, we first show a simple claim about compositions with truncation.
\begin{claim}
\label{lm:truncation}
Suppose the distribution $\calD$ is semi-uniform. We have $\property_{\calD}^t(d,\alpha)\subseteq\property_{\calD}(d,\alpha)\subseteq\property_{\calD}^t(d,\alpha+\frac{d}{tm})$.
\end{claim}
\begin{proof}
$\property_{\calD}^t(d,\alpha)\subseteq\property_{\calD}(d,\alpha)$ is obvious. To see $\property_{\calD}(d,\alpha)\subseteq\property_{\calD}^t(d,\alpha+\frac{d}{tm})$, we note that for any $g\in\property(d)$, for each $i$ such that $k_i>t$, substituting a function in $\calC_i^0$ for $g|_{X_i}$ causes at most a $\frac{1}{m}$ increase in the distance from $f\in\property_\calD(d,\alpha)$ to $g$. An easy observation that $|\{i:k_i>t\}|\leq\frac{d}{t}$ given $\sum\limits_{i=1}^mk_i\leq d$ completes the proof. 
\end{proof}

\begin{proof}(of Lemma \ref{thm:additive})
The algorithm first picks indices $1\leq i_1<i_2<\cdots i_l\leq m$ uniformly at random for $l=O(\frac{1}{\epsilon\mu^2}+\frac{1}{\epsilon^2})$. Then the algorithm asks for $O(\frac{Nm}{l})$ unlabeled examples to make sure with probability at least $\frac{11}{12}$, there are at least $N$ examples lying in $\bigcup\limits_{j=1}^lX_{i_j}$. These examples can be treated as drawn independently at random according to $\calD_{\mathbf i}$, where $\mathbf i=(i_1,i_2,\cdots,i_l)$. Finally, the algorithm calls the oracle to approximate the distance from $f$ to $\property^{t}((1+\frac{\mu}{2})\lambda l)$ truncated by $t=\frac{4\lambda}{\epsilon}$ on distribution $\calD_{\mathbf i}$ up to an additive error $\frac{\epsilon}{2}$ using these unlabeled examples and outputs what the oracle outputs. 

The correctness of the algorithm follows from the following two lemmas (with proofs in the appendices) and the Union Bound.
\end{proof}

\begin{lemma}
\label{lm:in}
Suppose $t=\frac{4\lambda}{m}$. If $f\in\property_{\calD}(\lambda m,\alpha)$, then choosing $l=O(\frac{1}{\epsilon\mu^2}+\frac{1}{\epsilon^2})$ is enough to make sure that with probability at least $\frac{5}{6}$, $f\in\property^{t}_{\calD_{\mathbf i}}((1+\frac{\mu}{2})\lambda l,\alpha+\frac{\epsilon}{2})$.
\end{lemma}

\begin{lemma}
\label{lm:notin}
Suppose $t=\frac{4\lambda}{m}$. If $f\notin\property_{\calD}((1+\mu)\lambda m,\alpha)$, then choosing $l=O(\frac{1}{\epsilon\mu^2}+\frac{1}{\epsilon^2})$ is enough to make sure that with probability at least $\frac{5}{6}$, $f\notin\property^{t}_{\calD_{\mathbf i}}((1+\frac{\mu}{2})\lambda l,\alpha-\frac{\epsilon}{2})$.
\end{lemma}


\begin{proof}(of Lemma \ref{lm:in})

By the choice of truncation $t=\frac{4\lambda}{\epsilon}$, according to Claim \ref{lm:truncation}, we know $f\in\property^t_{\calD}(\lambda m,\alpha+\frac{\epsilon}{4})$. Suppose $\dist_{\calD}(f,g)\leq \alpha+\frac{\epsilon}{4}$ for some $g\in\property^t(\lambda m)$. According to the Multiplicative Chernoff Bound for sampling without replacement, choosing $l=O(\frac{1}{\epsilon\mu^2})$ is enough to make sure that with probability at least $\frac{11}{12}$, $\exists g'$ s.t.\ $g'\in\property^{t}((1+\frac{\mu}{2})\lambda l)$ and $\dist_{\calD_{\mathbf i}}(g,g')=0$.\footnote{$g'$ is chosen such that $g'|_{X_{i}}\in\calC_{i}^0$ for all $i\notin \{i_1,i_2,\cdots,i_l\}$ and $g'|_{X_{i}}=g|_{X_{i}}$ for all $i\in\{i_1,i_2,\cdots,i_l\}$. The fact that the $k_i$'s of $g$ are bounded between 0 and $t=\frac{4\lambda}{\epsilon}$ allows us to use the Multiplicative Chernoff Bound.} According to the Chernoff Bound for sampling without replacement, choosing $l=O(\frac{1}{\epsilon^2})$ is enough to make sure that with probability at least $\frac{11}{12}$, $\dist_{\calD_{\mathbf i}}(f,g)\leq\alpha+\frac{\epsilon}{2}$. By the Union Bound, these two events happen at the same time with probability at least $\frac{5}{6}$, and in this case, $f\in\property^{t}_{\calD_{\mathbf i}}((1+\frac{\mu}{2})\lambda l,\alpha+\frac{\epsilon}{2})$. 
\end{proof}

\begin{proof}(of Lemma \ref{lm:notin})
According to Claim \ref{lm:truncation}, we know $f\notin\property^t_{\calD}((1+\mu)\lambda m,\alpha)$. Therefore, by definition, there exists $g\in\property^t((1+\mu)\lambda m)$ with the following two properties:\footnote{E.g., choose $g$ to be the closest or approximately-closest function in the class to $f$.  Note that $\property^t((1+\mu)\lambda m)$ can't be empty, because $\property^t((1+\mu)\lambda m)\supseteq\property^t(0)=\property(0)\neq\emptyset$.}
\begin{enumerate}
\item $\dist_{\calD}(f,g)>\alpha$;
\item $\forall g'\in \property^t((1+\mu)\lambda m),\dist_{\calD}(f,g')>\dist_{\calD}(f,g)-\frac{\epsilon}{4}\cdot\frac{l}{m}$.
\end{enumerate}
Suppose $g|_{X_i}\in\calC_{i}^{k_i}$ for $k_i\leq t=\frac{4\lambda}{\epsilon}$ satisfying $k:=\sum\limits_{i=1}^mk_i\leq(1+\mu)\lambda m$. We enlarge $k_i$ to $k'_i\in[k_i,t]$ to make sure that $k':=\sum\limits_{i=1}^mk'_i=(1+\mu)\lambda m$.\footnote{$k'_i$ doesn't have to be an integer. Also note that $mt=\frac{4\lambda}{\epsilon}\cdot m>4\lambda m>(1+\mu)\lambda m$.} According to the Multiplicative Chernoff Bound for sampling without replacement, choosing $l=O(\frac{1}{\epsilon\mu^2})$ is enough to make sure that with probability at least $\frac{11}{12}$, $\sum\limits_{j=1}^lk'_{i_j}\geq (1+\frac{\mu}{2})\lambda l$.

Now suppose it's the case that $\sum\limits_{j=1}^lk'_{i_j}\geq (1+\frac{\mu}{2})\lambda l$. Then, according to the second property of $g$, we know $$\forall g'\in\property^{t}((1+\frac{\mu}{2})\lambda l),\dist_{\calD_{\mathbf i}}(f,g')> \dist_{\calD_{\mathbf i}}(f,g)-\frac{\epsilon}{4}.$$ Otherwise, we can swap $g'$ for $g$ on $\bigcup\limits_{j=1}^lX_{i_j}$ causing a violation of the second property of $g$.

Finally, according to the Chernoff Bound for sampling without replacement, choosing $l=O(\frac{1}{\epsilon^2})$ is enough to make sure that with probability at least $\frac{11}{12}$, $\dist_{\calD_{\mathbf i}}(f,g)>\alpha-\frac{\epsilon}{4}$. Therefore, by the Union Bound, with probability at least $\frac{5}{6}$, $$\forall g'\in\property^{t}((1+\frac{\mu}{2})\lambda l),\dist_{\calD_{\mathbf i}}(f,g')> \dist_{\calD_{\mathbf i}}(f,g)-\frac{\epsilon}{4}>\alpha-\frac{\epsilon}{2},$$ a completion of the proof.
\end{proof}


\section{Proof of Theorem \ref{thm:main}}
\label{sec:proofmain}
\begin{proof}(of Theorem \ref{thm:main})
We use the definitions of $\interval(d)$ and $\interval_\calD(d,\alpha)$ in Section \ref{subsec:relatedinterval}. As pointed out in Section \ref{sec:interval}, we only need to consider $\calD$ as the uniform distribution over $[0,1]$ and we omit it for simplicity.  %\begin{claim}
%\label{lm:reductiontobicriteria}
%$\forall \epsilon\in(0,\frac{1}{2}),\forall \alpha\in[0,1],\forall d>\frac{2}{\epsilon},\interval((1+\frac{\epsilon}{2})d,\alpha-\epsilon)\subseteq\interval(d,\alpha)$.
%\end{claim}
%\begin{proof}
%$\forall f\in\interval((1+\frac{\epsilon}{2})d,\alpha-\epsilon),\exists g\in\interval((1+\frac{\epsilon}{2})d)$ s.t. $\dist(f,g)\leq \alpha-\epsilon$. Assume $g$ uses $k\leq (1+\frac{\epsilon}{2})d$ intervals. Without loss of generality, we can assume that $k\geq d$. We remove $\lceil\frac{\epsilon k}{2}\rceil$ shortest intervals from the $k$ intervals. The number of remaining intervals is at most $(1-\frac{\epsilon}{2})k\leq (1-\frac{\epsilon}{2})(1+\frac{\epsilon}{2})d\leq d$. The distance increase is upper bounded by $(\frac{\epsilon k}{2}+1)\cdot\frac{1}{k}\leq \frac{\epsilon}{2}+\frac{1}{d}\leq\epsilon$.
%\end{proof}





%Now we come back and prove Theorem \ref{thm:main}.
If $d\leq\frac{8}{\epsilon}$, we can simply do agnostic learning using $O(\frac{d}{\epsilon^2}\log\frac{1}{\epsilon})=O(\frac{1}{\epsilon^3}\log\frac{1}{\epsilon})$ queries and unlabeled examples. So in the rest of the proof, we assume $d>\frac{8}{\epsilon}$. We pick the largest positive integer $m$ satisfying $m\leq\frac{\epsilon d}{8}$ and we define $\lambda=\frac{d}{m}=O(\frac{1}{\epsilon})$.

Since the data distribution is assumed uniform on $[0,1]$, we can assume without loss of generality that our ground set $X$ is $[0,1]$ and $f\in\{0,1\}^X$. We evenly cut $X$ into $m$ pieces: $X_1=[0,\frac{1}{m}],X_2=(\frac{1}{m},\frac{2}{m}],X_3=(\frac{2}{m},\frac{3}{m}],\cdots,X_m=(\frac{m-1}{m},1]$. $\forall 1\leq i\leq m,\forall k\in\mathbb{N}$, we define $\calC_i^k$ to be the class of binary functions $f$ on $X_i$ such that $f^{-1}(1)$ is a union of at most $k$ intervals. Note that $\calC_i^0\neq\emptyset$. Therefore, we can define $\property$, the composition of $m$ additive properties as in Section \ref{sec:composition}. 

Note that for any $d'>0$ and any truncation $t>0$, the concept class $\property^{t}(d')$ has VC-dimension at most $2d'$. Therefore, according to the VC Theory for agnostic learning, for any $\mu\in(0,\frac{1}{2}),\epsilon'=\frac\epsilon 4,l=O(\frac{1}{\epsilon'\mu^2}+\frac{1}{\epsilon'^2})$, we have a $((1+\frac{\mu}{2})(1+\frac{\epsilon}{8})\lambda l,l,\frac{2(1+\frac{\epsilon}{8})\lambda}{\epsilon'},\epsilon')$ distance approximation oracle using $O(\frac{(1+\frac{\mu}{2})(1+\frac{\epsilon}{8})\lambda l}{{\epsilon'}^2}\log\frac{1}{\epsilon'})=O(\frac{l}{\epsilon'^2\epsilon}\log\frac{1}{\epsilon'})=O((\frac{1}{\epsilon'^3\epsilon\mu^2}+\frac{1}{\epsilon'^4\epsilon})\log\frac{1}{\epsilon'})=O((\frac 1{\epsilon^4\mu^2}+\frac 1{\epsilon^5})\log\frac 1\epsilon)$ queries and unlabeled examples simply by empirical risk minimization. By the Composition Lemma (Lemma \ref{thm:additive}), we have an algorithm that outputs $\hat\alpha$ such that $\forall f$,
\begin{enumerate}
\item $\forall \alpha$ s.t. $f\in\property((1+\frac{\epsilon}{8})\lambda m,\alpha)$, it holds with probability at least $\frac{2}{3}$ that $\hat\alpha\leq\alpha+2\epsilon'(=\alpha+\frac\epsilon 2)$;
\item $\forall \alpha$ s.t. $f\notin\property((1+\mu)(1+\frac{\epsilon}{8})\lambda m,\alpha)$, it holds with probability at least $\frac{2}{3}$ that $\hat\alpha>\alpha-2\epsilon'(=\alpha-\frac\epsilon 2)$.
\end{enumerate}

Choose $1+\mu=\frac{1+\frac{\epsilon}{4}}{1+\frac{\epsilon}{8}}$ and note that $\lambda m=d,\interval(d,\alpha)\subseteq\property(d+m,\alpha)\subseteq\property((1+\frac{\epsilon}{8})d,\alpha)$ and $\property((1+\frac{\epsilon}{4})d,\alpha)\subseteq \interval((1+\frac{\epsilon}{4})d,\alpha)$, we have $\forall f$,
\begin{enumerate}
\item $\forall \alpha$ s.t. $f\in\interval(d,\alpha)$, it holds with probability at least $\frac{2}{3}$ that $\hat\alpha\leq\alpha+\frac\epsilon 2$;
\item $\forall \alpha$ s.t. $f\notin\interval((1+\frac{\epsilon}{4})d,\alpha)$, it holds with probability at least $\frac{2}{3}$ that $\hat\alpha>\alpha-\frac \epsilon 2$.
\end{enumerate}

This is an $(\frac \epsilon 2,1+\frac{\epsilon}{4})$-bi-criteria tester for unions of $d$ intervals. According to the Composition Lemma (Lemma \ref{thm:additive}), the query complexity and the unlabeled sample complexity of the algorithm are $O((\frac{1}{\epsilon^4\mu^2}+\frac{1}{\epsilon^5})\log\frac{1}{\epsilon})=O(\frac{1}{\epsilon^6}\log\frac{1}{\epsilon})$ and $O((\frac{l}{\epsilon'^2\epsilon}\log\frac{1}{\epsilon'})\cdot\frac{m}{l})=O(\frac{d}{\epsilon^2}\log\frac{1}{\epsilon})$.

%Now we define $\epsilon'=\frac{\epsilon}{2}$ and by rewriting the second statement in an equivalent way, we get $\forall \alpha$,
%\begin{enumerate}
%\item $\forall f\in\interval(d,\alpha)$, it holds with probability at least $\frac{2}{3}$ that $\hat\alpha\leq\alpha+\frac{\epsilon}{2}<\alpha+\epsilon$;
%\item $\forall f\notin\interval((1+\frac{\epsilon}{4})d,\alpha-\frac{\epsilon}{2})$, it holds with probability at least $\frac{2}{3}$ that $\hat\alpha>(\alpha-\frac{\epsilon}{2})-\frac{\epsilon}{2}=\alpha-\epsilon$.
%\end{enumerate}

Finally, note that \citet{BBBY12} revealed a basic property of unions of $d$ intervals that $\interval((1+\frac\epsilon 4)d)\subseteq\interval(d,\frac\epsilon 2)$, implying $\interval((1+\frac\epsilon 4)d,\alpha)\subseteq \interval(d,\alpha+\frac\epsilon 2)$, which completes the proof.


%approximates the distance from $f$ to $\property((1+\frac{\epsilon}{8})\lambda m)$
%Also, we can define $\property(d,\alpha)$ as in Section \ref{subsec:composition}. An easy observation is that $\forall \delta>0,\forall\alpha\in[0,1],\interval(d,\alpha)\subseteq\property(d+m,\alpha)\subseteq\property((1+\frac{\epsilon}{8})d,\alpha)$ and $\property((1+\delta)(1+\frac{\epsilon}{8})d,\alpha)\subseteq \interval((1+\delta)(1+\frac{\epsilon}{8})d,\alpha)$.

%Therefore, the algorithm only needs to distinguish $f\in\property((1+\frac{\epsilon}{8})d,\alpha))=\property((1+\frac{\epsilon}{8})\lambda m,\alpha)$ and $f\notin \property((1+\frac{\epsilon}{4})d,\alpha+\frac{\epsilon}{2})=\property((1+\frac{\epsilon}{4})\lambda m,\alpha+\frac{\epsilon}{2})$. Now, if we define $\lambda'=(1+\frac{\epsilon}{8})\lambda,\delta'=\frac{1+\frac{\epsilon}{4}}{1+\frac{\epsilon}{8}}-1=\frac{\epsilon}{8+\epsilon},\epsilon'=\frac{\epsilon}{2}$, then we can rewrite our task as distinguishing $f\in\property(\lambda' m,\alpha)$ and $f\notin\property((1+\delta')\lambda' m,\alpha+\epsilon')$. Note that we have a $((1+\frac{\delta'}{2})\lambda' l,l,\frac{\lambda'}{\epsilon'},\alpha+\frac{\epsilon'}{3},\alpha+\frac{\epsilon'}{2})$-partial tester using $O(\frac{\lambda' l}{\epsilon'^2}\log\frac{1}{\epsilon'})=O(\frac{l}{\epsilon^3}\log\frac{1}{\epsilon})$ queries and unlabeled samples by agnostic learning. Also note that $O(\frac{1}{\epsilon'\delta'^2}+\frac{1}{\epsilon'^2})=O(\frac{1}{\epsilon^3})$. Therefore, by Theorem \ref{thm:additive}, there is an algorithm that can handle the task using $O(\frac{1}{\epsilon^6}\log\frac{1}{\epsilon})$ queries on $O(\frac{m}{\epsilon^3}\log\frac{1}{\epsilon})=O(\frac{d}{\epsilon^2}\log\frac{1}{\epsilon})$ unlabeled samples.
\end{proof}


\section{Proof of Theorem \ref{thm:pth}}
\label{sec:pthproof}
Before proving the theorem, we first show a simple estimator that works for any loss function $\mathrm{loss}(\cdot)$ bounded in $[0,1]$ with $L$-Lipschitz property\footnote{We say $\mathrm{loss}(\cdot)$ has $L$-Lipschitz property if $\forall x_1,x_2\in[0,1],|\loss(x_1)-\loss(x_2)|\leq L|x_1-x_2|$.} using $O(\frac{L^2}{\epsilon^4}\cdot\log\frac{1}{\epsilon})$ queries on $N+O(\frac{1}{\epsilon^2})$ unlabeled examples. The estimator runs for $O(\frac{1}{\epsilon^2})$ iterations and in each $i$th iteration, the estimator samples a fresh unlabeled example $x$ and then queries the labels of $w=O(\frac{L^2}{\epsilon^2}\log\frac{1}{\epsilon})$ examples $x_1,x_2,\cdots,x_w$ sampled independently at random uniformly from the $k$ nearest neighbors of $x$ in $S$. The estimator for this iteration is $E_i=\mathrm{loss}(|\frac{1}{w}\sum\limits_{j=1}^wf(x_j)-f(x)|)$. The final output of the estimator is the average of all $E_i$'s for all iterations $i$.

We prove Theorem \ref{thm:pth} by slightly modifying the above estimator's each iteration for $p$th-power loss. Instead of looking at the labels of $w$ examples, we only need to look at $p$ labels of $x_1,x_2,\cdots, x_p$, still sampled independently at random uniformly from the $k$ nearest neighbors of $x$ in $S$. In this case, $E_i$ is defined to be $\prod\limits_{j=1}^p|f(x_j)-f(x)|$. The final output of the estimator is still the average of $E_i$'s.
%The proof of Theorem \ref{thm:pth} is based on the following lemma.
%\begin{lemma}
%\label{lm:pth}
%Suppose the examples in the trainning set $T$ and the test point $x$ are sampled independently at random according to $\calD$. Suppose $x_1,x_2,\cdots,x_p$ are sampled independently at random uniformly from the $k$ nearest neighbors of $x$ in $T$. Suppose $e_i$ is defined to be $|f(x_i)-f(x)|$Then, $\mathbb E_{T,x}[\mathbb E_{x_1}[e_i]^p]=\mathbb E_{T,x,x_1,x_2,\cdots,x_p}[\prod\limits_{j=1}^pe_i]$.
%\end{lemma}
%\begin{proof}
%\begin{equation}
%\begin{split}
%&\mathbb E_{T,x}[|\mathbb E_{x_1}[f(x_1)-f(x)]|^p]\\
%=&\mathbb E_{T,x}[|(\mathbb E_{x_1}[f(x_1)-f(x)])^p|]\\
%=&\mathbb E_{T,x}[|\underbrace{(\mathbb E_{x_1}[f(x_1)-f(x)])(\mathbb E_{x_1}[f(x_1)-f(x)])\cdots (\mathbb E_{x_1}[f(x_1)-f(x)])}_{p}|]\\
%=&\mathbb E_{T,x}[|\prod\limits_{j=1}^p\mathbb E_{x_j}[f(x_j)-f(x)]|]\\
%=&\mathbb E_{T,x}[|\mathbb E_{x_1,x_2,\cdots,x_p}[\prod\limits_{j=1}^p(f(x_j)-f(x))]|]\\
%=&\mathbb E_{T,x}[\mathbb E_{x_1,x_2,\cdots,x_p}[\prod\limits_{j=1}^p|f(x_j)-f(x)|]]\\
%=&\mathbb E_{T,x,x_1,x_2,\cdots,x_p}[\prod\limits_{j=1}^p|f(x_j)-f(x)|]
%\end{split}
%\end{equation}
%\end{proof}
\begin{proof}(of Theorem \ref{thm:pth})
We use $e_j$ to denote $|f(x_j)-f(x)|$. To show the above estimator works, we first look at the value we want to estimate: $\mathbb E_{x}[(\err(x))^p]=\mathbb E_{x}[|\mathbb E_{x_1}[f(x_1)-f(x)]|^p]=\mathbb E_{x}[\mathbb E_{x_1}[|f(x_1)-f(x)|]^p]=\mathbb E_{x}[(\mathbb E_{x_1}[e_1])^p]$, where $x_1$ is sampled uniformly from the $k$ nearest neighbors of $x$ in $T$. Here, we can move the absolute value $|\cdot|$ inside because $f(x_1)-f(x)$ is either always non-negative (when $f(x)=0$) or always non-positive (when $f(x)=1$). Note that $x_1,x_2,\cdots,x_p$ are iid, so we know $\mathbb E_{x}[(\err(x))^p]=\mathbb E_{x}[ (\mathbb E_{x_1}[e_1])^p]=\mathbb E_{x}[\mathbb E_{x_1,x_2,\cdots,x_p}[e_1e_2\cdots e_p]]=\mathbb E_{x,x_1,x_2,\cdots,x_p}[\prod\limits_{j=1}^p|f(x_j)-f(x)|]$. According to the Chernoff Bound, the empirical mean of $\prod\limits_{j=1}^p|f(x_j)-f(x)|$ over $O(\frac 1{\epsilon^2})$ iid trials approximates the value $\mathbb E_{x}[(\err(x))^p]$ within additive error $\epsilon$ with probability at least $\frac 23$, which completes the proof.
\end{proof}

Theorem \ref{thm:pth} also holds naturally for Weighted Nearest Neighbor algorithms \citep{R66} with soft predictions, in which $\hat f(x)$ is a weighted average of $f(x')$ for all $x'\in S$ where the weights depend on the distances $\D(x',x)$, simply by sampling $x_1,x_2,\cdots,x_p$ iid from $S$ according to the weights. 

%In Theorem \ref{thm:pthpowerlower} (Section \ref{subsec:lower}), we will show a matching lower bound for the $O(\frac{p}{\epsilon^2})$ query complexity.

\section{Proof of Theorem \ref{thm:bestk}}
\label{sec:bestkproof}
\begin{lemma}
\label{lm:bestk}
Suppose $k_1\leq k_2$ and the loss function $\loss(\cdot)$ is $L$-Lipschitz. Then, $|\loss_{k_1}-\loss_{k_2}|\leq L\cdot(1-\frac{k_1}{k_2})$.
\end{lemma}
\begin{proof}
When the test point $x$ is chosen, we use $x_1,x_2,\cdots,x_{k_2}$ to denote the closest $k_2$ points to $x$ in $S$, arranged in non-decreasing order of their distances to $x$. Each $x_i$ might be random because ties might be broken randomly. We use $e_i$ to denote $|f(x_i)-f(x)|$. Note that we have $\loss_{k_1}=\mathbb E_{x,x_1,x_2,\cdots,x_{k_1}}[\loss(\frac{1}{k_1}\sum\limits_{i=1}^{k_1}e_i)]$ and $\loss_{k_2}=\mathbb E_{x,x_1,x_2,\cdots,x_{k_2}}[\loss(\frac{1}{k_2}\sum\limits_{i=1}^{k_2}e_i)]$. Therefore,
\begin{equation}
\begin{split}
&|\loss_{k_1}-\loss_{k_2}|\\
\leq &\mathbb E_{x,x_1,x_2,\cdots,x_{k_2}}[|\loss(\frac{1}{k_1}\sum\limits_{i=1}^{k_1}e_i)-\loss(\frac{1}{k_2}\sum\limits_{i=1}^{k_2}e_i)|]\\
\leq &L\cdot \mathbb E_{x,x_1,x_2,\cdots,x_{k_2}}[|\frac{1}{k_1}\sum\limits_{i=1}^{k_1}e_i-\frac{1}{k_2}\sum\limits_{i=1}^{k_2}e_i|]\\
= &L\cdot \mathbb E_{x,x_1,x_2,\cdots,x_{k_2}}[|(\frac{1}{k_1}-\frac{1}{k_2})\sum\limits_{i=1}^{k_1}e_i-\frac{1}{k_2}\sum\limits_{i=k_1+1}^{k_2}e_i|]\\
\leq &L\cdot \mathbb E_{x,x_1,x_2,\cdots,x_{k_2}}[\max\{(\frac{1}{k_1}-\frac{1}{k_2})\sum\limits_{i=1}^{k_1}e_i,\frac{1}{k_2}\sum\limits_{i=k_1+1}^{k_2}e_i\}]\\
\leq &L\cdot\max\{(\frac{1}{k_1}-\frac{1}{k_2})\cdot k_1,\frac{1}{k_2}\cdot(k_2-k_1)\}\\
=&L\cdot(1-\frac{k_1}{k_2})
\end{split}
\end{equation}
\end{proof}
\begin{proof}(of Theorem \ref{thm:bestk})
If we apply Lemma \ref{lm:bestk} to $p$th-power loss, which is $p$-Lipschitz, we know for any $1\leq \frac{k_2}{k_1}\leq \frac{p}{p-\epsilon}$, it holds that $|\loss_{k_1}-\loss_{k_2}|\leq\epsilon$. If we define $t=\lfloor\log_{\frac{p}{p-\frac{\epsilon}{3}}}N\rfloor,k_{2i}=\lfloor(\frac{p}{p-\frac{\epsilon}{3}})^i\rfloor,k_{2i+1}=\lceil(\frac{p}{p-\frac{\epsilon}{3}})^i\rceil$ for $i=0,1,2,\cdots,t$, then we know $\exists 0\leq i\leq 2t+1$ such that $k_i$ is $\frac{\epsilon}{3}$-approximately-best. By Theorem \ref{thm:pth}, we can estimate $\loss_{k_i}$ for every $0\leq i\leq 2t+1$ up to an additive error $\frac{\epsilon}{3}$ using $O(\frac{pt\log t}{\epsilon^2})$ queries on $N+O(\frac{t\log t}{\epsilon^2})$ unlabeled examples.\footnote{Repeat the estimator $O(\log t)$ times and take the median to boost its success probability to $1-O(\frac{1}{t})$.} The $k_i$ yielding the smallest approximation of $\loss_{k_i}$ is $\epsilon$-approximately-best. Note that $t=O(\frac{p\log N}{\epsilon})$, so the query complexity is $O(\frac{p^2\log N}{\epsilon^3}(\log\log N+\log p+\log\frac{1}{\epsilon}))$ and the unlabeled sample complexity is $N+O(\frac{p\log N}{\epsilon^3}(\log\log N+\log p+\log\frac{1}{\epsilon}))$.
\end{proof}

\section{Lower Bound Results for Estimating $k$-Nearest Neighbor Algorithms}
\label{sec:lower}
Our lower bound results in this section are stronger in the sense that the estimator has query access to $f$, knows the distribution to be the uniform distribution $\calU$ over a finite ground set $X$ and is only supposed to work on some fixed tie-breaking mechanism. Moreover, we don't require the estimator to have success probability at least $\frac{2}{3}$ for \emph{any} $S$; instead, the success probability is calculated over the random draw of $S$ and the internal randomness of the estimator. 

\begin{theorem}
\label{thm:pthpowerlower}Let $\calU$ be the uniform distribution over a finite ground set $X$. There exists a positive constant $c$ such that for any fixed $p\geq 1$, $\epsilon\in(0,\frac{1}{6\sqrt{e}})$ and oracle $M$ using any fixed tie-breaking mechanism, $\tsoft_{\calU}(f\qu,\epsilon,S,k)$ for $p$th-power loss requires at least $c\cdot\frac{p}{\epsilon^2}$ queries in the worst case over all finite metric spaces $(X,\D)$.
\end{theorem}

%Our goal of the testing algorithm is to approximate the accuracy of the $k$-nearest majority vote algorithm given $X,d,f$ and $N$. We assume that the distribution $\calD$ is unknown to the algorithm and the function $f$ is given by active access. 

%A natural question to ask is whether there is a better algorithm achieving less query complexity. By the following reduction theorem, we'll show a negative answer to this question under the assumption that the ``naive'' algorithm for $\aga$ cannot be improved in query complexity. The theorem also leads to an unconditional sample complexity lower bound for approximating the accuracy of $k$-nearest majority vote indicating the linear growth with respect to $k$ in the sample complexity is unimprovable.



%$\aga(\mathbf A, \gamma,\epsilon)$ has a simple algorithm requiring $O(\frac{1}{\gamma^2\epsilon^2}\log\frac{1}{\epsilon})$ queries as follows. The algorithm randomly picks $O(\frac{1}{\epsilon^2})$ arms. For each of the picked arms, the algorithm queries each arm $O(\frac{1}{\gamma^2}\log\frac{1}{\epsilon})$ times and think of it as ``good'' if more than half of the results are positive and ``bad'' otherwise. The algorithm outputs the fraction of the ``good'' arms among the picked arms.

%The following is the reduction theorem. Note that the theorem is stronger in the sense that the algorithm $A$ has query access and knows the distribution to be the uniform distribution $\calU$.
%The following theorem states a lower bound $\Omega(\frac{k}{\epsilon\log\frac{1}{\epsilon}})$ for the query complexity of $\thard$, implying that the linear growth with respect to $k$ in the query complexity of $\thard$ can't be improved.
\begin{theorem}
\label{thm:mv}
There exists a positive constant $c$ such that for any fixed $k\in\mathbb{N}^*,\epsilon\in(0,\frac{1}{4})$ and oracle $M$ using any fixed tie-breaking mechanism, $\thard_{\calU}(f\qu,\epsilon,S,k)$ requires at least $c\cdot\frac{k}{\epsilon\log\frac{1}{\epsilon}}$ queries in the worst case.
\end{theorem}
%\begin{Theorem}

%\end{Theorem}
Before we prove the above theorems in Sections \ref{subsec:proofpthpowerlower} and \ref{subsec:proofmv}, we first show some related definitions and results in the stochastic multi-armed bandit setting that will be useful in the proofs of the theorems.

\subsection{Counting and Approximating the Number of Good Arms}
\label{subsec:arm}
To show query complexity lower bound results for estimating the performance of $k$-Nearest Neighbor algorithms, we show reductions from two related problems in the stochastic multi-armed bandit setting: counting the number of good arms ($\cga$) and approximating the number of good arms ($\aga$).

The setting of stochastic multi-armed bandit problems \citep{R85} is as follows. The algorithm is given $n$ arms, denoted by $\mathbf A=(A_1,A_2,\cdots,A_n)$. Each arm is a distribution over $\mathbb R$ unknown to the algorithm. The algorithm adaptively accesses these arms to receive values independently sampled according to the distributions.

In this paper, we only consider arms with Bernoulli distributions. When given $\gamma\in(0,\frac{1}{2}]$, we define good arms to be arms with mean at least $\frac{1}{2}+\gamma$ and bad arms to be arms with mean at most $\frac{1}{2}-\gamma$.

The problem of $\cga(\mathbf A,\gamma)$ is, when given $\mathbf A$ in which every $A_i$ is either good or bad, to output the number of good arms among the given $n$ arms. The algorithm should output the correct answer with probability at least $\frac{2}{3}$.

The problem of $\aga(\mathbf A,\gamma,\epsilon)$ is a similar task to $\cga(\mathbf A,\gamma)$, except that we only need to approximate the correct answer up to an additive error $\epsilon n$.

The following lemma is developed by \citet{KCG16} as a useful tool for proving lower bounds in the stochastic multi-armed bandit setting.
\begin{lemma}[Change of measure]
\label{lm:changeofdistribution}
Suppose $\mathbf A=(A_1,A_2,\cdots,A_n)$ and $\mathbf A'=(A_1',A_2',\cdots,A_n')$ are two sequences of arms. Suppose algorithm $\calA$ takes $n$ arms as input. Suppose $\mathcal E$ is an event in the $\sigma$-field $\mathcal{F}_T$ for some almost-surely finite stopping time $T$ with respect to the filtration $\{\mathcal{F}_t\}_{t\geq 0}$. Suppose $\tau_i$ is the number of queries on $A_i$ made by the algorithm. Then,
$$\sum\limits_{i=1}^n\mathbb E_{\calA,\mathbf A}[\tau_i]\mathrm{KL}(A_i,A_i')\geq D(\Pr_{\calA,\mathbf A}[\calE],\Pr_{\calA,\mathbf A'}[\calE]).\footnote{$\mathrm{KL}(X,Y)$ denotes the Kullback-Leibler divergence from distribution $Y$ to distribution $X$. If the two distributions $X$ and $Y$ are Bernoulli with means $x$ and $y$, their Kullback-Leibler divergence is the relative entropy $D(x,y)=x\log\frac{x}{y}+(1-x)\log\frac{1-x}{1-y}$.}$$
\end{lemma}



A simple special case ($n=1$) of the lemma is that to distinguish a coin with mean $\mu_1$ from a coin with mean $\mu_2$ with success probability at least $1-\delta$, an algorithm needs at least $\frac{D(1-\delta,\delta)}{D(\mu_1,\mu_2)}=\Omega(\frac{1}{D(\mu_1,\mu_2)}\log\frac{1}{\delta})$ queries in expectation for $\mu_1\neq \mu_2$ and $0<\delta\leq\frac{2}{5}$.

\subsection{Proof of Theorem \ref{thm:pthpowerlower}}
\label{subsec:proofpthpowerlower}
\begin{proof}(of Theorem \ref{thm:pthpowerlower})
We define $\epsilon'=6\sqrt{e}\epsilon$. Note that $D(\frac{1-\epsilon'}{2p},\frac{1}{2p})=O(\frac{{\epsilon'}^2}{p})$ for $p\geq 1$ and $\epsilon'\in(0,1)$. Therefore, we only need to show that a  $\tsoft_{\calU}(f\qu,\epsilon,S,k)$ estimator implies an algorithm that distinguishes a coin of mean $\frac{1-\epsilon'}{2p}$ from a coin of mean $\frac{1}{2p}$ with success probability at least $\frac{3}{5}$ using at most the same number of queries. We construct the algorithm in the following way.

The algorithm first constructs a $\knnsoft$ instance with ground set $X$ and distance metric $\D$. We first choose $k=\lceil\frac{c'p^2}{\epsilon^2}\rceil, b=\lceil\frac{6}{\epsilon}\rceil,N=\lceil c''\cdot (1+b)k\rceil$ and $m=\lceil \frac{c'''N^2}{1+b}\rceil\geq\frac{6N}{(1+b)\epsilon}$. Here, $c',c''$ and $c'''$ are sufficiently large constants. $X$ consists of a star with $m$ centers and $bm$ leaves. Each center $C$ has a distance $\D_C\in(1,2)$ to every leaf in the star and different centers have different values of $\D_C$ to avoid ties. The distance between each pair of leaves is 2 and the distance between each pair of centers is 1. 

The algorithm then simulates the estimator $\tsoft_{\calU}(f\qu,\epsilon,S,k)$ on this $\knnsoft$ instance without knowing $f$ beforehand. Every time the estimator queries the label of a new example, it simulates the result as follows. If the example being queried is a leaf, the result is 1. If the example being queried is a center, the result is obtained to be the same result of an independent toss of the coin we want to distinguish. Finally, if the output of $\tsoft_{\calU}(f\qu,\epsilon,S,k)$ is above $\frac{1}{2}[(1-\frac{1}{2p})^p+(1-\frac{1-\epsilon'}{2p})^p]$, the algorithm then guesses the coin to have mean $\frac{1-\epsilon'}{2p}$. Otherwise, the algorithm guesses the coin to have mean $\frac{1}{2p}$.

Now we show that the above algorithm correctly distinguishes the coins with success probability at least $\frac{3}{5}$. The process of the algorithm, by interchanging the randomness of the labels (coin tosses) and the internal randomness of the $\tsoft_{\calU}(f\qu,\epsilon,S,k)$ estimator, can be viewed in the way that the true labels $f$ are determined before we run the $\tsoft_{\calU}(f\qu,\epsilon,S,k)$ estimator. The leaves all have label 1 and each center is independently labeled 0 or 1 according to the result of a toss of the coin. After the labels $f$ are decided, the $\tsoft_{\calU}(f\qu,\epsilon,S,k)$ estimator is then simulated  to approximate the $p$th-power loss of the $\knnsoft$ instance up to additive error $\epsilon$ with success probability at least $\frac{2}{3}$. 

Suppose the coin to be distinguished has mean $\mu$. Note that the total number of points in the ground set is $(1+b)m=\Omega(N^2)$, therefore we can make sure with probability at least $1-\frac{1}{40}$ that no two unlabeled examples lie on the same point. Because each random example has probability $\frac{1}{1+b}$ to lie in the centers and $N\geq c''\cdot (1+b)k$, therefore by choosing a sufficiently large $c''$, we can make sure with probability at least $1-\frac{1}{40}$ that in the unlabeled sample pool, there are at least $k$ examples lying at the centers. These two events happen at the same time with probability at least $1-\frac{1}{20}$ by the Union Bound. Conditioned on these two events happening, by a sufficiently large choice of $c'$, among those unlabeled examples lying at the centers, we can make sure that with probability at least $1-\frac{1}{20}$, the average of the labels of the $k$ examples with smallest $\D_C$ is contained in $(\mu-\frac{\epsilon}{6p},\mu+\frac{\epsilon}{6p})$. All these events happen at the same time with probability at least $(1-\frac{1}{20})^2\geq 1-\frac{1}{10}$, and in this case, every leaf outside the unlabeled pool $S$ has $L^1$ error in $(1-\mu-\frac{\epsilon}{6p},1-\mu+\frac{\epsilon}{6p})$ and thus has $p$th-power loss in $((1-\mu)^p-\frac{\epsilon}{6},(1-\mu)^p+\frac{\epsilon}{6})$. The total number of leaves in the unlabeled pool $S$ and centers is upper bounded by the size $N$ of the pool plus $m$, which contributes only a $\frac{N+m}{(b+1)m}\leq\frac{\epsilon}{3}$ fraction of the total number of points. Therefore, with probability at least $1-\frac{1}{10}$, the average $p$th-power loss of all points is contained in $((1-\mu)^p-\frac{\epsilon}{2},(1-\mu)^p+\frac{\epsilon}{2})$. %Finally, by Markov's Inequality, we know if we only consider the randomness of the labels, with probability at least $1-\frac{1}{10}$, the expected $p$th power loss of all points is contained in $((1-\mu)^p-\frac{\epsilon}{2},(1-\mu)^p+\frac{\epsilon}{2})$, which completes the proof.

Note that $(1-\frac{1-\epsilon'}{2p})^p-(1-\frac{1}{2p})^p>3\epsilon$, therefore the algorithm correctly guesses the mean of the coin with probability at least $(1-\frac{1}{10})\cdot\frac{2}{3}=\frac{3}{5}$.
\end{proof}

\subsection{Proof of Theorem \ref{thm:mv}}
\label{subsec:proofmv}
\begin{lemma}
\label{thm:reductionfrombandit}There exists a positive constant $c$ such that for any fixed $k\in \mathbb{N}^*$, $\epsilon\in (0,\frac{1}{4})$ and oracle $M$ using any fixed tie-breaking mechanism, if there is a $\thard_{\calU}(f\qu,\epsilon,S,k)$ estimator using at most $q$ queries in the worst case, then there is an $\aga(\mathbf A,\gamma,2\epsilon)$ algorithm using at most $O(q)$ queries in the worst case where $\gamma=\min\left\{\frac{1}{2},c\cdot \sqrt{\frac{\log\frac{1}{\epsilon}}{k}}\right\}$.
\end{lemma}
%Let's first consider the average error rate for points in satisfied stars.  , and in this case, all the leaves of each of these stars that are not in the unlabeled pool $S$ has error rate 1 if the star is good and 0 if the star is bad. The total number of the leaves in the unlabeled pool $S$ and the centers is upper bounded by the size $N$ of the pool plus $m$, which contributes only a $\frac{N+m}{(b+1)m}<\frac{\epsilon}{3}$ fraction of the total number of points. Therefore, with probability at least $1-\frac{1}{10}$, the average $p$th power loss of all points is contained in $((1-\mu)^p-\frac{\epsilon}{2},(1-\mu)^p+\frac{\epsilon}{2})$. 

%To show the claim, given the fraction of good arms to be $\xi$, we only need to show that if the labels of the centers are chosen independently at random such that each center of a star whose corresponding arm has mean $p$ has probability $p$ being labeled 1 and probability $1-p$ being labeled 0, then with probability $\frac{9}{10}=\frac{\frac{3}{5}}{\frac{2}{3}}$, the $\mv$ instance has error rate in $[\xi-\epsilon,\xi+\epsilon]$. In fact, with probability at least $1-\frac{\epsilon^2}{120}$, each star satisfies that there are more than $k$ training data located in its center (by the choice of $N$ and $c'$), more than half of the $k$-nearest are labeled 1 for good stars and less than half of the $k$-nearest are labeled 1 for bad stars (by the choice of $\gamma$ and $c$). By Markov's Inequality, with probability $1-\frac{\epsilon}{20}$, more than $1-\frac{\epsilon}{6}$ fraction of the stars satisfy the previous property. Conditioned on this happening, the instance has error rate in $[\xi-\frac{\epsilon}{2},\xi+\frac{\epsilon}{2}]$ because $b\geq\frac{6}{\epsilon}$ and $N\leq\frac{\epsilon}{6}\cdot mnb$. Then by Markov's inequality, with probability at least $1-\frac{1}{10}$ of the labeling, with probability at least $1-\frac{\epsilon}{2}$ of the sampling has error rate in $[\xi-\frac{\epsilon}{2},\xi+\frac{\epsilon}{2}]$, which means that with probability at least $\frac{9}{10}$ of the labeling, the error rate of the instance is in $[\xi-\epsilon,\xi+\epsilon]$. 

The above lemma shows that a query complexity lower bound for $\aga(\mathbf A, \gamma,\epsilon)$ can imply a query complexity lower bound for $\thard_{\calU}(f\qu,\epsilon,S,k)$. $\aga(\mathbf A, \gamma,\epsilon)$ has a simple algorithm requiring $O(\frac{1}{\gamma^2\epsilon^2}\log\frac{1}{\epsilon})$ queries as follows. The algorithm randomly picks $O(\frac{1}{\epsilon^2})$ arms. For each of the picked arms, the algorithm queries it $O(\frac{1}{\gamma^2}\log\frac{1}{\epsilon})$ times and thinks of it as ``good'' if more than half of the results are positive and ``bad'' otherwise. The algorithm outputs the fraction of ``good'' arms among the picked arms.

If we assume the simple $O(\frac{1}{\gamma^2\epsilon^2}\log\frac{1}{\epsilon})$ query complexity for $\aga$ is not improvable, then Lemma \ref{thm:reductionfrombandit} implies that the $O(\frac{k}{\epsilon^2})$ query complexity for $\thard$ is also not improvable. In other words, if for every sequences $\epsilon_n\rightarrow 0$ and $\gamma_n\rightarrow 0$, there exists a positive constant $c$ such that $\aga(\mathbf A,\epsilon_i,\gamma_i)$ needs at least $c\cdot \frac{1}{\gamma_i^2\epsilon_i^2}\log\frac{1}{\epsilon_i}$ queries in the worst case, then according to Lemma \ref{thm:reductionfrombandit}, we know for any sequences $\{k_n\},\{\epsilon_n\}$ such that $\epsilon_n\rightarrow 0,\frac{k_n}{\log\frac{1}{\epsilon_n}}\rightarrow\infty$, there exists a positive constant $c'$ such that the estimator $\thard_{\calU}(f\qu,\epsilon,S,k)$ for $\mv$ algorithms needs at least $c'\cdot\frac{k_i}{\epsilon_i^2}$ queries in the worst case.

\begin{proof}(of Lemma \ref{thm:reductionfrombandit})
Since the success probability can be boosted by repetition, we only show an $\aga(\mathbf A,\gamma,2\epsilon)$ algorithm with success probability at least $\frac{3}{5}$. Given any instance of $\aga(\mathbf A,\gamma,2\epsilon)$ with total number of arms equal to $n$, the algorithm constructs a ground set $X$ and the distance metric $\D$ on it to form a $\mv$ instance in the following way. We first choose $b=\lceil\frac{3}{\epsilon}\rceil,N=\lceil c'\cdot (1+b)n(k+\log\frac{1}{\epsilon})\rceil$ and $m=\lceil\frac{c''N^2}{(1+b)n}\rceil\geq\frac{3N}{(1+b)n\epsilon}$. Here, $c'$ and $c''$ are sufficiently large constants. $X$ consists of $n$ identical stars, each corresponds to an arm, with the distances between stars to be very large. Each star consists of $m$ centers and $bm$ leaves. Each center $C$ has a distance $\D_C\in(1,2)$ to every leaf in the same star and different centers have different values of $\D_C$ to avoid ties. The distance between each pair of leaves in the same star is 2 and the distance between each pair of centers in the same star is 1. 

The algorithm then simulates the estimator $\thard_{\calU}(f\qu,\epsilon,S,k)$ on this $\mv$ instance without knowing $f$ beforehand. Every time the estimator queries the label of a new example, it simulates the result as follows. If the example being queried is a leaf, the result is 0. If the example being queried is a center, the result is obtained to be the same result of an independent query to the corresponding arm. Finally, the algorithm outputs $\hat\alpha n$ when the $\thard_{\calU}(f\qu,\epsilon,S,k)$ estimator outputs $\alpha$.

%We run algorithm $A$ on ground set $X$ with distance metric $\D$ without knowing the labels $f$ beforehand. Each time $A$ queries the label of a new example, we answer in the following manner. If the example being queried is a leaf, we always answer 0, and if the example being queried is a center of the $i$th star, we query the $i$th arm and answer the result of the coin toss. Finally we output what $A$ outputs. We claim that this is an $\aga(\mathbf A,\gamma,2\epsilon)$ algorithm.
Now we show that the above is an $\aga(\mathbf A,\gamma,2\epsilon)$ algorithm with success probability at least $\frac{3}{5}$. The process of the algorithm, by interchanging the randomness of the labels (arms) and the internal randomness of the $\thard_{\calU}(f\qu,\epsilon,S,k)$ estimator, can be viewed in the way that the true labels $f$ are determined before we run the $\thard_{\calU}(f\qu,\epsilon,S,k)$ estimator. The leaves all have labels 0 and each center is independently labeled 0 or 1 according to the result of a query to the corresponding arm. After the labels $f$ are decided, the $\thard_{\calU}(f\qu,\epsilon,S,k)$ estimator is then simulated  to approximate the error rate of the $\mv$ instance up to additive error $\epsilon$ with success probability at least $\frac{2}{3}$. 

Let's say a star is good (bad) if it corresponds to a good (bad) arm. Suppose there are $\xi n$ good arms, and thus $\xi n$ good stars. Note that there are $(1+b)mn=\Omega(N^2)$ points in the ground set, we can make sure with probability at least $1-\frac{1}{20}$ that no two unlabeled examples lie on the same point, on which the following discussion is conditioned. Let's first fix a star $R$ whose corresponding arm has mean $\mu$. Because each random example has probability $\frac{1}{(1+b)n}$ to lie in the centers of $R$ and $N\geq c'\cdot (1+b)n(k+\log\frac{1}{\epsilon})$, therefore by choosing a sufficiently large $c'$, we can make sure with probability at most $\frac{\frac{\epsilon}{120}}{1-\frac{1}{20}}$ that in the unlabeled sample pool, there are less than $k$ examples lying at the centers of $R$. Therefore, by a sufficiently large choice of $c$, among those unlabeled examples lying at the centers of $R$, we can make sure that with probability at least $(1-\frac{\frac{\epsilon}{120}}{1-\frac{1}{20}})(1-\frac{\epsilon}{200})\geq 1-\frac{\frac{\epsilon}{60}}{1-\frac{1}{20}}$, the average of the labels of the $k$ examples with smallest $\D_C$ is contained in $(\mu-\gamma,\mu+\gamma)$, or $R$ is \emph{satisfied}. By Markov's Inequality, with probability at least $1-\frac{\frac{1}{20}}{1-\frac{1}{20}}$, or $1-\frac{1}{10}$ if we unwrap the conditional probability of $1-\frac{1}{20}$, at least a $(1-\frac{\epsilon}{3})$ fraction of all the $n$ stars are satisfied. In a satisfied star, any leaf that is not in the unlabeled pool has $L^1$ error 1 if the star is good and $L^1$ error 0 if the star is bad. Note that there are at most $N$ leaves in the unlabeled pool, contributing at most an $\frac{N}{(1+b)mn}\leq\frac{\epsilon}{3}$ fraction of the total number of points. Also there are only $mn$ centers in total, contributing at most an $\frac{mn}{(1+b)mn}\leq\frac{\epsilon}{3}$ fraction of the total number of points. Therefore, with probability at least $1-\frac{1}{10}$, the average error of all points is contained in $[\xi-\epsilon,\xi+\epsilon]$, which implies that with probability at least $(1-\frac{1}{10})\cdot\frac{2}{3}=\frac{3}{5}$, $\hat\alpha\in[\xi-2\epsilon,\xi+2\epsilon]$.
 \end{proof}
%Though we are not able to show an $\Omega(\frac{1}{\gamma^2\epsilon^2}\log\frac{1}{\epsilon})$ lower bound for approximating the fraction of good arms, by considering it's special case, \emph{counting good arms}, we can easily get an $\Omega(\frac{1}{\gamma^2\epsilon})$ lower bound, which implies by Theorem \ref{thm:reductionfrombandit} an $\Omega(\frac{k}{\epsilon\log\frac{1}{\epsilon}})$ lower bound for approximating the accuracy of $\mv$. Specifically, we show the following theorem.




 
 Before proving Theorem \ref{thm:mv}, we first show a query complexity lower bound for $\cga$.
\begin{lemma}
\label{lm:counting}
There exists a universal constant $c$ such that for any fixed $\gamma\in(0,\frac{1}{2}]$ and $n\in\mathbb N^*$, $\cga(\mathbf A,\gamma)$ requires at least $c\cdot\frac{n}{\gamma^2}$ queries in the worst case, where $n$ is the number of arms in $\mathbf A$.
\end{lemma}

\begin{proof}(of Lemma \ref{lm:counting})
We use $G$ to denote the good arm with mean $\frac{1}{2}+\gamma$ and $B$ to denote the bad arm with mean $\frac{1}{2}-\gamma$. Let's first consider the case where each of the $n$ arms is independently chosen to be $G$ or $B$ uniformly at random. Note that we require the probability of success to be at least $\frac 23$, so $\cga(\mathbf A,\gamma)$ can't always make less than $n$ queries because the probability of success is at most $\frac 12$ in this case. Therefore, $n$ is an obvious query complexity lower bound and in the rest of the proof we can assume $\gamma<\frac 14$.

We claim a stronger fact that for any $0\leq q\leq n$ and any instance consisting of $q$ $G$'s and $n-q$ $B$'s, $\cga(\mathbf A,\gamma)$ needs at least $c\cdot\frac{1}{\gamma^2}$ queries on \emph{every} of the $n$ arms. By symmetry between ``good'' and ``bad'', we only show that every $G$ arm needs to be queried at least $c\cdot\frac{1}{\gamma^2}$ times. The reason is as follows. Suppose $\mathbf A=(A_1,A_2,\cdots,A_n)$ in which $A_i=G$ for $1\leq i\leq q$ and $A_i=B$ otherwise. We define $\mathbf A'=(A_1',A_2',\cdots,A_n')$ in which $A'_i=G$ for $1\leq i\leq p-1$ and $A'_i=B$ otherwise. The only difference between $\mathbf A$ and $\mathbf A'$ is that $A_p=G$ while $A'_p=B$. We use $\calE$ to denote the event that $\cga(\mathbf A,\gamma)$ outputs $p$. By Lemma \ref{lm:changeofdistribution} and $\mathrm{KL}(G,B)=O(\gamma^2)$, we know $\mathbb E[\tau_p]\cdot O(\gamma^2)\geq D(\frac{2}{3},\frac{1}{3})=\Omega(1)$ and thus $\mathbb E[\tau_p]=\Omega(\frac{1}{\gamma^2})$. For similar reasons, we can show for all $1\leq i\leq p$ that $\mathbb E[\tau_i]=\Omega(\frac{1}{\gamma^2})$, which completes the proof.
\end{proof}

\begin{proof}(of Theorem \ref{thm:mv})
Lemma \ref{lm:counting} immediately implies the existence of a positive constant $c'$ such that for any fixed $\epsilon\in(0,\frac{1}{2})$ and $\gamma\in(0,\frac{1}{2}]$, $\aga(\mathbf A, \gamma,\epsilon)$ requires at least $c'\cdot\frac{1}{\gamma^2\epsilon}$ queries in the worst case by choosing $n=\lceil\frac{1}{2\epsilon}\rceil-1$. Then, by Lemma \ref{thm:reductionfrombandit}, we get an $\Omega(\frac{1}{
\left(\min\left\{\frac{1}{2},\sqrt{\frac{
	\log\frac{1}{\epsilon}
}{k}}\right\}\right)^2
}\cdot\frac{1}{2\epsilon})=\Omega(\frac{k}{\epsilon\log\frac{1}{\epsilon}})$ lower bound for $\thard_{\calU}(f\qu,\epsilon,S,k)$ for $k\in\mathbb N^*$ and $\epsilon\in(0,\frac{1}{4})$.
\end{proof}
%Before introducing the reduction, we first describe the setting for the problem of approximating the fraction of good arms. Suppose we are given $n$ arms $A_1,A_2,\cdots, A_n$, each $A_i$ as a 0-1 coin with mean $\mu_i$. We are also given a parameter $\gamma\in(0,\frac{1}{2}]$. We assume that the $n$ arms are either ``good'', i.e., $\mu_i\geq \frac{1}{2}+\gamma$, or ``bad'', i.e., $\mu_i\leq\frac{1}{2}-\gamma$. The algorithm can choose to query each arm for multiple times to receive the results of independent coin tosses. The goal of the algorithm is to approximate the fraction of good arms among the $n$ arms within additive error $\epsilon$ with success probability at least $\frac{2}{3}$.
\end{document}