%!TEX root = main.tex

\section{Proof of Hamiltonian Lemmas}
In this section, we prove Lemma \ref{lem:energy_nonconvex}, Lemma \ref{lem:energy_NCE} and Corollary \ref{cor:localball}, which are presented in Section \ref{sec:hamiltonian} and 
Section \ref{sec:imp_local}.  In section \ref{sec:counterex} we also give an example 
where standard AGD with negative curvature exploitation can increase the Hamiltonian.


Recall that we define the Hamiltonian as $E_t \defeq f(\x_{t}) + \frac{1}{2\eta}\norm{\v_{t}}^2$, where, for AGD, we define $\v_t = \x_t - \x_{t-1}$.
The first lemma shows that this Hamiltonian decreases in every step of~\nag~for mildly nonconvex functions.

\begingroup
\def\thetheorem{\ref{lem:energy_nonconvex}}
\begin{lemma}[Hamiltonian decreases monotonically]
  Assume that the function $f(\cdot)$ is $\ell$-smooth and set the learning rate to
be $\eta \le \frac{1}{2\ell}$, $\theta\in [2\eta \gamma,\frac{1}{2}]$ in ~\nag~(Algorithm \ref{algo:AGD}). Then, for every iteration $t$ where~\eqref{eq:certificate} does not hold, we have:
  \begin{equation*}
  E_{t+1} \le E_t - \frac{\theta}{2\eta}\norm{\v_t}^2 - \frac{\eta}{4}\norm{\grad f(\y_{t})}^2.
  \end{equation*}
\end{lemma}
\addtocounter{theorem}{-1}
\endgroup

\begin{proof}
Recall that the update equation of accelerated gradient descent has following form:
\begin{align*}
\x_{t+1} &\leftarrow \y_t - \eta \grad f (\y_t) \\
\y_{t+1} &\leftarrow \x_{t+1} + (1-\theta) (\x_{t+1} - \x_t).
\end{align*}
By smoothness, with $\eta \le \frac{1}{2\ell}$:
\begin{align}
f(\x_{t+1}) \le f(\y_t) - \eta\norm{\grad f(\y_t)}^2 + \frac{\ell\eta^2}{2}\norm{\grad f(\y_t)}^2
\le f(\y_t) - \frac{3\eta}{4}\norm{\grad f(\y_t)}^2, \label{eq:energy_smooth}
\end{align}
assuming that the precondition \eqref{eq:certificate} does not hold:
\begin{align}
f(\x_t) \ge f(\y_t) + \la \grad f(\y_t), \x_t - \y_t\ra -
\frac{\gamma}{2}\norm{\y_t - \x_t}^2,
\label{eq:energy_almost_convex}
\end{align}
and given the following update equation:
\begin{align}
\norm{\x_{t+1} - \x_t}^2
 =& \norm{\y_t - \x_t  - \eta \grad f (\y_t)}^2  \nn\\
 =& \left[(1-\theta)^2\norm{\x_t - \x_{t-1}}^2
 - 2\eta\la \grad f(\y_t), \y_t - \x_t \ra
 + \eta^2 \norm{\grad f(\y_t)}^2 \right], \label{eq:energy_momentum}
\end{align}
% Adding up Eq.\eqref{eq:energy_smooth} and \eqref{eq:energy_convex}, we have:
% \begin{equation}
% f(\x_{t+1}) \le  f(\x_t) + \la \grad f(\y_t), \y_t - \x_t \ra + \frac{\ell}{2} \norm{\y_t - \x_t}^2 - \frac{\eta}{2}\norm{\grad f(\y_t)}^2 \label{eq:momentum} 
% \end{equation}
we have:
\begin{align*}
f(\x_{t+1})
+ \frac{1}{2\eta}\norm{\x_{t+1} - \x_{t}}^2
\le& f(\x_t) + \la \grad f(\y_t), \y_t - \x_t \ra - \frac{3\eta}{4}\norm{\grad f(\y_t)}^2 \\
&+\frac{1+ \eta\gamma}{2\eta}(1-\theta)^2 \norm{\x_t - \x_{t-1}}^2 - \la \grad f(\y_t), \y_t - \x_t \ra
 + \frac{\eta}{2} \norm{\grad f(\y_t)}^2\\
\le& f(\x_{t}) + \frac{1}{2\eta}\norm{\x_{t} - \x_{t-1}}^2
 - \frac{2\theta-\theta^2 - \eta\gamma(1-\theta)^2}{2\eta}\norm{\v_{t}}^2 - \frac{\eta}{4}\norm{\grad f(\y_{t})}^2 \\
\le& f(\x_{t}) + \frac{1}{2\eta}\norm{\x_{t} - \x_{t-1}}^2 - \frac{\theta}{2\eta}\norm{\v_t}^2 - \frac{\eta}{4}\norm{\grad f(\y_{t})}^2.
\end{align*}
The last inequality uses the fact that $\theta \in [2\eta \gamma, \frac{1}{2}]$ 
so that $\theta^2 \le \frac{\theta}{2}$ and $\eta\gamma \le \frac{\theta}{2}$. 
We substitute in the definition of $\v_t$ and $E_t$ to finish the proof.
% \begin{align*}
% E_{t+1} \le& E_t
% - \frac{2\theta-\theta^2 - \eta\gamma(1-\theta)^2}{2\eta}\norm{\v_{t}}^2 - \frac{\eta}{4}\norm{\grad f(\y_{t})}^2 \\
% \le &  E_t
% - \frac{\theta}{2\eta}\norm{\v_t}^2 - \frac{\eta}{4}\norm{\grad f(\y_{t})}^2
% \end{align*}
\end{proof}
We see from this proof that~\eqref{eq:energy_almost_convex} relies on approximate convexity of $f(\cdot)$, which explains why in all existing proofs, the convexity between $\x_t$ and $\y_t$ is so important. A perhaps surprising fact to note is that the above proof can in fact go through even with mild nonconvexity (captured in line $8$ of Algorithm~\ref{algo:PAGD}).
Thus, high nonconvexity is the problematic situation.
%Nestrov's AGD is designed to work for convex case. In the nonconvex setting,
To overcome this, we need to slightly modify AGD so that the Hamiltonian is
decreasing. This is formalized in the following lemma.


\begingroup
\def\thetheorem{\ref{lem:energy_NCE}}
\begin{lemma}
Assume that $f(\cdot)$ is $\ell$-smooth and $\rho$-Hessian Lipschitz. For every iteration $t$ of Algorithm~\ref{algo:PAGD} where~\eqref{eq:certificate} holds (thus running NCE), we have:
\begin{equation*}
E_{t+1}\le E_t -\min\{\frac{s^2}{2\eta},  \frac{1}{2}(\gamma - 2\rho s) s^2\}.
\end{equation*}
\end{lemma}
\addtocounter{theorem}{-1}
\endgroup


\begin{proof}
When we perform an NCE step, we know that \eqref{eq:certificate} holds. In the first case ($\norm{\v_t} \ge s$), we set $\x_{t+1}  = \x_t$ and set the momentum $\v_{t+1}$ to zero, which gives:
\begin{align*}
E_{t+1} = f(\x_{t+1}) = f(\x_t) = E_{t} - \frac{1}{2\eta}\norm{\v_t}^2
\le E_{t} - \frac{s^2}{2\eta}.
\end{align*}
In the second case ($\norm{\v_t} \le s$), expanding in a Taylor series with Lagrange remainder, we have:
\begin{equation*}
f(\x_t) =  f(\y_t) + \la \grad f(\y_t), \x_t - \y_t \ra + \frac{1}{2} (\x_t - \y_t)\trans 
\hess f(\zeta_t) (\x_t - \y_t),
\end{equation*}
where $\zeta_t = \phi\x_t + (1-\phi)\y_t$ and $\phi \in [0, 1]$. Due to the certificate \eqref{eq:certificate} we have
\begin{equation*}
\frac{1}{2} (\x_t - \y_t)\trans 
\hess f(\zeta_t) (\x_t - \y_t) \le - \frac{\gamma}{2} \norm{\x_t - \y_t}^2.
\end{equation*}
On the other hand, clearly $\min\{\la \grad f(\x_t), \delta\ra, \la \grad f(\x_t), -\delta\ra  \} \le 0$. WLOG, suppose $\la \grad f(\x_t), \delta\ra \le 0$,
then, by definition of $\x_{t+1}$, we have:
\begin{equation*}
f(\x_{t+1}) \le f(\x_{t} + \delta) 
= f(\x_t) + \la \grad f(\x_t), \delta\ra + \frac{1}{2} \delta\trans \hess f(\zeta'_t) \delta
\le  f(\x_t) + \frac{1}{2} \delta\trans 
\hess f(\zeta'_t) \delta,
\end{equation*}
where $\zeta'_t = \x_t + \phi'\delta$ and $\phi' \in [0, 1]$. 
Since $\norm{\zeta_t - \zeta'_t} \le 2s$, $\delta$ also lines up
with $\y_t - \x_t$:
\begin{equation*}
\delta\trans 
\hess f(\zeta'_t) \delta
\le \delta\trans 
\hess f(\zeta_t) \delta
+ \norm{\hess f(\zeta'_t) - \hess f(\zeta_t) }\norm{\delta}^2
\le - \gamma \norm{\delta}^2 + 2\rho s\norm{\delta}^2.
\end{equation*}
Therefore, this gives
\begin{align*}
E_{t+1} = f(\x_{t+1}) \le f(\x_t) 
- \frac{1}{2}(\gamma - \rho s) s^2
\le E_{t} - \frac{1}{2}(\gamma - 2\rho s) s^2,
\end{align*}
which finishes the proof.
\end{proof}

The Hamiltonian decrease has an important consequence: if the Hamiltonian does not decrease much, then all the iterates are localized in a small ball around the starting point. Moreover, the iterates do not oscillate much in this ball. We called this the improve-or-localize phenomenon.

\begingroup
\def\thetheorem{\ref{cor:localball}}
\begin{corollary}[Improve or localize]
Under the same setting as in Lemma \ref{lem:energy_nonconvex}, if \eqref{eq:certificate} does not hold for all steps in $[t, t+T]$, we have:
\begin{equation*}
\sum_{\tau = t+1}^{t+T}\norm{\x_\tau - \x_{\tau-1}}^2
\le \frac{2\eta}{\theta} (E_t - E_{t+T}).
% \text{~~~and~~~} \norm{\x_{t+T} -\x_t}^2 \le \frac{2\eta T}{\theta}(E_t - E_{t+T}).
\end{equation*}
\end{corollary}
% \begin{corollary}
% Under the same setting as in Lemma \ref{lem:energy_nonconvex}, if NCE is not triggered in step $[t, t+T]$, then:
% \begin{equation*}
% \sum_{\tau = t+1}^{t+T}\norm{\x_\tau - \x_{\tau-1}}^2
% \le \frac{2\eta}{\theta} (E_t - E_{t+T})
% \text{~~~and~~~} \norm{\x_{t+T} -\x_t}^2 \le \frac{2\eta T}{\theta}(E_t - E_{t+T})
% \end{equation*}
% \end{corollary}
\addtocounter{theorem}{-1}
\endgroup


\begin{proof}
The proof follows immediately from telescoping the argument of Lemma \ref{lem:energy_nonconvex}.
% The proof of second inequality follows from triangle inequality and Cauchy-Swartz to first inequality as follows:
% \begin{equation*}
% \norm{\x_{t+T} -\x_t}^2 \le \left(\sum_{\tau = t+1}^{t+T}\norm{\x_\tau - \x_{\tau-1}}\right)^2
% \le T \left(\sum_{\tau = t+1}^{t+T}\norm{\x_\tau - \x_{\tau-1}}^2\right)
% \le \frac{2\eta T}{\theta}(E_t - E_{t+T})
% \end{equation*}
\end{proof}




% \begin{lemma}\label{lem:localball}
% For function $f$ satisfy assumption 1, 2, for $T \le O(\frac{1}{\theta})$, assume
% \begin{equation*}
% E_{t+T} - E_t \ge - O(c\sqrt{\frac{\epsilon^3}{\rho}})
% \end{equation*}
% then, we have:
% \begin{equation*}
% \norm{\x_{t+T} -\x_t} \le O(\sqrt{\frac{c\epsilon}{\rho}})
% \quad \text{~and~} \quad
% \sum_{\tau=t+1}^{t+T} \norm{\x_\tau - \x_{\tau-1}}^2
% \le \frac{\eta}{\theta}O(c\sqrt{\frac{\epsilon^3}{\rho}})
% \end{equation*}
% \end{lemma}
% \begin{proof}
% Cleary NEC must have not been triggered (otherwise, decrease enough energy).
% Therefore, by theorem \ref{thm:energy_nonconvex}, we know the cumulative energy dissipation:
% \begin{align*}
% \sum_{\tau=t+1}^{t+T} \frac{\theta}{2\eta}\norm{\v_\tau}^2 
% =\sum_{\tau=t+1}^{t+T} \frac{\theta}{2\eta}\norm{\x_\tau - \x_{\tau-1}}^2
% \le O(c\sqrt{\frac{\epsilon^3}{\rho}})
% \end{align*}
% By Cauchy-Swartz, we have:
% \begin{align*}
% \sum_{\tau=t+1}^{t+T} \norm{\x_\tau - \x_{\tau+1}} 
% \le \sqrt{T}\sqrt{\sum_{\tau=t+1}^{t+T}\norm{\x_\tau - \x_{\tau-1}}^2}
% \le O(\left(\frac{\eta}{\theta^2} c\sqrt{\frac{\epsilon^3}{\rho}}\right)^{\frac{1}{2}})
% = O(\sqrt{\frac{c\epsilon}{\rho}})
% \end{align*}
% \end{proof}


\subsection{AGD can increase the Hamiltonian under nonconvexity}
\label{sec:counterex}
In the previous section, we proved Lemma \ref{lem:energy_nonconvex} which requires $\theta \ge 2\eta\gamma$, that is, $\gamma \le \theta/(2\eta)$.
In this section, we show Lemma \ref{lem:energy_nonconvex} is almost tight in the sense 
that when $\gamma \ge 4\theta/\eta$ in \eqref{eq:certificate}, we have:
\begin{equation*}
f(\x_t) \le  f(\y_t) + \la \grad f(\y_t), \x_t - \y_t \ra - \frac{\gamma}{2} \norm{\x_t - \y_t}^2.
\end{equation*}
Monotonic decrease of the Hamiltonian may no longer hold, indeed, AGD can increase the Hamiltonian for those steps.

Consider a simple one-dimensional example, $f(x) = -\frac{1}{2}\gamma x^2$, where \eqref{eq:certificate} always holds. Define the initial condition $x_0 = -1, v_0 = 1/(1-\theta)$. By update equation in Algorithm \ref{algo:AGD}, the next iterate will be $x_1 = y_0 = 0$, and $v_1 = x_1 - x_0 = 1$. By the definition of Hamiltonian, we have:
\begin{align*}
E_0 =& f(x_0) + \frac{1}{2\eta}|v_0|^2 = -\frac{\gamma}{2} + \frac{1}{2\eta (1-\theta)^2}\\
E_1 =& f(x_1) + \frac{1}{2\eta}|v_1|^2 = \frac{1}{2\eta},
\end{align*}
since $\theta \le 1/4$. It is not hard to verify that whenever $\gamma \ge 4 \theta/\eta$, we will have $E_1 \ge E_0$; that is, the Hamiltonian increases in this step.

This fact implies that when we pick a large learning rate $\eta$ and small momentum parameter $\theta$ (both are essential for acceleration), standard AGD does not decrease the Hamiltonian in a very nonconvex region. We need another mechanism such as NCE to fix the monotonically decreasing property.
