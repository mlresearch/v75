\documentclass[final,12pt]{colt2018} 

% The following packages will be automatically loaded:
% amsmath, amssymb, natbib, graphicx, url, algorithm2e

%%%
% Packages
\usepackage{amssymb,amsmath,sectsty,url}
%\usepackage[letterpaper,hmargin=1.0in,vmargin=1.0in]{geometry}
%\usepackage[dvips,colorlinks,linkcolor=blue,citecolor=blue,filecolor=blue,urlcolor=blue]{hyperref}
%\usepackage[pdftex,colorlinks,linkcolor=blue,citecolor=blue,filecolor=blue,urlcolor=blue]{hyperref}
%\usepackage[dvips,colorlinks,linkcolor=black,citecolor=black,filecolor=black,urlcolor=black]{hyperref}
%\usepackage[colorlinks,linkcolor=black,citecolor=black,filecolor=black,urlcolor=black]{hyperref}
\usepackage{cleveref}
\usepackage{color}
\usepackage[boxed]{algorithm}
%\usepackage[margin=20pt,font=small,labelfont=bf]{caption}
\pagestyle{plain}

\usepackage{tikz}
\usepackage{graphicx}
\usepackage{nicefrac}
\usepackage{aliascnt}

\newcommand{\aref}[1]{\autoref{#1}}

\numberwithin{theorem}{section}

%General
\newcommand{\abs}[1]{\left|#1\right|} 
\newcommand{\norm}[1]{\lVert#1\rVert}

%Probability
\def\E{\mathbb E}
\newcommand{\Exp}{\mathop{\mathrm E}\displaylimits} % expectation
\newcommand{\Var}{{\bf Var}}
\newcommand{\Cov}{{\bf Cov}}
\newcommand{\1}{{\rm 1\hspace*{-0.4ex}%

\rule{0.1ex}{1.52ex}\hspace*{0.2ex}}}

%Algebraic Structures
\newcommand{\U}{\mathbf U}
\newcommand{\N}{\mathbb N}
\newcommand{\R}{\mathbb R}
\newcommand{\Z}{\mathbb Z}
\newcommand{\C}{\mathbb C}
\newcommand{\F}{\mathbb F}
\newcommand{\GF}{\mathbb{GF}}
\newcommand{\B}{\{ 0,1 \}}
\newcommand{\BM}{\{ -1,1 \}}

% Misc
\newcommand\polylog[1]{\ensuremath{\mathrm{polylog}\left(#1\right)}}
\newcommand\argmin{\ensuremath{\mathrm{argmin}}}
\newcommand\qed{\hfill$\blacksquare$}

%%%

\newaliascnt{claim}{theorem}
\newtheorem{claim}[claim]{Claim}
\aliascntresetthe{claim}
\crefname{claim}{Claim}{Claims}

%\newaliascnt{lemma}{theorem}
%\newtheorem{lemma}[lemma]{Lemma}
%\aliascntresetthe{lemma}
%\crefname{lemma}{Lemma}{Lemmas}

\title[Approximate Nearest Neighbors in Limited  Space]{Approximate Nearest Neighbors in Limited  Space}
\usepackage{times}
 % Use \Name{Author Name} to specify the name.
 % If the surname contains spaces, enclose the surname
 % in braces, e.g. \Name{John {Smith Jones}} similarly
 % if the name has a "von" part, e.g \Name{Jane {de Winter}}.
 % If the first letter in the forenames is a diacritic
 % enclose the diacritic in braces, e.g. \Name{{\'E}louise Smith}

 % Two authors with the same address
  % \coltauthor{\Name{Author Name1} \Email{abc@sample.com}\and
  %  \Name{Author Name2} \Email{xyz@sample.com}\\
  %  \addr Address}

 % Three or more authors with the same address:
 % \coltauthor{\Name{Author Name1} \Email{an1@sample.com}\\
 %  \Name{Author Name2} \Email{an2@sample.com}\\
 %  \Name{Author Name3} \Email{an3@sample.com}\\
 %  \addr Address}


 % Authors with different addresses:
 \coltauthor{\Name{Piotr Indyk} \Email{indyk@mit.edu}\\
 \addr CSAIL, MIT
 \AND
 \Name{Tal Wagner} \Email{talw@mit.edu}\\
 \addr CSAIL, MIT
 }

\begin{document}

\maketitle

\begin{abstract}
We consider the $(1+\epsilon)$-approximate nearest neighbor search problem: given a set $X$ of $n$ points in a $d$-dimensional space, build a data structure that, given any query point  $y$,  finds a point $x \in X$ whose distance to $y$ is at most $(1+\epsilon) \min_{x \in X} \|x-y\|$ for an accuracy parameter $\epsilon \in (0,1)$.  Our main result is a data structure that occupies only $O(\epsilon^{-2} n \log(n) \log(1/\epsilon))$
bits of space, assuming all point coordinates are integers in the range  $\{-n^{O(1)} \ldots n^{O(1)}\}$, i.e., the coordinates have $O(\log n)$ bits of precision. This improves over the best previously known space bound of         $O(\epsilon^{-2} n \log(n)^2)$, obtained via the randomized dimensionality reduction method of \cite{johnson1984extensions}.  We also consider the more general problem of estimating all distances from a collection of query points to all data points $X$, and provide almost tight upper and lower bounds for the space complexity of this problem. 
\end{abstract}

\begin{keywords}
nearest neighbor, quantization, distance estimation, metric compression, distance sketches, dimension reduction
\end{keywords}

Online learning algorithms are a key tool in web search and content optimization, adaptively learning what users want to see. In a typical application, each time a user arrives, the algorithm chooses among various content presentation options (\eg news articles to display), the chosen content is presented to the user, and an outcome (\eg a click) is observed. Such algorithms must balance \emph{exploration} (making potentially suboptimal decisions now for the sake of acquiring information that will improve decisions in the future) and \emph{exploitation} (using information collected in the past to make better decisions now). Exploration could degrade the experience of a current user, but improves user experience in the long run. This exploration-exploitation tradeoff is commonly studied in the online learning framework of \emph{multi-armed bandits}~\citep{Bubeck-survey12}.

Concerns have been raised about whether exploration in such scenarios could be unfair, in the sense that some individuals or groups may experience too much of the downside of exploration without sufficient upside \citep{bird2016exploring}. We formally study these concerns in the \emph{linear contextual bandits} model~\citep{Langford-www10,chu2011contextual}, a standard variant of multi-armed bandits appropriate for content personalization scenarios.  We focus on \emph{externalities} arising due to exploration, that is, undesirable side effects that the presence of one party may impose on another.


We first examine the effects of exploration at a group level.  We introduce the notion of a \emph{group externality} in an online learning system, quantifying how much the presence of one population (which we dub the majority) impacts the rewards of another (the minority). We show that this impact can be negative, and that, in a particular precise sense, no algorithm can avoid it. This cannot be explained by the absence of suitably good policies since our adoption of the linear contextual bandits framework implies the existence of a feasible policy that is simultaneously optimal for everyone. Instead, the problem is inherent to the process of exploration. We come to a surprising conclusion that more data can sometimes lead to worse outcomes for the users of an explore-exploit-based system. \looseness=-1

We next turn to the effect of exploration at an individual level. We interpret exploration as a potential externality imposed on the current user by future users of the system. Indeed, it is only for the sake of the future users that the algorithm would forego the action that currently looks optimal. To avoid this externality, one may use the greedy algorithm that always chooses the action that appears optimal according to current estimates of the problem parameters. While this greedy algorithm performs poorly in the worst case,
it tends to work well in many applications and experiments.\footnote{Both positive and negative findings are folklore. One way to precisely state the negative result is that the greedy algorithm incurs constant per-round regret with constant probability; while results of this form have likely been known for decades,
\citet[Corollary A.2(b)]{competingBandits-itcs16}
proved this for a wide variety of scenarios. Very recently, the good empirical performance has been confirmed by state-of-art experiments in \citet{practicalCB-arxiv18}.}

In a new line of work, \citet{bastani2017exploiting} and \citet{kannan2018smoothed}
analyzed conditions under which inherent diversity in the data makes explicit exploration unnecessary.
\citet{kannan2018smoothed} proved that the greedy algorithm achieves a regret rate of
$\tilde{O}(\sqrt{T})$ in expectation over small perturbations of the context vectors (which ensure sufficient data diversity). This is the best rate that can be achieved in the worst case (\ie for all problem instances, without data diversity assumptions), but it leaves open the possibilities that (i) another algorithm may perform much better than the greedy algorithm on some problem instances, or (ii) the greedy algorithm may perform much better than worst case under the diversity conditions. We expand on this line of work. We prove that under the same diversity conditions, the greedy algorithm almost matches the best possible Bayesian regret rate of \emph{any} algorithm \emph{on the same problem instance}. This could be as low as $\polylog(T)$ for some instances, and, as we prove, at most $\tilde{O}(T^{1/3})$ whenever the diversity conditions hold.


Returning to group-level effects, we show that under the same diversity conditions, the negative group externalities imposed by the majority essentially vanish if one runs the greedy algorithm. Together, our results illustrate a sharp contrast between the high individual and group externalities that exist in the worst case, and the ability to remove all externalities if the data is sufficiently diverse.   \looseness=-1

\xhdr{Additional motivation.} Whether and when explicit exploration is necessary is an important concern in the study of the exploration-exploitation tradeoff. Fairness considerations aside, explicit exploration is expensive. It is wasteful and risky in the short term, it adds a layer of complexity to algorithm design \citep{Langford-nips07,monster-icml14}, and its adoption at scale tends to require substantial systems support and buy-in from management \citep{MWT-WhitePaper-2016,DS-arxiv}. A system based on the greedy algorithm would typically be cheaper to design and deploy.

Further, explicit exploration can run into incentive issues in applications such as recommender systems. Essentially, when it is up to the users which products or experiences to choose and the algorithm can only issue recommendations and ratings, an explore-exploit algorithm needs to provide incentives to explore for the sake of the future users \citep{Kremer-JPE14,Frazier-ec14,Che-13,ICexploration-ec15,Bimpikis-exploration-ms17}. Such incentive guarantees tend to come at the cost of decreased performance, and rely on assumptions about human behavior. The greedy algorithm avoids this problem as it is inherently consistent with the users' incentives.



\xhdr{Additional related work.}
Our research draws inspiration from the growing body of work on fairness in machine learning~\cite[e.g.,][]{dwork2012fairness,hardt2016equality,kleinberg2017inherent,chouldechova2017fair}.  Several other authors have studied fairness in the context of the contextual bandits framework.  Our work differs from the line of research on meritocratic fairness in online learning \citep{kearns2017meritocratic,liu2017calibrated,joseph2016fairness}, which considers the allocation of limited resources such as bank loans and requires that nobody should be passed over in favor of a less qualified applicant. We study a fundamentally different scenario in which there are no allocation constraints and we would like to serve each user the best content possible.  Our work also differs from that of \citet{celis2017fair}, who studied an alternative notion of fairness in the context of news recommendations. According to this notion, all users should have approximately the same probability of seeing a particular type of content (e.g., Republican-leaning articles), regardless of their individual preferences, in order to mitigate the possibility of discriminatory personalization.

The data diversity conditions in \citet{kannan2018smoothed} and this paper are inspired by the smoothed analysis framework of \citet{SmoothedAnalysis-jacm04}, who proved that the expected running time of the simplex algorithm is polynomial for perturbations of any initial problem instance (whereas the worst-case running time has long been known to be exponential). Such disparity implies that very bad problem instances are brittle. 
We find a similar disparity for the greedy algorithm in our setting.



\xhdr{Our results on group externalities.}  A typical goal in online learning is to minimize \emph{regret}, the (expected) difference between the cumulative reward that would have been obtained had the optimal policy been followed at every round and the cumulative reward obtained by the algorithm.  We define a corresponding notion of \emph{minority regret}, the portion of the regret experienced by the minority.  Since online learning algorithms update their behavior based on the history of their observations, minority regret is influenced by the entire population on which an algorithm is run.  If the minority regret is much higher when a particular algorithm is run on the full population than it is when the same algorithm is run on the minority alone, we can view the majority as imposing a negative externality on the minority; the minority population would achieve a higher cumulative reward if the majority were not present. Asking whether this can ever happen
amounts to asking whether access to more data points can ever lead an explore-exploit algorithm to make inferior decisions. One might think that more data should always lead to better decisions and therefore better outcomes for the users.
Surprisingly, we show that this is not the case, even with a standard algorithm.

Consider LinUCB~\citep{Langford-www10,chu2011contextual,abbasi2011improved}, a standard algorithm for linear contextual bandits that is based on the principle of ``optimism under uncertainty.''  We provide a specific problem instance on which, after observing $T$ users, LinUCB would have a minority regret of $\Omega(\sqrt T)$ if run on the full population, but only constant minority regret if run on the minority alone. While stylized, this example is motivated by the problem of providing driving directions to different populations of users, about which fairness concerns have been raised~\citep{bird2016exploring}. Further, the situation is reversed on a slight variation of this example: LinUCB obtains constant minority regret when run on the full population and $\Omega(\sqrt T)$ on the minority alone.  That is, group externalities can be large and positive in some cases, and large and negative in others.

Although these regret rates are specific to LinUCB, we show that this phenomenon is, in some sense, unavoidable. Consider the minority regret of LinUCB when run on the full population and the minority regret that LinUCB would incur if run on the minority alone. We know that one may be much smaller or larger than the other. We ask whether any algorithm could  achieve the minimum of the two on every problem instance. Using a variation of the same problem instance, we prove that this is impossible; in fact, no algorithm could simultaneously approximate both up to any $o(\sqrt{T})$ factor. In other words, an externality-free algorithm would sometimes ``leave money on the table."


In terms of techniques, we rely on the special structure of our example, which can be viewed as an instance of the sleeping bandits problem~\citep{SleepingBandits-ml10}. This simplifies the behavior and analysis of LinUCB, allowing us to obtain the $O(1)$ upper bounds.  The lower bounds are obtained using KL-divergence techniques to show that the two variants of our example are essentially indistinguishable, and an algorithm that performs well on one must obtain $\Omega(\sqrt{T})$ regret on the other. \looseness=-1


\xhdr{Our results on the greedy algorithm.} We consider a version of linear contextual bandits in which the latent weight vector $\theta$ is drawn from a known prior. In each round, an algorithm is presented several actions to choose from, each represented by a \emph{context vector}. The expected reward of an action is a linear product of $\theta$ and the corresponding context vector. The tuple of context vectors is drawn independently from a fixed distribution. In the spirit of smoothed analysis, we assume that this distribution has a small amount of jitter. Formally, the tuple of context vectors is drawn from some fixed distribution, and then a small \emph{perturbation}---small-variance Gaussian noise---is added independently to each coordinate of each context vector. This ensures arriving contexts are diverse. We are interested in Bayesian regret, i.e., regret in expectation over the Bayesian prior. Following the literature, we are primarily interested in the dependence on the time horizon $T$. \looseness=-1

We focus on a batched version of the greedy algorithm, in which new data arrives to the algorithm's optimization routine in small batches, rather than every round. This is well-motivated from a practical perspective---in high-volume applications data usually arrives to the ``learner" only after a substantial delay \citep{MWT-WhitePaper-2016,DS-arxiv}---and is essential for our analysis.

Our main result is that the greedy algorithm matches the Bayesian regret of any algorithm up to polylogarithmic factors, for each problem instance, fixing the Bayesian prior and the context distribution. We also prove that LinUCB achieves regret $\tilde{O}(T^{1/3})$ for each realization of $\theta$. This implies a worst-case Bayesian regret of $\tilde{O}(T^{1/3})$ for the greedy algorithm under the perturbation assumption. \looseness=-1

Our results hold for both natural versions of the batched greedy algorithm, Bayesian and frequentist, henceforth called \BayesGreedy and \FreqGreedy. In \BayesGreedy, the chosen action maximizes expected reward according to the Bayesian posterior. \FreqGreedy estimates $\theta$ using ordinary least squares regression and chooses the best action according to this estimate. The results for \FreqGreedy come with additive polylogarithmic factors, but are stronger in that the algorithm does not need to know the prior. Further, the $\tilde{O}(T^{1/3})$ regret bound for \FreqGreedy is approximately prior-independent, in the sense that it applies even to very concentrated priors such as independent Gaussians with standard deviation on the order of $T^{-2/3}$.

The key insight in our analysis of \BayesGreedy is that any (perturbed) data can be used to simulate any other data, with some discount factor. The analysis of \FreqGreedy requires an additional layer of complexity. We consider a hypothetical algorithm that receives the same data as \FreqGreedy, but chooses actions based on the Bayesian-greedy selection rule. We analyze this hypothetical algorithm using the same technique as \BayesGreedy, and then upper bound the difference in Bayesian regret between the hypothetical algorithm and \FreqGreedy.

Our analyses extend to group externalities and (Bayesian) minority regret. In particular, we circumvent the impossibility result mentioned above. We prove that both \BayesGreedy and \FreqGreedy match the Bayesian minority regret of any algorithm run on either the full population or the minority alone, up to polylogarithmic factors

\xhdr{Detailed comparison with prior work.} We substantially improve over the $\tilde{O}(\sqrt{T})$ worst-case regret bound from \citet{kannan2018smoothed}, at the cost of some additional assumptions. First, we consider Bayesian regret, whereas their regret bound is for each realization of $\theta$.%
\footnote{Equivalently, they allow point priors, whereas our priors must have variance $T^{-O(1)}$.} Second, they allow the context vectors to be chosen by an adversary before the perturbation is applied. Third, they extend their analysis to a somewhat more general model, in which there is a separate latent weight vector for every action (which amounts to a more restrictive model of perturbations). However, this extension relies on the greedy algorithm being initialized with a substantial amount of data. The results of \citet{kannan2018smoothed} do not appear to have implications on group externalities.

\citet{bastani2017exploiting} show that the greedy algorithm achieves logarithmic regret in an alternative linear contextual bandits setting that is incomparable to ours in several important ways.
They consider two-action instances where the actions share a common context vector in each round, but are parameterized by different latent vectors. They ensure data diversity via a strong assumption on the context distribution. This assumption does not follow from our perturbation conditions; among other things, it implies that each action is the best action in a constant fraction of rounds. Further, they assume a version of Tsybakov's \emph{margin condition}, which is known to substantially reduce regret rates in bandit problems \citep[\eg see][]{Zeevi-colt10}.


\input{setting}
\input{basic}
\input{ann}
%We begin by considering the following simple question: how close is the behavior of two given Markov chains $P$ and $Q$?
%A natural notion of distance would tell us how easy it is to distinguish which Markov chain $P$ or $Q$ a word $w=s_0\to s_1\cdots\to s_\ell$ of certain length $\ell$ was generated from. 
Given two Markov chains $P$ and $Q$, we want to come up with a distance notion which captures how easy it is to distinguish which Markov chain $P$ or $Q$ a word $w=s_0\to s_1\cdots\to s_\ell$ of certain length $\ell$ was generated from (while being agnostic to the distribution of $s_0$). 
This distinguishability is precisely captured by the TV distance $\dtv{\word{P}{\ell}}{\word{Q}{\ell}}$ between {\em word distributions} 
$\word{P}{\ell}$, $\word{Q}{\ell}$ for words of length $\ell$ generated by Markov chains $P$ and $Q$ respectively. It is more convenient in our setting to use, instead of total variation distance, the square of the Hellinger distance $\hellingersq{\word{P}{\ell}}{\word{Q}{\ell}}$
or the closely related Bhattacharya coefficient\footnote{Hellinger distance is tightly related to the Bhattacharya coefficient between two distributions which is defined as
$BC(p,q) = \sum_{i \in [k]} \sqrt{p_i\cdot q_i}$. It captures similarity of two distributions and lies in $[0,1]$.}, which is useful for studying 
divergence of non-stationary and continuous Markov chains as was observed in~\cite{Kazakos78}. \cite{Kazakos78} establishes
nice recurrence relations for the Bhattacharya coefficient of two word distributions, which is captured by the matrix 
$\srprod{P}{Q}\eqdef\left[\sqrt{P_{ij}\cdot Q_{ij}}~ \right]_{i,j\in[n\times n]}$. %(see Appendix~\ref{app:sec_dist} for derivation of \eqref{eq:hellinger_square_algebraic} and missing proofs of the claims in this section)
%Precise calculation of $1-\hellingersq{\word{P}{\ell}}{\word{Q}{\ell}}$. 
\begin{lemma}[\cite{Kazakos78}] \label{lemma:kazakos lemma}
Suppose $P$ and $Q$ are Markov Chains over states $[n]$, $\vect{p}$ and $\vect{q}$ are probability distributions of the initial state. Let $\word{P}{\ell}$, $\word{Q}{\ell}$ 
be the distributions denoting a length $\ell$ trajectory of Markov Chains $P$ (resp. $Q$) starting at a random node $s_0$ sampled from $\vec{p}$ (resp. $\vec{q}$). Moreover, define 
the vector $\srprod{\vect{p}}{\vect{q}}\eqdef\left[\sqrt{p_s\cdot q_s}\right]_{s\in[n]}$ and the matrix $\srprod{P}{Q}\eqdef\left[\sqrt{P_{ij}\cdot Q_{ij}}~ \right]_{i,j\in[n\times n]}$. Then:
\be
\label{eq:hellinger_square_algebraic}
1-\hellingersq{\word{P}{\ell}}{\word{Q}{\ell}}=\srprodt{\vect{p}}{\vect{q}}\circ \left(\srprod{P}{Q}\right)^{\ell} \circ \onev,
\ee
\end{lemma}

%\begin{prevproof}{Lemma}{lemma:kazakos lemma}
%\begin{multline*}
%1-\hellingersq{\word{P}{\ell}}{\word{Q}{\ell}}=\sum_{w=s_0\ldots s_\ell}\sqrt{\Prlong[P]{w}\Prlong[Q]{w}}
%=\trans{\left[\sum_{\substack{w=s_0\ldots s_{\ell}\\s_\ell=s}}\sqrt{\Prlong[P]{w}\Prlong[Q]{w}}\right]}_{s\in[n]}\circ\onev\\
%=\trans{\left[\sum_{r\in[n]}\sqrt{\Prlong[P]{r\to s}\Prlong[Q]{r\to s}}\sum_{\substack{w=s_0\ldots s_{\ell-1}\\s_{\ell-1}=r}}\sqrt{\Prlong[P]{w}\Prlong[Q]{w}}\right]}_{s\in[n]}\circ\onev\\
%=\trans{\left[\sum_{\substack{w=s_0\ldots s_{\ell-1}\\s_{\ell-1}=r}}\sqrt{\Prlong[P]{w}\Prlong[Q]{w}}\right]}_{r\in[n]}\circ
%\begin{bmatrix}
%&\vdots&\\
%\cdots&\sqrt{P_{rs}\cdot Q_{rs}}&\cdots\\
%&\vdots&
%\end{bmatrix}_{r,s\in[n\times n]}
%\circ\onev
%\\
%=\trans{\left[\sum_{\substack{w=s_0\ldots s_{\ell-1}\\s_{\ell-1}=r}}\sqrt{\Prlong[P]{w}\Prlong[Q]{w}}\right]}_{r\in[n]}\circ\srprod{P}{Q}\circ\onev
%=\srprodt{\vect{p}}{\vect{q}}\circ \left(\srprod{P}{Q}\right)^{\ell} \circ \onev,
%%\label{eq:hellinger_square_algebraic}
%\end{multline*}
%\end{prevproof}
%An important observation is that the distance between $\word{P}{\ell}$ and $\word{Q}{\ell}$ above depends on the initial distribution of the first state in $w$, and also the length 
%$\ell$ of the word. 
There are two important parameters which affect the expression given by \cite{Kazakos78}. The first is the distributions of the starting states of the Markov chains ($\vect{p}, \vect{q}$) and 
the second is the length of the word ($l$). We want a notion of distance which is a scale-free non-negative real number. To achieve this, we study next how to eliminate 
%from the expression given by \cite{Kazakos78}, 
the dependencies on the starting state distributions ($\vect{p},\vect{q}$) and the word length ($l$).
\paragraph{Assumption on the starting state.} We study two scenarios for the choice of the starting state: (i) a {\bf worst-case} scenario where
both $P$ and $Q$ begin from the same state $i$ chosen in adversarial manner to make $P$ and $Q$ look as much alike as possible;
(ii) an {\bf average-case} scenario, where the initial distributions $\vect{p}=\vect{q}$ for $P$ and $Q$ either are given to us, or are related to $P$ and $Q$ in 
some natural way\footnote{For example $\vect{p}$ and $\vect{q}$ could be respective stationary distributions of $P$ and $Q$. However, we still assume identical initial distributions for $P$ and $Q$, i.e. $\vect{p}=\vect{q}$, as otherwise there might be a simpler trivial strategy to distinguish $P$ and $Q$ by observing only one initial sample from $\vect{p}$. Example~\ref{fig:example3} illustrates how 
two Markov chains can produce very similar distributions of words $\word{P}{\ell},\word{Q}{\ell}$ starting from any state for some large $\ell$, and yet have vastly 
different stationary distributions.}. 
Given the assumption on the starting state we want to answer the question of what $\ell$ to pick, so 
that $\word{P}{\ell}$ and $\word{Q}{\ell}$ are far apart in squared Hellinger distance (say $\ge 0.5$). 
Formally, we have the following respectively for the worst-case and average-case scenarios listed above:
%The worst-case and average-case scenarios can be formu we respectfully get 
\begin{align}
\label{eq:forall_states_eigenvalue}
\min_{\ell>0} \quad \ell:& \quad\quad \forall i\in[n] && 0.5 \ge 1-\hellingersq{\word{P}{\ell}}{\word{Q}{\ell}} = \onevti\circ \left(\srprod{P}{Q}\right)^{\ell} \circ \onev .\\
\min_{\ell>0} \quad \ell:  & && 0.5 \ge 1-\hellingersq{\word{P}{\ell}}{\word{Q}{\ell}} = \srprodt{\vect{p}}{\vect{q}}\circ \left(\srprod{P}{Q}\right)^{\ell} \circ \onev \nonumber
\end{align}
Due to the relation between Hellinger and total variation distances, an inequality similar to~\eqref{eq:forall_states_eigenvalue} holds
for $1-\dtv{\word{P}{\ell}}{\word{Q}{\ell}}$ as well but with a different constant on the left.\\

We call the minimal $\ell$ that satisfies $\dtv{\word{P}{\ell}}{\word{Q}{\ell}}\ge\frac{2}{3}$ for all starting states $i\in[n]$ (or for fixed starting distributions $\vect{p}=\vect{q}$)
the {\em minimal distinguishing length}. We note that \eqref{eq:forall_states_eigenvalue} gives us an estimate on $\ell$ up to a constant factor.

%Is there single natural parameter that captures closeness between $P$ and $Q$ without dependency on $\ell$ and initial state? 
Next we argue that when $\ell$ is large, the behavior of the RHS of~\eqref{eq:forall_states_eigenvalue} is governed by {\em the largest eigenvalue} 
$\eigi[1]=\specr{\srprod{P}{Q}}$ of $\srprod{P}{Q}$. 
%In a long run when $\ell\to\infty$ the parameter $\eigi[1]$ captures the rate at which the similarity between the two word distributions decreases (implying an increase in our ability to distinguish the two chains). 
In particular, by Perron-Frobenius theorem, we have that the largest eigenvalue of $\srprod{P}{Q}$ is non-negative and 
the corresponding left eigenvector $\eigvli[1]: \eigvlit[1]\circ\srprod{P}{Q}=\eigi[1]\cdot\eigvlit[1]$ 
has non-negative coordinates. In particular, if we choose initial distributions $\vect{p} = \vect{q}$ proportional to $\eigvli[1]$, then
\be
\trans{\vect{p}}\circ\left(\srprod{P}{Q}\right)^{\ell}\circ\onev=\eigi[1]^\ell\cdot\scalprod{\vect{p}}{\onev}=\eigi[1]^\ell.
\label{eq:largest_eigenvalue}
\ee

\begin{claim}
It is always true that $\eigi[1]=\specr{\srprod{P}{Q}}\le 1$. Moreover, $\eigi[1]=1$ iff $P$ and $Q$ have an identical essential communicating class.
\label{cl:eigval_less_than_one}
\end{claim} 
Proof of Claim~\ref{cl:eigval_less_than_one} is defered to Appendix~\ref{app:proofs_dist}.\\

We propose the use of the quantity $1-\specr{\srprod{P}{Q}}$ as a distance measure between Markov chains $P$ and $Q$. 
\begin{description}
\label{def:distance}
\item[Definition:] $\dist{P}{Q} \eqdef 1-\specr{\srprod{P}{Q}}$.
\end{description}
In particular in \eqref{eq:forall_states_eigenvalue} if $\vect{p} = \vect{q}$ is proportional to $\eigvli[1]$, then 
$\ell\cdot\ln(1-\eps)\le\ln 0.5\implies\ell\ge\frac{\ln 2}{2\eps}$. This shows that in the worst-case we need to observe a trajectory of length at least $\Omega(1/\eps)$  before we can satisfactorily distinguish the two chains.
%Also, $\eigi[1]=1$ means that $P$ and $Q$ are indistinguishable at least for a starting state in some connected component of $P$ and $Q$.
Note however that, in general, $\ell$ might need to be larger than 
$\Omega(\frac{1}{\eps})$ as is illustrated in Example~\ref{fig:example2}. However, we will see that in the case of symmetric Markov chains we observe a more regular behavior.
In the remainder of this section and the following sections we only consider symmetric Markov chains that avoid such irregular behavior and 
dependency on the starting state.


\paragraph{Word distance between Symmetric Markov Chains.} The stationary distribution for any symmetric Markov chain is the uniform distribution over all states. 
In this case the most natural starting distributions for the average-case part of equation~\eqref{eq:forall_states_eigenvalue} are $\vect{p}=\vect{q}=\frac{1}{n}\onev$.
%\be
%\min_{\ell>0} \quad\ell: \quad\quad\quad 0.9\ge 1-\hellingersq{\word{P}{\ell}}{\word{Q}{\ell}} = \frac{1}{n}\onevt\circ \left(\srprod{P}{Q}\right)^{\ell} \circ \onev.
%\label{eq:average_states_eigenvalue}
%\ee
%When we start observing a phenomenon 
%obeying Markov Chain $P$, it is reasonable instead of the worst-case assumption on the initial state assume that initial state is distributed 
%according to the stationary distribution of $P$ and measure the distance between $P$ and $Q$ for a random starting state. A natural question here is 
%what $\ell$ is enough to distinguish $P$ and $Q$ with significant probability, i.e.,
%\[
%\frac{1}{2}\ge 1-\hellingersq{\word{P}{\ell+1}}{\word{Q}{\ell+1}} = \frac{1}{n}\onevt\circ \left(\srprod{P}{Q}\right)^{\ell}\circ\onev
%\]
In this setting of symmetric Markov chains, we can provide sharp bounds on the minimal distinguishing length $\ell$.
\begin{claim}
The necessary and sufficient distinguishing length $\ell$, which allows to distinguish $P$ vs. $Q$ with high probability, 
is $\wTheta{\frac{1}{\eps}}$ (up to a $\log n$ factor), where $\eps=1-\specr{\srprod{P}{Q}}$ under both worst-case and average-case (we assume
$\vect{p}=\vect{q}=\frac{1}{n}\onev$) scenarios for the starting state.
\label{cl:symm_spectrum}
\end{claim}
Proof of Claim~\ref{cl:symm_spectrum} is given in Appendix~\ref{app:proofs_dist}.

%\label{cl:symm_spectrum_worst-case}
We note that, if one could pick the starting state instead of working with average-case or worst-case assumptions of Claim~\ref{cl:symm_spectrum}, then $\ell$ can be much smaller (see Example~\ref{fig:example5}). Claim~\ref{cl:symm_spectrum} gives a strong evidence that $1-\specr{\srprod{P}{Q}}$ 
is a meaningful and important parameter that captures closeness between $P$ and $Q$. In the following section we will use it as analytical proxy for the distance between 
Markov Chains\footnote{In general this notion of distance should be used with care. For instance, note that $\dist{P}{Q}=1-\specr{\srprod{P}{Q}}$, is not a metric. In particular, $
\dist{P}{Q}$ violates the triangle inequality ($\dist{M_1}{M_2}=\dist{M_2}{M_3}=0,$ but $\dist{M_1}{M_3}>0$ for some $M_1,M_2,M_3$) as is illustrated by Example~\ref{fig:example1}. 
We note that this problem can only appear for reducible chains, as is shown in Claim~\ref{cl:eigval_less_than_one}. Also it is not always possible to extend the sharp bounds on $\ell$ of 
Claim~\ref{cl:symm_spectrum} from symmetric Markov chains to non-symmetric Markov chains, even if both MC have the uniform distribution as their stationary distribution (see Example~\ref{fig:example4})
}.





%
%
%
%For both General and Symmetric Markov chains, $\eps=1-\specr{\srprod{P}{Q}}$ is a meaningful and important parameter that captures closeness between $P$ and $Q$.
%In the following sections we will use it as proxy for the distance between Markov Chains. 
%
%
%\paragraph{Average-case starting state.} Another reasonable approach is to assume that both Markov chains $P$ and $Q$ had enough time to mix before 
%our observation and, therefore, the initial distributions $\vect{p},\vect{q}$ for the starting state in $P$ and $Q$ are the respective stationary distributions.  
%
%
%
%However, there is a simple condition that quantifies how big the gap between the lower bound
%on $\ell$ from \eqref{eq:largest_eigenvalue} and an upper bound that is sufficient for \eqref{eq:forall_states_eigenvalue} could be: if the right eigenvector $\eigvi[1]$ of $\srprod{P}{Q}$
%($\srprod{P}{Q}\circ\eigvi[1]=\eigi[1]\cdot\eigvi[1]$) has bounded gap between its largest and smallest coordinates, i.e., 
%$\frac{\max_{i}\iprod{\eigvi[1]}{\onevi}}{\min_{j}\iprod{\eigvi[1]}{\onevi[j]}}\le\kappa$, then 
%\begin{multline*}
%\forall i\in[n]\quad\quad\quad\quad \onevti\circ \left(\srprod{P}{Q}\right)^{\ell}\circ\onev  \le  
%\onevti\circ\left(\srprod{P}{Q}\right)^{\ell}\circ\frac{\eigvi[1]}{\min_{j}\iprod{\eigvi[1]}{\onevi[j]}}\\
%=\frac{\iprod{\onevi}{\eigvi[1]}\eigi[1]^\ell}{\min_{j}\iprod{\eigvi[1]}{\onevi[j]}}
%\le\frac{\eigi[1]^\ell\max_{i}\iprod{\onevi}{\eigvi[1]}}{\min_{j}\iprod{\eigvi[1]}{\onevi[j]}}=\kappa\cdot\eigi[1]^\ell
%\end{multline*}
%
%
%%We present a novel definition of distance between two Markov chains which captures the difference in the words produced by the chains. 
%%Let $P$ and $Q$ be two Markov chains defined on a set $S$ of $n$ states. We also use $P$ to refer to the transition 
%%matrix of the Markov chain $P$. 
%%
%%Consider the matrix $\srprod{P}{Q}$. The distance between Markov chains $P$ and $Q$ is defined as $1-\specr{\srprod{P}{Q}}$. 
%
%
%\nick{Selling new distance (plan):}
%\begin{itemize}
%\item Given two Markov Chains $P$, $Q$ how to distinguish them with $m$ consequent samples?
%%We want a notion of distance that allows us to say which MC $P$ or $Q$ the word $w$ of certain length $\ell$ was generated from. This is precisely captured by total variation 
%%distance $\dtv{\word{P}{\ell}}{\word{Q}{\ell}}$ between word distributions of length $\ell$. This notion depends on the initial distribution of the first state in $w$, and 
%%also length $\ell$ of the word. Is there single natural parameter that captures closeness between $P$ and $Q$ without dependency on $\ell$ and initial state? 
%%\item As a proxy for $\dtv{\word{P}{\ell+1}}{\word{Q}{\ell+1}}$ we use $\hellingersq{\word{P}{\ell+1}}{\word{Q}{\ell+1}}$. 
%%Precise calculation of $1-\hellingersq{\word{P}{\ell+1}}{\word{Q}{\ell+1}}$. \nick{See section~\ref{sec:hellinger_sq_precise}.}
%%\begin{remark}
%%The Bhattacharya coefficient between two probability distributions $p$ and $q$ supported over $X$ is defined as
%%$BC(p,q) = \sum_{x \in X} \sqrt{p(x)q(x)}$
%%The Bhattacharya coefficient captures the similarity of two distributions under the Hellinger squared distance. 
%%We will show that Hellinger similarity between two Markov chains has a nice form which is captured by the matrix $\srprod{P}{Q}\eqdef\left[\sqrt{P_{ij}\cdot Q_{ij}}~ \right]_{i,j\in[n\times n]}$.
%%\end{remark} 
%
%%For the regime when $\ell$ is large, the behavior of RHS of the above expression is governed by the largest eigenvalue $\eigi[1]$ of 
%%$\srprod{P}{Q}$. In a long run when $\ell\to\infty$ the parameter $\eigi[1]$ captures the rate at which we can distinguish $P$ and $Q$.
%%We recall that by Perron-Frobenius theorem the largest eigenvalue of $\srprod{P}{Q}$ is positive (assuming non-degeneracy conditions on $\srprod{P}{Q}$) 
%%and corresponding eigenvector $\eigvli[1]: \eigvlit[1]\circ\srprod{P}{Q}=\eigi[1]\cdot\eigvlit[1]$ has all positive coordinates. 
%%In particular, if initial state is chosen from the distribution $\vect{p}$ proportional to $\eigvli[1]$, then
%%\be
%%\trans{\vect{p}}\circ\left(\srprod{P}{Q}\right)^{\ell}\circ\onev=\eigi[1]^\ell\cdot\scalprod{\vect{p}}{\onev}=\eigi[1]^\ell\le \frac{1}{2},
%%\label{eq:largest_eigenvalue}
%%\ee
%%If $\eigi[1]=1-\eps$ it means that $\ell\cdot\log(1-\eps)\le\log\frac{1}{2}\implies\ell\ge\frac{\log 2}{2\eps}$. Note that in general $\ell$ might need
%%to be much larger than this bound, as is illustrated in Example 1. However, there is a simple condition that quantifies how big the gap between the lower bound
%%on $\ell$ from \eqref{eq:largest_eigenvalue} and an upper bound that is sufficient for \eqref{eq:forall_states_eigenvalue} could be: if the right eigenvector $\eigvi[1]$ of $\srprod{P}{Q}$
%%($\srprod{P}{Q}\circ\eigvi[1]=\eigi[1]\cdot\eigvi[1]$) has bounded gap between its largest and smallest coordinates, i.e., 
%%$\frac{\max_{i}\iprod{\eigvi[1]}{\onevi}}{\min_{j}\iprod{\eigvi[1]}{\onevi[j]}}\le\kappa$, then 
%%\begin{multline*}
%%\forall i\in[n]\quad\quad\quad\quad \onevti\circ \left(\srprod{P}{Q}\right)^{\ell}\circ\onev  \le  
%%\onevti\circ\left(\srprod{P}{Q}\right)^{\ell}\circ\frac{\eigvi[1]}{\min_{j}\iprod{\eigvi[1]}{\onevi[j]}}\\
%%=\frac{\iprod{\onevi}{\eigvi[1]}\eigi[1]^\ell}{\min_{j}\iprod{\eigvi[1]}{\onevi[j]}}
%%\le\frac{\eigi[1]^\ell\max_{i}\iprod{\onevi}{\eigvi[1]}}{\min_{j}\iprod{\eigvi[1]}{\onevi[j]}}=\kappa\cdot\eigi[1]^\ell
%%\end{multline*}
   %
%%where $V\circ\Lambda\circ \inv{V}$ is eigenvalue decomposition of matrix $\srprod{P}{Q}$: columns of $V$ are right eigenvectors of $\srprod{P}{Q}$ and $\Lambda$ is diagonal matrix
%%with eigenvalues of $\srprod{P}{Q}$; $\eigvi$ and $\eigvli$ are corresponding to $\eigi$ right and left eigenvectors of $\srprod{P}{Q}$ normalized such that $\scalprod{\eigvli}{\eigvi}=1$. 
 %%
%%columns of matrix $U$ are composed of left eigenvectors $\eigvlit:\eigvlit\circ\srprod{P}{Q}=\eigi\cdot\eigvlit$, columns of matrix $V$ are right eigenvectors 
%%$\eigvi:\srprod{P}{Q}\circ\eigvi=\eigi\cdot\eigvi$, and $\Lambda$ is diagonal matrix composed of eigenvalues of $\srprod{P}{Q}$.
%%
%%first inequality holds true as $\scalprod{\eigvi}{\eigvi}=\twonorm{\eigvi}=1$, to get the first equality
%%we just rewrite $\srprodt{\vect{p}}{\vect{q}}$ in the eigenvector basis of $\srprod{P}{Q}$.
%
%\item Symmetric Markov Chains. 
%%In this case uniform distribution is stationary for both $P$ and $Q$. Hence, when we start observing a phenomenon 
%%obeying Markov Chain $P$, it is reasonable instead of the worst-case assumption on the initial state assume that initial state is distributed 
%%according to the stationary distribution of $P$ and measure the distance between $P$ and $Q$ for a random starting state. A natural question here is 
%%what $\ell$ is enough to distinguish $P$ and $Q$ with significant probability, i.e.,
%%\[
%%\frac{1}{2}\ge 1-\hellingersq{\word{P}{\ell+1}}{\word{Q}{\ell+1}} = \frac{1}{n}\onevt\circ \left(\srprod{P}{Q}\right)^{\ell}\circ\onev
%%\]
%%Note that $\srprod{P}{Q}$ is symmetric matrix and therefore we have
%%\[
%%\frac{1}{n}\onevt\circ \left(\srprod{P}{Q}\right)^{\ell}\circ\onev=\frac{1}{n}\onevt\circ\left(\sum_{i=1}^n\eigi\cdot\eigvi\circ\eigvit\right)^\ell\circ\onev
%%=\sum_{i=1}^n\eigi^\ell\cdot\frac{1}{n}\iprod{\onev}{\eigvi}^2=(*)
%%\]
%%Now we can write upper and lower bound on $(*)$ in terms of $\eigi[1]^\ell$ (assume that $\ell$ is even):
%%\begin{multline*}
%%\frac{\eigi[1]^\ell}{n}=\frac{\eigi[1]^\ell}{n}\twonorm{\eigvi[1]}^2\le\eigi[1]^\ell\cdot\frac{1}{n}\onenorm{\eigvi[1]}^2
%%\le(*)\le
%%\sum_{i=1}^n\eigi^\ell\cdot\frac{1}{n}\onenorm{\eigvi}^2\le\sum_{i=1}^n\eigi^\ell\cdot\twonorm{\eigvi}^2=\sum_{i=1}^n\eigi^\ell\le n\cdot\eigi[1]^\ell,  
%%\end{multline*}
%%where in the second inequality we used Perron-Frobenius theorem stating that all coordinates of $\eigvi[1]$ are non negative. 
%%Consequently, these bounds imply that $\ell=\Theta\left(\frac{1}{\eps}\right)$ up to a $\log n$ factor, if $\eigi[1]=1-\eps$. I.e.,
%%$\ell=\wTheta{\frac{1}{\eps}}$. Noticeably, similar upper bound on $\ell$ still holds for the worst-case starting state. In this case we have:
%%$(*)=\onevti\circ \left(\srprod{P}{Q}\right)^{\ell}\circ\onev$
%%\[
%%(*)=\sum_{i=1}^n\eigi^\ell\cdot\iprod{\onevi}{\eigvi}\cdot\iprod{\onev}{\eigvi}
%%\le\sum_{i=1}^n\eigi^\ell\cdot\infnorm{\eigvi}\cdot\onenorm{\eigvi}
%%\le\sum_{i=1}^n\eigi^\ell\cdot n\twonorm{\eigvi}^2\le n^2\cdot\eigi[1]^\ell.
%%\]
%%
%%For both General and Symmetric Markov chains, $\eps=1-\specr{\srprod{P}{Q}}$ is a meaningful and important parameter that captures closeness between $P$ and $Q$.
%%In the following sections we will use it as proxy for the distance between Markov Chains. 
%\item Time dependent Markov Chains. We observe a few i.i.d. runs.
%\end{itemize}
%
%
%\subsection{Bhattacharya coefficient}

%\paragraph{Fixed word length.} In some applications the length $\ell$ of the observed word might be given a priori. One such example corresponds to card riffle shuffle, 
%where the random choices in the process can be described (see Section~\ref{sec:shuffle} for more detail) as a Markov chain over $O(n^2)$ states ($n=52$ for the card deck), where the
%process terminates after $\ell=n$ steps. In this case we can expect a few i.i.d. samples of length-$\ell$ words. For such examples and more generally for the Markov Chains with 
%a specified number of steps $T$ it is natural to define
%$
%\dist{P}{Q}\eqdef\dtv{\word{P}{T}}{\word{Q}{T}}.
%%\label{eq:def_dist_fixed_time}
%$
%Note that now the distance $\dist{P}{Q}$ satisfies triangle inequality. Moreover, due to the relation between Hellinger and total variation distances we 
%can estimate $1-\frac{\distsq{P}{Q}}{2}\ge 1-\hellingersq{\word{P}{T}}{\word{Q}{T}}$, where the RHS term admits a nice analytical expression similar 
%to \eqref{eq:hellinger_square_algebraic}.



%\subsection{Hellinger Squared Distance between the Word Distributions}
%\label{sec:hellinger_sq_precise}
%Consider any Markov chain $P$. The sequence of states seen in any given run of $P$ is called a word generated by $P$. 
%The words of length $t$ generated by Markov chain $P$ have a distribution which is denoted by $\word{P}{t}$. We will 
%now derive an expression for $\hellinger{\word{P}{t}}{\word{Q}{t}}$.  We define $\hellinger{\word{P}{t}}{\word{Q}{t}}_s$ as follows
%$$HS_s\left(\word{P}{t},\word{Q}{t}\right) = \sum_{\substack{w~:~s_1\ldots s_t, \\ \text{s.t. } w_t=s}}  \sqrt{\Pr_P(w)\Pr_Q(w)}$$
%Note that the Hellinger squared similarity (or Bhattacharya coefficient) is,
%$$1-\hellingersq{\word{P}{t}}{\word{Q}{t}} = \sum_{s \in S} HS_s\left(\word{P}{t},\word{Q}{t}\right).$$
%We also define the $n$-vector $\vec{HS}\left(\word{P}{t},\word{Q}{t}\right)$ as,
%$$\vec{HS}\left(\word{P}{t},\word{Q}{t}\right) = \left(HS_s\left(\word{P}{t},\word{Q}{t}\right) \right)_{s \in S}.$$
%Therefore 
%$$1-\hellingersq{\word{P}{t}}{\word{Q}{t}} = \vec{HS}\left(\word{P}{t},\word{Q}{t}\right)^T \circ \onev.$$
%With these definitions in mind, we begin by considering $HS_s\left(\word{P}{t},\word{Q}{t}\right)$ for some state $s \in S$.
%\begin{eqnarray}
%&& HS_s\left(\word{P}{t},\word{Q}{t}\right) = \sum_{\substack{w~:~s_1\ldots s_t, \\ \text{s.t. } s_t=s}}  \sqrt{\Pr_P(w)\Pr_Q(w)} \\
%&=& \sum_{w~:~s_1\ldots s_{t-1}}  \sqrt{\Pr_P(s_1\ldots s_{t-1})\Pr_Q(s_1\ldots s_{t-1})P(s_{t-1},s)Q(s_{t-1},s)} \\
%&=& \vec{HS}\trans{\left(\word{P}{t-1},\word{Q}{t-1}\right)} \circ \srprod{P}{Q}^{(s)} \\
%\end{eqnarray}
%where $\srprod{P}{Q}^{(s)}$ denotes the $s^{th}$ column of the $\srprod{P}{Q}$ matrix. Therefore we arrive at the following recurrence.
%\begin{eqnarray*}
%\vec{HS}\trans{\left(\word{P}{t},\word{Q}{t}\right)} = \vec{HS}\trans{\left(\word{P}{t-1},\word{Q}{t-1}\right)} \circ \srprod{P}{Q}
%\end{eqnarray*}
%The above recurrence implies,
%$$\vec{HS}\trans{\left(\word{P}{t},\word{Q}{t}\right)} = \vec{HS}\trans{\left(\word{P}{1},\word{Q}{1}\right)} \circ \left(\srprod{P}{Q}\right)^{t-1}$$
%Therefore, 
%\begin{equation}
%\label{eq:dist}
%1-\hellingersq{\word{P}{t}}{\word{Q}{t}} = \srprodt{\vect{p}}{\vect{q}}\circ \left(\srprod{P}{Q}\right)^{t-1} \circ \onev
%\end{equation}




%\subsection{Examples}
%
%
%\begin{example}
%\label{example:two_components}
%Two disjoint connected components: $\dist{M_1}{M_2}=1-\specr{\srprod{M_1}{M_2}}$ is not a metric $1-\specr{\srprod{M_1}{M_2}}=1-\specr{\srprod{M_2}{M_3}}=0,$ but 
%$1-\specr{\srprod{M_1}{M_3}}> 0$.
%\end{example}
%
%\begin{example}
%\label{example:starting_position}
%Non symmetric Markov chains: importance of starting position. 
%$P$ -- oriented ring, $Q$ -- the same ring but with a state that has heavy self-loop on one of the vertices. 
%Distance $\dist{P}{Q}=1-\specr{\srprod{P}{Q}}$ can be $0$. However, from some states it takes up to $\Omega(n)$ steps to reach 
%vertex with the loop to see the difference.
%
%\begin{figure}[H]
%	\centering
%		\includegraphics[scale=0.40]{diagrams/example2.jpg}
%	\caption{Example 2.}
%	\label{fig:example2}
%\end{figure}
%\end{example}
%
%\begin{example}
%\label{example:stationary_testing}
%Non symmetric MC: small difference between $P$ and $Q$ ($\dist{\word{P}{\ell}}{\word{Q}{\ell}}$ is small), but stationary distributions $\vect{p}_0$ and $\vect{q}_0$
%are vastly different. $Q$ - a ring with an edge $e=(v_1v_2)$ is removed, $v_1$ has a self-loop instead; $P$ - a similar ring with a loop, but $e$ is not removed but has a lighter
%weight of $\frac{1}{\sqrt{n}}$, while self-loop at $v_1$ has a weight of $1-\frac{1}{\sqrt{n}}$. Stationary distributions: $\vect{q}_0=\trans{(1,0,\cdots,0)}$ and
%$\vect{p}_0=\trans{(\frac{\sqrt{n}}{n+\sqrt{n}-1},\frac{1}{n+\sqrt{n}-1},\dots,\frac{1}{n+\sqrt{n}-1})}$, while $\specr{\srprod{P}{Q}}=\sqrt{1-\frac{1}{\sqrt{n}}}$.   
%%\begin{figure}[H]
%	%\centering
%		%\includegraphics[scale=0.40]{diagrams/example3.jpg}
%	%\caption{Example 3.}
%	%\label{fig:example3}
%%\end{figure}
%
%\end{example}
%\begin{example}
%\label{example:same_stationary_large_dist}
%Non symmetric MC: same stationary (Uniform) distribution, $\dist{P}{Q}=1$, at average it takes $\Omega(n)$ steps to distinguish whether $P=Q$, or not.
%Two oriented cycles, 
%\[
%P\eqdef s_1\to s_2\to\cdots\to s_n\to s_1\quad\quad Q\eqdef s_1\to s_3\to s_4\cdots\to s_n\to s_2\to s_1.
%\]
%%\begin{figure}[H]
%	%\centering
%		%\includegraphics[scale=0.40]{diagrams/example4.jpg}
%	%\caption{Example 4.}
%	%\label{fig:example4}
%%\end{figure}
%\end{example}
%\begin{example}
%\label{example:symmetric}
%Symmetric Markov chains: $Q$ -- complete graph $Q_{ij}=1/n$; $P$ -- clique and disjoint vertex $P_{ij}=\frac{1}{n-1}, i,j\in[n-1]$, $P_{in}=P_{ni}=0, i\in[n-1],$ and 
%$P_{nn}=1$. Then $\eigi[1]=\sqrt{\frac{n-1}{n}}= 1 - \frac{1}{2n}+O(n^2)$, $\eigi[2]=\sqrt{\frac{1}{n}}$, $\eigi[3]=\cdots=\eigi[n]=0$. If Markov Chain starts from state $1$, 
%after one transition we would know almost certainly whether $w\sim P$, or $w\sim Q$. On the other hand, if $w$ starts from any other state, then it would take us about $n$ 
%observations to tell whether $w\sim P$, or $w\sim Q$. If we start from a random state, again we would need about $n$ steps to distinguish $P$ vs. $Q$.
%%\begin{figure}[H]
%	%\centering
%		%\includegraphics[scale=0.40]{diagrams/example5.jpg}
%	%\caption{Example 5.}
%	%\label{fig:example5}
%%\end{figure}
%\end{example}






% Acknowledgments---Will not appear in anonymized version
\acks{This work was supported by grants from the MITEI-Shell program, Amazon Research Award and Simons Investigator Award.}


\bibliography{sublinear_space_nn}

\newpage
\appendix
\input{sketch_proofs}
\input{nnlower}
\newcommand{\Rlin}{R^{\mathrm{lin}}}
\section{Lower bound}
\label{s:lower}
In this section we derive a lower bound for bandits with composite anonymous feedback. We do that through a reduction from the setting of linear bandits (in the probability simplex) to our setting. This reduction allows us to upper bound the regret of a linear bandit algorithm in terms of (a suitably scaled version of) the regret of an algorithm in our setting. Since the reduction applies to any instance of a linear bandit problem, we can use a known lower bound for the linear bandit setting to derive a corresponding lower bound for our composite setting.

Let $\Delta_K$ be the probability simplex in $\R^K$. At each round $t$, an algorithm $A$ for linear bandit optimization chooses an action $\bp_t\in\Delta_K$ and suffers loss $\bloss_t^{\top}\bp_t$, where $\bloss_t \in [0,1]^K$ is some unknown loss vector. The feedback observed by the algorithm at the end of round $t$ is the scalar $\bloss_t^{\top}\bp_t$. The regret suffered by algorithm $A$ playing actions $\bp_1,\dots,\bp_T$ is
\begin{equation}
\label{eq:lin-regret}
	\Rlin_T = \sum_{t=1}^T \bloss_t^{\top}\bp_t - \min_{\bp\in\Delta_K} \sum_{t=1}^T \bloss_t^{\top}\bp = \sum_{t=1}^T \bloss_t^{\top}\bp_t - \min_{i=1,\dots,K} \sum_{t=1}^T \loss_t(i)
\end{equation}
where we used the fact that a linear function on the simplex is minimized at one of the corners.
Let $\Rlin_T(A,\Delta_K)$ denote the worst case regret (over the oblivious choice of $\bloss_1,\dots,\bloss_T$) of algorithm $A$. Similarly, let $R_T(A_d,K,d)$ be the worst case regret (over the oblivious choice of loss components $\loss_t^{(s)}(i)$ for all $t$, $s$, and $i$) of algorithm $A_d$ for nonstochastic $K$-armed bandits with $d$-delayed composite anonymous feedback. Our reduction shows the following.
%
\begin{lemma}\label{lem:lower}
For any algorithm $A_d$ for $K$-armed bandits with $d$-delayed composite anonymous feedback, there exists an algorithm $A$ for linear bandits in $\Delta_K$ such that
$
	R_T(A_d,K,d) \ge d\,\Rlin_{T/d}(A,\Delta_K)
$.
\end{lemma}
%
%
Our reduction, described in detail in the proof of the above lemma (see the appendix), essentially builds the probability vectors $\bp_t$ played by $A$ based on the empirical distribution of actions played by $A_d$ during blocks of size $d$. Now, an additional lemma is needed (whose proof is given in the appendix).
\begin{lemma}
\label{l:shamir}
The regret of any algorithm $A$ for linear bandits in the simplex satisfies $\Rlin_T(A,\Delta_K) = \widetilde{\Omega}\big(\sqrt{KT}\big)$.
\end{lemma}
%
Using the above two lemmas we can prove the following theorem.
\begin{theorem}
For any algorithm $A_d$ for $K$-armed bandits with $d$-delayed composite anonymous feedback,
$
R_T(A_d,K,d)=\widetilde{\Omega}\big(\sqrt{dKT}\big)
$.
\end{theorem}
%
\begin{proof}
Fix an algorithm $A_d$. Using the reduction of~Lemma~\ref{lem:lower} gives an algorithm $A$ such that
$
	R_T(A_d,K,d) \ge d\,\Rlin_{T/d}(A,\Delta_K) = \widetilde{\Omega}\big(\sqrt{dKT}\big)
$,
where we used Lemma~\ref{l:shamir} with horizon $T/d$ to prove the $\widetilde{\Omega}$-equality.
\end{proof}
%
Although the loss sequence used to prove the lower bound for linear bandits in the simplex is stochastic i.i.d., the loss sequence achieving the lower bound in our delayed setting is not independent due to the deterministic loss transformation in the proof of Lemma~\ref{lem:lower} (which is defined independent of the algorithm, thus preserving the oblivious nature of the adversary).


\input{middleout}

\end{document}
