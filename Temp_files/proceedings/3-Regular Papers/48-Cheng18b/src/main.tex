%\let\proof\relax
%\let\endproof\relax

\documentclass[final,12pt]{colt2018} % Anonymized submission
% \documentclass{colt2017} % Include author names

% The following packages will be automatically loaded:
% amsmath, amssymb, natbib, graphicx, url, algorithm2e

\title[Non-Convex Matrix Completion Against a Semi-Random Adversary]{Non-Convex Matrix Completion Against a Semi-Random Adversary}
\usepackage{times}
 % Use \Name{Author Name} to specify the name.
 % If the surname contains spaces, enclose the surname
 % in braces, e.g. \Name{John {Smith Jones}} similarly
 % if the name has a "von" part, e.g \Name{Jane {de Winter}}.
 % If the first letter in the forenames is a diacritic
 % enclose the diacritic in braces, e.g. \Name{{\'E}louise Smith}

 % Two authors with the same address
 \coltauthor{\Name{Yu Cheng} \Email{yucheng@cs.duke.edu}\AND
  \Name{Rong Ge} \Email{rongge@cs.duke.edu}\\
  \addr Duke University}

 % Three or more authors with the same address:
 % \coltauthor{\Name{Author Name1} \Email{an1@sample.com}\\
 %  \Name{Author Name2} \Email{an2@sample.com}\\
 %  \Name{Author Name3} \Email{an3@sample.com}\\
 %  \addr Address}
 
 % Authors with different addresses:
% \coltauthor{\Name{Author Name1} \Email{abc@sample.com}\\
% \addr Address 1
% \AND
% \Name{Author Name2} \Email{xyz@sample.com}\\
% \addr Address 2
% }


%\usepackage{amsmath,amsfonts,amsthm,amssymb,color}

%\usepackage{fullpage}

%\newtheorem{theorem}{Theorem}[section]
%\newtheorem{corollary}[theorem]{Corollary}
%\newtheorem{lemma}[theorem]{Lemma}
%\newtheorem{observation}[theorem]{Observation}
%\newtheorem{proposition}[theorem]{Proposition}
%\newtheorem{conjecture}[theorem]{Conjecture}
%\newtheorem{claim}[theorem]{Claim}
%\newtheorem{fact}[theorem]{Fact}
%\newtheorem{assumption}[theorem]{Assumption}
%\theoremstyle{definition}
%\newtheorem{definition}[theorem]{Definition}
\newtheorem{problem}{Problem}
%\newtheorem{remark}[theorem]{Remark}

% Color edits
%\usepackage{color-edits}
%\usepackage[suppress]{color-edits}  % to get rid of all color

%\addauthor[Rong]{rg}{blue}
%\addauthor[Yu]{yc}{red}

% We now have the macros:
% - \ycedit{text}: prints text in red
% - \yccomment{text}: prints [Yu: text] in red
% - \ycmargincomment{text}: prints [Yu: text] in red in the margin
% - \ycdelete{text}: marks in the margin in red that Yu deleted here

% Yu
%\usepackage{hyperref}
%\usepackage[ruled,vlined]{algorithm2e}
%\usepackage{graphicx}
\usepackage{xspace}
\usepackage{ifthen}
%\usepackage{color}
%\newcommand{\todo}[1]{{\color{red} TODO: #1}}

\newcommand{\eps}{\ensuremath{\epsilon}\xspace}
\renewcommand{\tilde}{\widetilde}
\renewcommand{\hat}{\widehat}
\renewcommand{\bar}{\overline}
\newcommand{\eqdef}{:=}
%\newcommand{\eqdef}{\ensuremath{\overset{\mathrm{def}}{=}}\xspace}
\newcommand{\R}{\mathbb{R}}
\newcommand{\C}{\mathbb{R}}
\newcommand{\OO}{\mathcal{O}}

\DeclareMathOperator{\tr}{tr}
\newcommand{\norm}[1]{\lVert#1{\rVert}}
\newcommand{\normone}[1]{{\norm{#1}}_1}
\newcommand{\normtwo}[1]{{\norm{#1}}_2}
\newcommand{\norminf}[1]{{\norm{#1}}_\infty}
\newcommand{\fnorm}[1]{{\norm{#1}}_F}
\newcommand{\maxnorm}[1]{{\norm{#1}}_{\max}}
\providecommand{\expect}[2]{\ensuremath{\ifthenelse{\equal{#1}{}}{\mathbb{E}}{\mathbb{E}_{#1}}\left[#2\right]}\xspace}

% LP environment
\newcommand{\mini}[1]{\mbox{minimize} & {#1} &\\}
\newcommand{\maxi}[1]{\mbox{maximize} & {#1 } & \\}
\newcommand{\st}{\mbox{subject to} }
\newcommand{\con}[1]{&#1 & \\}
\newcommand{\qcon}[2]{&#1, & \mbox{for } #2.  \\}
\newenvironment{lp}{\begin{equation}  \begin{array}{lll}}{\end{array}\end{equation}}
\newenvironment{lp*}{\begin{equation*}  \begin{array}{lll}}{\end{array}\end{equation*}}

% Rong
\newcommand{\nnz}{\text{nnz}\xspace}
\newcommand{\inner}[1]{\langle #1\rangle}
\newcommand{\Us}{U^\star}
\newcommand{\Vs}{V^\star}
\newcommand{\Ms}{M^\star}
\newcommand{\Sigs}{\Sigma^\star}
\newcommand{\sigs}{\sigma^\star}


\begin{document}

%\title{Non-Convex Matrix Completion Against a Semi-Random Adversary}
%\author{Yu Cheng \qquad Rong Ge \\ Duke University}
%\date{}

\maketitle

\begin{abstract}
%System identification is a fundamental problem in time-series analysis, control
%theory, and reinforcement learning.
%%
%Despite its importance, a sharp non-asymptotic analysis for the number of
%trajectories from an unknown dynamical system needed to identify its parameters
%remains an open question, even in the special case when the dynamics are governed by linear
%equations.
%%
%In this paper, we take an important step towards a non-asymptotic theory for system identification.
%We prove that the ordinary least-squares (OLS) estimator attains nearly minimax
%optimal performance for the identification of linear dynamical systems from
%a single observed trajectory.
%%
%Our analysis relies on a generalization of Mendelson's small-ball method to dependent data,
%eschewing the use of standard mixing-time arguments.
%%
%We capture the correct
%signal-to-noise behavior of the problem, showing that \emph{more unstable} linear
%systems are \emph{easier} to estimate.
%%
%This behavior is qualitatively different from arguments which rely on mixing-time
%calculations that suggest that unstable systems are more difficult to estimate.
%%
%Finally, our proof techniques generalize to a class of linear response
%time-series.


We prove that the ordinary least-squares (OLS) estimator attains nearly minimax
optimal performance for the identification of linear dynamical systems from
a single observed trajectory.
%
Our upper bound relies on a generalization of Mendelson's small-ball method to dependent data,
eschewing the use of standard mixing-time arguments.
%
Our lower bounds reveal that these upper bounds match up to logarithmic factors.
%
In particular, we capture the correct
signal-to-noise behavior of the problem, showing that \emph{more unstable} linear
systems are \emph{easier} to estimate.
%
This behavior is qualitatively different from arguments which rely on mixing-time
calculations that suggest that unstable systems are more difficult to estimate.
%
We generalize our technique to provide bounds for a more general class of linear response
time-series.



\end{abstract}


%\pagenumbering{gobble}
%\newpage
%\pagenumbering{arabic}

\begin{keywords}
Matrix Completion, Non-Convex Optimization, Semi-Random Model.
\end{keywords}



Online learning algorithms are a key tool in web search and content optimization, adaptively learning what users want to see. In a typical application, each time a user arrives, the algorithm chooses among various content presentation options (\eg news articles to display), the chosen content is presented to the user, and an outcome (\eg a click) is observed. Such algorithms must balance \emph{exploration} (making potentially suboptimal decisions now for the sake of acquiring information that will improve decisions in the future) and \emph{exploitation} (using information collected in the past to make better decisions now). Exploration could degrade the experience of a current user, but improves user experience in the long run. This exploration-exploitation tradeoff is commonly studied in the online learning framework of \emph{multi-armed bandits}~\citep{Bubeck-survey12}.

Concerns have been raised about whether exploration in such scenarios could be unfair, in the sense that some individuals or groups may experience too much of the downside of exploration without sufficient upside \citep{bird2016exploring}. We formally study these concerns in the \emph{linear contextual bandits} model~\citep{Langford-www10,chu2011contextual}, a standard variant of multi-armed bandits appropriate for content personalization scenarios.  We focus on \emph{externalities} arising due to exploration, that is, undesirable side effects that the presence of one party may impose on another.


We first examine the effects of exploration at a group level.  We introduce the notion of a \emph{group externality} in an online learning system, quantifying how much the presence of one population (which we dub the majority) impacts the rewards of another (the minority). We show that this impact can be negative, and that, in a particular precise sense, no algorithm can avoid it. This cannot be explained by the absence of suitably good policies since our adoption of the linear contextual bandits framework implies the existence of a feasible policy that is simultaneously optimal for everyone. Instead, the problem is inherent to the process of exploration. We come to a surprising conclusion that more data can sometimes lead to worse outcomes for the users of an explore-exploit-based system. \looseness=-1

We next turn to the effect of exploration at an individual level. We interpret exploration as a potential externality imposed on the current user by future users of the system. Indeed, it is only for the sake of the future users that the algorithm would forego the action that currently looks optimal. To avoid this externality, one may use the greedy algorithm that always chooses the action that appears optimal according to current estimates of the problem parameters. While this greedy algorithm performs poorly in the worst case,
it tends to work well in many applications and experiments.\footnote{Both positive and negative findings are folklore. One way to precisely state the negative result is that the greedy algorithm incurs constant per-round regret with constant probability; while results of this form have likely been known for decades,
\citet[Corollary A.2(b)]{competingBandits-itcs16}
proved this for a wide variety of scenarios. Very recently, the good empirical performance has been confirmed by state-of-art experiments in \citet{practicalCB-arxiv18}.}

In a new line of work, \citet{bastani2017exploiting} and \citet{kannan2018smoothed}
analyzed conditions under which inherent diversity in the data makes explicit exploration unnecessary.
\citet{kannan2018smoothed} proved that the greedy algorithm achieves a regret rate of
$\tilde{O}(\sqrt{T})$ in expectation over small perturbations of the context vectors (which ensure sufficient data diversity). This is the best rate that can be achieved in the worst case (\ie for all problem instances, without data diversity assumptions), but it leaves open the possibilities that (i) another algorithm may perform much better than the greedy algorithm on some problem instances, or (ii) the greedy algorithm may perform much better than worst case under the diversity conditions. We expand on this line of work. We prove that under the same diversity conditions, the greedy algorithm almost matches the best possible Bayesian regret rate of \emph{any} algorithm \emph{on the same problem instance}. This could be as low as $\polylog(T)$ for some instances, and, as we prove, at most $\tilde{O}(T^{1/3})$ whenever the diversity conditions hold.


Returning to group-level effects, we show that under the same diversity conditions, the negative group externalities imposed by the majority essentially vanish if one runs the greedy algorithm. Together, our results illustrate a sharp contrast between the high individual and group externalities that exist in the worst case, and the ability to remove all externalities if the data is sufficiently diverse.   \looseness=-1

\xhdr{Additional motivation.} Whether and when explicit exploration is necessary is an important concern in the study of the exploration-exploitation tradeoff. Fairness considerations aside, explicit exploration is expensive. It is wasteful and risky in the short term, it adds a layer of complexity to algorithm design \citep{Langford-nips07,monster-icml14}, and its adoption at scale tends to require substantial systems support and buy-in from management \citep{MWT-WhitePaper-2016,DS-arxiv}. A system based on the greedy algorithm would typically be cheaper to design and deploy.

Further, explicit exploration can run into incentive issues in applications such as recommender systems. Essentially, when it is up to the users which products or experiences to choose and the algorithm can only issue recommendations and ratings, an explore-exploit algorithm needs to provide incentives to explore for the sake of the future users \citep{Kremer-JPE14,Frazier-ec14,Che-13,ICexploration-ec15,Bimpikis-exploration-ms17}. Such incentive guarantees tend to come at the cost of decreased performance, and rely on assumptions about human behavior. The greedy algorithm avoids this problem as it is inherently consistent with the users' incentives.



\xhdr{Additional related work.}
Our research draws inspiration from the growing body of work on fairness in machine learning~\cite[e.g.,][]{dwork2012fairness,hardt2016equality,kleinberg2017inherent,chouldechova2017fair}.  Several other authors have studied fairness in the context of the contextual bandits framework.  Our work differs from the line of research on meritocratic fairness in online learning \citep{kearns2017meritocratic,liu2017calibrated,joseph2016fairness}, which considers the allocation of limited resources such as bank loans and requires that nobody should be passed over in favor of a less qualified applicant. We study a fundamentally different scenario in which there are no allocation constraints and we would like to serve each user the best content possible.  Our work also differs from that of \citet{celis2017fair}, who studied an alternative notion of fairness in the context of news recommendations. According to this notion, all users should have approximately the same probability of seeing a particular type of content (e.g., Republican-leaning articles), regardless of their individual preferences, in order to mitigate the possibility of discriminatory personalization.

The data diversity conditions in \citet{kannan2018smoothed} and this paper are inspired by the smoothed analysis framework of \citet{SmoothedAnalysis-jacm04}, who proved that the expected running time of the simplex algorithm is polynomial for perturbations of any initial problem instance (whereas the worst-case running time has long been known to be exponential). Such disparity implies that very bad problem instances are brittle. 
We find a similar disparity for the greedy algorithm in our setting.



\xhdr{Our results on group externalities.}  A typical goal in online learning is to minimize \emph{regret}, the (expected) difference between the cumulative reward that would have been obtained had the optimal policy been followed at every round and the cumulative reward obtained by the algorithm.  We define a corresponding notion of \emph{minority regret}, the portion of the regret experienced by the minority.  Since online learning algorithms update their behavior based on the history of their observations, minority regret is influenced by the entire population on which an algorithm is run.  If the minority regret is much higher when a particular algorithm is run on the full population than it is when the same algorithm is run on the minority alone, we can view the majority as imposing a negative externality on the minority; the minority population would achieve a higher cumulative reward if the majority were not present. Asking whether this can ever happen
amounts to asking whether access to more data points can ever lead an explore-exploit algorithm to make inferior decisions. One might think that more data should always lead to better decisions and therefore better outcomes for the users.
Surprisingly, we show that this is not the case, even with a standard algorithm.

Consider LinUCB~\citep{Langford-www10,chu2011contextual,abbasi2011improved}, a standard algorithm for linear contextual bandits that is based on the principle of ``optimism under uncertainty.''  We provide a specific problem instance on which, after observing $T$ users, LinUCB would have a minority regret of $\Omega(\sqrt T)$ if run on the full population, but only constant minority regret if run on the minority alone. While stylized, this example is motivated by the problem of providing driving directions to different populations of users, about which fairness concerns have been raised~\citep{bird2016exploring}. Further, the situation is reversed on a slight variation of this example: LinUCB obtains constant minority regret when run on the full population and $\Omega(\sqrt T)$ on the minority alone.  That is, group externalities can be large and positive in some cases, and large and negative in others.

Although these regret rates are specific to LinUCB, we show that this phenomenon is, in some sense, unavoidable. Consider the minority regret of LinUCB when run on the full population and the minority regret that LinUCB would incur if run on the minority alone. We know that one may be much smaller or larger than the other. We ask whether any algorithm could  achieve the minimum of the two on every problem instance. Using a variation of the same problem instance, we prove that this is impossible; in fact, no algorithm could simultaneously approximate both up to any $o(\sqrt{T})$ factor. In other words, an externality-free algorithm would sometimes ``leave money on the table."


In terms of techniques, we rely on the special structure of our example, which can be viewed as an instance of the sleeping bandits problem~\citep{SleepingBandits-ml10}. This simplifies the behavior and analysis of LinUCB, allowing us to obtain the $O(1)$ upper bounds.  The lower bounds are obtained using KL-divergence techniques to show that the two variants of our example are essentially indistinguishable, and an algorithm that performs well on one must obtain $\Omega(\sqrt{T})$ regret on the other. \looseness=-1


\xhdr{Our results on the greedy algorithm.} We consider a version of linear contextual bandits in which the latent weight vector $\theta$ is drawn from a known prior. In each round, an algorithm is presented several actions to choose from, each represented by a \emph{context vector}. The expected reward of an action is a linear product of $\theta$ and the corresponding context vector. The tuple of context vectors is drawn independently from a fixed distribution. In the spirit of smoothed analysis, we assume that this distribution has a small amount of jitter. Formally, the tuple of context vectors is drawn from some fixed distribution, and then a small \emph{perturbation}---small-variance Gaussian noise---is added independently to each coordinate of each context vector. This ensures arriving contexts are diverse. We are interested in Bayesian regret, i.e., regret in expectation over the Bayesian prior. Following the literature, we are primarily interested in the dependence on the time horizon $T$. \looseness=-1

We focus on a batched version of the greedy algorithm, in which new data arrives to the algorithm's optimization routine in small batches, rather than every round. This is well-motivated from a practical perspective---in high-volume applications data usually arrives to the ``learner" only after a substantial delay \citep{MWT-WhitePaper-2016,DS-arxiv}---and is essential for our analysis.

Our main result is that the greedy algorithm matches the Bayesian regret of any algorithm up to polylogarithmic factors, for each problem instance, fixing the Bayesian prior and the context distribution. We also prove that LinUCB achieves regret $\tilde{O}(T^{1/3})$ for each realization of $\theta$. This implies a worst-case Bayesian regret of $\tilde{O}(T^{1/3})$ for the greedy algorithm under the perturbation assumption. \looseness=-1

Our results hold for both natural versions of the batched greedy algorithm, Bayesian and frequentist, henceforth called \BayesGreedy and \FreqGreedy. In \BayesGreedy, the chosen action maximizes expected reward according to the Bayesian posterior. \FreqGreedy estimates $\theta$ using ordinary least squares regression and chooses the best action according to this estimate. The results for \FreqGreedy come with additive polylogarithmic factors, but are stronger in that the algorithm does not need to know the prior. Further, the $\tilde{O}(T^{1/3})$ regret bound for \FreqGreedy is approximately prior-independent, in the sense that it applies even to very concentrated priors such as independent Gaussians with standard deviation on the order of $T^{-2/3}$.

The key insight in our analysis of \BayesGreedy is that any (perturbed) data can be used to simulate any other data, with some discount factor. The analysis of \FreqGreedy requires an additional layer of complexity. We consider a hypothetical algorithm that receives the same data as \FreqGreedy, but chooses actions based on the Bayesian-greedy selection rule. We analyze this hypothetical algorithm using the same technique as \BayesGreedy, and then upper bound the difference in Bayesian regret between the hypothetical algorithm and \FreqGreedy.

Our analyses extend to group externalities and (Bayesian) minority regret. In particular, we circumvent the impossibility result mentioned above. We prove that both \BayesGreedy and \FreqGreedy match the Bayesian minority regret of any algorithm run on either the full population or the minority alone, up to polylogarithmic factors

\xhdr{Detailed comparison with prior work.} We substantially improve over the $\tilde{O}(\sqrt{T})$ worst-case regret bound from \citet{kannan2018smoothed}, at the cost of some additional assumptions. First, we consider Bayesian regret, whereas their regret bound is for each realization of $\theta$.%
\footnote{Equivalently, they allow point priors, whereas our priors must have variance $T^{-O(1)}$.} Second, they allow the context vectors to be chosen by an adversary before the perturbation is applied. Third, they extend their analysis to a somewhat more general model, in which there is a separate latent weight vector for every action (which amounts to a more restrictive model of perturbations). However, this extension relies on the greedy algorithm being initialized with a substantial amount of data. The results of \citet{kannan2018smoothed} do not appear to have implications on group externalities.

\citet{bastani2017exploiting} show that the greedy algorithm achieves logarithmic regret in an alternative linear contextual bandits setting that is incomparable to ours in several important ways.
They consider two-action instances where the actions share a common context vector in each round, but are parameterized by different latent vectors. They ensure data diversity via a strong assumption on the context distribution. This assumption does not follow from our perturbation conditions; among other things, it implies that each action is the best action in a constant fraction of rounds. Further, they assume a version of Tsybakov's \emph{margin condition}, which is known to substantially reduce regret rates in bandit problems \citep[\eg see][]{Zeevi-colt10}.



%!TEX root = main.tex

\section{Preliminaries}
In this section, we will review some well-known results on~\gd~and~\nag~in the strongly convex setting, 
and existing results on convergence of~\gd~to second-order stationary points. 
% The pseudocode for these algorithms is given in Algorithms~\ref{algo:gd} and~\ref{algo:AGD} respectively.

% \cnote{Show gradient descent in equation}

\subsection{Notation}
Bold upper-case letters ($\A, \B$) denote matrices and bold lower-case letters ($\x, \y$) denote vectors. 
For vectors $\norm{\cdot}$ denotes the $\ell_2$-norm. For matrices, $\norm{\cdot}$ denotes the spectral norm and $\lambda_{\min}(\cdot)$ denotes the minimum eigenvalue.
For $f: \R^d \rightarrow \R$, $\grad f(\cdot)$ and  $\hess f(\cdot)$ denote its gradient and Hessian respectively, and $f^\star$ denotes its global minimum.
% Other than Section \ref{sec:related}, 
We use $O(\cdot), \Theta(\cdot), \Omega(\cdot)$ to hide absolute constants, and $\tilde{O}(\cdot), \tilde{\Theta}(\cdot), \tilde{\Omega}(\cdot)$ to hide absolute constants and polylog factors for all problem parameters. 
% \praneeth{I think it will be cleaner to make the dependence on smoothness parameters in Table~\ref{tab:main} and edit this statement} \jccomment{Then I also need to add function value dependence, maybe too complicated to compare}.\praneeth{The issue with this is that $O()$ is not just hiding constants but also problem dependent parameters. May be mention this explicitly in the caption to the table.} 
% We let $\ball^{(d)}_\x(r)$ denote the d-dimensional ball centered at $\x$ with radius $r$; when it is clear from context, we simply denote it as $\ball_\x(r)$. We use $\proj_{\mathcal{X}}(\cdot)$ to denote projection onto the set $\mathcal{X}$. Distance and projection are always defined in a Euclidean sense.


% \pn{Talk about ignoring $\log d$ factors in notation.}

\subsection{Convex Setting}\label{sec:prelim_convex}
% \begin{figure}[t]
% \begin{minipage}{0.5\textwidth}
% 	\begin{algorithm}[H]
% 	\caption{\gd($\x_0, \eta$)}\label{algo:gd}
% 	\begin{algorithmic}[1]
% 		\For{$t = 0, 1, \ldots, T $}
% 		\State $\x_{t+1} \leftarrow \x_t - \eta \grad f (\x_t)$
% 		\EndFor
% 		\State \textbf{return} $\x_T$
% 	\end{algorithmic}
% 	\end{algorithm}
% 	\vspace{0.5cm}
% \end{minipage}
% \begin{minipage}{.5\textwidth}

% \end{minipage}
% \end{figure}
To minimize a function $f(\cdot)$,~\gd ~performs the following sequence of steps:
\begin{equation*}
\x_{t+1} = \x_{t}- \eta \grad f(\x_t).
\end{equation*}
The suboptimality of~\gd~and the improvement achieved by~\nag~can be clearly illustrated for the case of smooth and strongly convex functions. %The definitions of smoothness and strong convexity are as follows.
\begin{definition}\label{def:smooth}
A differentiable function $f(\cdot)$ is \textbf{$\ell$-smooth (or $\ell$-gradient Lipschitz)} if:
\begin{equation*}
\norm{\grad f(\x_1) - \grad f(\x_2)} \le \ell \norm{\x_1 - \x_2} \quad \forall \; \x_1, \x_2.
\end{equation*}
\end{definition}
\noindent
The gradient Lipschitz property asserts that the gradient can not change too rapidly in a small local region.
\begin{definition}\label{def:convex}
A twice-differentiable function $f(\cdot)$ is \textbf{$\alpha$-strongly convex} if
$\lambda_{\min}(\hess f(\x)) \ge \alpha, \;  \forall \; \x$.
% $f(\x_2) \ge f(\x_1) + \la \grad f(\x_1), \x_2 - \x_1 \ra + \frac{\alpha}{2}\norm{\x_2 - \x_1}^2, \quad \forall \; \x_1, \x_2.$
\end{definition}
Let $\fstar \defeq \min_{\y}f(\y)$. A point $\x$ is said to be \textbf{$\epsilon$-suboptimal} if $f(\x)  \le  \fstar + \epsilon$. The following theorem gives the convergence rate of GD and AGD for smooth and strongly convex functions.
\begin{theorem}[\cite{nesterov2004introductory}]\label{thm:gd_convex}
Assume that the function $f(\cdot)$ is $\ell$-smooth and $\alpha$-strongly convex. Then, for any $\epsilon>0$,
the iteration complexities to find an $\epsilon$-suboptimal point are as follows:
\begin{itemize}
\item GD with $\eta  = 1/\ell$: \quad $O((\ell/\alpha) \cdot \log ((f(\x_0) - \fstar)/\epsilon))$
\item AGD (Algorithm~\ref{algo:AGD}) with $\eta = 1/\ell$ and $\theta = \sqrt{\alpha/\ell}$:
\quad$O(\sqrt{\ell/\alpha} \cdot \log ((f(\x_0) - \fstar)/\epsilon))$.
\end{itemize}
% ~\gd~with $\eta = \frac{1}{\ell}$ will output an \ESP ~in iterations:
% \begin{equation*}
% O\left(\frac{\ell}{\alpha}\log \frac{f(\x_0) - \fstar}{\epsilon}\right).
% \end{equation*}
\end{theorem}

The number of iterations of GD depends linearly on the ratio $\ell/\alpha$, which is called the condition number of $f(\cdot)$ since $\alpha \I \preceq\hess f(\x) \preceq \ell \I $. Clearly $\ell \geq \alpha$ and hence condition number is always at least one. Denoting the condition number by ${\cn}$, we highlight two important aspects of~\nag: (1) the momentum parameter satisfies $\theta = 1/\sqrt{\cn}$ and (2) \nag~improves upon GD by a factor of $\sqrt{\cn}$. 
% The following theorem gives the convergence rate of~\nag~for these problems.
% \begin{theorem}[\cite{nesterov2004introductory}]\label{thm:agd_convex}
% Assume that the function $f(\cdot)$ is $\ell$-smooth and convex. Then, for any $\epsilon>0$,~\nag~with $\eta = \frac{1}{\ell}$ and $\theta = \Theta(\sqrt{\frac{\alpha}{\ell}}) $ will output an~\ESP~in iterations:
% \begin{equation*}
% O\left(\sqrt{\frac{\ell}{\alpha}}\log \frac{f(\x_0)-\fstar }{\epsilon}\right).
% \end{equation*}
% \end{theorem}
% \noindent
% Note that the rate here improves upon that of~\gd~by a factor of $\sqrt{\frac{\ell}{\alpha}}$ i.e., squareroot of the condition number.
%say something about condition number.

\subsection{Nonconvex Setting}
For nonconvex functions finding global minima is NP-hard in the worst case. The best one can hope for in this setting is convergence to stationary points. There are various levels of stationarity.
\begin{definition}
$\x$ is an \textbf{\EFSP} of function $f(\cdot)$ if $\norm{\grad f(\x)} \le \epsilon$.
\end{definition}
\noindent
As mentioned in Section~\ref{sec:intro}, for most nonconvex problems encountered in practice, a majority of first-order stationary points turn out to be saddle points. Second-order stationary points require not only zero gradient, but also positive semidefinite Hessian, ruling out most saddle points.
%Therefore, this paper focus on finding second-order stationary point.
%In order to discuss Hessian-related properties meaningfully, we first need to assert Hessian smoothness condition.
Second-order stationary points are meaningful, however, only when the Hessian is continuous.
% second order stationary points which means that in addition to being first order stationary points, the Hessian at these points is almost positive semidefinite. This is meaningful only if the Hessian does not change arbitrarily (and perhaps have large negative eigenvalues) in a small neighborhood around this point. In other words, finding second order stationary points is meaningful only if the Hessian is continuous.
%\cnote{Should we talk the case where gradient is Lipschitz but Hessian is not?}
% \begin{theorem}[\citep{nesterov1998introductory}]\label{thm:grad_smooth}
% Assume that the function $f(\cdot)$ is $\ell$-smooth. Then, for any $\epsilon>0$, gradient descent will output an \EFSP ~in iterations:
% \begin{equation*}
% \frac{\ell(f(\x_0) - f^\star)}{\epsilon^2}.
% \end{equation*}
% \end{theorem}
\begin{definition}\label{def:HessianLip}
A twice-differentiable function $f(\cdot)$ is \textbf{$\rho$-Hessian Lipschitz} if:
\begin{equation*}
\norm{\hess f(\x_1) - \hess f(\x_2)} \le \rho \norm{\x_1 - \x_2} \quad \forall \; \x_1, \x_2.
\end{equation*}
\end{definition}
%\noindent
% For Hessian Lipschitz functions, we recall the definition of second order stationary points from~\cite{nesterov2006cubic}.
\begin{definition}[\cite{nesterov2006cubic}]\label{def:SOSP}
For a $\rho$-Hessian Lipschitz function $f(\cdot)$, $\x$ is an \textbf{\ESSP} if:
% $\norm{\grad f(\x)} \le \epsilon$ and $\lambda_{\min}(\hess f(\x)) \ge - \sqrt{\rho \epsilon}$.
\begin{equation*}
\norm{\grad f(\x)} \le \epsilon \quad\text{and}\quad \lambda_{\min}(\hess f(\x)) \ge - \sqrt{\rho \epsilon}.
\end{equation*}
\end{definition}
\noindent
The following theorem gives convergence rate of perturbed~\gd~to second-order stationary points.
%See~\cite{jin2017escape} for a detailed description of the algorithm.
\begin{theorem}[\citep{jin2017escape}]\label{thm:perturbed_GD}
Assume that the function $f(\cdot)$ is $\ell$-smooth and $\rho$-Hessian Lipschitz. Then, for any $\epsilon>0$, perturbed GD outputs an \ESSP ~w.h.p in iterations:
\begin{equation*}
\otilde{\frac{\ell(f(\x_0) - \fstar)}{\epsilon^2}}.
\end{equation*}
\end{theorem}
\noindent
Note that this rate is essentially the same as that of~\gd~for convergence to first-order stationary points. In particular, it only has polylogarithmic dependence on the dimension.


% !TEX root = main.tex

\section{Pre-Processing: Reweighting the Entries}
\label{sec:bss}

In this section, we present a nearly-linear time algorithm for Problem~\ref{prob:main}.
As we discussed in Section~\ref{sec:laplacian}, Problem~\ref{prob:main} is equivalent to the problem of approximating the identity matrix (Problem~\ref{prob:identity}). We prove the following theorem for Problem~\ref{prob:identity}:

\begin{theorem}[Our Preprocessing Algorithm]
\label{thm:bss}
Fix $0 \le \eps, \beta \le 1/10$, and a graph Laplacian $L \in \R^{n \times n}$.
Given a set of $m$ vectors $\{v_i\}_{i=1}^m$, where each $v_i = L^{-1/2} b_i$ for some $b_i$ representing an edge (with only two non-zero entries, one $+1$ and one $-1$).
Assume there exist weights $w_i \ge 0$ such that~\footnote{We assume $0<\beta\le\frac{1}{10}$ is given, because we can do a binary search by running our algorithm and see if it succeeds.
It is worth mentioning that we never explicitly compute any $v_i = L^{-1/2} b_i$. %We will maintain a set of weights $\{\tilde w_i\}$ which will eventually satisfy the claim.
See Appendix~\ref{app:bss} for more details.}
\[
\textstyle (1-\beta) I \preceq \sum_{i=1}^m w_i v_i v_i^\top \preceq (1+\beta) I.
\]
We can find a set of weights $\tilde w_i \ge 0$ in $\tilde O(m / \eps^{O(1)})$ time, such that with high probability,
\[
\textstyle (1-O(\beta)-\eps)I \preceq \sum_{i=1}^m \tilde w_i v_i v_i^\top \preceq I.
\]
\end{theorem}

%In other words, we can solve the following class of SDP approximately in nearly linear time.
%\begin{lp}
%\mini{\normtwo{I - \sum_{i=1}^m w_i v_i v_i^\top}}
%\st \con{w_i \ge 0}
%\end{lp}
We adapt techniques from recent developments on linear-sized graph sparsification~\citep{BatsonSS12, AllenLO15, LeeS17}.
The main difference between our problem and the graph sparsification problem is the following: instead of assuming $\sum_i v_i v_i^\top = I$, we only know the \emph{existence} of an unknown set $S$ such that \mbox{$\sum_{i\in S} v_i v_i^\top = I$}.
%In other words, the graph that we are trying to approximate (e.g., the complete bipartite graph) has spectral properties different from the input graph.
This prevents us from using some of the well-known techniques in graph sparsification (e.g., sampling by effective resistance \cite{SpielmanS11, LeeS15}).
%In other words, our goal is to find a set of weights $\{w_i\}_i$ so that $\sum_i w_i v_i v_i^\top$ is as close to $I$ as possible.
For the same reason, any simple reweighting algorithms that are oblivious to whether a good set $S$ exists will not work.

One of our main contributions is to identify that the framework of \cite{BatsonSS12} is not only limited to graph sparsification.
The fact that the algorithm picks edges \emph{deterministically} makes it much more powerful,
and the analysis only requires the \emph{existence} of a ``good'' edge to add in each iteration.
%Roughly speaking, in \citep{BatsonSS12} a good edge exists by an averaging argument over the entire set, while in our setting we average over the unknown subset $S$.
%Our approach in this section is most directly inspired by the recent work of~\cite{LeeS17}.
On the technical level, our work departs from previous works in two important ways: (1) our algorithm works even when the hidden set $S$ has sum only \emph{approximately} equal to $I$; and (2) our analysis is considerably simpler, partly because we do not require the output weights to be sparse. % (i.e., only a small number of the output $\tilde w_i$'s are non-zero).

%We follow the same algorithmic framework used by all previous algorithms.
We first give an overview of the framework of~\cite{BatsonSS12}.
We will maintain two barrier values $\ell < u$, and a weighted sum of the rank-one matrices $A = \sum_{i=1}^m w_i v_i v_i^\top$ such that $\ell I \prec A \prec u I$.
The plan is to start with some constants $\ell < 0 < u$, $A = 0$, and gradually increase the weights $\{w_i\}_i$, $u$ and $\ell$, while making sure that $A$ stays between the two barriers $u I$ and $\ell I$.
If we can increase $u$ and $\ell$ at roughly the same rate, the condition number of $A$ will become smaller.

Our approach in this section is most directly inspired by the recent work of~\cite{LeeS17}.
We use the following potential function proposed in \citep{LeeS17} to measure how far $A$ is from the barriers (both $uI$ and $\ell I$):
%\begin{align*}
%\Phi_{u,\ell}(A) & = \Phi_u(A) + \Phi_\ell(A), \text{ where} \\
%\Phi_u(A) & = \tr \exp \left((u I - A)^{-1}\right), \\
%\Phi_\ell(A) & = \tr \exp \left((A - \ell I)^{-1}\right).
%\end{align*}
\begin{align*}
\Phi_{u,\ell}(A) & = \Phi_u(A) + \Phi_\ell(A), \\
\text{ where } \Phi_u(A) & = \tr \exp \left((u I - A)^{-1}\right), \text{ and } \Phi_\ell(A) = \tr \exp \left((A - \ell I)^{-1}\right).
\end{align*}
If $A$ is far from $u I$ and $\ell I$, then all eigenvalues of $uI - A$ and $A - \ell I$ are large and $\Phi_{u,\ell}(A)$ is small.
The potential function is going to guide us on how to increase the weights $w_i$ so that $A$ stays away from the barriers.
% Intuitively, we want to expand $A$ in the direction that $(A - \ell I)$ is small, while at the same time avoid directions that $(u I - A)$ is small.
The derivatives of the potential functions with respect to $A$ are
%\begin{align*}
%\nabla \Phi_u (A) & = \exp \left((u I - A)^{-1}\right) (u I - A)^{-2}, \\ %\text{ and} \\
%\nabla \Phi_\ell (A) & = - \exp \left((A - \ell I)^{-1}\right) (A - \ell I)^{-2}.
%\end{align*}
\begin{align*}
\nabla \Phi_u (A) & = \exp \left((u I - A)^{-1}\right) (u I - A)^{-2}, \text{ and } 
\nabla \Phi_\ell (A) = - \exp \left((A - \ell I)^{-1}\right) (A - \ell I)^{-2}.
\end{align*}

For notational convenience, we define $C_{-} = \nabla \Phi_u (A)$, $C_{+} = -\nabla \Phi_\ell (A)$, and $C = C_+ - C_-$.
Note that when $\ell I \prec A \prec u I$, both $C_+$ and $C_-$ are PSD matrices.
The first order approximation of the potential function is
$\Phi_{u,\ell}(A+\Delta)
 \approx \Phi_{u,\ell}(A) + \nabla \Phi_{u}(A) \bullet \Delta + \nabla \Phi_{\ell}(A) \bullet \Delta 
% = \Phi_{u,\ell}(A) + C_- \bullet \Delta - C_+ \bullet \Delta
 = \Phi_{u,\ell}(A) - C \bullet \Delta$.
%\begin{align*}
%\Phi_{u,\ell}(A+\Delta)
% & \approx \Phi_{u,\ell}(A) + \nabla \Phi_{u}(A) \bullet \Delta + \nabla \Phi_{\ell}(A) \bullet \Delta \\
% & = \Phi_{u,\ell}(A) + C_- \bullet \Delta - C_+ \bullet \Delta
% = \Phi_{u,\ell}(A) - C \bullet \Delta.
%\end{align*}

We want $\Phi_{u,\ell}(A+\Delta)$ to be small, which guarantees that $A+\Delta$ is far away from $\ell I$ and $u I$.
Therefore, in each iteration, we seek a matrix $\Delta$ such that
\begin{enumerate}
\item[(1)] $\Delta$ is small enough for the first-order approximation of $\Phi_{u,\ell}(A+\Delta)$ to be accurate; and
\item[(2)] $\Delta$ maximizes $C \bullet \Delta$, the reduction to (first-order approximation of) the potential function.
\end{enumerate}

Formally, let $\rho = (\lambda_{\min}\{u I - A, A - \ell I\})^2$.
%As shown in~\citep{LeeS17}, 
When $0 \preceq \Delta \preceq \eps \rho I$, the first-order approximation of $\Phi_{u,\ell}(A+\Delta)$  is accurate (see Lemma~\ref{lem:potential-FO} in Appendix~\ref{app:bss}).
We are interested in the following SDP:
\begin{lp}
\label{eqn:sdp-oracle}
\maxi {C \bullet X}
\st \con{X \preceq \eps \rho I, \quad X = \sum_{i=1}^m x_i v_i v_i^\top \text{ (which implies $0 \preceq X$)},}
\end{lp}

%Ideally, we would like to have $X = \eps \rho I$ and $\delta_{u}=\delta_{\ell}=\eps\rho$, so that $A$ grows equally in each dimension.
%The upper and lower barriers would increase at the same rate, and the potential function would remain unchanged: $\Phi_{u+\eps\rho,\ell+\eps\rho}(A+\eps\rho I) = \Phi_{u,\ell}(A)$.
%When $X = \delta \rho I$, the objective value of the SDP is $C \bullet X = \delta \rho \tr(C)$.
%While this is too good to be true, we will show that we can find an $X$ that is almost as good. % with $C \bullet X \approx C \bullet \eps\rho I = \eps\rho\tr(C)$.

We give a full description of our algorithm in Algorithm~\ref{alg:ls17}.

\begin{algorithm2e}
%\begin{algorithm}
  \caption{Find $A = \sum_i w_i v_i v_i^\top \approx I$.}
  \label{alg:ls17}
  \SetAlgoVlined
  \SetKwInOut{Input}{Input}
%  \SetKwInOut{Output}{Output}
  \Input{$\{v_i\}_{i=1}^m$, $\eps \le 1/10$.}
%  \Output{}
  $j = 0$, $A_0 = 0$, $\ell_0 = -\frac{1}{4}$, $u_0 = \frac{1}{4}$\; 
  \While{$u_j - \ell_j \le 1$}{
   Let $\rho \in [1-\eps, 1] \cdot (\lambda_{\min}\{u_j I - A_j, A_j - \ell I_j\})^2$\;
   Let $\Delta_j$ be an approximate solution to the SDP \eqref{eqn:sdp-oracle} with $C = -\left(\nabla \Phi_{u_j}(A_j)+\nabla \Phi_{\ell_j}(A_j)\right)$\;
   $\delta_{u,j}=\frac{\eps\rho}{2} \cdot \frac{(1+\beta+5\eps)}{1-2\eps}$, $\delta_{\ell,j}=\frac{\eps\rho}{2}\cdot\frac{(1-\beta-5\eps)}{1+2\eps}$\;
   $A_{j+1} = A_j + \Delta_j$, $u_{j+1} = u_j + \delta_{u,j}$, $\ell_{j+1} = \ell_j + \delta_{\ell,j}$; \, $j = j + 1$\;
  }
  \Return{$A_j / u_j$}\;
%\end{algorithm}
\end{algorithm2e}

We will use the following lemmas (Lemmas~\ref{lem:sdp-sol}~and~\ref{lem:phi-no-increase}) to analyze Algorithm~\ref{alg:ls17} and prove Theorem~\ref{thm:bss}.
Lemma~\ref{lem:sdp-sol} shows that the SDP in \eqref{eqn:sdp-oracle} admits a good solution, and we can solve it approximately in nearly-linear time.
Lemma~\ref{lem:phi-no-increase} says that the potential function $\Phi_{u_j,\ell_j}(A_j)$ never increases, which guarantees that $A_j$ is far away from both $u_j I$ and $\ell_j I$ for all $j$.

%It is worth noting that, since we only know the existence of a set of weights whose sum is \emph{approximately} $I$, we have to change the rate that we increase the lower and upper barriers to cope with this error (at rate $1 \pm O(\beta + \eps)$ rather than $1 \pm O(\eps)$).

\begin{lemma}
\label{lem:sdp-sol}
Fix $0 < \beta,\eps \le 1/10$.
%Let $\{v_i\}_{i=1}^m$ be the input vectors in Theorem~\ref{thm:main}.
%, and a graph Laplacian $L \in \R^{n \times n}$.
%Given a set of $m$ vectors $\{v_i = L^{-1/2} b_i\}_{i=1}^m$, assume there exist weights $w_i \ge 0$ such that
%\[
%(1-\beta) I \preceq \sum_{i=1}^m w_i v_i v_i^\top \preceq (1+\beta) I.
%\]
In any iteration $j$ of Algorithm~\ref{alg:ls17}, given $A_j = \sum_{i=1}^m w_i v_i v_i^\top$ (implicitly by the weights $\{w_i \ge 0\}_{i=1}^m$) and corresponding barrier values $\ell = \ell_j$ and $u = u_j$,
\begin{enumerate}
\item[(1)] We can compute $\rho \in [1-\eps, 1] \cdot (\lambda_{\min}\{u I - A, A - \ell I\})^2$ w.h.p. in time $\tilde O(m / \eps^{O(1)})$.
\item[(2)] Let $C, C_{+}, C_{-}$ be defined as above. %$C_{-} = \nabla \Phi_u (A)$, $C_{+} = -\nabla \Phi_\ell (A)$, and $C = C_+ - C_-$.
%The SDP in \eqref{eqn:sdp-oracle} admits a solution $X^*$ with
%$
%C \bullet X^* \ge \eps \rho \left(\frac{1-\beta}{1+\beta}\tr(C_+) - \tr(C_-)\right).
%$
%Moreover, we can compute a solution $X = \sum_{i=1}^m \tilde w_i v_i v_i^\top$ (represented by the weights $\tilde w$) in time $\tilde O(m / \eps^{O(1)})$ such that, with high probability,
%Moreover, 
We can compute a set of weights $\{x_i \ge 0\}_{i=1}^m$ in time $\tilde O(m / \eps^{O(1)})$, such that w.h.p. for $X = \sum_{i=1}^m x_i v_i v_i^\top$,
\[
C \bullet X \ge \frac{\eps \rho}{2} \left((1-\beta-\eps)\tr(C_+) - (1+\beta+\eps)\tr(C_-)\right).
\]
\end{enumerate} 

\end{lemma}

\begin{lemma}
\label{lem:phi-no-increase}
Fix $0 < \beta, \eps \le 1/10$.
Let $A_{j+1} = A_j + \Delta_j$ denote the matrix in the $j$-th iteration of Algorithm~\ref{alg:ls17}.
If $\Delta_j$ is an approximate solution to the SDP that satisfies Lemma~\ref{lem:sdp-sol}, then we have $\Phi_{u_{j+1},\ell_{j+1}}(A_{j+1}) \le \Phi_{u_j,\ell_j}(A_{j})$.
\end{lemma}

We defer the proofs of these two lemmas to Appendix~\ref{app:bss}. Now we are ready to prove Theorem~\ref{thm:bss}.

\begin{proof}[Proof of Theorem~\ref{thm:bss}]
First, we show that $(1-O(\beta+\eps)) I \preceq A_j / u_j \preceq I$.
The condition number of $A_j$ is upper bounded by
$
\frac{u_j}{\ell_j} = \left(1-\frac{u_j-\ell_j}{u_j}\right)^{-1},
$
hence it suffices to show that $\frac{u_j-\ell_j}{u_j} = O(\beta + \eps)$.
Since $u_j - \ell_j > 1$ when the algorithm terminates,
\[
\frac{u_j - \ell_j}{u_j} < \frac{2(u_j - \ell_j) - 1}{u_j - \tfrac{1}{4}} = 2 \frac{(u_j - u_0) - (\ell_j - \ell_0)}{u_j - u_0}
  \le 2 \max_j\frac{\delta_{u,j} - \delta_{\ell,j}}{\delta_{u,j}} = O(\beta + \eps).
\]


Next, we analyze the running time of Algorithm~\ref{alg:ls17}.
The initial value of the potential function is $\Phi_{u_0, \ell_0}(0) = 2 \tr \exp (I) = 2n$.
By Lemma~\ref{lem:phi-no-increase} and a union bound over $j$, we have that with high probability, $\Phi_{u_j, \ell_j}(A_j) \le 2n$ for all $j \le O(\log n/\eps^2)$.
To see that $A_j$ must be far away from the barriers, consider only the contribution of $\lambda_{\min}(u_j I - A_j )$ to the potential function:
\[
2n \ge \Phi_{u_j}(A_j) = \tr \exp\left((u_j I - A_j)^{-1}\right) \ge \exp \left(\lambda_{\min}(u_j I - A_j)^{-1} \right).
\]
It follows that $\lambda_{\min}(u_j I - A) = \Omega(\log^{-1} n)$, and similarly $\lambda_{\min}(A_j - \ell_j I) = \Omega(\log^{-1} n)$.
Therefore, we know that $\rho \ge (1-\eps) (\lambda_{\min}\{u_j I - A_j, A_j - \ell I_j\})^2 = \Omega(\log^{-2} n)$, and $\delta_{u,j} - \delta_{\ell,j} = \Omega(\eps^2 \cdot \log^{-2} n)$ for all $j$.
Since the algorithm starts with $u_0 > \ell_0$ and terminates when $u_j - \ell_j > 1$, the number of iterations is at most $1/(\min_j (\delta_{u,j} - \delta_{\ell,j})) = O\left(\eps^{-2} \log^2 n \right)$.

It remains to show that each iteration takes nearly-linear time.
We maintain the matrices $A_j$ and $\Delta_j$ implicitly by the corresponding sets of weights, and add their weights together to get $A_{j+1}$. % $A_{j+1} = A_j + \Delta_j$.
The input and output of the SDP are also represented implicitly by the weights.
By Lemma~\ref{lem:sdp-sol}, we can compute $\rho$ and find a near-optimal solution to the SDP in \eqref{eqn:sdp-oracle} in time $\tilde O(m / \eps^{O(1)})$.
The overall running time is $\tilde O(\frac{\log^2 n}{\eps^2} \cdot \frac{m}{\eps^{O(1)}}) = \tilde O(m / \eps^{O(1)})$.
\end{proof}


%Difference compared to \cite{LeeS17}:
%(1) The growth rate of $u$ and $\ell$ is different compared to \cite{LeeS17}, which is necessary because...
%(2) No sparsity requirement so no need to do sampling, which allows us to remove the expectation in the analysis of potential functions.
%(We simplify the analysis of \cite{LeeS17}.)

%\subsection{From Graphs to Weight Matrices: Proof of Theorem~\ref{thm:main}}
\subsection{From Graphs to Weight Matrices: Consequences of Spectral Similarity}
\label{sec:lap-to-adj}
In this section, %we prove our main result (Theorem~\ref{thm:main}).
we show that the weights computed by Algorithm~\ref{alg:ls17} are useful for the semi-random matrix completion problem.
More specifically, we prove the following corollary of Theorem~\ref{thm:bss}.
Corollary~\ref{cor:preprocess} provides the spectral property that is crucial to our analysis of non-convex matrix-completion algorithms in Section~\ref{sec:matcomp}. % (see Lemma~\ref{lem:deterministc_main}).

\begin{corollary}
\label{cor:preprocess}
Fix $\beta > 0$.
Consider the matrix completion problem with ground truth $\Ms \in \R^{n_1 \times n_2}$.
There exists $p = O\left(\frac{\log n}{n_1 \beta^2}\right)$ such that if every entry of $\Ms$ is observed with probability at least $p$,
then w.h.p., we can compute a weight matrix $W \in \R^{n_1 \times n_2}$ in time $\tilde O(m / \beta^{O(1)})$, such that $W$ is supported on the observed entries, $\norminf{W} \le n_2$, $\normone{W} \le n_1$, and $\norm{W-J} = O(\beta \sqrt{n_1 n_2})$.
\end{corollary}
Recall that $J$ is the all ones matrix, $n = n_1 + n_2$, and we assume $n_1 \le n_2$.

Corollary~\ref{cor:preprocess} follows from Lemma~\ref{lem:random-graph}, Theorem~\ref{thm:bss}, and Lemma~\ref{lem:Lclose-Aclose}.
Lemma~\ref{lem:random-graph} provides concentration bounds for random matrices, which implies that when $p$ is large enough, the semi-random input contains a good subset of observations.
We can then apply Theorem~\ref{thm:bss} to show that our preprocessing algorithm can recover a good set of weights (i.e., a weighted graph that is spectrally similar to the complete bipartite graph).
Finally, Lemma~\ref{lem:Lclose-Aclose} shows that the closeness in the Laplacians of two graphs implies the closeness in their (normalized) adjacency matrices. %, which is a key property that we use in our analysis of non-convex matrix-completion algorithms in Section~\ref{sec:matcomp}. % (see Lemma~\ref{lem:deterministc_main}).

\begin{lemma}
\label{lem:random-graph}
Let $G$ denote the $n_1 \times n_2$ complete bipartite graph.
We write $n = n_1 + n_2$ for the number of vertices, and $m = n_1 n_2$ for the number of edges.
Let $H$ denote a random graph generated by including each edge of $G$ independently with probability $p$.
W.h.p, we can re-weight edges in $H$ so that the Laplacian matrix $L_H$ is $\eps$-spectrally similar with $L_G$, where $\eps = O\left(\sqrt{\frac{n \log n}{p m}}\right)$.
\end{lemma}
%For complete bipartite graphs, all edges have the same effective resistance. Therefore, uniform sampling among all the edges will produce a good spectral sparsifier %(see, e.g., \citep{RudelsonV07,SpielmanS11}). The lemma allows reweighting on $H$ because we sample with replacement and sum up the weights if an edge is chosen more than once.

%Next, we show that the closeness in the Laplacian matrices of two graphs implies the closeness in their (normalized) adjacency matrices.

\begin{lemma}
\label{lem:Lclose-Aclose}
Let $L = D - A$ and $\tilde L = \tilde{D} - \tilde{A}$ be two graph Laplacians, where $D$ is the degree matrix and $A$ is the adjacency matrix of the graph.
If $(1-\eps) L \preceq \tilde L \preceq L$, then we have:% (i) $(1-\eps) D_{i,i} \le \tilde{D}_{i,i} \le D_{i,i}$, and (ii) $\norm{D^{-1/2}(\tilde{A} - A)D^{-1/2}} \le 3\eps$.
\begin{enumerate}
\item[(1)] $(1-\eps) D_{i,i} \le \tilde{D}_{i,i} \le D_{i,i}$.
\item[(2)] $\norm{D^{-1/2}(\tilde{A} - A)D^{-1/2}} \le 3\eps$.
\end{enumerate}
\end{lemma}
We defer the proofs of Lemmas~\ref{lem:random-graph}~and~\ref{lem:Lclose-Aclose} to Appendix~\ref{apx:lap-to-adj}, and first prove Corollary~\ref{cor:preprocess}.
%\begin{proof}
%For (1), the spectral similarity between $L$ and $\tilde L$ implies that $(1-\eps) x^\top L x \le x^\top \tilde{L} x \le x^\top L x$ for all $x \in \R^n$.
%In particular, this holds for all standard basis vectors, so $(1-\eps) D_{i,i} \le \tilde{D}_{i,i} \le D_{i,i}$.
%
%For (2), we know that $0 \preceq L - \tilde{L} \preceq \eps L$ and similarly $0 \preceq D-\tilde{D} \preceq \eps D$, and therefore
%\begin{align*}
%\norm{D^{-1/2}(\tilde{A} - A)D^{-1/2}}
%& = \norm{D^{-1/2}(\tilde{D} - D + L - \tilde{L})D^{-1/2}} \\
%& \le \norm{D^{-1/2} (D - \tilde D) D^{-1/2}} + \norm{D^{-1/2} (L - \tilde L) D^{-1/2}} \\
%& \le \eps \norm{I} + \eps \norm{D^{-1/2} L D^{-1/2}} \le 3\eps. %\tag*{\jmlrQED} %\qedhere
%\end{align*}
%The last step uses the fact that eigenvalues of a normalized Laplacian matrix $D^{-1/2} L D^{-1/2}$ are always between $0$ and $2$.
%%\renewcommand{\jmlrQED}{}
%\end{proof}

\begin{proof}[Proof of Corollary~\ref{cor:preprocess}]
Let $G$ denote the $n_1 \times n_2$ complete bipartite graph ($n_1 \le n_2$).
Let $H$ be the graph corresponds to the entries revealed randomly, and let $H'$ denote the graph after the adversary added extra edges.
By Lemma~\ref{lem:random-graph}, for $p = O\left(\frac{\log n}{n_1 \beta^2}\right)$, with high probability, there exists edge weights for $H$ such that $(1-\beta) L_G \preceq L_H \preceq (1+\beta) L_G$.
% We assume this high probability event happens for the rest of the proof.

Because the edges of $H'$ is a superset of the edges of $H$, there exist edge weights for $H'$ such that the same condition holds.
Since the vectors $\{L_G^{-1/2} b_e\}_{e \in H'}$ satisfy the condition in Problem~\ref{prob:identity}, we can invoke Theorem~\ref{thm:bss} with $\eps = \beta$ to obtain a set of weights $\{\tilde w_e\}_{e \in H'}$ such that $(1-O(\beta))L_G \preceq \sum_{e \in H'} \tilde w_e b_e b_e^\top \preceq L_G$ in time $\tilde O(m / \beta^{O(1)})$, where $m$ is the number of edges in $H'$.

Let $A$ denote the adjacency matrix of $G$, and let $A'$ denote the adjacency matrix of $H'$ with weights $\tilde w_e$.
Since both $A'$ and $A_G$ include only edges in the complete bipartite graph, we can write
\[
A = \left(\begin{array}{cc} 0 & J \\ J^\top & 0 \end{array}\right), \qquad
A' = \left(\begin{array}{cc} 0 & W \\ W^\top & 0 \end{array}\right), \qquad
\]
where $J$ is the all ones matrix and $W \in \R^{n_1 \times n_2}$ contains the edge weights $\tilde w_e$ ($W_{i,j} = \tilde w_e$ for every $(i,j) \in H'$, and $W_{i,j} = 0$ otherwise).
By Lemma~\ref{lem:Lclose-Aclose}, the row sum of $W$ is at most $n_2$ for every row, and the column sum of $W$ is at most $n_1$ for every column.
Again by Lemma~\ref{lem:Lclose-Aclose}, $\norm{D^{-1/2} (A' - A_{G}) D^{-1/2}} \le O(\beta)$, which implies that $\norm{W - J} = \norm{A - A'} = O(\beta \sqrt{n_1 n_2})$.
\end{proof}

% !TEX root = main.tex

\section{Application to Matrix Completion}
\label{sec:matcomp}
Most analysis of matrix completion relies on the fact that the observed entries are sampled uniformly at random. Let $W_{i,j} = 1/p$ if entry $(i,j)$ is observed. This assumption is mostly used to prove {\em concentration inequalities} related to the norm of low-rank matrices $\|M\|_W^2$.  In particular, the following two lemmas are used in most papers.

Lemma~\ref{lem:tangent} shows that for an $M$ that is in the ``tangent space'' (the linear space of $\Us X^\top + Y(\Vs)^\top$), the norm of $Z$ is preserved after we restrict to the observed entries. 
Lemma~\ref{lem:Delta_mc} shows that the norm is preserved for every incoherent matrix $XY^\top$.

\begin{lemma}[\cite{recht2011simpler}]
\label{lem:tangent}
Suppose $\Ms = \Us\Sigs(\Vs)^\top$. Suppose entries are revealed with probability $p$ independently, weight matrix $W_{i,j} = 1/p$ if $(i,j)$ is revealed and $0$ otherwise. 
For any $0 < \delta < 1$, when $p \ge \Omega(\frac{\mu r}{\delta^2 n} \log n)$, with high probability over the randomness of $W$ we have
\[
|\|M\|_W - \|M\|_F| \le \delta \|M\|_F,
\]
for any matrix $M\in \R^{n_1\times n_2}$ of the form $M = \Us X^\top + Y(\Vs)^\top$.
\end{lemma}

\begin{lemma}
\label{lem:Delta_mc}
% Let set $A \subset [d]\times[d]$, and $\bar{\Omega}$ be the random set so that every element of $A$ is in $\bar{\Omega}$ with probability $p$. 
% Then,
Let $W$ be a random matrix where $W_{i,j} = 1/p$ with probability $p$, and $W_{i,j} = 0$ otherwise.
There exist universal constants $c_1$ and $c_2$, so that for any $\delta>0$, if $p \ge c_1  \frac{\log n}{\delta^2 n_1}$, then with probability at least $1-\frac{1}{2}n^{-4}$, we have for any matrices $X \in \R^{n_1\times r}, Y\in\R^{n_2\times r}$:
\begin{equation*}
\norm{XY^\top}^2_{W}  \le (1+\delta) \norm{X}_F^2\norm{Y}_F^2 + c_2 \sqrt{\tfrac{n}{p}}\norm{X}_F\norm{Y}_F\cdot \max_i\norm{X_i} \cdot \max_j\norm{Y_j}.
\end{equation*}
\end{lemma}

However, neither of these lemmas is applicable in our semi-random setting, because the weight matrix $W$ is no longer chosen randomly by nature.

Lemma~\ref{lem:tangent} is used in both the convex analysis (e.g.,~\citep{recht2011simpler}) and many local analyses for the non-convex methods (e.g.,~\citep{sun2015guaranteed}). Deterministic versions of Lemma~\ref{lem:tangent} include Assumption A2 in \citep{BhojanapalliJ14} and Assumption A3 in \citep{LiLR16}. Unfortunately, we do not know whether Assumption A2 in \citep{BhojanapalliJ14} is true even for random matrices, and we cannot guarantee Assumption A3 in \citep{LiLR16} because it is a condition that depends on the (unknown) ground truth.

Since we do not know how to obtain a deterministic version of Lemma~\ref{lem:tangent}, we turn our attention to Lemma~\ref{lem:Delta_mc}.
Lemma~\ref{lem:Delta_mc} is only used in more recent non-convex analyses (e.g.,~\citep{sun2015guaranteed, GeJZ17}).
We replace Lemma~\ref{lem:Delta_mc} with the following (stronger version of the) lemma, which states that if $W$ is close to the all ones matrix $J$ (which is guaranteed by our preprocessing algorithm), then the norm of $\|XY^\top\|_F$ is preserved after we weight the entries by $W$.
Recently, \citep{chen2017memory} has independently obtained a deterministic inequality similar to Lemma~\ref{lem:deterministc_main}.

\begin{lemma}[Preserving the Norm via Spectral Properties]
\label{lem:deterministc_main}
For any matrices $X\in \R^{n_1\times r}$, $Y \in \R^{n_2\times r}$, and $W \in \R^{n_1 \times n_2}$, we have
\[
|\|XY^\top\|_W^2 - \|XY^\top\|_F^2| \le \|W-J\| \|X\|_F\|Y\|_F\max_{i}\|X_i\|\max_{i}\|Y_i\|.
\]
\end{lemma}

\begin{proof}
Recall that $X_i$ is the $i$-th row of $X$, and $X\odot X \in \R^{n_1 \times r^2}$ is the Katri-Rao product.
% Similarly define $Y\odot Y$.
We have 
\[\inner{(X\odot X)_i, (Y\odot Y)_j} = \inner{X_i\otimes X_i, Y_j\otimes Y_j} = (X_i^\top Y_j)^2 = (X Y^\top)_{i,j}^2.\]

As a result, we know $\|XY^\top\|_W^2 = \inner{XY^\top, XY^\top}_W = \tr((X\odot X)^\top W (Y\odot Y))$, and $\|XY^\top\|_F^2 = \inner{XY^\top, XY^\top} = \tr((X\odot X)^\top J (Y\odot Y))$.

We can also bound the Frobenius norm of the two product by: $\|X\odot X\|_F \le \|X\|_F\max_{i}\|X_i\|$, $\|Y\odot Y\|_F \le \|Y\|_F\max_{i}\|Y_i\|$. Therefore,
\begin{align*}
|\inner{XY^\top, XY^\top}_W - \inner{XY^\top, XY^\top}|
& = |\tr((X\odot X)^\top (W-J) (Y\odot Y))| \\
& \le \|X\odot X\|_F \|(W-J) (Y\odot Y)\|_F \\
& \le \|W-J\| \|X\odot X\|_F \|(Y\odot Y)\|_F  \\
& \le \|W-J\| \|X\|_F\|Y\|_F\max_{i}\|X_i\|\max_{i}\|Y_i\|. \tag*{\jmlrQED} %\qedhere
\end{align*}
\renewcommand{\jmlrQED}{}
\end{proof}

Using this lemma, as well as techniques in \citep{GeJZ17}, we can prove the following theorem (see Appendix~\ref{app:matrix} for its proof):
%\begin{theorem} \label{thm:symmetric}
%For the asymmetric Objective~\eqref{eqn:symmetricobj}, if we choose $\alpha^2 = \frac{C\mu r\sigs_1}{n}$ and $\lambda = \frac{C^2 n}{\mu r\kappa^\star}$ where $C$ is a large enough universal constant. Assume the weight matrix satisfies $\normone{W_i} \le 2n$ for all rows $i$, and $\|W-J\| \le \epsilon \frac{cn}{\mu^2 r^2 (\kappa^\star)^2}$ where $\epsilon < 1$ and $c>0$ is a small enough universal constant, then for any local minimum of $f(U)$ we have $\|UU^\top-\Ms\|_F^2 \le \epsilon \|\Ms\|_F^2$. 
%\end{theorem}
%
%The theorem can be generalized to asymmetric case:
\begin{theorem} \label{thm:asymmetric_local}
For matrix completion problem with ground truth $\Ms \in \R^{n_1 \times n_2}$, let $\mu, r, \sigs_1, \kappa^\star$ be the incoherence parameter, rank, largest singular value and condition number of $\Ms$.
Fix any error parameter $0 < \eps < 1$.
Suppose the weight matrix $W \in \R^{n_1 \times n_2}$ satisfies $\norminf{W} \le n_2$, $\normone{W} \le n_1$, and $\|W-J\| \le \frac{\eps c\sqrt{n_1n_2}}{\mu^3 r^3 (\kappa^\star)^3}$ for a small enough universal constant $c$.
Let $\alpha_1^2 = \frac{C\mu r\sigs_1}{n_1},\alpha_2^2 = \frac{C\mu r\sigs_1}{n_2}$, $\lambda_1 = \frac{C^2 n_1}{\mu r\kappa^\star}$, and $\lambda_2 = \frac{C^2 n_1}{\mu r\kappa^\star}$ for some large enough universal constant $C$. Then, any local minimum $(U, V)$ of the asymmetric Objective~\eqref{eqn:asymmetricobj} satisfies $\|UV^\top-\Ms\|_F^2 \le \epsilon \|\Ms\|_F^2$. 
\end{theorem}

%Our main Theorem (Theorem~\ref{thm:main}) follows immediately by combining this theorem with Theorem~\ref{???}.

Our main result (Theorem~\ref{thm:main}) follows immediately from Corollary~\ref{cor:preprocess} and Theorem~\ref{thm:asymmetric_local}.

We can choose $\beta = O(\eps / (\mu^3 r^3 (\kappa^\star)^3))$ in Corollary~\ref{cor:preprocess} so that our preprocessing algorithm produces a weight matrix $W$ that satisfies the requirements of Theorem~\ref{thm:asymmetric_local}.
%Therefore, $W$ satisfies the conditions in Theorem~\ref{thm:asymmetric_local} if we set $\beta = O(\eps / (\mu^3 r^3 (\kappa^\star)^3))$.
By Theorem~\ref{thm:asymmetric_local}, the non-convex objective with weight matrix $W$ has no bad local optima.
The pre-processing time is $\tilde O(m / \beta^{O(1)}) = \tilde O(m \cdot \mathrm{poly}(\mu, r, \kappa^\star, \eps^{-1}))$.%, which is nearly linear in $m$.

\section{Conclusion}
We study the problem of incentivizing exploration with heterogeneous
user preferences.
We proposed an algorithm that achieves expected cumulative regret
$O(\ARMNUM \e^{2/\MinProb} + \ARMNUM \log^3(T))$,
using expected cumulative payments of $O(\ARMNUM^2 \e^{2/\MinProb})$.
It is possible to improve these bounds to polynomial (in \ARMNUM and
$1/\MinProb$) when \MinProb is known or the preference distribution is
discrete.
In fact, we conjecture that this should be possible even in the full
generality of our model.
As a first step towards such a polynomial bound, we can obtain an exponential dependence on
$1/(\MinProb \ARMNUM)$ by changing the probability threshold to be $\frac{1}{\ARMNUM\log(s)}$
\footnote{This will lead to a different dependence on $\ARMNUM$
in the regret bound as well as the payment bound}, which gives polynomial dependence unless some
arm has a much smaller fraction of the population preferring it.

Taking this goal one step further, we would like to 
develop algorithms that do not require all arms to be preferred by a
strictly positive fraction of agents.
An alternate algorithm might only incentivize an arm if its estimated
attribute vector is close enough to a Pareto frontier.
The regret will then be $\Omega(\log(T))$ when at least one arm falls
below the Pareto frontier, as we no longer have free exploration of
all arms. 
It is likely that a bound will deteriorate as the number of such
unpreferred arms increases.

Finally, it would be desirable to generalize to utility
functions beyond inner products.
We believe that similar results hold for arbitrary
Lipschitz-continuous utility functions of the arm's attribute vector,
and that only minor modifications are necessary to the algorithm and
proofs.


\section*{Acknowledgments} This work is supported by NSF CCF-1704656. We thank Qingqing Huang, Andrej Risteski, Srinadh Bhojanapalli, Yin Tat Lee for discussions at various stages of the work. Yu Cheng is also supported in part by NSF CCF-1527084, CCF-1535972, CCF-1637397, IIS-1447554, and NSF CAREER Award CCF-1750140.

%\bibliographystyle{alpha}
\bibliography{names,conferences,spectral,matrix_ref,semi_random}

\appendix

% !TEX root = main.tex

\section{Counter Examples for Non-convex Approaches}
\label{app:examples}

In this section, we give counter-examples to some non-convex methods for matrix completion. For simplicity, we give examples for the {\em weighted} version of the non-convex objective. These examples can be translated to the semi-random adversary model using standard sampling techniques.

We will give the counter-examples in a simpler setting where $\Ms$ is known to be symmetric, we have $\Ms$ is an $n\times n$ matrix that can be decomposed as $\Ms = \Us(\Us)^\top$. In this case, we optimize:
\begin{equation}
\label{eqn:symmetricobj}
\min \; f(U) = \frac{1}{2}\|UU^\top - \Ms\|_W^2 + Q(U). 
\end{equation}
Here $Q(U)$ is the regularizer $\lambda \sum_{i=1}^n (\normtwo{U_i} - \alpha)_+^4$ (where $x_+ = \max\{x,0\}$). Parameters $\lambda, \alpha$ in the regularizer is specified later (see Lemma~\ref{lem:symmetricnormbound}).
%
%Here $\lambda, \alpha$ are parameters to be specified (see Lemma~\ref{lem:symmetricnormbound})% \yccomment{Where did we specify them?}
%, $x_+ = \max\{x,0\}$. Also, let $Q(U)$ be the regularizer $\lambda \sum_{i=1}^n (\normtwo{U_i} - \alpha)_+^4$ (where $x_+ = \max\{x,0\}$). 
Our examples also work for the asymmetric Objective~\eqref{eqn:asymmetricobj}.

\paragraph{Example Where Objective~\eqref{eqn:symmetricobj} Has Spurious Local Minimum.}

We first give an example where Objective~\eqref{eqn:symmetricobj} has a spurious local minimum. This is a simple rank 1 case where the intended solution $\Ms = u^\star (u^\star)^\top$ is the all ones matrix, and $u^\star = (1,1,...,1)^\top$ is the all ones vector.

In this example, all vectors will be represented by two blocks of size $n/2 \times 1$, and the value within each block will be the same; similarly all matrices will be partitioned into $2\times 2$ blocks (where each block has size $n/2\times n/2$), and entries within each block are the same.

For this example, we choose 
\[
u = \left(\begin{array}{c} \beta \\ -\beta \end{array}\right), \quad W = \left(\begin{array}{cc} \gamma J & J \\ J & \gamma J \end{array}\right),
\]
for any parameters $2^{-1/4} < \beta \le \frac{9}{10}$ and $\gamma = \frac{1+\beta^2}{1-\beta^2} < 10$.
%This example fits in the semi-random model, because the adversary will reveal each entry of $\Ms$ with probability $p_{i,j} \in [p, \gamma p]$. %each entry of $\Ms$ is revealed with probability at least $\gamma$. %constant probability.

\begin{lemma}
For the setting of $\Ms$, $u$, $W$ stated above, the objective function \eqref{eqn:symmetricobj} has a local minimum at $u$.
\end{lemma}

We will prove this lemma by second-order sufficient condition: gradient is 0 and Hessian is positive definite. Clearly $u$ is incoherent, so the incoherence regularizer does not matter. We first check the gradient of $f(u)$ is 0. This is due to our choice of $\gamma$ satisfies $\gamma(\beta^2-1) + (\beta^2+1) = 0$:
\[
\nabla f(u) = [(W + W^\top) * (uu^\top - \Ms)] u = 2[W * (uu^\top - \Ms)] u = 0.
\]
Next, we consider the Hessian of $f(u)$.
For any vector $\delta \in \R^n$, we know that
\[
\frac{1}{2}\delta^\top [\nabla^2 f(u)] \delta = \|\delta u^\top\|_W^2 + \inner{2uu^\top - \Ms,\delta\delta^\top}_W.
\]
We show this is strictly greater than 0 for all $\delta \neq 0$.
Let $\delta = \left( \begin{array}{c} \delta_1 \\ \delta_2 \end{array} \right)$ for $\delta_1, \delta_2 \in \R^{n/2}$.
Notice that $\|\delta u^\top\|_W^2$ is non-negative, therefore, we have
\begin{align*}
& \|\delta u^\top\|_W^2 + \inner{2uu^\top - \Ms,\delta\delta^\top}_W \\
& \ge \inner{2uu^\top - \Ms,\delta\delta^\top}_W \\
% & = 2 \sum_{i,j} W_{ij} u_i u_j \delta_i \delta_j - \sum_{i,j} W_{ij} \Ms_{ij} \delta_i \delta_j \\
& = 2\beta^2 \left(\gamma\normone{\delta_1}^2 + \gamma\normone{\delta_2}^2 - 2 \normone{\delta_1} \normone{\delta_2} \right) - \left(\gamma\normone{\delta_1}^2 + \gamma\normone{\delta_2}^2 + 2 \normone{\delta_1} \normone{\delta_2}\right) \\
& \ge 2\beta^2 (\gamma-1) \left(\normone{\delta_1}^2 + \normone{\delta_2}^2\right) - (\gamma+1) \left(\normone{\delta_1}^2 + \normone{\delta_2}^2\right) \\
& = (2\beta^2 (\gamma-1) - (\gamma+1))\left(\normone{\delta_1}^2 + \normone{\delta_2}^2\right) > 0.
\end{align*}

The last step is due to $\delta \neq 0$ and our choice of $\beta$ and $\gamma$.
%We can verify the first term is positive when $\beta \to 1$ and $\gamma \to 0$.

\paragraph{Example Where SVD Initialization Gives the Wrong Subspace.}

Many other non-convex methods depend on SVD to do initialization (e.g., \citep{jain2013low, hardt2014understanding}). Now we give an example where SVD cannot find the subspace correctly.

This is a rank-$2$ example.
For simplicity, all vectors are divided into blocks of size $n/4 \times 1$, and matrices are divided into blocks of size $n/4\times n/4$.
We write these as $4 \times 2$ or $4\times 4$ matrices, and they should be interpreted as blocks (constant multiplied by $J_{n/4\times 1}$ or $J_{n/4\times n/4}$).
The matrix $\mbox{Diag}(4,1)$ is a $2 \times 2$ diagonal matrix.
\[
\Ms = \left(\begin{array}{cc}1 & 1 \\ 1 & 1 \\ 1 & -1 \\ 1 & -1\end{array}\right) \mbox{Diag}(4,1)\left(\begin{array}{cccc}1 & 1 & 1 & 1 \\ 1 & 1 & -1 & -1\end{array}\right)
\]

\[
W = \left(\begin{array}{cccc}2 & 1 & 2 & 1 \\ 1 & 2 & 1 & 2 \\2 &1 & 2 & 1  \\ 1 & 2 & 1 & 2\end{array}\right); \qquad W*\Ms =  \left(\begin{array}{cccc}10 & 5 & 6 & 3 \\ 5 & 10 & 3 & 6 \\6 &3 & 10 & 5  \\ 3 & 6 & 5 & 10\end{array}\right)
\]

\begin{lemma}
Under the above setting for $\Ms, W$, the top two singular vectors of $W*\Ms$ are (up to normalization) $(1,1,1,1)$ and $(1,-1,1,-1)$. The principled angle between this subspace and the true subspace $\mbox{span}(\Ms)$ is 90 degrees.
\end{lemma}

This lemma is easy to verify numerically by computing the SVD of the $ 4\times 4$ matrix. The SVD of the original matrix follows the same block structure.

\paragraph{Converting Weighted Examples to Semi-Random Examples.}
In order to get counter examples in the semi-random model, pick a probability of observation $p$ (we need $p \ge \mbox{poly}(r) \log(n) /n$, and in our examples $p$ can be as large as $1/10$).
The semi-random adversary will reveal entry $(i,j)$ with probability $p_{i,j} = p W_{i,j} \ge p$ for the $W$ given in the examples.
This way, the expectation of the observed entries is exactly $W * \Ms$ (after scaling by $1/p$), and the expectation of the objective function is equal to $\|UU^\top - \Ms\|_W^2$.

For the first example, by standard concentration results, we know that the gradient and Hessian of the objective function are close to their expectations, so there is a local minimum near $u$.
For the second example, by standard random matrix theory, we know when $p$ is large enough the top singular space of the observed matrix is close to the top singular space of $W*\Ms$ (and thus far from the correct space).

%In fact, for the counterexamples we can consider a weaker version of the semi-random adversary. This adversary looks at $\Ms$,  and is allowed to increase the reveal probabilities of the entries. After the perturbation of semi-random adversary, entry $(i,j)$ is revealed with probability $p_{i,j}$ where $p_{i,j}\ge p$ for all $(i,j)$. In this case, the expected objective function is similar to a weighted low-rank factorization problem. In particular, let $W$ be a matrix where $W_{i,j} = \frac{1}{p} p_{i,j}$, the expected objective function is going to be $\|UV^\top - \Ms\|_W^2$. All of our counter-examples work in this setting where the adversary is weaker; on the other hand our algorithm works for a stronger adversary that can also look at the actual entries that are revealed.

%In order to get an example for the semi-random adversary, pick a probability of observation $p$ ($p$ needs to be larger than $\mbox{poly}(r)/d$, and in our examples $p$ can be as large as $1/2$). The semi-random adversary will reveal entry $i,j$ with probability $p W_{i,j}$. For the first example, the expectation of Objective~\eqref{eqn:symmetricobj} is exactly equal to the weighted version. By standard concentration results we know the gradient and Hessian are close to the expectation, and there is a local minimum near the given point. For the second example, the expectation of the observed entries is exactly $W*\Ms$. Therefore by standard random matrix theory, we know when $p$ is large enough the top singular space of the observed matrix is close to the top singular space of $W*\Ms$ (and thus far from the correct space).

% !TEX root = main.tex

\section{Omitted Proofs from Section~\ref{sec:bss}}
\label{app:bss}

In this section, we give more details about Section~\ref{sec:bss} and prove Lemmas~\ref{lem:sdp-sol}~and~\ref{lem:phi-no-increase}.

Recall that we are given a set of input vectors $\{v_i\}_{i=1}^m$ where each $v_i = L^{-1/2} b_i$ for a fixed Laplacian $L$ and some $b_i$ representing an edge, and we assume that there exist weights $w_i$ such that $(1-\beta) I \preceq \sum_{i=1}^m w_i v_i v_i^\top \preceq (1+\beta) I$.
The goal is to compute a set of weights $\tilde w_i$ in nearly-linear time so that $(1-O(\beta)-\eps)I \preceq \sum_{i=1}^m \tilde w_i v_i v_i^\top \preceq I$.

We maintain barrier values $u$ and $\ell$, and a weighted sum $A = \sum_{i=1}^m \tilde w_i v_iv_i^\top$ such that $\ell I \prec A \prec uI$.
It is worth noting that we never explicitly compute the vectors $v_i = L^{-1/2} b_i$.  We store $A$ by keeping track of the weights $w_i$.
When updating the weights, we approximate the quantities $v_i^\top C v_i$ (for some matrix $C = C(A)$) simultaneously for all $i$ in nearly-linear time (see Lemma~\ref{lem:sdp-computation}).

We use the following potential functions proposed in \citep{LeeS17} to measure how far $A$ is away from the barriers:
\begin{align*}
\Phi_{u,\ell}(A) & = \Phi_u(A)+\Phi_{\ell}(A), \text{ where} \\
\Phi_u(A) & = \tr \exp \left((u I - A)^{-1}\right), \\
\Phi_\ell(A) & = \tr \exp \left((A - \ell I)^{-1}\right).
\end{align*}

The derivatives of the potential functions with respect to $A$ are
\begin{align*}
\nabla \Phi_u (A) & = \exp \left((u I - A)^{-1}\right) (u I - A)^{-2}, \\ %\text{ and} \\
\nabla \Phi_\ell (A) & = - \exp \left((A - \ell I)^{-1}\right) (A - \ell I)^{-2}.
\end{align*}

By convexity we have
\[
\Phi_{u,\ell}(A+\Delta) \ge \Phi_{u,\ell}(A) + \nabla \Phi_{u}(A) \bullet \Delta + \nabla \Phi_{\ell}(A) \bullet \Delta.
\]

The following lemma from \cite{LeeS17} shows that when $\Delta$ is small, the first-order approximation of $\Phi_{u,\ell}(A+\Delta)$ is a good estimation.

\begin{lemma}[\cite{LeeS17}]
\label{lem:potential-FO}
Let $A$ be a symmetric matrix.
Let $\ell < u$ be barrier values such that $u - \ell \le 1$ and $\ell I \prec A \prec u I$.
Assume that $0 \preceq \Delta$, $\Delta \preceq \eps (uI - A)^2$, and $\Delta \preceq \eps (A - \ell I)^2$ for $\eps \le 1/10$. Then,
\begin{alignat*}{3}
\Phi_{u}(A+\Delta) & \le \Phi_u(A) + (1+2\eps) \nabla \Phi_u(A) \bullet \Delta &&= \Phi_u(A) + (1+2\eps) C_- \bullet \Delta, \\
\Phi_{\ell}(A+\Delta) & \le \Phi_\ell(A) + (1-2\eps) \nabla \Phi_\ell(A) \bullet \Delta &&= \Phi_\ell(A) - (1-2\eps) C_+ \bullet \Delta; \\
\Phi_{u}(A-\Delta) & \le \Phi_u(A) - (1-2\eps) \nabla \Phi_u(A) \bullet \Delta &&= \Phi_u(A) - (1-2\eps) C_- \bullet \Delta, \\
\Phi_{\ell}(A-\Delta) & \le \Phi_\ell(A) - (1+2\eps) \nabla \Phi_\ell(A) \bullet \Delta &&= \Phi_\ell(A) + (1+2\eps) C_+ \bullet \Delta.
\end{alignat*}
\end{lemma}

%Recall that $C = -\left[(1+2\delta) \nabla \Phi_u(A) + (1-2\delta) \nabla \Phi_\ell(A)\right]$.
%Lemma~\ref{lem:potential-FO} implies that $\Phi_{u,\ell}(A+\Delta) \le \Phi_{u,\ell}(A) - C \bullet \Delta$.
Recall that for notational convenience, we write 
\begin{alignat*}{3}
C_+ & = -\nabla \Phi_\ell (A) &&= \exp \left((A - \ell I)^{-1}\right) (A - \ell I)^{-2}, \\
C_- & = \nabla \Phi_u (A) &&= \exp \left((u I - A)^{-1}\right) (u I - A)^{-2}.
\end{alignat*}

When $\ell I \prec A \prec u I$, both $C_+$ and $C_-$ are PSD matrices.
Recall that $\rho = (\lambda_{\min}\{uI-A, A-\ell I\} )^2$.
We are interested in the following packing SDP:
\begin{lp*}
\maxi {(C_+ - C_-) \bullet X} \tag{\ref{eqn:sdp-oracle}}
\st \con{X \preceq \eps \rho I, \quad X = \sum_{i=1}^m x_i v_i v_i^\top \text{ (which implies $0 \preceq X$)}.}
\end{lp*}

The constraint $X \preceq \eps \rho I$ implies that $X \preceq \eps(uI-A)^2$ and $X \preceq \eps(A-\ell I)^2$.
Thus, by Lemma~\ref{lem:potential-FO}, when $\eps \le 1/10$, the first-order approximation of the potential function is accurate:
\[
\Phi_{u,\ell}(X) - (C_+ - C_-) \bullet X \le \Phi_{u,\ell}(A+X) \le \Phi_{u,\ell}(A) - ((1-2\eps) C_+ - (1+2\eps)C_- ) \bullet X.
\]
Let $C = C_+ - C_-$. The SDP in \eqref{eqn:sdp-oracle} is naturally trying to find an $X$ to maximize the $C \bullet X$, while making sure $C \bullet X$ is a good approximation to the reduction of the potential function.

Ideally, we would like to have $X = \eps \rho I$ so that $A$ grows equally in each dimension, and the potential function stays the same:
\[
\Phi_{u+\eps\rho,\ell+\eps\rho}(A+\eps\rho I) = \Phi_{u,\ell}(A).
\]
When $X = \eps \rho I$, the objective value of the SDP is
\[
(C_+ - C_-) \bullet X = \eps \rho (\tr(C_+) - \tr(C_-)).
\]
While in general $I$ may not be in the span of the rank-one matrices, we will show in Lemma~\ref{lem:sdp-computation} that we can compute an $X$ with
\[
(C_+ - C_-) \bullet X \ge \frac{\eps \rho}{2} \left((1-\beta-2\eps)\tr(C_+) - (1+\beta+2\eps)\tr(C_-)\right).
\]
We first prove Lemma~\ref{lem:phi-no-increase}, which states that the potential function does not increase if $\Delta_j$ in each iteration satisfies Lemma~\ref{lem:sdp-computation}.

\begin{proof}[Proof of Lemma~\ref{lem:phi-no-increase}]
We want to show $\Phi_{u_{j+1},\ell_{j+1}}(A_{j+1}) \le \Phi_{u_{j},\ell_{j}}(A_{j})$, where $A_{j+1} = A_j + \Delta_j$.

When we increase the weights $\{w_i\}$ and expand $A$, the lower barrier potential $\Phi_\ell$ decreases (since $A$ gets farther away from $\ell I$) and the upper barrier $\Phi_u$ increases (since $A$ gets closer to $u I$).
When we increase the barrier values $u$ and $\ell$, the opposite happens: $\Phi_\ell$ increases and $\Phi_u$ decreases.
Intuitively, the proof works by carefully increasing $u$ and $\ell$ to cancel out the change due to adding $\Delta_j$, while making sure both $u$ and $\ell$ increase at roughly the same rate.

Recall that $C_+ = -\nabla \Phi_{\ell_j} (A_j)$ and $C_- = \nabla \Phi_{u_j} (A_j)$.
Formally, we have
\begin{align}
& \Phi_{u_{j+1},\ell_{j+1}}(A_{j+1}) \notag \\
% & = \Phi_{u_j+\delta_{u,j}, \ell_j+\delta_{\ell,j}}(A_{j+1}) \notag \\
& = \Phi_{u_j}(A_{j+1}-\delta_{u,j} I) + \Phi_{\ell_j}(A_{j+1}-\delta_{\ell,j} I) \notag \\
& \le \Phi_{u_j,\ell_j}(A_{j+1}) - (1-2\eps) \nabla \Phi_{u_j}(A_{j+1}) \bullet (\delta_{u,j} I) - (1+2\eps) \nabla \Phi_{\ell_j}(A_{j+1}) \bullet (\delta_{\ell,j} I) \tag{Lemma~\ref{lem:potential-FO}}\\
& \le \Phi_{u_j,\ell_j}(A_{j+1}) - (1-2\eps) \nabla \Phi_{u_j}(A_{j}) \bullet (\delta_{u,j} I) - (1+2\eps) \nabla \Phi_{\ell_j}(A_{j}) \bullet (\delta_{\ell,j} I) \label{eqn:nabla-Aj1} \\
& = \Phi_{u_j,\ell_j}(A_{j+1}) - (1-2\eps) \delta_{u,j} \tr(C_-) + (1+2\eps) \delta_{\ell,j} \tr(C_+) \label{eqn:phi-change-1}.
\end{align}
Our choice of $\delta_{u,j}, \delta_{\ell,j}$ satisfies that $\delta_{u,j}, \delta_{\ell,j} \le \eps \rho$ when $\eps, \beta \le 1/10$, which allows us to apply Lemma~\ref{lem:potential-FO}.
Step~\eqref{eqn:nabla-Aj1} uses $A_{j} \preceq A_{j+1}$ and the fact that, for any $\ell I \prec A_1 \preceq A_2 \prec u I$, we have $0 \preceq \nabla \Phi_u (A_1) \preceq \nabla \Phi_u (A_2)$ and $0 \preceq -\nabla \Phi_\ell (A_2) \preceq -\nabla \Phi_\ell (A_1)$.

We continue to bound the change of the potential function when we set $A_{j+1} = A_j + \Delta_j$ for any $\Delta_j$ that satisfies Lemma~\ref{lem:sdp-computation}.
\begin{align}
& \Phi_{u_j,\ell_j}(A_{j+1})
= \Phi_{u_j,\ell_j}(A_{j} + \Delta_j) \notag \\
& \le \Phi_{u_j, \ell_j}(A_j) + (1+2\eps) C_- \bullet \Delta_j - (1-2\eps) C_+ \bullet \Delta_j \tag{Lemma~\ref{lem:potential-FO}} \\
& = \Phi_{u_j, \ell_j}(A_j) - (C_+ - C_-) \bullet \Delta_j + 2\eps (C_+ + C_-) \bullet \Delta_j \notag \\
& = \Phi_{u_j, \ell_j}(A_j) - (C_+ - C_-) \bullet \Delta_j + 2\eps^2\rho \tr(C_+ + C_-) \tag{$\Delta_j \le \eps\rho I$} \notag \\
& \le \Phi_{u_j, \ell_j}(A_j) - \frac{\eps\rho}{2}\left[(1-\beta-\eps)\tr(C_+) - (1+\beta+\eps)\tr(C_+)\right] + 2\eps^2 \rho\tr(C_+ + C_-) \tag{Lemma~\ref{lem:sdp-computation}} \\
& = \Phi_{u_j, \ell_j}(A_j) + \frac{\eps\rho(1+\beta+5\eps)}{2} \tr(C_-) - \frac{\eps\rho(1-\beta-5\eps)}{2} \tr(C^+) \label{eqn:phi-change-2}.
\end{align}
We conclude the proof by comparing lines \eqref{eqn:phi-change-1} and \eqref{eqn:phi-change-2}, and setting $\delta_{u,j} = \frac{\eps\rho(1+\beta+5\eps)}{2(1-2\eps)}$ and $\delta_{\ell,j} = \frac{\eps\rho(1-\beta-5\eps)}{2(1+2\eps)}$.
The trace terms cancel out and we get $\Phi_{u_{j+1},\ell_{j+1}}(A_{j+1}) \le \Phi_{u_{j},\ell_{j}}(A_{j})$ as needed.
\end{proof}

We remark that the best we can hope for is to increase the upper and lower barriers at a rate of $1+\beta$ vs. $1-\beta$ (which is the case as $\eps \to 0$), because the ground-truth is a set of weights $w_i$ with $(1-\beta) I \preceq \sum_i w_i v_i v_i^\top \preceq (1+\beta) I$. Our algorithm only achieves a ratio of $1+O(\beta+\eps)$ vs. $1-O(\beta+\eps)$ for several reasons: (1) the error in the first-order approximation of the potential function, (2) we solve the SDP approximately, and (3) we use Taylor expansion and Johnson-Lindenstrauss to speed up the computation.



We break Lemma~\ref{lem:sdp-sol} into two lemmas and prove them separately.
Lemma~\ref{lem:sdp-sol-exist} states that SDP~\eqref{eqn:sdp-oracle} has a good solution. Lemma~\ref{lem:sdp-computation} shows that we can compute $\rho$ and solve this packing SDP~\eqref{eqn:sdp-oracle} approximately in nearly-linear time.

\begin{lemma}
\label{lem:sdp-sol-exist}
Let $A$ be a symmetric matrix. Let $\ell < u$ be barrier values such that $\ell I \prec A \prec u I$.
The SDP in \eqref{eqn:sdp-oracle} has a solution $X$ with
\[
C \bullet X \ge \eps \rho \left(\frac{1-\beta}{1+\beta}\tr(C_+) - \tr(C_-)\right).
\]
\end{lemma}
\begin{proof}%[Proof of Lemma~\ref{lem:sdp-sol}]
%Ideally, we would like to have $X = \eps \rho I$ so that $A$ grows equally in each dimension, and the potential function stays the same:
%\[
%\Phi_{u+\eps\rho,\ell+\eps\rho}(A+\eps\rho I) = \Phi_{u,\ell}(A).
%\]
When $X = \eps \rho I$, the objective value is $C \bullet X = \eps \rho \tr(C)$.
%While in general $I$ may not be in the span of the rank-one matrices, we will show that there is a good solution $X$ with $C \bullet X$ close to $\eps \rho \tr(C)$.
% Moreover, we can compute an approximately optimal solution to this SDP in nearly-line time.
Note that $C$ is not PSD, so when $X \approx \eps \rho I$, we need to split $C$ into the difference of two PSD matrices $C_+$ and $C_-$ to bound the error.

Recall that there exists a set of weights $w_i$ with $(1-\beta) I \preceq \sum_{i=1}^m w_i v_i v_i^\top \preceq (1+\beta)I$.
We look at a solution of this SDP with $X = \frac{\rho}{1+\beta} \sum_{i=1}^m w_i v_i v_i^\top$.
It follows directly that $X$ is feasible since $X$ is a weighted sum of $v_i v_i^\top$ and $X \preceq \eps \rho I$.
For the objective value, since $\frac{1-\beta}{1+\beta} \eps \rho I \preceq X \preceq \eps \rho I$,
\begin{equation*}
%\mathrm{OPT} & \ge 
(C_+ - C_-) \bullet X
  \ge C_+ \bullet (\frac{1-\beta}{1+\beta} \eps \rho I) - C_- \bullet (\eps \rho I)
  = \eps\rho\left(\frac{1-\beta}{1+\beta} \tr(C_+) - \tr(C_-)\right). \tag*{\jmlrQED} %\qedhere 
\end{equation*}
\renewcommand{\jmlrQED}{}
\end{proof}

We now provide details about how to implement Algorithm~\ref{alg:ls17} in nearly-linear time.
We remark that similar implementations were shown in~\cite{AllenLO15, LeeS15, LeeS17}.

\begin{lemma}
\label{lem:sdp-computation}
Fix $0 < \beta,\eps \le 1/10$. %Let $\{v_i\}_{i=1}^m$ be the set of input vectors given in Theorem~\ref{thm:main}.
%, and a graph Laplacian $L \in \R^{n \times n}$.
%Given a set of $m$ vectors $\{v_i = L^{-1/2} b_i\}_{i=1}^m$, assume there exist weights $w_i \ge 0$ such that
%\[
%(1-\beta) I \preceq \sum_{i=1}^m w_i v_i v_i^\top \preceq (1+\beta) I.
%\]
Given an $n \times n$ matrix $A = \sum_{i=1}^m w_i v_i v_i^\top$ represented by a set of weights $\{w_i \ge 0\}_{i=1}^m$, let $\ell < u$ be barrier values such that $u - \ell \le 1$ and $(\ell + g) I \prec A \prec (u - g) I$ for some gap $g = \Omega(\log^{-2} n)$.

We can compute $\rho$ and weights $\{\tilde w_i\}_{i=1}^m$ in $\tilde O(m / \eps^{5})$ time, such that with high probability,
\begin{enumerate}
\item[(1)] $\rho \in [1-\eps, 1] \cdot (\lambda_{\min}\{u I - A, A - \ell I\})^2$;
\item[(2)] $X = \sum_{i=1}^m \tilde x_i v_i v_i^\top$ satisfies $X \preceq \eps\rho I$ and
\[
(C_+ - C_-) \bullet X \ge \frac{\eps \rho}{2} \left((1-\beta-\eps)\tr(C_+) - (1+\beta+\eps)\tr(C_-)\right).
\]
\end{enumerate}
\end{lemma}
\begin{proof}
Recall that $v_i = L^{-1/2} b_i$ for a fixed Laplacian $L$.
Let $\hat L = \sum_i w_i b_i b_i^\top$ be the Laplacian specified by the weights of $A$.
In this proof, we will frequently use the following fact,
\[
A = \sum_{i=1}^m w_i v_i v_i^\top = L^{-1/2} \sum_{i=1}^m w_i b_i b_i^\top L^{-1/2} = L^{-1/2} \hat L L^{-1/2}.
\]

\begin{enumerate}
\item[(1)] We show how to compute $\rho \in [1-\eps, 1] \cdot \lambda_{\min}(u I - A)^2$. The approach is similar for $A - \ell I$.
It is sufficient to compute $\rho \approx_{\eps/2} \lambda_{\min}(u I - A)^2$.~\footnote{We write $a \approx_{\eps} b$ for $\exp(-\eps) a \le b \le \exp(\eps) a$. This extends naturally to PSD matrices, where $A \approx_{\eps} B$ means $\exp(-\eps)A \preceq B \preceq \exp(\eps) A$. It is sufficient to approximate $\rho$ up to a factor of $1\pm \exp(\eps/2)$ because $\exp(\eps) \le \frac{1}{1-\eps}$.}
By Lemma~\ref{lem:mat-taylor}, there exists a degree $\tilde O\left(\frac{\log(1/\eps)}{g}\right) = \tilde O(\log^2 n \log(1/\eps))$ polynomial $p(A)$ such that $p(A) \approx_{\eps/4} (u I - A)^{-2}$.
%This polynomial can be viewed as a low-degree polynomial of $A$, which we abuse notation and denote by $p(A)$.
Since $\lambda_{\max} (p(A)) \approx_{\eps/4} \lambda_{\max} ((u I - A)^{-2}) = \left(\lambda_{\min} (u I - A)^2\right)^{-1}$, it is sufficient to approximate $\lambda_{\max} (p(A))$.

Observe that for any $n \times n$ PSD matrix $M$,
\[
\lambda_{\max}(M) \le \left(\tr\left(M^{2k}\right)\right)^{1/2k} \le n^{1/2k} \lambda_{\max} (M).
\]
In particular, for $k = O(\log n / \eps)$ we can get $\left(\tr(p(A)^{2k})\right)^{1/2k} \approx_{\eps/4} \lambda_{\max}(p(A))$,
and thus, we can return $\rho = \left(\tr(p(A)^{2k})\right)^{-1/2k} \approx_{\eps/2} \lambda_{\min}(u I - A)^2$.

It remains to show that we can approximate
\[
\tr(p(A)^{2k}) = \tr(p(L^{-1/2} \hat L L^{-1/2})^{2k}) = \tr(p(L^{-1} \hat L)^{2k}).
\]
Let $M = p(L^{-1} \hat L)^{k}$ so $\tr(p(A)^{2k}) = \tr(M^2)$.
We approximate each diagonal entry of $M^2$ by writing it as $\left(M^2\right)_{i,i} = \chi_i^\top M M \chi_i = \normtwo{M \chi_i}^2$, where $\chi_i$ denote the $i$-th standard basis vector.
By the Johnson-Lindenstrauss lemma, we can generate a random $O(\log n/\eps^2) \times n$ matrix $Q$, so that with high probability, for all $1 \le i \le n$,
\[
\normtwo{M \chi_i}^2 = \normtwo{Q M \chi_i}^2.
\]
Note that $Q M \chi_i$ is the $i$-th column of $QM$.
We can compute (approximately) $Q M = \left(M Q^\top\right)^\top$ by multiplying each column of $Q^\top$ through $M$.

This can be done in time $\tilde O(n/\eps^5)$, because $Q^\top$ has $O(\log n/\eps^2)$ columns, and matrix-vector multiplication with $M = p(L^{-1} \hat L)^{k}$ can be implemented using $k \cdot \mbox{deg}(p) = \tilde O(\log^3 n / \eps)$ matrix-vector multiplications with $L^{-1} \hat L$. We will show that matrix-vector multiplication with $L^{-1} \hat L$ can be done in time $\tilde O(n / \eps^2)$, so the overall running time is $\tilde O(n / \eps^5)$.

Recall that the number of edges in $\hat L$ is at most $m$.
Let $m'$ denote the number of edges in $L$.
W.l.o.g., we can assume both $m, m' = O(n/\eps^2)$ by sparsifying the input graphs first.
Therefore, one matrix-vector multiplication with $L^{-1} \hat L$ can be done in time $\tilde O(n/\eps^2)$, by first multiplying the vector through $\hat L$, and then solving a linear system in $L$ in $\tilde O(m' \log (1/\eps))$ time~\citep{SpielmanT14, KoutisMP11, KelnerOSZ13, PengS14, ChengCLPT15, CohenKMPPRX14, KyngS16}.

\item[(2)]
Since we represent the variable $X$ of the SDP by a set of weights $\{x_i\}_{i=1}^m$, the objective function is of the form $C \bullet X = C \bullet \left(\sum_{i=1}^m x_i v_i v_i^\top\right) = \sum_{i=1}^m x_i (v_i^\top C v_i)$.
Let $c \in \R^m$ be a vector with $c_i = v_i^\top C v_i$.
The SDP in \eqref{eqn:sdp-oracle} can be rewritten as
\begin{lp*}
\maxi {c^\top x}
\st \con{\sum_{i=1}^m x_i (v_i v_i^\top) \preceq \eps \rho I.}
\end{lp*}

This is a packing SDP that can be solved in polylogarithmic iterations (see, e.g., \citep{JainY11, AllenLO16, PengTZ16}).
Formally, we use Lemma~\ref{lem:alo16} from \cite{AllenLO16}.
Because Lemma~\ref{lem:alo16} returns a solution $X$ with $\expect{}{C \bullet X} \ge \frac{4}{5} \mathrm{OPT}$, it must return a $\frac{3}{5}$ approximation with probability at least $1/2$.
Since $\frac{3}{5} > \frac{1+\beta}{2}$, we can invoke Lemma~\ref{lem:alo16} $O(\log n)$ times so that we get $\frac{1+\beta}{2}$-approximation with high probability. We assume this event happens for the rest of the proof.

Let $c_i^+ = v_i^\top C_+ v_i$ and $c_i^- = v_i^\top C_- v_i$.
If we can approximate $c^+$ and $c^-$ by a (multiplicative) factor of $1 \pm \frac{\eps}{2}$, we have
\begin{align*}
(c^+ - c^-)^\top x
& \ge \frac{1+\beta}{2} \mathrm{OPT} - \frac{\eps}{2} (c^+ + c^-)^\top x \tag{Lemma~\ref{lem:alo16}} \\
& \ge \frac{1+\beta}{2}  \mathrm{OPT} - \frac{\eps^2 \rho}{2} \tr(C^+ + C^-) \tag{$X \preceq \eps \rho I$} \\
& \ge \frac{1+\beta}{2} \eps \rho \left(\frac{1-\beta}{1+\beta}\tr(C_+) - \tr(C_-)\right) - \frac{\eps^2 \rho}{2} \tr(C^+ + C^-) \tag{Lemma~\ref{lem:sdp-sol-exist}}. \\
& = \frac{\eps \rho}{2} \left( (1-\beta-\eps) \tr(C_+) - (1+\beta+\eps) \tr(C_-) \right).
\end{align*}

Finally, 
%Before we can invoke Lemma~\ref{lem:alo16}, we need to approximate $c^+$ and $c^-$ in nearly-linear time.
%To make our proof more self-contained,
we will show how to approximate $c^-_i$ by a factor of $1 \pm \frac{\eps}{2}$ for all $1 \le i \le m$ in time $\tilde O(m / \eps^{O(1)})$.
The algorithms for approximating $c^+_i$ and implementing the oracle required by Lemma~\ref{lem:alo16} follow from a similar approach.\footnote{
We remark that the problem of approximating these quantities is akin to that of approximating (relative) effective resistances~\citep{SpielmanS11, AllenLO15, LeeS15}, and a nearly-linear time algorithm for computing the same quantities was shown in~\citep{LeeS17}.}

Recall that $C_- = \exp((uI - A)^{-1}) (uI-A)^{-2}$, where $A = L^{-1/2} \hat L L^{-1/2}$.
By Lemma~\ref{lem:mat-taylor} the assumption that $g = \Omega(\log^{-2} n)$, there exists a degree $\tilde O\left(\frac{\log(1/\eps)}{g^2}\right) = \tilde O(\log^4 n \log(1/\eps))$ polynomial $q(A)$ such that
\[
q(A) \approx_{\eps/6} \exp\left(\frac{1}{2}(uI - A)^{-1}\right) (uI-A)^{-1}.
\]
Because both sides are matrix polynomials of $A$, we can diagonalize them simultaneously so that the approximation only happens to the eigenvalues. Therefore,
\[
\left(q(A)\right)^2 \approx_{\eps/3} \exp((uI - A)^{-1}) (uI-A)^{-2},
\]
which implies $v_i^\top q(A)^2 v_i \in [1\pm \frac{\eps}{2}] v_i^\top C_- v_i$, since $\exp(\eps/3) \le 1+\eps/2$ when $\eps \le \frac{1}{10}$.

Recall that $m$, $m'$ denotes the number of edges in $\hat L$ and $L$.
Recall that $L = B^\top B$ where $B \in \R^{m' \times n}$ is the edge-vertex incident matrix of $L$.
Fix some $1 \le i \le m$.
Let $v_i = L^{-1/2} b_i$ where $b_i = \chi_u - \chi_{u'}$ for the $i$-th edge $(u, u')$.

For any $1 \le i \le m$, we have
\begin{align*}
v_i^\top C_- v_i
& \approx_{\eps/3} \normtwo{q(A) v_i}^2 \\
& = \normtwo{q(L^{-1/2} \hat L L^{-1/2}) L^{-1/2} b_i}^2 \\
& = \normtwo{L^{1/2} q(L^{-1} \hat L) L^{-1} b_i}^2 \\
& = \normtwo{B q(L^{-1} \hat L) L^{-1} b_i}^2 \\
& = \normtwo{B q(L^{-1} \hat L) L^{-1} (\chi_u - \chi_{u'})}^2.
\end{align*}
So the quantities $\{c^-_i\}_{i=1}^m$ are just the squared distances between the $m$-dimensional points $\{B q(L^{-1} \hat L) L^{-1} \chi_u\}_{u\in V}$.
We invoke the Johnson-Lindenstrauss lemma and generate a random $O(\log n/\eps^2) \times m$ matrix $Q$, so that with high probability, for all $1 \le u, u' \le n$,
\[
\normtwo{Bq(L^{-1} \hat L)L^{-1}(\chi_u - \chi_{u'})} \approx_{\eps/6} \normtwo{QBq(L^{-1} \hat L)L^{-1}(\chi_u - \chi_{u'})}.
\]
Recall that $A_i$ is the $i$-th row of a matrix $A$.
Let $Z = QBq(L^{-1} \hat L)L^{-1}$ and $Y = QBq(L^{-1} \hat L)$.
Both $Y$ and $Z$ have $O(\log n/\eps^2)$ rows and $n$ columns.
We have $Z^\top = L^{-1} Y^\top$, which allows us to approximate each $(Z_i)^\top = L^{-1} (Y_i)^\top$ by solving a linear system in $L$.
The time it takes to solve $O(\log n/\eps^2)$ linear systems in $L$ is $\tilde O(n/\eps^4)$, because we can assume $m, m' = O(n/\eps^2)$ by sparsifying the input graphs.

We can compute $Y^\top = q(L^{-1} \hat L) B^\top Q^\top$ in $\tilde O(n/\eps^4)$ time, since we can perform matrix-vector multiplication with $q(L^{-1} \hat L)$ in time $\tilde O(m/\eps^2)$, and $B^\top Q^\top$ can be computed in $\tilde O(n/\eps^4)$ time because $B$ has $2m' = O(n/\eps^2)$ non-zeros and $Q$ has $O(\log n/\eps^2)$ rows. \jmlrQED %\qedhere
\end{enumerate}
\renewcommand{\jmlrQED}{}
\end{proof}

The overall running time of Algorithm~\ref{alg:ls17} is $\tilde O(m / \eps^7)$, because there are at most $O(\log n / \eps^2)$ iterations (shown in Section~\ref{sec:bss}), and each iteration can be implemented to run in time $\tilde O(m / \eps^{5})$ by Lemma~\ref{lem:sdp-computation}.

One of our main contributions is conceptual: we show that the framework of~\cite{BatsonSS12} can be applied to a much broader settings to obtain scalable algorithms.
On a technical level, because there exists a hidden set $S$ whose sum is only \emph{approximately} equal to $I$, the optimal solution to the SDP will be worse, so we need to carefully control the error caused by this, and move the barriers at a slightly different rate.
Our analysis is considerably simpler than that in~\citep{LeeS17}, partly because we do not require the output weights to be sparse;
We also take care of two minor issues with~\citep{LeeS17}: They assumed $\rho$ can be computed exactly for simplicity, and they proved Taylor expansion of $C_-$ can be truncated (where it should be $C_-^{1/2}$ as in Lemma~\ref{lem:mat-taylor}).

\begin{lemma}[\cite{AllenLO16}]
\label{lem:alo16}
Consider the following SDP with $M_i \succeq 0$ and $c \in \R^m$.
\begin{lp*}
\maxi{c^\top x}
\st \con{x \ge 0}
    \con{\sum_{i=1}^m x_i M_i \preceq I.}
\end{lp*}
Suppose $c$ is given explicitly and we have access to $M_i$ via an oracle $\OO_{\eta, \delta}$ which on input $x \in \R^m$ outputs a vector $y \in \R^m$ such that
\[
y_i \in (1 \pm \frac{\delta}{2}) \; M_i \bullet \exp \left(\eta \Bigl(\sum_{i} x_i M_i - I \Bigr) \right)
\]
% for $\eta = (4/\delta) \log(nm/\delta)$
in time $T_{\eta, \delta}$ for any $x \in \R^m$ such that $x \ge 0$ and $\sum_{i=1}^m x_i M_i \preceq 2I$.
% \todo{$n$ needs to appear in this theorem.}
Then, we can output an $x$ in time $\tilde O(T_{\eta, \delta} \log m / \delta^3)$ such that
\[
\expect{}{c^\top x} \ge (1-O(\delta)) \cdot \mathrm{OPT} \text{ with } \sum_{i=1}^m x_i M_i \preceq I.
\]
\end{lemma}



\begin{lemma}
\label{lem:mat-taylor}
Let $A$ be a real symmetric matrix. When $(u-1) I \prec A \prec (u-g) I$ for some $0 < g < 1$, we can compute
\begin{enumerate}
\item[(1)] A polynomial $p(A)$ of degree $O\left(\frac{\log(1/(\eps g))}{g}\right)$ such that $p(A) \approx_{\eps} (uI-A)^{-2}$.
\item[(2)] A polynomial $q(A)$ of degree $O\left(\frac{\log(1/(\eps g))}{g^2}\right)$ such that $q(A) \approx_{\eps} \exp\left(\frac{1}{2}(uI - A)^{-1}\right) (uI-A)^{-1}$.
\end{enumerate}
\end{lemma}
\begin{proof}
The lemma is proved by truncating Taylor expansions.
Because $A$ is symmetric, the matrix polynomials $p(A)$ and $q(A)$ can be diagonalized simultaneously with $A$.
Therefore, it is sufficient to prove such polynomials exist for scalars.

\begin{enumerate}
\item[(1)] Let $f(x) = x^{-2}$. Let $p(a) = \hat p(u-a)$, and we define $\hat p(\cdot)$ to be the first $d$ terms of the Taylor expansion of $f(x)$ at $x=1$.
\[
f(x) = \sum_{i=0}^d \frac{f^{(i)}(1)}{i!}(x-1)^n + \frac{1}{d!} \int_1^{x} f^{(d+1)}(t) (x-t)^d dt.
\]
We know that all eigenvalues of $(uI-A)$ are in the interval $(g, 1)$.
For any $x \in (g, 1)$, there exists some $d = \left(g^{-1} \log(1/(\eps g)) \right)$ such that the remainder of the Taylor series satisfies
\begin{align*}
\left| f(x) - \sum_{i=0}^d \frac{f^{(i)}(1)}{i!}(x-1)^n \right|
  & = \left| \frac{1}{d!} \int_1^{x} f^{(d+1)}(t) (x-t)^d dt \right| \\
  & = \frac{(1-x)^{d+1} (1+x+dx)}{x^2} \\
  & \le (1-g)^{d+1}\frac{d+2}{g^2} \le \eps.
\end{align*}
\item[(2)] Let $h(x) = \exp\left(\frac{1}{2}x^{-1}\right) x^{-1}$.
Let $q(a) = \hat q(u-a)$, and we define $\hat q(\cdot)$ to be the first $d$ terms of the Taylor expansion of $h(x)$ at $x=1$.

For any $x \in (g, 1)$ and $t \in [x, 1]$, $h$ is holomorphic on a neighborhood of the ball $B := \{z \in \C : |z - t| \le r \}$ for $r = t - x/2$, so we can bound the coefficients of the Taylor expansion using Cauchy's estimates.
\[
\frac{1}{(d+1)!} h^{(d+1)}(t) \le r^{-d-1} \sup_{z \in B} |h(z)| \le r^{-d-1} \cdot 2 \exp(x^{-1}) x^{-1}.
\]
There exists some $d = O\left(g^{-2} \log(1/(\eps g))\right)$ such that the remainder at $x \in (g, 1)$ satisfies
\begin{align*}
\left| h(x) - \hat q(x) \right|
  & = \left| \frac{1}{d!} \int_1^{x} h^{(d+1)}(t) (x-t)^d dt \right| \\
  & \le 2(d+1) \exp(x^{-1}) x^{-1} \int_x^1 \frac{(t-x)^d}{(t-x/2)^{d+1}} dt \\
  & \le 4(d+1) \exp(x^{-1}) x^{-2} \int_x^1 \bigl(1-\frac{x}{2}\bigr)^d dt \\
  & \le 4(d+1) \exp(g^{-1}) g^{-2} \bigl(1-\frac{g}{2}\bigr)^d \le \eps. \tag*{\jmlrQED} %\qedhere
\end{align*}
\end{enumerate}
\renewcommand{\jmlrQED}{}
\end{proof}



\section{Omitted Proof from Section~\ref{sec:lap-to-adj}}
\label{apx:lap-to-adj}

In this section, we restate Lemmas~\ref{lem:random-graph}~and~\ref{lem:Lclose-Aclose} and prove them.

\noindent {\bf Lemma~\ref{lem:random-graph}}
{\enspace \em
Let $G$ denote the $n_1 \times n_2$ complete bipartite graph.
We write $n = n_1 + n_2$ for the number of vertices, and $m = n_1 n_2$ for the number of edges.
Let $H$ denote a random graph generated by including each edge of $G$ independently with probability $p$.
with high probability, we can re-weight edges of $H$ so that the (weighted) Laplacian matrix $L_H$ is $\eps$-spectrally similar with $L_G$, where $\eps = O\left(\sqrt{\frac{n \log n}{p m}}\right)$.
}
\begin{proof}
For complete bipartite graphs, all edges have the same effective resistance, so uniform sampling among all the edges will produce a good spectral sparsifier.

Formally, we can use the main result of~\cite{SpielmanS11}: Fix any $0 < \eps < 1$. For sufficiently large $n$ and all graphs on $n$ vertices, there is a universal constant $C$ so that sampling $C n \log n/\eps^2$ edges independently (with sample probability $p_e$ proportional to [edge weight $\times$ effective resistant]) produces an $\eps$-spectral sparsifier with high probability.
The lemma allows reweighting on $H$ because when we include an edge we give it weight $1/p_e$.

At the core of~\citep{SpielmanS11} are matrix concentration inequalities~\citep{RudelsonV07, AhlswedeW02, Tropp12}.
Note that the original proof in~\citep{SpielmanS11} used sampling with replacement and holds only with constant probability, but the analysis can be adapted to show that sampling by effective resistance without replacement works with high probability.
\end{proof}

\noindent {\bf Lemma~\ref{lem:Lclose-Aclose}}
{\enspace \em
Let $L = D - A$ and $\tilde L = \tilde{D} - \tilde{A}$ be two graph Laplacians, where $D$ is the degree matrix and $A$ is the adjacency matrix of the graph.
If $(1-\eps) L \preceq \tilde L \preceq L$, then we have:
\begin{enumerate}
\item[(1)] $(1-\eps) D_{i,i} \le \tilde{D}_{i,i} \le D_{i,i}$.
\item[(2)] $\norm{D^{-1/2}(\tilde{A} - A)D^{-1/2}} \le 3\eps$.
\end{enumerate}
}
\begin{proof}
For (1), the spectral similarity between $L$ and $\tilde L$ implies that $(1-\eps) x^\top L x \le x^\top \tilde{L} x \le x^\top L x$ for all $x \in \R^n$.
In particular, this holds for all standard basis vectors, so $(1-\eps) D_{i,i} \le \tilde{D}_{i,i} \le D_{i,i}$.

For (2), we know that $0 \preceq L - \tilde{L} \preceq \eps L$ and similarly $0 \preceq D-\tilde{D} \preceq \eps D$, and therefore
\begin{align*}
\norm{D^{-1/2}(\tilde{A} - A)D^{-1/2}}
& = \norm{D^{-1/2}(\tilde{D} - D + L - \tilde{L})D^{-1/2}} \\
& \le \norm{D^{-1/2} (D - \tilde D) D^{-1/2}} + \norm{D^{-1/2} (L - \tilde L) D^{-1/2}} \\
& \le \eps \norm{I} + \eps \norm{D^{-1/2} L D^{-1/2}} \le 3\eps. %\tag*{\jmlrQED} %\qedhere
\end{align*}
The last step uses the fact that eigenvalues of a normalized Laplacian matrix $D^{-1/2} L D^{-1/2}$ are always between $0$ and $2$.
%\renewcommand{\jmlrQED}{}
\end{proof}

% !TEX root = main.tex

\section{Using Deterministic Conditions for Matrix Completion}
\label{app:matrix}
In this section, we prove Theorem~\ref{thm:asymmetric_local}.

We use Lemma~\ref{lem:deterministc_main} and the techniques in \citep{GeJZ17} to show that all local minima of the non-convex objective functions are close to the ground truth.
%As in the original proof, 
We first restate the objective functions: Equation~\eqref{eqn:symmetricobj} for the symmetric case and Equation~\eqref{eqn:asymmetricobj} for the asymmetric case.
\begin{align*}
\min \; f(U) &= \frac{1}{2}\|UU^\top - \Ms\|_W^2 + Q(U),  \tag{\ref{eqn:symmetricobj}} \\
\min \; f(U, V) &= 2\|UV^\top - \Ms\|_W^2 + \frac{1}{2} \|U^\top U-V^\top V\|_F^2 + Q(U,V), \tag{\ref{eqn:asymmetricobj}}
\end{align*}
where $x_+ = \max\{x,0\}$, $Q(U) = \lambda \sum_{i=1}^n (\normtwo{U_i} - \alpha)_+^4$, and $Q(U,V) = \lambda_1 \sum_{i=1}^{n_1} (\normtwo{U_i} - \alpha_1)_+^4 +\lambda_2\sum_{i=1}^{n_2} (\normtwo{V_i} - \alpha_2)_+^4$.

We start with an overview of the analysis in \citep{GeJZ17} in Appendix~\ref{app:matrix-overview}.
Because Lemma~\ref{lem:tangent} is no longer true in the semi-random setting, we cannot use the proof of \cite{GeJZ17} in a black-box way.
We will handle symmetric (Appendix~\ref{app:matrix-symmetric}) and asymmetric (Appendix~\ref{app:matrix-asymmetric}) cases separately.

\subsection{Overview of the Analysis in \citep{GeJZ17}}
\label{app:matrix-overview}

We give a brief overview of the techniques in \citep{GeJZ17}. The materials in this section are independent of the concentration bounds, so they remain valid in the semi-random model.

\paragraph{Measuring Distance between Matrices.} The first problem in analyzing Objective~\eqref{eqn:symmetricobj} is that the optimal solution is not unique: given a matrix $\Ms = \Us(\Us)^\top$, for any orthonormal matrix $R$ we also have $\Ms = (\Us R)(\Us R)^\top$. To take this symmetry into account, we define the distance between two matrices as follows:

\begin{definition}\label{def:difference}
Given matrices $U, \Us \in \R^{n\times r}$, their difference is defined to be $\Delta = U - \Us R$, where $R\in \R^{r\times r}$ is an $r\times r$ orthonormal matrix that minimizes $\|U - \Us R\|_F^2$.
\end{definition}

The benefit of this definition of distance is summarized in the following lemma:

\begin{lemma}[Lemma 6 in \citep{GeJZ17}]
\label{lem:normconnect}
Given matrices $U, \Us \in \R^{n\times r}$, let $M = UU^\top$ and $\Ms = \Us(\Us)^\top$, let $\Delta$ be the difference defined in Definition~\ref{def:difference}, then
\[
\|\Delta\Delta^\top\|_F^2 \le 2\|M - \Ms\|_F^2,
\]
and
\[
\sigs_r\|\Delta\|_F^2 \le \frac{1}{2(\sqrt{2}-1)} \|M - \Ms\|_F^2.
\]
\end{lemma}

The lemma states that when $\Delta$ is large, $M$ is also far from $\Ms$. This would not be true if we simply defined $\Delta = U - \Us$ without considering the best rotation of $\Us$.
From now on, we will always assume $\Us$ is {\em aligned} with $U$ in the sense that $R = I$ and $\Delta = U - \Us$ (this can be guaranteed by choosing the appropriate global optimum that $U$ is comparing to). 

\paragraph{Main Proof for the Symmetric Case.}
First, we introduce notations for the Hessian. The Hessian of $f(U)$ is a 4-th order tensor (because the variable $U$ is a matrix). For an $n\times r$ matrix $X$, we use $[\nabla^2 f(U)](X)$ to denote the quadratic form of the Hessian evaluated at $X$. The Hessian is positive semidefinite (PSD), iff $[\nabla^2 f(U)](X) \ge 0$ for every $X$.

The main idea of \cite{GeJZ17} is to focus on the direction of $\Delta$:
To prove $UU^\top = \Ms$, instead of using $\nabla f(U) = 0$ and $\nabla^2 f(U)$ is PSD, it is sufficient to work with $\inner{\nabla f(U), \Delta} = 0$ and $[\nabla^2 f(U)](\Delta) \ge 0$.
%Intuitively, this direction brings us closer to the global optimum $\Us$.
The next lemma, which is the main lemma in \citep{GeJZ17}, derives a particular inequality that is very useful in proving convergence.
Lemma~\ref{lem:gjzmain_sym} is proved by simplifying the second-order term $[\nabla^2 f(U)](\Delta)$ given that the first-order term $\inner{\nabla f(U), \Delta}$ is 0.

\begin{lemma}[Lemma 7 in \citep{GeJZ17}]
\label{lem:gjzmain_sym}
Let $M = UU^\top$ and $\Delta$ is the difference of $U$ and $\Us$ as in Definition~\ref{def:difference}, if $U$ is a local minimum of Objective \eqref{eqn:symmetricobj}, then
\[
0 \le [\nabla^2 f(U)](\Delta) = \|\Delta\Delta^\top\|_W^2 - 3\|M-\Ms\|_W^2 + ([\nabla^2 Q(U)](\Delta) -4\inner{\nabla Q(U),\Delta}).
\]
\end{lemma}

To see why this inequality is useful intuitively, assume the regularizer term is 0 (the current vector is incoherent so the incoherence regularizer is not active), and assume further the $W$-norms are very close to Frobenius norm (which is essentially guaranteed by Lemmas~\ref{lem:tangent}~and~\ref{lem:Delta_mc} when entries are observed randomly), then we have
\[
\|\Delta\Delta^\top\|_F^2 - 3\|M-\Ms\|_F^2 \ge 0.
\]
However, by Lemma~\ref{lem:normconnect} we know $\|\Delta\Delta^\top\|_F^2 \le 2\|M-\Ms\|_F^2$, so the only way this equation can hold is if $\|M-\Ms\|_F = 0$, and therefore, all local optima are global.

Finally, we state the lemma that shows the regularizer term is indeed small.\footnote{
The constant in Lemma~\ref{lem:extra_bound_symmetric} is slightly different from that of \citep{GeJZ17}, but it follows from the same proof by choosing a larger universal constant $C$.}

\begin{lemma}[Lemma 11 in \citep{GeJZ17}]
\label{lem:extra_bound_symmetric}
Let $U$ and $\Delta$ be defined as above.
Choose $\alpha^2 = \frac{C\mu r\sigs_1}{n}$ and $\lambda = \frac{C^2 n}{\mu r\kappa^\star}$ where $C$ is a large enough universal constant, then we have
\[
([\nabla^2 Q(U)](\Delta) -4\inner{\nabla Q(U),\Delta}) \le 0.1\sigs_r \|\Delta\|_F^2.
\]
\end{lemma}

\paragraph{Reduction from Asymmetric Case to the Symmetric Case.}

To handle asymmetric matrices, \citep{GeJZ17} gives a way to essentially reduce asymmetric matrices to symmetric matrices. 

For variables $U,V$ and optimal solution $\Us,\Vs$, we define the following matrices:
\[
Z = 
\begin{pmatrix}
 U\\
 V
\end{pmatrix}; \;
Z^\star = 
\begin{pmatrix}
 \Us\\
 \Vs
\end{pmatrix}; \quad
N = ZZ^\top; \; N^\star = (Z^\star)(Z^\star)^\top.
\]

In the asymmetric setting, we consider $\Delta=\begin{pmatrix}
 \Delta_U\\
 \Delta_V
\end{pmatrix}$ as the difference between $Z$ and $Z^\star$ as in Definition~\ref{def:difference}, and we also rotate $Z^\star$ so that $\Delta = Z-Z^\star$.

Roughly speaking, we want to design an objective function that reduces the asymmetric case to a symmetric matrix completion problem with variables $Z$ and ground truth $N^\star$.
This is impossible if we only focus on the term $2 \|UV^\top - \Ms\|_W^2$, because it does not depend on the diagonal blocks of $(N - N^\star)$.
Since we cannot observe the diagonal blocks of $N^\star$, we try to add a term so that the Hessian of $f(Z)$ acts like a block identity tensor on $N$.
The additional term $\frac{1}{2} \|U^\top U-V^\top V\|_F^2$ is introduced for exactly this purpose.
%\begin{lemma}\label{lem:symmetricconvert}
%We have
%\[2\|UV^\top - \Ms\|_W^2 + \frac{1}{2} \|U^\top U-V^\top V\|_F^2 = \frac{1}{2}\|N-N^\star\|_{\bar{W}}^2,\]
%In particular, $\|\bar{W} - J\| = 2\|W-J\|$.
%\end{lemma}
%
%The lemma follows from direct calculation. This lemma justifies why we need the additional term $\frac{1}{2} \|U^\top U-V^\top V\|_F^2$: it fills in the diagonal blocks for the matrix $N$, $N^\star$. Using this lemma, we can rewrite the objective function as
%\[
%\min f(Z) = \frac{1}{2}\|ZZ^\top - N^\star\|_{\bar{W}}^2 + Q(Z). 
%\]
%Here $Q(Z) = Q(U,V) =: Q_1(U)+Q_2(V)$ is the same regularizer. The formula is then essentially the same as Equation \eqref{eqn:symmetricobj}, and indeed \cite{GeJZ17} proved a very similar main lemma:

Let $Q(Z) = Q(U,V)$ be the same regularizer as in Objective~\eqref{eqn:asymmetricobj}. \cite{GeJZ17} proved the following lemma:

\begin{lemma}[Essentially Lemma 16 in \citep{GeJZ17}]
\label{lem:gjzmain_asym}
Let $Z$, $Z^\star$, $N$, $N^\star$, and $\Delta$ be defined as above, if $Z$ is a local minimum of Objective~\eqref{eqn:asymmetricobj}, then
\[
0 \le [\nabla^2 f(Z)](\Delta) \le \|\Delta\Delta^\top\|_{\bar{W}}^2 - 3\|N-N^\star\|_{\bar{W}}^2 + ([\nabla^2 Q(Z)](\Delta) -4\inner{\nabla Q(Z),\Delta}).
\]
where
\[
\bar{W} = \begin{pmatrix}
J & 2W - J \\
2W^\top - J & J
\end{pmatrix}.
\]
\end{lemma}

Similar to the symmetric case, Lemma~\ref{lem:gjzmain_asym} is proved by simplifying the second-order term $[\nabla^2 f(Z)](\Delta)$ given that the first-order term $\inner{\nabla f(Z), \Delta}$ is 0.

Notice that we have $\|\bar{W} - J\| = 2\|W-J\|$.
If our preprocessing algorithm guarantees that $W$ is close to $J$, then $\bar W$ is close to $J$ as well.

Finally, we have a corresponding lemma that shows the regularization term is small.

\begin{lemma}[Lemma 22 in \citep{GeJZ17}]\label{lem:extra_bound_asymmetric}
Let $Z$ and $\Delta$ be defined as above.
Choose $\alpha_1^2 = \frac{C\mu r\sigs_1}{n_1},\alpha_2^2 = \frac{C\mu r\sigs_1}{n_2}$ and $\lambda_1 = \frac{C^2 n_1}{\mu r\kappa^\star},\lambda_2 = \frac{C^2 n_1}{\mu r\kappa^\star}$ where $C$ is a large enough universal constant, then we have
\[
([\nabla^2 Q(Z)](\Delta) -4\inner{\nabla Q(Z),\Delta}) \le 0.1\sigs_r \|\Delta\|_F^2.
\]
\end{lemma}



\subsection{Proof of Our Symmetric Case}
\label{app:matrix-symmetric}
We first prove a variant of Lemma 9 in \citep{GeJZ17} in the semi-random model. Lemma~\ref{lem:symmetricnormbound} shows that any local minima of Objective~\eqref{eqn:symmetricobj} have bounded row norms.

\newcommand{\grad}{\nabla}
\newcommand{\poly}{\mbox{poly}}

\begin{lemma} \label{lem:symmetricnormbound}
When the weight matrix $W$ satisfies $\norminf{W} \le n$, choose $\alpha^2 = \frac{C\mu r\sigs_1}{n}$ and $\lambda = \frac{C^2 n}{\mu r\kappa^\star}$ where $C$ is a large enough universal constant. For Objective~\eqref{eqn:symmetricobj}, we have for any matrix $U$ with $\grad f (U) = 0$, 
\begin{equation*}
\max_{i}\normtwo{U_i}^2 = O\left(\frac{\mu^2 r^2 \kappa^\star \sigs_1}{n}\right).
\end{equation*}
\end{lemma}
% where $(\frac{epsilon}{\lambda})^{\frac{2}{3}} \le \frac{\sqrt{\mu r} \cdot \sigs_1 }{\lambda} $

\begin{proof}
Recall that $U_i \in \R^{1 \times r}$ is the $i$-th row of $U \in \R^{n \times r}$ and $e_i \in \R^{r \times 1}$ is the $i$-th standard basis vector.

%By Lemma 18 in \citep{GeJZ17}, 
The gradient $\nabla f(U)$ is equal to $2(W*(M-\Ms))U + \nabla Q(U)$, where
\[ \nabla Q(U) = 4\lambda \sum_{i=1}^n (\normtwo{U_i} - \alpha)_+^3 \frac{e_i U_i}{\normtwo{U_i}^2}. \]
%\[ \nabla Q(U) = 4\lambda \sum_{i=1}^n (\|U_i\| - \alpha)_+^3 \frac{e_i (U_i)^\top}{\|U_i\|^2}. \]

Let $i^\star$ be the row index with the maximum $\ell_2$-norm, if $\normtwo{U_{i^\star}} \le 2\alpha$ then we are done. On the other hand, if $\normtwo{U_{i^\star}} > 2\alpha$, we will consider the gradient along $e_{i^\star}U_{i^\star}$. % $e_{i^\star}(U_{i^\star})^\top$
We have
\begin{align*}
0 & = \inner{\nabla f(U), e_{i^\star}U_{i^\star}} \\
& = \inner{e_{i^\star}^\top[2(W*(UU^\top - \Ms))U + \nabla Q(U)], U_{i^\star}} \\
& \ge 4\lambda (\normtwo{U_{i^\star}}-\alpha)_+^3\normtwo{U_{i^\star}} - 2\inner{e_{i^\star}^\top \Ms, e_{i^\star}^\top UU^\top}_W \\
& \ge \frac{\lambda}{2}\normtwo{U_{i^\star}}^4 - 2n \maxnorm{\Ms} \maxnorm{UU^\top} \\
& \ge \frac{\lambda}{2}\normtwo{U_{i^\star}}^4 - 2\mu r\sigs_1\normtwo{U_{i^\star}}^2.
\end{align*}

The third step removes the term $\inner{e_{i^\star}^\top U U^\top U, U_{i^\star}} = \fnorm{e_{i^\star}^\top U U^\top}^2 \ge 0$.
The fourth step uses that $\normtwo{U_{i^\star}} > 2\alpha$ and $\norminf{W} \le n$ (every row of $W$ has $\ell_1$-norm at most $n$).
The last step is due to $\maxnorm{UU^\top} = \normtwo{U_{i^\star}}^2$; and $\maxnorm{\Ms} = \max_{i,j} \inner{U_i, V_j} \le \max_{i,j} \normtwo{U_i} \normtwo{V_j} \le \frac{\sigs_1 \mu r}{n}$ because $\Ms$ is incoherent.
As a result, we know that $\normtwo{U_{i^\star}}^2 \le \frac{4\mu r\sigs_1}{\lambda} = O\left(\frac{\mu^2 r^2 \kappa^\star \sigs_1}{n}\right)$ by our choice of $\lambda$.
\end{proof}

Next, we will show that all local minima are close to the ground truth.

\begin{lemma}
Fix any error parameter $0 < \eps < 1$.
For a weight matrix $W$ such that $\norminf{W} \le n$ and $\|W-J\| \le \frac{\eps cn}{\mu^2 r^2 (\kappa^\star)^2}$ for a small enough universal constant $c$, any local minimum $U$ of Objective~\eqref{eqn:symmetricobj} satisfies $\|UU^\top-\Ms\|_F^2 \le \epsilon \|\Ms\|_F^2$. 
\end{lemma}

\begin{proof}
By Lemma~\ref{lem:gjzmain_sym}, we know that every local minimum of $f(U)$ satisfies
\[
\|\Delta\Delta^\top\|_W^2 - 3\|UU^\top - \Ms\|_W^2 + \left([\nabla^2 Q(U)](\Delta) - 4\inner{\nabla Q(U), \Delta}\right) \ge 0.
\]

We will bound these three terms. First, by Lemma~\ref{lem:symmetricnormbound}, we know $\normtwo{\Delta_i}^2 \le O\left(\frac{\mu^2 r^2 \kappa^\star \sigs_1}{n}\right)$. 

For the first term $\|\Delta\Delta^\top\|_W^2$, we can directly apply Lemma~\ref{lem:deterministc_main}:
\begin{align*}
\|\Delta\Delta^\top\|_W^2 & \le \|\Delta\Delta^\top\|_F^2 + \|W-J\|\|\Delta\|_F^2 \max_i \|\Delta_i\|^2 \\
& \le \|\Delta\Delta^\top\|_F^2 + \|W-J\|\cdot O\left(\frac{\mu^2 r^2 \kappa^\star \sigs_1}{n}\right)\cdot \|\Delta\|_F^2 \\
& \le \|\Delta\Delta^\top\|_F^2 + 0.1\sigs_r \|\Delta\|_F^2.
\end{align*}

The last inequality uses the fact that $\|W-J\| \le \frac{cn}{\mu^2 r^2 (\kappa^\star)^2}$ for a small enough constant $c$. 

For the second term $\|UU^\top - \Ms\|_W^2$, we invoke Lemma~\ref{lem:deterministc_main} with $X = (U, \Us)$ and $Y = (U, -\Us)$.
Notice that $X Y^\top = UU^\top - \Ms$.
Moreover, we know that $\fnorm{X} \le \|U\|_F + \|\Us\|_F \le 2\|\Us\|_F + \|\Delta\|_F$. Similarly, the row norms of $X$ is still upper bounded by $O\left(\frac{\mu^2 r^2 \kappa^\star \sigs_1}{n}\right)$. Therefore,
\begin{align*}
\|UU^\top - \Ms\|_W^2 & \ge \|UU^\top - \Ms\|_F^2 - \|W-J\|\|(U,\Us)\|_F^2 \max_i \normtwo{(U,\Us)_i}^2 \\
& \ge \|UU^\top - \Ms\|_F^2 - \|W-J\|\cdot O\left(\frac{\mu^2 r^2 \kappa^\star \sigs_1}{n}\right)\cdot (2\|\Us\|_F + \|\Delta\|_F)^2 \\
& \ge \|UU^\top - \Ms\|_F^2 - 0.1 \epsilon\sigs_r \|\Us\|_F^2 -0.1\sigs_r \|\Delta\|_F^2.
\end{align*}

Again, the last step uses the fact that $\|W-J\| \le \frac{\eps cn}{\mu^2 r^2 (\kappa^\star)^2}$ for a small enough constant $c$. 

Finally, the third term is bounded by $0.1\sigs_r\|\Delta\|_F^2$ by Lemma~\ref{lem:extra_bound_symmetric}. Combining all these terms,
\begin{align*}
0 & \le \|\Delta\Delta^\top\|_W^2 - 3\|UU^\top - \Ms\|_W^2 + \left([\nabla^2 Q(U)](\Delta) - 4\inner{\nabla Q(U), \Delta}\right) \\
& \le \|\Delta\Delta^\top\|_F^2 + 0.2\sigs_r \|\Delta\|_F^2 - 3\left(\|UU^\top - \Ms\|_F^2 - 0.1 \epsilon\sigs_r \|\Us\|_F^2 -0.1\sigs_r \|\Delta\|_F^2\right)  \\
% Too wide for COLT template
%& \le \left(\|\Delta\Delta^\top\|_F^2 + 0.1\sigs_r \|\Delta\|_F^2\right) - 3\left(\|UU^\top - \Ms\|_F^2 - 0.1 \epsilon\sigs_r \|\Us\|_F^2 -0.1\sigs_r \|\Delta\|_F^2\right) + 0.1\sigs_r \|\Delta\|_F^2 \\
& \le -\|UU^\top - \Ms\|_F^2 + 0.5 \sigs_r\|\Delta\|_F^2 + 0.3\epsilon \sigs_r \|\Us\|_F^2 \\
& \le -0.3\|UU^\top - \Ms\|_F^2 + 0.3 \epsilon\sigs_r \|\Us\|_F^2.
\end{align*}

Here the calculations use the inequalities in Lemma~\ref{lem:normconnect}. 

As a result, $\|UU^\top - \Ms\|_F^2 \le \epsilon \sigs_r \|\Us\|_F^2$.
We conclude the proof by noting that $\sigs_r \|\Us\|_F^2 = \sigs_r \sum_{i=1}^r \sigs_i \le \sum_{i=1}^r (\sigs_i)^2 = \|\Ms\|_F^2$, because the singular values of $\Us$ are $\sqrt{\sigs_i}$'s.
\end{proof}

\subsection{Our Asymmetric Case: Proof of Theorem~\ref{thm:asymmetric_local}}
\label{app:matrix-asymmetric}

The proof for the asymmetric case (Objective~\eqref{eqn:asymmetricobj}) is very similar.
Recall that $\Ms = \Us(\Vs)^\top$, where $\Us \in \R^{n_1\times r}$ and $\Vs = \R^{n_2\times r}$.
We again start by bounding the row norms of $U$ and $V$.

\begin{lemma} \label{lem:asymmetricnormbound}
Suppose $\norminf{W} \le n_2$ and $\normone{W} \le n_1$. Choose $\alpha_1^2 = \frac{C\mu r\sigs_1}{n_1},\alpha_2^2 = \frac{C\mu r\sigs_1}{n_2}$ and $\lambda_1 = \frac{C^2 n_1}{\mu r\kappa^\star},\lambda_2 = \frac{C^2 n_1}{\mu r\kappa^\star}$ where $C$ is a large enough universal constant. For $f$ as in Objective~\eqref{eqn:asymmetricobj} and any matrix $Z = \begin{pmatrix}
U \\ V
\end{pmatrix}$ with $\grad f(Z) = 0$, we have
\begin{equation*}
\max_{i}\normtwo{U_i}^2 = O\left(\frac{\mu^3 r^3 (\kappa^\star)^2 \sigs_1}{n_1}\right); \quad 
\max_{i}\normtwo{V_i}^2 = O\left(\frac{\mu^3 r^3 (\kappa^\star)^2 \sigs_1}{n_2}\right).
\end{equation*}
\end{lemma}

\begin{proof}
Without loss of generality, we assume $\sqrt{n_1}\max_i\normtwo{U_i} \ge \sqrt{n_2}\max_i\normtwo{V_i}$, so it is enough to upper bound $\max_i \normtwo{U_i}$. The gradient can be computed as follows.
\begin{align*}
\grad f(Z) & =
4 \begin{pmatrix}
[W*(M -\Ms)] V \\
[W*(M -\Ms)]^\top U
\end{pmatrix}
+
2 \begin{pmatrix}
 U(U^\top U - V^\top V) \\
 V(V^\top V - U^\top U)
\end{pmatrix}
+
 \grad Q(Z),
\end{align*}
where %we have $\nabla Q(Z)$ as follows:
\begin{equation*}
\grad Q(Z) = 4\lambda_1 \sum_{i=1}^{n_1}(\normtwo{Z_i} - \alpha_1)^3_{+}\frac{e_i Z_i}{\normtwo{Z_i}^2} 
+ 4\lambda_2 \sum_{i=n_1+1}^{n_1+n_2}(\normtwo{Z_i} - \alpha_2)^3_{+}\frac{e_i Z_i}{\normtwo{Z_i}^2}.
\end{equation*}

First, we observe that $\inner{\nabla Q(Z), Z} \ge 0$, therefore,
\begin{align*}
0 & = \inner{\nabla f(Z), Z} \\
& = 2 \|U^\top U - V^\top V\|_F^2 + 8 \inner{M - \Ms, M}_W + \inner{\nabla Q(Z),Z} \\
& \ge 2 \|U^\top U - V^\top V\|_F^2 - 8 \inner{\Ms, M}_W \\
& \ge 2 \|U^\top U - V^\top V\|_F^2 - 8n_1n_2 \maxnorm{\Ms} \maxnorm{M}. \\
& \ge 2 \|U^\top U - V^\top V\|_F^2 - 8\sqrt{n_1n_2}\mu r\sigs_1 \maxnorm{M}.
\end{align*}

Let $i^\star = \arg\max_i \normtwo{U_i}$, if $\normtwo{U_{i^\star}} \le 2 \alpha_1$ then we are done. On the other hand, if $\normtwo{U_{i^\star}} > 2\alpha_1$, we know $\maxnorm{M} \le \max_{i,j} \normtwo{U_i}\normtwo{V_j} \le \sqrt{n_1/n_2} \normtwo{U_{i^\star}}^2$. 
As a result, we know 
\[
\|U^\top U - V^\top V\|_F^2 \le 4\sqrt{n_1n_2}\mu r\sigs_1\|M\|_\infty \le 4n_1\mu r \sigs_1\normtwo{U_{i^\star}}^2.
\]

Let $Q(Z) = Q(U,V) = Q_1(U)+Q_2(V)$.
Consider the gradient of $f(Z)$ along the direction $e_{i^\star} Z_{i^\star}$, we have
\begin{align*}
0 & = \inner{\nabla f(Z), e_{i^\star} Z_{i^\star}} = \inner{e_{i^\star}^\top\nabla f(Z), Z_{i^\star}} \\
& = \inner{e_{i^\star}^\top\left[4(W*(M-\Ms))V + 2 U(U^\top U-V^\top V) + \nabla Q_1(U)\right], U_{i^\star}} \\
& \ge 4\lambda_1(\normtwo{U_{i^\star}}-\alpha_1)_+^3\normtwo{U_{i^\star}} - 4\inner{e_{i^\star}^\top \Ms, e_{i^\star}^\top M}_W - 2 \|U^\top U-V^\top V\|_F \normtwo{U_{i^\star}}^2 \\
& \ge \frac{\lambda_1}{2}\normtwo{U_{i^\star}}^4 - 4n_2 \maxnorm{\Ms} \maxnorm{M} - 2 \|U^\top U-V^\top V\|_F \normtwo{U_{i^\star}}^2. \\
& \ge \frac{\lambda_1}{2}\normtwo{U_{i^\star}}^4 - 4\mu r\sigs_1\normtwo{U_{i^\star}}^2 - 4\sqrt{n_1\mu r \sigs_1}\normtwo{U_{i^\star}}^3. 
\end{align*}

This implies 
\[
\frac{\lambda_1}{2}\normtwo{U_{i^\star}}^2 \le 4\mu r\sigs_1 + 4\sqrt{n_1\mu r \sigs_1}\normtwo{U_{i^\star}}.
\]
Therefore, $\normtwo{U_{i^\star}}^2 = O\bigl(\max\{\frac{\mu r\sigs_1}{\lambda_1}, \frac{n_1\mu r \sigs_1}{\lambda_1^2}\}\bigr) = O\left(\frac{\mu^3 r^3 (\kappa^\star)^2 \sigs_1}{n_1}\right)$ by our choices of $\alpha$'s and $\lambda$'s. 
\end{proof}

Next, we will prove that all local minima are close to the ground truth.

\begin{lemma}
Fix any error parameter $0 < \eps < 1$.
Suppose the weight matrix $W$ satisfies that $\norminf{W} \le n_2$, $\normone{W} \le n_1$, and $\|W-J\| \le \frac{\eps c\sqrt{n_1n_2}}{\mu^3 r^3 (\kappa^\star)^3}$ for a small enough universal constant $c$. Then, any local minimum $(U, V)$ of Objective~\eqref{eqn:asymmetricobj} has $\|UV^\top-\Ms\|_F^2 \le \epsilon \|\Ms\|_F^2$. 
\end{lemma}

\begin{proof}
By Lemma~\ref{lem:gjzmain_asym} we know for every local minimum of $f(U,V)$ satisfies
\[
\|\Delta\Delta^\top\|_{\bar{W}}^2 - 3\|N - N^\star\|_{\bar{W}}^2 + \left([\nabla^2 Q(Z)](\Delta) - 4\inner{\nabla Q(Z), \Delta}\right) \ge 0.
\]

We will bound these three terms. First, by Lemma~\ref{lem:asymmetricnormbound} we know the rows of $\Delta_U$ have squared $\ell_2$-norm at most $ O\left(\frac{\mu^3 r^3 (\kappa^\star)^2 \sigs_1}{n_1}\right)$, and the rows of $\Delta_V$ have squared $\ell_2$-norm at most $ O\left(\frac{\mu^3 r^3 (\kappa^\star)^2 \sigs_1}{n_2}\right)$.

For the first term $\|\Delta\Delta^\top\|_{\bar{W}}^2$, by the definition of $\bar W$,
\begin{align*}
\norm{\Delta \Delta^\top}_{\bar W}^2 & = \norm{\Delta_U \Delta_U^\top}_F^2 + \norm{\Delta_V \Delta_V^\top}_F^2 - 2 \norm{\Delta_U \Delta_V^\top}_F^2 + 4 \norm{\Delta_U \Delta_V^\top}_W^2 \\
  & = \norm{\Delta \Delta^\top}_{F}^2 + 4(\norm{\Delta_U \Delta_V^\top}_W^2 - \norm{\Delta_U \Delta_V^\top}_F^2).
\end{align*}
%This is close to $\norm{\Delta \Delta^\top}_{F}^2$ if we can show $\norm{\Delta_U \Delta_V^\top}_W^2 \approx \|\Delta_U\Delta_V^\top\|_F^2$.
We can directly apply Lemma~\ref{lem:deterministc_main} to $\norm{\Delta_U \Delta_V^\top}_W^2$.
\begin{align*}
\|\Delta_U\Delta_V^\top\|_W^2 & \le \|\Delta_U\Delta_V^\top\|_F^2 + \|W-J\| \cdot \|\Delta_U\|_F \cdot \|\Delta_V\|_F \cdot \max_{i=1}^{n_1} \normtwo{\Delta_i} \cdot \max_{j=n_1+1}^{n_1+n_2} \normtwo{\Delta_j} \\
& \le \|\Delta_U\Delta_V^\top\|_F^2 + \|W-J\|\cdot O\left(\frac{\mu^3 r^3 (\kappa^\star)^2 \sigs_1}{\sqrt{n_1n_2}}\right)\cdot \|\Delta\|_F^2 \\
& \le \|\Delta_U\Delta_V^\top\|_F^2 + 0.01\sigs_r \|\Delta\|_F^2.
\end{align*}

Here the last inequality uses the fact that $\|W-J\| \le \frac{c\sqrt{n_1n_2}}{\mu^3 r^3 (\kappa^\star)^3}$ for a small enough constant $c$. 

For the second term, we can relate $\bar W$-norm to $W$-norm similarly, which allows us to focus on the $W$-norm of the off-diagonal blocks, $\|UV^\top - \Ms\|_W^2$.
We then invoke Lemma~\ref{lem:deterministc_main} with $X = (U, \Us)$ and $Y = (V, -\Us)$.
We know $X Y^\top = U V^\top - \Ms$ and $\|X\|_F \le \|U\|_F + \|\Us\|_F \le 2\|\Us\|_F + \|\Delta\|_F$ (the same upper bound holds for $\|Y\|_F$ because $\|\Us\|_F = \|\Vs\|_F$). The row norms of $X$ is still bounded by $O\left(\frac{\mu^2 r^2 \kappa^\star \sigs_1}{n_1}\right)$ (and similarly for $Y$ except the denominator is $n_2$).
%As a result,
\begin{align*}
& \|UV^\top - \Ms\|_W^2 \\
& \ge \|UV^\top - \Ms\|_F^2 - \|W-J\| \cdot \|(U,\Us)\|_F \|(V,-\Vs)\|_F \cdot \max_i \normtwo{(U,\Us)_i} \cdot \max_j \normtwo{(V,\Vs)_j} \\
& \ge \|UV^\top - \Ms\|_F^2 - \|W-J\| \cdot O\left(\frac{\mu^3 r^3 (\kappa^\star)^2 \sigs_1}{\sqrt{n_1n_2}}\right)\cdot (2\|\Us\|_F + \|\Delta\|_F)^2 \\
& \ge \|UV^\top - \Ms\|_F^2 - 0.05 \epsilon\sigs_r \|\Us\|_F^2 -0.01\sigs_r \|\Delta\|_F^2.
\end{align*}

Again, the last step uses the fact that $\|W-J\| \le \frac{\eps cn}{\mu^3 r^3 (\kappa^\star)^3}$ for a small enough constant $c$. 

Finally, the third term is bounded by $0.1\sigs_r\|\Delta\|_F^2$ by Lemma~\ref{lem:extra_bound_asymmetric}. We combine all these terms and apply Lemma~\ref{lem:normconnect},
\begin{align*}
0 &\le \|\Delta\Delta^\top\|_{\bar{W}}^2 - 3\|N- N^\star\|_{\bar{W}}^2 + \left([\nabla^2 Q(Z)](\Delta) - 4\inner{\nabla Q(Z), \Delta}\right) \\
& \le \|\Delta\Delta^\top\|_F^2 + 0.14\sigs_r \|\Delta\|_F^2 - 3\left(\|N- N^\star\|_F^2 - 0.2 \epsilon\sigs_r \|\Us\|_F^2 -0.01\sigs_r \|\Delta\|_F^2\right) \\
% Too wide for COLT template
%& \le \left(\|\Delta\Delta^\top\|_F^2 + 0.04\sigs_r \|\Delta\|_F^2\right) - 3\left(\|N- N^\star\|_F^2 - 0.2 \epsilon\sigs_r \|\Us\|_F^2 -0.01\sigs_r \|\Delta\|_F^2\right) + 0.1\sigs_r \|\Delta\|_F^2 \\
& \le -\|N- N^\star\|_F^2 + 0.17 \sigs_r\|\Delta\|_F^2 + 0.6 \epsilon \sigs_r \|\Us\|_F^2 \\
& \le -0.6\|N- N^\star\|_F^2 + 0.6 \epsilon\sigs_r \|\Us\|_F^2.
\end{align*}

As a result, $\|M-\Ms\|_F^2 \le \|N-N^\star\|_F^2 \le \epsilon \sigs_r \|\Us\|_F^2 \le \epsilon \|\Ms\|_F^2$. %This finishes the proof.
\end{proof}

\end{document}
