\subsection{General Lemmas}

% We begin by showing that under our assumptions, the measure of
% problematic arms cannot be too large.

% \begin{lemma} \label{lem:sdelta}
% $\AlmostTied{\delta} \leq \TieDensity \cdot \delta$ for all $\delta$.
% \end{lemma}

% \begin{emptyproof}
% Using the upper bound from Assumption~\ref{A1}, we can bound

% \begin{align*}
% \AlmostTied{\delta}
% & = \sum_{i,i'} \ProbC{\AgV \cdot (\ArmV{i} - \ArmV{i'}) \leq \delta}%
%     {\AgV \in \FirstTwo{i}{i'}}
%   \cdot \Prob{\AgV \in \FirstTwo{i}{i'}}\\
% & \leq \sum_{i,i'} \TieDensity \cdot \delta
%     \cdot \Prob{\AgV \in \FirstTwo{i}{i'}}
% \; = \; \TieDensity \cdot \delta. \QED
% \end{align*}
% \end{emptyproof}

We begin by bounding the length of any phase,
and more generally, the number of arm pulls by any given set
  of agents.
The bound of Lemma~\ref{lem:phase-length} will support bounding the
regret of early rounds (before tail bounds have kicked in).
The proof of Lemma~\ref{lem:phase-length} and all other proofs missing
in the main paper may be found in the appendix.
  
\begin{lemma} \label{lem:phase-length}
For any $s\geq 3$, the expected length of phase $s$ is at most
$\ARMNUM \cdot \log(s)$ time steps.

More generally, for any set of types $A$, the expected number of times that 
an agent with a type in $A$ appears in a phase $s$ is at most 
$f(A) \cdot \ARMNUM \cdot \log(s)$,
where $f(A) := \Prob[\AgV \sim \AgentDist]{\AgV \in A}$.
\end{lemma}

%For any $s\geq 3$, the expected length of phase $s$ is at most
%$\ARMNUM \cdot \log(s)$ time steps.
                  
We now state the key technical lemma which captures the intuition that
the estimates of the arms' attribute vectors become
more accurate with increasing phases $s$.

\begin{lemma} \label{lem:round-prob}
Recall the noise is a mean-zero sub-Gaussian($\sigma^2$) random variable.
Let \LatePhase be fixed, 
and let $x_n, x'_n > 0$ be sequences satisfying 
$\sqrt{0.6 n \cdot \log (\log_{1.1}(n) + 1) + \frac{n x_n^2}{16 \sigma^2}}
\leq \frac{n x'_n}{2 \sigma}$,
for all $n \geq \LatePhase$.
Let $\tau_s$ be a stopping time
(which may depend on the entire past history)
which is almost surely in phase $s$,
i.e., satisfying $\tau_s \in [t_s, t_{s+1})$ almost surely.
Then, for any arm $i$, attribute $j$, and phase $s \geq \LatePhase$,
we have 
$\Prob{\AccE{\tau_s}{i}{j}{x'_s}}
\geq 1 - 24 \exp\left(-\frac{1.8 s x_s^2}{16 \sigma^2} \right)$.
\end{lemma}

Next, we show the complementary result:
when the event \AccEU{t}{x} happens, no myopic agent incurs large regret.

\begin{lemma} \label{lem:right-choice}
Let $x > 0$ be arbitrary.
When \AccEU{t}{x} happens,
no agent \AgV will pull a highly suboptimal arm, i.e., an arm $i$ with 
$\AgV \cdot (\ArmV{\Best{\AgV}} - \ArmV{i}) > 2\Diam d x$.
\end{lemma}


