\documentclass[final,12pt]{colt2018}
% \usepackage[anon]{jmlr2e}


% Any additional packages needed should be included after jmlr2e.
% Note that jmlr2e.sty includes epsfig, amssymb, natbib and graphicx,
% and defines many common macros, such as 'proof' and 'example'.
%
% It also sets the bibliographystyle to plainnat; for more information on
% natbib citation styles, see the natbib documentation, a copy of which
% is archived at http://www.jmlr.org/format/natbib.pdf


\usepackage[]{algorithm}
\usepackage[noend]{algorithmic}

%\usepackage[utf8]{inputenc} % allow utf-8 input
%\usepackage[T1]{fontenc}    % use 8-bit T1 fonts
%\usepackage{microtype}      % microtypography
%\usepackage{xr} 

\usepackage{url}            % simple URL typesetting
\usepackage{amsfonts}       % blackboard math symbols
\usepackage{amsmath}
\usepackage{nicefrac}       % compact symbols for 1/2, etc.
%\usepackage{amsthm}
%\usepackage{mathtools}

%\usepackage{comment}
%\usepackage{subcaption}
%\usepackage{xcolor}
%\usepackage{float}
\usepackage{bm}
\usepackage{bbm}
\usepackage{ifthen}
\usepackage{xspace}

%\usepackage{color-edits}
% home-made package by David that provides the following macros:
% 1. \pfedit{}: prints argument in orange
% 2. \pfcomment{}: prints argument in orange in [Peter: #1]
% 3. \pfmargincomment{}: prints argument in orange in the margin as
% [Peter: #1]
% 4. \pfdelete{}: instead of text, marks a note in the margin in orange that says
% "Peter deleted here"
% Package has option [suppress], which gets rid of all color and all comments
% and [showdeletions], which shows deleted text in color/strikeout.
%\addauthor[Peter]{pf}{orange}
%\addauthor[Bangrui]{bc}{green}
%\addauthor[David]{dk}{red}

%\newcommand{\bccomment}[1]{{\color{blue}BC: #1}}
%\newcommand{\pfcomment}[1]{{\color{blue}PF: #1}}

\providecommand{\SET}[1]{\ensuremath{\{ #1 \}}\xspace}
\providecommand{\Abs}[1]{\ensuremath{| #1 |}\xspace}
\providecommand{\Set}[2]{\ensuremath{\SET{#1 \mid #2}}\xspace}

\providecommand{\PROB}{\ensuremath{{\rm Prob}}\xspace}
\providecommand{\Prob}[2][]{\ensuremath{%
\ifthenelse{\equal{#1}{}}{\PROB\left[#2\right]}{\PROB_{#1}\left[#2\right]}}\xspace}
\providecommand{\ProbC}[3][]{\Prob[#1]{#2\;|\;#3}}
\providecommand{\Expect}[2][]{\ensuremath{%
\ifthenelse{\equal{#1}{}}{\mathbb{E}}{\mathbb{E}_{#1}}%
\left[#2\right]}\xspace}
\providecommand{\ExpectC}[3][]{\Expect[#1]{#2\;|\;#3}}

\newcommand{\e}{\mathrm{e}}
\newcommand{\dd}{\mathrm{d}}
\newcommand{\quarter}{\ensuremath{\frac{1}{4}}\xspace}

\providecommand{\Kth}[1]{\ensuremath{{#1}^{\rm th}}}

\def\QED{{\phantom{x}} \hfill \ensuremath{\rule{1.3ex}{1.3ex}}}

\newcommand{\extraproof}[1]{\rm \trivlist \item[\hskip \labelsep{\bf Proof of #1. }]}
\def\endextraproof{\QED \endtrivlist}

\def\emptyproof{\rm \trivlist \item[\hskip \labelsep{\bf Proof. }]}
\def\endemptyproof{\endtrivlist}

\newcommand{\emptyextraproof}[1]{\rm \trivlist \item[\hskip \labelsep{\bf Proof of #1. }]}
\def\endemptyextraproof{\endtrivlist}

\newenvironment{rtheorem}[3][]{

\bigskip

\noindent \ifthenelse{\equal{#1}{}}{\bf #2 #3}{\bf #2 #3 (#1)}
\begin{it}
}{\end{it}}

\newenvironment{rlemma}[3][]{

\bigskip

\noindent \ifthenelse{\equal{#1}{}}{\bf #2 #3}{\bf #2 #3 (#1)}
\begin{it}
}{\end{it}}

\newcommand{\argmax}{\mathop{\mathrm{argmax}}}
\newcommand\floor[1]{\lfloor#1\rfloor}
\newcommand\ceil[1]{\lceil#1\rceil}
\newcommand{\fracpartial}[2]{\frac{\partial #1}{\partial  #2}}
\newcommand{\R}{\mathbb{R}} % real numbers

\newtheorem{assumption}{Assumption}

% Definitions of handy macros can go here

\newcommand{\vc}[1]{\bm{#1}} % vectors

\newcommand{\Normal}[2]{\ensuremath{\mathrm{subG}(#1,#2)}\xspace}
% normal distribution

% \newcommand{\dataset}{{\cal D}} % used anywhere?

%%%%%%%%%%%%%%%%%%%%%%%%%%%%%%%%%%%%
% Agreed upon symbols without macros
%%%%%%%%%%%%%%%%%%%%%%%%%%%%%%%%%%%%

% dimension for attribute vector: d
% stopping time: \tau
% phase of the algorithm: s


% To define and add in appropriate spots

% \Delta for utility difference
% \bar{\zeta} for error in utility difference
% q_{\delta} for mass of users


%%%%%%%%%%%%%%%%%%%%%%%%%%%%%%%%%%%%%
% Stuff related to Arms and Arm Pulls
%%%%%%%%%%%%%%%%%%%%%%%%%%%%%%%%%%%%%

\newcommand{\ARMNUM}{\ensuremath{N}\xspace} % number of arms

\newcommand{\ArmV}[1]{\ensuremath{\vc{\mu}_{#1}}\xspace}
% vector for location of an arm

\newcommand{\Arm}[2]{\ensuremath{\mu_{#1}^{(#2)}}\xspace}
% individual entries of the ith arm's location vector
% Argument 1: arm index
% Argument 2: coordinate

\newcommand{\NoiseV}[1][]{\ensuremath{\vc{\zeta}_{#1}}\xspace}
% vector of noise added to arm

\newcommand{\Noise}[2][]{\ensuremath{%
\ifthenelse{\equal{#1}{}}{\zeta_{#2}}{\zeta_{#1,#2}}\xspace}}
% individual entries of noise vector

\newcommand{\ObsV}[1]{\ensuremath{\vc{y}_{#1}}\xspace}
% observation of an arm: location plus noise

\newcommand{\ArmEV}[2]{\ensuremath{\vc{\hat{\mu}}_{#1,#2}}\xspace} 
% empirical estimator of the arm location based on samples
% Argument 1: time step
% Argument 2: arm

\newcommand{\ArmE}[3]{\ensuremath{\hat{\mu}_{#1,#2}^{(#3)}}\xspace}
% coordinates of empirical estimater of the arm location
% Argument 1: time step
% Argument 2: arm
% Argument 3: coordinate/attribute

\newcommand{\ErrV}[2]{\ensuremath{\vc{\epsilon}_{#1,#2}}\xspace} 
% error in the empirical estimator of the arm location
% Argument 1: time step
% Argument 2: arm

\newcommand{\Err}[3]{\ensuremath{\epsilon_{#1,#2}^{(#3)}}\xspace}
% coordinates of error in the empirical estimater of the arm location
% Argument 1: time step
% Argument 2: arm
% Argument 3: coordinate/attribute

\newcommand{\PullProb}[2]{\ensuremath{\phi_{#1,#2}}\xspace}
% probability that a random myopic agent will pull the arm
% Argument 1: time step
% Argument 2: arm
% \phi is a placeholder for now.


%%%%%%%%%%%%%%%%%%%%%%%%%
% Stuff related to agents
%%%%%%%%%%%%%%%%%%%%%%%%%

\newcommand{\AgV}{\ensuremath{\vc{\theta}}\xspace}
% vector for location of an agent, without subscript

\newcommand{\Ag}[1]{\ensuremath{\theta_{#1}}\xspace}
% individual entries of an agent location vector without time step

\newcommand{\AgentV}[1]{\ensuremath{\AgV_{#1}}\xspace}
% vector for location of an agent, subscripted by time of arrival

\newcommand{\Agent}[2]{\ensuremath{\theta_{#1,#2}}\xspace}
% individual entries of the agent location vector

\newcommand{\Best}[1]{\ensuremath{B_{#1}}\xspace}
% Best arm for a given agent

\newcommand{\Second}[1]{\ensuremath{B'_{#1}}\xspace}
% Second-best arm for a given agent

%\newcommand{\FirstTwo}[2]{\ensuremath{\Omega_{#1,#2}}\xspace}
% Set of agents who have i as their first choice and i' as their second



%%%%%%%%%%%%%%%%%%%%%%%%%%%%%%%%%%%%%
% Stuff related to agent distribution
%%%%%%%%%%%%%%%%%%%%%%%%%%%%%%%%%%%%%

\newcommand{\AgentDist}{\ensuremath{f}\xspace}
% distribution of locations for agents

\newcommand{\Diam}{\ensuremath{D}\xspace}
% "diameter" of support of agent distribution, [0,D]^d

\newcommand{\MinProb}{\ensuremath{p}\xspace}
% minimum probability of support for any arm

\newcommand{\TieDensity}{\ensuremath{L}\xspace}
% upper bound on the derivative of the density of near-ties

\newcommand{\TieCutoff}{\ensuremath{\hat{z}}\xspace}
% cutoff point below which the linear bound on near-ties holds

\newcommand{\AlmostTied}[1]{\ensuremath{q(#1)}\xspace}
% probability mass of almost tied agents across all arms

%\newcommand{\ClearPref}[1]{\ensuremath{p(#1)}\xspace}
% probability mass of agents with a clear preference for their first choice,
% minimized over all arms.
% not used anywhere.


%%%%%%%%%%%%%%%%%%%%%%%%%%%
% Stuff related to policies
%%%%%%%%%%%%%%%%%%%%%%%%%%%

\newcommand{\POLICY}{\ensuremath{\mathcal A}\xspace} % algorithm/policy
\newcommand{\Pay}[2]{\ensuremath{c_{#1,#2}}\xspace}
% payment offered for arms:
% Argument 1: time step
% Argument 2: arm

\newcommand{\PayA}[1]{\ensuremath{c_{#1}}\xspace}
% payment actually made in step t, i.e., c_{t,\Pull{t}}

\newcommand{\TotalPay}[1]{\ensuremath{C_{#1}}\xspace}
% cumulative payment made up to and including step t.

\newcommand{\Pull}[1]{\ensuremath{i_{#1}}\xspace}
% arm that was actually pulled at time step t

\newcommand{\NumPull}[2]{\ensuremath{m_{#1,#2}}\xspace}
% number of pulls of arm i up to time t:
% Argument 1: time step
% Argument 2: arm

\newcommand{\Regret}[1]{\ensuremath{r_{#1}}\xspace}
% regret incurred in time step t

\newcommand{\TotalRegret}[1]{\ensuremath{R_{#1}}\xspace}
% total regret incurred up to and including time step t

\newcommand{\MAXR}{\ensuremath{R}\xspace}
% upper bound on the maximum regret in any one round

\newcommand{\History}[1]{H_{#1}}
% History of pulls, payments, and observations

\newcommand{\EarlyCut}[2][]{\ensuremath{%
\ifthenelse{\equal{#1}{}}{\gamma(#2)}{\gamma^{#1}(#2)}}\xspace}

%%%%%%%%%%%%%%%%%%%%%%%%%%%%%%%%%%
% Stuff related to specific bounds
%%%%%%%%%%%%%%%%%%%%%%%%%%%%%%%%%%

\newcommand{\LatePhase}{\ensuremath{s_0}\xspace}
% lower bound after which concentration lemma holds

\newcommand{\EvenLaterPhase}{\ensuremath{s_1}\xspace}
% lower bound after which payments become unlikely



\newcommand{\AccE}[4]{\ensuremath{{\mathcal E}_{#1,#2,#3}(#4)}\xspace}
% event that estimate of a specific arm's attribute is sufficiently accurate
% Argument 1: time step
% Argument 2: arm index
% Argument 3: attribute
% Argument 4: error bound

\newcommand{\AccEU}[2]{\ensuremath{{\mathcal E}_{#1}(#2)}\xspace}
% event that all estimates of all arms' attributes are sufficiently accurate
% Argument 1: time step
% Argument 2: error bound


%%%%%%%%%%%%%%%%%%%%%%%%%%%%%%%%
% Stuff related to payment proof
%%%%%%%%%%%%%%%%%%%%%%%%%%%%%%%%

\newcommand{\SP}{\ensuremath{\omega}\xspace}
% sample path

\newcommand{\Env}[1]{\ensuremath{{\mathcal L}_{#1}}\xspace}
% envelope
% Argument: level of envelope


\title{Incentivizing Exploration by Heterogeneous Users}
\usepackage{times}

\coltauthor{\Name{Bangrui Chen} \Email{bc496@cornell.edu} \\
\Name{Peter I.\ Frazier} \Email{pf98@cornell.edu} \\
\addr Operations Research and Information Engineering\\
Cornell University\\
New York, NY 14850, USA
\AND
\Name{David Kempe} \Email{david.m.kempe@gmail.com}  \\
\addr Department of Computer Science\\
University of Southern California\\
Los Angeles, CA 90089, USA}

\begin{document}

\maketitle

\begin{abstract}

\begin{abstract}
%System identification is a fundamental problem in time-series analysis, control
%theory, and reinforcement learning.
%%
%Despite its importance, a sharp non-asymptotic analysis for the number of
%trajectories from an unknown dynamical system needed to identify its parameters
%remains an open question, even in the special case when the dynamics are governed by linear
%equations.
%%
%In this paper, we take an important step towards a non-asymptotic theory for system identification.
%We prove that the ordinary least-squares (OLS) estimator attains nearly minimax
%optimal performance for the identification of linear dynamical systems from
%a single observed trajectory.
%%
%Our analysis relies on a generalization of Mendelson's small-ball method to dependent data,
%eschewing the use of standard mixing-time arguments.
%%
%We capture the correct
%signal-to-noise behavior of the problem, showing that \emph{more unstable} linear
%systems are \emph{easier} to estimate.
%%
%This behavior is qualitatively different from arguments which rely on mixing-time
%calculations that suggest that unstable systems are more difficult to estimate.
%%
%Finally, our proof techniques generalize to a class of linear response
%time-series.


We prove that the ordinary least-squares (OLS) estimator attains nearly minimax
optimal performance for the identification of linear dynamical systems from
a single observed trajectory.
%
Our upper bound relies on a generalization of Mendelson's small-ball method to dependent data,
eschewing the use of standard mixing-time arguments.
%
Our lower bounds reveal that these upper bounds match up to logarithmic factors.
%
In particular, we capture the correct
signal-to-noise behavior of the problem, showing that \emph{more unstable} linear
systems are \emph{easier} to estimate.
%
This behavior is qualitatively different from arguments which rely on mixing-time
calculations that suggest that unstable systems are more difficult to estimate.
%
We generalize our technique to provide bounds for a more general class of linear response
time-series.



\end{abstract}

\end{abstract}

\begin{keywords}
Incentivizing Exploration, Multi-Armed Bandits, Social Learning
\end{keywords}

     

% !TEX root = onlinevarinancebandits.tex

\section{Introduction}
% Structure:
% \begin{itemize}
% \item The approach of (Regularized) ERM and its importance in Machine Learning.
% Solving such problems with sequential optimization algorithms such as SGD/SVRG/Online K-means.
% Maybe focus on SGD as a running example.
% \item  Mentioned the alternative option is uniform sampling. Describe/illustrate how importance sampling can be used to improve the performance. Give references. 
% \item Describe how the variance of the estimates is a natural measure of performance in this setting. Mention that low variance translates to better performance, e.g. for SGD. 
% \textbf{Online Problem:}
% Similarly to  Duchi/EPFL  we formulate importance sampling an online convex optimization problem.
% Describe the approaches of Duchi/EPFL say very nice things about them give them credit and discuss the limitations of their results/approaches.
% \item State our result. State our contributions+ discuss the improvements over previous work: \\
% (i) tighter regret guarantees with respect to the simplex.\\
% (ii) Showing that regret minimization makes sense in this setting.\\
% (iii) (Hopefully) complementary lower bounds. \\
% (iii) Efficient experimental implementation showing the benefits of the proposed method
% \kl{This is for COLT, for ArXiv put the experiments in (ii) place}

% Discuss the technical challenges of our work, specifically the fact that the costs are unbounded + the bandit feedback.
% Discuss the new regularization that we introduce, its benefits (closed form formula for the FTRL) + the challenge.
% Mention other settings with unbounded losses, e.g. log loss in portfolio selection.
 
%  \textbf{Related work}
%  Who should we cite? Look at Jaggi/Duchi/EPFL for references.

% \kl{Mention (where?) that we can use our approach for coordinate descent.}

% \end{itemize}
%Among the most important paradigms in machine learning is Empirical Risk Minimization (ERM) , which is often the strategy of choice due to its generality and statistical efficiency.
Empirical risk minimization (ERM) is among the most important paradigms in machine learning, and  is often the strategy of choice due to its generality and statistical efficiency.
In ERM, we draw a set of  samples $\D=\{x_1,\ldots,x_n\}\subset \X$ from the underlying data distribution, and we aim to find a solution $w\in\W$ that minimizes the empirical risk,  %The empirical risk serves as a proxy to the expected loss which is often. 
%the objective is to  find a solution $w\in\W$ that minimizes the empirical risk based on a collection of $n$ samples $\D=\{x_1,\ldots,x_n\}\subset \X $:
\begin{equation} \label{eq:ERM}
  \min_{w\in\W }L(w) := \frac{1}{n}  \sum_{i=1}^n \ell (x_i, w),
\end{equation}
where $\ell: \mathcal{X} \times \W \rightarrow \reals$ is a given loss function, and $\W\subseteq \reals^d$ is usually a compact domain.

In this work we are interested in sequential procedures for minimizing the ERM objective, and relate to such methods as \emph{ERM solvers}.
More concretely, we focus on the regime where the number of samples $n$ is very large,  and it is therefore desirable to employ ERM solvers that only require  few passes over the dataset. There exists a rich arsenal of such efficient solvers which have been investigated throughout the years, with the canonical example from this category being  Stochastic Gradient Descent (SGD).


% among are SVRG \citep{johnson2013accelerating} and SAGA \citep{defazio2014saga},
%
%
% such efficient sequential solvers have been developed throughout the years, with the canonical example from this category being  Stochastic Gradient Descent (SGD).

Typically, such methods  require an unbiased estimate of the loss function at each round, which is usually  generated   by sampling a few points uniformly at random from the dataset.
However, by employing uniform sampling, these methods are insensitive to the intrinsic structure of the data. In case of SGD, for example, some data points might produce large gradients, but they are nevertheless assigned the same probability of being sampled as any other point. This ignorance often results in high-variance estimates, which is likely to degrade the performance.

The above issue can be mended by employing non-uniform importance sampling.
And indeed, we have recently witnessed several  techniques to do so: %techniques.
%In recent years several approaches have been developed in order to address this issue.
\citet{zhao2015stochastic} and similarly \citet{needell2014stochastic}, suggest using prior knowledge on the gradients of each data point in order to devise predefined importance sampling distributions.  \citet{NIPS2017_7025} devise adaptive sampling techniques guided by a robust optimization approach. These are only a few examples of a larger body of work 
 \citep{bouchard2015online, alain2015variance, csiba2016importance}.

Interestingly, the recent works of \cite{pmlr-v70-namkoong17a} and \cite{salehi2017} formulate the task of devising importance sampling distributions as an online learning problem with bandit feedback. In this context, they  think of the algorithm, which adaptively chooses the distribution, as a player that competes against the ERM solver. The goal of the player is to minimize the cumulative variance of the resulting (gradient) estimates.  Curiously, both methods rely on some form of the ``linearization trick''\footnote{ By ``linearization trick'' we mean that these methods update according to a first order approximation  of the costs rather than the costs themselves.} 
%\footnote{If $g_t$ is a subgradient of the convex function $f:S\rightarrow \mathbb{R}$ at $w_t$, then $f(w_t) - f(u) \leq (w_t-u)^\intercal g_t$,  $\forall u \in S$.} 
 to resort to the analysis of the EXP3  \citep{auer2002nonstochastic}.

On the other hand, the theoretical guarantees of the above methods are somewhat limited. Strictly speaking, none of them provides regret guarantees with respect to the best fixed distribution in hindsight:  \citet{pmlr-v70-namkoong17a} only compete with the best distribution among a \emph{subset} of the simplex (around the uniform distribution).  Conversely, \cite{salehi2017} compete against a solution which might perform worse than the best in hindsight up to a multiplicative factor of $3$.

In this work, we adopt the above mentioned online learning formulation, and design novel importance sampling techniques. 
Our adaptive sampling procedure is simple and efficient, and 
in contrast to previous work, we are able to provide regret guarantees with respect to the best fixed point among the simplex.
As our contribution, we
\vspace{-1.5mm}
\begin{itemize}
\setlength\itemsep{0.05em}
\item motivate theoretically why regret minimization is meaningful in this setting, 
\item propose a novel bandit algorithm for variance reduction ensuring regret  of~$\tO(n^{1/3}T^{2/3})$,
\item empirically validate our method, and provide an efficient implementation\footnote{The source code is available at  \url{https://github.com/zalanborsos/online-variance-reduction}}.
\end{itemize}
On the technical side, we do not rely on a ``linearization trick'' but rather directly employ a scheme based on the classical 
 Follow-the-Regularized-Leader approach. 
Our analysis entails several technical challenges, most notably handling  unbounded cost functions while only receiving partial (bandit) feedback. Our design and analysis draws inspiration from the seminal works of  \citet{auer2002nonstochastic}  and 
\cite{Abernethy08}. 
Although we present our method for choosing \emph{data points}, it naturally applies to choosing \emph{coordinates in coordinate descent} or even \emph{blocks} of thereof \citep{allen2016even,perekrestenko2017faster, nesterov2012efficiency, necoara2011random}.
More broadly, the proposed algorithm can be incorporated in \emph{any sequential algorithm} that relies on an unbiased estimation of the loss. A prominent  application of our method is variance reduction for SGD, which can be achieved by considering  gradient norms as  losses, i.e., replacing $\ell(w,x_i) \leftrightarrow \|\nabla \ell(w,x_i)\|$. With this modification, our method is minimizing the cumulative variance of the gradients throughout the optimization process.

%defining the bandit feedback as the norm of the gradient estimate. 
The paper is organized as follows. In Section \ref{sec:Motivation}, we formalize the online learning setup of variance reduction, and motivate why regret is a suitable performance measure. As the first step of our analysis, we investigate the full information setting in Section \ref{sec:full-info}, which serves as a mean for studying the bandit setting in Section \ref{sec:bandit}. Finally, we validate our method empirically, and provide the detailed discussion of the results in Appendix  \ref{sec:experiments}. 



%We achieve this by relying on classical results from Follow-the-Regularized-Leader framework, where the regularizer is chosen to suit the setting of variance reduction with importance sampling.

%\newpage
%
%
% rely on solving a a ro
%
%There exists
%
%In 
%Addressing the regime where the number of samples $n$ is very large, efficient \emph{sequential} procedures have been developed, that perform only a few passes over the dataset. These methods usually require an unbiased estimate of the loss function at each round, and they generate the estimate by sampling a few points uniformly at random from the dataset. The canonical example from this category is Stochastic Gradient Descent (SGD).
%
%However, by employing uniform sampling, these methods are agnostic to the intrinsic structure of the data. In case of SGD, for example, some data points might produce large gradients, but they are nevertheless assigned the same probability of being sampled as any other point. This ignorance often results in high-variance estimates.
%
%References:
%\begin{itemize}
%\item \textbf{Variance reduction with uniform sampling}: SVRG \citep{johnson2013accelerating} and SAGA \citep{defazio2014saga}, \cite{xiao2014proximal}
%\item \textbf{Variance reduction with importance sampling (points): } \citep{needell2014stochastic, zhao2015stochastic, bouchard2015online, csiba2016importance,alain2015variance, NIPS2017_7025}
%\item \textbf{Importance sampling, but without direct interpretation as variance reduction (points): } \cite{strohmer2009randomized}, 
%\item \textbf{Coordinate descent, with non-uniform sampling:} \cite{allen2016even,perekrestenko2017faster, nesterov2012efficiency, necoara2011random}
%\item \textbf{Bandits, both points and coordinates:} \cite{pmlr-v70-namkoong17a}, \cite{salehi2017}, \cite{salehi2017stochastic}
%\end{itemize}
%
%
%This issue is addressed by several variance reduction techniques, some prominent examples being  SVRG \citep{johnson2013accelerating} and SAGA \citep{defazio2014saga}. In case of (strongly) convex loss functions, the reduced variance directly translates to improved convergence bounds. \textcolor{red}{An important class of methods that allow variance reduction interpretation}  rely on the technique of importance sampling \citep{needell2014stochastic, zhao2015stochastic, bouchard2015online, csiba2016importance,alain2015variance%, allen2016even, perekrestenko2017faster%
%, NIPS2017_7025}, where the sampling distribution is either fixed or adaptive over the iterations. Since competing against \textcolor{red}{the optimal per-round} sampling distributions is usually computationally infeasible, the methods are often compared to the optimal sampling distribution \emph{in hindsight}.
%
%An interesting idea of the recent works of \cite{pmlr-v70-namkoong17a} and \cite{salehi2017} is to formulate the task of finding a competitive sampling distribution under importance sampling for variance reduction as a \emph{bandit problem}. In this setting, no-regret assures that the chosen distribution performs close to the optimal stationary distribution in hindsight. Both methods rely on some form of the ``linearization trick'' \citep{shalev2012online} to resort to the analysis of the EXP3  \citep{auer2002nonstochastic} and obtain similar algorithms. These methods show convincing convergence guarantees for stochastic optimization for convex ERM both theoretically and empirically.
%
%On the other hand, the two methods come with limitations. While the latter is guaranteed to approximate the variance under the optimal sampling distribution in hindsight within a factor of 3, the former incorporates a KL-projection step in order to ensure that the sampling probabilities are larger than some threshold $\pmin$ --- an additional hyperparameter that affects the convergence bounds.
%
%In this work, we pursue the same idea of employing bandit optimization for variance reduction and we design an algorithm that suffers from none of the limitations mentioned above. We achieve this by relying on classical results from Follow-the-Regularized-Leader framework, where the regularizer is chosen to suit the setting of variance reduction with importance sampling. As our contribution, we
%\vspace{-1.5mm}
%\begin{itemize}
%\setlength\itemsep{0.05em}
%\item motivate theoretically why regret minimization is meaningful in this setting,
%\item propose and analyze a novel bandit algorithm for variance reduction,
%\item empirically validate our method and provide an efficient implementation\footnote{The source code is available at  \url{https://github.com/zalanborsos/online-variance-reduction}}.
%\kl{Dont forget to remove this footnote in the anonimized COLT submission!}
%\end{itemize}
%The analysis entails several technical challenges, most notably handling  unbounded cost functions and unbounded regularizers. Although we present our method for choosing \emph{datapoints} in an optimization problem, it naturally applies to choosing \emph{coordinates in coordinate descent} or even \emph{blocks} of thereof. More broadly, the proposed algorithm can be incorporated in \emph{any sequential algorithm} that relies on unbiased estimation of the loss.
%\kl{add citation about coordinate descent, see Cevher and Jaggi}

\section{Preliminaries}
\label{sec:prob}

We consider a multi-armed bandit setting with \ARMNUM arms.
Arm payoffs are determined by $d$ \emph{attributes} or \emph{features};
hence, arms are identified with vectors $\ArmV{i} \in \R^d$.
The \ArmV{i} are (adversarially) fixed, and unknown to the agents
and the principal.
When arm $i$ is pulled, its current utility-relevant features are
determined as $\ArmV{i} + \NoiseV$, where \NoiseV is a mean-zero
independent continuous sub-Gaussian%
\footnote{For simplicity of notation, we assume that the variance proxy
  $\sigma^2$ is uniform across time steps.
  The analysis extends easily when the variance proxy changes over time.}
noise vector $\NoiseV \sim \Normal{\vc{0}}{\sigma^2 I_{d}}$.
Here, $I_d$ denotes the $d \times d$ identity matrix.

At each time $t$, a new user (or \emph{agent}) arrives,
whose feature vector (which we also call his \emph{type})
$\AgentV{t} \in \R^d$ is drawn from a known distribution \AgentDist.
Depending on context, we will identify agents with their arrival time
$t$ or their type \AgentV{t}.
When agent $t$ pulls arm $\Pull{t} := i$,
he and all future agents observe a vector
$\ObsV{t} = \ArmV{i} + \NoiseV[t]$ for arm $i$,
and his reward is $\AgentV{t} \cdot \ObsV{t}$,
i.e., agents have linear preferences.
Although the model is linear, it permits additional flexibility
through the addition of derived attributes that are nonlinear
transformation of some original attribute vector.
One can simply define $\Ag{j} = h_j(\AgV)$ for some known
nonlinear functions $h_j$ and $j=d+1,\ldots,d'$, and then increase the
number of attributes to $d'$.

For each time $t$ and arm $i$, let \NumPull{t}{i} be the number of
times that arm $i$ has been pulled (strictly) before time $t$.
An agent at time $t$ estimates arm $i$'s attribute vector as the
average of vectors observed during the arm's past pulls:
$\ArmEV{t}{i} = \frac{1}{\NumPull{t}{i}+1} \cdot
(\ArmEV{0}{i} + \sum_{t'<t: \Pull{t'} = i} \ObsV{t'})$;
here, \ArmEV{0}{i} is a single draw $\ArmV{i} + \NoiseV$ for arm $i$.
(In other words, we assume each arm is pulled once for free at time 0.)
For a justification of the assumption that agents can observe
the actual noisy vectors \ObsV{t'} of past pulls, see
Section~\ref{sec:introduction}.
  
Since each user only pulls an arm once, users are \emph{myopic}:
in the absence of incentives, user $t$ will pull an arm from
$\argmax_i \AgentV{t} \cdot \ArmEV{t}{i}$.
To incentivize users to explore more, the principal can offer
\emph{payments} $\Pay{t}{i}$ to user $t$ for pulling arm $i$.
Then, user $t$ will pull an\footnote{We assume that ties are broken in
  favor of an arm with largest payment \Pay{t}{i}.}
% ; this assumption is
%   only for notational convenience, and can of course be avoided by
%   raising payments infinitesimally.
%   \pfcomment{In the finite preference setting this is clear, but it's a little bit delicate when there are continuum preferences, since regardless of the payment we offer there may always be support within our preference distribution for a $\theta$ that causes a tie.  In this continuum setting it should instead be possible to perturb infinitessimally so that we avoid ties with probability 1.}  
% }
arm $i$ maximizing $(\Pay{t}{i} + \AgentV{t} \cdot \ArmEV{t}{i})$.
The principal cannot observe \AgentV{t},
and only knows the distribution \AgentDist from which it is drawn.
Her goal is to reduce both the cumulative
\emph{regret} experienced by all users up to time $T$,
and the total \emph{payment} she makes to the users.

We define the regret at time $t$ as
$\Regret{t} = (\max_{i} \AgentV{t} \cdot \ArmV{i}) - \AgentV{t} \cdot \ArmV{\Pull{t}}$,
and the cumulative regret up to time $T$ as
$\TotalRegret{T} = \sum_{t=1}^{T} \Regret{t}$.
Similarly, $\PayA{t} = \Pay{t}{\Pull{t}}$ is the actual incentive
payment at time $t$,
and the cumulative payment up to time $T$ is
$\TotalPay{T} = \sum_{t=1}^{T} \PayA{t}$.
More formally, the principal's goal is to find a policy
\POLICY for offering payments under which both the cumulative expected
regret
$\Expect{\TotalRegret{T}}$ the cumulative expected payment
$\Expect{\TotalPay{T}}$ are small.

We assume above that the distribution $\AgentDist$ over preference
vectors is known to the principal.
In practice, a principal would estimate this distribution from agents'
selections and the attribute estimates and offered payments on which
they were based.
While we do not show it, we hypothesize that our regret guarantees
only change by constants as long as the principal's estimate of the
conditional probability of pulling an arm is correct to within a
constant factor.

% \pfedit{
% We assume above that the distribution $\AgentDist$ over preference vectors is known.
% In practice, one would estimate this from agents' selections and attribute estimates $\ArmEV{t}{i}$
% and offered payments $\Pay{t}{i}$ at the time of selection.  
% In the process of doing such an estimation procedure, one could also select nonlinear transformations of existing attributes (increasing $d$) to improve the predictive accuracy of the assumed linear model.   
% Our analysis assumes such a procedure has already been conducted.
% This assumption is most appropriate when the number of dimensions $d$ is small relative to the number of items $N$, so that the time required to estimate the distribution over preference vectors is small relative to the time required to learn about items.
% }
%\pfcomment{One modeling assumption I don't say anything about is that $\AgentDist$ is known.  I thought about adding some text saying that it can actually be estimated from user selections as we go, which is true, but it isn't clear that we can learn it much quicker than the item attributes, in general.  Another tact is to argue that our method is robust to mis-estimation of this, which seems true intuitively, but I'm not sure how to argue it here.  Probably it is possible to desgin an algorithm that learns this distribution along with item attributes.}

To support the formulation of our results and the analysis,
we define the following additional notation.
We let
$\Best{\AgV} \in \argmax_i \AgV \cdot \ArmV{i}$
and
$ \Second{\AgV} \in \argmax_{i \neq \Best{\AgV}} \AgV \cdot \ArmV{i}$
denote the (indices of) the best and second-best arms for an agent
with attribute vector \AgV,
breaking ties arbitrarily (but consistently).
Notice that based on Assumption~\ref{A1} below,
\Best{\AgV} is unique with probability 1.

% \dkdelete{This behavior may be recovered if agents are Bayesian and
%   share a common non-informative prior distribution that is constant
%   over $\mathbb{R}^m$ and know $\sigma^2$.  In this case, the
%   posterior distribution on $u_{i}$ at time $t$ is multivariate normal
%   with mean $u_{i,t}$, and the expected value of $\theta_t \cdot u_i$
%   under this posterior conditioned on $\theta_t$ is $\theta_t \cdot
%   u_{i,t}$ (see Equation 2.13 in Section 2.5, \cite{Ge04}).
%   Alternatively, one may simply take our assumption that agents use
%   the average as their estimate of an attribute value directly without
%   such a Bayesian justification.}
% \dkcomment{I wonder what's the best way to discuss this.}

% \paragraph{Properties of \AgentDist}
Our algorithms rely on the following three assumptions on the agent distribution
\AgentDist.

\begin{assumption}[Compact Support] \label{A2}
\AgentDist has a compact support set contained in $[0,\Diam]^d$.
\end{assumption}

Let $\MinProb = \min_{i} \Prob[\AgV \sim \AgentDist]{\Best{\AgV} = i}$
denote the minimum (over all arms) fraction of users that prefer any
particular arm.

\begin{assumption}[Every arm is someone's best] \label{A3}
Each arm $i$ has a strictly positive proportion of users for whom $i$
is the best arm; that is, $\MinProb > 0$.
\end{assumption}

%Let $\FirstTwo{i}{i'} = \Set{\AgV}{\Best{\AgV} = i, \Second{\AgV} = i'}$
%be the set of agent attribute vectors whose best arm is $i$ and
%second-best arm is $i'$.
% Let $F_{i,i'}$ be the cumulative distribution function,
% $F_{i,i'}(z) = \Prob[\AgV \sim \AgentDist]{\AgV \in \FirstTwo{i}{i'}
%   \mbox{ and } (\ArmV{i}-\ArmV{i'}) \cdot \AgV \leq z}$
% of $(\ArmV{i}-\ArmV{i'}) \cdot \AgV$,
% on $\AgV \in \Omega_{i,i^{'}}$.
% In words, $F_{i,i'}$ is the distribution of the \emph{strength} of the
% preference of a random agent for arm $i$ over arm $i'$

% $\Best{\AgV} \in \argmax_i \AgV \cdot \ArmV{i}$
% and
% $ \Second{\AgV} \in \argmax_{i \neq \Best{\AgV}} \AgV \cdot \ArmV{i}$

Let \AlmostTied{z} be the cumulative distribution function (CDF)
$\AlmostTied{z} = \Prob[\AgV \sim \AgentDist]{%
  (\ArmV{\Best{\AgV}}-\ArmV{\Second{\AgV}}) \cdot \AgV \leq z}$.
In words, \AlmostTied{z} is the CDF of the \emph{strength} of the
preference of a random agent for his best arm over his second-best arm.

\begin{assumption}[Not too many near-ties] \label{A1}
Near-ties have vanishing probability; 
that is, there exist constants $\TieCutoff > 0, \TieDensity$ such that 
$\AlmostTied{z} \leq \TieDensity \cdot z$ for all $z \leq \TieCutoff$.
\end{assumption}

Assumption~\ref{A1} implies that ties happen with probability $0$.
One special case of interest we discuss below is $\TieDensity = 0$.
This case arises when there is a cutoff
\TieCutoff such that only a measure-zero set of agents has two best arms
within utility \TieCutoff of each other.
In particular, $\TieDensity = 0$ happens whenever \AgentDist is supported on a finite set of types,
and each type has a unique best arm.

\paragraph{Roles of parameters:}
Our problem setting is characterized by many parameters.
Of these, we consider \ARMNUM, $\MinProb \leq 1/\ARMNUM$ and $T$ to be
the key parameters, and also give some prominence to
\TieDensity to illustrate an interesting phenomenon.
On the other hand, we consider
$\sigma, \Diam, d$, and \TieCutoff to be constant.
We track all parameters through most of our proofs,
but report final results in terms of only the key parameters,
except where we illustrate a specific point.
In particular, we consider $d$ a constant, because users
typically evaluate products (or restaurants, etc.) by a small number of
relevant features.


\section{Overview of Results and Discussion}

Our main algorithm is presented as
Algorithm~\ref{alg:basic-incentivizing}
in Section~\ref{sec:ub}.
Our main result is the following pair of theorems%
\footnote{Recall that we omit the dependence on parameters other than
  \ARMNUM, \MinProb, $T$ unless making a particular point.},
analyzing the payments and regret of the algorithm.

\begin{theorem} \label{rst:budget}
The expected total payment of
Algorithm~\ref{alg:basic-incentivizing} is at most
$O \left(\ARMNUM^2 \cdot \e^{2/\MinProb} \right)$.
\end{theorem}

\begin{theorem} \label{rst:regret}
For any time horizon $T$, the expected cumulative regret for
Algorithm~\ref{alg:basic-incentivizing} up to time $T$ is bounded
above by 
$O \left(\ARMNUM \cdot \e^{2/\MinProb} + \TieDensity \ARMNUM \log^3(T) \right)$.
\end{theorem}

When $\TieDensity = 0$, the bound of Theorem~\ref{rst:regret} is
constant in $T$;
thus, the algorithm achieves constant regret using constant
expected payments. 
As discussed in Section~\ref{sec:prob},
the case $\TieDensity = 0$ arises, for instance, for
discrete agent distributions with finite support.
In fact, when $\TieDensity = 0$, one can also modify the algorithm to
reduce the dependence on \MinProb from exponential to polynomial.
Theorem~\ref{rst:discrete} (given later) states that the modified
algorithm achieves expected regret
$O \left(\ARMNUM/\MinProb \right)$
with expected payments of
$O \left(\ARMNUM^2/\MinProb \right)$.
  
The fact that constant regret can be achieved with constant payment
(independent of $T$) when $\TieDensity = 0$ suggests aiming for a
constant bound more generally, i.e., for $\TieDensity > 0$.
That such a bound is unachievable is shown in Appendix~\ref{sec:lb},
where we show a lower bound of $\Omega(\log(T))$ on
the expected regret of any algorithm.
The instance is simple: it has two arms, one with known attributes; 
in addition, one draw from the other arm is observed in
each step $t$ even when it is not pulled.
While the probability of pulling the wrong arm decreases over time, it
does not do so fast enough, causing the stated regret.

The exponential dependence on $1/\MinProb$ implies an exponential
dependence on \ARMNUM (because $\MinProb \leq 1/\ARMNUM$).
This exponential dependence arises from a need to continue to
incentivize arm pulls to ensure that nearly tied agents learn their
best arms quickly.
Aside from the assumption that $\TieDensity = 0$,
another assumption allows us to eliminate this exponential dependence.
Namely, when \MinProb (or a lower bound on it) is known ahead of time,
the algorithm can be modified to incentivize arms less aggressively.
As shown in Theorem~\ref{rst:known-p},
the modified algorithm has expected regret at most
$O \left(\frac{\ARMNUM}{\MinProb^3}
+\frac{\ARMNUM \TieDensity \log^3(T)}{\MinProb} \right)$,
with expected payments of at most
$O \left(\frac{\ARMNUM^2}{\MinProb^{5/2}} \right)$.

We compare these bounds to those for standard bandits,
focusing on the dependence on $T$.  
The standard bandit setting is the case when the agent types \AgV
are concentrated on a single point, and agents pull arms at the
principal's direction without requiring payment.
(This setting violates our Assumption~\ref{A3},
so our bounds do not apply to it.)
Then, the payment is $0$ and the expected regret scales as
$\Theta(\log(T))$ \cite[Theorem 2.1]{bubeck2012regret}.

Our algorithm's payment is constant in $T$,
while its regret is $O(\log^3(T))$ in general with a lower bound of
$\Omega(\log(T))$;
when preferences are discrete, our algorithm's regret is
constant in $T$.
Thus, viewed solely in terms of the dependence on $T$,
the best performance achievable seems comparable to that in a standard
multi-armed bandit problem;
but when preferences are discrete, the constant regret
surpasses the $\Theta(\log(T))$ achievable in the standard multi-armed
bandit setting.
This may seem surprising, because the principal in our setting has
both less control and less information than in the standard bandit setting.
The result arises because heterogeneity in preferences provides
free exploration, and allows all of the arms to be pulled infinitely
often without incurring regret once estimates are accurate enough.

While heterogeneity in preferences enables this free exploration,
heterogeneity alone is not always sufficient for enabling performance
improvements compared to the standard bandit setting.
Indeed, suppose that agents are still heterogeneous,
but the principal pulls arms directly.
Unless the principal can also observe the agents' types,
she will be unable to correctly choose each agent's preferred arm,
even with infinite exploration of arm attributes.
Regret will then grow as $\Omega(T)$. 

Thus, reaping the benefits of (unobserved) heterogeneous preferences
requires the principal to give up direct control of the arms,
providing agents the autonomy they need to express their private
information about their own preferences.
Our results show that simple arm-based incentives are sufficient
to overcome the apparent challenges created by this abdication of
control.


\section{Main Algorithm and Analysis}
\label{sec:ub}
The algorithm achieving the bounds of
Theorems~\ref{rst:budget} and~\ref{rst:regret} is simple.
It mostly allows agents to exploit, but when an arm is sufficiently
unlikely to be pulled,
it incentivizes this arm with a payment high enough
to guarantee that the next agent pulls it.
This way, the algorithm ensures that each arm is pulled often enough.

More precisely, the algorithm divides time into \emph{phases}
$s = 1, 2, 3, \ldots$.
Phase $s$ starts when each arm has been pulled at least $s$ times.
We indicate the start time of phase $s$ by $t_s$. An arm $i$ is \emph{payment-eligible} at time $t$ (in phase $s$)
if both of the following hold:

\begin{itemize}
\item $i$ has been pulled at most%
\footnote{in fact: exactly, since the algorithm entered phase $s$}
$s$ times up to time $t$, i.e., $\NumPull{t}{i} \leq s$.
\item 
The conditional probability of pulling arm $i$ is less than
$1/\log(s)$ given the current estimates \ArmEV{t}{i'} of the arms'
attribute vectors.  In other words, we require
$\PullProb{t}{i} < 1/\log(s)$ where 
$\PullProb{t}{i} = \Prob[\AgV \sim \AgentDist]{\AgV \cdot \ArmEV{t}{i} > \AgV
\cdot \ArmEV{t}{i'} \mbox{ for all } i' \neq i \mid \ArmEV{t}{i'}\ \forall i'}$
is the probability that arm $i$ will be pulled
by the next (random) agent based on the current estimates. 
Our assumption of a continuous noise distribution and one free pull
ensure that ties in $\AgV \cdot \ArmEV{t}{i}$ between arms occur with
probability $0$.
\end{itemize}


When multiple arms are payment-eligible, the algorithm picks one arbitrarily to incentivize.
When the algorithm decides to incentivize an arm $i$,
it offers ``whatever it takes,'' i.e., offers a payment of
$\Pay{t}{i} = \max_{\AgV,i'} \AgV \cdot (\ArmEV{t}{i'} - \ArmEV{t}{i})$.
The maximum for \AgV is taken over the support of \AgentDist;
recall that we assumed this support to be compact.
The payment \Pay{t}{i} may appear unnecessarily high.
Indeed, it suffices to
incentivize only a $1/\log(s)$ fraction of the agents,
and our bounds also hold for an alternate version of our algorithm that 
offers payment
$\Pay{t}{i} = \sup \Set{c \geq 0}{%
\Prob[\AgV \sim \AgentDist]{c + \AgV \cdot \ArmEV{t}{i} \geq \max_{i'\ne i} \AgV \cdot \ArmEV{t}{i'}} \leq 1/\log(s)}$.
(This definition ensures that \Pay{t}{i} is well-defined
and incentivizes at least a $1/\log(s)$ measure of agents,
even if $f$ has discrete points.)
However, we focus on the higher payments for simplicity of presentation.

Notice that \PullProb{t}{i} depends on the estimates for \emph{all}
arms; thus, by pulling another arm $i'$, an arm $i$ may become
payment-eligible, or cease to be so.
Algorithm~\ref{alg:basic-incentivizing} gives the full details.


\begin{algorithm}
\caption{Algorithm: Incentivizing Exploration \label{alg:basic-incentivizing}}
\begin{algorithmic}
\STATE Set the current phase number $s = 1$.
\COMMENT{Each arm is pulled once initially ``for free.''}
\FOR{time steps $t = 1, 2, 3, \ldots$}
\IF{$\NumPull{t}{i} \geq s+1$ for all arms $i$}
\STATE Increment the phase $s = s + 1$.
\ENDIF
\IF{there is a payment-eligible arm $i$}
\STATE Let $i$ be an arbitrary payment-eligible arm.
\STATE Offer payment
$\Pay{t}{i} = \max_{\AgV,i'} \AgV \cdot (\ArmEV{t}{i'} - \ArmEV{t}{i})$
for pulling arm $i$
(and payment 0 for all other arms).
\ELSE
\STATE Let agent $t$ play myopically, i.e., offer payments 0 for all arms.
\ENDIF
\ENDFOR
\end{algorithmic}
\end{algorithm}

The high-level idea in the proofs of our main results, Theorems~\ref{rst:budget} and~\ref{rst:regret},
is the following.
Because the algorithm ensures that each arm is pulled
``frequently enough,''
the estimates \ArmEV{t}{i} become gradually more accurate in the
phase number $s$.
Thus, the fraction of agents who misidentify their best arm decreases.
Because each arm has enough agents that would prefer it based
  on its true attribute vector, 
once the arms' attribute vectors are learned well enough,
the algorithm will not need to incentivize any more,
resulting in a payment bound independent of $T$.
Similarly, instantaneous regret will decrease, and mostly accrue
due to ``problematic'' agents who are nearly tied in their preferences
between their top two arms.
The detailed analysis following this outline is complicated by
dependence between the agents' arm pulls and the
estimates which in turn are based on past arm pulls.
We begin with several technical lemmas that are used for both the
payment and regret bounds.

To formally reason about the event that the estimates of arms'
attributes vectors are accurate enough --- or fail to be so ---
we define the events
$\AccE{t}{i}{j}{x} := [|\ArmE{t}{i}{j} - \Arm{i}{j}| \leq x]$
that attribute $j$ of arm $i$ at time $t$ is estimated to
within accuracy $x$ or better.
Then, 
$\AccEU{t}{x} = \bigcap_{i,j} \AccE{t}{i}{j}{x}$
is the event that at time $t$, all arm attribute
estimates are accurate to within $x$ simultaneously.
We will show that for suitable choices of $t, x$,
the events \AccE{t}{i}{j}{x}
(and hence, by a union bound, \AccEU{t}{x})
have high probability,
and that when they do, myopic agents do not make large mistakes.

%We now prove our main results, Theorems~\ref{rst:budget} and~\ref{rst:regret}.



\subsection{General Lemmas}

% We begin by showing that under our assumptions, the measure of
% problematic arms cannot be too large.

% \begin{lemma} \label{lem:sdelta}
% $\AlmostTied{\delta} \leq \TieDensity \cdot \delta$ for all $\delta$.
% \end{lemma}

% \begin{emptyproof}
% Using the upper bound from Assumption~\ref{A1}, we can bound

% \begin{align*}
% \AlmostTied{\delta}
% & = \sum_{i,i'} \ProbC{\AgV \cdot (\ArmV{i} - \ArmV{i'}) \leq \delta}%
%     {\AgV \in \FirstTwo{i}{i'}}
%   \cdot \Prob{\AgV \in \FirstTwo{i}{i'}}\\
% & \leq \sum_{i,i'} \TieDensity \cdot \delta
%     \cdot \Prob{\AgV \in \FirstTwo{i}{i'}}
% \; = \; \TieDensity \cdot \delta. \QED
% \end{align*}
% \end{emptyproof}

We begin by bounding the length of any phase,
and more generally, the number of arm pulls by any given set
  of agents.
The bound of Lemma~\ref{lem:phase-length} will support bounding the
regret of early rounds (before tail bounds have kicked in).
The proof of Lemma~\ref{lem:phase-length} and all other proofs missing
in the main paper may be found in the appendix.
  
\begin{lemma} \label{lem:phase-length}
For any $s\geq 3$, the expected length of phase $s$ is at most
$\ARMNUM \cdot \log(s)$ time steps.

More generally, for any set of types $A$, the expected number of times that 
an agent with a type in $A$ appears in a phase $s$ is at most 
$f(A) \cdot \ARMNUM \cdot \log(s)$,
where $f(A) := \Prob[\AgV \sim \AgentDist]{\AgV \in A}$.
\end{lemma}

%For any $s\geq 3$, the expected length of phase $s$ is at most
%$\ARMNUM \cdot \log(s)$ time steps.
                  
We now state the key technical lemma which captures the intuition that
the estimates of the arms' attribute vectors become
more accurate with increasing phases $s$.

\begin{lemma} \label{lem:round-prob}
Recall the noise is a mean-zero sub-Gaussian($\sigma^2$) random variable.
Let \LatePhase be fixed, 
and let $x_n, x'_n > 0$ be sequences satisfying 
$\sqrt{0.6 n \cdot \log (\log_{1.1}(n) + 1) + \frac{n x_n^2}{16 \sigma^2}}
\leq \frac{n x'_n}{2 \sigma}$,
for all $n \geq \LatePhase$.
Let $\tau_s$ be a stopping time
(which may depend on the entire past history)
which is almost surely in phase $s$,
i.e., satisfying $\tau_s \in [t_s, t_{s+1})$ almost surely.
Then, for any arm $i$, attribute $j$, and phase $s \geq \LatePhase$,
we have 
$\Prob{\AccE{\tau_s}{i}{j}{x'_s}}
\geq 1 - 24 \exp\left(-\frac{1.8 s x_s^2}{16 \sigma^2} \right)$.
\end{lemma}

Next, we show the complementary result:
when the event \AccEU{t}{x} happens, no myopic agent incurs large regret.

\begin{lemma} \label{lem:right-choice}
Let $x > 0$ be arbitrary.
When \AccEU{t}{x} happens,
no agent \AgV will pull a highly suboptimal arm, i.e., an arm $i$ with 
$\AgV \cdot (\ArmV{\Best{\AgV}} - \ArmV{i}) > 2\Diam d x$.
\end{lemma}



\subsection{Bounding the Total Payment}

As a first step towards bounding the total payment (and also regret),
we show that for sufficiently late phases,
under the event \AccEU{t}{x} for suitably small $x$,
the algorithm does not offer any payments.

\begin{lemma} \label{lem:no-incentives}
Fix an arm $i$.
Let $s \geq \exp(2/\MinProb)$, and let $\tau_s$ be the (random)
time when arm $i$ is pulled for the \Kth{s} time.
Let $\hat{x} = \frac{1}{2\Diam d}
  \cdot \min(\TieCutoff, \frac{\MinProb}{2\TieDensity})$.
Under \AccEU{\tau_s}{\hat{x}},
this pull of arm $i$ is not incentivized.
\end{lemma}

Towards bounding the algorithm's total payment, we now bound by a
constant the \emph{number} of rounds in which the algorithm makes a
payment.
This bound also turns out to be useful for bounding the total regret.

\begin{lemma} \label{lem:numP}
The expected number of time steps in which
Algorithm~\ref{alg:basic-incentivizing}
makes any payment is at most $O\left( N\exp\left(\frac{2}{p}\right) \right)$.
\end{lemma}

\begin{proof}
We partition phases into early and late phases.
For each of the early phases,
we crudely bound the number of payments by \ARMNUM,
using that each arm is incentivized at most once per phase.
For later phases,
we use Lemma~\ref{lem:no-incentives},
which rules out any incentives unless large misestimates of the arm
locations occur, which is exponentially unlikely by 
Lemma~\ref{lem:round-prob}.

To make this intuition precise, we set
$x = x' = \frac{1}{2 \Diam d} \cdot \min(\TieCutoff, \frac{\MinProb}{2 \TieDensity})$
(which is independent of the phase number $s$
and is equal to $\hat{x}$ defined in Lemma~\ref{lem:no-incentives}).
We set the cutoff point between early and late phases to
$\EvenLaterPhase = \max(2, \frac{30 \sigma^3}{x^3}, \exp(\frac{2}{\MinProb}))$.
We verify below that
$\sqrt{0.6 n \cdot \log (\log_{1.1}(n) + 1) + \frac{n x^2}{16 \sigma^2}}
\leq \frac{n x}{2 \sigma}$
for all $n \geq \max(2, \frac{30 \sigma^3}{x^3})$.

Fix an arm $i$, let $s \geq \EvenLaterPhase$, and 
define $\tau_s$ as in Lemma~\ref{lem:no-incentives}.
By Lemma~\ref{lem:round-prob} (with our choice of $x = x'$),
and a union bound over all $i,j$, we bound the probability
$\Prob{\AccEU{\tau_s}{x}}
\geq 1 - 24 \ARMNUM d \cdot \exp\left(-\frac{1.8 s x^2}{16 \sigma^2} \right)$.
And by Lemma~\ref{lem:no-incentives},
under the event ${\mathcal E}_{\tau_s}(x)$,
arm $i$ is not payment-eligible.

Thus, for any $s \geq \EvenLaterPhase$,
the \Kth{s} pull of arm $i$ is incentivized  
with probability at most
$24 \ARMNUM d \cdot \exp \left(
- \frac{1.8 s}{256 \Diam^2 d^2 \sigma^2}
  \cdot \min(\TieCutoff, \frac{\MinProb}{\TieDensity})^2
\right)$.
Summing over all arms and phases $s$,
adding the at most $\ARMNUM \EvenLaterPhase$ incentivizations in
the first \EvenLaterPhase phases, 
and using that $\exp(-x) \leq 1-x/2$ for $x \leq 1$,
the expected total number of arm incentivizations is at most
\begin{align*}
\lefteqn{\ARMNUM \EvenLaterPhase
  + 24 \ARMNUM^2 d \cdot \frac{1}{1 - \exp \left(
- \frac{1.8}{256 \Diam^2 d^2 \sigma^2}
  \cdot \min(\TieCutoff, \frac{\MinProb}{\TieDensity})^2
  \right)}}
\\ & = O\left( \max \left(
     \frac{\ARMNUM \TieDensity^3 \Diam^3 d^3 \sigma^3}{\MinProb^3},
\frac{\ARMNUM \Diam^3 d^3 \sigma^3}{\TieCutoff^3},
     \ARMNUM \exp \left(\frac{2}{\MinProb} \right) \right)
+ \max \left(
     \frac{\ARMNUM^2 d^3 \TieDensity^2 \sigma^2 \Diam^2}{\MinProb^2},
\frac{\ARMNUM^2 d^3 \sigma^2 \Diam^2}{\TieCutoff^2}
\right) \right)
\\ & = O\left( \ARMNUM \exp \left(\frac{2}{\MinProb}\right) \right).
\end{align*}

It remains to show that
$\sqrt{0.6 n \cdot \log (\log_{1.1}(n) + 1) + \frac{n x^2}{16 \sigma^2}}
\leq \frac{n x}{2 \sigma}$
for all $n \geq \max(2, \frac{30 \sigma^3}{x^3})$.
We first use that $\sqrt{\cdot}$ is sublinear to bound
\[
  \sqrt{0.6 n \cdot \log (\log_{1.1}(n) + 1) + \frac{n x^2}{16 \sigma^2}}
\; \leq \; \sqrt{0.6 n \cdot \log (\log_{1.1}(n) + 1)} + \frac{n x}{4 \sigma}
\]
(here we use the fact $\sqrt{n} \leq n$).
Next, we show that
$\sqrt{0.6 n \log (\log_{1.1}(n) + 1)} \leq \frac{n x}{4 \sigma}$;
then, adding the two terms gives the desired bound.

By squaring the claimed statement and rearranging,
it is equivalent to show that
$\frac{n}{\log(\log_{1.1}(n)+1)} \geq \frac{9.6\sigma^2}{x^2}$.
Because $n \geq 2$,
a numerical calculation and derivative test shows that
$\frac{n}{\log(\log_{1.1}(n)+1)} \geq n^{2/3}$,
and because $n \geq \frac{30 \sigma^3}{x^3}$,
we get that
$n^{2/3} \geq \frac{9.6 \sigma^2}{x^2}$, completing the proof.
\end{proof}

It would be desirable to simply identify a constant upper bound on the
payment made each time.
Unfortunately, while the agent types are drawn from a bounded support,
the noise in arm locations is not;
hence, with small probability, arm locations may be grossly
misestimated, resulting in high incentive payments.
As a result, the actual analysis of the total payment is significantly more
intricate;
the proof of Theorem~\ref{rst:budget} is given in
Appendix~\ref{sec:payment-proof}.


\begin{algorithm} [t]
\begin{algorithmic}[1]
\FOR{$t=1, \ldots, T$}
			\STATE Define $x(l) = \Pi^\Phi_{K_l}(x_{t-1})$, increase $l$ from $l=f_t(v_t)$, until $\norm{\nabla \Phi(x(l)) - \nabla \Phi(x_{t-1})}_* = \eta \norm{\nabla f_t (x(l))}_*$. 
			\label{projection-step-regret}
			\STATE $x_t =  x(l)$.
			\ENDFOR
\end{algorithmic}
\caption{(Dual) \ouralg}
\label{alg: online-projection-regret}
\end{algorithm}

%We now move from competitive ratio to regret, specifically dynamic regret.  
While the online algorithms community typically focuses on competitive ratio, regret is typically the focus of the online learning community. The difference in performance metrics leads to differences in the settings considered.  In the previous section, we studied locally polyhedral cost functions, while here we focus on cost functions that are continuously differentiable and have a minimizer $v_t$ in the interior of the feasible set $\mathcal{X}$.\footnote{Any convex function can be approximated by a convex functions with these properties, e.g., see \cite{nesterov2005}.% and \cite{wright1999}.
}  

Interestingly, it is has been shown that the change in metric from competitive ratio to regret has a fundamental impact on the type of algorithms that perform well. Concretely, it has been shown that no single algorithm can perform well across (static) regret and competitive ratio \citep{andrew2013}.  Consequently, it is not surprising that we find a different choice of balance in \ourack\ is needed to obtain the no-regret performance guarantees.  Specifically, in contrast to the results of the previous section, which focus on a form of balance in the primal setting, in this section we focus on balance in the dual setting, where we compare costs as measured in the dual norm, $\norm{x}_* = \max_{\norm{z} \le 1} \langle z, x \rangle$. 

We show that choosing  $l$ to balance between the switching cost in the dual space and the size of the gradient leads to an online algorithm with small dynamic regret. It is worth emphasizing that, in contrast to the results of the previous section, we balance the switching cost against the marginal hitting cost $\norm{\nabla f_t(x_t)}_*$ instead of $f_t(x_t)$. A formal description of the instantiation of \ourack\ for regret is given in Algorithm \ref{alg: online-projection-regret}, which can be implemented efficiently via bisection (Lemma \ref{lem: continuity-ratio}).% implies that the right balance can be computed efficiently via bisection when implementing Step \ref{projection-step-regret}. Our main result in this section bounds the dynamic regret of Algorithm \ref{alg: online-projection-regret}. 

\begin{theorem}
Consider $\Phi$ that is an $m$-strongly convex function in $\norm{\cdot}$ with $\norm{\nabla \Phi(x)}_*$ bounded above by $G$ and $\nabla \Phi(0) = 0$. Then the $L$-constrained dynamic regret of Algorithm \ref{alg: online-projection-regret} is $\leq \frac{GL}{\eta}  + \frac{T\eta}{2m}.$
\label{thm: online-projection-regret}
\end{theorem} 

While the result above does not depend on knowing the parameters of the instance, if we know the parameters $T$, $D$ and $L$ ahead of time then we can optimize the balance parameter $\eta$ as follows.

\begin{corollary}\label{cor: online-projection-regret}
When $\eta =\sqrt{\frac{2GLm}{T}}$,  Algorithm \ref{alg: online-projection-regret} has $L$-constrained dynamic regret $\leq \sqrt{\frac{2GLT}{m}}.$ 
\end{corollary}

One interesting aspect of this result is that it has a form similar to the dynamic regret bound on OGD in Theorem 2 of \cite{zinkevich2003}.  Both are independent of the dimension of the decision space $d$, assuming the diameter of the space is normalized to $1$. The key difference is that the bound in Corollary \ref{cor: online-projection-regret} is \emph{independent of the size of the gradients of the cost functions}, unlike in the case of OGD. This can be viewed as a significant benefit that results from the fact that \ourack\ steps in a direction normal to where it lands, rather than where it starts.  

Finally, note that Theorem \ref{thm: online-projection-regret} and Corollary \ref{cor: online-projection-regret} additionally provide bounds on the static regret of \ourack, by setting $L=D$. In this case, Corollary \ref{cor: online-projection-regret} gives a bound of $O(\sqrt{T})$ which matches the lower bound  in the setting where there are no switching costs \citep{hazan2016}.

%Note that we always have $L \leq DT$, since the offline can move at most $D$ each step. With this choice of $L$, the time-averaged dynamic regret is at most $\sqrt{\frac{2GD}{m}}$. Furthermore, when the offline optimal is static, i.e., $x_1^* = x_2^* = \ldots = x_T$, then $L = \norm{x^* - x_0} \le D$, hence the regret is upper bounded by $D\sqrt{2T}$. \gautam{does this imply the static regret is sublinear??} \niangjun{Yes, but remember we have one-step lookahead advantage here, that's why I don't want to directly compare with regret.} \gautam{Can you explain why it implies the static regret is sublinear? It seems to me this only holds if the dynamic optimal happens to be static, which is very rarely the case.}\niangjun{the definition of static regret is comparing against the optimal that is static}

%Now let us end this section with a proof of Theorem \ref{thm: online-projection-regret}.

%
%
%\begin{align*}
%\norm{x_t - x_t^*}^2 &= \norm{ x_{t-1} - x_t^* - \eta \nabla_t}^2 \\
%& = \norm{x_{t-1} - x_t^*}^2 - 2\eta \langle \nabla_t, x_{t-1} - x_t^* \rangle + \eta^2 \norm{\nabla_t}^2,
%\end{align*}
%rearranging, we have 
%\begin{align}
%\langle \nabla_t, x_{t-1} - x_t^* \rangle = \frac{1}{2\eta} (\norm{x_{t-1} - x_t^*}^2 - \norm{x_t - x_t^*}^2) + \frac{\eta}{2} \norm{\nabla_t}^2. 
%\label{eqn: grad-ineq}
%\end{align}
%Substitute \eqref{eqn: grad-ineq} into \eqref{eqn: per-step-diff} and summing over $t$, we have 
%\begin{align*}
%\sum_{t=1}^T f_t(x_t) - f_t(x_t^*)  & \le  \frac{1}{2\eta} \sumt ( \norm{x_t - x_t^*}^2 - \norm{x_{t-1} - x_t^*}^2 ) - \frac{\eta}{2}\norm{\nabla_t}^2 \\
%& \le \frac{1}{2\eta}  ( \norm{x_0}^2 - \norm{x_T}^2)  + \sumt  \frac{1}{\eta} \langle x_t^*, x_t - x_{t-1} \rangle - \frac{\eta}{2}\norm{\nabla_t}^2 \\
%& \overset{(a)}{\le} \sumt   \frac{1}{\eta}\langle x_t, x_t^* - x_{t+1}^*\rangle - \frac{\eta}{2} \norm{\nabla_t}^2\\
%& \overset{(b)}{\le}  \sumt \frac{1}{\eta} \norm{x_t}\norm{x_t^* - x_{t+1}^*} - \frac{\eta}{2} \norm{\nabla_t}^2\\
%& \overset{(c)}{\le} \frac{ D L}{\eta} - \sumt \frac{\eta}{2}\norm{\nabla_t}^2,
%\end{align*}
%where $(a)$ is because $x_0 = 0$ and denoting $x_{t+1}^* = 0$, $(b)$ is due to Cauchy-Schwarz inequality, and $(c)$ is because of the assumption $\norm{x_t} \le D$ for all $x_t \in \mathcal{X}$ and the definition of $L$. 
%
%Therefore, the dynamic regret of Algorithm \ref{alg: online-projection-regret} can be upper bounded by:
%\begin{align*}
%&\sumt f_t(x_t) + \norm{x_t - x_{t-1}} -  ( f_t(x_t^*) + \norm{x_t - x_{t-1}}) \\
%\le &L \left( \frac{D}{\eta} - 1\right) + \eta \sumt (  \norm{\nabla_t} - \frac{1}{2} \norm{\nabla_t}^2) \\
% \le &L\left(\frac{D}{\eta} - 1 \right) + \frac{T\eta}{2} - \frac{\eta}{2} \sumt (\norm{\nabla_t} -1 )^2, 
%\end{align*}
%throwing away the negative terms completes the proof. 





\subsection{Tighter Bounds Under Additional Assumptions}
%: Discrete Preferences and Known \MinProb}

The proofs of Theorem~\ref{rst:budget} and Theorem~\ref{rst:regret}
incur exponential (in \ARMNUM) payment and regret in the initial
phases because the threshold $1/\log(s)$ required for incentivization
decreases slowly. 
This slow decrease is needed to bound the regret in the later phases
when the concentration inequality kicks in as in
Equation~(\ref{equ:small_regret_bound}).
In this section, we discuss two restrictions under which we can
modify the algorithm slightly and provide stronger bounds,
avoiding this exponential dependence.
Both problem settings are special cases of the more general
setting previously considered. 

\subsubsection{Discrete Preferences}
\label{subsec:discrete}
Discrete preferences by agents are captured by the following
strengthening of Assumption~\ref{A1},
which states that the agents who are close to tied between
two arms have measure 0:

\begin{assumption}[Discrete Preferences]
\label{a:discrete}
There is a positive $\hat{z}$ such that
$\AlmostTied{\hat{z}} = 0$.
\end{assumption}

When Assumption~\ref{a:discrete} holds,
we restrict the payment-eligibility criterion by only incentivizing
arms with much smaller probability to be pulled:
an arm $i$ is payment-eligible at time $t$ in phase $s$ when both of
the following are true: 

\begin{itemize}
\item $i$ has been pulled at most $s$ times up to time $t$,
i.e., $\NumPull{t}{i} \leq s$.
\item Based on the current estimates \ArmEV{t}{i'} of all arms' attribute vectors,
the probability of pulling arm $i$ is less than $1/s$
(compared to $1/\log(s)$ in the general algorithm).
\end{itemize}

We refer to this modified version of
Algorithm~\ref{alg:basic-incentivizing} as the
\emph{Discrete-Preference Algorithm.}
We outline a proof of the following result for this algorithm under
Assumption~\ref{a:discrete} in Appendix~\ref{sec:discussion-proof1}.

\begin{theorem}
\label{rst:discrete}
Under Assumption~\ref{a:discrete}, the Discrete-Preference Algorithm
has expected payment bounded above by 
$O \left(\ARMNUM^2/\MinProb \right)$
and expected regret bounded above by $O(\ARMNUM / \MinProb)$.
\end{theorem}

\subsubsection{Known \MinProb}
An alternative useful assumption is that \MinProb
(or a strictly positive lower bound on \MinProb) is known. 

\begin{assumption}[Known $\MinProb$]
\label{a:known-p}
A strictly positive lower bound on \MinProb is known in advance.
\end{assumption}

When this assumption holds, we choose a different modification in the
definition of payment eligibility.
Let $s_0 = \exp(2/\MinProb)$.
An arm $i$ is \emph{payment-eligible} at time $t$ (in phase $s$)
if both:
\begin{itemize}
\item $i$ has been pulled at most
$s$ times up to time $t$, i.e., $\NumPull{t}{i} \leq s$.
\item Based on the current estimates \ArmEV{t}{i'} of all arms' attribute vectors,
the probability of pulling arm $i$ is less than $1/\log(s+s_0)$.
\end{itemize}

This knowledge of \MinProb allows the algorithm to incentivize
significantly fewer arm pulls.
We refer to this modified version of
Algorithm~\ref{alg:basic-incentivizing} as the \emph{Known-\MinProb
Algorithm.}
We outline a proof of the following result for this algorithm under
Assumption~\ref{a:known-p} in Appendix~\ref{sec:discussion-proof2}.

\begin{theorem}
\label{rst:known-p}
Under Assumption~\ref{a:known-p}, the Known-\MinProb Algorithm has
expected payment  bounded above by  
$O(\ARMNUM^2 \cdot \max(1,(\TieDensity/\MinProb)^{5/2}))$ 
and expected regret bounded above by
%\begin{align*}
$O \left( \frac{\ARMNUM^2}{\MinProb^2}
%  + \frac{\ARMNUM^2}{\hat{z}^2}+\frac{\ARMNUM}{\hat{z}^3}
  +\frac{\ARMNUM \TieDensity \log^3(T)}{\MinProb}\right)$.
%  \end{align*}
\end{theorem}



\section{Conclusion}
We study the problem of incentivizing exploration with heterogeneous
user preferences.
We proposed an algorithm that achieves expected cumulative regret
$O(\ARMNUM \e^{2/\MinProb} + \ARMNUM \log^3(T))$,
using expected cumulative payments of $O(\ARMNUM^2 \e^{2/\MinProb})$.
It is possible to improve these bounds to polynomial (in \ARMNUM and
$1/\MinProb$) when \MinProb is known or the preference distribution is
discrete.
In fact, we conjecture that this should be possible even in the full
generality of our model.
As a first step towards such a polynomial bound, we can obtain an exponential dependence on
$1/(\MinProb \ARMNUM)$ by changing the probability threshold to be $\frac{1}{\ARMNUM\log(s)}$
\footnote{This will lead to a different dependence on $\ARMNUM$
in the regret bound as well as the payment bound}, which gives polynomial dependence unless some
arm has a much smaller fraction of the population preferring it.

Taking this goal one step further, we would like to 
develop algorithms that do not require all arms to be preferred by a
strictly positive fraction of agents.
An alternate algorithm might only incentivize an arm if its estimated
attribute vector is close enough to a Pareto frontier.
The regret will then be $\Omega(\log(T))$ when at least one arm falls
below the Pareto frontier, as we no longer have free exploration of
all arms. 
It is likely that a bound will deteriorate as the number of such
unpreferred arms increases.

Finally, it would be desirable to generalize to utility
functions beyond inner products.
We believe that similar results hold for arbitrary
Lipschitz-continuous utility functions of the arm's attribute vector,
and that only minor modifications are necessary to the algorithm and
proofs.


\acks{PF and BC were partially supported by NSF CMMI-1254298 and AFOSR
FA9550-15-1-0038.
DK was supported in part by NSF grant 1423618.}

\bibliography{reference}

\appendix

\section{A Lower Bound of $\Omega(\log(T))$} \label{sec:lb}

We saw in Theorem~\ref{rst:discrete} that 
when agent preferences for their best arm are sufficiently clear,
in the sense that $\TieDensity = 0$, the regret of
Algorithm~\ref{alg:basic-incentivizing} is bounded by a constant.
One may conjecture that this should hold more generally,
in that the regret of the (fewer and fewer) agents on the boundary
between close arms should go to 0, while their fraction also goes to 0.
In this section, we establish a lower bound,
showing that even in very simple settings, a (logarithmic) dependence
on $T$ is typically unavoidable for \emph{any} incentivization strategy.

We consider an instance with two arms,
whose attribute vectors are $(0,0)$ and $(0,1)$, respectively.
Agent types are distributed uniformly on the (edge of the)
two-dimensional unit square%
\footnote{While our model technically assumed that all agent coordinates are
non-negative, we could simply shift the unit square.
The present choice is solely for ease of notation.}  
$\Set{\AgV}{\max(|\Ag{1}|, |\Ag{2}|) = 1}$,
with density $\frac{1}{8}$.
Because $\AgV \cdot (0,0) = 0$ and $\AgV \cdot (0,1) = \Ag{2}$,
the best choice for agent \AgV is arm $(0,1)$ iff $\Ag{2} > 0$,
i.e., the top half of the unit square prefers the arm $(0,1)$,
and the bottom half prefers the arm $(0,0)$.

Since we are proving a lower bound, we give the algorithm the
following extra two advantages:
(1) there is no noise in the observations of the arm $(0,0)$,
and all agents know its location deterministically.
(2) in each time step, regardless of which arm is pulled, the algorithm
and all agents observe a pull from arm $(0,1)$.
For simplicity of notation, we set the standard deviation of the arm
$(0,1)$ to $\sigma = 1$;
different values only lead to a scaling of the results.

Under these advantages, myopic play is clearly optimal, so it suffices
to bound the regret of the myopic algorithm which never incentivizes
agents. 

Because a pull of arm $(0,1)$ is observed in each time step,
after $t$ rounds, the estimate \ArmEV{t}{(0,1)} is of the form
$(0,1) + (\Noise[t]{1}, \Noise[t]{2})$,
where $\Noise[t]{1} \sim \Normal{0}{\sqrt{1/t}}$
and $\Noise[t]{2} \sim \Normal{0}{\sqrt{1/t}})$.
We lower-bound the regret in step $t$ by focusing on the event that
both normal noise coordinates are non-negative,
which by symmetry has probability \quarter:

\begin{align*}
\Expect{\Regret{t}} 
  & \geq \ExpectC{\Regret{t}}{\Noise[t]{1} > 0, \Noise[t]{2} > 0}
         \cdot \Prob{\Noise[t]{1} > 0, \Noise[t]{2} > 0}
  \; = \; \quarter \ExpectC{\Regret{t}}{\Noise[t]{1} > 0, \Noise[t]{2} > 0}.
\end{align*}

For the moment, focus on some time step $t$,
and write $\NoiseV = \NoiseV[t]$.
Then, there are two types of agents who pull the wrong arm and incur
regret:
\begin{enumerate}
\item If $\Ag{2} > 0$ and $\Ag{1} \Noise{1} + \Ag{2} (1+\Noise{2}) < 0$
then \AgV should pull $(0,1)$,
but will wrongly pull $(0,0)$ and incur regret \Ag{2}.
The range of \AgV making the wrong choice is thus
$0 < \Ag{2} < \frac{- \Noise{1}}{1+\Noise{2}} \cdot \Ag{1}$.
Since we are proving a lower bound, we only focus on the
case $\Ag{1} = -1$, and ignore the case $\Ag{2} = 1$
(which is rare, since it requires \Noise{1} to be large).
Thus, the set of agents incurring regret contains the set
$\Set{\AgV = (-1, \Ag{2})}{0 < \Ag{2} < \frac{\Noise{1}}{1+\Noise{2}}}$.

% For any particular value of \Ag{2},
% the length of the interval of corresponding \Ag{1} is
% \begin{align*}
%   \sqrt{1-\Ag{2}^2} - \frac{1+\Noise{2}}{\Noise{1}} \cdot \Ag{2}
%   & \geq 1 - \Ag{2} - \frac{1+\Noise{2}}{\Noise{1}} \cdot \Ag{2}
%   \; = \; 1 - \frac{1+\Noise{1}+\Noise{2}}{\Noise{1}} \cdot \Ag{2}.
% \end{align*}

\item If $\Ag{2} < 0$ and $\Ag{1} \Noise{1} + \Ag{2} (1+\Noise{2}) > 0$,
then \AgV should pull $(0,0)$,
but will wrongly pull $(0,1)$ and incur regret $-\Ag{2}$.
This region and its regret are rotationally symmetric to the previous
case.
\end{enumerate}

Hence, the expected regret for given \Noise{1}, \Noise{2} is at least
%\begin{align*}
$2 \int_0^{\frac{\Noise{1}}{1+\Noise{2}}}
  \Ag{2} \cdot \frac{1}{8} \dd \Ag{2}
=
\frac{1}{8} \cdot \left(\frac{\Noise{1}}{1+\Noise{2}}\right)^2$.
%\end{align*}
The expected regret, conditioned on
$\Noise{1} > 0$ and $\Noise{2} > 0$, is therefore

\begin{align*}
\ExpectC{\Regret{t}}{\Noise{1} > 0, \Noise{2} > 0}
 & \geq \frac{1}{\Prob{\Noise{1}>0,\Noise{2}>0}} \cdot \frac{1}{8}
    \int_{0}^{\infty} \int_{0}^{\infty}
    \left( \frac{\Noise{1}}{1+\Noise{2}} \right)^2 \cdot
    \frac{\e^{-\frac{t \Noise{1}^2}{2}} \sqrt{t}}{\sqrt{2\pi}} \dd\Noise{1}
    \frac{\e^{-\frac{t \Noise{2}^2}{2}} \sqrt{t}}{\sqrt{2\pi}} \dd\Noise{2} \\
 & = \frac{1}{4\pi} \int_{0}^{\infty} \int_{0}^{\infty}
   \left( \frac{\Noise{1}}{\sqrt{t}+\Noise{2}} \right)^2
   \e^{-\Noise{1}^2/2} \e^{-\Noise{2}^2/2} \dd \Noise{1} \dd \Noise{2}.
\end{align*}

We want to show that the expected regret per time step decreases only
at a rate of $\Omega(1/t)$, and thereto consider the limit of the
ratio of the two quantities:

\begin{align*}
\lim_{t \to \infty} \frac{\ExpectC{\Regret{t}}{\Noise{1} > 0, \Noise{2} > 0}}{\frac{1}{t}}
  & \geq
    \lim_{t \to \infty} \left( \frac{1}{4\pi} \cdot
    \int_{0}^{\infty} \int_{0}^{\infty}
    t \cdot \left( \frac{\Noise{1}}{\sqrt{t}+\Noise{2}} \right)^2
    \e^{-\Noise{1}^2/2} \e^{-\Noise{2}^2/2} \dd \Noise{1} \dd \Noise{2} \right) \\
  & = \frac{1}{4\pi} \cdot \int_{0}^{\infty} \int_{0}^{\infty}
    \lim_{t \to \infty}
    \left( t \cdot \left( \frac{\Noise{1}}{\sqrt{t}+\Noise{2}} \right)^2 \right)
    \e^{-\Noise{1}^2/2} \e^{-\Noise{2}^2/2} \dd \Noise{1} \dd \Noise{2} \\
  & = \frac{1}{4\pi} \cdot \int_{0}^{\infty} \int_{0}^{\infty}
    \e^{-\Noise{1}^2/2} \e^{-\Noise{2}^2/2} \dd \Noise{1} \dd \Noise{2}
  \; = \; \frac{1}{4 \pi}; 
\end{align*}

the second line is due to the monotone convergence theorem,
which can be applied because
$t \left(\frac{\Noise{1}}{\sqrt{t}+\Noise{2}}\right)^2$
is strictly increasing in $t$.
Because the expected regret in each time step is at least
$\Omega(1/t)$, the total expected regret is at least
$\Omega(\sum_{t=1}^{T}\frac{1}{t}) = \Omega(\log(T))$.


\section{Proof of Lemma~\ref{lem:phase-length}}
\label{sec:lemma4-proof}

\begin{rlemma}{Lemma}{\ref{lem:phase-length}}
For any $s\geq 3$, the expected length of phase $s$ is at most
$\ARMNUM \cdot \log(s)$ time steps.

More generally, for any set of types $A$, the expected number of times that 
an agent with a type in $A$ appears in a phase $s$ is at most 
$f(A) \cdot \ARMNUM \cdot \log(s)$,
where $f(A) := \Prob[\AgV \sim \AgentDist]{\AgV \in A}$.
\end{rlemma}

\begin{proof}
We show the second claim for general $A$,
which implies the first claim by taking $A$ to be the support of $\AgentDist$.

Fix a phase $s$, and let $S_t$ be the set of arms that have been
pulled at most $s$ times by time $t$.
Let $\tau_m$ be the maximum of $t_s$ (the start of phase $s$)
and the time $t$ when $S_t$ first contains (at most) $m$ elements.
The phase ends when $S_t$ contains no elements, i.e., $\tau_0 = t_{s+1}$.
We consider the expected number of pulls during times
$t \in [\tau_m, \tau_{m-1})$ made by agents whose types are in $A$. 

Consider a counter starting from $0$ at time $\tau_m$ that increments
each time an agent arrives with a type in $A$,
and that is stopped the first time an arm in $S_t$ is pulled.
For any stopping time $t \in [\tau_m, \tau_{m-1})$, 
the conditional probability that the counter increments,
given the history of pulls, payments, and observations at time $t$,
is $f(A)$.
The conditional probability that the counter is stopped by that
agent's pull is at least $1/\log(s)$.
To see this, consider two cases.
In the first case, there is at least one payment-eligible arm,
in which case the principal incentivizes an arm in $S_t$.
In the second case, there are no payment-eligible arms.
Then, by definition of payment-eligibility,
each arm $i \in S_t$ has conditional probability at least $1/\log(s)$ of being pulled.

The expected value of this counter when it is stopped is bounded above
by the expected stopped value of another counter whose conditional
probability of being stopped is exactly $1/\log(s)$.
(This can be shown more formally via a coupling argument.) 
Let $V_\ell$ be the conditional expectation of this second counter's value when it is stopped,
given that its current value is $\ell$ and it has not been stopped.
Observe that $V_\ell = \ell + V_0$.
Enumerating outcomes
(either the counter increments or not; either it stops or it continues)
gives us the recurrence relation
$V_0 = f(A) \frac{1}{\log(s)} + f(A) (1-\frac{1}{\log(s)}) V_1
   + (1-f(A))(1-\frac{1}{\log(s)}) V_0$.
Using that $V_1 = 1 + V_0$, this equation simplifies to 
$V_0 = f(A) + (1-\frac{1}{\log(s)}) V_0$.
Noting that $V_0$ is finite and solving for $V_0$ gives us that
$V_0 = f(A) \log(s)$.
Thus, the conditional expected number of pulls (given the history at time $\tau_m$)
by an agent with a type in $A$ during times in $[\tau_m, \tau_{m-1})$
is no more than $f(A) \log(s)$.

We may write the total number of pulls by agents with types in $A$
during phase $s$ as a sum of this quantity over $m$ ranging from
$|S_{t_s}|$ down to 1.
Noting that $|S_{t_s}| \leq \ARMNUM$ and taking the expectation of
this sum completes the proof. 
\end{proof}







\section{Proof of Lemma~\ref{lem:round-prob}}
\label{sec:lemma5-proof}

\begin{rlemma}{Lemma}{\ref{lem:round-prob}}
Recall the noise is a mean-zero sub-Gaussian($\sigma^2$) random variable.
Let \LatePhase be a phase cutoff, 
and let $x_n, x'_n > 0$ be functions satisfying that
$\sqrt{0.6 n \cdot \log (\log_{1.1}(n) + 1) + \frac{n x_n^2}{16 \sigma^2}}
\leq \frac{n x'_n}{2 \sigma}$,
for all $n \geq \LatePhase$.
Let $\tau_s$ be a stopping time
(which may depend on the entire past history)
which is almost surely in phase $s$,
i.e., satisfying $\tau \in [t_s, t_{s+1})$ almost surely.

Then, for any arm $i$, attribute $j$, and phase $s \geq \LatePhase$,
we have that
$\Prob{\AccE{\tau_s}{i}{j}{x'_s}}
\geq 1 - 24 \exp\left(-\frac{1.8 s x_s^2}{16 \sigma^2} \right)$.
\end{rlemma}

The proof of Lemma~\ref{lem:round-prob} is based on an adaptive
concentration inequality due to \cite{zhao2016adaptive},
given as Lemma~\ref{lem:ACI-inequality}.

\begin{lemma}[Corollary 1 of \cite{zhao2016adaptive}]
\label{lem:ACI-inequality}
Let $X_i$ be zero-mean $1/2$-subgaussian random variables,
and $\SET{S_n = \sum_{i=1}^n X_i, n \geq 1}$ the corresponding random walk.
Let $J$ be any stopping time with respect to $\SET{X_1, X_2, \ldots}$.
(We allow $J$ to take the value $\infty$,
defining $\Prob{J = \infty} = 1 - \lim_{n \to \infty} \Prob{J \leq n}$.)
Define 
$g(n)  = \sqrt{0.6 n \cdot \log (\log_{1.1}(n) + 1) + n \cdot b}$.

Then, 
$\Prob{J < \infty \mbox{ and } S_J \geq g(J)} \leq 12 \e^{-1.8 b}$.
\end{lemma}

\begin{extraproof}{Lemma~\ref{lem:round-prob}}
Fix an arm $i$ and attribute $j$.
By the assumptions of the lemma,
the stopping time $\tau_s$ is such that almost surely,
each arm --- and in particular arm $i$ --- has been pulled at least
$s \geq \LatePhase$ times at time $\tau_s$.
Define $J$ to be the number of times that $i$ has been pulled at
time $\tau_s$.
For any $n \geq 1$, let $k_n$ be the time step right after arm $i$ has
been pulled for the \Kth{n} time. Define
$S_n := \frac{n \cdot (\ArmE{k_n}{i}{j} - \Arm{i}{j})}{2 \sigma}$
to be the sum of all attribute-$j$ noise components up to and
including the \Kth{n} pull of arm $i$,
renormalized to be a mean-zero sub-Gaussian($1/2$) random variable.
The $S_n$ define an unbiased half-subgaussian random walk,
and we can therefore apply Lemma~\ref{lem:ACI-inequality} to them and
the stopping time $J$.
Specifically, we set $b = \frac{x_J^2}{16 \sigma^2}$,
and obtain that

\begin{align*}
\Prob{J < \infty \mbox{ and } S_J \geq 
\sqrt{0.6 J \cdot \log (\log_{1.1}(J) + 1) + \frac{J x_J^2}{16 \sigma^2}}}
& \leq 12 \exp \left( \frac{-1.8 J x_J^2}{16 \sigma^2} \right).
\end{align*}

Applying Lemma~\ref{lem:ACI-inequality} to $-S_n$ with the same choice
of $b$, and taking a union bound over both cases, we obtain that

\begin{align*}
\Prob{J < \infty \mbox{ and } |S_J| \geq 
\sqrt{0.6 J \cdot \log (\log_{1.1}(J) + 1) + \frac{J x_J^2}{16 \sigma^2}}}
& \leq 24 \exp \left( \frac{-1.8 J x_J^2}{16 \sigma^2} \right).
\end{align*}

Because $J$ will be finite with probability 1, we can drop the
$J < \infty$ part of the event:
\begin{align*}
& \Prob{S_J \geq 
\sqrt{0.6 J \cdot \log (\log_{1.1}(J) + 1) + \frac{J^2 x_J^2}{16
    \sigma^2}}}\\
= & \Prob{J < \infty \mbox{ and } S_J \geq 
\sqrt{0.6 J \cdot \log (\log_{1.1}(J) + 1) + \frac{J^2 x_J^2}{16
    \sigma^2}}}.
\end{align*}

In the high-probability case, we now apply the assumed inequality
between $x_J$ and $x'_J$, to obtain that
\[
|S_J| \; \leq \;
\sqrt{0.6 J \cdot \log (\log_{1.1}(J) + 1) + \frac{J x_J^2}{16 \sigma^2}}
\; \leq \; \frac{J x'_J}{2 \sigma}.
\]

Substituting the definition of $S_J$ and canceling common terms,
the inequality implies that
$|\ArmE{k_J}{i}{j} - \Arm{i}{j}| \leq x'_J$.
The choice of $J$ ensures that
$\ArmE{k_J}{i}{j} = \ArmE{\tau_s}{i}{j}$,
and we have thus shown that
$|\ArmE{\tau_s}{i}{j} - \Arm{i}{j}| \leq x'_s$.
\end{extraproof}




\section{Proof of Lemma~\ref{lem:right-choice}}
\label{sec:lemma7-proof}

\begin{rlemma}{Lemma}{\ref{lem:right-choice}}
Let $x > 0$ be arbitrary.
When \AccEU{t}{x} happens,
no agent \AgV will pull a highly suboptimal arm, i.e., an arm $i$ with 
$\AgV \cdot (\ArmV{\Best{\AgV}} - \ArmV{i}) > 2\Diam d x$.
\end{rlemma}

\begin{proof}
By definition, when \AccEU{t}{x} happens,
for all arms $i$ and attributes $j$,
all arm attribute estimates are accurate to within $x$,
in the sense that
$|\ArmE{t}{i}{j} - \Arm{i}{j}| \leq x$.

Consider any agent type \AgV.
Let $i \neq \Best{\AgV}$ be any arm
with much smaller true reward than the best arm:
$\AgV \cdot (\ArmV{\Best{\AgV}} - \ArmV{i}) > 2\Diam d x$.
Because each coordinate of \ArmEV{t}{\Best{\AgV}} and of
\ArmEV{t}{i} is estimated accurately to within $x$, 
we get that 
$\AgV \cdot (\ArmEV{t}{\Best{\AgV}} - \ArmV{\Best{\AgV}})
\geq - \Diam d x$
and
$\AgV \cdot (\ArmEV{t}{i} - \ArmV{i}) \leq \Diam d x$.
Hence, 

\begin{align}
\AgV \cdot (\ArmEV{t}{\Best{\AgV}} - \ArmEV{t}{i})
& =
\AgV \cdot (\ArmEV{t}{\Best{\AgV}} - \ArmV{\Best{\AgV}})
+ \AgV \cdot (\ArmV{\Best{\AgV}} - \ArmV{i})
+ \AgV \cdot (\ArmV{i} - \ArmEV{t}{i}) \nonumber\\
& > -\Diam d x + 2 \Diam d x - \Diam d x
\; = \; 0, \label{equ:choice}
\end{align}
which means that the agent with type \AgV will not pull arm $i$.
\end{proof}


\section{Proof of Lemma~\ref{lem:no-incentives}}
\label{sec:lemma8-proof}

\begin{rlemma}{Lemma}{\ref{lem:no-incentives}}
Fix an arm $i$.
Let $s \geq \exp(2/\MinProb)$, and let $\tau_s$ be the (random)
time when arm $i$ is pulled for the \Kth{s} time.
Let $\hat{x} = \frac{1}{2\Diam d}
  \cdot \min(\TieCutoff, \frac{\MinProb}{2\TieDensity})$.
Under \AccEU{\tau_s}{\hat{x}},
this pull of arm $i$ is not incentivized.
\end{rlemma}

\begin{proof}
By Lemma~\ref{lem:right-choice},
under the event \AccEU{\tau_s}{\hat{x}},
all agent types \AgV with
$\AgV \cdot (\ArmV{\Best{\AgV}} - \ArmV{\Second{\AgV}})
> 2\Diam d \hat{x}$
will pull their best arm \Best{\AgV}.
Notice that $2\Diam d \hat{x}
= \min(\TieCutoff, \frac{\MinProb}{2 \TieDensity})$

By Assumption~\ref{A1},
the measure of agents (across all arms)
whose best and second-best arm differ in utility by less
than $\min(\TieCutoff, \frac{\MinProb}{2 \TieDensity})$
is at most
$\TieDensity \cdot \min(\TieCutoff, \frac{\MinProb}{2 \TieDensity})
\leq \frac{\MinProb}{2}$.
In particular, this bound holds for agents whose best arm is $i$.
By Assumption~\ref{A3}, at least a measure \MinProb of agents has $i$
as their best arm, and thus, at least a measure $\frac{\MinProb}{2}$
will myopically pull arm $i$.
Because $1/\log(s) \leq \frac{\MinProb}{2}$ for
$s \geq \exp(2/\MinProb)$, arm $i$ is not payment-eligible at time
$\tau_s$.
\end{proof}


\section{Proof of Theorem~\ref{rst:budget}}
\label{sec:payment-proof}

We restate the theorem here for convenience:

\begin{rtheorem}{Theorem}{\ref{rst:budget}}
The expected total payment of
Algorithm~\ref{alg:basic-incentivizing} is at most
$O \left(\ARMNUM^2 \cdot \e^{2/\MinProb} \right)$.
%The payment budget for Algorithm~\ref{alg:basic-incentivizing} is bounded above by 
%\begin{align}
% O \left(\ARMNUM^2 \MAXR \Diam^2 \TieDensity d^3 \sigma\cdot \exp\left(\frac{2}{\MinProb}\right) \right).  
%\pfedit{O \left(\ARMNUM^2  d \cdot (\MAXR + \Diam d \sigma)\cdot \max\bigg\{\exp\left(\frac{2}{p}\right),
%\left( \frac{\sigma \Diam d}{\TieCutoff} \right)^{5/2},
%\left( \frac{\sigma \TieDensity \Diam d}{\MinProb} \right)^{5/2}\bigg\}
%\right).}\nonumber 
%\end{align}
\end{rtheorem}

The high-level idea of the proof is motivated by Lemma~\ref{lem:numP},
which shows that the expected \emph{number} of payments is constant.
Unfortunately, in contrast to the regret, there is no hard upper bound
on the payment in any one round.
If a draw of a particular arm comes out wildly inaccurate --- which is
an event of low but positive probability ---
then agents may require very large incentives to pull this arm again
in the future (and correct the inaccurate estimate).
The high payments are offset by the exceedingly low probability of
having to incur them, but a rigorous analysis requires some care:
if a high payment is required in one phase, this indicates a very
inaccurate estimate, which may require multiple phases to correct.
Hence, we need to handle dependency of payments across time steps and
phases.

To reason about such estimation errors formally, we define
\emph{envelopes} of sample paths.
A sample path \SP captures all the random events affecting the
algorithm, i.e., the random draws \AgentV{t} of agents and the
noise \NoiseV[t] in the draws of the pulled arms,
with an infinite time horizon.

With foresight, we define $g(s, \ell) := \frac{12 \sigma \ell}{s^{2/5}}$.
Let $\ErrV{t}{i} = \ArmEV{t}{i} - \ArmV{i}$ be the estimation
error for the attribute vector \ArmV{i} at time $t$,
with components \Err{t}{i}{j}.
For any sample path \SP, let $s (t,\SP)$ be the phase number that the
algorithm is in at time $t$ with the sample path \SP.
Define the sets

\begin{align*}
\hat{\mathcal{L}}_\ell
  & = \Set{\SP}{|\Err{t}{i}{j}(\SP)| \leq g(s(t,\SP),\ell)
    \mbox{ for all } i,j,t},\\
\Env{1} & = \hat{\mathcal{L}}_1,\\ 
\Env{\ell} & = \hat{\mathcal{L}}_{\ell} \setminus \hat{\mathcal{L}}_{\ell-1}
  \quad \mbox{ for } \ell \geq 2.
\end{align*}

% \begin{align*}
% L'[\ell](t)
%   & = \Set{\SP}{|\Err{t}{i}{j}(\SP)|\leq g(s(t,\SP),\ell), \forall i,j\}
% \end{align*}
We call \Env{\ell} the \Kth{\ell} envelope.
In words, \Env{\ell} consists of all sample paths such that at all times
$t$, all coordinates of all arm estimation errors are bounded by
$g(s, \ell)$,
but for at least one time $t$, at least one coordinate of one arm
estimation error is \emph{not} bounded by $g(s, \ell-1)$.
When \SP is clear, we omit it in the notation for
\Err{t}{i}{j}, payments, etc.
The importance of envelopes is that for small $\ell$, the payments are
tightly bounded, while for large $\ell$, the cumulative probability of
the sample paths in \Env{\ell} is small.
This is captured by the following two lemmas.

\begin{lemma} \label{lem:sample-path-payment}
If $\SP \in \Env{\ell}$ and $s(t,\SP) = s$, then
the payment in step $t$ is upper-bounded by
$\bar{c} (s,\ell) = \MAXR + 2 \Diam d \cdot g(s,\ell)$.
\end{lemma}

\begin{proof}
The maximum payment is upper-bounded by the maximum perceived
difference in value for any agent type and any two arms:
\begin{align*}
\bar{c} (s, \ell) & \leq 
\max_{\AgV} \left(  \max_{i} \AgV \cdot \ArmEV{t}{i}
                 - \min_{i'} \AgV \cdot \ArmEV{t}{i'} \right) \\
& = \max_{\AgV} \left( \max_{i}\AgV \cdot (\ErrV{t}{i}+\ArmV{i})
                    - \min_{i'}\AgV \cdot (\ErrV{t}{i'}+\ArmV{i'}) \right) \\
& \leq \max_{\AgV} \left(  \max_{i} \AgV \cdot \ArmV{i}
                        - \min_{i'} \AgV \cdot \ArmV{i'}
                        + \max_{i} \AgV \cdot \ErrV{t}{i}
                        - \min_{i'} \AgV \cdot \ErrV{t}{i'} \right) \\
& \leq \MAXR + 2 \Diam d \cdot g(s,\ell). 
\end{align*}
The final inequality used the definition of the envelope.
\end{proof}

\begin{lemma} \label{lem:envelope-probability}
For every $\ell \geq 2$, we have that
$\Prob{\SP \in \Env{\ell}} \leq 24 \ARMNUM d\exp(-1.8(\ell-1)^2)$. 
\end{lemma}

\begin{proof}
For \SP to be in \Env{\ell}, by definition,
for at least one time $t$,
at least one coordinate of at least one arm's estimation error must
exceed $g(s(t,\SP),\ell-1)$.
For now, fix an arm $i$ and coordinate $j$.

Define $x_s := \frac{4 \sigma (\ell-1)}{s^{1/2}}$ and
$x'_s := g(s, \ell-1) = \frac{12 \sigma (\ell-1)}{s^{2/5}}$.

Recall that $\NumPull{t}{i}(\SP) \geq s(t,\SP)$ is the number of times
that arm $i$ has been pulled by time $t$ under \SP.
Let the random stopping time $\tau_{i,j}$ be the first value of
\NumPull{t}{i} (i.e., the first pull of arm $i$) such that 
the estimation error of attribute $j$ of arm $i$ exceeds
$g(\NumPull{t}{i}, \ell-1)$.
$\tau_{i,j} = \infty$ means that the estimation error never exceeds 
$g(\NumPull{t}{i}, \ell-1)$.
We next verify that $x_s, x'_s$ as we defined them satisfy the 
assumption 
\[
  \sqrt{0.6 n \cdot \log (\log_{1.1}(n) + 1) + \frac{n x_n^2}{16 \sigma^2}}
  \; \leq \; \frac{n x'_n}{2 \sigma}
\]
of Lemma~\ref{lem:round-prob}, for all $n \geq 1$.
First, we show that
$\sqrt{0.6n \log(\log_{1.1}(n)+1)} \leq 5 n^{3/5}$
for all $n \geq 1$.
By squaring the inequality and canceling out a factor $n$,
the statement is equivalent to showing that
$25 n^{1/5} \geq \log(\log_{1.1}(n)+1)$.
Indeed, a stronger statement holds, namely, that
$25 n^{1/5} \geq \log_{1.1}(n)+1$.  
To see this, notice that the derivative of the left-hand side is
always strictly larger than the derivative of the right-hand side,
so the difference between the sides is minimized at $n=1$,
where it is positive. 
Using this inequality and subadditivity of $\sqrt{\cdot}$, we can bound

\[
  \sqrt{0.6 n \cdot \log (\log_{1.1}(n) + 1) + \frac{n x_n^2}{16 \sigma^2}}
\; \leq \;
  5 n^{3/5} + \frac{\sqrt{n} x_n}{4 \sigma}
\; = \; 
  5 n^{3/5} + (\ell-1)
  \; \leq \; \frac{n x'_n}{2 \sigma}.
\]
Applying Lemma~\ref{lem:round-prob} to $\tau_{i,j}$,
we conclude that the probability that the error in coordinate $j$ of
arm $i$ exceeds $x'_s = g(s,\ell-1)$ at time $\tau_{i,j}$
(and hence at any time, by definition of $\tau_{i,j}$)
is at most
$24 \exp\left(-\frac{1.8 s x_s^2}{16 \sigma^2} \right)
\; = \; 24 \exp(-1.8 (\ell-1)^2)$.
Now, taking a union bound over all arms $i$ and
coordinates $j$ completes the proof.
\end{proof}

Next, we show that for any sample path \SP in the envelope \Env{\ell},
we can bound the total number of payments made in terms of $\ell$.
Define $h(\ell) := \max \left( \exp(\frac{2}{\MinProb}),
\left( \frac{24 \sigma \ell \Diam d}{\TieCutoff} \right)^{5/2},
\left( \frac{48 \sigma \ell \TieDensity \Diam d}{\MinProb} \right)^{5/2}
\right)$.

\begin{lemma} \label{lem:envelope-payments}
Let \SP be a sample path in \Env{\ell}.
Then, under \SP, the algorithm makes payments at most for the first 
$h(\ell)$ phases. 
\end{lemma}

\begin{proof}
The proof is similar to that of Lemma~\ref{lem:numP}.
Fix a sample path $\SP \in \Env{\ell}$.
Consider a phase $s > h(\ell)$ and a time $t$ in phase $s$.
Because $\SP \in \Env{\ell}$, in particular, all estimation errors are
bounded at time $t$, in the sense that the event \AccEU{t}{g(s,\ell)}
happens.
Because $s \geq h(\ell)$, we get that
$g(s,\ell) \leq \min(\frac{\TieCutoff}{2\Diam d}, \frac{\MinProb}{4 \Diam d \TieDensity})$
as well as $s \geq \exp(2/\MinProb)$.
Therefore, we can apply Lemma~\ref{lem:no-incentives} and conclude
that the pull at time $t$ was not incentivized.
\end{proof}

\begin{extraproof}{Theorem~\ref{rst:budget}}
We can write the total expected payment as
\begin{align*}
  \Expect{\sum_{t=1}^{\infty} \PayA{t}}
& = \sum_{\ell = 1}^{\infty} \sum_{\SP \in \Env{\ell}}
  \Prob{\SP} \cdot \sum_{t=1}^{\infty} \PayA{t}(\SP)
\; = \; \sum_{\ell = 1}^{\infty} \sum_{\SP \in \Env{\ell}}
  \Prob{\SP} \cdot \sum_{s=1}^{\infty} \sum_{t: s(t,\SP) = s} \PayA{t}(\SP).
\end{align*}

By Lemma~\ref{lem:envelope-payments},
the payments are 0 for $s > h(\ell)$,
and by Lemma~\ref{lem:sample-path-payment},
when $\SP \in \Env{\ell}$ and $s(t,\SP) = s$, we can bound
$\PayA{t}(\SP) \leq \MAXR + 2 \Diam d \cdot g(s,\ell)$.
Furthermore, in any one phase, because each arm is incentivized at
most once, there are at most \ARMNUM payments total.
Substituting these bounds, we obtain that

\begin{align*}
  \Expect{\sum_{t=1}^{\infty} \PayA{t}}
& \leq \sum_{\ell = 1}^{\infty} \sum_{\SP \in \Env{\ell}} \Prob{\SP} \cdot
  \sum_{s=1}^{h(\ell)} \ARMNUM \cdot (\MAXR + 2 \Diam d \cdot g(s,\ell))
\\ & = \ARMNUM \cdot
  \sum_{\ell = 1}^{\infty} \Prob{\SP \in \Env{\ell}} \cdot
  \sum_{s=1}^{h(\ell)} \left( \MAXR + \frac{24 \Diam d \sigma \ell}{s^{2/5}} \right).
\end{align*}

We now lower-bound $s^{2/5} \geq 1$, split off the term for $\ell=1$
and bound $\Prob{\SP \in \Env{1}} \leq 1$, and apply
Lemma~\ref{lem:envelope-probability} to the remaining
$\Prob{\SP \in \Env{\ell}}$ terms, to bound

\begin{align*}
\Expect{\sum_{t=1}^{\infty} \PayA{t}}
& \leq
\ARMNUM \cdot \sum_{s=1}^{h(1)} (\MAXR + 24 \Diam d \sigma)
+ \ARMNUM \cdot
  \sum_{\ell = 2}^{\infty} 24 \ARMNUM d\exp(-1.8(\ell-1)^2) \cdot
  \sum_{s=1}^{h(\ell)} (\MAXR + 24 \Diam d \sigma \ell)
\\ & \leq
\ARMNUM \cdot h(1) \cdot (\MAXR + 24 \Diam d \sigma)
+ 24 \ARMNUM^2 d \cdot \sum_{\ell = 2}^{\infty}
     \exp(-1.8(\ell-1)^2) \cdot h(\ell) \cdot (\MAXR + 24 \Diam d \sigma \ell).
\end{align*}
Because $h(\ell)$ and $(\MAXR + 24 \Diam d \sigma \ell)$ grow
polynomially in $\ell$, whereas $\exp(-1.8(\ell-1)^2)$ decreases
exponentially in $\ell$, the sum is dominated by its first term, and
the overall expected payment is bounded by

\begin{align*}
& O \left(\ARMNUM^2 d \cdot (\MAXR + \Diam d \sigma) \cdot h(1)
  \right)
\\ = 
& O \left(\ARMNUM^2  d \cdot (\MAXR + \Diam d \sigma)\cdot
  \max \left( \exp \left(\frac{2}{\MinProb}\right),
		    \left( \frac{\sigma \Diam d}{\TieCutoff} \right)^{5/2},
    \left( \frac{\sigma \TieDensity \Diam d}{\MinProb} \right)^{5/2} \right)
\right)
\\ =
& O \left( \ARMNUM^2 \cdot \exp \left(\frac{2}{\MinProb} \right) \right).
\end{align*}
\end{extraproof}


\section{Proof Sketch of Theorem~\ref{rst:discrete}}
\label{sec:discussion-proof1}

\begin{rtheorem}{Theorem}{\ref{rst:discrete}}
Under Assumption~\ref{a:discrete}, the Discrete-Preference Algorithm
has expected payment bounded above by 
$O \left(\ARMNUM^2/\MinProb \right)$
and expected regret bounded above by $O(\ARMNUM / \MinProb)$.
\end{rtheorem}

\begin{proof}
The proof of the expected payment bound is the same as the proof of
Theorem~\ref{rst:budget},
except that we now define
$h(\ell) := \max \left( \frac{2}{\MinProb},
\left( \frac{24 \sigma \ell \Diam d}{\TieCutoff} \right)^{5/2} \right)$.
%\pfedit{and $s_1 = \max(2,\frac{30\sigma^3}{x^3}, \frac2{\MinProb})$}.

To prove the bound on the expected regret,
one can first prove tightened versions of Lemmas~\ref{lem:no-incentives} and \ref{lem:numP},
which replace the $\exp(2/\MinProb)$ term with $2/\MinProb$ and use $L=0$.
In return, the length of phase $s$ is now bounded only by $\ARMNUM s$
instead of $\ARMNUM \log(s)$,
and the expected number of times steps in which a payment is made is bounded by
$O(\ARMNUM/\MinProb + \ARMNUM^2)$.
Substituting these changes into the proof of Theorem~\ref{rst:regret},
we obtain the bound $O(\MAXR(\ARMNUM/\MinProb + \ARMNUM^2) + \ARMNUM \MAXR +
\ARMNUM \MAXR) = O(\ARMNUM/\MinProb)$ since $\MinProb \le 1/\ARMNUM$.
\end{proof}

% \pfcomment{
% The calculation I did for the bound on the expected payment is,
% \begin{align*}
% &O \left(\ARMNUM^2 d \cdot (\MAXR + \Diam d \sigma) \cdot h(1) \right) \\ 
% &=O \left(\ARMNUM^2  d \cdot (\MAXR + \Diam d \sigma)\cdot
%     \max\bigg\{\frac{2}{p},
%     \left( \frac{\sigma \Diam d}{\TieCutoff} \right)^{5/2}
%     \bigg\}
% \right)
% \end{align*}
% }
% \pfcomment{
% The calculation I did for the bound on the regret is:

% \begin{itemize}
% \item regret from times when we pay is $RN(N + 1/p)$
% \item regret from phases with $\gamma(s) >= \hat{z}$ is $O(NR D^6 d^6 \sigma^6 / \hat{z}^6)$
% \item regret from pulls with small regret is 0
% \item regret from pulls with big regret is $O(NR)$
% \end{itemize}
% }

\section{Proof Sketch of Theorem~\ref{rst:known-p}}
\label{sec:discussion-proof2}


\begin{rtheorem}{Theorem}{\ref{rst:known-p}}
% Under Assumption~\ref{a:known-p}, the Known-\MinProb Algorithm has an expected payment bounded above by 
% $\pfedit{O\left(N^2 \max\{1, (L/p)^{5/2}\}\right)}$.
% and an expected regret bounded above by
% \bcedit{
% \begin{align*}
% O\left(\frac{\ARMNUM^2}{\MinProb^2}
%     + \frac{\ARMNUM^2}{\hat{z}^2} + \frac{\ARMNUM}{\hat{z}^3}
%     + \frac{NL\log^3(T)}{p}    
%     \right)
% \end{align*}
% }
Under Assumption~\ref{a:known-p}, the Known-\MinProb Algorithm has
expected payment bounded above by  
$O(\ARMNUM^2 \cdot \max(1,(\TieDensity/\MinProb)^{5/2}))$ 
and expected regret bounded above by
%\begin{align*}
$O \left( \frac{\ARMNUM^2}{\MinProb^2}
%  + \frac{\ARMNUM^2}{\hat{z}^2}+\frac{\ARMNUM}{\hat{z}^3}
  +\frac{\ARMNUM \TieDensity \log^3(T)}{\MinProb}\right)$.
%  \end{align*}
\end{rtheorem}

\begin{proof}
The proof of the expected payment bound follows Lemma~\ref{rst:budget},
except we define $h(\ell)$ without including the $\exp(2/\MinProb)$ term,
instead using
$h(\ell) := \max \left(
  \left(\frac{24\sigma \ell\Diam d}{\TieCutoff} \right)^{5/2},
  \left( \frac{48\sigma \ell\TieDensity \Diam d}{\MinProb} \right)^{5/2}
\right)$.
The resulting bound on the expected payment is
\begin{align*}
O\left( \ARMNUM^2 d \cdot (\MAXR+\Diam d\sigma) \cdot
  \max \left(
    \left(\frac{\sigma \Diam d}{\TieCutoff} \right)^{5/2},
    \left( \frac{\sigma \TieDensity \Diam d}{\MinProb} \right)^{5/2}
    \right) \right)
& = O\left(\ARMNUM^2 \max (1, (\TieDensity/\MinProb)^{5/2}) \right).
\end{align*}

The proof of the expected regret bound first establishes tightened versions of Lemmas~\ref{lem:no-incentives} and \ref{lem:numP},
proving the following upper bound on the number of time steps in which a payment is made:
    
\begin{align*}
O\left(\frac{\ARMNUM^2 \TieDensity^3 \Diam^3 d^3 \sigma^3}{\MinProb^2}
    + \frac{\ARMNUM^2 d^3 \sigma^2 \Diam^2}{\TieCutoff^2} + \frac{\ARMNUM \Diam^3 d^3 \sigma^3}{\TieCutoff^3}
  \right)
& = O \left( \frac{\ARMNUM^2}{\MinProb^2} \right).
\end{align*}

The proof of this result follows that of Lemma~\ref{lem:numP},
but the less aggressive incentivization allows us to define
$\EvenLaterPhase = \max(2, \frac{30 \sigma^3}{x^3})$
since $\frac{1}{\log(s+s_0)} \leq \frac{\MinProb}{2}$ is true for all $s$.

Using this tighter bound on the number of incentivizations,
and the fact that phases now last at most $\ARMNUM \log(s+s_0)$ steps in expectation,
we can bound the regret in Equation~\eqref{equ:small_regret_bound} by
%$71.11 \ARMNUM\Diam^2 d^2\sigma^2 \TieDensity\log(T)(\log(T)+1)\log(T+s_0)$
$O\left(\frac{\ARMNUM\Diam^2 d^2\sigma^2 \TieDensity \cdot (\log(T))^3}{\MinProb}\right)$.
\end{proof}


% \input{technical-proofs}

\end{document}
