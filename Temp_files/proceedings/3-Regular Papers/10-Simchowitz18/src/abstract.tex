\begin{abstract}
%System identification is a fundamental problem in time-series analysis, control
%theory, and reinforcement learning.
%%
%Despite its importance, a sharp non-asymptotic analysis for the number of
%trajectories from an unknown dynamical system needed to identify its parameters
%remains an open question, even in the special case when the dynamics are governed by linear
%equations.
%%
%In this paper, we take an important step towards a non-asymptotic theory for system identification.
%We prove that the ordinary least-squares (OLS) estimator attains nearly minimax
%optimal performance for the identification of linear dynamical systems from
%a single observed trajectory.
%%
%Our analysis relies on a generalization of Mendelson's small-ball method to dependent data,
%eschewing the use of standard mixing-time arguments.
%%
%We capture the correct
%signal-to-noise behavior of the problem, showing that \emph{more unstable} linear
%systems are \emph{easier} to estimate.
%%
%This behavior is qualitatively different from arguments which rely on mixing-time
%calculations that suggest that unstable systems are more difficult to estimate.
%%
%Finally, our proof techniques generalize to a class of linear response
%time-series.


We prove that the ordinary least-squares (OLS) estimator attains nearly minimax
optimal performance for the identification of linear dynamical systems from
a single observed trajectory.
%
Our upper bound relies on a generalization of Mendelson's small-ball method to dependent data,
eschewing the use of standard mixing-time arguments.
%
Our lower bounds reveal that these upper bounds match up to logarithmic factors.
%
In particular, we capture the correct
signal-to-noise behavior of the problem, showing that \emph{more unstable} linear
systems are \emph{easier} to estimate.
%
This behavior is qualitatively different from arguments which rely on mixing-time
calculations that suggest that unstable systems are more difficult to estimate.
%
We generalize our technique to provide bounds for a more general class of linear response
time-series.



\end{abstract}
