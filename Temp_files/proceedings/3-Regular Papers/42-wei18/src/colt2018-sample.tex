%\documentclass[anon,12pt]{colt2018} % Anonymized submission
\documentclass[final, 12pt]{colt2018} % Include author names

% The following packages will be automatically loaded:
% amsmath, amssymb, natbib, graphicx, url, algorithm2e

\title[More Adaptive Algorithms for Adversarial Bandits]{More Adaptive Algorithms for Adversarial Bandits}
\usepackage{float}
%\restylefloat{table}

\usepackage{algorithm}
\usepackage{times}
\usepackage{makecell}
\usepackage[final]{showlabels}
%\usepackage{tablefootnote}
\usepackage{threeparttable}
\allowdisplaybreaks
%\usepackage[ruled]{algorithm2e}
 %vlined, ,linesnumbered
\usepackage{bbm}
\usepackage{enumerate}
%\usepackage{amsmath,amsthm,amsfonts,amssymb,mathrsfs}
%\addbibresource{main.bib}
\newcommand{\p}{\prime}
\DeclareMathOperator*{\argmin}{argmin}
\DeclareMathOperator*{\argmax}{argmax}
\newcommand{\inner}[1]{ \left\langle {#1} \right\rangle }
\newcommand{\inn}[1]{ \langle {#1} \rangle }
\newcommand{\absolute}[1]{ \left\lvert {#1} \right\rvert }
\newcommand{\abs}[1]{ \lvert {#1} \rvert }
\usepackage{amsmath}
\newcommand\norm[1]{\left\lVert#1\right\rVert}
\newcommand{\reg}{\text{\rm Reg}}
%\newtheorem{lemma}{Lemma}
%\newtheorem{theorem}{Theorem}
\newtheorem{cor}[theorem]{Corollary}
%\newtheorem{remark}[theorem]{Remark}
%\newtheorem{remark}{Remark}
%\newtheorem{prop}{Proposition}
%\newtheorem{definition}{Definition}
%\newtheorem{assumption}{Assumption}
 % Use \Name{Author Name} to specify the name.
 % If the surname contains spaces, enclose the surname
 % in braces, e.g. \Name{John {Smith Jones}} similarly
 % if the name has a "von" part, e.g \Name{Jane {de Winter}}.
 % If the first letter in the forenames is a diacritic
 % enclose the diacritic in braces, e.g. \Name{{\'E}louise Smith}

 % Two authors with the same address
  % \coltauthor{\Name{Author Name1} \Email{abc@sample.com}\and
  %  \Name{Author Name2} \Email{xyz@sample.com}\\
  %  \addr Address}

 % Three or more authors with the same address:
 % \coltauthor{\Name{Author Name1} \Email{an1@sample.com}\\
 %  \Name{Author Name2} \Email{an2@sample.com}\\
 %  \Name{Author Name3} \Email{an3@sample.com}\\
 %  \addr Address}


 % Authors with different addresses:
 \coltauthor{\Name{Chen-Yu Wei} \Email{chenyu.wei@usc.edu} \\
 \addr University of Southern California
 \AND
 \Name{Haipeng Luo} \Email{haipengl@usc.edu}\\
 \addr University of Southern California
 }

\begin{document}

\maketitle

\begin{abstract}
\begin{abstract}
%System identification is a fundamental problem in time-series analysis, control
%theory, and reinforcement learning.
%%
%Despite its importance, a sharp non-asymptotic analysis for the number of
%trajectories from an unknown dynamical system needed to identify its parameters
%remains an open question, even in the special case when the dynamics are governed by linear
%equations.
%%
%In this paper, we take an important step towards a non-asymptotic theory for system identification.
%We prove that the ordinary least-squares (OLS) estimator attains nearly minimax
%optimal performance for the identification of linear dynamical systems from
%a single observed trajectory.
%%
%Our analysis relies on a generalization of Mendelson's small-ball method to dependent data,
%eschewing the use of standard mixing-time arguments.
%%
%We capture the correct
%signal-to-noise behavior of the problem, showing that \emph{more unstable} linear
%systems are \emph{easier} to estimate.
%%
%This behavior is qualitatively different from arguments which rely on mixing-time
%calculations that suggest that unstable systems are more difficult to estimate.
%%
%Finally, our proof techniques generalize to a class of linear response
%time-series.


We prove that the ordinary least-squares (OLS) estimator attains nearly minimax
optimal performance for the identification of linear dynamical systems from
a single observed trajectory.
%
Our upper bound relies on a generalization of Mendelson's small-ball method to dependent data,
eschewing the use of standard mixing-time arguments.
%
Our lower bounds reveal that these upper bounds match up to logarithmic factors.
%
In particular, we capture the correct
signal-to-noise behavior of the problem, showing that \emph{more unstable} linear
systems are \emph{easier} to estimate.
%
This behavior is qualitatively different from arguments which rely on mixing-time
calculations that suggest that unstable systems are more difficult to estimate.
%
We generalize our technique to provide bounds for a more general class of linear response
time-series.



\end{abstract}

\end{abstract}

\begin{keywords}
multi-armed bandit, semi-bandit, adaptive regret bounds, optimistic online mirror descent, increasing learning rate
\end{keywords}

\section{Introduction}
Online learning algorithms are a key tool in web search and content optimization, adaptively learning what users want to see. In a typical application, each time a user arrives, the algorithm chooses among various content presentation options (\eg news articles to display), the chosen content is presented to the user, and an outcome (\eg a click) is observed. Such algorithms must balance \emph{exploration} (making potentially suboptimal decisions now for the sake of acquiring information that will improve decisions in the future) and \emph{exploitation} (using information collected in the past to make better decisions now). Exploration could degrade the experience of a current user, but improves user experience in the long run. This exploration-exploitation tradeoff is commonly studied in the online learning framework of \emph{multi-armed bandits}~\citep{Bubeck-survey12}.

Concerns have been raised about whether exploration in such scenarios could be unfair, in the sense that some individuals or groups may experience too much of the downside of exploration without sufficient upside \citep{bird2016exploring}. We formally study these concerns in the \emph{linear contextual bandits} model~\citep{Langford-www10,chu2011contextual}, a standard variant of multi-armed bandits appropriate for content personalization scenarios.  We focus on \emph{externalities} arising due to exploration, that is, undesirable side effects that the presence of one party may impose on another.


We first examine the effects of exploration at a group level.  We introduce the notion of a \emph{group externality} in an online learning system, quantifying how much the presence of one population (which we dub the majority) impacts the rewards of another (the minority). We show that this impact can be negative, and that, in a particular precise sense, no algorithm can avoid it. This cannot be explained by the absence of suitably good policies since our adoption of the linear contextual bandits framework implies the existence of a feasible policy that is simultaneously optimal for everyone. Instead, the problem is inherent to the process of exploration. We come to a surprising conclusion that more data can sometimes lead to worse outcomes for the users of an explore-exploit-based system. \looseness=-1

We next turn to the effect of exploration at an individual level. We interpret exploration as a potential externality imposed on the current user by future users of the system. Indeed, it is only for the sake of the future users that the algorithm would forego the action that currently looks optimal. To avoid this externality, one may use the greedy algorithm that always chooses the action that appears optimal according to current estimates of the problem parameters. While this greedy algorithm performs poorly in the worst case,
it tends to work well in many applications and experiments.\footnote{Both positive and negative findings are folklore. One way to precisely state the negative result is that the greedy algorithm incurs constant per-round regret with constant probability; while results of this form have likely been known for decades,
\citet[Corollary A.2(b)]{competingBandits-itcs16}
proved this for a wide variety of scenarios. Very recently, the good empirical performance has been confirmed by state-of-art experiments in \citet{practicalCB-arxiv18}.}

In a new line of work, \citet{bastani2017exploiting} and \citet{kannan2018smoothed}
analyzed conditions under which inherent diversity in the data makes explicit exploration unnecessary.
\citet{kannan2018smoothed} proved that the greedy algorithm achieves a regret rate of
$\tilde{O}(\sqrt{T})$ in expectation over small perturbations of the context vectors (which ensure sufficient data diversity). This is the best rate that can be achieved in the worst case (\ie for all problem instances, without data diversity assumptions), but it leaves open the possibilities that (i) another algorithm may perform much better than the greedy algorithm on some problem instances, or (ii) the greedy algorithm may perform much better than worst case under the diversity conditions. We expand on this line of work. We prove that under the same diversity conditions, the greedy algorithm almost matches the best possible Bayesian regret rate of \emph{any} algorithm \emph{on the same problem instance}. This could be as low as $\polylog(T)$ for some instances, and, as we prove, at most $\tilde{O}(T^{1/3})$ whenever the diversity conditions hold.


Returning to group-level effects, we show that under the same diversity conditions, the negative group externalities imposed by the majority essentially vanish if one runs the greedy algorithm. Together, our results illustrate a sharp contrast between the high individual and group externalities that exist in the worst case, and the ability to remove all externalities if the data is sufficiently diverse.   \looseness=-1

\xhdr{Additional motivation.} Whether and when explicit exploration is necessary is an important concern in the study of the exploration-exploitation tradeoff. Fairness considerations aside, explicit exploration is expensive. It is wasteful and risky in the short term, it adds a layer of complexity to algorithm design \citep{Langford-nips07,monster-icml14}, and its adoption at scale tends to require substantial systems support and buy-in from management \citep{MWT-WhitePaper-2016,DS-arxiv}. A system based on the greedy algorithm would typically be cheaper to design and deploy.

Further, explicit exploration can run into incentive issues in applications such as recommender systems. Essentially, when it is up to the users which products or experiences to choose and the algorithm can only issue recommendations and ratings, an explore-exploit algorithm needs to provide incentives to explore for the sake of the future users \citep{Kremer-JPE14,Frazier-ec14,Che-13,ICexploration-ec15,Bimpikis-exploration-ms17}. Such incentive guarantees tend to come at the cost of decreased performance, and rely on assumptions about human behavior. The greedy algorithm avoids this problem as it is inherently consistent with the users' incentives.



\xhdr{Additional related work.}
Our research draws inspiration from the growing body of work on fairness in machine learning~\cite[e.g.,][]{dwork2012fairness,hardt2016equality,kleinberg2017inherent,chouldechova2017fair}.  Several other authors have studied fairness in the context of the contextual bandits framework.  Our work differs from the line of research on meritocratic fairness in online learning \citep{kearns2017meritocratic,liu2017calibrated,joseph2016fairness}, which considers the allocation of limited resources such as bank loans and requires that nobody should be passed over in favor of a less qualified applicant. We study a fundamentally different scenario in which there are no allocation constraints and we would like to serve each user the best content possible.  Our work also differs from that of \citet{celis2017fair}, who studied an alternative notion of fairness in the context of news recommendations. According to this notion, all users should have approximately the same probability of seeing a particular type of content (e.g., Republican-leaning articles), regardless of their individual preferences, in order to mitigate the possibility of discriminatory personalization.

The data diversity conditions in \citet{kannan2018smoothed} and this paper are inspired by the smoothed analysis framework of \citet{SmoothedAnalysis-jacm04}, who proved that the expected running time of the simplex algorithm is polynomial for perturbations of any initial problem instance (whereas the worst-case running time has long been known to be exponential). Such disparity implies that very bad problem instances are brittle. 
We find a similar disparity for the greedy algorithm in our setting.



\xhdr{Our results on group externalities.}  A typical goal in online learning is to minimize \emph{regret}, the (expected) difference between the cumulative reward that would have been obtained had the optimal policy been followed at every round and the cumulative reward obtained by the algorithm.  We define a corresponding notion of \emph{minority regret}, the portion of the regret experienced by the minority.  Since online learning algorithms update their behavior based on the history of their observations, minority regret is influenced by the entire population on which an algorithm is run.  If the minority regret is much higher when a particular algorithm is run on the full population than it is when the same algorithm is run on the minority alone, we can view the majority as imposing a negative externality on the minority; the minority population would achieve a higher cumulative reward if the majority were not present. Asking whether this can ever happen
amounts to asking whether access to more data points can ever lead an explore-exploit algorithm to make inferior decisions. One might think that more data should always lead to better decisions and therefore better outcomes for the users.
Surprisingly, we show that this is not the case, even with a standard algorithm.

Consider LinUCB~\citep{Langford-www10,chu2011contextual,abbasi2011improved}, a standard algorithm for linear contextual bandits that is based on the principle of ``optimism under uncertainty.''  We provide a specific problem instance on which, after observing $T$ users, LinUCB would have a minority regret of $\Omega(\sqrt T)$ if run on the full population, but only constant minority regret if run on the minority alone. While stylized, this example is motivated by the problem of providing driving directions to different populations of users, about which fairness concerns have been raised~\citep{bird2016exploring}. Further, the situation is reversed on a slight variation of this example: LinUCB obtains constant minority regret when run on the full population and $\Omega(\sqrt T)$ on the minority alone.  That is, group externalities can be large and positive in some cases, and large and negative in others.

Although these regret rates are specific to LinUCB, we show that this phenomenon is, in some sense, unavoidable. Consider the minority regret of LinUCB when run on the full population and the minority regret that LinUCB would incur if run on the minority alone. We know that one may be much smaller or larger than the other. We ask whether any algorithm could  achieve the minimum of the two on every problem instance. Using a variation of the same problem instance, we prove that this is impossible; in fact, no algorithm could simultaneously approximate both up to any $o(\sqrt{T})$ factor. In other words, an externality-free algorithm would sometimes ``leave money on the table."


In terms of techniques, we rely on the special structure of our example, which can be viewed as an instance of the sleeping bandits problem~\citep{SleepingBandits-ml10}. This simplifies the behavior and analysis of LinUCB, allowing us to obtain the $O(1)$ upper bounds.  The lower bounds are obtained using KL-divergence techniques to show that the two variants of our example are essentially indistinguishable, and an algorithm that performs well on one must obtain $\Omega(\sqrt{T})$ regret on the other. \looseness=-1


\xhdr{Our results on the greedy algorithm.} We consider a version of linear contextual bandits in which the latent weight vector $\theta$ is drawn from a known prior. In each round, an algorithm is presented several actions to choose from, each represented by a \emph{context vector}. The expected reward of an action is a linear product of $\theta$ and the corresponding context vector. The tuple of context vectors is drawn independently from a fixed distribution. In the spirit of smoothed analysis, we assume that this distribution has a small amount of jitter. Formally, the tuple of context vectors is drawn from some fixed distribution, and then a small \emph{perturbation}---small-variance Gaussian noise---is added independently to each coordinate of each context vector. This ensures arriving contexts are diverse. We are interested in Bayesian regret, i.e., regret in expectation over the Bayesian prior. Following the literature, we are primarily interested in the dependence on the time horizon $T$. \looseness=-1

We focus on a batched version of the greedy algorithm, in which new data arrives to the algorithm's optimization routine in small batches, rather than every round. This is well-motivated from a practical perspective---in high-volume applications data usually arrives to the ``learner" only after a substantial delay \citep{MWT-WhitePaper-2016,DS-arxiv}---and is essential for our analysis.

Our main result is that the greedy algorithm matches the Bayesian regret of any algorithm up to polylogarithmic factors, for each problem instance, fixing the Bayesian prior and the context distribution. We also prove that LinUCB achieves regret $\tilde{O}(T^{1/3})$ for each realization of $\theta$. This implies a worst-case Bayesian regret of $\tilde{O}(T^{1/3})$ for the greedy algorithm under the perturbation assumption. \looseness=-1

Our results hold for both natural versions of the batched greedy algorithm, Bayesian and frequentist, henceforth called \BayesGreedy and \FreqGreedy. In \BayesGreedy, the chosen action maximizes expected reward according to the Bayesian posterior. \FreqGreedy estimates $\theta$ using ordinary least squares regression and chooses the best action according to this estimate. The results for \FreqGreedy come with additive polylogarithmic factors, but are stronger in that the algorithm does not need to know the prior. Further, the $\tilde{O}(T^{1/3})$ regret bound for \FreqGreedy is approximately prior-independent, in the sense that it applies even to very concentrated priors such as independent Gaussians with standard deviation on the order of $T^{-2/3}$.

The key insight in our analysis of \BayesGreedy is that any (perturbed) data can be used to simulate any other data, with some discount factor. The analysis of \FreqGreedy requires an additional layer of complexity. We consider a hypothetical algorithm that receives the same data as \FreqGreedy, but chooses actions based on the Bayesian-greedy selection rule. We analyze this hypothetical algorithm using the same technique as \BayesGreedy, and then upper bound the difference in Bayesian regret between the hypothetical algorithm and \FreqGreedy.

Our analyses extend to group externalities and (Bayesian) minority regret. In particular, we circumvent the impossibility result mentioned above. We prove that both \BayesGreedy and \FreqGreedy match the Bayesian minority regret of any algorithm run on either the full population or the minority alone, up to polylogarithmic factors

\xhdr{Detailed comparison with prior work.} We substantially improve over the $\tilde{O}(\sqrt{T})$ worst-case regret bound from \citet{kannan2018smoothed}, at the cost of some additional assumptions. First, we consider Bayesian regret, whereas their regret bound is for each realization of $\theta$.%
\footnote{Equivalently, they allow point priors, whereas our priors must have variance $T^{-O(1)}$.} Second, they allow the context vectors to be chosen by an adversary before the perturbation is applied. Third, they extend their analysis to a somewhat more general model, in which there is a separate latent weight vector for every action (which amounts to a more restrictive model of perturbations). However, this extension relies on the greedy algorithm being initialized with a substantial amount of data. The results of \citet{kannan2018smoothed} do not appear to have implications on group externalities.

\citet{bastani2017exploiting} show that the greedy algorithm achieves logarithmic regret in an alternative linear contextual bandits setting that is incomparable to ours in several important ways.
They consider two-action instances where the actions share a common context vector in each round, but are parameterized by different latent vectors. They ensure data diversity via a strong assumption on the context distribution. This assumption does not follow from our perturbation conditions; among other things, it implies that each action is the best action in a constant fraction of rounds. Further, they assume a version of Tsybakov's \emph{margin condition}, which is known to substantially reduce regret rates in bandit problems \citep[\eg see][]{Zeevi-colt10}.



\section{Problem Setup and Algorithm Overview}
\label{section:notations}
We consider the combinatorial bandit problem with semi-bandit feedback, which subsumes the classic multi-armed bandit problem. The learning process proceeds for $T$ rounds. In each round, the learner selects a subset of arms, denoted by a binary vector $b_t$ from a predefined action set $\mathcal{X}\subseteq \{0,1\}^K$, and suffers loss $b_t^\top \ell_t$, where $\ell_t \in [-1,1]^K$ is a loss vector decided by an adversary. The feedback received by the learner is the vector $(b_{t,1}\ell_{t,1}, \ldots, b_{t,K}\ell_{t,K})$,
or in other words, the loss of each chosen arm. For simplicity, we assume that the adversary is oblivious and the loss vectors $\ell_1, \ldots, \ell_T$ are decided ahead of time independent of the learner's actions.

The learner's goal is to minimize the {\it regret}, which is the gap between her accumulated loss and that of the best fixed action $b^*\in\mathcal{X}$. Formally the regret is defined as
\begin{align*}
\reg_T\triangleq \sum_{t=1}^T b_t^\top \ell_{t}- \sum_{t=1}^T b^{*\top}\ell_{t}, \text{ where } b^*\triangleq \min_{b\in\mathcal{X}} \sum_{t=1}^T b^\top \ell_{t}. 
\end{align*}

In the special case of multi-armed bandit, the action set $\mathcal{X}$ is $\{\mathbf{e}_1, \mathbf{e}_2, \ldots, \mathbf{e}_K\}$ where $\mathbf{e}_i$ denotes the $i$-th standard basis vector. In other words, in each round the learner picks one arm $i_t \in [K] \triangleq \{1,2,\ldots,K\}$ (corresponding to $b_t = \mathbf{e}_{i_t}$), and receives the loss $\ell_{t,i_t}$. We denote the best arm by $i^* \triangleq \min_{i \in [K]} \sum_{t=1}^T \ell_{t, i}$.

%The OMD framework relies heavily on the notion of Bregman divergence. 
\paragraph{Notation.}
For a convex function $\psi$ defined on a convex set $\Omega$, the Bregman divergence of two points $u, v\in \Omega$ with respect to $\psi$ is defined as $D_{\psi}(u,v)\triangleq\psi(u)-\psi(v)-\inn{\nabla\psi(v), u-v}$. The log-barrier used in this work is of the form $\psi(u)=\sum_{i=1}^K \frac{1}{\eta_i}\ln \frac{1}{u_i}$ for some learning rates $\eta_1, \ldots, \eta_K \geq 0$ and $u \in \text{conv}(\mathcal{X})$, the convex hull of $\mathcal{X}$. With $h(y)\triangleq y-1-\ln y$, the Bregman divergence with respect to the log-barrier is: 
%\begin{align*}
$D_{\psi}(u,v)=\sum_{i=1}^K \frac{1}{\eta_i} \left(\ln \frac{v_i}{u_i} + \frac{u_i-v_i}{v_i}\right)=\sum_{i=1}^K \frac{1}{\eta_i} h\left(\frac{u_i}{v_i}\right).$
%\end{align*}
%where $h(y)\triangleq y-1-\ln y$. %Note that $h(y)$ is always positive, and is increasing when $y\geq 1$. 

%Other notations we use in the paper are as below: $[N]$ denotes $\{1,2,\ldots, N\}$; for any $v=(v_1, \ldots, v_K)\in \mathbb{R}^K$, $v^2$ denotes the vector $(v_1^2, \ldots, v_K^2)$; for a binary vector $b\in \{0,1\}^K$, we write $i\in b$ if $b_i=1$; $\mathbf{1}$ is the all-one vector $(1,1,\ldots, 1)$; $\text{conv}(\mathcal{X})$ is the convex hull of $\mathcal{X}$; $\tilde{\mathcal{O}}(\cdot)$ is the big-O notation that can hide $(\ln T)^n$ factors for $n\in \mathbb{R}_+$, and $\tilde{\Omega}$ is the big-$\Omega$ notation that hides $(\ln T)^{-n}$. Although we also use $\Omega$ to denote $\text{conv}(\mathcal{X})$, it should be clear from the contexts which one we are referring to. 

The all-zero and all-one vector are denoted by $\mathbf{0}$ and $\mathbf{1}$ respectively.
$\Delta_K$ represents the ($K-1$)-dimensional simplex.
For a binary vector $b$ we write $i\in b$ if $b_i=1$. 
Denote by $K_0 = \max_{b \in \mathcal{X}}\|b\|_0$ the maximum number of arms an action in $\mathcal{X}$ can pick.
Note that for MAB, $K_0$ is simply $1$.

We define $\ell_0 = \mathbf{0}$ for notational convenience. 
At round $t$, for an arm $i$ we denote its accumulated loss by $L_{t,i}\triangleq \sum_{s=1}^t \ell_{s,i}$,
its average loss by $\mu_{t,i} \triangleq \frac{1}{t}L_{t,i}$,
its (unnormalized) variance by $Q_{t,i}\triangleq \sum_{s=1}^t (\ell_{s,i}-\mu_{t,i})^2$,
and its first-order path-length by $V_{t,i}\triangleq \sum_{s=1}^t \absolute{\ell_{s,i}-\ell_{s-1,i}}$. 
%and its second-order path-length by $D_{t,i}\triangleq \sum_{s=1}^t (\ell_{s,i}-\ell_{s-1,i})^2$.
For MAB, we define $\alpha_i(t)$ to be the most recent time when arm $i$ is picked prior to round $t$ ,
that is, $\alpha_i(t) = \max\{s < t : i_s = i\}$ (or $0$ if the set is empty).
%Similarly, in the analysis we define $\beta_i(t)$ to be the next time when arm $i$ is picked starting from round $t$,
%that is, $\beta_i(t) = \min\{s \geq t : i_s = i\}$ (or $T+1$ if the set is empty). 

\subsection{Algorithm Overview}
As mentioned our algorithm falls into the OMD framework that operates on the set $\Omega = \text{conv}(\mathcal{X})$.
The vanilla OMD formula for the bandit setting is $w_{t} = \argmin_{w\in\Omega} \{ \inn{w, \hat{\ell}_{t-1}}+D_{\psi}(w,w_{t-1}) \}$
for some regularizer $\psi$ and some (unbiased) estimator $\hat{\ell}_{t-1}$ of the true loss $\ell_{t-1}$.
The learner then picks an action $b_t$ randomly such that $\mathbb{E}[b_t] = w_t$, and constructs the next loss estimator $\hat{\ell}_t$ based on the bandit feedback.
Our algorithm, however, requires several extra ingredients. The generic update rule is
\begin{align}
w_t &= \argmin_{w\in\Omega} \left\{ \inn{w, m_t}+D_{\psi_t}(w,w_t^\p) \right\},\label{eqn:update_rule_1}\\ 
w_{t+1}^\p &= \argmin_{w\in\Omega} \left\{ \inn{w, \hat{\ell}_t+a_t}+D_{\psi_t}(w,w_t^\p) \right\}  \label{eqn:update_rule_2}.
\end{align}

\begin{algorithm}[t]
\DontPrintSemicolon
\caption{\small \textbf{B}arrier-\textbf{R}egularized with \textbf{O}ptimism and \textbf{AD}aptivity \textbf{O}nline \textbf{M}irror \textbf{D}escent (\textsc{\textbf{Broad-OMD}})}
\label{alg:general}
\textbf{Define}: $\Omega=\text{conv}(\mathcal{X})$, $\psi_t(w)=\sum_{i=1}^K \frac{1}{\eta_{t,i}}\ln\frac{1}{w_i}$. \\
\textbf{Initialize}: $w_1^\p = \argmin_{w\in \Omega}\psi_1(w)$.\\
\For{$t=1, 2, \ldots, T$}{
   $w_t = \argmin_{w\in\Omega} \big\{ \inner{w,m_t} + D_{\psi_t}(w, w_t^\p)\big\}$. \\
   Draw $b_t\sim w_t$, suffer loss $b_t^\top \ell_t$, and observe $\{b_{t,i}\ell_{t,i}\}_{i=1}^K$. \\
   Construct $\hat{\ell}_t$ as an unbiased estimator of $\ell_t$. \\
   Let $a_{t,i}=\begin{cases}
       6\eta_{t,i}w_{t,i}(\hat{\ell}_{t,i}-m_{t,i})^2,  &\text{(Option I)}\\
       0. &\text{(Option II)}
       \end{cases}$\\
   $w_{t+1}^\p=\argmin_{w\in\Omega} \big\{ \langle w,\hat{\ell}_t+ a_t\rangle  +D_{\psi_t}(w, w^\p_t) \big\}.$ 
}    
\end{algorithm}

 \renewcommand{\arraystretch}{1.4}
\begin{table}[t] 
  \centering
  \caption{Different configurations of \textsc{Broad-OMD} and regret bounds for MAB. 
  See Section~\ref{section:notations} and the corresponding sections for the meaning of notation. 
  For the last two rows, to obtain parameter-free algorithms one needs to apply a doubling trick to decrease the learning rate.}
  \begin{threeparttable}
  \begin{tabular}{ | c | c | c | c | c | c | }
    \hline
    Sec. & Option & $m_{t,i}$ & $\hat{\ell}_{t,i}$ & $\eta_{t,i}$ & $\mathbb{E}[\reg_T]$ in $\tilde{\mathcal{O}}$ \\ \hline
    \ref{subsubsection:variation bound} & I & $\tilde{\mu}_{t-1,i}$ & $\frac{(\ell_{t,i}-m_{t,i})\mathbbm{1}\{i_t=i\}}{w_{t,i}}+m_{t,i}$ & fixed & $\sqrt{KQ_{T,i^*}}$ \\ \hline
    \ref{subsubsection:path-length} &  I & $\ell_{\alpha_i(t),i}$ & $\frac{(\ell_{t,i}-m_{t,i})\mathbbm{1}\{i_t=i\}}{\bar{w}_{t,i}}+m_{t,i}$ & increasing & $K\sqrt{V_{T,i^*}}$ \\ \hline
    \ref{subsection:first_order_better_k} & II & $\ell_{\alpha_i(t),i}$ & $\frac{(\ell_{t,i}-m_{t,i})\mathbbm{1}\{i_t=i\}}{w_{t,i}}+m_{t,i}$ & fixed & $ \sqrt{K\sum_{i=1}^K V_{T,i}}$ \\  \hline
   \ref{section:best of both worlds} & II & $\ell_{t,i_t}$ &  $\frac{\ell_{t,i}\mathbbm{1}\{i_t=i\}}{w_{t,i}}$ & fixed & $\min\{ \sqrt{KL_{T,i^*}}, \frac{K}{\Delta}\}$ \\ \hline
  \end{tabular}
  \end{threeparttable}
  \label{table:summary}
\end{table}

Here, we still play randomly according to $w_t$, which is now updated to minimize its loss with respect to $m_t \in [-1,1]^K$, 
an {\it optimistic prediction} of the true loss vector $\ell_t$,
penalized by a Bregman divergence term associated with a {\it time-varying} regularizer $\psi_t$.
In addition, we maintain a sequence of auxiliary points $w_t^\p$ that is updated using the loss estimator $\hat{\ell}_t$ and an extra {\it correction term} $a_t$.

When $a_t = \mathbf{0}$, this is studied in~\citep{rakhlin2013online} under the name optimistic OMD. 
When $a_t \neq \mathbf{0}$, the closest algorithm to this variant of OMD is its FTRL version studied by~\citet{steinhardt2014adaptivity}.
However, while $\psi_t$ is fixed for all $t$ in~\citep{steinhardt2014adaptivity},\footnote{%
\citet{steinhardt2014adaptivity} also uses the notation $\psi_t$, but it corresponds to putting $a_t$ into a fixed regularizer.}
some of our results crucially rely on using time-varying $\psi_t$ (which corresponds to time-varying learning rate)
and also the OMD update form instead of FTRL. 

It is well known that the classic Exp3 algorithm falls into this framework with $m_t = a_t = \mathbf{0}$ and $\psi_t$ being the (negative) entropy.
To obtain our results, first, it is crucial to use the log-barrier as the regularizer instead, that is, $\psi_t(w)=\sum_{i=1}^K \frac{1}{\eta_{t,i}}\ln\frac{1}{w_i}$
for some individual and time-varying learning rates $\eta_{t,i}$.
Second, we focus on two options of $a_t$.
For results that depend on some quantity of only the best arm, we use a sophisticated choice of $a_t$ that we explain in details in Section~\ref{section:Option I}.
For the other results we simply set $a_t = \mathbf{0}$.
With the choices of $m_t, \hat{\ell}_t$, and $\eta_{t}$ open, we present this generic framework in Algorithm~\ref{alg:general}
and name it \textsc{Broad-OMD} (short for Barrier-Regularized with Optimism and ADaptivity Online Mirror Descent).

In Section~\ref{section:Option I} and~\ref{section:Option II} respectively, we prove general regret bounds for \textsc{Broad-OMD} 
with Option I and Option II, followed by specific applications in the MAB setting achieved via specific choices of $m_{t}, \hat{\ell}_t$, and $\eta_{t}$.
The results and the corresponding configurations of the algorithm are summarized in Table \ref{table:summary}.  



\paragraph{Computational efficiency.} 
The sampling step $b_t \sim w_t$ can be done efficiently as long as $\Omega$ can be described by a polynomial number of constraints.
The optimization problems in the update rules of $w_t$ and $w_t'$ are convex and can be solved by general optimization methods.
For many special cases, however, these two computational bottlenecks have simple solutions.
Take MAB as an example, $w_t$ directly specifies the probability of picking each arm,
and the optimization problems can be solved via a simple binary search~\citep{agarwal2017corralling}.


\section{\textsc{Broad-OMD} with Option I}
\label{section:Option I}
In this section we focus on \textsc{Broad-OMD} with Option I.
We first show a general lemma that update rules~\eqref{eqn:update_rule_1} and~\eqref{eqn:update_rule_2} guarantee,
no matter what regularizer $\psi_t$ is used and what $a_t, m_t$, and $\hat{\ell}_t$ are.

\begin{lemma}
\label{thm:general_instantaneous}
For the update rules~\eqref{eqn:update_rule_1} and~\eqref{eqn:update_rule_2}, if the following condition holds:
\begin{align}
\inn{w_t-w^\p_{t+1}, \hat{\ell}_t-m_t+a_t} \leq \inn{w_t, a_t}, \label{eqn:condition1} 
\end{align}
then for all $u\in \Omega$, we have
\begin{align}
\inn{w_t-u, \hat{\ell}_t}\leq D_{\psi_t}(u,w_t^\p)-D_{\psi_t}(u,w^\p_{t+1})+\inn{u,a_t}-A_t,\label{eqn:regret_bound:bregman}
\end{align}
where $A_t\triangleq D_{\psi_t}(w_{t+1}^\p, w_t)+D_{\psi_t}(w_t, w_t^\p)\geq 0$.
\end{lemma}
 
%The condition \eqref{eqn:condition1} is essentially the one stated in Theorem 3.1 of \cite{steinhardt2014adaptivity}, but we translate it from FTRL to OMD language. One can indeed see that the regret has a term $\inn{u,a_t}$ that adapts to the regret benchmark. 

%Besides, the regularizers we use in our algorithm are \textit{log-barriers}. Therefore, we name this algorithm the Barrier-Regularized with Optimism and ADaptivity Online Mirror Descent, abbreviated \textsc{Broad-OMD}. The pseudocode is shown in Algorithm \ref{alg:general}. 

%\textsc{Broad-OMD} can be applied to all combinatorial bandit problems with semi-bandit feedback. The classic multi-armed bandit problem is a special case of it. The algorithm is designed with the typical trick to feed unbiased loss estimators to a full-information algorithm. 

%To see the usefulness of the bound in Lemma \ref{thm:general_instantaneous}, let's now assume that $\psi_t=\psi$ for all $t$, and assume \eqref{eqn:condition1} indeed holds. Then by telescoping and dropping negative terms, Lemma \ref{thm:general_instantaneous} implies $\sum_{t=1}^{T} \inn{w_t-u,\hat{\ell}_t}\leq D_{\psi}(u,w^\p_1)+\sum_{t=1}^T \inn{u,a_t}$. We note that the regret achieved by Optimistic OMD is $\sum_{t=1}^{T} \inn{w_t-u,\hat{\ell}_t}\leq D_{\psi}(u,w^\p_1)+\sum_{t=1}^T D_\psi(w_t, w_{t+1}^\p)$. One can see that the second term with \textsc{Broad-OMD} is adaptive to the regret comparator, while Optimistic-OMD's does not. 

The important part of bound~\eqref{eqn:regret_bound:bregman} is the term $\inn{u,a_t}$, 
which allows us to derive regret bounds that depend on only the comparator $u$.
The key is now how to configure the algorithm such that condition~\eqref{eqn:condition1} holds, 
while leading to a reasonable bound~\eqref{eqn:regret_bound:bregman} at the same time. 

In the work of~\citep{steinhardt2014adaptivity} for full-information problems, $a_t$ can be defined as $a_{t,i} = \eta_{t,i}(\ell_{t,i}-m_{t,i})^2$,
which suffices to derive many interesting results.
However, in the bandit setting this is not applicable since $\ell_t$ is unknown.
The natural first attempt is to replace $\ell_t$ by $\hat{\ell}_t$, but one would quickly realize the common issue in the bandit literature:
$\hat{\ell}_{t,i}$ is often constructed via inverse propensity weighting, and thus $(\hat{\ell}_{t,i}-m_{t,i})^2$ can be of order $1/w_{t,i}^2$, which is too large.

Based on this observation, our choice for $a_t$ is $a_{t,i}=6\eta_{t,i}w_{t,i}(\hat{\ell}_{t,i}-m_{t,i})^2$ (the constant $6$ is merely for technical reasons). 
The extra term $w_{t,i}$ can then cancel the aforementioned large term $1/w_{t,i}^2$ in expectation, similar to the classic trick done in the analysis of Exp3~\citep{auer2002nonstochastic}.

%More precisely, with the above choice of $a_{t,i}$ and $\hat{\ell}_{t,i}$, we have 
%\begin{align}
%\mathbb{E}_{b_t}[\inn{u,a_t}]=\mathbb{E}_{b_t}\left[ \sum_{i=1}^K 18\eta u_i(\ell_{t,i}-m_{t,i})^2 \frac{\mathbbm{1}\{i\in b_t\}}{w_{t,i}} \right]=18\sum_{i=1}^K \eta_{t,i}u_{i}(\ell_{t,i}-m_{t,i})^2.\label{eqn:no_explode}
%\end{align}
%Inserting this into \eqref{eqn:regret_bound:a_t_neq_0}, we see that the regret bound here is only related to the estimation errors on the coordinates the benchmark takes (i.e., $u$), rather than those the learner chooses (i.e., $b_t$). This is essentially what \textit{adaptivity} means. 

%After we add this $w_{t,i}$, we turn to worry about whether \eqref{eqn:condition1} can hold, because now its right-hand side is tighter. By some standard analysis in OMD, the left-hand side of \eqref{eqn:condition1} can often be bounded by a constant times \sloppy$(\hat{\ell}_{t}-m_t+a_t)^\top {\nabla^{-2}\psi_t(w_t)}(\hat{\ell}_{t}-m_t+a_t)$. The value of this quantity is smaller as the regularizer $\psi_t$ is more curved (i.e., larger Hessian). Therefore, in order to make \eqref{eqn:condition1} hold, we are forced to use the log-barrier regularizer instead of the commonly used negative-entropy regularizer. 

Note that with a smaller $a_t$, condition~\eqref{eqn:condition1} becomes more stringent.
The entropy regularizer used in~\citep{steinhardt2014adaptivity} no longer suffices to maintain such a condition.
Instead, it turns out that the log-barrier regularizer used by \textsc{Broad-OMD} addresses the issue, as shown below.

\begin{theorem}
\label{lemma:MAB_condition}
If the following three conditions hold for all $t,i$: 
(i) $\eta_{t,i}\leq \frac{1}{162}$,
(ii) $w_{t,i}\abs{\hat{\ell}_{t,i}-m_{t,i}}\leq 3$,
(iii) $\sum_{i=1}^K \eta_{t,i}w_{t,i}^2(\hat{\ell}_{t,i}-m_{t,i})^2\leq \frac{1}{18},$
then \textsc{Broad-OMD} with $a_{t,i}=6\eta_{t,i}w_{t,i}(\hat{\ell}_{t,i}-m_{t,i})^2$ guarantees condition~\eqref{eqn:condition1}.
%\begin{align}
%\inn{w_t-w_{t+1}^\p, \hat{\ell}_t-m_t+ a_t} \leq \inn{w_t, a_t}. \label{eqn:condition_hold}
%\end{align}
Moreover, it guarantees for any $u\in \Omega$ (recall $h(y) = y - 1 - \ln y \geq 0$), 
\begin{align}
\sum_{t=1}^T \inn{w_t-u, \hat{\ell}_t}\leq \sum_{i=1}^K \left( \frac{\ln\frac{w^\p_{1,i}}{u_i}}{\eta_{1,i}} + \sum_{t=1}^T \left(\frac{1}{\eta_{t+1,i}}-\frac{1}{\eta_{t,i}}\right)h\left(\frac{u_i}{w_{t+1,i}^\p}\right) \right) +\sum_{t=1}^T \inn{u,a_t}. \label{eqn:regret_bound:a_t_neq_0} 
\end{align}
\end{theorem}

The three conditions of the theorem are usually trivially satisfied as we will show.
Note that $h(\cdot)$ is always non-negative. Therefore, if the sequence $\{\eta_{t,i}\}_{t=1}^{T+1}$ is non-decreasing for all $i$,\footnote{%
One might notice that $\eta_{T+1, i}$ is not defined here.
Indeed this term is artificially added only to make the analysis of Section~\ref{subsubsection:path-length} more concise, and $\eta_{T+1,i}$ can be any positive number.
In Algorithm~\ref{alg:increasing} we give it a concrete definition.
} 
the term $\sum_{t=1}^T \left(\frac{1}{\eta_{t+1,i}}-\frac{1}{\eta_{t,i}}\right)h\left(\frac{u_i}{w_{t+1,i}^\p}\right)$ in bound~\eqref{eqn:regret_bound:a_t_neq_0} 
is non-positive. For some results we can simply discard this term, while for others, this term becomes critical.
On the other hand, the term $\ln\frac{w^\p_{1,i}}{u_i}$ appears to be infinity if we want to compare with the best fixed action (where $u_i = 0$ for some $i$).
However, this can be simply resolved by comparing with some close neighbor of the best action in $\Omega$ instead, similar to~\citep{foster2016learning, agarwal2017corralling}.

One can now derive different results using Theorem~\ref{lemma:MAB_condition} with specific choices of $\hat{\ell}_t$ and $m_t$.
As an example, we state the following corollary by using a variance-reduced importance-weighted estimator $\hat{\ell}_t$ as in~\citep{rakhlin2013online}.
\begin{cor}
\label{cor:clear_corollary}
\textsc{Broad-OMD} with $a_{t,i}=6\eta_{t,i}w_{t,i}(\hat{\ell}_{t,i}-m_{t,i})^2$, any $m_{t,i} \in [-1, 1]$, $\hat{\ell}_{t,i}=\frac{(\ell_{t,i}-m_{t,i})\mathbbm{1}\{i\in b_t\}}{w_{t,i}}+m_{t,i}$, and \sloppy $\eta_{t,i}=\eta \leq \frac{1}{162K_0}$ enjoys the following regret bound:
%(for MAB, the last constraint can be $\eta_{t,i}=\eta\leq\frac{1}{162}$) 
%Under the same settings and conditions as in Theorem \ref{lemma:MAB_condition}, if $\eta_{t,i}=\eta$ for all $t,i$, and $\hat{\ell}_t$ is constructed as $\hat{\ell}_{t,i}=\frac{(\ell_{t,i}-m_{t,i})\mathbbm{1}\{i\in b_t\}}{w_{t,i}}+m_{t,i}$, then \textsc{Broad-OMD} enjoys the following regret bound:
\begin{align*}
\mathbb{E}\left[\reg_T\right]= \mathbb{E}\left[ \sum_{t=1}^T \inn{b_t-b^*, \ell_t} \right] \leq \frac{K\ln T}{\eta} + 6\eta\mathbb{E}\left[\sum_{t=1}^T \sum_{i: i\in b^*} (\ell_{t,i}-m_{t,i})^2\right] +\mathcal{O}(K_0). 
\end{align*}
%For MAB, the last $\mathcal{O}(K)$ term can be replaced by $\mathcal{O}(1)$. 
\end{cor}
%\begin{proof}{\textbf{(sketch)}}
%Picking $u=b^*$ in \eqref{eqn:regret_bound:a_t_neq_0}, then we have: 
%\begin{align*}
%\mathbb{E}_{b_t}[\inn{u,a_t}]=C\eta\mathbb{E}_{b_t}\left[ \sum_{i=1}^K u_{i}w_{t,i}\times \frac{(\ell_{t,i}-m_{t,i})^2\{i\in b_t\}}{w_{t,i}^2} \right]=C\eta\sum_{i: i\in b^*} (\ell_{t,i}-m_{t,i})^2. 
%\end{align*}
%Also note the facts $\mathbb{E}\left[\sum_{t=1}^T \inn{w_t-u, \hat{\ell}_t}\right]=\mathbb{E}\left[ \sum_{t=1}^T \inn{b_t-b^*, \ell_t} \right]$, and $\mathbb{E}[B]=\mathcal{O}(1)$. Putting everything into \eqref{eqn:regret_bound:a_t_neq_0}, we get the desired bound.
%\end{proof}
%Here let's see what benefits does the bound \eqref{eqn:regret_bound:a_t_neq_0} bring to us. Assume for now that $\eta_{t,i}=\eta$ for all $t, i$, and the unbiased estimator $\hat{\ell}_t$ is constructed as $\hat{\ell}_{t,i}=\frac{(\ell_{t,i}-m_{t,i})\mathbbm{1}\{i\in b_t\}}{w_{t,i}}+m_{t,i}$ for a general $m_{t,i}$. Picking $u=b^*$, then $\mathbb{E}_{b_t}[\inn{u,a_t}]=C\eta\mathbb{E}_{b_t}\left[ \sum_{i=1}^K u_{i}w_{t,i}\times \frac{(\ell_{t,i}-m_{t,i})^2\{i\in b_t\}}{w_{t,i}^2} \right]=C\eta \sum_{i=1}^K u_{i}(\ell_{t,i}-m_{t,i})^2=C\eta\sum_{i: i\in b^*} (\ell_{t,i}-m_{t,i})^2$. Furthermore, note that $\mathbb{E}\left[\sum_{t=1}^T \inn{w_t-u, \hat{\ell}_t}\right]=\mathbb{E}\left[ \sum_{t=1}^T \inn{b_t-b^*, \ell_t} \right]$. Thus, taking expectation on both sides of \eqref{eqn:regret_bound:a_t_neq_0} yields
%\begin{align}
%\mathbb{E}\left[ \sum_{t=1}^T \inn{b_t-b^*, \ell_t} \right] \leq \frac{K\ln T}{\eta} + C\eta\sum_{t=1}^T \sum_{i: i\in b^*} (\ell_{t,i}-m_{t,i})^2 +\mathcal{O}(1). 
%\end{align}
%if we ignore the negative terms and use the fact $\mathbb{E}[B]=\mathcal{O}(1)$. 

One can see that the expected regret in Corollary \ref{cor:clear_corollary} only depends on the squared estimation error of $m_t$ for the actions that $b^*$ chooses! This is exactly the counterpart of results in~\citep{steinhardt2014adaptivity}, but for the more challenging combinatorial semi-bandit problem. 
Note that our dependence on $K_0$ is also optimal~\citep{audibert2013regret}.

In the following subsections, we invoke Theorem \ref{lemma:MAB_condition} with different choices of $\hat{\ell}_{t}$ and $m_t$ to obtain various more concrete adaptive bounds. %Although Theorem \ref{lemma:MAB_condition} is true for all combinatorial semi-bandit problems, 
For simplicity, we state these results only in the MAB setting, but they can be straightforwardly generalized to the semi-bandit case. 



%%%%%%%%%%%%%%%%%%%%%%%%%%%
%  Variation Bounds  
%%%%%%%%%%%%%%%%%%%%%%%%%%%
\subsection{Variance Bound}
\label{subsubsection:variation bound}
Our first application of \textsc{Broad-OMD} is an adaptive bound that depends on the variance of the best arm, that is, a bound of order $\tilde{\mathcal{O}}\left(\sqrt{KQ_{T,i^*}}\right)=\tilde{\mathcal{O}}\left(\sqrt{K\sum_{t=1}^T(\ell_{t,i^*}-\mu_{T,i^*})^2}\right)$.  
According to Corollary~\ref{cor:clear_corollary}, if we were able to use $m_t = \mu_T$, with a best-tuned $\eta$ the bound is obtained immediately.
The issue is of course that $\mu_T$ is unknown ahead of time. 
In fact, even setting $m_t = \mu_{t-1}$ is infeasible due to the bandit feedback.

Fortunately this issue was already solved by~\citet{hazan2011better} via the ``reservoir sampling'' technique. 
%For the details of this technique, the readers are referred to \citep{hazan2011better}. 
The high level idea is that one can spend a small portion of time on estimating $\mu_{t}$ on the fly. More precisely, by performing uniform exploration with probability $\min\left\{1, \frac{MK}{t}\right\}$ at time $t$ for some parameter $M$, one can obtain an estimator $\tilde{\mu}_{t}$ of $\mu_t$  such that $\mathbb{E}[\tilde{\mu}_{t}] = \mu_{t}$ and $\text{Var}[\tilde{\mu}_{t,i}]\leq \frac{Q_{t,i}}{Mt}$ (see~\citep{hazan2011better} for details). 
Then we can simply pick $m_t=\tilde{\mu}_{t-1}$ %and construct $\hat{\ell}_t$ as $\hat{\ell}_{t,i}=\frac{(\ell_{t,i}-m_{t,i})\mathbbm{1}\{i_t=i\}}{w_{t,i}}+m_{t,i}$. 
and prove the following result.
\begin{theorem}
\label{cor:variance_bound}
\textsc{Broad-OMD} with reservoir sampling~\citep{hazan2011better}, $a_{t,i}=6\eta_{t,i}w_{t,i}(\hat{\ell}_{t,i}-m_{t,i})^2$, $m_{t,i}=\tilde{\mu}_{t-1,i}$, $\hat{\ell}_{t,i}=\frac{(\ell_{t,i}-m_{t,i})\mathbbm{1}\{i_t=i\}}{w_{t,i}}+m_{t,i}$, and $\eta_{t,i}=\eta\leq \frac{1}{162}$ guarantees
\begin{align*}
\mathbb{E}\left[\reg_T\right]= \mathcal{O}\left(\frac{K\ln T}{\eta}+\eta Q_{T,i^*} + K(\ln T)^2\right).
\end{align*}
With the optimal tuning of $\eta$, the regret is thus of order $\tilde{\mathcal{O}}\left(\sqrt{KQ_{T,i^*}}+K\right)$.
\end{theorem}

\subsection{Path-length Bound}
\label{subsubsection:path-length}

Our second application is to obtain path-length bounds.
The counterpart in the full-information setting is a bound in terms of the second-order path-length $\sum_{t=1}^T(\ell_{t,i^*}-\ell_{t-1,i^*})^2$~\citep{steinhardt2014adaptivity}. 
Again, in light of Corollary~\ref{cor:clear_corollary}, if we were able to pick $m_t = \ell_{t-1}$ the problem would be solved.
The difficulty is again that $\ell_{t-1}$ is not fully observable.

%However in the bandit setting, to get a path-length bound like this seems to be a much more challenging problem, and it is still not clear how to achieve this bound or whether it is possible at all. The difficulty lies in that the learner only observes the loss of one arm in each round, and thus the full-information approach that lets $m_{t,i}=\ell_{t-1,i}$ for all $i$ \citep{chiang2012online, rakhlin2013online, steinhardt2014adaptivity} is no longer feasible. 

While it is still not clear how to achieve such a second-order path-length bound or whether it is possible at all,
we propose a way to obtain a slightly weaker first-order path-length bound 
$\tilde{\mathcal{O}}\left(K\sqrt{V_{T,i^*}}\right)=\tilde{\mathcal{O}}\Big(K\sqrt{\sum_{t=1}^T\abs{\ell_{t,i^*}-\ell_{t-1,i^*}}}\Big)$.
Note that in the worst case this is $\sqrt{K}$ times worse than the optimal regret $\tilde{\mathcal{O}}(\sqrt{TK})$.

%Fortunately, we can still obtain some weaker path-length bounds. In this section, an algorithm achieving the expected regret of \sloppy$\tilde{\mathcal{O}}\left(K\sqrt{V_{T,i^*}}\right)=\tilde{\mathcal{O}}\left(K\sqrt{\sum_{t=1}^T\abs{\ell_{t,i^*}-\ell_{t-1,i^*}}}\right)$ is introduced. This bound is weaker in the sense that it is a first-order rather than a second-order bound, and its dependency on $K$ is slightly worse than an ordinary MAB algorithm. We will see another path-length bound that achieves $\tilde{\mathcal{O}}\left(\sqrt{K\sum_{i=1}^K V_{T,i}}\right)$ in Section \ref{section:Option II}. 

The idea is to set $m_{t,i}$ to be the most recent observed loss of arm $i$, that is, $m_{t,i}=\ell_{\alpha_i(t),i}$, where $\alpha_i(t)$ is defined in Section \ref{section:notations}.
%, and let $\hat{\ell}_{t,i}=\frac{(\ell_{t,i}-m_{t,i})\mathbbm{1}\{i_t=i\}}{w_{t,i}}+m_{t,i}$. 
While the estimation error $(\ell_{t,i}-\ell_{\alpha_i(t),i})^2$ could be much larger than $(\ell_{t,i}-\ell_{t-1,i})^2$, the quantity we aim for, 
observe that %if $t-\alpha_i(t)$ is small, the two estimation errors should be close; and 
if $t-\alpha_i(t)$ is large, it means that arm $i$ has bad performance before time $t$ so that the learner seldom draws arm $i$.
In this case, the learner might have accumulated negative regret with respect to arm $i$, which can potentially be used to compensate the large estimation error. 

To formalize this intuition, we go back to the bound in Theorem~\ref{lemma:MAB_condition} and examine the key term $\sum_{t=1}^T \inn{u, a_t}$
after plugging in $u = \mathbf{e}_i$ for some arm $i$, $m_{t,i}=\ell_{\alpha_i(t),i}$, and $\hat{\ell}_{t,i}=\frac{(\ell_{t,i}-m_{t,i})\mathbbm{1}\{i_t=i\}}{w_{t,i}}+m_{t,i}$. 
We assume $\eta_{t,i}=\eta$ for simplicity and also use the fact $w_{t,i}\abs{\hat{\ell}_{t,i}-m_{t,i}}\leq 2$.
We then have
\begin{align}
\sum_{t=1}^T \inn{u,a_t}&=6\eta\sum_{t=1}^T w_{t,i}(\hat{\ell}_{t,i}-\ell_{\alpha_i(t),i})^2 \leq 12\eta\sum_{t=1}^T \abs{\hat{\ell}_{t,i}-\ell_{\alpha_i(t),i}}
= 12\eta\sum_{t: i_t=i} \frac{\abs{\ell_{t,i}-\ell_{\alpha_i(t),i}} }{w_{t,i}} \nonumber \\
&\leq 12\eta\sum_{t: i_t=i} \frac{\sum_{s=\alpha_i(t)+1}^t \abs{\ell_{s,i}-\ell_{s-1,i}} }{w_{t,i}}  
\leq12\eta \left(\max_{t\in[T]} \frac{1}{w_{t,i}}\right) V_{T,i}. \label{eqn:path_length_trick}
\end{align}
%where $\beta_{i}(t)$ is also defined in Section \ref{section:notations}, and $T_i$ is the last round arm $i$ is drawn in the whole horizon. 

%One can observe that the path length of arm $i$ is penalized with a larger factor (i.e., larger $\frac{1}{w_{\beta_i(t),i}}$) when it is less often drawn around time $t$ (i.e., smaller $w_{\beta_i(t),i}$). Similar to what we just described, this large factor has the chance to be compensated by previously accumulated negative regret. 

Therefore, the term $\sum_{t=1}^T \inn{u,a_t}$ is close to the first-order path-length but with an extra factor $\max_{t\in[T]} \frac{1}{w_{t,i}}$.
To cancel this potentially large factor, we adopt the increasing learning rate schedule recently used in~\citep{agarwal2017corralling}. 
The idea is that the term $h\big(\frac{u_i}{w_{t+1,i}^\p}\big)$ in Eq.~\eqref{eqn:regret_bound:a_t_neq_0} is close to $\frac{1}{w_{t+1,i}}$ if $u_i$ is close to $1$.
If we increase the learning rate whenever we encounter a large $\frac{1}{w_{t+1,i}}$, 
then $\Big(\frac{1}{\eta_{t+1,i}}-\frac{1}{\eta_{t,i}}\Big)h\Big(\frac{u_i}{w_{t+1,i}^\p}\Big)$ becomes a large negative term in terms of $\frac{-1}{w_{t+1,i}}$,
which exactly compensates the term $\sum_{t=1}^T \inn{u,a_t}$.

To avoid the learning rates increased by too much, 
similarly to~\citep{agarwal2017corralling} we use some individual threshold ($\rho_{t,i}$) to decide when to increase the learning rate
and update these thresholds in some doubling manner. 
Also, we mix $w_t$ with a small amount of uniform exploration to further ensure that it cannot be too small.
The final algorithm, call \textsc{Broad-OMD+}, is presented in Algorithm~\ref{alg:increasing} (only for the MAB setting for simplicity).
We prove the following theorem.
%The idea is to let the $-D_\psi(u,w_{t+1}^\p)$ term in Lemma \ref{thm:general_instantaneous} come into help. Similar to \citep{agarwal2017corralling}, with this increasing learning rate mechanism, we need to control the magnitude of the loss estimator $\hat{\ell}_t$, so we mix a small enough probability (i.e., $\frac{1}{KT}\mathbf{1}$) to $w_t$.  We rewrite our algorithm as \textsc{Broad-OMD+} in Algorithm \ref{alg:increasing}. We write it in the MAB setting for simplicity, and we also fix the choices of $a_t$, $\hat{\ell}_t$ in the algorithm. 


\begin{algorithm}[t]
\DontPrintSemicolon
\caption{\textsc{Broad-OMD}+ (specialized for MAB)}
\label{alg:increasing}
\textbf{Define:} $\kappa=e^{\frac{1}{\ln T}}$, $\psi_t(w)= \sum_{i=1}^K \frac{1}{\eta_{t,i}} \ln \frac{1}{w_{i}}$. \\
\textbf{Initialize}: $w^\p_{1, i} = 1/K$, $\rho_{1, i} = 2K$ for all $i \in [K]$.\\
\For{$t=1, 2, \ldots, T$}{
   $w_t = \argmin_{w\in\Delta_K} \big\{ \inner{w,m_t} + D_{\psi_t}(w, w_t^\p)\big\}$. \\
   $\bar{w}_{t} = (1-\frac{1}{T})w_{t} + \frac{1}{KT}\mathbf{1}$. \\
   Draw $i_t\sim \bar{w}_t$, suffer loss $\ell_{t,i_t}$, and let $\hat{\ell}_{t,i}=\frac{(\ell_{t,i}-m_{t,i})\mathbbm{1}\{i_t=i\}}{\bar{w}_{t,i}}+m_{t,i}$.\\
   Let $a_{t,i}=6\eta_{t,i}w_{t,i}(\hat{\ell}_{t,i}-m_{t,i})^2$.
   \\
   $w_{t+1}^\p=\argmin_{w\in\Delta_K} \big\{ \langle w,\hat{\ell}_t+a_t\rangle  +D_{\psi_t}(w, w^\p_t) \big\}.$ \\
   \For{$i=1, \ldots, K$}{
      \lIf{$\frac{1}{\bar{w}_{t,i}} > \rho_{t,i}$}{
         $\rho_{t+1,i}=\frac{2}{\bar{w}_{t,i}}$, $\eta_{t+1,i}=\kappa\eta_{t,i}$. 
      }
      \lElse{
         $\rho_{t+1,i}=\rho_{t,i}$, $\eta_{t+1,i}=\eta_{t,i}$.
      }
   }
}    
\end{algorithm}

%One can verify that even with the above modifications, Lemma \ref{thm:general_instantaneous} and \ref{lemma:MAB_condition} still hold because we do not change the definitions of $w_t$ or $w_{t+1}^\p$ (i.e., update rules \eqref{eqn:update_rule_1}, \eqref{eqn:update_rule_2} remain the same). Based on them, we can prove the following theorem. 
\begin{theorem}
\label{thm:path_length}
\textsc{Broad-OMD+} with $m_{t,i}=\ell_{\alpha_i(t), i}$ and $\eta_{1,i}=\eta\leq \frac{1}{810}$ guarantees 
\begin{align*}
\mathbb{E}\left[\reg_T \right] \leq \frac{2K\ln T}{\eta} + \mathbb{E}[\rho_{T+1,i^*}]\left( \frac{-1}{40\eta\ln T} + 90\eta V_{T,i^*} \right) + \mathcal{O}\left( 1 \right)
\end{align*}
when $T\geq 3$. Picking $\eta = \min\Big\{\frac{1}{810}, \frac{1}{60\sqrt{V_{T,i^*} \ln T}}\Big\}$ so that the second term is non-positive leads to $\mathbb{E}\left[\reg_T \right] = \tilde{\mathcal{O}}\left( K\sqrt{V_{T,i^*}}+K \right)$. 
\end{theorem}

\section{\textsc{Broad-OMD} with Option II}
\label{section:Option II}
%As mentioned, \textsc{Broad-OMD} with $a_t=\mathbf{0}$ is simply an optimistic OMD with the log-barrier regularizer. However, we will see its several interesting applications that were not exploited in previous works. The following lemma holds for general semi-bandit problems. 

In this section, we move on to discuss \textsc{Broad-OMD} with Option II, that is, $a_t = \mathbf{0}$. 
We also fix $\eta_{t,i} = \eta$, although in the doubling trick discussed later, different values of $\eta$ will be used for different runs of  \textsc{Broad-OMD}.
Again we start with a general lemma that holds no matter what regularizer $\psi_t$ is used and what $m_t$ and $\hat{\ell}_t$ are.

\begin{lemma}
\label{lemma:simple_lemma}
For the update rules~\eqref{eqn:update_rule_1} and~\eqref{eqn:update_rule_2} with $a_t=\mathbf{0}$, we have for all $u\in \Omega$, 
\begin{align*}
\inn{w_t-u, \hat{\ell}_t}\leq D_{\psi_t}(u,w_t^\p)-D_{\psi_t}(u,w^\p_{t+1})+\inn{w_t-w_{t+1}^\p, \hat{\ell}_t-m_t}-A_t, 
\end{align*}
where $A_t\triangleq D_{\psi_t}(w_{t+1}^\p, w_t)+D_{\psi_t}(w_t, w_t^\p)\geq 0$.
\end{lemma}

The proof is standard as in typical OMD analysis. The next theorem then shows how the term $\inn{w_t-w_{t+1}^\p, \hat{\ell}_t-m_t}$ is further bounded
when $\psi_t$ is the log-barrier as in \textsc{Broad-OMD}. 

\begin{theorem}
\label{lemma:second_order_regret_bound}
If the following three conditions hold for all $t,i$: 
(i) $\eta \leq \frac{1}{162}$,
(ii) $w_{t,i}\abs{\hat{\ell}_{t,i}-m_{t,i}}\leq 3$,
(iii) $\eta\sum_{i=1}^K w_{t,i}^2(\hat{\ell}_{t,i}-m_{t,i})^2\leq \frac{1}{18}$
(same as those in Theorem \ref{lemma:MAB_condition}), %and furthermore assume $\{\eta_{t,i}\}$ are non-decreasing for all $i$,
then \textsc{Broad-OMD} with $a_t=\mathbf{0}$ guarantees for any $u\in \Omega$, 
\begin{align}
\sum_{t=1}^T \inn{w_t-u, \hat{\ell}_t}\leq \sum_{i=1}^K  \frac{\ln \frac{w^\p_{1,i}}{u_i}}{\eta}  +3\eta\sum_{t=1}^T\sum_{i=1}^K w_{t,i}^2(\hat{\ell}_{t,i}-m_{t,i})^2-\sum_{t=1}^T A_t. \label{eqn:second_order_regret_bound}
\end{align}
For MAB, the last term can further be lower bounded by $\sum_{t=1}^T A_t \geq \frac{1}{48\eta}\sum_{t=2}^T \sum_{i=1}^K\frac{(w_{t,i}-w_{t-1,i})^2}{w_{t-1,i}^2}$.
\end{theorem}

In bound~\eqref{eqn:second_order_regret_bound}, 
the first term can again be bounded by $\frac{K \ln T}{\eta}$ via picking an appropriate $u$.
The last negative term is useful when we use the algorithm to play games, which is discussed in Section~\ref{subsection:games}.
The second term is the key term, which, compared to the key term $\sum_{t=1}^T \inn{u,a_t}$ in Eq.~\eqref{eqn:regret_bound:a_t_neq_0} for \textsc{Broad-OMD} with Option I,
has an extra $w_{t,i}$ and is in terms of all arms instead of the arms that $u$ picks.
As a comparison to Corollary~\ref{cor:clear_corollary}, if we pick $\hat{\ell}_{t,i}=\frac{(\ell_{t,i}-m_{t,i})\mathbbm{1}\{i\in b_t\}}{w_{t,i}}+m_{t,i}$,
we obtain an expected regret bound in terms of $\mathbb{E}\left[ \sum_{t=1}^T \sum_{i \in b_t} (\ell_{t,i} - m_{t,i})^2 \right] = 
\mathbb{E}\left[ \sum_{t=1}^T \sum_{i =1}^K w_{t,i} (\ell_{t,i} - m_{t,i})^2 \right]$,
which is not as easy to interpret as the bound in Corollary~\ref{cor:clear_corollary}.
However, in the following subsections we will discuss in details how to apply bound~\eqref{eqn:second_order_regret_bound} to obtain more concrete results.

%if we select $\eta_{t,i}=\eta$, and $u=\left(1-\frac{1}{T}\right)\mathbf{e}_{i^*}+\frac{1}{T}w_1^\p$, which makes $\frac{w_{1,i}^\p}{u_i}\leq T$, then it implies
%\begin{align}
%\sum_{t=1}^T\inn{w_t-\mathbf{e}_{i^*}, \hat{\ell}_t} \leq \frac{K\ln T}{\eta} + 3\eta\sum_{t=1}^T \sum_{i=1}^K w_{t,i}^2(\hat{\ell}_{t,i}-m_{t,i})^2 + B,  \label{eqn:double_trick_bound1}
%\end{align}
%where $B=\sum_{t=1}^T \inn{-\frac{1}{T}\mathbf{e}_{i^*}+\frac{1}{KT}\mathbf{1}, \hat{\ell}_t}$. 

Before that, we point out that since the bound is now in terms of all arms, % instead of the comparator $u$,
we can in fact apply a doubling trick to make the algorithm parameter-free!
The idea is that 
%The benefit of \eqref{eqn:double_trick_bound1} is that the first two terms on the right-hand side is independent of the regret comparator $\mathbf{e}_{i^*}$, 
%and thus we can use the standard doubling trick to tune the learning rate: 
as long as the observable term $3\eta\sum_{s=1}^t \sum_{i=1}^K w_{s,i}^2(\hat{\ell}_{s,i}-m_{s,i})^2$ becomes larger than $\frac{K\ln T}{\eta}$ at some round $t$, 
we half the learning rate $\eta$ and restart the algorithm. 
This avoids the need for optimal tuning done in Section~\ref{section:Option I}.
%Besides, $\mathbb{E}[B]=\mathcal{O}(1)$ for any $i^*$, 
%so we can fairly ignore the $B$ term in \eqref{eqn:double_trick_bound1} if we only care about the expected regret. 
We formally present the algorithm in Algorithm \ref{alg:doubling} (in Appendix~\ref{app:doubling_trick}) and show its regret bound below.

\begin{theorem}
\label{thm:doubling_trick_theorem}
If conditions (ii) and (iii) in Theorem~\ref{lemma:second_order_regret_bound} hold, then Algorithm \ref{alg:doubling} guarantees
\begin{align*}
\mathbb{E}[\reg_T]=\mathcal{O}\left(\sqrt{(K\ln T)\mathbb{E}\left[\sum_{t=1}^T\sum_{i=1}^Kw_{t,i}^2(\hat{\ell}_{t,i}-m_{t,i})^2\right]}+K_0K\ln T\right).
\end{align*}
\end{theorem}
In the following subsections, we instantiate Theorem~\ref{lemma:second_order_regret_bound} or~\ref{thm:doubling_trick_theorem} with different $m_{t}$ and $\hat{\ell}_t$. Again, for simplicity we only focus on the MAB setting. 

\subsection{Another Path-length Bound}
\label{subsection:first_order_better_k}
If we configure \textsc{Broad-OMD} with Option II in the same way as in Section~\ref{subsubsection:path-length},
that is, $m_{t,i}=\ell_{\alpha_i(t),i}$ and $\hat{\ell}_{t,i}=\frac{(\ell_{t,i}-m_{t,i})\mathbbm{1}\{i_t=i\}}{w_{t,i}}+m_{t,i}$.
Then the key term in Eq.~\eqref{eqn:second_order_regret_bound} can be bounded as follows:
\begin{align}
&\sum_{t=1}^T \sum_{i=1}^K  w_{t,i}^2(\hat{\ell}_{t,i}-m_{t,i})^2= \sum_{t=1}^T\sum_{i=1}^K (\ell_{t,i}-\ell_{\alpha_i(t),i})^2\mathbbm{1}\{i_t=i\} 
= \sum_{i=1}^K \sum_{t:i_t=i} (\ell_{t,i}-\ell_{\alpha_i(t),i})^2 \nonumber \\
&\leq 2 \sum_{i=1}^K \sum_{t:i_t=i} \abs{\ell_{t,i}-\ell_{\alpha_i(t),i}} 
\leq 2 \sum_{i=1}^K \sum_{t:i_t=i} \sum_{s=\alpha_i(t)+1}^t\abs{\ell_{s,i}-\ell_{s-1,i}} \leq 2 \sum_{i=1}^K V_{T,i}. \label{eqn:path_length_calculation_1}
\end{align}
Unlike Eq.~\eqref{eqn:path_length_trick}, this is bounded even without the help of negative regret, but the price is that now the regret depends on the sum of all arms' path-length. With this calculation, we obtain the following corollary.
\begin{cor}
\label{cor:path_length_bound_1}
\textsc{Broad-OMD} with $a_{t,i}=0$, $m_{t,i}=\ell_{\alpha_i(t),i}$, $\hat{\ell}_{t,i}=\frac{(\ell_{t,i}-m_{t,i})\mathbbm{1}\{i_t=i\}}{w_{t,i}}+m_{t,i}$, and $\eta_{t,i}=\eta\leq \frac{1}{162}$ guarantees 
\begin{align*}
\mathbb{E}\left[\reg_T\right]\leq\mathcal{O}\left(\frac{K\ln T}{\eta}\right) + 6\eta\sum_{i=1}^K V_{T,i} -\mathbb{E}\left[\sum_{t=2}^{T} \sum_{i=1}^K\frac{(w_{t,i}-w_{t-1,i})^2}{48\eta w_{t-1,i}^2} \right]\leq \mathcal{O}\left( \frac{K\ln T}{\eta} + \eta\sum_{i=1}^K V_{T,i}  \right). 
\end{align*}
Using the doubling trick (Algorithm~\ref{alg:doubling}), we achieve expected regret $\tilde{\mathcal{O}}\left(\sqrt{K\sum_{i=1}^K V_{T,i}} + K\right)$.
\end{cor}

%Similarly to the discussion in the introduction, 
This new path-length bound could be $\sqrt{K}$ times better than the one in Section~\ref{subsubsection:path-length} in some cases,
but $\sqrt{T}$ times larger in others.
The extra advantage, however, is the negative term in the regret,\footnote{%
In fact, similar negative term, coming from the term $A_t$ in Lemma~\ref{thm:general_instantaneous}, also exists (but is omitted) in the bound of Theorem~\ref{thm:path_length}.
However, it is not clear to us how to utilize it in the same way as in Section~\ref{subsection:games} if we also want to exploit the other negative term coming from increasing learning rates.
} 
explicitly spelled out in Corollary~\ref{cor:path_length_bound_1},
which we discuss next.

\subsubsection{Fast convergence in bandit games}
\label{subsection:games}

It is well-known that in a repeated two-player zero-sum game, 
if both players play according to some no-regret algorithms,
then their average strategies converge to a Nash equilibrium~\citep{freund1999adaptive}.
Similar results for general multi-player games have also been discovered.
The convergence rate of these results is governed by the regret bounds of the learning algorithms,
and several recent works (such as those mentioned in the introduction) have developed adaptive algorithms with regret much smaller than the worst case $\mathcal{O}(\sqrt{T})$ 
by exploiting the special structure in this setup,
which translates to convergence rates faster than $1/\sqrt{T}$ in computing equilibriums.

%Here we investigate the application of \textsc{Broad-OMD} with path-length bound in multi-player repeated games. The game proceeds in rounds: in each round, every player takes an action and receives some utility, which is jointly determined by the actions of all players in that round. The goal of each player is to maximize his/her own accumulated utility. 

%It is known that if all players use no-regret algorithms to play the game, their average strategies converge to some sort of equilibrium, with the convergence rate governed by the regret bounds. One question that arises is whether there exists some family of algorithms such that if all players use algorithms from that family, the convergence can be faster. Researches in this line can be found in, e.g., \citep{daskalakis2015near, rakhlin2013optimization, syrgkanis2015fast, foster2016learning}. 

One way to obtain such fast rates is exactly via path-length regret bounds as shown in~\citep{rakhlin2013optimization, syrgkanis2015fast}. In these works, the convergence rate $1/T$ is achieved when the players have full-information feedback. 
We generalize their results to the bandit setting, and show that convergence rate of $ 1/T^{\frac{3}{4}} $ can be obtained. Though faster than $1/\sqrt{T}$, it is still slower than $1/T$ compared to the full-information setting, which is due to the fact that in bandit we only have first-order instead of second-order path-length bound. We detail the proofs and the remaining open problems in Appendix~\ref{appendix:game}. 
%For simplicity we only consider two-player zero-sum games, but the generalization to multi-player games is straightforward.

%While there are still many open problems in this area, especially when the players receive bandit feedback, our \textsc{Broad-OMD} with path-length bound indeed provides faster convergence rate in this kind of games. For simplicity, below we consider only two-player zero-sum repeated games. 

%Specifically, the game is defined by an unknown matrix $G\in[-1,1]^{M\times N}$
%where entry $G(i,j)$ specifies the loss (or reward) for Player 1 (or Player 2) if Player 1 picks row $i$ while Player 2 picks column $j$.
%The players play the game repeatedly for $T$ rounds.
%At round $t$, Player 1 randomly picks a row $i_t \sim x_t$ for some $x_t \in \Delta_M$
%while Player 2 randomly picks a column $j_t \sim y_t$ for some $y_t \in \Delta_N$.
%In~\citep{syrgkanis2015fast}, the feedbacks they receive are the vectors $Gy_t$ and $x_t^\top G$ respectively.
%As a natural extension to the bandit setting, we consider a setting where the feedbacks are the scalar values $\mathbf{e}_{i_t}^\top Gy_t$
%and $x_t^\top G\mathbf{e}_{j_t}$ respectively, that is, the expected loss/reward for the players' own realized actions (over the opponent's randomness). 

%It is clear that each player is essentially facing an MAB problem and thus can employ an MAB algorithm.
%Specifically, if both players apply Exp3 for example, their expected average strategies converge to a Nash equilibrium at rate $1/\sqrt{T}$.
%However, if instead Player 1 applies \textsc{Broad-OMD} configured as in Corollary~\ref{cor:path_length_bound_1},
%then her regret has a path-length term that can be bounded as follows:
%\begin{align*}
%\sum_{i=1}^K \sum_{t=2}^T\left| \mathbf{e}_{i}^\top Gy_t -  \mathbf{e}_{i}^\top Gy_{t-1}\right|
%\leq \sum_{i=1}^K \sum_{t=2}^T\left\| \mathbf{e}_{i}^\top G \right\|_\infty \|y_t - y_{t-1}\|_1 \leq K \sum_{t=2}^T \|y_t - y_{t-1}\|_1,
%\end{align*}
%which is closely related to the negative regret term in Corollary~\ref{cor:path_length_bound_1}
%for Player 2 if she also employs the same \textsc{Broad-OMD}.
%The cancellation of these terms then lead to faster convergence rate.
%Under this setting, we can define the regret for the two players: 
%\begin{gather*}
%\text{Reg}_T^1 \triangleq \sum_{t=1}^T \mathbf{e}_{i_t}^\top G\mathbf{e}_{j_t} -\sum_{t=1}^T\min_{x\in \Delta_M} x^\top G\mathbf{e}_{j_t},\\
%\text{Reg}_T^2 \triangleq \sum_{t=1}^T \max_{y\in \Delta_N} \mathbf{e}_{i_t}^\top Gy -\sum_{t=1}^T \mathbf{e}_{i_t}^\top G\mathbf{e}_{j_t}.
%\end{gather*}
%The following theorem show that if the two players both run \textsc{Broad-OMD} with $a_t=\mathbf{0}$ and use $\mathbf{e}_{i_t}^\top Gy_t$ or $x_t^\top G\mathbf{e}_{j_t}$ as their loss/reward at round $t$, then the convergence rate is faster than $\tilde{\Theta}(T^{-\frac{1}{2}})$. 
%\begin{theorem}
%\label{thm:fast_convergence_theorem}
%For the setting described above, if both players run \textsc{Broad-OMD} configured as in Corollary~\ref{cor:path_length_bound_1} except that $\eta_{t,i}=\eta= (M+N)^{-\frac{1}{4}}T^{-\frac{1}{4}}$, then their expected average strategies converge to Nash equilibriums at the rate of $\tilde{\mathcal{O}}\left((M+N)^{\frac{5}{4}}/T^{\frac{3}{4}}\right)$, that is,
%\begin{align*}
%\max_{y\in \Delta_N} \mathbb{E}[\bar{x}]^\top Gy \leq \text{\rm Val} + \tilde{\mathcal{O}}((M+N)^{\frac{5}{4}}/T^{\frac{3}{4}}) \quad\text{and}\quad
%\min_{x\in \Delta_M}x^\top G\mathbb{E}[\bar{y}] \geq \text{\rm Val} - \tilde{\mathcal{O}}((M+N)^{\frac{5}{4}}/T^{\frac{3}{4}}),
%\end{align*}
%where $\bar{x}=\frac{1}{T}\sum_{t=1}^T x_t, \bar{y}=\frac{1}{T}\sum_{t=1}^T y_t$ and 
%$\text{\rm Val}= \min\limits_{x\in \Delta_M}\max\limits_{y\in \Delta_N} x^\top Gy = \max\limits_{y\in \Delta_N}\min\limits_{x\in \Delta_M} x^\top Gy$.
%\end{theorem}

%When one player uses \textsc{Broad-OMD} while the other uses a \textit{stable} algorithm (defined in the following theorem), the one using \textsc{Broad-OMD} can have expected regret better than $\tilde{\mathcal{O}}(\sqrt{T})$. 

%\begin{theorem}
%\label{theorem:better_for_stable}
%Under the described setting, if Player 1 uses \textsc{Broad-OMD} with $a_t=\mathbf{0}$, while Player 2 uses a $\kappa$-stable algorithm, that is, an algorithm with $\norm{y_t-y_{t-1}}_1=\mathcal{O}(\kappa)$ for all $t$. Then Player 1 with learning rate $\tilde{\Theta}(\kappa^{-\frac{1}{2}}T^{-\frac{1}{2}})$ guarantees $\mathbb{E}[\text{Reg}_T^1]=\tilde{\mathcal{O}}(M\sqrt{T\kappa})$. 
%\end{theorem}
%For example, if Player 2 uses the vanilla Exp3 with learning rate proportional to $\frac{1}{\sqrt{T}}$, then $\kappa=\mathcal{O}\left(\frac{1}{\sqrt{T}}\right)$, and Player 1 can have $\tilde{\mathcal{O}}(T^{\frac{1}{4}})$ regret by selecting $\tilde{\Theta}(T^{-\frac{1}{4}})$ learning rate; if both players runs \textsc{Broad-OMD} with $\tilde{\Theta}(T^{-\frac{1}{3}})$ learning rate, then they can both achieve $\tilde{\mathcal{O}}(T^{\frac{1}{3}})$ regret (omitting the dependency on $M$ or $N$ for simplicity).  

%As shown by the theorem, we obtain convergence rate faster than $1/\sqrt{T}$,
%but still slower than the $1/T$ rate compared to the full-information setup of~\citep{syrgkanis2015fast},
%due to the fact that we only have first-order instead of second-order path-length bound.

%Note that~\citet{rakhlin2013optimization} also studies two-player zero-sum games with bandit feedback
%but with an unnatural restriction that in each round the players play the same strategy for four times.
%\citet{foster2016learning} greatly weakened the restriction, but their algorithm only converges to some approximation of Val.
%For further comparisons, the readers are referred to the comparisons to~\citep{syrgkanis2015fast}
%in \citep{foster2016learning}.
%^We also point out that the question raised in \citep{rakhlin2013optimization} remains open: if the players only receive the realized loss/reward $\mathbf{e}_{i_t}^\top G\mathbf{e}_{j_t}$ as feedback (a more natural setup), can the convergence rate to Val be faster than $1/\sqrt{T}$?
%When every player only receives bandit feedbacks (i.e., in each round a player only observes the utility/expected utility corresponding to the action he/she takes), this problem is less understood. \cite{rakhlin2013optimization} proposed an algorithm in two-player zero-sum games such that if both players use this algorithm, their convergence rate can be of order $\tilde{\mathcal{O}}(1/T)$. However, they have the strong assumption that in each round the players play the same strategy for four times, and each time observe the expected utility as the feedback. \cite{foster2016learning} greatly weakened the above assumption, requiring that in each round, every player only plays once and only receives the realized utility as the feedback; however, the convergence notion they defined is different from previous works: they considered the convergence to an \textit{approximate} equilibrium. Below we give another result for the bandit feedback scenario and establish a $\tilde{O}(1/T^{\frac{3}{4}})$ convergence rate, which is incomparable to previous results. For simplicity, we consider only two-player zero-sum games. The convergence notion we use is more similar to \citep{rakhlin2013optimization, syrgkanis2015fast}'s, that is, convergence to the exact game value or exact minimax/maximin strategies. 

\subsection{Adapting to Stochastic Bandits}
\label{section:best of both worlds}
Our last application is to obtain an algorithm that simultaneously enjoys near optimal regret in both adversarial and stochastic setting. 
Specifically, the stochastic setting we consider here is as follows: there exists an arm $a^*$ and some fixed gap $\Delta > 0$ such that 
$\mathbb{E}_{\ell_t}\left[\ell_{t,i}-\ell_{t,a^*} | \ell_1, \ldots, \ell_{t-1}\right]\geq \Delta$ for all $i \neq a^*$ and $t\in[T]$.
In other words, arm $a^*$'s expected loss is always smaller than those of other arms by a fixed amount.
The classic i.i.d. MAB~\citep{lai1985asymptotically} is clearly a special case of ours.
Unlike the i.i.d. setting, however, we require neither independence nor identical distributions.

Note that $a^*$ can be different from the empirically best arm $i^*$ defined in Section~\ref{section:notations}. 
The expected regret in this setting is still with respect to $i^*$ and further takes into consideration the randomness over losses. 
In other words, we care about $\mathbb{E}_{\ell_1, \ldots, \ell_T}\left[\mathbb{E}_{i_1, \ldots, i_T}[\text{Reg}_T]\right]$, abbreviated as $\mathbb{E}[\text{Reg}_T]$ still.  

We invoke \textsc{Broad-OMD} with $a_t=\mathbf{0}$, $\hat{\ell}_{t,i}=\frac{\ell_{t,i}\mathbbm{1}\{i_t=i\}}{w_{t,i}}$ being the typical importance-weighted unbiased estimator,
and a somewhat special choice of $m_{t}$: $m_{t,i}=\ell_{t,i_t}$ for all $i$. 
%This $\hat{\ell}_{t}$ is just the vanilla inverse propensity weighted estimator without the prediction term, and therefore it is still an unbiased estimator of $\ell_{t}$. The more curious part is $m_{t}$. 
This choice of $m_t$ is seemingly invalid since it depends on $i_t$, which is drawn after we have constructed $w_t$ based on $m_t$ itself.
However, note that because $m_t$ now has identical coordinates, we have
$w_t = \argmin_{w\in\Delta_K} \big\{ \inner{w,m_t} + D_{\psi_t}(w, w_t^\p)\big\} = \argmin_{w\in\Delta_K} \big\{D_{\psi_t}(w, w_t^\p)\big\} = w_t'$, independent of the actual value of $m_t$.
Therefore, the algorithm is still valid and is in fact equivalent to the vanilla log-barrier OMD of~\citep{foster2016learning}.
Also note that we cannot define $\hat{\ell}_{t}$ as in previous sections (in terms of $m_t$) since it is not an unbiased estimator of $\ell_t$ anymore (due to the randomness of $m_t$).

%In \textsc{Broad-OMD}, $m_{t}$ is used to construct $w_{t}$, and then $i_t$ is further drawn based on $w_t$. Therefore, it seems unreasonable to define $m_t$ based on $i_t$. However, with this special choice of $m_t$, we claim that without knowing $i_t$, the learner can still successfully construct the correct $w_t$. This is simply because $m_t$'s components are all the same, so $w_t$ will always be the same as $w_t^\p$. We more formally prove this fact in Lemma \ref{lemma:same_point}. Therefore, the player just plays $w_t^\p$, and the algorithm proceeds just like the vanilla log-barrier OMD! So what is the benefit of this set of choices? 

Although the algorithm is the same, using our analysis framework we actually derive a tighter bound in terms of the following quantity based on
Theorem~\ref{lemma:second_order_regret_bound}:
$\sum_{t=1}^T\sum_{i=1}^K w_{t,i}^2(\hat{\ell}_{t,i}-\ell_{t,i_t})^2=\sum_{t=1}^T\sum_{i=1}^K (\ell_{t,i}\mathbbm{1}\{i_t=i\}-w_{t,i}\ell_{t,i_t})^2$.
It turns out that based on this quantity alone, one can derive both a ``small-loss'' bound for the adversarial setting and a logarithmic bound for the stochastic setting
as shown below.
We emphasize that the doubling trick of Algorithm~\ref{alg:doubling} is essential to make the algorithm parameter-free,
which is another key difference from~\citep{foster2016learning}.

%It turns out the benefit is in the analysis: the introduction of this $m_{t}$ tightens the regret bound for log-barrier OMD for free! Note that Theorem \ref{lemma:second_order_regret_bound} is still valid, and now the regret depends on a term $\sum_{t=1}^T\sum_{i=1}^K w_{t,i}^2(\hat{\ell}_{t,i}-\ell_{t,i_t})^2=\sum_{t=1}^T\sum_{i=1}^K (\ell_{t,i}\mathbbm{1}\{i_t=i\}-w_{t,i}\ell_{t,i_t})^2$.The following theorem shows that the regret having this term has both-of-both-world implications. 

\begin{theorem}
\label{thm:best of both}
\textsc{Broad-OMD} with $a_t = 0$, $m_{t,i}=\ell_{t,i_t}$, $\hat{\ell}_{t,i}=\frac{\ell_{t,i}\mathbbm{1}\{i_t=i\}}{w_{t,i}}$, and
the doubling trick (Algorithm~\ref{alg:doubling}), guarantees 
\begin{equation}\label{eqn:new_excess_loss_bound}
\mathbb{E}\left[\reg_T\right]=\mathcal{O}\left(\sqrt{(K\ln T)\mathbb{E}\left[ \sum_{t=1}^T\sum_{i=1}^K (\ell_{t,i}\mathbbm{1}\{i_t=i\}-w_{t,i}\ell_{t,i_t})^2 \right]} + K\ln T \right).
\end{equation}
This bound implies that in the stochastic setting, we have $\mathbb{E}\left[\reg_T\right] = \mathcal{O}\left(\frac{K\ln T}{\Delta}\right)$, while in the adversarial setting, we have 
%$\mathbb{E}\left[\reg_T\right] = \mathcal{O}\left(\sqrt{KL\ln T} + K\ln T\right)$, where $L\triangleq \mathbb{E}\left[\sum_{t=1}^T \ell_{t,i_t}^2 \right]$. %\leq \sum_{t=1}^T \norm{\ell_t}_\infty^2$. 
$\mathbb{E}\left[\reg_T\right] = \mathcal{O}\left(\sqrt{KL_{T,i^*}\ln T}+K\ln T\right)$ assuming non-negative losses.
%If we further assume losses are non-negative, i.e., $\ell_{t,i}\in [0,1]$ for all $t,i$, then we further have the small-loss bound $\mathcal{O}\left(\sqrt{KL_{T,i^*}\ln T}+K\ln T\right)$ in the adversarial setting. 

\end{theorem}

\section{Conclusions and Discussions}
In this work we develop and analyze a general bandit algorithm using techniques such as optimistic mirror descent, log-barrier regularizer, increasing learning rate, and so on.
We show various applications of this general framework, obtaining several more adaptive algorithms that improve previous works.
Future directions include 1) improving the dependence on $K$ for the path-length results; 2) obtaining second-order path-length bounds;
3) generalizing the results to the linear bandit problem.

\paragraph{Acknowledgement.}
CYW is grateful for the support of NSF Grant \#1755781. The authors would like to thank Chi-Jen Lu for posing the problem of bandit path-length, and to thank Chi-Jen Lu and Yi-Te Hong for helpful discussions in this direction. 

\bibliography{colt2018-sample} 

%\section*{Acknowledgement}

%Y. Fei and Y. Chen were partially supported by the National Science
%Foundation CRII award 1657420 and grant 1704828.


\appendix
%\appendixpage

\section{Additional notations}

We define the shorthand $\error\coloneqq\norm[\Yhat-\Ystar]1$. For
a matrix $\M$, we write $\norm[\M]{\infty}\coloneqq\max_{i,j}\left|M_{ij}\right|$
as its entry-wise $\ell_{\infty}$ norm, and $\opnorm{\M}$ as its
spectral norm (maximum singular value).  We let $\I$ and $\OneMat$
be the $\num\times\num$ identity matrix and all-one matrix, respectively.
For a real number $x$, $\left\lceil x\right\rceil $ denotes its
ceiling. We denote by $\clustset a\coloneqq\left\{ i\in\left[\num\right]:\labelstar(i)=a\right\} $
the set of indices of points in cluster $a$, and we define $\size\coloneqq\left|\clustset a\right|=\frac{\num}{\numclust}$. 

\section{Proof of Theorem \ref{thm:ip_sdp}\label{sec:proof_ip_sdp}}

\subsection{Initial steps}

We assume $\error>0$ since otherwise we are done. We can write $\Adj=\C+\C\t-2\H\H\t$,
where $\H$ is a matrix whose $i$-th row is the point $\h_{i}$ and
$\C$ is a matrix where the entries in the $i$-th row are identical
and equal to $\norm[\h_{i}]2^{2}$. Since the row-sum constraint in
the program (\ref{eq:SDP1}) ensures that the matrix $\Yhat-\Ystar$
has zero row sum, we have $\left\langle \Yhat-\Ystar,\C\right\rangle =\left\langle \Yhat-\Ystar,\C\t\right\rangle =0$
which implies $\left\langle \Yhat-\Ystar,\C+\C\t\right\rangle =0$.

Let $\G\coloneqq\H-\E\H$ be a matrix of entries in $\H$ with their
means removed. We can compute
\begin{align*}
\H\H\t & =\left(\G+\E\H\right)\left(\G+\E\H\right)\t\\
 & =\G\G\t+\G\left(\E\H\right)\t+\left(\E\H\right)\G\t+\left(\E\H\right)\left(\E\H\right)\t
\end{align*}
and 
\[
\E\H\H^{\top}=\E\G\G\t+\left(\E\H\right)\left(\E\H\right)\t.
\]
Therefore 
\[
\H\H\t-\E\H\H^{\top}=\left(\G\G\t-\E\G\G\t\right)+\G\left(\E\H\right)\t+\left(\E\H\right)\G\t.
\]
Let $\U\in\real^{\num\times\numclust}$ be the matrix of the left
singular vectors of $\Ystar$. For any $\M\in\real^{\num\times\num}$,
define the projection $\PT\left(\M\right)\coloneqq\U\U\t\M+\M\U\U\t-\U\U\t\M\U\U\t$
and its orthogonal complement $\PTperp\left(\M\right)\coloneqq\M-\PT\left(\M\right)$.
The fact that $\Yhat$ is optimal and $\Ystar$ is feasible to the
program (\ref{eq:SDP1}) implies 
\begin{align*}
0 & \leq-\frac{1}{2}\left\langle \Yhat-\Ystar,\Adj\right\rangle \\
 & =\left\langle \Yhat-\Ystar,\H\H\t-\E\H\H^{\top}\right\rangle +\left\langle \Yhat-\Ystar,\E\H\H^{\top}\right\rangle \\
 & =\left\langle \Yhat-\Ystar,\G\G\t-\E\G\G\t+\G\left(\E\H\right)\t+\left(\E\H\right)\G\t\right\rangle +\left\langle \Yhat-\Ystar,\E\H\H^{\top}\right\rangle \\
 & =\left\langle \Yhat-\Ystar,\PT\left(\G\G\t-\E\G\G\t\right)\right\rangle +\left\langle \Yhat-\Ystar,\PTperp\left(\G\G\t-\E\G\G\t\right)\right\rangle \\
 & \quad+2\left\langle \Yhat-\Ystar,\G\left(\E\H\right)\t\right\rangle +\left\langle \Yhat-\Ystar,\E\H\H^{\top}\right\rangle \\
 & \eqqcolon S_{1}+S_{2}+2S_{3}+S_{4}.
\end{align*}
We may control $S_{1}$, $S_{2}$ and $S_{4}$ using the following. 
\begin{prop}
\label{prop:S1} If $\snr^{2}\geq C\left(\sqrt{\frac{\numclust\vecdim}{\num}\log\left(\num\numclust\right)}+\sqrt{\frac{\numclust}{\num}}\log\left(\num\numclust\right)\right)$
for some universal constant $C>0$, then $S_{1}\leq\frac{1}{100}\minsep^{2}\error$
with probability at least $1-\left(2\num\right)^{-2}$.
\end{prop}

\begin{prop}
\label{prop:S2} If $\snr^{2}\geq C\numclust\left(\sqrt{\frac{\vecdim}{\num}}+1\right)$
for some universal constant $C>0$, then $S_{2}\leq\frac{1}{100}\minsep^{2}\error$
with probability at least $1-2e^{-\num}$.
\end{prop}

\begin{prop}
\label{prop:S4} We have $S_{4}=-\frac{1}{2}\sum_{a\ne b}T_{ab}\minsep_{ab}^{2}\le-\frac{1}{4}\minsep^{2}\error$
where $T_{ab}\coloneqq\sum_{i\in\clustset a,j\in\clustset b}\left(\Yhat-\Ystar\right)_{ij}$. 
\end{prop}
The proofs are given in Sections \ref{sec:proof_S1}, \ref{sec:proof_S2}
and \ref{sec:proof_S4}, respectively. Combining the above propositions,
we have $S_{1}+S_{2}\le-\frac{1}{2}S_{4}$ and therefore 
\begin{equation}
0\leq S_{3}+\frac{1}{4}S_{4}\eqqcolon S_{0}\label{eq:error_S3_bound}
\end{equation}
with probability at least $1-\left(2\num\right)^{-C'}-2e^{-\num}$
for some universal constant $C'>0$.

Let $\B\coloneqq\Yhat-\Ystar$. We have 
\begin{align*}
S_{3} & =\sum_{j}\sum_{a}\sum_{i\in C_{a}}B_{ji}\left\langle \Mean_{a},\g_{j}\right\rangle \\
 & =\size\sum_{j}\sum_{a}\left\langle \Mean_{a},\g_{j}\right\rangle \left(\frac{1}{\size}\sum_{i\in\clustset a}B_{ji}\right)\\
 & =\size\sum_{j}\sum_{a\ne\labelstar(j)}\left\langle \Mean_{a}-\Mean_{\labelstar(j)},\g_{j}\right\rangle \left(\frac{1}{\size}\sum_{i\in\clustset a}B_{ji}\right)
\end{align*}
where the last step holds since $\sum_{a\ne\labelstar(j)}\left(\sum_{i\in\clustset a}B_{ji}\right)=-\sum_{i\in\clustset a:a=\labelstar(j)}B_{ji}$
for each $j\in\left[\num\right]$ which follows from the row-sum constraint
of program (\ref{eq:SDP1}). By Proposition \ref{prop:S4}, we have
\begin{align*}
S_{4} & =-\size\sum_{j}\sum_{a\ne\labelstar(j)}\frac{1}{2}\minsep_{\labelstar(j),a}^{2}\left(\frac{1}{\size}\sum_{i\in\clustset a}B_{ji}\right).
\end{align*}
Therefore, we have 
\[
S_{0}=\size\sum_{j}\sum_{a\ne\labelstar(j)}\left(\left\langle \Mean_{a}-\Mean_{\labelstar(j)},\g_{j}\right\rangle -c\minsep_{\labelstar(j),a}^{2}\right)\left(\frac{1}{\size}\sum_{i\in\clustset a}B_{ji}\right)
\]
where $c=\frac{1}{8}$.

To control $S_{0}$, we define $\beta_{ja}\coloneqq\left\langle \Mean_{a}-\Mean_{\labelstar(j)},\g_{j}\right\rangle -c\minsep_{\labelstar(j),a}^{2}$
and consider the program 
\begin{align}
\max_{\X}\  & \sum_{j}\sum_{a\ne\labelstar(j)}\beta_{ja}X_{ja}\nonumber \\
\text{s.t.}\  & 0\leq X_{ja}\leq1,\qquad\forall a\ne\labelstar(j),j\in\left[\num\right]\nonumber \\
 & \sum_{a\ne\labelstar(j)}X_{ja}\leq1,\qquad\forall j\in\left[\num\right]\label{eq: int_opt}\\
 & \sum_{j}\sum_{a\ne\labelstar(j)}X_{ja}=R,\nonumber 
\end{align}
where $R\in(0,\num]$. Let us denote by $V(R)$ the optimal value
of the above program and we let $V(R)=-\infty$ if the program is
infeasible. The constraints of program (\ref{eq:SDP1}) implies that
$\frac{\error}{2\size}\in(0,\num]$ and 
\[
\sum_{j\in\left[\num\right]}\sum_{a\ne\labelstar(j)}\left(\sum_{i\in\clustset a}B_{ji}\right)=\frac{\error}{2}.
\]
Hence, by Equation (\ref{eq:error_S3_bound}), we have 
\begin{equation}
0\leq S_{0}\leq\size\cdot V\left(\frac{\error}{2\size}\right).\label{eq:basic_ineq_upper_bound_V}
\end{equation}


\subsection{Controlling $\protect\error$ by LP}

We show that $\error$ is upper bounded by the objective value of
an LP that is related to program (\ref{eq: int_opt}). If $\error=0$
then the conclusion of Theorem \ref{thm:ip_sdp} holds trivially.
For $\error>0$, we have the following cases:
\begin{enumerate}
\item If $\frac{\error}{2\size}\in(0,1]$, it follows from Equation (\ref{eq:basic_ineq_upper_bound_V})
that the error $\error$ must satisfy 
\[
0\le V\left(\frac{\error}{2\size}\right)=\beta^{*}\frac{\error}{2\size}\le\beta^{*}\left\lceil \frac{\error}{2\size}\right\rceil =V\left(\left\lceil \frac{\error}{2\size}\right\rceil \right)
\]
where $\beta^{*}\coloneqq\max_{j\in\left[\num\right],a\ne\labelstar(j)}\beta_{ja}$.
This implies 
\begin{align*}
\frac{\error}{2\size}\le\left\lceil \frac{\error}{2\size}\right\rceil  & \le\max\left\{ R\in\{0,1,.\ldots\}:V(R)\ge0\right\} .
\end{align*}
\item If $\frac{\error}{2\size}>1$, it follows from Equation (\ref{eq:basic_ineq_upper_bound_V})
that the error $\error$ must satisfy 
\[
0\le V\left(\frac{\error}{2\size}\right)\le\max\left\{ V\left(\left\lceil \frac{\error}{2\size}\right\rceil \right),V\left(\left\lfloor \frac{\error}{2\size}\right\rfloor \right)\right\} =\max\left\{ V\left(\left\lceil \frac{\error}{2\size}\right\rceil \right),V\left(\left\lceil \frac{\error}{2\size}\right\rceil -1\right)\right\} .
\]
In other words, we have
\begin{align*}
\frac{\error}{2\size}\le\left\lceil \frac{\error}{2\size}\right\rceil  & \le\max\left\{ R\in\{0,1,.\ldots\}:V(R)\vee V(R-1)\ge0\right\} \\
 & =1+\max\left\{ R\in\{0,1,.\ldots\}:V(R)\ge0\right\} .
\end{align*}
Note that $\left\lceil \frac{\error}{2\size}\right\rceil \ge2$, and
therefore we must have $1\le\max\left\{ R\in\{0,1,.\ldots\}:V(R)\ge0\right\} $.
This implies 
\[
\frac{\error}{2\size}\le2\max\left\{ R\in\{0,1,.\ldots\}:V(R)\ge0\right\} .
\]
\end{enumerate}
Consequently, we have 
\[
\frac{\error}{2\size}\le2\max\left\{ R\in\{0,1,.\ldots\}:V(R)\ge0\right\} .
\]


\subsection{Converting LP to IP}

We are now ready to formally establish a connection between the error
of the SDP (\ref{eq:SDP1}) and that of the Oracle IP (\ref{eq:oracleIP}),
by relating $\max\left\{ R\in\{0,1,.\ldots\}:V(R)\ge0\right\} $ to
the quantity (\ref{eq:IPerror}). Note that if $R\ge0$ is an integer,
then there exists an optimal solution $\left\{ w_{ja}\right\} $ of
program (\ref{eq: int_opt}) such that $w_{ja}\in\{0,1\}$ for all
$j\in[\num],a\in[\numclust]$. Therefore, if $R\in\{0,1,\ldots\}$
is an integer, then 
\begin{equation}
V(R)=\IP_{1}(R)\coloneqq\left\{ \begin{aligned}\max_{\X}\  & \sum_{j}\sum_{a\ne\labelstar(j)}\beta_{ja}X_{ja}\\
\text{s.t.}\  & X_{ja}\in\{0,1\},\qquad\forall a\ne\labelstar(j),j\in\left[\num\right]\\
 & \sum_{a\ne\labelstar(j)}X_{ja}\leq1,\qquad\forall j\in\left[\num\right]\\
 & \sum_{j}\sum_{a\ne\labelstar(j)}X_{ja}=R
\end{aligned}
\right\} .\label{eq:IP1}
\end{equation}
Combining the last two display equations we obtain that
\begin{align}
\frac{\error}{2\size} & \le2\max\left\{ R\in\{0,1,.\ldots\}:\IP_{1}(R)\ge0\right\} \nonumber \\
 & \overset{}{=}2\cdot\left\{ \begin{aligned}\max_{R,\X}\; & R\\
\text{s.t.}\; & R\in\{0,1,\ldots\}\\
 & \sum_{j}\sum_{a\ne\labelstar(j)}\beta_{ja}X_{ja}\ge0\\
 & X_{ja}\in\{0,1\},\qquad\forall a\ne\labelstar(j),j\in\left[\num\right]\\
 & \sum_{a\ne\labelstar(j)}X_{ja}\leq1,\qquad\forall j\in\left[\num\right]\\
 & \sum_{j}\sum_{a\ne\labelstar(j)}X_{ja}=R,
\end{aligned}
\right\} \nonumber \\
 & =2\cdot\IP_{2}\coloneqq2\cdot\left\{ \begin{aligned}\max_{\X}\; & \sum_{j}\sum_{a\ne\labelstar(j)}X_{ja}\\
\text{s.t.}\; & \sum_{j}\sum_{a\ne\labelstar(j)}\beta_{ja}X_{ja}\ge0\\
 & X_{ja}\in\{0,1\},\qquad\forall a\ne\labelstar(j),j\in\left[\num\right]\\
 & \sum_{a\ne\labelstar(j)}X_{ja}\leq1,\qquad\forall j\in\left[\num\right]
\end{aligned}
\right\} .\label{eq:error_bound2}
\end{align}

Let us reparameterize the integer program $\IP_{2}$ by a change of
variable. Recall that 
\[
\mathcal{F}\coloneqq\left\{ \F\in\{0,1\}^{\num\times\numclust}:\F\one_{\numclust}=\one_{\num}\right\} 
\]
is the set of all possible assignment matrices and $\F^{*}\in\mathcal{F}$
is the true assignment matrix; that is, $F_{ja}^{*}=\indic\left\{ a=\labelstar(j)\right\} $
for all $j\in[\num],a\in[\numclust]$. Consider any feasible solution
$\X$ of $\IP_{2}$; here for each $j\in[\num]$, we may fix $X_{j,\labelstar(j)}=-\sum_{a\neq\labelstar(j)}X_{ja}$
\textemdash{} doing so does not affect the feasibility and objective
value of $\X$ w.r.t. $\IP_{2}$. Define the new variable $\F\coloneqq\F^{*}+\X\in\mathcal{F}$.
The objective value and constraints of the old variable $\X$ can
be mapped to those of $\F$; in particular, we have 
\begin{align*}
\sum_{j}\sum_{a\ne\labelstar(j)}X_{ja} & =\sum_{j}\sum_{a\ne\labelstar(j)}(F_{ja}-F_{ja}^{*})=\frac{1}{2}\norm[\F-\F^{*}]1
\end{align*}
and
\begin{align*}
\left.\begin{array}{c}
X_{ja}\in\{0,1\},\forall a\ne\labelstar(j),j\in\left[\num\right]\\
\sum_{a\ne\labelstar(j)}X_{ja}\leq1,\forall j\in\left[\num\right]\\
X_{j,\labelstar(j)}=-\sum_{a\neq\labelstar(j)}X_{ja},\forall j\in[\num]
\end{array}\right\}  & \Longleftrightarrow\F\in\mathcal{F}
\end{align*}
and
\[
\sum_{j}\sum_{a\ne\labelstar(j)}\beta_{ja}X_{ja}\overset{(i)}{=}\sum_{j}\sum_{a}\beta_{ja}X_{ja}=\sum_{j}\sum_{a}\beta_{ja}F_{ja}-\sum_{j}\sum_{a}\beta_{ja}F_{ja}^{*}\overset{(ii)}{=}\sum_{j}\sum_{a}\beta_{ja}F_{ja},
\]
where steps $(i)$ and $(ii)$ both follow from the fact that $\beta_{j,\labelstar(j)}=0,\forall j.$
It follows that $\IP_{2}$ has the same optimal value as a corresponding
integer program in terms of $\X$; in particular, we have
\[
\IP_{2}=\IP_{3}\coloneqq\left\{ \begin{aligned}\max_{\F}\; & \frac{1}{2}\norm[\F-\F^{*}]1\\
\text{s.t.}\; & \sum_{j}\sum_{a}\beta_{ja}F_{ja}\ge0\\
 & \F\in\mathcal{F}
\end{aligned}
\right\} .
\]
Combining with equation (\ref{eq:error_bound2}), we see that the
error $\error$ satisfies
\begin{equation}
\frac{\error}{2\size}\le2\cdot\IP_{3}.\label{eq:error_bound3}
\end{equation}

We further simplify the first constraint in $\IP_{3}$. Recall that
$\bar{\h}_{i}\coloneqq\Mean_{\labelstar(i)}+(2c)^{-1}\g_{i}$ for
each $i\in[\num]$. Note that $\left\{ \bar{\h}_{i}\right\} $ can
be viewed as data points generated from the Sub-Gaussian Mixture Model
but with $(2c)^{-1}$ times the standard deviation. By definition
of $\beta_{ja}$, we have 
\begin{align*}
\beta_{ja} & =\left\langle \Mean_{a}-\Mean_{\labelstar(j)},\g_{j}\right\rangle -c\minsep_{\labelstar(j),a}^{2}\\
 & =c\left(2\left\langle \Mean_{a}-\Mean_{\labelstar(j)},(2c)^{-1}\g_{j}\right\rangle -\minsep_{\labelstar(j),a}^{2}\right)\\
 & =c\left(2\left\langle \Mean_{a}-\Mean_{\labelstar(j)},(2c)^{-1}\g_{j}\right\rangle -\norm[\Mean_{a}-\Mean_{\labelstar(j)}]2^{2}\right)\\
 & =c\left(2\left\langle \Mean_{a}-\Mean_{\labelstar(j)},(2c)^{-1}\g_{j}\right\rangle -\norm[\Mean_{a}-\Mean_{\labelstar(j)}]2^{2}-\norm[(2c)^{-1}\g_{j}]2^{2}+\norm[(2c)^{-1}\g_{j}]2^{2}\right)\\
 & =c\left(-\norm[\Mean_{\labelstar(j)}-\Mean_{a}+(2c)^{-1}\g_{j}]2^{2}+\norm[(2c)^{-1}\g_{j}]2^{2}\right)\\
 & =c\left(-\norm[\bar{\h}_{j}-\Mean_{a}]2^{2}+\norm[(2c)^{-1}\g_{j}]2^{2}\right).
\end{align*}
For any $\F\in\mathcal{F}$, we then have
\begin{align*}
\sum_{j}\sum_{a}\beta_{ja}F_{ja} & =c\sum_{j}\sum_{a}\left(-\norm[\bar{\h}_{j}-\Mean_{a}]2^{2}+\norm[(2c)^{-1}\g_{j}]2^{2}\right)F_{ja}\\
 & =c\left(-\sum_{j}\sum_{a}\norm[\bar{\h}_{j}-\Mean_{a}]2^{2}F_{ja}+\sum_{j}\norm[(2c)^{-1}\g_{j}]2^{2}\sum_{a}F_{ja}\right)\\
 & \overset{(i)}{=}c\left(-\sum_{j}\sum_{a}\norm[\bar{\h}_{j}-\Mean_{a}]2^{2}F_{ja}+\sum_{j}\norm[(2c)^{-1}\g_{j}]2^{2}\sum_{a}F_{ja}^{*}\right)\\
 & =c\left(-\sum_{j}\sum_{a}\norm[\bar{\h}_{j}-\Mean_{a}]2^{2}F_{ja}+\sum_{j}\sum_{a}\norm[(2c)^{-1}\g_{j}]2^{2}F_{ja}^{*}\right)\\
 & =c\left(-\sum_{j}\sum_{a}\norm[\bar{\h}_{j}-\Mean_{a}]2^{2}F_{ja}+\sum_{j}\sum_{a}\norm[\bar{\h}_{j}-\Mean_{\labelstar(j)}]2^{2}F_{ja}^{*}\right)\\
 & \overset{(ii)}{=}c\left(-\sum_{j}\sum_{a}\norm[\bar{\h}_{j}-\Mean_{a}]2^{2}F_{ja}+\sum_{j}\sum_{a}\norm[\bar{\h}_{j}-\Mean_{a}]2^{2}F_{ja}^{*}\right),
\end{align*}
where step $(i)$ holds because $\sum_{a}F_{ja}=1=\sum_{a}F_{ja}^{*},\forall j$,
and step $(ii)$ holds because $F_{ja}^{*}=1$ only if $a=\labelstar(j)$.
Again recall the shorthand
\[
\eta(\F)\coloneqq\sum_{j}\sum_{a}\norm[\bar{\h}_{j}-\Mean_{a}]2^{2}F_{ja}.
\]
We have the more compact expression
\begin{equation}
\sum_{j}\sum_{a}\beta_{ja}F_{ja}=c\left(\eta(\F^{*})-\eta(\F)\right)\label{eq:eta_func_equivalence}
\end{equation}
It follows that for any $\F\in\mathcal{F}$, the first constraint
in $\IP_{3}$ is satisfied if and only if 
\[
\eta(\F)\le\eta(\F^{*}).
\]
Combining with the (\ref{eq:error_bound3}), we obtain that
\[
\frac{\error}{2\size}\le2\cdot\IP_{3}=2\cdot\left\{ \begin{aligned}\max_{\F}\; & \frac{1}{2}\norm[\F-\F^{*}]1\\
\text{s.t.}\; & \eta(\F)\le\eta(\F^{*})\\
 & \F\in\mathcal{F}
\end{aligned}
\right\} .
\]
Rearranging terms, we have the bound
\begin{equation}
\error\le2\size\cdot\max\left\{ \norm[\F-\F^{*}]1:\eta(\F)\le\eta(\F^{*}),\F\in\mathcal{F}\right\} .\label{eq:error_bound3a}
\end{equation}
The result follows from the fact that $\norm[\Ystar]1=\num\size$
and $\norm[\F^{*}]1=\num$. 

\subsection{Proof of Proposition \ref{prop:S1} \label{sec:proof_S1}}

In this section we control $S_{1}$. We can further decompose $S_{1}$
as 
\begin{align*}
S_{1} & =\left\langle \Yhat-\Ystar,\U\U\t\left(\G\G\t-\E\G\G\t\right)\right\rangle +\left\langle \Yhat-\Ystar,\left(\G\G\t-\E\G\G\t\right)\U\U\t\right\rangle \\
 & \qquad-\left\langle \Yhat-\Ystar,\U\U\t\left(\G\G\t-\E\G\G\t\right)\U\U\t\right\rangle \\
 & \leq2\left|\left\langle \Yhat-\Ystar,\U\U\t\left(\G\G\t-\E\G\G\t\right)\right\rangle \right|+\left|\left\langle \Yhat-\Ystar,\U\U\t\left(\G\G\t-\E\G\G\t\right)\U\U\t\right\rangle \right|\\
 & \eqqcolon2T_{1}+T_{2}
\end{align*}
By the generalized Holder's inequality, we have 
\begin{align*}
T_{1} & \leq\error\cdot\norm[\U\U\t\left(\G\G\t-\E\G\G\t\right)]{\infty}
\end{align*}
and 
\begin{align*}
T_{2} & =\left|\left\langle \Yhat-\Ystar,\U\U\t\left(\G\G\t-\E\G\G\t\right)\U\U\t\right\rangle \right|\\
 & =\left|\left\langle \left(\Yhat-\Ystar\right)\U\U\t,\U\U\t\left(\G\G\t-\E\G\G\t\right)\right\rangle \right|\\
 & \leq\error\cdot\norm[\U\U\t\left(\G\G\t-\E\G\G\t\right)]{\infty}
\end{align*}
where the last inequality holds since 
\[
\norm[\left(\Yhat-\Ystar\right)\U\U\t]1\leq\norm[\Yhat-\Ystar]1=\error.
\]
Combining the above, we have 
\[
S_{1}\leq3\error\cdot\norm[\U\U\t\left(\G\G\t-\E\G\G\t\right)]{\infty}.
\]

Note that there are $m=\num\numclust$ distinct random variables in
$\U\U\t\left(\G\G\t-\E\G\G\t\right)$ and let us call them $X_{1},\ldots,X_{m}$.
For each $i$, we can see that $X_{i}$ is the average of $\size$
entries in $\G\G\t-\E\G\G\t$ and we let $\B_{i}$ be an $\num\times\num$
matrix with $\size$ entries equal to 1 and the others equal to 0
such that $\size X_{i}=\left\langle \B_{i},\G\G\t-\E\G\G\t\right\rangle $.
To proceed, we need the Hanson-Wright inequality (an extension of
Exercise 6.2.7 on pp.$\ $140 in \citet{vershynin2017high}).
\begin{lem}[Higher-dimensional Hanson-Wright inequality]
\emph{ \label{lem:hanson-wright} }Let $\x_{1},\ldots,\x_{N}$ be
independent, mean zero, sub-Gaussian random vectors in $\real^{M}$.
Let $\B$ be an $N\times N$ matrix. For every $t\geq0$ and some
universal constant $c>0$, we have 
\[
\P\left[\left|\sum_{i,j}B_{ij}\left\langle \x_{i},\x_{j}\right\rangle -\E\sum_{i,j}B_{ij}\left\langle \x_{i},\x_{j}\right\rangle \right|\geq t\right]\leq4\exp\left[-c\min\left(\frac{t^{2}}{K^{4}M\norm[\B]F^{2}},\frac{t}{K^{2}\norm[\B]{}}\right)\right]
\]
where $K\coloneqq\max_{i}\norm[\x_{i}]{\psi_{2}}.$ 
\end{lem}
The proof is given in Section \ref{sec:proof_hanson_wright}. Using
Lemma \ref{lem:hanson-wright}, we see that for any $t\ge0$ 
\[
\P\left\{ \size X_{i}\geq t\right\} =\P\left\{ \left\langle \B_{i},\G\G\t-\E\G\G\t\right\rangle \geq t\right\} \leq4\exp\left[-c\min\left(\frac{t^{2}}{K^{4}\vecdim\size},\frac{t}{K^{2}\sqrt{\size}}\right)\right].
\]
We can choose $t^{*}=DK^{2}\sqrt{\size}\left(\sqrt{\vecdim\log m}+\log m\right)$
with $K=\sgnorm$ and $D>0$ a universal constant. Apply the union
bound, we have 
\[
S_{1}\leq3\error\cdot\frac{1}{\size}\cdot t^{*}
\]
with probability at least $1-m\cdot\P\left\{ \size X\geq t\right\} \geq1-\exp\left(-C'\log m\right)=1-m^{-C'}$
where $C'>0$ is a universal constant. The result follows from the
condition of the proposition.

\subsection{Proof of Proposition \ref{prop:S2} \label{sec:proof_S2}}

In this section we control $S_{2}$. We have 
\begin{align*}
S_{2} & =\left\langle \PTperp\left(\Yhat-\Ystar\right),\G\G\t-\E\G\G\t\right\rangle \\
 & \leq\Tr\left[\PTperp\left(\Yhat-\Ystar\right)\right]\cdot\opnorm{\G\G\t-\E\G\G\t}\\
 & \le\frac{\error}{\size}\cdot\opnorm{\G\G\t-\E\G\G\t}.
\end{align*}
Let $\Var\left(g_{ij}\right)=\std^{2}$. We record a fact about the
sub-Gaussian property of columns of $\G$. 
\begin{fact}
\label{fact:satisfy_cond_gauss_choas_operator_norm_bound} Let $\x\in\real^{\num}$
be an arbitrary column of $\G$. We have 
\[
\norm[\left\langle \x,\w\right\rangle ]{\psi_{2}}\leq C\frac{\sgnorm}{\std}\sqrt{\E\left\langle \x,\w\right\rangle ^{2}}\qquad\text{for any }\w\in\real^{\num},
\]
where $C>0$ is a universal constant and $C\frac{\sgnorm}{\std}\ge1$.
\end{fact}
The proof is given in Section \ref{sec:proof_satisfy_cond}. Applying
Lemma \ref{lem:subg_cov_mat_bound} with $\rho_{0}=\frac{\sgnorm}{\std}$,
we have 
\[
\opnorm{\frac{1}{d}\G\G\t-\frac{1}{d}\E\G\G\t}\leq C_{1}\rho_{0}^{2}\left(\sqrt{\frac{2\num}{\vecdim}}+\frac{2\num}{\vecdim}\right)\opnorm{\frac{1}{d}\E\G\G\t}
\]
with probability at least $1-2e^{-\num}$. Here we let $m=\vecdim,u=\num$
and define $\x_{i}$ to be the $i$-th column of $\G$ and $\x$ to
be a vector independent of but identically distributed as each column of $\G$ (note that columns of $\G$ are identically
distributed). We also use the fact that $\E\x\x\t=\frac{1}{\vecdim}\E\G\G\t=\std^{2}\I$.
Multiplying $\vecdim$ on both sides of the above equation yields
\[
\opnorm{\G\G\t-\E\G\G\t}\leq C_{1}\left(\sqrt{\frac{2\num}{\vecdim}}+\frac{2\num}{\vecdim}\right)\vecdim\sgnorm^{2}.
\]
Hence, we have 
\[
S_{2}\le\frac{\error}{\size}\cdot C_{1}\left(\sqrt{\frac{2\num}{\vecdim}}+\frac{2\num}{\vecdim}\right)\vecdim\sgnorm^{2}=2C_{1}\error\numclust\left(\sqrt{\frac{\vecdim}{\num}}+1\right)\frac{\minsep^{2}}{\snr^{2}}
\]
The result follows from the condition of the proposition.

\subsection{Proof of Proposition \ref{prop:S4} \label{sec:proof_S4}}

We can compute 
\[
\left(\E\H\H\t\right)_{ij}=\begin{cases}
\vecdim\std^{2}+\norm[\Mean_{\labelstar(i)}]2^{2} & \text{if }i=j\\
\norm[\Mean_{\labelstar(i)}]2^{2} & \text{if }i\ne j\text{ and }\labelstar(i)=\labelstar(j)\\
\left\langle \Mean_{\labelstar(i)},\Mean_{\labelstar(j)}\right\rangle  & \text{otherwise}.
\end{cases}
\]
We partition the matrix $\Yhat-\Ystar$ into $\numclust^{2}$ of $\size\times\size$
blocks, and note that $T_{ab}$ denotes the sum of entries within
the $(a,b)$-th block. The constraints of program (\ref{eq:SDP1})
implies that 
\begin{enumerate}
\item $T_{aa}\leq0$ for each $a\in\left[\numclust\right]$ and $T_{ab}\geq0$
for each $a\ne b\in\left[\numclust\right]$;
\item $T_{ab}=T_{ba}$ for each $a,b\in\left[\numclust\right]$;
\item $-T_{aa}=\sum_{b\in\left[\numclust\right]:b\ne a}T_{ab}$ for each
$a\in\left[\numclust\right]$;
\item $-\sum_{a\in\left[\numclust\right]}T_{aa}+\sum_{a,b\in\left[\numclust\right]:a\ne b}T_{ab}=\error$
and thus $-\sum_{a\in\left[\numclust\right]}T_{aa}=\sum_{a,b\in\left[\numclust\right]:a\ne b}T_{ab}=\frac{\error}{2}$.
\end{enumerate}
Since $\Yhat-\Ystar$ has zero diagonal, we can write 
\begin{align*}
S_{4} & =\sum_{a\in\left[\numclust\right]}T_{aa}\norm[\Mean_{a}]2^{2}+2\sum_{a,b\in\left[\numclust\right]:a<b}T_{ab}\left\langle \Mean_{a},\Mean_{b}\right\rangle \\
 & =-\sum_{a,b\in\left[\numclust\right]:a<b}T_{ab}\minsep_{ab}^{2}\\
 & =-\frac{1}{2}\sum_{a,b\in\left[\numclust\right]:a\ne b}T_{ab}\minsep_{ab}^{2}\\
 & \leq-\frac{1}{2}\sum_{a,b\in\left[\numclust\right]:a\ne b}T_{ab}\minsep^{2}\\
 & =-\frac{1}{4}\minsep^{2}\error.
\end{align*}


\section{Proof of Theorem \ref{thm:ip_exp_rate}\label{sec:proof_ip_exp_rate}}

We define the shorthand 
\[
\iperror\coloneqq\max\left\{ \frac{1}{2}\norm[\F-\F^{*}]1:\eta(\F)\le\eta(\F^{*}),\F\in\mathcal{F}\right\} .
\]
It is not hard to see that $\iperror$ takes integer values in $[0,\num]$.
If $\iperror=0$ then we are done. We therefore focus on the case
$\iperror\in\left[\num\right]$.

Suppose $\iperror>3\num\numclust e^{-\snr^{2}/C_{0}^{2}}$ for a fixed
$C_{0}>D/c$. Note that 
\[
3\num\numclust e^{-\snr^{2}/C_{0}^{2}}\overset{(i)}{\le}\num\numclust\cdot\frac{1}{\numclust}\cdot e^{-\snr^{2}/\left(2C_{0}^{2}\right)}\le\num e^{-\snr^{2}/\left(2C_{0}^{2}\right)}<\num
\]
where step $(i)$ holds since we have assumed $\snr^{2}\ge\consts\numclust$
for some universal constant $\consts>0$. We record an important result
for our proof.
\begin{lem}
\label{lm:order_stats} Let $m\ge4$ and $g\ge1$ be integers. Let
$\X\in\real^{m\times g}$ be a matrix such that each $X_{ja}$ is
a sub-Gaussian random variable with its mean equal to $\lambda_{ja}$
and its sub-Gaussian norm no larger than $\rho_{ja}$, and each pair
$X_{ja}$ and $X_{ib}$ are independent for $j\ne i$ and $a,b\in\left[g\right]$.
Then for some universal constant $D>0$ and for any $\beta\in(0,m]$,
we have 
\begin{align*}
\sum_{j,a}X_{ja}M_{ja} & \le D\sqrt{\left\lceil \beta\right\rceil \left(\sum_{j,a}\rho_{ja}^{2}M_{ja}\right)\log\left(3mg/\beta\right)}+\sum_{j,a}\lambda_{ja}M_{ja},\\
 & \qquad\quad\forall\M\in\left\{ 0,1\right\} ^{m\times g}:\M\one_{g}\le\one_{m},\norm[\M]1=\left\lceil \beta\right\rceil ,
\end{align*}
with probability at least $1-\frac{1.5}{m}$.
\end{lem}
The proof is given in Section \ref{sec:proof_lm_order_stat}. Define
the set 
\[
\calM\coloneqq\left\{ \M\in\left\{ 0,1\right\} ^{\num\times\numclust}:\M\one_{\numclust}\le\one_{\num},\norm[\M]1=\iperror,M_{j,\labelstar(j)}=0\ \forall j\in\left[\num\right]\right\} .
\]
For any $\F$ feasible to $\IP_{3}$, we have 
\begin{align*}
0 & \le\frac{1}{c}\left(\eta(\F^{*})-\eta(\F)\right)\\
 & \overset{(i)}{=}\sum_{j\in[\num]}\sum_{a\in[\numclust]}\beta_{ja}F_{ja}\\
 & =\sum_{\left(j,a\right):F_{ja}=1,a\ne\labelstar(j)}\beta_{ja}\\
 & \le\max_{\M\in\calM}\sum_{j}\sum_{a\ne\labelstar(j)}\beta_{ja}M_{ja}\\
 & \overset{(ii)}{\le}\max_{\M\in\calM}\left[D\sqrt{\iperror\sgnorm^{2}\left(\sum_{j}\sum_{a\ne\labelstar(j)}\minsep_{\labelstar(j),a}^{2}M_{ja}\right)\log\left(3\num\left(\numclust-1\right)/\iperror\right)}-c\sum_{j}\sum_{a\ne\labelstar(j)}\minsep_{\labelstar(j),a}^{2}M_{ja}\right]\\
 & \le\max_{\M\in\calM}\left[D\sqrt{\iperror\sgnorm^{2}\left(\sum_{j}\sum_{a\ne\labelstar(j)}\minsep_{\labelstar(j),a}^{2}M_{ja}\right)\frac{\snr^{2}}{C_{0}^{2}}}-c\sum_{j}\sum_{a\ne\labelstar(j)}\minsep_{\labelstar(j),a}^{2}M_{ja}\right]\\
 & \le\left(\frac{D}{C_{0}}-c\right)\cdot\max_{\M\in\calM}\sum_{j}\sum_{a\ne\labelstar(j)}\minsep_{\labelstar(j),a}^{2}M_{ja}
\end{align*}
where step $(i)$ holds by Equation (\ref{eq:eta_func_equivalence}),
step $(ii)$ holds by Lemma \ref{lm:order_stats} with $g=\numclust-1$
since only $\numclust-1$ entries of $\left\{ \beta_{ja}\right\} $
are considered for each $j$ in the sum above $(ii)$, and the last
step holds since $\iperror\minsep^{2}\le\sum_{j}\sum_{a\ne\labelstar(j)}\minsep_{\labelstar(j),a}^{2}M_{ja}$.
Since $C_{0}>D/c$ and $\sum_{j}\sum_{a\ne\labelstar(j)}\minsep_{\labelstar(j),a}^{2}M_{ja}>0$,
the RHS above is negative, which is a contradiction. Hence, we must
have $\iperror\le3\num\numclust e^{-\snr^{2}/C_{0}^{2}}\le\num e^{-\snr^{2}/\left(2C_{0}^{2}\right)}$
and the result follows from the fact that $\norm[\F^{*}]1=\num$.

\section{Proof of technical results}

In this section we provide the proofs of the technical results used
in the proofs of our main theorems.

\subsection{Proof of Lemma \ref{lem:hanson-wright}\label{sec:proof_hanson_wright}}

We record the following lemma (Exercise 6.2.7 on pp.$\ $140 in \citet{vershynin2017high}).
\begin{lem}[Higher-dimensional Hanson-Wright inequality]
\emph{} \label{lem:hanson_wright_hdp} Let $\x_{1},\ldots,\x_{N}$
be independent, mean zero, sub-Gaussian random vectors in $\real^{M}$.
Let $\B=\left\{ B_{ij}\right\} $ be an $N\times N$ matrix. There
exists some universal constant $c>0$ such that for every $t\geq0$
\[
\P\left[\left|\sum_{i,j:i\ne j}^{N}B_{ij}\left\langle \x_{i},\x_{j}\right\rangle \right|\geq t\right]\leq2\exp\left[-c\min\left(\frac{t^{2}}{K^{4}M\norm[\B]F^{2}},\frac{t}{K^{2}\opnorm{\B}}\right)\right]
\]
where $K\coloneqq\max_{i}\norm[\x_{i}]{\psi_{2}}.$ 
\end{lem}
With this result, we only need to prove the same tail bound for $\P\left[\left|\sum_{i=1}^{N}B_{ii}\left(\norm[\x_{i}]2^{2}-\E\norm[\x_{i}]2^{2}\right)\right|\geq t\right]$.
To prove that, we cite another useful lemma (Theorem 2.8.2 on pp.$\ $36
in \citet{vershynin2017high}).
\begin{lem}[Bernstein's inequality for sub-exponential random variables]
\emph{}\label{lem:bernstein-subexp} Let $X_{1},\ldots,X_{N}$ be
independent, mean zero, sub-exponential random variables, and $\a\in\real^{N}$.
Then for every $t\geq0$, we have 
\[
\P\left[\left|\sum_{i=1}^{N}a_{i}X_{i}\right|\geq t\right]\leq2\exp\left[-c\min\left(\frac{t^{2}}{K_{1}^{2}\norm[\a]2^{2}},\frac{t}{K_{1}\norm[\a]{\infty}}\right)\right]
\]
where $K_{1}\coloneqq\max_{i}\norm[X_{i}]{\psi_{1}}$.
\end{lem}
Here, $\norm[\cdot]{\psi_{1}}$ denotes the sub-exponential norm;
see \citet{vershynin2017high} for more details. We work under the
premise of Lemma \ref{lem:hanson_wright_hdp}. Since $\x_{i}$ are
independent sub-Gaussian random vectors, each $\norm[\x_{i}]2^{2}-\E\norm[\x_{i}]2^{2}$
is the sum of $M$ independent, mean zero, sub-exponential random
variables with sub-exponential norm equal to $K^{2}$. Then Lemma
\ref{lem:bernstein-subexp} implies 
\[
\P\left[\left|\sum_{i=1}^{N}B_{ii}\left(\norm[\x_{i}]2^{2}-\E\norm[\x_{i}]2^{2}\right)\right|\geq t\right]\leq2\exp\left[-c\min\left(\frac{t^{2}}{K^{4}M\norm[\B]F^{2}},\frac{t}{K^{2}\opnorm{\B}}\right)\right]
\]
as required. 

\subsection{Proof of Fact \ref{fact:satisfy_cond_gauss_choas_operator_norm_bound}
\label{sec:proof_satisfy_cond}}

We prove the following equivalent statement 
\[
\norm[\left\langle \x,\w\right\rangle ]{\psi_{2}}^{2}\leq C\frac{\sgnorm^{2}}{\std^{2}}\E\left\langle \x,\w\right\rangle ^{2}\qquad\text{for any }\w\in\real^{\num},
\]
where $C>0$ is a universal constant and $C\frac{\sgnorm^{2}}{\std}\ge1.$
We first establish a relationship between $\sgnorm^{2}$ and $\Var\left(x_{1}\right)$:
Proposition 2.5.2 on pp. 24 of \citet{vershynin2017high} implies
that $\frac{C'\sgnorm^{2}}{\std^{2}}\ge\frac{1}{2}$ for some universal
constant $C'>0$. Hence, we have 
\begin{align*}
\norm[\left\langle \x,\w\right\rangle ]{\psi_{2}}^{2} & \overset{(i)}{\le}2C'\sum_{i\in\left[\num\right]}w_{i}^{2}\norm[x_{i}]{\psi_{2}}^{2}\\
 & =2C'\frac{\sgnorm^{2}}{\std^{2}}\sum_{i\in\left[\num\right]}w_{i}^{2}\std^{2}\\
 & \overset{(ii)}{=}2C'\frac{\sgnorm^{2}}{\std^{2}}\E\left\langle \x,\w\right\rangle ^{2},
\end{align*}
where $(i)$ holds according to Proposition 2.6.1 on pp. 28 of \citet{vershynin2017high},
and $(ii)$ holds since $x_{i}$ are i.i.d.$\ $and $\E x_{i}=0$.
Letting $C=2C'$ completes the proof.

\subsection{Proof of Lemma \ref{lm:order_stats} \label{sec:proof_lm_order_stat}}

We define 
\begin{align*}
L_{\M} & \coloneqq\sum_{j,a}\left(X_{ja}-\lambda_{ja}\right)M_{ja},\\
R_{\beta,\M} & \coloneqq D\sqrt{\left\lceil \beta\right\rceil \left(\sum_{j,a}\rho_{ja}^{2}M_{ja}\right)\log\left(3mg/\beta\right)},\\
\calM_{\beta} & \coloneqq\left\{ \M\in\left\{ 0,1\right\} ^{m\times g}:\M\one_{g}\le\one_{m},\norm[\M]1=\left\lceil \beta\right\rceil \right\} .
\end{align*}
To establish a uniform bound in $\beta$, we apply a discretization
argument to the possible values of $\beta$. Define the shorthand
$E\coloneqq(0,m]$. We can cover $E$ by the sub-intervals $E_{t}\coloneqq(t-1,t]$
for $t\in[m]$. For each $t\in[m]$ we define the probability

\begin{align*}
\alpha_{t} & \coloneqq\P\left\{ \exists\beta\in E_{t},\exists\M\in\calM_{\beta}:L_{\M}>R_{\beta,\M}\right\} .
\end{align*}
We bound each of these probabilities: 

\begin{align}
\alpha_{t} & \overset{(i)}{\le}\P\left\{ \exists\M\in\calM_{t}:L_{\M}>R_{t,\M}\right\} \nonumber \\
 & \leq\P\left\{ \bigcup_{\M\in\calM_{t}}\left\{ L_{\M}>R_{t,\M}\right\} \right\} \nonumber \\
 & \leq\sum_{\M\in\calM_{t}}\P\left\{ L_{\M}>R_{t,\M}\right\} ,\label{eq:double union bd on normal-1-1-2}
\end{align}
where step $(i)$ holds since $\beta\in E_{t}$ implies $\beta\le\left\lceil \beta\right\rceil =t$. 

Note that each $X_{ja}-\lambda_{ja}$ is an independent zero-mean
sub-Gaussian random variable and the squared sub-Gaussian norm of
$L_{\M}$ is at most $C_{\psi_{2}}\sum_{j,a}\rho_{ja}^{2}M_{ja}$
where $C_{\psi_{2}}>0$ is a universal constant. We apply Hoeffding
inequality (Lemma \ref{lem:hoeffding}) to bound the probability on
the RHS of (\ref{eq:double union bd on normal-1-1-2}): 
\begin{align*}
\P\left\{ L_{\M}>R_{t,\M}\right\}  & \leq\exp\left\{ -\frac{cD^{2}t\left(\sum_{j,a}\rho_{ja}^{2}M_{ja}\right)\log(3mg/t)}{C_{\psi_{2}}\sum_{j,a}\rho_{ja}^{2}M_{ja}}\right\} \\
 & \leq\exp\left\{ -4t\log(3mg/t)\right\} 
\end{align*}
where $c>0$ is a universal constant. Plugging this back to (\ref{eq:double union bd on normal-1-1-2}),
we have for each $t\in\left[m\right]$, 
\begin{align}
\alpha_{t} & \leq\sum_{\M\in\calM_{t}}\exp\left\{ -4t\log(3mg/t)\right\} \nonumber \\
 & =\binom{m}{t}g^{t}\exp\left\{ -4t\log(3mg/t)\right\} \nonumber \\
 & \leq\left(\frac{me}{t}\right)^{t}g^{t}\exp\left\{ -4t\log(3mg/t)\right\} \nonumber \\
 & \leq\exp\left\{ t\log(3mg/t)+t-4t\log(3mg/t)\right\} \nonumber \\
 & \leq\exp\left\{ -t\log(3mg/t)\right\} =\left(\frac{t}{3mg}\right)^{t},\label{eq:binom coeff bound-1-2}
\end{align}
where the last inequality follows from $t\leq t\log(3mg/t)$ for $t\in\left[m\right]$.
It follows that 
\begin{align*}
 & \quad\P\left\{ \exists\beta\in E,\exists\M\in\calM_{\beta}:L_{\M}>R_{\beta,\M}\right\} \\
 & \leq\P\left\{ \bigcup_{t=1}^{m}\left\{ \exists\beta\in E_{t},\exists\M\in\calM_{\beta}:L_{\M}>R_{\beta,\M}\right\} \right\} \\
 & \le\sum_{t=1}^{m}\alpha_{t}\\
 & \leq\sum_{t=1}^{m}\left(\frac{t}{3mg}\right)^{t}\eqqcolon P_{1}(m).
\end{align*}

It remains to show that $P_{1}(m)\leq\frac{1.5}{m}$. Since 
\begin{align*}
P_{1}(m) & \leq\sum_{t=1}^{m}\left(\frac{t}{3m}\right)^{t}\\
 & \le\frac{1}{3m}+\sum_{t=2}^{m}\left(\frac{t}{3m}\right)^{t}\\
 & \le\frac{1}{3m}+m\cdot\max_{t=2,3,\ldots,m}\left(\frac{t}{3m}\right)^{t},
\end{align*}
the proof is completed if for each integer $t=2,3,\ldots,m$, we can
show the bound $\left(\frac{t}{3m}\right)^{t}\leq\frac{1}{m^{2}}$,
or equivalently $f(t)\coloneqq t(\log3m-\log t)\geq2\log m.$ Since
$t\le m$, $f(t)$ has derivative 
\[
f'(t)=\log3m-\log t-1\ge\log3m-\log\left(\frac{3m}{3}\right)-1=\log3-1\ge0.
\]
Therefore, $f(t)$ is non-decreasing for $2\le t\le m$ and therefore
$f(t)\ge f(2)=2\log3m-2\log2\ge2\log m.$ Hence, $P_{1}(m)\le\frac{1.5}{m}$. 

\section{Proof of Theorem \ref{thm:cluster_error_rate}\label{sec:proof_cluster_error_rate}}

We only need to prove the first part of the theorem. The second part
follows immediately from the first part and Theorem \ref{cor:SDP_exp_rate}.

The proof follows similar lines as those of Theorem 17 and Lemma 18
in \citet{makarychev2016learning}. In the rest of the section, we
work under the context of Algorithms \ref{alg:apx_clustering} and
\ref{alg:est_clustering}. Recall that $\numclust'=\left|\left\{ B_{t}\right\} _{t\ge1}\right|$
and we let $\epsilon\coloneqq\norm[\Yhat-\Ystar]1/\norm[\Ystar]1$.
We have the following lemma.
\begin{lem}
\label{lem:apx_clustering} There exists a partial matching $\perm'$
between $\left[\numclust\right]$ and $\left[\numclust'\right]$ and
a universal constant $C>0$ such that 
\[
\left|\bigcup_{t=\perm'(a)}\clustset a\cap B_{t}\right|\ge\left(1-C\epsilon\right)\num.
\]
\end{lem}
The proof is given in Section \ref{sec:proof_apx_clustering}. The
next lemma concerns the quality of clustering by Algorithm \ref{alg:clustering}.
\begin{lem}
\label{lem:clustering} There exists a permutation $\perm$ on $\left[\numclust\right]$
and a universal constant $C>0$ such that 
\[
\left|\bigcup_{t=\perm(a)}\clustset a\cap U_{t}\right|\ge\left(1-C\epsilon\right)\num.
\]
\end{lem}
The proof is given in Section \ref{sec:proof_clustering}. The result
follows from combining the above lemmas and the fact that 
\[
\misrate(\LabelHat,\LabelStar)=1-\frac{1}{\num}\max_{\perm\in S_{\numclust}}\left|\bigcup_{t=\perm(a)}\clustset a\cap U_{t}\right|.
\]


\subsection{Proof of Lemma \ref{lem:apx_clustering}\label{sec:proof_apx_clustering}}

We define $\y_{a}$ to be an arbitrary row of $\Ystar$ whose index
is in $\clustset a$.
\begin{align*}
G_{a} & \coloneqq\left\{ i\in\clustset a:\norm[\Yhat_{i\bullet}-\y_{a}]1\leq\frac{\size}{8}\right\} ,\qquad\forall a\in\left[\numclust\right]\\
G & \coloneqq\bigcup_{a\in\left[\numclust\right]}G_{a},\\
H & \coloneqq\vertexset\backslash G.
\end{align*}

We construct a partial matching $\perm'$ between sets $\clustset a$
and $B_{t}$ by matching every cluster $\clustset a$ with the first
$B_{t}$ that intersects $G_{a}$, and we let $\perm'(a)=t$. Since
each $i\in\left[\num\right]$ belongs to some $B_{t}$, we are able
to match every $\clustset a$ with some $B_{t}$. The fact that we
cannot match two distinct clusters $\clustset a$ and $\clustset b$
with the same $B_{t}$ as well as the rest of the proof are given
by the following fact.
\begin{fact}
\label{fact:apx_clustering} We have 
\begin{enumerate}
\item For each $a\in\left[\numclust\right]$ and $t\in\left[\numclust'\right]$
such that $t=\perm'(a)$, we have $B_{t}\cap G_{b}=\emptyset$ for
any $b\in\left[\numclust\right]\backslash\left\{ a\right\} $ and
$B_{t}\subset G_{a}\cup H$;
\item For each $a\in\left[\numclust\right]$ and $t\in\left[\numclust'\right]$
such that $t=\perm'(a)$, we have 
\[
\left|B_{t}\cap\clustset a\right|\geq\left|G_{a}\right|-\left|B_{t}\cap H\right|.
\]
\item We have 
\[
\sum_{t=\perm'(a)}\left|B_{t}\cap\clustset a\right|\ge\left|\vertexset\right|-2\left|H\right|.
\]
\item There exists a universal constant $C>0$ such that $\left|H\right|\leq C\epsilon\num$.
\end{enumerate}
\end{fact}
The proof is given below.

\subsubsection{Proof of Fact \ref{fact:apx_clustering}\label{sec:proof_fact_apx_clustering}}
\begin{enumerate}
\item Suppose that there exist $B_{t}$ and $b\in\left[\numclust\right]$
such that $b\ne a$ and $B_{t}\cap G_{b}\ne\emptyset$. Let $u\in B_{t}\cap G_{a}$
and $v\in B_{t}\cap G_{b}$. Since $G_{a}$ and $G_{b}$ are disjoint,
we know that $u\ne v$. Let $w\in B_{t}$. Then we have 
\begin{align*}
\norm[\Yhat_{u\bullet}-\Yhat_{w\bullet}]1 & \leq\frac{\size}{4}\\
\norm[\Yhat_{v\bullet}-\Yhat_{w\bullet}]1 & \leq\frac{\size}{4}.
\end{align*}
Therefore 
\[
\norm[\Yhat_{u\bullet}-\Yhat_{v\bullet}]1\leq\norm[\Yhat_{u\bullet}-\Yhat_{w\bullet}]1+\norm[\Yhat_{v\bullet}-\Yhat_{w\bullet}]1\le\frac{\size}{2}.
\]
This implies 
\begin{align*}
\norm[\y_{a}-\y_{b}]1 & \leq\norm[\y_{a}-\Yhat_{u\bullet}]1+\norm[\Yhat_{u\bullet}-\Yhat_{v\bullet}]1+\norm[\y_{b}-\Yhat_{v\bullet}]1\\
 & \leq\frac{\size}{8}+\frac{\size}{2}+\frac{\size}{8}<\size,
\end{align*}
which is a contradiction to the fact that $\norm[\y_{a}-\y_{b}]1=2\size$.
To complete the proof, we note that for any $i\in B_{t}$ we have
either $i\in G_{a}$ or $i\in H$.
\item Fix $i\in G_{a}$ for some $a\in\left[\numclust\right]$. For any
$j\in G_{a}$ we have $j\in B(i)$ since 
\[
\norm[\Yhat_{i\bullet}-\Yhat_{j\bullet}]1\le\norm[\y_{a}-\Yhat_{i\bullet}]1+\norm[\y_{a}-\Yhat_{j\bullet}]1\le\frac{\size}{4}.
\]
Therefore, by definition 
\[
\left|B_{t}\right|\ge\left|B(i)\right|\ge\left|G_{a}\right|.
\]
We have 
\begin{align*}
\left|B_{t}\cap\clustset a\right| & \overset{(i)}{\ge}\left|B_{t}\cap G_{a}\right|\\
 & =\left|B_{t}\right|-\left|B_{t}\backslash G_{a}\right|\\
 & \overset{(ii)}{=}\left|B_{t}\right|-\left|B_{t}\cap H\right|\\
 & \ge\left|G_{a}\right|-\left|B_{t}\cap H\right|,
\end{align*}
where step $(i)$ holds since $G_{a}\subset\clustset a$ and step
$(ii)$ holds since $B_{t}\subset G_{a}\cup H$.
\item Summing the LHS of the above equation over $t=\perm'(a)$ gives 
\begin{align*}
\sum_{t=\perm'(a)}\left|B_{t}\cap\clustset a\right| & =\sum_{a\in\left[\numclust\right]}\left|G_{a}\right|-\sum_{t=\perm'(a)}\left|B_{t}\cap H\right|\\
 & \ge\sum_{a\in\left[\numclust\right]}\left|G_{a}\right|-\sum_{t\ge1}\left|B_{t}\cap H\right|\\
 & \overset{(i)}{=}\left|G\right|-\left|\vertexset\cap H\right|\\
 & =\left|\vertexset\right|-2\left|H\right|,
\end{align*}
where step $(i)$ holds since $B_{t}\cap H$ are disjoint and $\bigcup_{t\geq1}B_{t}=\vertexset$.
\item We have 
\[
\left|H\right|\cdot\frac{\size}{8}\leq\sum_{i\in H}\norm[\Yhat_{i\bullet}-\y_{\labelstar(i)}]1\leq\norm[\Yhat-\Ystar]1\leq\epsilon\norm[\Ystar]1=\epsilon\cdot\num\size
\]
where the last step follows from the fact that $\norm[\Ystar]1=\num\size$.
The result follows.
\end{enumerate}

\subsection{Proof of Lemma \ref{lem:clustering}\label{sec:proof_clustering}}

Let $\perm'$ be the partial matching between $\clustset a$ and $B_{t}$
from Lemma \ref{lem:apx_clustering}. Define $\perm(a)=\perm'(a)$
for $\perm'(a)\le\numclust$. If the resulting $\perm$ is a partial
permutation, we extend $\perm$ to a permutation defined on $\left[\numclust\right]$
in an arbitrary way. We may assume that $\left\{ U_{t}\right\} _{t\in\left[\numclust\right]}$
are $\left\{ B_{t}\right\} _{t\in\left[\numclust\right]}$ WLOG, and
that $U_{t}$ consists of $B_{t}$ and some elements from sets $B_{u}$
with $u>\numclust$. We have 
\begin{align*}
\left|\bigcup_{t=\perm(a)}\clustset a\cap U_{t}\right| & \ge\left|\bigcup_{t=\perm'(a)\le\numclust}\clustset a\cap B_{t}\right|\\
 & =\left|\bigcup_{t=\perm'(a)}\clustset a\cap B_{t}\right|-\left|\bigcup_{t=\perm'(a)>\numclust}\clustset a\cap B_{t}\right|\\
 & \ge\left(1-C'\epsilon\right)\num-\left|\bigcup_{t=\perm'(a)>\numclust}\clustset a\cap B_{t}\right|
\end{align*}
where $C'>0$ is a universal constant. Define 
\begin{align*}
T_{1} & \coloneqq\left\{ t>\numclust:t=\perm'(a)\text{ for some }a\in\left[\numclust\right]\right\} ,\\
T_{2} & \coloneqq\left\{ t\in\left[\numclust\right]:t\ne\perm'(a)\text{ for any }a\in\left[\numclust\right]\right\} .
\end{align*}
Note that $\left|T_{1}\right|=\left|T_{2}\right|$ and for any $t_{1}\in T_{1}$
and $t_{2}\in T_{2}$ we have $\left|B_{t_{1}}\right|\le\left|B_{t_{2}}\right|$.
Therefore, 
\begin{align*}
\left|\bigcup_{t=\perm'(a)>\numclust}\clustset a\cap B_{t}\right| & \le\left|\bigcup_{t\in T_{1}}B_{t}\right|\\
 & \le\left|\bigcup_{t\in T_{2}}B_{t}\right|\\
 & \le\left|\vertexset\right|-\left|\bigcup_{t=\perm'(a)}\clustset a\cap B_{t}\right|\\
 & =C'\epsilon\num.
\end{align*}
The result follows by setting $C\coloneqq2C'$.

\section{Proof of Theorem \ref{thm:mean_estimation_error}\label{sec:proof_mean_estimation_error}}

Let $\Var\left(g_{ij}\right)=\std^{2}$. For $a\in\left[\numclust\right]$,
define $\clustest a\coloneqq\left\{ i\in\left[\num\right]:\labelhat_{i}=a\right\} $
the estimated clusters encoded in $\LabelHat$, and recall that our
cluster center estimators are defined by $\Meanhat_{a}\coloneqq\size^{-1}\sum_{i\in\clustest a}\h_{i}$.
We assume $\left\{ \clustest a\right\} $ achieves the lowest clustering
error as given in Theorem \ref{thm:cluster_error_rate} WLOG. For
each $a\in\left[\numclust\right]$, we have 
\begin{align*}
\norm[\Meanhat_{a}-\Mean_{a}]2 & \le\norm[\frac{1}{\size}\sum_{i\in\clustest a}\h_{i}-\frac{1}{\size}\sum_{j\in\clustset a}\h_{j}]2+\norm[\frac{1}{\size}\sum_{j\in\clustset a}\h_{j}-\Mean_{a}]2\\
 & \eqqcolon Q_{1}+Q_{2}.
\end{align*}


\subsection{Controlling $Q_{1}$}

Define $\epsilon\coloneqq\misrate(\LabelHat,\LabelStar)$. We work
on the event that the result Theorem \ref{thm:cluster_error_rate}
is true. We have 
\[
Q_{1}=\frac{1}{\size}\norm[\sum_{i\in\clustest a\backslash\clustset a}\h_{i}-\sum_{j\in\clustset a\backslash\clustest a}\h_{j}]2
\]
Note that $\left|\clustest a\backslash\clustset a\right|=\left|\clustset a\backslash\clustest a\right|$
so we can pair each point in $\clustest a\backslash\clustset a$ with
a point in $\clustset a\backslash\clustest a$. Let us pair $i$th
point in $\clustest a\backslash\clustset a$ with $j(i)$th point
in $\clustset a\backslash\clustest a$, and define $\calM\coloneqq\left\{ \left(i,j(i)\right)\right\} $.
We have $\left|\calM\right|\le\num\epsilon$ and we can write 
\begin{align*}
Q_{1} & =\frac{1}{\size}\norm[\sum_{(i,j(i))\in\calM}\left(\h_{i}-\h_{j(i)}\right)]2\\
 & \le\frac{1}{\size}\sum_{(i,j(i))\in\calM}\norm[\h_{i}-\h_{j(i)}]2\\
 & \le\frac{1}{\size}\sum_{(i,j(i))\in\calM}\left(\minsep_{\labelstar(i),\labelstar(j(i))}+\norm[\g_{i}-\g_{j(i)}]2\right)\\
 & \le\frac{1}{\size}\sum_{(i,j(i))\in\calM}\left(C_{q}\minsep+\norm[\g_{i}-\g_{j(i)}]2\right),
\end{align*}
where the last step holds for some universal constant $C_{q}>0$ given
that $\max_{a,b\in\left[\numclust\right]}\minsep_{ab}\le C_{q}\minsep$.
By Theorem 3.1.1 on pp.$\ $41 of \citet{vershynin2017high}, $\frac{1}{\sqrt{2}\std}\norm[\g_{i}-\g_{j(i)}]2-\sqrt{\vecdim}$
is a sub-Gaussian random variable with sub-Gaussian norm at most $C_{\psi_{2}}\frac{\sgnorm^{2}}{\std^{2}}$
where $C_{\psi_{2}}>0$ is a universal constant. Then Lemma \ref{lem:hoeffding}
implies that 
\[
\P\left[\frac{1}{\sqrt{2}\std}\norm[\g_{i}-\g_{j(i)}]2-\sqrt{\vecdim}\ge C\frac{\sgnorm^{2}}{\std^{2}}\sqrt{\log\num}\right]\le\num^{-C'}
\]
for some universal constants $C,C'>2$. By the union bound and the
facts that $\left|\calM\right|\le\num$ and $\std\lesssim\sgnorm$,
we have 
\[
\max_{(i,j)\in\calM}\norm[\g_{i}-\g_{j(i)}]2\le C_{g}\left(\sgnorm\sqrt{2\vecdim}+C\sgnorm\sqrt{2\log\num}\right)
\]
with probability at least $1-n^{-C_{1}}$ where $C_{g},C_{1}>0$ are
universal constants. 

Therefore, we have 
\begin{align*}
Q_{1} & \le C_{0}\left(\minsep+\sgnorm\sqrt{\vecdim}+\sgnorm\sqrt{\log\num}\right)\cdot\numclust\exp\left[-\frac{\snr^{2}}{\conste}\right]\\
 & \le C_{0}\left(\minsep+\sgnorm\sqrt{\vecdim}+\sgnorm\sqrt{\log\num}\right)\cdot\exp\left[-\frac{\snr^{2}}{2\conste}\right]
\end{align*}
for some universal constant $C_{0},\conste>0$ with probability at
least $1-n^{-C_{1}}$, where the last step holds since $\snr^{2}\ge\numclust$.
The fact that $e^{x}\ge1+x>x$ for any $x$ implies 
\[
\exp\left[-\frac{\snr^{2}}{4\conste}\right]\le\frac{4\conste}{\snr^{2}}=\frac{\sgnorm}{\minsep}\cdot\frac{4\conste}{\snr}\le4\conste\frac{\sgnorm}{\minsep}
\]
where the last step holds since we have $\snr\ge1$ by the conditions
of Theorem \ref{thm:cluster_error_rate}. Hence, we have 
\begin{align*}
Q_{1} & \le C_{0}\sgnorm\left(4\conste+\sqrt{\vecdim}+\sqrt{\log\num}\right)\cdot\exp\left[-\frac{\snr^{2}}{4\conste}\right]\\
 & \leq C_{1}\sgnorm\left(1+\sqrt{\vecdim}+\sqrt{\log\num}\right)\cdot\exp\left[-\frac{\snr^{2}}{4\conste}\right]\\
 & \leq2C_{1}\sgnorm\left(\sqrt{\vecdim}+\sqrt{\log\num}\right)\cdot\exp\left[-\frac{\snr^{2}}{4\conste}\right]
\end{align*}
where $C_{1}>0$ is a universal constant.

\subsection{Controlling $Q_{2}$}

We have 
\[
Q_{2}=\norm[\frac{1}{\size}\sum_{j\in\clustset a}\g_{j}]2.
\]
We see that $\frac{1}{\size}\sum_{j\in\clustset a}g_{ji}$ has variance
$\frac{1}{\size}\std^{2}$. By Proposition 2.6.1 on pp.$\ $28 and
Theorem 3.1.1 on pp. 41 of \citet{vershynin2017high}, $\frac{\sqrt{\size}}{\std}\norm[\frac{1}{\size}\sum_{j\in\clustset a}\g_{j}]2-\sqrt{\vecdim}$
is a sub-Gaussian random variable with sub-Gaussian norm at most $C_{\psi_{2}}\frac{\sgnorm^{2}}{\std^{2}}$
where $C_{\psi_{2}}>0$ is a universal constant. Then Lemma \ref{lem:hoeffding}
implies that 
\[
\P\left[\frac{\sqrt{\size}}{\std}\norm[\frac{1}{\size}\sum_{j\in\clustset a}\g_{j}]2-\sqrt{\vecdim}\ge C\frac{\sgnorm^{2}}{\std^{2}}\sqrt{\log\num}\right]\le\num^{-C'}
\]
for some universal constants $C,C'>0$. Since $\std\lesssim\sgnorm$,
there exists a universal constant $C_{0}>0$ such that 
\[
Q_{2}\leq C_{0}\sgnorm\left(\sqrt{\frac{\numclust\vecdim}{\num}}+\sqrt{\frac{\numclust\log\num}{\num}}\right)
\]
with probability at least $1-\num^{-C'}$. 

\section{Technical lemmas}

The following lemma is Theorem 2.6.2 on pp.$\ $28 in \citet{vershynin2017high}.
\begin{lem}[General Hoeffding's inequality]
\emph{ \label{lem:hoeffding} }Let $X_{1},\ldots,X_{N}$ be independent,
mean zero, sub-Gaussian random variables. Then, for every $t\geq0$
we have 
\[
\P\left[\left|\sum_{i=1}^{N}X_{i}\right|\geq t\right]\leq2\exp\left[-\frac{ct^{2}}{\sum_{i=1}^{N}\norm[X_{i}]{\psi_{2}}^{2}}\right],
\]
where $c>0$ is a universal constant.
\end{lem}
The following lemma is Exercise 4.7.3 in \citet{vershynin2017high}.
\begin{lem}[Tail bound of covariance matrix of sub-Gaussians]
\emph{ }\label{lem:subg_cov_mat_bound} Let $\x$ be a sub-Gaussian
vector and let $\x_{1},\ldots,\x_{m}$ be independent samples of $\x$.
Let $m$ be a positive integer and define 
\begin{align*}
\boldsymbol{\Sigma} & \coloneqq\E\x\x\t,\\
\boldsymbol{\Sigma}_{m} & \coloneqq\frac{1}{m}\sum_{i=1}^{m}\x_{i}\x_{i}\t.
\end{align*}
Let $\rho_{0}\ge1$ be such that 
\[
\norm[\left\langle \x,\w\right\rangle ]{\psi_{2}}\leq\rho_{0}\sqrt{\E\left\langle \x,\w\right\rangle ^{2}}\qquad\text{for any }\w\in\real^{N}.
\]
For any $u\geq0$, we have for a universal constant $C>0$, 
\[
\opnorm{\boldsymbol{\Sigma}_{m}-\boldsymbol{\Sigma}}\leq C\rho_{0}^{2}\left(\sqrt{\frac{N+u}{m}}+\frac{N+u}{m}\right)\opnorm{\boldsymbol{\Sigma}}
\]
with probability at least $1-2e^{-u}$.
\end{lem}


\end{document}
