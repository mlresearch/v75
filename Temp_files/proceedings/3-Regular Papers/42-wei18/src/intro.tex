
The adversarial Multi-Armed Bandits (MAB) problem~\citep{auer2002nonstochastic} is a classic online learning problem with partial information feedback.
In this problem, at each round the learner selects one of the $K$ arms while simultaneously the adversary decides the loss of each arm,
then the learner suffers and observes (only) the loss of the picked arm.
The goal of the learner is to minimize the regret, that is, the difference between her total loss and the total loss of the best fixed arm.
The classic Exp3 algorithm~\citep{auer2002nonstochastic} achieves a regret bound of order $\tilde{\mathcal{O}}(\sqrt{TK})$ after $T$ rounds,\footnote{%
Throughout the paper we use the notation $\tilde{\mathcal{O}}(\cdot)$ to suppress factors that are poly-logarithmic in $T$ and $K$.}
which is worst-case optimal up to logarithmic factors.

There are several existing works on deriving more adaptive bandit algorithms, replacing the dependence on $T$ in the regret bound
by some data-dependent quantity that is $\mathcal{O}(T)$ in the worst-case but could be potentially much smaller in benign environments.
Examples of such data-dependent quantities include the loss of the best arm~\citep{allenberg2006hannan, foster2016learning} or the empirical variance of all arms~\citep{hazan2011better, bubeck2017sparsity}. 
Extensions to more general settings such as semi-bandit, two-point bandit, and graph bandit have also been studied~\citep{neu2015first, chiang2013beating, lykouris2017small}.
These adaptive algorithms not only enjoy better performance guarantees,
but also have important applications for other areas such as game theory~\citep{foster2016learning}.

In this work, we propose a novel and generic bandit algorithm in the more general semi-bandit setting
(formally defined in Section~\ref{section:notations}).
By instantiating this generic algorithm differently,
we obtain various adaptive algorithms with new data-dependent expected regret bounds that improve previous work.
When specified to the MAB setting with $\ell_{t,i} \in [-1,1]$ denoting the loss of arm $i$ at time $t$ (and $\ell_{0,i} \triangleq 0$),
these bounds replace the dependence on $T$ by (also see Table~\ref{table:summary} for a summary):
\begin{itemize}
\item 
$\sum_{t=1}^T (\ell_{t,i^\star} - \frac{1}{T}\sum_{s=1}^T\ell_{s,i^\star})^2$,
that is, the (unnormalized) variance of the best arm $i^\star$.
Similar existing bounds of~\citep{hazan2011better, hazan2011simple, bubeck2017sparsity} replace $T$ by the average of the variances of all arms.
In general these two are incomparable. 
However, note that the variance of the best arm is always bounded by $K$ times the average variance,
while it is possible that the latter is of order $\Theta(T)$ and the former is only $\mathcal{O}(1)$. (Section~\ref{subsubsection:variation bound})

\item
$K\sum_{t=1}^T |\ell_{t,i^\star} - \ell_{t-1,i^\star}|$, that is, ($K$ times) the first-order path-length of the best arm. (Section~\ref{subsubsection:path-length})

\item
$\sum_{i=1}^K\sum_{t=1}^T |\ell_{t,i} - \ell_{t-1,i}|$, that is, the sum of the first-order path-lengths of all arms.
Importantly, there is also an additional negative term in the regret similar to the one of~\citep{syrgkanis2015fast} for the full information setting.
This implies a fast convergence rate of order $1/T^\frac{3}{4}$ for several game playing settings with bandit feedback. 
(Sections~\ref{subsection:first_order_better_k})

\item
A new quantity in terms of some second-order excess loss (see Eq.~\eqref{eqn:new_excess_loss_bound} for the exact form). 
While the bound is not easy to interpret on it own, 
it in fact automatically and simultaneously implies the so-called ``small-loss'' bound $\tilde{\mathcal{O}}\Big(\sqrt{K\sum_{t=1}^T \ell_{t,i^\star}}\Big)$,\footnote{%
Assuming that losses are non-negative in this case as it is common for small-loss bounds.
}
{\it and} logarithmic regret $\mathcal{O}(\frac{K\ln T}{\Delta})$ if there is an arm whose expected loss is always smaller than those of other arms by a fixed gap $\Delta$
(e.g. the classic i.i.d. MAB setting~\citep{lai1985asymptotically}). (Section~\ref{section:best of both worlds})

\end{itemize}

These bounds are incomparable in general. 
All of them have known counterparts in the full information setting (see for example~\citep{steinhardt2014adaptivity} and~\citep{de2014follow}),
but are novel in the bandit setting to the best of our knowledge.
Note that for the first two results that depend on some quantities of only the best arm, 
we require tuning a learning rate parameter in terms of these (unknown) quantities.
Obtaining the same results with parameter-free algorithms remains open, even for the full information setting.
However, for the other results, we indeed provide parameter-free algorithms based on a variant of the doubling trick.

Our general algorithm falls into the Online Mirror Descent (OMD) framework (see for example~\citep{hazan2016introduction})
with the ``log-barrier'' as the regularizer, originally proposed in~\citep{foster2016learning}.
However, to obtain our results, two extra crucial ingredients are needed:
\begin{itemize}
\item
First, we adopt the ideas of optimism and adaptivity from~\citep{steinhardt2014adaptivity},
which roughly speaking amounts to incorporating a correction term as well as an optimistic prediction into the loss vectors.
In~\citep{steinhardt2014adaptivity}, this technique was developed in the Follow-the-Regularized-Leader (FTRL) framework,\footnote{%
Although it was confusingly referred as OMD in~\citep{steinhardt2014adaptivity}.}
but it is in fact crucial here to re-derive it in the OMD framework (due to the next ingredient).
The challenges here are to come up with the right correction terms and optimistic predictions.

\item
Second, we apply an individual and increasing learning rate schedule for one of the path-length results.
Such increasing learning rate schedule was originally proposed in~\citep{bubeck2016kernel} and also recently used in~\citep{agarwal2017corralling},
but for different purposes.

\end{itemize}

Although most algorithmic techniques we use in this work have been studied before,
combining all of them, in the general semi-bandit setting, requires novel and non-trivial analysis.
The use of log-barrier in the semi-bandit setting is also new as far as we know.

\paragraph{Related work.}
There is a rich literature in deriving adaptive algorithms and regret bounds for online learning with full information feedback
(see recent work~\citep{luo2015achieving, koolen2015second, van2016metagrad, orabona2016coin, cutkosky2017online} and references therein),
as well as the stochastic bandit setting (such as~\citep{garivier2011kl, lattimore2015optimally, degenne2016anytime}).
Similar results for the adversarial bandit setting, however, are relatively sparse and have been mentioned above.
While obtaining regret bounds that depend on the quality of the best action is common in the full information setting,
it is in fact much more challenging in the bandit setting, 
and the only existing result of this kind is the ``small-loss'' bound~\citep{allenberg2006hannan, foster2016learning}.
We hope that our work opens up more possibilities in obtaining these results,
despite some recent negative results discovered by~\citet{gerchinovitz2016refined}.

\citet{chiang2013beating} proposed bandit algorithms with second-order path-length bounds, but their work requires stronger two-point feedback.
The implication of path-length regret bounds on faster convergence rate for computing equilibriums was studied in~\citep{syrgkanis2015fast}.
Other examples of adaptive online learning leading to faster convergence in game theory include~\citep{rakhlin2013optimization, daskalakis2015near, foster2016learning}.

There exist several bandit algorithms that achieve almost optimal regret in both the adversarial setting ($\mathcal{O}(\sqrt{TK})$) and 
the i.i.d. setting ($\mathcal{O}(\sum_{i: \Delta_i \neq 0}\frac{\ln T}{\Delta_i})$ where $\Delta_i$ is the gap between the expected loss of arm $i$ and the one of the optimal arm)~\citep{bubeck2012best, seldin2014one, auer2016algorithm, seldin2017improved}.
Our results in Section~\ref{section:best of both worlds} have slightly weaker guarantee for the i.i.d. setting (at most $K$ times worse specifically)
since it essentially replaces all $\Delta_i$ by $\min_{i: \Delta_i \neq 0} \Delta_i$.
On the other hand, however, our results have several advantages compared to previous work.
First, our guarantee for the adversarial setting is stronger since it replaces the dependence on $T$ by the loss of the best arm.
Second, our logarithmic regret result applies to not just the simple i.i.d. setting, 
but the more general setting mentioned above where neither independence nor identical distributions is required.
Our dependence on $\ln T$ is also better than previous works,
resolving an open problem raised by~\citet{seldin2017improved}.
Finally, our algorithm and analysis are also arguably much simpler, without performing any stationarity detection or gap estimation.
Indeed, the result is in some sense algorithm-independent and solely through a new adaptive regret bound Eq.~\eqref{eqn:new_excess_loss_bound},
similar to the results in the full-information setting such as~\citep{gaillard2014second}.

Using a self-concordant barrier as regularizer was proposed in the seminal work of~\citep{abernethy2008competing} for general linear bandit problems.
The log-barrier is technically not a barrier for the decision set of the semi-bandit problem, 
but still it exhibits many similar properties as shown in our proofs.
Optimistic FTRL/OMD was developed in~\citep{chiang2012online, rakhlin2013online}.
As pointed out in~\citep{steinhardt2014adaptivity}, incorporating correction terms in the loss vectors can also be viewed as using adaptive regularizers,
which was studied in several previous works, mostly for the full information setting (see~\citep{mcmahan2017survey} for a survey).



