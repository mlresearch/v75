\section{Proof of Theorem~\ref{thm:applyKatyusha}}\label{sec:katyushaproof}

In this section, we prove our primary result, stated in \Cref{thm:applyKatyusha}.
Specifically, we further examine whether our strong initialization distance bound will allow us to improve the running time with black-box accelerated stochastic gradient descent methods.
The first step towards this is to apply smoothing reductions to our objective function.

\subsection{Smoothing the Objective Function and Adding Strong Convexity}

As before, we let $\xx^* = \arg\min_{\xx} \norme{\AA\xx - \bb}$. For clarity, we will borrow some of the notation from \cite{AllenZhuH16} to more clearly convey their black-box reductions.% to reduce the objective function to one with smoothness and strong convexity, along with its corresponding runtime and approximation guarantees.

\begin{definition}
	A function $f(x)$ is $(L,\sigma)$-\textit{smooth-sc} if it is both $L$-\textit{smooth} and $\sigma$-\textit{strongly-convex}.
\end{definition}

\begin{definition}
	An algorithm $\calA(f(x),x_0)$ is a $\textsc{Time}_\calA(L,\sigma)$-\textit{minimizer} if $f(x)$ is  $(L,\sigma)$-\textit{smooth-sc} and $\textsc{Time}_\calA(L,\sigma)$ is the time it takes $\calA$ to produce $x'$ such that $f(x') - f(x^*) \leq \frac{f(x_0) - f(x^*)}{4}$ for any starting $x_0$.
\end{definition}

Allen-Zhu and Hazan assume access to efficient $\textsc{Time}_\calA(L,\sigma)$-\textit{minimizer} algorithms, and show how a certain class of objective functions can be slightly altered to meet the smoothness and strong convexity conditions to apply these algorithms without losing too much in terms of error and runtime.


\begin{theorem}[Theorem C.2 from \cite{AllenZhuH16}]\label{thm:reductionAlgo}
	Consider the problem of minimizing an objective function
	\[f(x) = \frac{1}{n}\sum_{i=1}^n f_i(x) \]
	such that each $f_i(\cdot)$ is $G$-Lipschitz continuous. Let $\xx_0$ be a starting vector such that $f(\xx_0) - f(\xx^*) \leq \Delta$ and $\|\xx_0 - \xx^*\|_2^2 \leq \Theta$. 
	Then there is a routine that takes as input a $\textsc{Time}_\calA(L,\sigma)$-\textit{minimizer}, $\calA$, alongside $f(x)$ and $x_0$, with $\beta_0 = \Delta/G^2$, $\sigma_0 = \Delta/\Theta$ and $T = \log_2(\Delta/\epsilon)$, and produces $x_T$ satisfying $f(x_T) - f(x^*) \leq O(\epsilon)$ in total running time \[\sum_{t=0}^{T-1}\textsc{Time}_{\calA}(2^t/\beta_0,\sigma_0\cdot 2^{-t}).\] 
\end{theorem}


It is then straightforward to show that our objective function fits the necessary conditions to utilize \Cref{thm:reductionAlgo}.


\begin{lemma}\label{lem:minEqualSumLipsFunctions}
	If $\AA$ is IRB, then the function $\norm{\AA\xx - \bb}_1$ can be written as $\frac{1}{n}\sum_{i=1}^n f_i(\xx)$ such that each $f_i(\cdot)$ is $O(\sqrt{nd})$-Lipschitz continuous.
\end{lemma}


\begin{proof}
	By the definition of 1-norm,
	\[\norm{\AA\xx - \bb}_1 = \sum_{i=1}^n\abs{\AA_{i,:}\xx-\bb_i}.\]
	We then set $f_i(\xx) = n \cdot \abs{\AA_{i,:}\xx-\bb_i}$ and the result follows from Lemma~\ref{lem:easyLipscitz}.
\end{proof}

We can then incorporate our objective into the routine from \Cref{thm:reductionAlgo}, along with our initialization of $\xx_0$.

\begin{lemma}\label{lem:applyReduction}
	Let $\calA$ be a $\textsc{Time}_\calA(L,\sigma)$-\textit{minimizer}, along with objective $\norm{\AA\xx -\bb}_1$ such that $\AA$ is IRB and $\xx_0 = \AA^T \bb$, then the routine from \Cref{thm:reductionAlgo} produces
	$\xx_T$ satisfying $f(\xx_T) - f(\xx^*) \leq O(\epsilon)$ in total running time \[ \sum_{t=0}^{T-1}\textsc{Time}_\calA\left(O\left(\frac{nd2^t}{\Delta}\right),O\left(\frac{n\Delta}{d \cdot f(\xx^*)^2 2^t}\right)\right).\] 
\end{lemma}

\begin{proof}
	Lemma~\ref{lem:minEqualSumLipsFunctions} implies that we can apply \Cref{thm:reductionAlgo} where $G = O(\sqrt{nd})$, and Lemma~\ref{lem:easyInit} gives $\Theta = O(\frac{d}{n}f(x^*)^2)$.
	We then obtain $\beta_0 = O(\frac{\Delta}{nd})$ and $\sigma_0 = O(\frac{n\Delta}{d f(x^*)^2})$ and substitute these values in the running time of \Cref{thm:reductionAlgo}.
\end{proof}

\subsection{Applying Katyusha Accelerated SGD}\label{subsec:applykatyusha}

Now that we have shown how our initialization can be plugged into the smoothing construction of \cite{AllenZhuH16}, we simply need an efficient $\textsc{Time}_\calA(L,\sigma)$-\textit{minimizer} to obtain all the necessary pieces to prove our primary result.

\begin{theorem}[Corollary 3.8 in \cite{AllenZhu17}]\label{thm:katyushaRuntime}
	There is a routine that is a $\textsc{Time}_\calA(L,\sigma)$-\textit{minimizer} where $\textsc{Time}_\calA(L,\sigma) = d \cdot O(n + \sqrt{nL/\sigma})$.
\end{theorem}

We can then precondition the matrix to give our strong bounds on the initialization distance of $\xx_0$ from the optimal $\xx^*$, which allows us to apply the smoothing reduction and Katyusha accelerated gradient descent more efficiently. 

\applyKatyusha*
\begin{proof}	
	Once again, by preconditioning with Lemma~\ref{lem:lewisAndRotate} and error $O(\epsilon)$ we obtain a matrix $\tilde{\AA}\UU \in \R^{N \times d}$ and a vector $\tilde{\bb} \in \R^n$ in time $O(nnz(\AA)\log{n} + d^{\omega-1}\min\{d\epsilon^{-2}\log{n},n\})$.
	We utilize the routine in \Cref{thm:katyushaRuntime} as the $\textsc{Time}_\calA(L,\sigma)$-\textit{minimizer} for Lemma~\ref{lem:applyReduction}, and plug the time bounds in to achieve an absolute error of $O(\delta)$ in the preconditioned objective function with the following running time:
	\begin{align*} 
	d \cdot O\left( \sum_{t=0}^{T-1}N + \sqrt{N\cdot\left(\frac{Nd 2^{t}}{\Delta}\right)\left(\frac{d\cdot f(x^*)^2 2^{t}}{N\Delta}\right)}\right) 
	&= d \cdot O\left( N\log{\frac{\Delta}{\delta}} +  \frac{d\sqrt{N}\cdot f(x^*)}{\Delta}\sum_{t=0}^{T-1} 2^{t}\right) \\ &= d \cdot O\left( N\log{\frac{\Delta}{\delta}} + \frac{d\sqrt{N}\cdot f(x^*)}{\delta}\right) .
	\end{align*}

	To achieve our desired relative error of $\epsilon$ we need to set $\delta = O(\epsilon f(\xx^*))$. Technically, this means that the input to gradient descent will require at least a constant factor approximation to $f(\xx^*)$. We will show in \Cref{subsec:binarySearch} that we can assume that we have such an approximation at the cost of a factor of $\log{n}$ in the running time. We assume that $f(\xx^*) \geq n^{-c}$ for some fixed constant\footnote{Note that if $f(\xx^*) = 0$, then our initialization $\xx_0 = \AA^T \bb$ will be equal to $\xx^*$.} $c$ in order to upper bound $\log\frac{\Delta}{\delta}$, which gives a runtime of $O\left(d N\log (n\epsilon^{-1}) + \frac{d\sqrt{N}\cdot }{\epsilon}\right)$.
		
	Here, we used the fact that $N = O(d\epsilon^{-2}\log{n})$, but can also assume that computationally, $N \leq n$, as will be addressed in \Cref{subsec:simulatedSplitting}. This gives a runtime of 
	\[
	O\left(\min\{d^{2.5} \epsilon^{-2} \sqrt{\log {n}} , nd\log (n/\epsilon) + \sqrt{n}d^2\epsilon^{-1} \}\right),
	\] which, combined with our preconditioning runtime (where $\Upsilon$ is a lower order term if we assume the $\epsilon$ is at most polynomially small in $n$) and the factor $\log{n}$ overhead from estimating $f(\xx^*)$ which we address in \Cref{subsec:binarySearch}, gives the desired runtime. Furthermore, since the error in our preconditioning was $O(\epsilon)$, by Lemma~\ref{lem:lewisAndRotate} we have achieved a solution with $O(\epsilon)$ relative error in the original problem.
\end{proof}

%\subsection{Simulated Sampling of $\AA$}\label{subsec:simulatedSplitting}

In Lemma~\ref{lem:lewisAndRotate}, our primary preconditioning lemma, we set $N$ to be the minimum of $n$ and $O(d\epsilon^{-2}\log{n})$.
However, all of our sampling above assumed that $O(d\epsilon^{-2}\log{n})$ rows were sampled to achieve certain matrix concentration results.
Accordingly, we will still assume that $O(d\epsilon^{-2}\log{n})$ rows are sampled, but show that we can reduce the computational cost of any duplicate rows to $O(1)$, and hence the computation factor of $N$ can be assumed to be $\min\{n,O(d\epsilon^{-2}\log{n})\}$. The sampling procedure itself can be done in about $O(n)$ time. At the end of this section, we explain how the running time of Katyusha can be made to depend on $n$, rather than $d\epsilon^{-2}\log n$.

Ultimately, our proof of Lemma~\ref{lem:lewisAndRotate} will critically use the fact that $\tilde{\AA}$ has $O(d\epsilon^{-2}\log{n})$ rows in several places.
The following lemmas will then show how we can reduce this computation for duplicate rows, allowing us to substitute $n$ for $O(d\epsilon^{-2}\log{n})$ in the running time when $n \ll d\epsilon^{-2}\log{n}$.

\begin{lemma}\label{lem:duplicateRowsForAtransposeA}
	Let $\tilde{\AA} \in \R^{N \times d}$ be a matrix with at most $n$ unique rows, and for each unique row, we are given the number of copies in $\tilde{\AA}$. Then computing $\tilde{\AA}^T\tilde{\AA}$ takes at most $O(nd^{\omega -1})$ time.
\end{lemma}

\begin{proof}
	By definition 
	\[
	\tilde{\AA}^T\tilde{\AA} = \sum_i \tilde{\AA}_{i,:}^T\tilde{\AA}_{i,:}.
	\]
	Therefore, if we have $k$ copies of row $\tilde{\AA}_{i,:}$, we know that they contribute $k\tilde{\AA}_{i,:}^T\tilde{\AA}_{i,:}$ to the summation. Accordingly, if we replaced all of them with one row $\sqrt{k}\tilde{\AA}_{i,:}$, then this row would contribute an equivalent amount to the summation. As a result, we can combine all copies of unique rows to achieve an $n \times d$ matrix $\tilde{\AA'}$ and compute $\tilde{\AA'}^T\tilde{\AA'}$ which will be equivalent to $\tilde{\AA}^T\tilde{\AA}$.
\end{proof}

\begin{corollary}\label{cor:duplicateRowsRotAndInit}
	Let $[\tilde{\AA}\; \tilde{\bb}] \in \R^{N \times (d+1)}$ be a matrix with at most $n$ unique rows, and for each unique row, we are given the number of copies in $[\tilde{\AA}\; \tilde{\bb}]$. Then computing $\tilde{\AA}\UU$ where $\UU \in \R^{d\times d}$, and computing $\tilde{\AA}^T\tilde{\bb}$ takes $O(nd^{\omega - 1})$ and $O(nd)$ time, respectively.
\end{corollary}


\begin{proof}
	We can similarly use the fact that $\tilde{\AA}_{i,:}\UU$ is equivalent for all copies of $\tilde{\AA}_{i,:}$ and combine $k$ copies into the row $k\tilde{\AA}_{i,:}$.
	
	Analogously, we have $\tilde{\AA}^T\tilde{\bb} = \sum_i\tilde{\AA}_{i,:}^T\tilde{\bb}_i$, so we can combine duplicate rows.
\end{proof}

Furthermore, we need to show that we can efficiently sample $O(d\epsilon^{-2}\log{n})$ rows (ideally in $O(n)$-time) even when $O(d\epsilon^{-2}\log{n}) \gg n$. We will achieve this through known results on fast binomial distribution sampling.

\begin{theorem}[Theorem 1.1 in \cite{Farach-Colton2015}]\label{thm:binSampling}
	Given a binomial distribution $B(n,p)$ for $n \in \mathbb{N}$, $p \in \mathbb{Q}$, drawing a sample from it takes $O(\log^2{n})$ time using $O(n^{1/2 + \epsilon})$ space w.h.p., after $O(n^{1/2+\epsilon})$-time preprocessing for small $\epsilon > 0$. The preprocessing does not depend on $p$ and can be used for any $p'\in \mathbb{Q}$ and for any $n' \leq n$.
	
\end{theorem}

This result implies that sampling $m$ items independently can be done more efficiently if $m \gg n$, where we are only concerned with the number of times each item in the state space is sampled.
\begin{corollary}\label{cor:binSampling}
	Given a probability distribution $\mathcal{P} = (p_1,...,p_n)$ over a state space of size $n$, sampling $m$ items independently from $\mathcal{P}$ takes $O(m^{1/2 + \epsilon} + n\log^2{n})$-time.
	
\end{corollary}

\begin{proof}
	Note that sampling independently $m$ times is equivalent to determining how many of each item is sampled by using the binomial distribution and updating after each item. More specifically, we can iterate over all $i \in [n]$ and draw $k_i \sim B(m,p_i)$, then update $m$ to be $m - k_i$ and scale up each $p_j$ (where $j > i$) by $(1-p_i)^{-1}$. It is straightforward to make the scaling up of each $p_j$ efficient, and according to Theorem~\ref{thm:binSampling} we can obtain the binomial sample in $O(\log^2{n})$-time.
	
	
	Furthermore, because $m$ is decreasing at each iteration, we can use the original preprocessing in Theorem~\ref{thm:binSampling} for each step to achieve our desired running time.	
\end{proof}


\begin{corollary}\label{cor:sampleEfficiently}
	Given a matrix $\AA \in \R^{n \times d}$, with a probability distribution over each row, we can produce a matrix
	$\tilde{\AA} \in \R^{O(d\epsilon^{-2}\log{n})\times d}$ according to the given distribution in time at most
	$O\left(\min(d\epsilon^{-2}\log{n}, (d\epsilon^{-2}\log{n})^{1/2 + o(1)} + n\log^2{n})\right).$
\end{corollary}

Finally, our application of \Cref{thm:katyushaRuntime} assumes that it is given an $N \times d$ matrix, but we assumed that the computational cost could assume $N = \min\{n,O(d\epsilon^{-2}\log{n})\}$. A closer examination of Algorithm 2 in \cite{AllenZhu17}, which is the routine for \Cref{thm:katyushaRuntime}, shows that the factor of $N$ comes from a full gradient calculation, which can be done more quickly by combining rows in an equivalent manner to the lemma and corollary above.