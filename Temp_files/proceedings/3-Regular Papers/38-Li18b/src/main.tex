\documentclass[final,12pt]{colt2018}

\usepackage{algorithm}
\usepackage{algorithmic}
\renewcommand{\algorithmicrequire}{\textbf{Input:}}
\renewcommand{\algorithmicensure}{\textbf{Output:}}

\usepackage{thmtools,thm-restate}

% For notes
\def\shownotes{0}  %set 1 to show author notes
\ifnum\shownotes=1
\newcommand{\authnote}[2]{{$\ll$\textsf{\footnotesize #1 notes: #2}$\gg$}}
\else
\newcommand{\authnote}[2]{}
\fi
\newcommand{\yingyu}[1]{{\color{blue}\authnote{Yingyu}{{#1}}}}
\newcommand{\yuanzhi}[1]{{\color{red}\authnote{Yuanzhi}{{#1}}}}


\usepackage{multirow}


\newcommand{\var}{\textsf{Var}}
\newcommand{\veps}{\varepsilon}
\newcommand{\bI}{\bold{I}}
\newcommand{\bM}{\bold{M}}
\newcommand{\bA}{\bold{A}}
\newcommand{\bX}{\bold{X}}
\newcommand{\bY}{\bold{Y}}
\newcommand{\bV}{\bold{V}}
\newcommand{\bU}{\bold{U}}
\newcommand{\bSigma}{\bold{\Sigma}}	
\newcommand{\E}{\mathbb{E}}
\DeclareMathOperator*{\argmin}{arg\,min}
\DeclareMathOperator*{\sign}{sign}




\title{Learning Mixtures of Linear Regressions with Nearly Optimal Complexity}
\usepackage{times}

\coltauthor{\Name{Yuanzhi Li} \Email{yuanzhil@cs.princeton.edu}\\
 \addr Princeton University, Computer Science Department
 \AND
 \Name{Yingyu Liang} \Email{yliang@cs.wisc.edu}\\
 \addr University of Wisconsin-Madison, Computer Sciences Department
 }

\begin{document}

\maketitle


\begin{abstract}
Mixtures of Linear Regressions (MLR) is an important mixture model with many applications. In this model, each observation is generated from one of the several unknown linear regression components, where the identity of the generated component is also unknown. Previous works either assume strong assumptions on the data distribution or have high complexity. This paper proposes a fixed parameter tractable algorithm for the problem under general conditions, which achieves global convergence and the sample complexity scales nearly linearly in the dimension. In particular, different from previous works that require the data to be from the standard Gaussian, the algorithm allows the data from Gaussians with different covariances. When the conditional number of the covariances and the number of components are fixed, the algorithm has nearly optimal sample complexity $N = \tilde{O}(d)$ as well as nearly optimal computational complexity $\tilde{O}(Nd)$, where $d$ is the dimension of the data space. To the best of our knowledge, this approach provides the first such recovery guarantee for this general setting.
\end{abstract}


Online learning algorithms are a key tool in web search and content optimization, adaptively learning what users want to see. In a typical application, each time a user arrives, the algorithm chooses among various content presentation options (\eg news articles to display), the chosen content is presented to the user, and an outcome (\eg a click) is observed. Such algorithms must balance \emph{exploration} (making potentially suboptimal decisions now for the sake of acquiring information that will improve decisions in the future) and \emph{exploitation} (using information collected in the past to make better decisions now). Exploration could degrade the experience of a current user, but improves user experience in the long run. This exploration-exploitation tradeoff is commonly studied in the online learning framework of \emph{multi-armed bandits}~\citep{Bubeck-survey12}.

Concerns have been raised about whether exploration in such scenarios could be unfair, in the sense that some individuals or groups may experience too much of the downside of exploration without sufficient upside \citep{bird2016exploring}. We formally study these concerns in the \emph{linear contextual bandits} model~\citep{Langford-www10,chu2011contextual}, a standard variant of multi-armed bandits appropriate for content personalization scenarios.  We focus on \emph{externalities} arising due to exploration, that is, undesirable side effects that the presence of one party may impose on another.


We first examine the effects of exploration at a group level.  We introduce the notion of a \emph{group externality} in an online learning system, quantifying how much the presence of one population (which we dub the majority) impacts the rewards of another (the minority). We show that this impact can be negative, and that, in a particular precise sense, no algorithm can avoid it. This cannot be explained by the absence of suitably good policies since our adoption of the linear contextual bandits framework implies the existence of a feasible policy that is simultaneously optimal for everyone. Instead, the problem is inherent to the process of exploration. We come to a surprising conclusion that more data can sometimes lead to worse outcomes for the users of an explore-exploit-based system. \looseness=-1

We next turn to the effect of exploration at an individual level. We interpret exploration as a potential externality imposed on the current user by future users of the system. Indeed, it is only for the sake of the future users that the algorithm would forego the action that currently looks optimal. To avoid this externality, one may use the greedy algorithm that always chooses the action that appears optimal according to current estimates of the problem parameters. While this greedy algorithm performs poorly in the worst case,
it tends to work well in many applications and experiments.\footnote{Both positive and negative findings are folklore. One way to precisely state the negative result is that the greedy algorithm incurs constant per-round regret with constant probability; while results of this form have likely been known for decades,
\citet[Corollary A.2(b)]{competingBandits-itcs16}
proved this for a wide variety of scenarios. Very recently, the good empirical performance has been confirmed by state-of-art experiments in \citet{practicalCB-arxiv18}.}

In a new line of work, \citet{bastani2017exploiting} and \citet{kannan2018smoothed}
analyzed conditions under which inherent diversity in the data makes explicit exploration unnecessary.
\citet{kannan2018smoothed} proved that the greedy algorithm achieves a regret rate of
$\tilde{O}(\sqrt{T})$ in expectation over small perturbations of the context vectors (which ensure sufficient data diversity). This is the best rate that can be achieved in the worst case (\ie for all problem instances, without data diversity assumptions), but it leaves open the possibilities that (i) another algorithm may perform much better than the greedy algorithm on some problem instances, or (ii) the greedy algorithm may perform much better than worst case under the diversity conditions. We expand on this line of work. We prove that under the same diversity conditions, the greedy algorithm almost matches the best possible Bayesian regret rate of \emph{any} algorithm \emph{on the same problem instance}. This could be as low as $\polylog(T)$ for some instances, and, as we prove, at most $\tilde{O}(T^{1/3})$ whenever the diversity conditions hold.


Returning to group-level effects, we show that under the same diversity conditions, the negative group externalities imposed by the majority essentially vanish if one runs the greedy algorithm. Together, our results illustrate a sharp contrast between the high individual and group externalities that exist in the worst case, and the ability to remove all externalities if the data is sufficiently diverse.   \looseness=-1

\xhdr{Additional motivation.} Whether and when explicit exploration is necessary is an important concern in the study of the exploration-exploitation tradeoff. Fairness considerations aside, explicit exploration is expensive. It is wasteful and risky in the short term, it adds a layer of complexity to algorithm design \citep{Langford-nips07,monster-icml14}, and its adoption at scale tends to require substantial systems support and buy-in from management \citep{MWT-WhitePaper-2016,DS-arxiv}. A system based on the greedy algorithm would typically be cheaper to design and deploy.

Further, explicit exploration can run into incentive issues in applications such as recommender systems. Essentially, when it is up to the users which products or experiences to choose and the algorithm can only issue recommendations and ratings, an explore-exploit algorithm needs to provide incentives to explore for the sake of the future users \citep{Kremer-JPE14,Frazier-ec14,Che-13,ICexploration-ec15,Bimpikis-exploration-ms17}. Such incentive guarantees tend to come at the cost of decreased performance, and rely on assumptions about human behavior. The greedy algorithm avoids this problem as it is inherently consistent with the users' incentives.



\xhdr{Additional related work.}
Our research draws inspiration from the growing body of work on fairness in machine learning~\cite[e.g.,][]{dwork2012fairness,hardt2016equality,kleinberg2017inherent,chouldechova2017fair}.  Several other authors have studied fairness in the context of the contextual bandits framework.  Our work differs from the line of research on meritocratic fairness in online learning \citep{kearns2017meritocratic,liu2017calibrated,joseph2016fairness}, which considers the allocation of limited resources such as bank loans and requires that nobody should be passed over in favor of a less qualified applicant. We study a fundamentally different scenario in which there are no allocation constraints and we would like to serve each user the best content possible.  Our work also differs from that of \citet{celis2017fair}, who studied an alternative notion of fairness in the context of news recommendations. According to this notion, all users should have approximately the same probability of seeing a particular type of content (e.g., Republican-leaning articles), regardless of their individual preferences, in order to mitigate the possibility of discriminatory personalization.

The data diversity conditions in \citet{kannan2018smoothed} and this paper are inspired by the smoothed analysis framework of \citet{SmoothedAnalysis-jacm04}, who proved that the expected running time of the simplex algorithm is polynomial for perturbations of any initial problem instance (whereas the worst-case running time has long been known to be exponential). Such disparity implies that very bad problem instances are brittle. 
We find a similar disparity for the greedy algorithm in our setting.



\xhdr{Our results on group externalities.}  A typical goal in online learning is to minimize \emph{regret}, the (expected) difference between the cumulative reward that would have been obtained had the optimal policy been followed at every round and the cumulative reward obtained by the algorithm.  We define a corresponding notion of \emph{minority regret}, the portion of the regret experienced by the minority.  Since online learning algorithms update their behavior based on the history of their observations, minority regret is influenced by the entire population on which an algorithm is run.  If the minority regret is much higher when a particular algorithm is run on the full population than it is when the same algorithm is run on the minority alone, we can view the majority as imposing a negative externality on the minority; the minority population would achieve a higher cumulative reward if the majority were not present. Asking whether this can ever happen
amounts to asking whether access to more data points can ever lead an explore-exploit algorithm to make inferior decisions. One might think that more data should always lead to better decisions and therefore better outcomes for the users.
Surprisingly, we show that this is not the case, even with a standard algorithm.

Consider LinUCB~\citep{Langford-www10,chu2011contextual,abbasi2011improved}, a standard algorithm for linear contextual bandits that is based on the principle of ``optimism under uncertainty.''  We provide a specific problem instance on which, after observing $T$ users, LinUCB would have a minority regret of $\Omega(\sqrt T)$ if run on the full population, but only constant minority regret if run on the minority alone. While stylized, this example is motivated by the problem of providing driving directions to different populations of users, about which fairness concerns have been raised~\citep{bird2016exploring}. Further, the situation is reversed on a slight variation of this example: LinUCB obtains constant minority regret when run on the full population and $\Omega(\sqrt T)$ on the minority alone.  That is, group externalities can be large and positive in some cases, and large and negative in others.

Although these regret rates are specific to LinUCB, we show that this phenomenon is, in some sense, unavoidable. Consider the minority regret of LinUCB when run on the full population and the minority regret that LinUCB would incur if run on the minority alone. We know that one may be much smaller or larger than the other. We ask whether any algorithm could  achieve the minimum of the two on every problem instance. Using a variation of the same problem instance, we prove that this is impossible; in fact, no algorithm could simultaneously approximate both up to any $o(\sqrt{T})$ factor. In other words, an externality-free algorithm would sometimes ``leave money on the table."


In terms of techniques, we rely on the special structure of our example, which can be viewed as an instance of the sleeping bandits problem~\citep{SleepingBandits-ml10}. This simplifies the behavior and analysis of LinUCB, allowing us to obtain the $O(1)$ upper bounds.  The lower bounds are obtained using KL-divergence techniques to show that the two variants of our example are essentially indistinguishable, and an algorithm that performs well on one must obtain $\Omega(\sqrt{T})$ regret on the other. \looseness=-1


\xhdr{Our results on the greedy algorithm.} We consider a version of linear contextual bandits in which the latent weight vector $\theta$ is drawn from a known prior. In each round, an algorithm is presented several actions to choose from, each represented by a \emph{context vector}. The expected reward of an action is a linear product of $\theta$ and the corresponding context vector. The tuple of context vectors is drawn independently from a fixed distribution. In the spirit of smoothed analysis, we assume that this distribution has a small amount of jitter. Formally, the tuple of context vectors is drawn from some fixed distribution, and then a small \emph{perturbation}---small-variance Gaussian noise---is added independently to each coordinate of each context vector. This ensures arriving contexts are diverse. We are interested in Bayesian regret, i.e., regret in expectation over the Bayesian prior. Following the literature, we are primarily interested in the dependence on the time horizon $T$. \looseness=-1

We focus on a batched version of the greedy algorithm, in which new data arrives to the algorithm's optimization routine in small batches, rather than every round. This is well-motivated from a practical perspective---in high-volume applications data usually arrives to the ``learner" only after a substantial delay \citep{MWT-WhitePaper-2016,DS-arxiv}---and is essential for our analysis.

Our main result is that the greedy algorithm matches the Bayesian regret of any algorithm up to polylogarithmic factors, for each problem instance, fixing the Bayesian prior and the context distribution. We also prove that LinUCB achieves regret $\tilde{O}(T^{1/3})$ for each realization of $\theta$. This implies a worst-case Bayesian regret of $\tilde{O}(T^{1/3})$ for the greedy algorithm under the perturbation assumption. \looseness=-1

Our results hold for both natural versions of the batched greedy algorithm, Bayesian and frequentist, henceforth called \BayesGreedy and \FreqGreedy. In \BayesGreedy, the chosen action maximizes expected reward according to the Bayesian posterior. \FreqGreedy estimates $\theta$ using ordinary least squares regression and chooses the best action according to this estimate. The results for \FreqGreedy come with additive polylogarithmic factors, but are stronger in that the algorithm does not need to know the prior. Further, the $\tilde{O}(T^{1/3})$ regret bound for \FreqGreedy is approximately prior-independent, in the sense that it applies even to very concentrated priors such as independent Gaussians with standard deviation on the order of $T^{-2/3}$.

The key insight in our analysis of \BayesGreedy is that any (perturbed) data can be used to simulate any other data, with some discount factor. The analysis of \FreqGreedy requires an additional layer of complexity. We consider a hypothetical algorithm that receives the same data as \FreqGreedy, but chooses actions based on the Bayesian-greedy selection rule. We analyze this hypothetical algorithm using the same technique as \BayesGreedy, and then upper bound the difference in Bayesian regret between the hypothetical algorithm and \FreqGreedy.

Our analyses extend to group externalities and (Bayesian) minority regret. In particular, we circumvent the impossibility result mentioned above. We prove that both \BayesGreedy and \FreqGreedy match the Bayesian minority regret of any algorithm run on either the full population or the minority alone, up to polylogarithmic factors

\xhdr{Detailed comparison with prior work.} We substantially improve over the $\tilde{O}(\sqrt{T})$ worst-case regret bound from \citet{kannan2018smoothed}, at the cost of some additional assumptions. First, we consider Bayesian regret, whereas their regret bound is for each realization of $\theta$.%
\footnote{Equivalently, they allow point priors, whereas our priors must have variance $T^{-O(1)}$.} Second, they allow the context vectors to be chosen by an adversary before the perturbation is applied. Third, they extend their analysis to a somewhat more general model, in which there is a separate latent weight vector for every action (which amounts to a more restrictive model of perturbations). However, this extension relies on the greedy algorithm being initialized with a substantial amount of data. The results of \citet{kannan2018smoothed} do not appear to have implications on group externalities.

\citet{bastani2017exploiting} show that the greedy algorithm achieves logarithmic regret in an alternative linear contextual bandits setting that is incomparable to ours in several important ways.
They consider two-action instances where the actions share a common context vector in each round, but are parameterized by different latent vectors. They ensure data diversity via a strong assumption on the context distribution. This assumption does not follow from our perturbation conditions; among other things, it implies that each action is the best action in a constant fraction of rounds. Further, they assume a version of Tsybakov's \emph{margin condition}, which is known to substantially reduce regret rates in bandit problems \citep[\eg see][]{Zeevi-colt10}.


%!TEX root = LWM.tex
\subsection{Related Work}

Most directly related to our work is a recent series of papers by 
\cite{faradonbeh17a,faradonbeh17b}, who study the
linear system identification problem by proving a non-asymptotic rate
on the convergence of the OLS estimator to the true system matrices.
%Faradonbeh et al.\ distinguish between two cases: (a) when the $\Ast$ matrix
%is stable (i.e. spectral radius $\rho(\Ast)$ is bounded by one), and
%(b) when it is not.
In the regime where $\Ast$ is stable, Faradonbeh et al.\ recover 
a similar rate as our result. The major
difference is that the dependence of their analysis on the spectral properties of
$\Ast$ are qualitatively suboptimal, and difficult to interpret precisely.
%
Their analysis is based on separately establishing concentration of the
sample covariance matrix $\sum_{t=1}^{T} X_t X_t^\top$ to the stationary
covariance matrix and bounding the martingale difference term $\sum_{t=1}^{T} X_t \noise_t$.
This decoupled analysis inevitably picks up a dependence on the condition
number of the stationary covariance matrix, which means that
as the system becomes more unstable, their bound deteriorates. 
Indeed, such a strategy is unable to provide any insight into the behavior of OLS when,
for example, $\Ast$ is a scaled orthogonal matrix.
%\maxs{What does their bound say about marginally stable? Does it even hold} 
On the other hand, our analysis does not decouple the two terms, and as
a result our bounds only degrade in the \emph{logarithm} of the condition
number of the finite-time controllability Gramian $\Gamma_T$.
\cite{faradonbeh17a} also provide a bound in the \emph{unstable regime}, which
we believe can be sharpened using our analysis techniques which couple the
covariate- and noise-processes. We leave this to future work. Moreover,
our analysis of one-dimensional, unstable systems corroborates the linear
convergence behavior that \cite{faradonbeh17a} obtain for ``explosive'' systems,
which are systems where
\emph{all} eigenvalues of $\Ast$ lie outside the complex unit disk.


Another closely related work is the scalar analysis by \cite{rantzer18}.
In fact, our proof technique for scalar systems can be seen as an extension of
his technique. The main difference is that by more carefully tracking the terms
that appear in the moment generating function of the noise and covariate processes, we are able to discriminate behaviors that
arise when $\Ast$ is stable versus unstable, and uncover a linear rate of convergence
in the unstable regime.

Our result qualitatively matches the behavior of the
rate given in \cite{dean17},
in that the key spectral quantity governing the rate of convergence is the
minimum eigenvalue of the finite-time controllability Gramian.
The major difference is that the analysis
in Dean et al.\ uses multiple independent trajectories, and discards all but the last
state-transition in each trajectory. This decouples the covariates, and reduces
the analysis to that of random design linear regression with independent covariates.
We note, however, that the analysis in Dean et al.\ applies even when $\Ast$ is
unstable.

More broadly, there has been recent interest in non-asymptotic analysis of linear system
identification problems. Some of the earlier non-asymptotic literature in system identification
include \cite{campi2002finite} and \cite{vidyasagar2008learning}.
The results provided in this line of work are often quite conservative,
featuring quantities which are exponential in the degree of the system.
Furthermore, the rates given are often difficult to interpret.
More recently, \cite{shah12} pose the problem of recovering
a single-input, single-output (SISO) LTI system from linear measurements in the frequency domain as a sparse recovery
problem, proving polynomial sample complexity for recovery in the $\calH_2$-norm.
\cite{hardt16} show that under fairly restrictive assumptions on the
$\Ast$ matrix, projected gradient descent recovers the state-space representation 
of an LTI system with only a polynomial number of samples. 
The analysis from both Shah et al. and Hardt et al. both degrade polynomially
in $\frac{1}{1-\rho(\Ast)}$, where $\rho(\Ast)$ is the spectral radius of underlying $\Ast$.
%
On the other hand, \cite{hazan17} propose a new spectral filtering algorithm 
for online prediction of linear systems where the rates do not degenerate as $\rho(\Ast) \to 1$,
with the caveat that the analysis only applies to symmetric $\Ast$ matrices. \cite{hazan18} extends the analysis to diagonalizable matrices, but the obtained error rates are polynomial in problem parameters. Both works also consider the more general setting where $X_t$ is observed indirectly via $Y_t = CX_t$ for an unknown observation matrix $C$.
%
Moreover, the main metric of interest in both \cite{hardt16} and \cite{hazan17,hazan18}
is the prediction error.  It is not clear how
prediction error guarantees can be used in downstream robust control synthesis, 
whereas the operator norm bounds we provide can be used as direct
inputs into robust synthesis for optimal control problems~\citep{dean17}.


The most well-established technique in the statistics literature for dealing
with non-independent, time-series data is the use of mixing-time arguments~\citep{yu94}.
In the machine learning literature, mixing arguments have been used to develop
generalization bounds~\citep{mohri07,mohri08,kuznetsov17,mcdonald17b} which
are analogous to the classical generalization bounds for i.i.d.\ data.
As mentioned previously, a fundamental limitation of mixing-time arguments is that
the bounds all degrade as the mixing-time increases. This has two implications for
linear system identification: (a) none of these existing results can correctly capture
the qualitative behavior as the $\Ast$ matrix reaches instability, and (b)
these techniques cannot be applied to the regime where $\Ast$ is unstable, for which
estimation is not only well-posed, but should be quite easy.
It is for these reasons we do not pursue such arguments in this work.



%\maxs{what about Cyril's paper?}
%\stephen{TODO: maybe make a short blurb about Mendelson's small ball method.}

%\stephen{---------------------------------------------------------}
%
%\paragraph{Estimation papers.}
%
%\stephen{michigan paper~\cite{faradonbeh17a,faradonbeh17b}}
%\\
%\stephen{dean et al~\cite{dean17}}
%\\
%\stephen{cyril zhang paper~\cite{hazan17}}
%\\
%\stephen{tengyu paper~\cite{hardt16}}
%\\
%\stephen{rantzer paper}
%\\
%\stephen{campi papers}
%
%\paragraph{Mixing papers.}
%
%\stephen{yu94~\cite{yu94}, mohri~\cite{mohri07,mohri08}, kuznetsov, mcdonald and shalizi~\cite{mcdonald17b}, my recent LQR+LSTD paper}

%%% Local Variables:
%%% mode: latex
%%% TeX-master: t
%%% End:

\section{Problem Definition and Our Result} \label{sec:preli}

In the Mixtures of Linear Regressions (MLR) model, the data $(x, \alpha) \in \mathbb{R}^{d+1}$ is generated by
\begin{align} \label{def:mlr}
  z \sim \mbox{multinomial}(p), ~x \sim \mathcal{D}_z,~ \alpha = \langle w_z, x \rangle
\end{align}
where $p \in \mathbb{R}^k$ is the proportion of different components satisfying $\sum_{i=1}^k p_i=1$, $\mathcal{D}_i$ is the distribution of the $i$-th component, and $\{w_i \in \mathbb{R}^d\}_{i=1}^k $ are the ground truth parameters. The goal is then to recover $\{w_i\}_i$ given a dataset $\{(x_\ell, \alpha_\ell)\}_{\ell=1}^{N}$, where each $(x_\ell, \alpha_\ell)$ is i.i.d. generated by (\ref{def:mlr}).


\paragraph{Notations.} $[k]$ is used to denote the set $\{1, 2, \ldots, k\}$. With high probability or w.h.p.\ means with probability $1 - d^{-C}$ for some sufficiently large constant $C>1$. $1_\set{E}$ is the indicator function of the event $\set{E}$.


\paragraph{Assumptions.} We make the following assumptions about the distributions $\mathcal{D}_i$'s and $w_i$'s.
\begin{enumerate}
\item[\textbf{(A1)}] Each $\mathcal{D}_i = \mathcal{N}(0, \bSigma_i^2)$, where $\bI \preceq  \bSigma_i \preceq \sigma \bI$ for some $\sigma \ge 1$.

\item[\textbf{(A2)}] For every $i \in [k]$, $p_i \ge p_{\min}$ for some $p_{\min} > 0$.

\item[\textbf{(A3)}] Each $\| w_i\|_2 \leq1$, and for some $\Delta \in (0,1)$, 
$
  \| w_i - w_j \|_2 \geq \Delta
$
for any $i \neq j \in [k]$.
\end{enumerate}

Assumption \textbf{(A1)} allows the data $x$ in different components to come from Gaussian distributions with different unknown covariances.\footnote{In the standard linear regression model, the covariance of $x$ can be assumed to be the identity by doing a linear transformation. However, in the mixture of linear regression models, different components have different covariances and thus can not be simultaneously transformed to the identity since which data point comes from which component is unknown.}
This is more general than all the previous works that assume they all come from the standard Gaussian distribution. This also causes difficulties in applying known techniques for MLR, and thus requires new algorithmic approaches.  Moreover, our result can also be easily generalized to the case that the mixtures come from \emph{different} subspaces. That is, there can be zero singular values for $\Sigma_i$'s and the \emph{non-zero} singular values of each component is in $[1, \sigma]$. 

Assumption \textbf{(A2)} controls the imbalance of the components. We should require that there are enough data from each component so that it is possible to recover the corresponding parameter. On the other hand, our technique can also be generalized to the case when there is enough difference between the probabilities. In this case, we could also treat some components as noise and only recover the leading ones. 

Assumption \textbf{(A3)} assumes that the ground truth parameters are separated vectors, which is indeed required for exact recovery. Previous works also in general have some form of separation assumptions, many of which are much more sophisticated than ours (e.g.,~\citep{zhong2016mixed,yi2016solving}). 

\paragraph{Our result.} 
We are now ready to present our result formally.

%
%\begin{restatable}[Main]{theorem}{maintheorem}
%\label{thm:main} 
%Assume the model and the assumptions. Then there is an algorithm that takes $N=d \log\left(\frac{d}{\veps}\right)\cdot \left(\frac{\sigma}{\Delta p_{\min}} \right)^{O(k)} +  \left( \frac{\sigma }{\Delta p_{\min} \veps} \right)^{O(k^2)}$ data points and in time $Nd \cdot \textrm{polylog}(k, d, \sigma, \frac{1}{\Delta}, \frac{1}{p_{\min}}, \frac{1}{\veps}) $  outputs a set of vectors $\{v_i\}_{i=1}^k$ that with high probability satisfy
%$$
  %\|v_i  - w_{\pi(i)} \|_2 \le \veps, \forall i \in [k], ~\mbox{for some permutation $\pi$}.
%$$ 
%\end{restatable}
%
%
%\begin{corollary}
%\label{cor:main} 
%In the simpler setting where $k,\sigma, p_{\min}, \Delta$ are constants, there is an algorithm that takes $N=O\left(d \log\left(\frac{d}{\veps}\right)\right) +  \textrm{poly}(1/\veps)$ data points and in time $Nd \cdot \textrm{polylog}(d, \frac{1}{\veps}) $  outputs a set of vectors $\{v_i\}_{i=1}^k$ that with high probability satisfy
%$$
  %\|v_i  - w_{\pi(i)} \|_2 \le \veps, \forall i \in [k], ~\mbox{for some permutation $\pi$}.
%$$ 
%\end{corollary}
%

\begin{restatable}[Main]{theorem}{maintheorem}
\label{thm:main} 
Assume the model~(\ref{def:mlr}) and assumptions \textbf{(A1)}-\textbf{(A3)}. Then Algorithm~\ref{alg:mlr} takes 
%$N=d \log\left(\frac{d}{\veps}\right)\cdot \left(\frac{\sigma}{\Delta p_{\min}} \right)^{O(k)} +  \left( \frac{\sigma }{\Delta p_{\min} \veps} \right)^{O(k^2)}$ 
$N=d \log\left(\frac{d}{\veps}\right)\cdot \textrm{poly}\left(\frac{k\sigma}{\Delta p_{\min}} \right) +  \left( \frac{\sigma }{\Delta p_{\min} } \right)^{O(k^2)}$ 
data points and in time $Nd \cdot \textrm{polylog}(k, d, \sigma, \frac{1}{\Delta}, \frac{1}{p_{\min}}, \frac{1}{\veps}) $  outputs a set of vectors $\{v_i\}_{i=1}^k$ that with high probability satisfy
$$
  \|v_i  - w_{\pi(i)} \|_2 \le \veps, \forall i \in [k], ~\mbox{for some permutation $\pi$}.
$$ 
\end{restatable}

The theorem shows that the proposed algorithm achieves global convergence. The run time is polylog in $1/\veps$ for recovery error $\veps$, i.e., the algorithm can achieve exact recovery efficiently. Furthermore, in the case where $k, \sigma,$ $p_{\min}$, and $\Delta$ are fixed constants, the sample complexity is nearly linear in the dimension $d$ of the data space, which is nearly optimal in the key parameter $d$. 
%This case already subsumes the settings studied in previous works. 
The algorithm still works for wider range of $k, \sigma$, $p_{\min}$, and $\Delta$, but with an exponential dependence on $k$. 
%The dependence on $d$ and $k$ still matches the best known guarantees~\citep{zhong2016mixed}, while our results hold for the more general setting with different variances.
%
%We also note that if we use a different existing algorithm as a subroutine in Algorithm~\ref{alg:1_d} then $N = d 
%\log\left(\frac{d}{\veps}\right)\cdot \left(\frac{\sigma}{\Delta p_{\min}} \right)^{O(k)} + 
%\log\left(\frac{d}{\veps}\right)\cdot \left(\frac{\sigma}{\Delta p_{\min}} \right)^{O(k^2)}$. In any case we achieve $N = d \log\left(\frac{d}{\veps}\right)\cdot \left(\frac{\sigma}{\Delta p_{\min}} \right)^{O(k)} + n$ where 
%%$n = \min\left\{\log\left(\frac{d}{\veps}\right)\cdot \left(\frac{\sigma}{\Delta p_{\min}} \right)^{O(k^2)}, \textrm{poly}(\frac{\sigma k}{\Delta p_{\min} \veps}) + \left(k \log \frac{\sigma k}{\Delta p_{\min} \veps} \right)^{O(k^4)}\right\}$ 
%$n$ is a minor term. 


Table~\ref{tab:previous} shows the comparison with some recent works.
Since for $k=2$ our settings and results subsumes the existing ones, we mainly compare to previous works handling multiple components $k \ge 2$. Algorithms using the tensor method have $\text{poly}(1/\veps)$ dependence~\citep{chaganty2013spectral,yi2014alternating,sedghi2016provable}.
This can be improved by using tensor method only for initialization. 
\citep{zhong2016mixed} provided such an algorithm fixed parameter tractable in the number of components, achieving $N = \tilde{O}(k^k d)$ sample complexity and $\tilde{O}(Nd)$ computational complexity. However, the result is only for the case where the components have data $x$ from the same distribution $\set{D}_i = \mathcal{N}(0, \bI)$. \citep{yi2016solving} provided an algorithm with sample complexity nearly linear in $d$ and polynomial in $k$ but again it is only for the case with $\set{D}_i = \mathcal{N}(0, \bI)$, and furthermore, the sample complexity depends on the minimal singular value of certain moment matrix, which can also be  $\left( \frac{1}{\Delta} \right)^{k}$ small in our setting.  
\citep{sedghi2016provable} provided algorithms for the case where there are $k\geq 2$ components and $\mathcal{D}_i$ are the same (but can be distributions other than Gaussians). It is based on tensor methods and when applied to Gaussian inputs has high sample and computational complexity. 

%Comparison with some recent works are presented in Table~\ref{tab:previous}, and some discussions involving technical details are deferred to Section~\ref{sec:discussion}. 

We also note that it is interesting to compare to results for learning mixture of Gaussians. When the covariance matrix is not axis-aligned, to the best of our knowledge, there is no algorithm for learning  mixture of Gaussians with sample complexity linear in the dimension. Thus, solving the mixture of Gaussian first and then rescale the covariances to identity would clearly fail in our setting. Our result shows how to make use of this small amount of side information (the label $\alpha$) to lower the sample and computational complexity significantly. We refer to for example~\citep{ashtiani2017sample} for some discussions. 


%1. \citep{yi2016solving}
%2. \citep{zhong2016mixed}
%3. tensor: \citep{chaganty2013spectral,sedghi2016provable} 
%4. convex: \citep{chen2014convex}
%5. EM \citep{klusowski2017estimating}

\begin{table}
	\centering
\scriptsize
		\begin{tabular}{c| c| c | c}
		\hline
			    & main model assumptions  &  sample complexity $N$   & computational complexity \\
		 \hline
\multirow{2}{*}{\citep{yi2016solving}}  &  $\set{D}_i = \set{N}(0, \bI), k \ge 2$,  separation $\Delta > 0$,  & \multirow{2}{*}{ $\text{poly}(k) \frac{d}{\sigma_k^5 \Delta^2} $ }          &   \multirow{2}{*}{$\text{poly}(k) d^3$ } \\
    & singular value of some moment matrix $\sigma_k$ & & 
\\ \hline
\citep{zhong2016mixed}  &  $\set{D}_i = \set{N}(0, \bI), k \ge 2$, separation $\Delta > 0$          &  $ O(d (k \log(d))^k)$ & $O(Nd \log(d/\veps))$ 
\\ \hline
%3. tensor: \citep{chaganty2013spectral} &   & 
%\\ \hline
\multirow{2}{*}{\citep{sedghi2016provable}}  &  $\set{D}_i$ are the same, $k \ge 2$,  &  \multirow{2}{*}{$O\left(\frac{k^4 d^3}{\veps^2 s^2}\right) $ for Gaussian input} & \multirow{2}{*}{much higher than $\tilde{O}(d^2)$ }
\\
& singular values of weight matrix $\ge s>0$ & &
\\ \hline
%4. convex: \citep{chen2014convex}  & & 
%\\ \hline
\multirow{2}{*}{\citep{klusowski2017estimating}} & $\set{D}_i = \set{N}(0, \bI)$, $k = 2$, & \multirow{2}{*}{$\tilde{O}(d)$}  & \multirow{2}{*}{$\tilde{O}(Nd)$ }
\\
& local convergence of EM algorithm  & &
\\ \hline \hline
		 \multirow{2}{*}{Ours}   &  $\set{D}_i = \set{N}(0, \bSigma_i^2), \bI \preceq \bSigma_i \preceq \sigma\bI, k \ge 2$, &  \multirow{2}{*}{$ d \log\left(\frac{d}{\veps}\right) \textrm{poly}\left(\frac{k\sigma}{\Delta} \right)$ + minor term}   & \multirow{2}{*}{$\tilde{O}(Nd)$  }
		\\
		& separation $\|w_i - w_j\| \ge \Delta > 0 (\forall i\neq j)$&  
		\\
		\hline
		\end{tabular}
	\caption{Comparison with some recent related works. Please refer to the papers for details about the model assumptions and dependence on some other less important parameters, which are omitted here for clarity. In particular, the separation parameters in the related work have different meaning from ours and more complicated. \yingyu{k2}}
	\label{tab:previous}
\end{table}
\normalsize


%!TEX root = main.tex
%\vspace{-.5cm}
\section{Proof Sketch} \label{sec:pfsketch}
\vspace{-1.22pt}
We provide an overview of the arguments that comprise the proof of Theorem \ref{thm:main} (full details are deferred to \myapp{app_pfsketch}). We highlight three key steps. First, since we assume the iterates $x_n$
produced from SGD converge to within $\sim O(\sqrt{\gamma_n})$ of $x_\star$, we can perform a
Taylor expansion of the recursion in \eq{grad_desc}, to relate the points $x_n$ on the manifold $\M$ to vectors $\Delta_n$ in the tangent space $T_{x_\star}\M$. This
generates a (perturbed) linear recursion governing the evolution of the vectors $\Delta_n \in T_{x_\star} \M$.
Recall that as $x_\star$ is unknown, $\Delta_n$ is not accessible, but is primarily a tool for our analysis. Second, we can show a fast $O(\frac{1}{n})$ convergence rate for the averaged vectors $\bar{\Delta}_n \in T_{x_\star} \M$, using techniques from the Euclidean setting.
Finally, we once again use a local expansion of \eq{ave_grad_desc} to connect the averaged tangent vectors $\bar \Delta_n$ to the streaming, Riemannian average $\tilde \Delta_n$---transferring the fast rate for the inaccessible vector $\bar{\Delta}_n$ to the computable point $\tilde x_n$.
Throughout our analysis we extensively use Assumption~\ref{assump:manifold}, which restricts the iterates $x_n$ to the subset $\X$.
\vspace{-4.11pt}
\subsection{From $\M$ to $T_{x_\star}\M$ } \label{sec:pfsketch1}
\vspace{-.0856cm}
We begin by linearizing the progress of the SGD iterates $x_n$ in the tangent space of $x_\star$ by considering the evolution of $\Delta_n = R_{x_\star}^{-1}(x_n)$.
\begin{itemize}
\vspace*{-6pt}
  \item First, as the $\Delta_n$ are all defined in the vector space $T_{x_\star} \M$, Taylor's theorem applied  to $R_{x_\star}^{-1} \circ R_{x_n}:T_{x_n} \M \to T_{x_\star} \M$ along with \eq{grad_desc} allows us to conclude that \vspace*{-4pt}
  \[
  \D_{n+1} = \D_n - \gamma_{n+1} [\te{x_\star}{x_n}]^{-1} (\nabla f_{n+1}(x_n)) + O(\gamma_{n+1}^2).
  \vspace*{-6pt}
  \]
  See Lemma \ref{lem:tangent_rec} for more details.
 \vspace*{-6pt}
 \item Second, we use the manifold version of Taylor's theorem and appropriate Lipschitz conditions on the gradient to further expand the gradient term $ \tp{x_n}{x_\star} \nabla f_{n+1}(x_n)$ as \vspace*{-6pt}
  \[
 \tp{x_n}{x_\star} \nabla f_{n+1}(x_n)=\Hess f(x_\star)\Delta_n + \nabla f_{n+1}(x_\star)+ \xi_{n+1}+O(\Vert \Delta_n\Vert^2), \vspace*{-6pt}
  \]
  where the noise term is controlled as $\E[\ \xi_{n+1}\vert\mathcal F_{n}]=0$, and $\E[\Vert \xi_{n+1}\Vert^2 \vert\mathcal F_{n}]=O( \Vert\Delta_n\Vert^2)$. See Lemma \ref{lem:tangent_rec_2} for more details.
 \vspace*{-6pt}
 \item Finally, we argue that the operator $ [\te{x_\star}{x_n}]^{-1}\tp{x_\star}{x_n} : T_{x_\star}\M \to  T_{x_\star}\M$ is a local isometry up to second-order terms:
 $
    [\te{x_\star}{x_n}]^{-1}\tp{x_\star}{x_n} = I + O(\norm{\Delta_n}^2),
  $
  which crucially rests on the fact $R$ is a second-order retraction. See Lemma \ref{lem:tangent_rec_3} for more details.

 \item  \vspace*{-6pt} Assembling the aforementioned lemmas allows us to derive a (perturbed) linear recursion, governing the tangent vectors $\{ \Delta_n \}_{n \geq 0}$ as \vspace*{-6pt}
  \begin{equation} \label{eq:final_proof_sketch}
    \D_{n+1} = \D_n - \gamma_{n+1} \Hess f(x_\star) \D_n  -\gamma_{n+1} \nabla f_{n+1}(x_\star)  -\gamma_{n+1}\xi_{n+1}   +  O(\norm{\D_n}^2\gamma_n + \gamma_n^2).  \vspace*{-6pt}
  \end{equation}
  See Theorem \ref{thm:linear} for more details.
\end{itemize}
\vspace{-.6cm}
\subsection{Averaging in $T_{x_\star} \M$} \label{sec:pfsketch2}
\vspace{-.0856cm}
Our next step is to prove both asymptotic and non-asymptotic convergence rates for a general, perturbed linear recursion (resembling \eq{final_proof_sketch}) of the form,
\begin{align}
  \D_{n}=\D_{n-1} -\gamma_n \Hess f(x_\star) \D_{n-1}+ \gamma_n (\eps_n+\xi_{n}+e_{n}),\label{eq:rec_with_error}
\end{align}
under appropriate assumptions on the error $\{ e_n \}_{n \geq 0}$ and noise $\{ \eps_n \}_{n \geq 0}$, $\{ \xi_n \}_{n \geq 0}$ sequences detailed in \myapp{conv_rates}. Under these assumptions we can derive an asymptotic rate for the average, $\bar{\Delta}_n = \frac{1}{n}\sum_{i=1}^{n} \Delta_i$, under a first-moment condition on $e_n$:
  \[
  \sqrt n \bar{\Delta}_n  \overset{D}{\to} \mathcal N (0,  \Hess f(x_\star)^{-1}\Sigma \Hess f(x_\star)^{-1}),
  \]
  and, under a slightly stronger second-moment condition on $e_n$ we have:
  \[
    \mathbb{E}[\Vert \bar{\Delta}_n \Vert ^2] \leq \frac{1}{n} \tr [\Hess f(x_\star)^{-1} \Sigma \Hess f(x_\star)^{-1}] +  O(n^{-2\alpha}) + O(n^{\alpha-2}),
  \]
  where $\Sigma$ denotes the asymptotic covariance of the noise $\eps_n$. The proof techniques are similar to those of \citet{polyak1992acceleration} and \citet{moulines2011non} so we do not detail them here. See Theorems \ref{thm:asymp_ave} and \ref{thm:nonasymp_ave} for more details.
  Note that $\bar{\Delta}_n$ is \textit{not} computable, but does have an interesting interpretation as an upper bound on the Riemannian center-of-mass, $K_n = \arg \min_{x \in \M}\sum_{i=1}^{n} \norm{R_{x}^{-1}(x_i)}^2$, of a set of iterates $\{ x_n \}_{n \geq 0}$ in $\M$
   \citep[see \mysec{com} and][for more details]{Afs11}.
\vspace{-3.11pt}
\subsection{From $T_{x_\star} \M$ back to $\M$} \label{sec:pfsketch3}
\vspace{-.0856cm}
Using the previous arguments, we can conclude that the averaged vector $\bar{\Delta}_n$ obeys both asymptotic and non-asymptotic Polyak-Ruppert-type results. However, $\bar{\Delta}_n$
is \textit{not} computable. Rather, $\tilde{\Delta}_n = R_{x_\star}^{-1}(\tilde{x}_n)$ corresponds to the computable, Riemannian streaming average $\tilde{x}_n$ defined in \eq{ave_grad_desc}. In order to conclude our result, we argue that $\tilde{\Delta}_n = R_{x_\star}^{-1}(\tilde{x}_n)$ and $\bar{\Delta}_n$ are close up to $O(\gamma_n)$ terms. The argument proceeds in two steps:
\begin{itemize}
\vspace*{-6pt}
  \item Using the fact that $x \to \norm{R_{x_\star}^{-1}(x)}^2$ is retraction convex we can conclude that $\E[\norm{\Delta_n}^2] = O(\gamma_n)$ implies that  $\E[\Vert \tilde{\Delta}_n\Vert^2] = O(\gamma_n)$ as well. See Lemma \ref{lem:avg_iters} for more details.
  \vspace*{-6pt}
  \item Then, we can locally expand \eq{ave_grad_desc} to find that,
  \vspace*{-6pt}
  \[
    \tilde{\Delta}_{n+1} = \tilde{\Delta}_n + \frac{1}{n+1}(\Delta_{n+1}-\tilde{\Delta}_n)+\tilde{e}_n,
 \vspace*{-6pt} \]
  where $\E[\Vert \tilde{e}_n\Vert] = O(\frac{\gamma_n}{n+1})$. Rearranging and summing this recursion shows that $\tilde{\Delta}_n = \bar{\Delta}_n+e_n$ for $\E[\Vert e_n\Vert] = O(\gamma_n)$, showing these terms are close. See Lemma \ref{lem:stream_avg_iters} for details.
  \vspace*{-6pt}
\end{itemize}

\section{Proposed algorithm: \ragd}
\begin{algorithm}[hbtp]
	\caption{Riemannian-Nesterov($x_0, \gamma_0, \{h_k\}_{k=0}^{T-1}, \{\beta_k\}_{k=0}^{T-1}$)} \label{alg:riemannian-ag}
	\SetAlgoLined
	\SetKwInput{KwData}{Parameters}
	\KwData{initial point $x_0\in\mathcal{X}$, $\gamma_0>0$, step sizes $\{h_k\le\frac{1}{L}\}$, shrinkage parameters $\{\beta_k>0\}$}
	initialize $v_0 = x_0$\\
	\For{$k=0,1,\dots,T-1$}{
		Compute $\alpha_k\in(0,1)$ from the equation
		$\alpha_k^2 = h_k\cdot\left((1-\alpha_k)\gamma_k + \alpha_k\mu\right)$\\
		Set $\overline{\gamma}_{k+1} = (1-\alpha_k)\gamma_k + \alpha_k\mu$\\
\nl	\label{ln:y_k}	Choose $y_k = \Exp_{x_k}\left(\frac{\alpha_k\gamma_k}{\gamma_k+\alpha_k\mu}\Exp_{x_k}^{-1}(v_k)\right)$\\
		Compute $f(y_k)$ and $\nabla f(y_k)$\\
\nl	\label{ln:x_k+1}	Set $x_{k+1} = \Exp_{y_k}\left(-h_k\nabla f(y_k)\right)$\label{eq:x-k+1} \\ 
\nl	\label{ln:v_k+1}	Set $v_{k+1} = \Exp_{y_k}\left(\frac{(1-\alpha_k)\gamma_k}{\overline{\gamma}_{k+1}} \Exp_{y_k}^{-1}(v_k) - \frac{\alpha_k}{\overline{\gamma}_{k+1}} \nabla f(y_k)\right)$\label{eq:v-k+1}\\ 
		Set $\gamma_{k+1} = \frac{1}{1+\beta_k}\overline{\gamma}_{k+1}$
	}
	{\bf Output:} $x_T$
\end{algorithm}

\begin{figure}[hbt]
	\centering \def\svgwidth{200pt}
	\input{figures/alg1.pdf_tex} \def\svgwidth{200pt} 
	\input{figures/alg1-2.pdf_tex}
	\caption{Illustration of the geometric quantities in Algorithm \ref{alg:riemannian-ag}. \textbf{Left:} iterates and minimizer $x^*$ with $y_{k}$'s tangent space shown schematically. \textbf{Right:} the inverse exponential maps of relevant iterates in $y_{k}$'s tangent space. Note that $y_k$ is on the geodesic from $x_k$ to $v_k$ (Algorithm \ref{alg:riemannian-ag}, Line \ref{ln:y_k}); $\Exp_{y_k}^{-1}(x_{k+1})$ is in the opposite direction of $\mathrm{grad} f(y_k)$ (Algorithm \ref{alg:riemannian-ag}, Line \ref{ln:x_k+1}); also note how $\Exp_{y_k}^{-1}(v_{k+1})$ is constructed (Algorithm \ref{alg:riemannian-ag}, Line \ref{ln:v_k+1}).}
\end{figure}

Our proposed optimization procedure is shown in Algorithm \ref{alg:riemannian-ag}. We assume the algorithm is granted access to oracles that can efficiently compute the exponential map and its inverse, as well as the Riemannian gradient of function $f$. In comparison with Nesterov's accelerated gradient method in vector space \citep[p.76]{nesterov2004introductory}, we note two important differences: first, instead of linearly combining vectors, the update for iterates is computed via exponential maps; second, we introduce a paired sequence of parameters $\{(\gamma_k, \overline{\gamma}_k)\}_{k=0}^{T-1}$, for reasons that will become clear when we analyze the convergence of the algorithm. 

Algorithm \ref{alg:riemannian-ag} provides a general scheme for Nesterov-style algorithms on Riemannian manifolds, leaving the choice of many parameters to users' preference. To further simplify the parameter choice as well as the analysis, we note that the following specific choice of parameters
\[ \gamma_0\equiv\gamma = \frac{\sqrt{\beta^2+4(1+\beta)\mu h}-\beta}{\sqrt{\beta^2+4(1+\beta)\mu h}+\beta}\cdot \mu, \qquad h_k\equiv h, \forall k\ge 0, \qquad \beta_k\equiv \beta > 0, \forall k\ge 0, \]
which leads to Algorithm \ref{alg:constant-step}, a constant step instantiation of the general scheme. We leave the proof of this claim as a lemma in the Appendix.

\begin{algorithm}[hbtp]
	\caption{Constant Step Riemannian-Nesterov($x_0, h, \beta$)}  \label{alg:constant-step}
	\SetAlgoLined
	\SetKwInput{KwData}{Parameters}
	\KwData{initial point $x_0\in\mathcal{X}$, step size $h\le\frac{1}{L}$, shrinkage parameter $\beta > 0$}
	initialize $v_0 = x_0$\\
	set $\alpha = \frac{\sqrt{\beta^2+4(1+\beta)\mu h}-\beta}{2}$,~ $\gamma = \frac{\sqrt{\beta^2+4(1+\beta)\mu h}-\beta}{\sqrt{\beta^2+4(1+\beta)\mu h}+\beta}\cdot \mu$,~ $\overline{\gamma} = (1+\beta)\gamma$\\
	\For{$k=0,1,\dots,T-1$}{
		Choose $y_k = \Exp_{x_k}\left(\frac{\alpha\gamma}{\gamma+\alpha\mu}\Exp_{x_k}^{-1}(v_k)\right)$\\
		Set $x_{k+1} = \Exp_{y_k}\left(-h\nabla f(y_k)\right)$ \\ 
		Set $v_{k+1} = \Exp_{y_k}\left(\frac{(1-\alpha)\gamma}{\overline{\gamma}} \Exp_{y_k}^{-1}(v_k) - \frac{~\alpha~}{~\overline{\gamma}~} \nabla f(y_k)\right)$
	}
	{\bf Output:} $x_T$
\end{algorithm}

We move forward to analyzing the convergence properties of these two algorithms in the following two sections. In Section \ref{sec:general-analysis}, we first provide a novel generalization of Nesterov's estimate sequence to Riemannian manifolds, then show that if a specific tangent space distance comparison inequality (\ref{eq:base-change-assumption}) always holds, then Algorithm \ref{alg:riemannian-ag} converges similarly as its vector space counterpart. In Section \ref{sec:constant-step-analysis}, we establish sufficient conditions for this tangent space distance comparison inequality to hold, specifically for Algorithm \ref{alg:constant-step}, and show that under these conditions Algorithm \ref{alg:constant-step} converges in $O\left(\sqrt{\frac{L}{\mu}}\log(1/\epsilon)\right)$ iterations, a faster rate than the $O\left(\frac{L}{\mu}\log(1/\epsilon)\right)$ complexity of Riemannian gradient descent.


%%% Local Variables:
%%% mode: latex
%%% TeX-master: "colt2018"
%%% End:


\section{Conclusion}
We study the problem of incentivizing exploration with heterogeneous
user preferences.
We proposed an algorithm that achieves expected cumulative regret
$O(\ARMNUM \e^{2/\MinProb} + \ARMNUM \log^3(T))$,
using expected cumulative payments of $O(\ARMNUM^2 \e^{2/\MinProb})$.
It is possible to improve these bounds to polynomial (in \ARMNUM and
$1/\MinProb$) when \MinProb is known or the preference distribution is
discrete.
In fact, we conjecture that this should be possible even in the full
generality of our model.
As a first step towards such a polynomial bound, we can obtain an exponential dependence on
$1/(\MinProb \ARMNUM)$ by changing the probability threshold to be $\frac{1}{\ARMNUM\log(s)}$
\footnote{This will lead to a different dependence on $\ARMNUM$
in the regret bound as well as the payment bound}, which gives polynomial dependence unless some
arm has a much smaller fraction of the population preferring it.

Taking this goal one step further, we would like to 
develop algorithms that do not require all arms to be preferred by a
strictly positive fraction of agents.
An alternate algorithm might only incentivize an arm if its estimated
attribute vector is close enough to a Pareto frontier.
The regret will then be $\Omega(\log(T))$ when at least one arm falls
below the Pareto frontier, as we no longer have free exploration of
all arms. 
It is likely that a bound will deteriorate as the number of such
unpreferred arms increases.

Finally, it would be desirable to generalize to utility
functions beyond inner products.
We believe that similar results hold for arbitrary
Lipschitz-continuous utility functions of the arm's attribute vector,
and that only minor modifications are necessary to the algorithm and
proofs.


\acks{Yingyu Liang would like to acknowledge that support for this research was provided by the Office of the Vice Chancellor for Research and Graduate Education at the University of Wisconsin –Madison with funding from the Wisconsin Alumni Research Foundation.}




\bibliography{bibfile}


\appendix
\section{Proof of Warm Start for Learning One of the Weights}
\label{sec:proof_one}


We prove the following lemma related to the output of Algorithm~\ref{alg:one}.

\medskip
\noindent
\textbf{Lemma~\ref{lem:warmstart}}
{\it
With probability at least $1-\delta$, $\min_i \|w_i - a_T\|_2 \le O(\sigma^2\veps)$.
}
\medskip

Before proving this lemma, we first need the following lemma about the clustering, which is crucial for constructing the coefficients. As we shall see, we will use this lemma on $r_i = \| \bSigma_i (w_i - a_t) \|_2^2$. Roughly speaking, $f(\sqrt{r_i})$ is the weight of $\bSigma_i^2$ and $f'(\sqrt{r_i})$ is the weight of $\bSigma_i^2 (w_i - a_t) $. Therefore, we would like $f(\sqrt{r_i})$ to be small compare to  $f'(\sqrt{r_i})$ to identify the subspace spanned by $\bSigma_i^2 (w_i - a_t) $. 

%\begin{lemma}[Coefficients]\label{lem:cluster}
%For every $k \geq 2$, every $\rho > 1$, every $r_1, \cdots, r_k \in [\frac{1}{\rho}, \rho]$, and every $\veps > 0$, one can find in time $O(k\log k)$ an integer $0<s\le k$ and centers $1/\rho \leq z_1 \leq  \cdots \leq z_s \leq \rho$ such that for $f(x) = \prod_{p = 1}^s (x^2 - z_p)$ the following holds.
%\begin{enumerate}
%\item For $r = \min\{r_i\}_{i = 1}^k$ and every $i \in [k]$, $|f(\sqrt{r_i})| \leq \veps |\sqrt{r}f'(\sqrt{r}) |  $. 
%\item $|\sqrt{r}f'(\sqrt{r})| \geq  \left( \frac{ \veps}{\rho} \right)^k$. 
%\item For all $x$ with $x^2\in [1/\rho, \rho]$, $|f'(x)| \leq 2k \rho^k$ and $|f''(x)| \leq 4k^2 \rho^k$.
%\end{enumerate}
%\end{lemma}

%\clusterlemma* % restatable doesn't work

\medskip
\noindent
\textbf{Lemma~\ref{lem:cluster} (Coefficients)}
{\it
For every $k \geq 2$, every $\rho > 1$, every $r_1, \cdots, r_k \in [\frac{1}{\rho}, \rho]$, and every $\veps > 0$, one can find in time $O(k\log k)$ an integer $0<s\le k$ and centers $1/\rho \leq z_1 \leq  \cdots \leq z_s \leq \rho$ such that for $f(x) = \prod_{p = 1}^s (x^2 - z_p)$ the following holds.
\begin{enumerate}
\item For $r = \min\{r_i\}_{i = 1}^k$ and every $i \in [k]$, $|f(\sqrt{r_i})| \leq \veps |\sqrt{r}f'(\sqrt{r}) |  $. 
\item $|\sqrt{r}f'(\sqrt{r})| \geq  \left( \frac{ \veps}{\rho} \right)^k$. 
\item For all $x$ with $x^2\in [1/\rho, \rho]$, $|f'(x)| \leq 2k \rho^k$ and $|f''(x)| \leq 4k^2 \rho^k$.
\end{enumerate}
}
\medskip

\begin{proof}[Proof of Lemma \ref{lem:cluster}]
%Let us without loss of generality assume that $r=r_1 \leq r_2 \leq \cdots \leq r_k$.  Let us define $z_1 = r_1$, and let $j \in [k ]$ be the smallest  index such that $r_j \geq z_1 + \frac{\veps}{{\rho}}$. If no suck index exists, we let $j = k+ 1$.  If $j \leq k$, let us define for $s \leq k$:
%\begin{align}
%z_2 = z_1 + \veps,  \quad z_3 = r_j, z_4 = r_{j + 1}, \cdots, z_s = r_k.
%\end{align}
%
%Otherwise if $j = k + 1$, we just pick $s = 2$ and let $z_1 = r_1$ and $z_2 = z_1 + \veps$. Therefore, we still have that $s \leq k$. 
Let us without loss of generality assume that $r=r_1 \leq r_2 \leq \cdots \leq r_k$.  Let us define $z_1 = r_1$, and let $j \in [k ]$ be the smallest  index such that $r_j \geq z_1 + \frac{\veps}{{\rho}}$. If no such index exists, we let $s = 1$ and the statements in the lemma are true.  If such $j$ exists, let us define:
\begin{align}
z_2 = r_j, z_3 = r_{j + 1}, \cdots, z_s = r_k.
\end{align}

Now, we know that 
\begin{align}
|\sqrt{r}f'(\sqrt{r})| &= 2 r \prod_{p = 2}^{s} |r - z_p| \geq  \left( \frac{ \veps}{{\rho}} \right)^k.
\end{align}

On the other hand, for every $i \geq j$, $f(\sqrt{r_i}) = 0$. For $i < j$ we have: 
\begin{align}
|f(\sqrt{r_i})| &=  |r_i - r|\prod_{p = 2}^{s} |r_i - z_p|
\\
&\leq \frac{\veps}{{\rho}} \prod_{p = 2}^{s} |r_i - z_p| \leq \veps r \prod_{p = 2}^{s} |r - z_p|  \leq \veps  |\sqrt{r} f'(\sqrt{r})|.
\end{align}


We now consider the derivative and second order derivative of $f(x)$ for $x^2 \in [0, \rho]$. By elementary calculation, we know that 
\begin{align}
|f'(x)| &= \left|\sum_{p = 1}^s 2x \prod_{q \not= p} (x^2 - z_q) \right|
\\
& \leq 2\sum_{p = 1}^s |x |  \prod_{q \not= p} \left| x^2 - z_q \right|
\\
& \leq 2 k \rho^k.
\end{align}

Similarly we can get that $|f''(x)| \leq 4k^2 \rho^k$. 
\end{proof}

We also need the following bound for the $k$-SVD of a matrix.

\begin{lemma}\label{lem:k_SVD}
Let $\bX_1, \cdots, \bX_k$ be $k$ rank-one matrices in $\mathbb{R}^{d \times d}$ such that each $\bX_i = x_i x_i^{\top}$, for every $\veps \geq 0$, every PSD matrix $\bM \in \mathbb{R}^{d \times d}$ such that 
\begin{align}
\left\| \bM - \sum_{i = 1}^k \bX_i \right\|_2 \leq \veps \| \bX_1 \|_2
\end{align}
Let $\bU \in \mathbb{R}^{d \times k}$ be the matrix consists of the top-k singular vectors of $\bM$, then we have
\begin{align}
\| x_1^{\top}\bU \|_2 \geq \left(1 - (\veps k)^{1/3}\right)\| x_1 \|_2
\end{align}

\end{lemma}
\begin{proof}[Proof of Lemma \ref{lem:k_SVD}]
Let us denote $\sigma_1 \geq  \cdots \geq \sigma_k \geq \sigma_{k + 1} =  0$ as the $k + 1$ singular values of $\sum_{i = 1}^k \bX_i$ with corresponding singular vectors $v_1, \cdots, v_k$ (and $v_{k + 1}$). For every $v_i$, by definition
\begin{align}
v_i^{\top} \left(\sum_{j = 1}^k \bX_j \right)v_i = \sigma_i
\end{align}
So we have $v_i^{\top} \bX_1 v_i  \leq \sigma_i$. Let $\bV_i  \in \mathbb{R}^{d \times i}$ defined as $\bV_i= (v_1, \cdots, v_i)$. By Gap-free Wedin theorem in~\citep{allen2016lazysvd} (see Lemma~\ref{lem:gapfree_wedin}), we know that 
\begin{align}
\| (\bI - \bU \bU^{\top}) \bV_i \|_2 \leq \frac{\veps \| x_1 \|_2^2}{\sigma_i}.
\end{align}
Thus, $\|x_1^{\top} (\bV_i  \bV_i^{\top})(\bI - \bU \bU^{\top}) \|_2 \leq \frac{\veps \| x_1 \|_2^3}{\sigma_i}$. 

On the other hand, since $x_1 \in \text{span}\{v_1, \cdots, v_k \}$, 
\begin{align}
\|x_1^{\top} (\bI - \bV_i  \bV_i^{\top}) \|_2 &= \|x_1^{\top} (\bV_{k} \bV_{k}^{\top} - \bV_i  \bV_i^{\top}) \|_2 
\\
&\leq  \sum_{j = i + 1}^k |x_i^{\top} v_k| \leq k \sqrt{\sigma_{i + 1}}.
\end{align}
Therefore, we know that 
\begin{align}
\| x_1^{\top}  (\bI - \bU \bU^{\top}) \|_2 \leq \frac{\veps \| x_1 \|_2^3}{\sigma_i} + k \sqrt{\sigma_{i + 1}}.
\end{align}

If $\sigma_1 \geq  \frac{\| x_1\|_2^2 \veps^{2/3}}{k^{2/3}}$, by picking $i$ to the largest index in $[k]$ such that $\sigma_i \geq \frac{\| x_1\|_2^2 \veps^{2/3}}{k^{2/3}}$,  we get that 
\begin{align}
\| x_1^{\top}  (\bI - \bU \bU^{\top}) \|_2 \leq (\veps k)^{1/3} \| x_1 \|_2
\end{align}

If $\sigma_1 \leq  \frac{\| x_1\|_2^2 \veps^{2/3}}{k^{2/3}}$, then we can just use $\|x_1^{\top}  \|_2 \leq k \sqrt{\sigma_1}$ to complete the proof.
%
%Let us denote the singular values of $\bM$ as $\sigma_1 \geq \sigma_2 \geq \cdots \geq \sigma_d \geq 0$ with corresponding singular vectors $x_1, \cdots , x_d$. By Wedin's theorem, we know that for every $i \in [d]$
%\begin{align}
%
%\end{align}
%
%
%
%
%By Wely's theorem, we know that $\sigma_{k + 1}(\bM) \leq  \veps \| \bX_1 \|_2 = \veps \| x_1 \|_2^2$. Thus, multiplying $x_1$ with $\bM$ we have:
%\begin{align}
%x_1^{\top} \bM x_1 = \|x_1^{\top} \bU \|_2^2
%\end{align}
\end{proof}



We are now ready to prove the following important lemma about the correlation between $\bU$ and $\bSigma_i^2(w_i - a_t)$.

\begin{lemma}\label{lem:correlation}
Let $j = \argmin_{1\le i\le k}  \|\bSigma_i (w_i - a_t) \|_2$, we have that in the $t$-th iteration of Algorithm~\ref{alg:one}, the $\bU_t$ satisfies
\begin{align}
\frac{\|\bU_t^{\top} \bSigma_j^2 (w_j - a_t) \|_2}{\| \bSigma_j^2 (w_j - a_t)  \|_2 } \geq \frac{1}{2}.
\end{align}

\end{lemma}
\begin{proof}[Proof of Lemma \ref{lem:correlation}]
Suppose $z \sim \mathcal{N}(0, \bSigma^2)$, we know that $z = \bSigma g$ where $g \sim \mathcal{N}(0, \bI)$. For every vector $a$,
\begin{align}
\E\left[\langle z, a \rangle^{2p} z z^{\top}\right] &= \bSigma \E\left[ \langle g, \bSigma a \rangle^{2p} g g^{\top}\right]  \bSigma
\\
&=  (2p - 1)!! \bSigma \left( 2p  \bSigma a a^{\top} \bSigma \| \bSigma a\|_2^{2p - 2}  + \| \bSigma a \|_2^{2p} \bI \right) \bSigma
\\
& =   (2p - 1)!!  \| \bSigma a\|_2^{2p} \left( 2p \frac{ \bSigma^2 a a^{\top} \bSigma^2 }{ \|\bSigma a \|_2^2 }+  \bSigma^2 \right).
\end{align}
Thus, we have
\begin{align}
\frac{1}{(2p - 1)!!}\E\left[ \alpha_i^{2p} x_i x_i^{\top} \right] &= \sum_{i = 1}^k p_i   \| \bSigma_i (w_i - a_t)\|_2^{2p} \left( 2p \frac{ \bSigma_i^2 (w_i - a_t) (w_i - a_t)^{\top} \bSigma_i^2 }{ \|\bSigma_i (w_i - a_t) \|_2^2 }+  \bSigma_i^2 \right).
\end{align}


Since in the $t$-th iteration, the labels $\alpha_i$ we fit to Algorithm~\ref{alg:pww} comes from $\alpha_{\ell} = \langle x_{\ell}, w^{(\ell)} - a_t \rangle$, we know that 
\begin{align}
\E[\bM] =  \sum_{i = 1}^k p_i    \sum_{p = 0}^k \left( c_p \| \bSigma_i (w_i - a_t)\|_2^{2p} \left( 2p \frac{ \bSigma_i^2 (w_i - a_t) (w_i - a_t)^{\top} \bSigma_i^2 }{ \|\bSigma_i (w_i - a_t) \|_2^2 }+ \bSigma_i^2  \right) \right).
\end{align}
Let us define the signal matrix $\bX_i$ as 
\begin{align}
\bX_i &=   \frac{ \bSigma_i^2 (w_i - a_t) (w_i - a_t)^{\top} \bSigma_i^2  }{ \|\bSigma_i (w_i - a_t) \|_2^2 }  \left( \sum_{p = 0}^k  2p c_p \| \bSigma_i (w_i - a_t)\|_2^{2p} \right) 
\\
&= \frac{ \bSigma_i^2 (w_i - a_t) (w_i - a_t)^{\top} \bSigma_i^2  }{ \|\bSigma_i (w_i - a_t) \|_2^2 }  \left( f'( \| \bSigma_i (w_i - a_t) \|_2) \| \bSigma_i (w_i - a_t) \|_2 \right)
\end{align}
and the noise matrix $\bY_i$ as
\begin{align}
\bY_i &=\bSigma_i^2 \left( \sum_{p = 0}^k  c_p \| \bSigma_i (w_i - a_t)\|_2^{2p} \right) 
\\
&= \bSigma_i^2 f(\| \bSigma_i (w_i - a_t)\|_2)
\end{align}
such that 
\begin{align}
\E[\bM] = \sum_{i=1}^k p_i (\bX_i + \bY_i).
\end{align}


For $j = \argmin\{\| \bSigma_i (w_i - a_t) \|_2) \}_{i = 1}^k$, let us denote 
$$
\beta := f'( \| \bSigma_j (w_j - a_t) \|_2) \| \bSigma_j (w_j - a_t) \|_2.
$$


Let us recall that $\veps^{(g)}$ is the error incurred when estimating $\{\| \bSigma_i (w_i - a_t) \|_2\}_{i = 1}^k$. $\veps^{(p)}$ is the error when constructing the coefficients of the polynomial (for sufficiently large  $\rho$ such that $\rho \geq  \max\{\| \bSigma_i (w_i - a_t) \|_2^2) \}_{i = 1}^k$ as we will show later in this proof). Thus, by Lemma~\ref{lem:cluster}, we know that 
\begin{align}
\| \bY_i \|_2& \leq \| \bSigma_i^2 \|_2 | f(\| \bSigma_i (w_i - a_t)\|_2)|
\\
& \leq \| \bSigma_i^2 \|_2(| f(\sigma_i)| + 2k \rho^k \left|\sigma_i - \| \bSigma_i (w_i - a_t)\|_2 \right| )
\\
&\leq  \| \bSigma_i^2 \|_2 ( \veps^{(p)} \beta + 4k \rho^k \veps^{(g)}).
\end{align} 
Similarly we have
\begin{align}
\|\bX_j\|_2 \geq \sigma_{\min}(\bSigma_j^2) \beta.
\end{align}
And we have $\beta \geq \left(\frac{\veps^{(p)}}{\rho} \right)^k - 8 k^2 \rho^k \veps^{(g)} \sigma^2$.  


Notice that $ \min\{\| \bSigma_i (w_i - a_t) \|_2) \}_{i = 1}^k \leq  \min\{\| \bSigma_i (w_i ) \|_2) \}_{i = 1}^k$, which implies that $\| a_1 \|_2 \leq \sigma^4$. Therefore, we can take $\rho =O\left( \max\left\{2\sigma^{10}, \frac{1}{\veps} \right\}\right)$. Thus, by our choice of parameter, we know that for $\veps^{(e)} \leq \frac{1}{100 k} $, 
\begin{align}
\left\|\E[\bM] - \sum_{i= 1}^k p_i \bX_i \right\|_2 \leq \veps^{(e)}\| \bX_j \|_2/2.
\end{align}
Using the sample complexity bound Lemma~\ref{lem:gsb}, by our choice of $m$ we know that 
\begin{align}
\left\|\bM - \E[\bM] \right\|_2 \leq \veps^{(e)}\| \bX_j \|_2/2.
\end{align}
Thus, apply Lemma \ref{lem:k_SVD} on $\bM$ we know that 
\begin{align}
\frac{\|\bU_t^{\top} \bX_j \bU_t \|_2}{\|\bX_j\|_2} \geq 1 - \left(\veps^{(e)}k\right)^{1/3} \geq \frac{3}{4}.
\end{align}
Indeed, this also implies that 
\begin{align}
\frac{\|\bU_t^{\top} \bSigma_j^2 (w_j - a_t) \|_2}{\| \bSigma_j^2 (w_j - a_t)  \|_2 } \geq \frac{1}{2}
\end{align}
completing the proof.
\end{proof}


Now we can prove the main lemma regarding the per-iteration improvement of Algorithm~\ref{alg:one}.

%\oneiterlemma* % doens't work


\medskip
\noindent
\textbf{Lemma~\ref{lem:one_iter} (Coefficients)}
{\it
For every $t \in \{0, 1, \cdots, T - 1\}$ and $\delta > 0$, as long as $\sigma_t = \Omega ( \sigma \veps)$, then with probability at least $1 - \delta$,
$$
\sigma_{t + 1}^2 \leq \left(1 - \frac{1}{200 k \sigma} \right) \sigma_t^2.
$$
}
\medskip

\begin{proof}[Proof of Lemma~\ref{lem:one_iter}]
At $t$-th iteration let $j = \argmin \{ \|\bSigma_i (w_i - a_t) \|_2\}_{i = 1}^k$, we know that 
\begin{align}
\frac{\|\bU_t^{\top} \bSigma_j^2 (w_j - a_t) \|_2}{\| \bSigma_j^2 (w_j - a_t)  \|_2 } \geq \frac{1}{2}.
\end{align}

By definition, $v = \frac{\bU_t \gamma}{\| \bU_t \gamma \|_2}$ for $\gamma \in \mathcal{N}(0, \bI)$. Thus, using elementary calculation of Gaussian random variables, we have: with probability at least $1/4$, 
\begin{align}
\frac{v^{\top} \bSigma_j^2 (w_j - a_t) }{ \| \bSigma_j^2 (w_j- a_t)  \|_2 } \geq \frac{1}{10 \sqrt{k}}
\end{align}
which implies that
\begin{align}
\left\|\bSigma_j (w_j - a_t  - \eta  v) \right\|_2^2 &= \left\|\bSigma_j (w_j - a_t ) \right\|_2^2 - 2  \eta \langle \bSigma_j (w_j - a_t), \bSigma_j  v\rangle + \eta^2 \| \bSigma_j  v \|_2^2
\\
&=  \left\|\bSigma_j (w_j - a_t ) \right\|_2^2  - 2 \eta \langle \bSigma_j^2 (w_j - a_t),  v\rangle + \eta^2 \| \bSigma_j  v \|_2^2
\\
& \leq  \left\|\bSigma_j (w_j - a_t ) \right\|_2^2  - \frac{\eta}{5 \sqrt{k}} \| \bSigma_j^2 (w_j - a_t) \|_2 + \eta^2  \sigma.
\end{align}


Let $\eta = \frac{\| \bSigma_j^2 (w_j - a_t) \|_2 }{10 \sigma \sqrt{k}}$. Then we know that 
$$
\left\|\bSigma_j (w_j - a_t  - \eta v) \right\|_2^2 \leq \left(1 - \frac{1}{100 k \sigma} \right)  \left\|\bSigma_j (w_j - a_t ) \right\|_2^2.
$$


Thus, since we can estimate $\left\|\bSigma_j (w_j - a_t  - \eta  v) \right\|_2$ up to accuracy $\veps/(k\sigma)$ using the algorithm proposed in~\citep{moitra2010settling}, as long as $\sigma_{t} = \Omega\left( \sigma\veps \right)$, we will have that $\sigma_{t + 1}^2 \leq  \left(1 - \frac{1}{200 k \sigma} \right)\sigma_t^2$. 
\end{proof}

This immediately leads to the main lemma regarding the output of Algorithm~\ref{alg:one}.


%\warmstartlemma*

\begin{proof}[Proof of Lemma~\ref{lem:warmstart}]
By Lemma~\ref{lem:one_iter}, and by the choice of the parameters in the algorithm,
$
  \sigma_T \le O(\sigma \veps).
$
Then for $j = \min_i \{\|\Sigma_i (w_i - a_T)\|_2\}$ we have 
$
  \|\Sigma_j (w_j - a_T)\|_2 \le O(\sigma \veps)
$
and thus
$ 
  \|w_j - a_T\|_2 \le O(\sigma^2 \veps).
$
\end{proof}

\section{Proof for Learning One of the Weights from Warm Start} \label{sec:proof_gd}


Without loss of generality, let us assume that we have an $v$ such that $\| v - w_1 \|_2$ is reasonably small. We will show that the update rule used in the algorithm can recover $w_1$ up to error $\veps$ with this $v$. 
%In particular, we consider the following algorithm: with $v^{(1)} = v_1$, each iteration, update
%$$v^{(t + 1)} = v^{(t)} + \eta_t \E \left[\frac{\sign(\langle w - v^{(t)}, x \rangle)}{|\langle w - v^{(t)}, x \rangle| + \zeta} x \right]$$
%
%Which is equivalent to the gradient descent update on the following concave objective function:
%$$f(v) = -  \E \left [\log (|\langle w - v, x \rangle| + \zeta)  \right]$$
It is equivalent to (the empirical version of) the gradient descent update to minimize the following concave objective function:
$$
  g(v) = \E \left [\log (|\alpha - \langle v, x \rangle| + \zeta)  \right].
$$


%\begin{lemma}[Gradient descent] \label{lem:gradient}
%Suppose $\| w_1 - v \| \le \zeta / \sigma$ and $\Delta > (\zeta / \sigma + 32\zeta/p_{\min})$. Let $\eta_t = c\frac{\zeta p_{\min}}{d} \times \left( 1 - c^3\frac{p_{\min}^2}{d} \right)^t$ for a sufficiently small constant $c>0$, and each gradient step uses $\text{poly}\left(\frac{1}{\zeta}, \frac{1}{p_{\min}}\right)$ examples. Then with high probability, after $T = O\left(\frac{d}{p_{\min}^2}  \log \frac{\zeta}{\veps}\right)$ steps,
%$
  %\| w_1 - v^{(T)} \| \le \veps.
%$
%\end{lemma}

%\gradientlemma*


\medskip
\noindent
\textbf{Lemma~\ref{lem:gradient} (Gradient descent)}
{\it
Suppose there exists $i \in [k]$ such that $\| w_i - v \|_2 \le \zeta / \sigma$. Then with high probability, Algorithm~\ref{alg:gd} outputs a vector $v^{(T)}$ such that
$
  \| w_{i} - v^{(T)} \| \le \veps.
$
}
\medskip

\begin{proof}[Proof of Lemma~\ref{lem:gradient}]
First, suppose we have the gradient on the expectation, i.e., we have $\nabla g(v^{(t)})$. For this gradient descent update rule,
by Lemma~\ref{lem:inverse_gaussian}, we know that 
\begin{align*}
 \left\langle - \nabla g(v^{(t)}), w_1 - v^{(t)} \right\rangle 
  & = \E \left[\frac{\sign(\alpha - \langle v^{(t)}, x \rangle) \langle w_1 - v^{(t)}, x \rangle}{|\alpha - \langle v^{(t)}, x \rangle| + \zeta}  \right] 
	\\
	& = p_1 \E_{y \sim \mathcal{N}(0, 1)} \E \left[\frac{\sign(\langle \bSigma_1 (w_1 - v^{(t)}), y \rangle) \langle \bSigma_1 (w_1 - v^{(t)}), y \rangle}{|\langle \bSigma_1(w_1 - v^{(t)}), y \rangle| + \zeta}  \right]
  \\
  & ~~ + \sum_{j = 2}^k p_j \E_{y \sim \mathcal{N}(0, 1)} \E \left[\frac{\sign(\langle \bSigma_j (w_j - v^{(t)}), y \rangle) \langle \bSigma_j (w_1 - v^{(t)}), y \rangle}{|\langle \bSigma_j(w_j - v^{(t)}), y \rangle| + \zeta}  \right]
  \\
  & \geq \frac{1}{4} p_1 \frac{\| \bSigma_1 (w_1 - v^{(t)})\|_2 }{\|\bSigma_1 (w_1 - v^{(t)}) \|_2+ \zeta} - \sum_{j = 2}^k p_j \frac{\|\bSigma_1 (w_1 - v^{(t)}) \|_2} {\| \bSigma_j (w_j - v^{(t)})\|_2 }.
\end{align*}

Note that our assumption on $\zeta$ satisfies that
\begin{align} \label{eqn:grad_condition}
  \|\bSigma_1 (w_1 - v^{(t)}) \|_2 \leq \zeta, \quad \|\bSigma_j (w_j - v^{(t)}) \|_2 \geq 32 \zeta / p_{\min}, j\neq 1,
\end{align}

Therefore, a direct calculation shows that 
$$
  \left\langle -\nabla g(v^{(t)}), w_1 - v^{(t)} \right\rangle 
	\geq \frac{p_{\min}}{32 } \frac{\|\bSigma_1 (w_1 - v^{(t)}) \|_2}{\zeta} 
	\geq   \frac{p_{\min} \| w_1 - v^{(t)} \|_2}{32 \zeta}.
$$

However, we only have the empirical version of the gradient given as
$$
 -\tilde\nabla g(v^{(t)}) = \E_{(x_\ell,\alpha_\ell)} \nabla g_\ell(v), \mbox{~where~} -\nabla g_\ell(v^{(t)}) = \frac{\sign(\alpha_\ell - \langle v^{(t)}, x_\ell \rangle)}{|\alpha_\ell - \langle v^{(t)}, x_\ell \rangle| + \zeta} x_\ell.
$$

To apply concentration bound on the empirical version, we know that for for every example $(x,\alpha)$, 
$$
  \left\|\frac{\sign(\alpha - \langle  v^{(t)}, x \rangle)}{|\alpha - \langle v^{(t)}, x \rangle| + \zeta} x  \right\|_2 \leq \frac{\|x\|_2}{\zeta}. 
$$

Moreover, we know that the true gradient satisfies
$$
  \left\langle - \nabla g(v^{(t)}) , \frac{ w_1 - v^{(t)}}{\| w_1 - v^{(t)}\|_2}  \right\rangle \geq \frac{p_{\min}}{32 \zeta} 
$$

For every example $(x, \alpha)$, we have
$$
  \left|\left\langle  \frac{\sign(\alpha - \langle v^{(t)}, x \rangle) x }{|\alpha -  \langle v^{(t)}, x \rangle| + \zeta}   , \frac{ w_1 - v^{(t)}}{\| w_1 - v^{(t)}\|_2} \right\rangle \right|
	\leq 
	\frac{\left|\left\langle\frac{ w_1 - v^{(t)}}{\| w_1 - v^{(t)}\|_2}, x  \right\rangle\right|}{\zeta}.
$$

Using an elementary concentration bound of Gaussian random variables, we know that with $\text{poly}\left(\frac{1}{\zeta}, \frac{1}{p_{\min}}, \sigma\right)$ examples, the estimated gradient $\tilde{\nabla} g(v^{(t)})$ satisfies with high probability that
$$
  \| \tilde{\nabla } g(v^{(t)}) \|_2 \leq \frac{4 \sqrt{d}}{\zeta}, 
	\quad 
	\left\langle -\tilde{ \nabla} g(v^{(t)}) , \frac{ w_1 - v^{(t)}}{\| w_1 - v^{(t)}\|_2}  \right\rangle \geq \frac{p_{\min}}{64 \zeta}.
$$ 


Then when $\eta_t = c\frac{\zeta p_{\min} \| w_1 - v^{(t)}\|_2}{ d}$ for a sufficiently small constant $c>0$, and using the assumptions on $v^{(0)}$ and $\Delta$ to satisfy the condition (\ref{eqn:grad_condition}), by induction, we have
$$
 \| w_1 - v^{(t + 1)}\|_2^2 \leq \left( 1 - \Omega\left( \frac{p_{\min}^2}{ d} \right)\right) \| w_1 - v^{(t)} \|_2^2
$$
completing the proof.
\end{proof}

\section{Proof for Learning All the weights} \label{sec:proof_all}

%\maintheorem* % doesn't work
%\begin{restatable}[Main]{theorem}{maintheorem}\label{thm:main} 
\noindent
\textbf{Theorem~\ref{thm:main} (Main)} 
{\it
Assume the model~(\ref{def:mlr}) and assumptions \textbf{(A1)}-\textbf{(A3)}. Then Algorithm~\ref{alg:mlr} takes $N=d \log\left(\frac{d}{\veps}\right)\cdot \left(\frac{\sigma}{\Delta p_{\min}} \right)^{O(k)} +  \left( \frac{\sigma }{\Delta p_{\min} \veps} \right)^{O(k^2)}$ data points and in time $Nd \cdot \textrm{polylog}(k, d, \sigma, \frac{1}{\Delta}, \frac{1}{p_{\min}}, \frac{1}{\veps}) $  outputs a set of vectors $\{v_i\}_{i=1}^k$ that with high probability satisfy
$$
  \|v_i  - w_{\pi(i)} \|_2 \le \veps, \forall i \in [k], ~\mbox{for some permutation $\pi$}.
$$  
}
%\end{restatable}


\begin{proof}[Proof of Theorem~\ref{thm:main}]
The theorem follows from Lemma~\ref{lem:gradient} and Lemma~\ref{lem:one_iter}, the guarantees for the two subroutines used. Note that we recovers each weight up to $\veps_g \le \left(\frac{p_{\min} \Delta}{\sigma d}\right)^{\Omega(k^2)}$. 
Therefore, only a $\left(\frac{p_{\min} \Delta}{\sigma d}\right)^{\Omega(k^2)}$ fraction of data points from this component are not removed, and only a $\left(\frac{p_{\min} \Delta}{\sigma d}\right)^{\Omega(k^2)}$ fraction of data points from other components get removed. These only causes polynomially small errors to the quantities computed in later steps and can be tolerated by our analysis. 
\end{proof}

\section{Tools}


We shall use the following bounds on the Gaussian moments and it's concentration.

\begin{lemma}
Let $g \sim \mathcal{N}(0, \bI)$, then for every unit vector $w$, we have that for every non-negative integer $p$, 
$$
 \E\left[\langle w, g \rangle^{2p} g g^{\top}\right] =  (2p + 1)!! w w^{\top} + (2p - 1)!! (\bI - w w^{\top} ).
$$
\end{lemma}


Using a standard Matrix Bernstein bound, we can get:
\begin{lemma}[Gaussian sample bound]\label{lem:gsb}
Let $g \sim \mathcal{N}(0, \bSigma^2)$, let $g_1, \cdots, g_m$ be $m$ independent samples of $g$.  Then for every vector $w$ and every non-negative integer $p$ and every $\delta > 0$, we have that 
\begin{align}
\Pr\left[\left\|\frac{1}{m}\sum_{i = 1}^m \langle w, g_i \rangle^{2p} g_i g_i^{\top} - \E\left[\langle w, g \rangle^{2p} g g^{\top}\right] \right\|_2 = \Omega\left( \sqrt{  \frac{\| \bSigma w \|_2^{4p} \left\| \bSigma \right\|^4_2 d \log \frac{1}{\delta}}{ m} } \right)\right] \leq \delta
\end{align}
\end{lemma}




The following lemma gives an estimation of a (modified) inverse Gaussian, which is used for analyzing the gradient descent step of our algorithm.

\begin{lemma}\label{lem:inverse_gaussian}
Suppose $y \sim \mathcal{N}(0, \bI)$. For every $\zeta > 0$, for every vectors $a, b \in \mathbb{R}^d$, with $\rho = \frac{\langle a, b \rangle}{\|a\|_2 \|b\|_2}$,
$$ 
  \frac{1}{4}\frac{\rho \|a\|_2}{\zeta + \|b\|_2}  \leq \E\left[\frac{\sign(\langle b, y \rangle)\langle a , y \rangle}{| \langle b, y \rangle |+ \zeta} \right]  \leq \frac{\rho \|a\|_2}{\|b\|_2} \leq \frac{ \|a\|_2}{\|b\|_2}.
$$
\end{lemma}


\begin{proof}[Proof of Lemma \ref{lem:inverse_gaussian}]
Without loss of generality assume $b = \|b\|_2 e_1$ and $a =  \|a\|_2 (\rho e_1 + \sqrt{1 - \rho^2} e_2 )$. Then 
\begin{align*}
\E\left[\frac{\sign(\langle b, y \rangle)\langle a , y \rangle}{| \langle b, y \rangle |+ \zeta} \right]&= \E\left[\frac{ \|a\|_2 (\rho y_1+ \sqrt{1 - \rho^2} y_2 ) \sign(y_1)}{ \| b\|_2 |y_1| + \zeta} \right] 
\\
& = \rho \|a\|_2 \E\left[ \frac{|y _1|}{\|b\|_2 |y_1 |+ \zeta} \right]
\end{align*}
We know that 
$$  
  \frac{|y _1|}{\|b\|_2 |y_1 |+ \zeta} \leq \frac{1}{\|b\|_2},
$$
and when $|y_1| \ge 1$
$$
  \frac{|y _1|}{\|b\|_2 |y_1 |+ \zeta} \ge \frac{1}{\zeta + \|b\|_2}.
$$
%On the other hand,
%$$ \frac{|y _1|}{\|b\|_2 |y_1 |+ \zeta} \geq  \left\{ \begin{array}{ll}
        %\frac{1}{\zeta + \|b\|_2} & \mbox{if $|y_1| \geq 1$};\\
          %\frac{|y_1 |}{\zeta + \|b\|_2} & \mbox{if $|y_1| < 1$}.\end{array} \right.  $$
Therefore, we have
$$ \frac{1}{4}\frac{\rho \|a\|_2}{\zeta + \|b\|_2}  \leq \E\left[\frac{\sign(\langle b, y \rangle)\langle a , y \rangle}{| \langle b, y \rangle |+ \zeta} \right]  \leq \frac{\rho \|a\|_2}{\|b\|_2}.
$$
where the first inequality follows from $\E[1_{|y_1| \ge 1}] \ge 1/4$.
\end{proof}

We will also need the Gap-Free Wedin Theorem from~\citep{allen2016lazysvd}. 

\begin{lemma}[Gap-Free Wedin Theorem, Lemma B.3 in~\citep{allen2016lazysvd}] \label{lem:gapfree_wedin}
For $\veps \ge 0$, let $A, B$ be two PSD matrices such that $\|A - B\|_2 \le \veps$. For every $\mu \ge 0, \tau > 0$, let $U$ be the column orthonormal matrix consisting of eigenvectors of $A$ with eigenvalue $\le \mu$, let $V$ be column orthonormal matrix consisting of eigenvectors of $B$ with eigenvalue $\ge \mu + \tau$, then we have:
$$
  \| U\top V\| \le \frac{\epsilon}{\tau}.
$$
\end{lemma}


\end{document}
