In the stochastic bandit problem, the goal is to maximize an unknown function via a sequence of noisy evaluations. Typically, the observation noise is assumed to be independent of the evaluation point and to satisfy a tail bound uniformly on the domain; a restrictive assumption for many applications. In this work, we consider bandits with heteroscedastic noise, where we explicitly allow the noise distribution to depend on the evaluation point. We show that this leads to new trade-offs for information and regret, which are not taken into account by existing approaches like upper confidence bound algorithms (UCB) or Thompson Sampling. To address these shortcomings, we introduce a frequentist regret analysis framework, that is similar to the Bayesian framework of \cite{RussoLearningOptimizeInformationDirected2014}, and we prove a new high-probability regret bound for general, possibly randomized policies, which depends on a quantity we refer to as \emph{regret-information ratio}. From this bound, we define a frequentist version of Information Directed Sampling (IDS) to minimize the regret-information ratio over all possible action sampling distributions. This further relies on concentration inequalities for online least squares regression in separable Hilbert spaces, which we generalize to the case of heteroscedastic noise. We then formulate several variants of IDS for linear and reproducing kernel Hilbert space response functions, yielding novel algorithms for Bayesian optimization. We also prove frequentist regret bounds, which in the homoscedastic case recover known bounds for UCB, but can be much better when the noise is heteroscedastic. Empirically, we demonstrate in a linear setting with heteroscedastic noise, that some of our methods can outperform UCB and Thompson Sampling, while staying competitive when the noise is homoscedastic. 