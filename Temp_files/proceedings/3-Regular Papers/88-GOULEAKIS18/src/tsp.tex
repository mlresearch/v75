% !TeX root = main.tex
\subsection{Optimal Travelling Salesman Tour} \label{ssec:tspIOpt}

In this section we examine the \textit{metric travelling salesman problem} where we are given $n$ points (each provided by one record in $\Workers$) in a metric space
$\mathcal{X}$ and we wish to find the length of the minimum cycle going through each point in the set $\Truth\subseteq \Workers$ of correct answers. As usual we let
$\vec{x}_\Workers$ be the input vector with record values whose coordinates are points in the metric space $\mathcal{X}$. Our goal is to find a certification
scheme for this metric travelling salesman problem. That is, the algorithm should either output a sufficiently accurate value (according to \ref{eq:verH0}) for the
minimum weight cycle going through the points in $\vec{x}_\Truth$  or find a invalid record \footnote{Note that throughout this paper we don't consider the
computational complexity of the problems, since we are more interested in the number of verifications needed. Besides that in the case of Euclidean TSP we could
use the $(1 + \eps)$-approximation algorithm that we know in order to get similar results and avoid $\NP$-completeness.}. The following lemma combined with Theorem \ref{thm:wcont} give us
the desired result.

\begin{lemma}\label{lm:tsp}
    Let $f : \Domain^* \to \reals$ be the function mapping a set of points in a metric space $\mathcal{X}$ to their minimum TSP tour and let $v_1v_2\dots v_n$ be the
  minimum TSP tour. Also, let $\vec{w} \in \reals_+^n=(w_1,\dots,w_n)$, where
  $w_i=d(v_{i-1},v_i)+d(v_i,v_{i+1})$ and the second indices are mod $n$. Then, $f$ is $\vec{w}$-continuous.
\end{lemma}

\begin{proof}
According to definition \ref{def:wcont}, we need to show that for any $S\subseteq\Workers$:

    \begin{equation}\label{tsp:ineq} f(\xw) \le f(\vec{x}_{\Workers\setminus S}) + \sum_{i \in S} w_i
    \end{equation}
To see why this inequality is satisfied, %consider the walk that contains
let $T_R$ be the minimum TSP tour going through the points in $R=\Workers\setminus S$ and $T_\Workers=v_1v_2\dots v_n$ be the minimum TSP tour that goes through all the points in the
set $\Workers\supseteq R$. Now let $j_1<j_2\dots <j_r$ be the indices at which the points of the set $R$ appear in this TSP tour.  Consider two consecutive points
$v_{j_k},v_{j_{k+1}}$ in this sequence and let $P_k=\{v_{j_k+1},v_{j_k+2},\dots, v_{j_{k+1}-1}\}$ be the set of consecutive points in the tour $T_\Workers$ between
$v_{j_k}$ and $v_{j_{k+1}}$. Clearly, $\forall k: P_k\subseteq  S$ and therefore the weights of those points appear in the sum that is in the rhs of
equation \eqref{tsp:ineq}. Now consider the two paths $p_{1,k}=v_{j_k},v_{j_k+1},\dots, v_{j_{k+1}-1}$ and $p_{2,k}=v_{j_k+1},v_{j_k+1},\dots, v_{j_{k+1}}$ which are
both part of $T_\Workers$. We have that:
\[
\sum_{i\in P_k} w_i = d(v_{j_k},v_{j_k+1})+d(v_{j_{k+1}-1},v_{j_{k+1}})+2\cdot \sum_{i=j_k+1}^{j_{k+1}-2} d(v_i,v_{i+1})=l(p_{1,k})+l(p_{2,k})
\]
where $l(\cdot)$ denotes the length of a path. %Suppose that, without loss of generality, $l(p_{1,k})<l(p_{2,k})$.
We now consider the walk that goes through all the vertices in $\Workers$ and has the following two properties:
\begin{itemize}
  \item It respects the order in which the vertices in $R$ are visited by $T_R$
  \item Between any two consecutive such vertices, it follows whichever path among $p_{1,k}$ and $p_{2,k}$ has smaller length in the forward and then backwards
direction.
\end{itemize}

We know that $f(\xw)$ smaller or equal to the walk we have just defined, since the walk goes through all the given points and even repeats the points in
$R$ \footnote{Since we are working on a metric space, skipping points in the order that we visit them can only decrease the cost.}. Thus,

\begin{align*}
 f(\xw) &\le f(\vec{x}_{\Workers\setminus S}) + \sum_{k=1}^s 2\cdot \min\{(d(v_{j_k},v_{j_k+1}),d(v_{j_{k+1}-1})\}+ 2\cdot \sum_{i=j_k+1}^{j_{k+1}-2} d(v_i,v_{i+1}) \\
 &\le f(\vec{x}_{\Workers\setminus S}) + \sum_{k=1}^s\sum_{i\in P_k}w_i\\
 &\le f(\vec{x}_{\Workers\setminus S}) + \sum_{i \in  S} w_i
\end{align*}
\end{proof}

Using lemma \ref{lm:tsp} and theorem \ref{thm:wcont}, we get the following corollary:

\begin{corollary}
    Let $f : \Domain^* \to \reals$ be the function mapping a set of points in a metric space $\mathcal{X}$ to their minimum TSP tour. Then, there exists a
  verification scheme that uses at most $O(\frac{1}{\eps}\log(\frac{1}{\delta}))$ verifications per correction.
\end{corollary}

\begin{proof}
This is a straightforward application of lemma \ref{lm:tsp} and theorem \ref{thm:wcont} since $\sum_{i\in \Workers} w_i$ contains each of the edges in the optimal TSP tour $T_\Workers$ exactly twice. Thus,
\[
\sum_{i\in \Workers} w_i=2f(\vec{x}_\Workers)
\]
\end{proof}
