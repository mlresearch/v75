\section{Applications of Theorem \ref{thm:wcont}}

% !TeX root = main.tex
\subsection{Optimal Travelling Salesman Tour} \label{ssec:tspIOpt}

In this section we examine the \textit{metric travelling salesman problem} where we are given $n$ points (each provided by one record in $\Workers$) in a metric space
$\mathcal{X}$ and we wish to find the length of the minimum cycle going through each point in the set $\Truth\subseteq \Workers$ of correct answers. As usual we let
$\vec{x}_\Workers$ be the input vector with record values whose coordinates are points in the metric space $\mathcal{X}$. Our goal is to find a certification
scheme for this metric travelling salesman problem. That is, the algorithm should either output a sufficiently accurate value (according to \ref{eq:verH0}) for the
minimum weight cycle going through the points in $\vec{x}_\Truth$  or find a invalid record \footnote{Note that throughout this paper we don't consider the
computational complexity of the problems, since we are more interested in the number of verifications needed. Besides that in the case of Euclidean TSP we could
use the $(1 + \eps)$-approximation algorithm that we know in order to get similar results and avoid $\NP$-completeness.}. The following lemma combined with Theorem \ref{thm:wcont} give us
the desired result.

\begin{lemma}\label{lm:tsp}
    Let $f : \Domain^* \to \reals$ be the function mapping a set of points in a metric space $\mathcal{X}$ to their minimum TSP tour and let $v_1v_2\dots v_n$ be the
  minimum TSP tour. Also, let $\vec{w} \in \reals_+^n=(w_1,\dots,w_n)$, where
  $w_i=d(v_{i-1},v_i)+d(v_i,v_{i+1})$ and the second indices are mod $n$. Then, $f$ is $\vec{w}$-continuous.
\end{lemma}

\begin{proof}
According to definition \ref{def:wcont}, we need to show that for any $S\subseteq\Workers$:

    \begin{equation}\label{tsp:ineq} f(\xw) \le f(\vec{x}_{\Workers\setminus S}) + \sum_{i \in S} w_i
    \end{equation}
To see why this inequality is satisfied, %consider the walk that contains
let $T_R$ be the minimum TSP tour going through the points in $R=\Workers\setminus S$ and $T_\Workers=v_1v_2\dots v_n$ be the minimum TSP tour that goes through all the points in the
set $\Workers\supseteq R$. Now let $j_1<j_2\dots <j_r$ be the indices at which the points of the set $R$ appear in this TSP tour.  Consider two consecutive points
$v_{j_k},v_{j_{k+1}}$ in this sequence and let $P_k=\{v_{j_k+1},v_{j_k+2},\dots, v_{j_{k+1}-1}\}$ be the set of consecutive points in the tour $T_\Workers$ between
$v_{j_k}$ and $v_{j_{k+1}}$. Clearly, $\forall k: P_k\subseteq  S$ and therefore the weights of those points appear in the sum that is in the rhs of
equation \eqref{tsp:ineq}. Now consider the two paths $p_{1,k}=v_{j_k},v_{j_k+1},\dots, v_{j_{k+1}-1}$ and $p_{2,k}=v_{j_k+1},v_{j_k+1},\dots, v_{j_{k+1}}$ which are
both part of $T_\Workers$. We have that:
\[
\sum_{i\in P_k} w_i = d(v_{j_k},v_{j_k+1})+d(v_{j_{k+1}-1},v_{j_{k+1}})+2\cdot \sum_{i=j_k+1}^{j_{k+1}-2} d(v_i,v_{i+1})=l(p_{1,k})+l(p_{2,k})
\]
where $l(\cdot)$ denotes the length of a path. %Suppose that, without loss of generality, $l(p_{1,k})<l(p_{2,k})$.
We now consider the walk that goes through all the vertices in $\Workers$ and has the following two properties:
\begin{itemize}
  \item It respects the order in which the vertices in $R$ are visited by $T_R$
  \item Between any two consecutive such vertices, it follows whichever path among $p_{1,k}$ and $p_{2,k}$ has smaller length in the forward and then backwards
direction.
\end{itemize}

We know that $f(\xw)$ smaller or equal to the walk we have just defined, since the walk goes through all the given points and even repeats the points in
$R$ \footnote{Since we are working on a metric space, skipping points in the order that we visit them can only decrease the cost.}. Thus,

\begin{align*}
 f(\xw) &\le f(\vec{x}_{\Workers\setminus S}) + \sum_{k=1}^s 2\cdot \min\{(d(v_{j_k},v_{j_k+1}),d(v_{j_{k+1}-1})\}+ 2\cdot \sum_{i=j_k+1}^{j_{k+1}-2} d(v_i,v_{i+1}) \\
 &\le f(\vec{x}_{\Workers\setminus S}) + \sum_{k=1}^s\sum_{i\in P_k}w_i\\
 &\le f(\vec{x}_{\Workers\setminus S}) + \sum_{i \in  S} w_i
\end{align*}
\end{proof}

Using lemma \ref{lm:tsp} and theorem \ref{thm:wcont}, we get the following corollary:

\begin{corollary}
    Let $f : \Domain^* \to \reals$ be the function mapping a set of points in a metric space $\mathcal{X}$ to their minimum TSP tour. Then, there exists a
  verification scheme that uses at most $O(\frac{1}{\eps}\log(\frac{1}{\delta}))$ verifications per correction.
\end{corollary}

\begin{proof}
This is a straightforward application of lemma \ref{lm:tsp} and theorem \ref{thm:wcont} since $\sum_{i\in \Workers} w_i$ contains each of the edges in the optimal TSP tour $T_\Workers$ exactly twice. Thus,
\[
\sum_{i\in \Workers} w_i=2f(\vec{x}_\Workers)
\]
\end{proof}


% !TeX root = main.tex
\subsection{Steiner tree} \label{ssec:steinerIOpt}

In the classic Steiner tree problem, the input is a positively weighted graph $G=(V,E,w)$ and the set of vertices $V$ is partitioned into two disjoint sets $T$ and
$U$ such that $V=T\cup U$. Usually $T$ is called the set of \emph{terminal} nodes and $U$ the set of \emph{Steiner} nodes. The goal is to compute a connected
subgraph of $G$ that has the smallest possible weight and has a set of vertices $T\subseteq V^\prime \subseteq V$ that includes all \emph{terminal} nodes and any number of steiner nodes.

  Here, we are going to examine the Steiner tree problem in the following setting: We are given a fixed graph $G=(V,E)$ on $\vert V\vert$ vertices and we also have
$\vert \Workers\vert$ values from the set of records $\Workers$. Each record is a node from the set $V$ claiming that this node is in the set $T\subseteq V$ of
terminal nodes that need to be connected by the tree. However, the records might be invalid and the algorithm is allowed to do verifications on those records.
Let $\vec{x}_\Workers$ be the input vector whose coordinates are vertices claimed to be in the set $T$ of terminal nodes. Similarly, let $\vec{x}_A$ be a vector
containing only a subset $A\subseteq \Workers$ of those vertices. Our goal is again to be able to either output a sufficiently accurate answer for the cost of the
optimal Steiner tree of find an invalid record.

As in the previous section we are going to use theorem \ref{thm:wcont} to achieve this. The conditions of theorem \ref{thm:wcont} are satisfied in this case due to
the following lemma:

\begin{lemma}\label{lm:steiner}
    Let $G=V,E$ be a graph and $f_G : V^* \to \reals$ be the function mapping a set of vertices $T\subseteq V$ to the minimum cost of a steiner tree connecting the
  vertices in $T$.  Then, there exists a vector $\vec{w}\in \reals_+^n$ such that $f$ is $\vec{w}$-continuous and also
  $\sum_{i\in \Workers}w_i = O(f_G(\vec{x}_\Workers))$.
\end{lemma}

\begin{proof}
We need to show that there exists a vector $\vec{w}\in \reals_+^n=(w_1,\dots,w_n)$, such that for any $S\subseteq\Workers$, the following inequality holds:

 \begin{equation}\label{steiner:ineq} f(\xw) \le f(\vec{x}_{\Workers\setminus S}) + \sum_{i \in S} w_i
 \end{equation}

We start by introducing some notation. Let $t$ be a tree subgraph of $G$. We denote by $H_t$ the Eulerian graph that results when we double each edge in $t$. Also,
let $t_A$ denote the optimal Steiner tree for the set $A\subseteq V$ of terminal nodes. Thus, $\forall A: f(\vec{x}_A)=cost(t_A)$.

Now let $t_R$ be the optimal Steiner tree for some set $R=\Workers\setminus S\subseteq V$ of terminal nodes. In order to show equation \eqref{steiner:ineq}, it suffices to show that
there exists a tree $t$ and a vector $\vec{w}\in \reals_+^n$, such that $t$ is a valid Steiner tree for the set $\Workers$ of terminal nodes and its cost is:
$cost(t)\le cost(t_R)+\sum_{i\in S}w_i$.

In other words, we would like to find a weight vector $\vec{w}\in \reals_+^n$, such that starting from the Steiner tree $t_R$ and using the weight assigned to the
set $ S=\Workers\setminus R$ as budget, we are able to construct a Steiner tree the \emph{covers} the set $\Workers$. To keep the number of verifications low, we also
require this vector to be such that $\sum_{i\in \Workers}w_i =O(f_G(\vec{x}_\Workers))$.

Now fix a specific Euler tour (i.e an ordering of the nodes) $U_\Workers$ for the graph $H_{t_\Workers}$ and also fix an Euler tour  $U_R$ for the graph $H_{t_R}$.
Note that the cost of each Euler tour is exactly twice the cost of the corresponding Steiner tree (e.g $cost(U_R)=2cost(t_R)$ where $cost(\cdot)$ denotes the sum of
weights of all edges in the Euler tour or the tree).

We define each weight $w_i$ to be the length of the path from the predecessor to the successor of node $i$ in the ordering $U_\Workers$.

Our goal is to find a new Euler tour which directly corresponds to a valid Steiner tree \footnote{That is, the traversing each edge of that tree twice and in
opposite directions.} for the set $\Workers$ and is within our budget $\sum_{i\in S}w_i$.

Now let $U_\Workers=v_1v_2\dots v_n$ be the ordering in which the terminal nodes are visited in the Euler tour of $H_{t_\Workers}$ and $j_1<j_2\dots <j_r$ be the
indices at which the points of the set $R=\Workers\setminus S$ appear in this Euler tour.  Consider two consecutive points $v_{j_k},v_{j_{k+1}}$ in this sequence and let
$P_k=\{v_{j_k+1},v_{j_k+2},\dots, v_{j_{k+1}-1}\}\subseteq  S$ be the set of consecutive points in the Euler tour $U_\Workers$ between $v_{j_k}$ and
$v_{j_{k+1}}$. Note that the sets $P_k$ are mutually disjoint and therefore: $\sum_{k=1}^r \sum_{i\in P_k}w_i \le \sum_{i\in  S}w_i$. Also,
$\sum_{i\in P_k}w_i$ is enough budget to add the set of nodes $P_k$ in the ordering $U_R$ between $v_{j_k}$ and $v_{j_{k+1}}$. \footnote{To be more precise here, we
need an argument similar to the two paths argument in the proof of lemma \ref{lm:tsp}.}  By repeating this for all $k\in [r]$, we get the desired Steiner tree $t$
that \emph{covers} all nodes in $\Workers$ and is such that:

\[
2\cdot cost(t_\Workers)\le 2\cdot cost(t) \le 2\cdot cost(t_R) +   \sum_{i\in  S}w_i \Rightarrow
\]


\[
 cost(t_\Workers)\le  cost(t_R) +   \sum_{i\in  S}\frac{w_i}{2} \Leftrightarrow
\]


\[
 f(\vec{x}_\Workers)\le  f(\vec{x}_{\Workers\setminus S}) +   \sum_{i\in S}w_i^\prime
\]where $w_i^\prime=\frac{w_i}{2}$.

Thus, $f$ is $\frac{\vec{w}}{2}$-continuous and also $\sum_{i\in \Workers}w_i^\prime = \frac{1}{2}\cdot 2\cdot cost(U_\Workers)=2\cdot f(\vec{x}_\Workers)$.
\end{proof}


The following corollary is a direct application of  lemma \ref{lm:steiner} and theorem \ref{thm:wcont}:

\begin{corollary}
Let $G=V,E$ be a graph and $f_G : V^* \to \reals$ be the function mapping a set of vertices $T\subseteq V$ to the minimum cost of a steiner tree connecting the
vertices in $T$. Then, there exists a verification scheme that uses at most $O(\frac{1}{\eps}\log(\frac{1}{\delta}))$ verifications per correction.
\end{corollary}


% !TeX root = main.tex
\section{Proof of Theorem~\ref{thm:reduction}}\label{app:weak}

We will first provide a simple analysis when the function $f$ is \emph{increasing} with record values and later extend to the general case.

\begin{proof}[Proof of Theorem \ref{thm:reduction} for increasing functions]
  Our weak correction scheme works by repeating the certification process enough times so that the number of times it failed is less than the number of
times it succeeded. In particular, we model this procedure as a random walk on the integers starting from point C and ending once it reaches 0. We move
to the right whenever the round of verifications (i.e an execution of the certification scheme) reveals some invalid record, and we move to the left
otherwise.

  The random walk is guaranteed to return to the origin eventually since if all invalid records are removed the certification scheme will not be able to
find any additional invalid record. The only case that the weak correction scheme fails is if it returns early without removing enough invalid records
having a value larger than $f(\xt)/(1-\eps)$. In such a case, at all points of the random walk the estimate was always larger than $f(\xt)/(1-\eps)$ which
means that the random walk was biased with probability at least $2/3$ to the right. The probability that such a biased random walk reaches the origin is
at most $\left( \frac {1/3} {2/3} \right)^C = 2^{-C}$. Setting $C = \log(1/\delta)$ times guarantees a probability of error $\delta$.
The number of verifications performed if $k$ invalid records are found is $(C + 2 k) q(n,\eps)$, thus the total verification complexity is
$O(q(n,\eps) \log(1/\delta))$.
\end{proof}

We will now remove the assumption that the function $f$ is increasing.
We again use the same random walk that starts at $C$ and ends at 0 as before. However, instead of outputting the result of the function $f$ on the final subset of records (after all deletions), we will consider every possible intermediate subset of records during the random walk as a candidate for producing an $(1+\varepsilon)$-approximate solution. Note that, at each step $i$ of the random walk, we run a certification scheme on some set $S_i\subseteq \Workers$. We define a subset $S\subseteq\Workers$ to be ``bad'' if $\frac{f(\vec{x}_{\Truth})}{f(\vec{x}_S)} \not\in \left[ 1 - \eps, \frac{1}{1 - \eps} \right]$ and to be ``good'' otherwise.

By the definition of the certification scheme, if the set $S$ is ``bad'', then an invalid record is found with probability at least $2/3$, in which case the random walk moves to the right. Otherwise, we do not have any guarantee on how the random walk will behave.

However, if at all steps the probability of finding an invalid record is more than $3/5$, then the probability that the random walk reaches 0 is less than $(\frac {2/5} {3/5})^C = (\frac 2 3)^C < \delta/2$ for $C=O(\log(1/\delta))$. Thus given that we returned, with high probability, there must be some set $S_i$ for which the correction scheme accepts with probability more than $2/5$. Note that, this can only be true if the set $S_i$ is good since $2/5 > 1/3$.

At this point, given a list of these subsets, our goal is to find a ``good'' subset for which the certification scheme accepts with probability more than $1/3$. We know that a ``good'' subset exists for which the acceptance probability is more than $2/5$. We view the certification process for a subset $S$ as sampling from a Bernoulli random variable. We say that a set $S$ has probability $p$ if the certification process on the set $S$ does not find an invalid record with probability $p$.

Let $Test(S,\gamma)$ be a test that accepts if the probability of a set $S$ is more than $2/5$ (call such a set ``very good'') and rejects if it is less than $1/3$. Such a test fails with probability $\gamma$ requiring $O(\log(1/\gamma))$ samples.

The main idea behing this algorithm is to iteratively run $Test(S,\gamma)$ for all candidate subsets $S$ with varying error probabilities $\gamma$ to throw out the failing ones until a significant fraction of the subsets in our pool is ``good''.  When this happens, we pick a subset at random and check if it is actually ``good'' by running $Test(S,\gamma)$ with small $\gamma$. We repeat this until we actually find a good subset and output the value on the function $f$ on that subset. To ensure that this will eventually happen, we choose parameters appropriately, so that a constant fraction of the ``bad'' subsets fail while the ``good'' subsets pass the certifications with high enough probability.


Let $K$ be the number of candidate subsets $S_i$. We have that $K$ is equal to the number of invalid records found during the random walk process.

Our algorithm proceeds in rounds until there are at most $K / \log K$ sets remaining. In the $t$-th round:
\begin{itemize}
  \item The algorithm runs $Test(S_i,10^{-t})$ for every set $S_i$ and discards all sets that fail.
  \item If the number of remaining sets is did not drop by a factor of 2 the algorithm stops and returns a set $S_i$ uniformly at random from the remaining sets.
\end{itemize}
If the algorithm has not returned after $\log \log K$ steps, then it runs $Test(S_i,1/K^2)$ for every remaining set $S_i$ and returns one that passes the test.

The proposed algorithm returns a ``good'' set with probability more than $3/5 - o(1)$.
First, notice that a ``very good'' set will be discarded in the first $\log \log K$ rounds with probability at most $\sum_t 10^{-t} \le \frac {1/10} {1 - 1/10} = 1/9$.
Hence, if the algorithm did not return after $\log \log K$ rounds, the last step returns a ``good'' set with high probability.

Now, suppose the algorithm returns at some round $t$. Let $K_{t-1}$ be the total remaining sets before round $t$. The probability that the number $B_t$ of ``bad'' sets remaining after round $t$ to be more than $K_{t-1}/5$ is at most:
$$\Prob\left(B_t > K_{t-1}/5\right) \le \exp(-B_{t-1}/10) \le \exp(-K_{t-1}/50) \le \exp(-K/(50 \log K))$$
This is an exponentially small probability and by a union bound over all $\log \log K$ rounds it is still negligible.

Thus, assuming that $B_t < K_{t-1}/5$ and $K_t > K_{t-1}/2$, a set chosen uniformly at random is ``bad'' with probability $B_t/K_t \le 2/5$.

Therefore, a good set is chosen with probability at least $3/5 - o(1)$ and thus by repeating $O(\log(1/\delta))$ times and choosing the median of the values $f(\vec{x}_S)$, we have that with probability $1 - \delta/2$, $\frac{f(\vec{x}_{\Truth})}{f(\vec{x}_S)} \in \left[ 1 - \eps, \frac{1}{1 - \eps} \right]$.


The total number times the certification scheme is called is $O(\log(1/\delta) ) \sum_t O(2^{-t} K \log 10^{t}) = O( K \log(1/\delta) )$.

Thus, the verification complexity of the weak correction scheme is equal to $O(q(n,\eps) \log(1/\delta))$ and the Theorem follows.


\section{Missing Proof of Section \ref{sec:scorr}} \label{sec:app:scorr}

\begin{proof}[Proof of Lemma \ref{lem:sumC}]
  We let the random variable $M$ to be the total number of verifications until we found $k$ valid records and let $\mathcal{M}$ be the set of samples that we observed. Also we define 
$Z = \abs{\mathcal{M} \cap \mathcal{T}}/M = k/M$. We claim that
\[ \Prob\left(\frac{\sum_{i \in \Truth} x_i}{\sum_{i \in \Workers} x_i} \in \left[ 1 - \eps, \frac{1}{1 - \eps} \right] \cdot Z \right) \ge 1 - \delta \]

\noindent given that $\abs{\mathcal{M} \cap \mathcal{T}} \ge k$.

  Let $q = \frac{\sum_{i \in \Truth} x_i}{\sum_{i \in \Workers} x_i}$. For tha sake of contradiction let $Z > \frac{1}{1 - \eps} \frac{\sum_{i \in \Truth} x_i}{\sum_{i \in \Workers} x_i}$ then 
$M < (1 - \eps) k / q$. Hence the expected number of valid records if we draw $M$ samples according to the described distribution is at most $(1 - \eps) k$. But know using simple Chernoff 
bounds and the fact that $k \ge \frac{1}{\eps^2}\log (2/\delta)$ we get that with probability at most $\delta / 2$ the number of valid records found is at least $k$. 

  Similarly we can show that if $Z < (1 - \eps) \frac{\sum_{i \in \Truth} x_i}{\sum_{i \in \Workers} x_i}$ then with probability at most $\delta / 2$ the number of valid records found is at most $k$. Hence we have
\begin{align*}
  \Prob(\abs{\mathcal{M} \cap \mathcal{T}} = k) & = \Prob\left(\abs{\mathcal{M} \cap \mathcal{T}} = k \mid q \in \left[ 1 - \eps, \frac{1}{1 - \eps} \right] \cdot Z \right) \Prob\left( q \in \left[ 1 - \eps, \frac{1}{1 - \eps} \right] \cdot Z \right) \\
               & + \Prob\left(\abs{\mathcal{M} \cap \mathcal{T}} = k \mid q < (1 - \eps) Z \right) \Prob\left( q < (1 - \eps) Z \right) \\
               & + \Prob\left(\abs{\mathcal{M} \cap \mathcal{T}} = k \mid q > \frac{1}{1 - \eps} Z \right) \Prob\left( q > \frac{1}{1 - \eps} Z \right)
\end{align*}
\noindent but from the definition of the strong correction scheme $\Prob(\abs{\mathcal{M} \cap \mathcal{T}} = k) = 1$ and as we proved 
\[ \Prob\left(\abs{\mathcal{M} \cap \mathcal{T}} = k \mid q < (1 - \eps) Z \right) \le \delta / 2 ~~ \text{ and } \]
\[ \Prob\left(\abs{\mathcal{M} \cap \mathcal{T}} = k \mid q > \frac{1}{1 - \eps} Z \right) \le \delta / 2 \]
\noindent therefore
\begin{align*}
  \Prob(\abs{\mathcal{M} \cap \mathcal{T}} = k) = 1 & \le \Prob\left(\abs{\mathcal{M} \cap \mathcal{T}} = k \mid q \in \left[ 1 - \eps, \frac{1}{1 - \eps} \right] \cdot Z \right) \Prob\left( q \in \left[ 1 - \eps, \frac{1}{1 - \eps} \right] \cdot Z \right) + \delta \\
\end{align*}
\noindent which implies
\[ \Prob\left( q \in \left[ 1 - \eps, \frac{1}{1 - \eps} \right] \cdot Z \right) \ge 1 - \delta. \]
  This finally implies that our estimator is in the correct range
\[ \Prob\left( Z \cdot \sum_{i \in \Workers} x_i \in \left[ 1 - \eps, \frac{1}{1 - \eps} \right] \cdot \sum_{i \in \Truth} x_i \right) \ge 1 - \delta. \]

  To see that $\Theta\left(\frac{1}{\eps^2}\log (1/\delta)\right)$ are also necessary let $x_1 = \cdots = x_n = 1 / n$ and let 
$\abs{\Truth} = \abs{\Workers} q$ where $q = 1/2$. This instance is identical with estimating the
bias of a Bernoulli random variable with error at most $\eps$ and since all the $x_i$'s are equal we can assume without loss 
of generality that at each step we take a uniform sample from $\Workers$. But it is well known that for
estimating a Bernoulli random variable within $\eps$ with probability of failure at most $\delta$ we need at least 
$\Theta\left(\frac{1}{\eps^2}\log (1/\delta)\right)$ total samples. Half of those samples are expected to be correct samples
and hence the verification complexity for any strong correction scheme is also at least 
$\Theta\left(\frac{1}{\eps^2}\log (1/\delta)\right)$.
\end{proof}


\begin{proof}[Proof of Lemma \ref{lem:maxOfSumC}]
  We consider a partition $\mathcal{J}$ of $\Workers$ into $n/c^2$ sets of size $c^2$ each with $x_i = 1$ for all $i \in \Workers$. Let $S$ be a certification scheme that verifies less than $n / 4 c^2$ records. Then there 
exists a set $A \in \mathcal{J}$ such that
\[ \Prob( S \text{ verifies some } j \in A ) < 1/4. \]
\noindent We prove this by contradiction. Let $\Prob( C \text{ verifies some } j \in A ) \ge 1/4$ for all $A \in \mathcal{J}$. Then 
$\Exp[\text{verification by } C] \ge \sum_{A \in \mathcal{J}} \Prob( C \text{ verifies some } j \in A ) \ge n / 4 c^2$ and hence we have a contradiction on the assumption that $S$ verifies less than $n / 4 c^2$ records. Let $\hat{s}$
be the output estimator of $S$ then we have that
\begin{align*}
  \Prob\left(\hat{s} \in \left[\frac{1}{c}, c\right] \cdot f(\xt)\right) & = \Prob\left(\hat{s} \in \left[\frac{1}{c}, c\right] \cdot f(\xt) \mid  S \text{ verifies some } j \in A\right) \Prob\left( S \text{ verifies some } j \in A\right) \\
                                                                         & + \Prob\left(\hat{s} \in \left[\frac{1}{c}, c\right] \cdot f(\xt) \mid  S \text{ does not verify } A\right) \Prob\left( S \text{ does not verify } A\right) \\
                                                                         & < 1/4 + \Prob\left(\hat{s} \in \left[\frac{1}{c}, c\right] \cdot f(\xt) \mid  S \text{ does not verify } A\right)
\end{align*}
\noindent Now if we fix $Q \subseteq \nats$, we observe that the quantity $\Prob\left(\hat{s} \in Q \mid  S \text{ does not verify } A\right)$ does not depend on $\Truth \cap A$ since we are conditioning on the event that $S$
does not verify any record in $A$. Now let $j_B$ be an arbitrary record from the set $B \in \mathcal{J}$. We consider the following two possibilities for the set $\Truth$.
\[ \Truth_0 = \bigcup_{B \in \mathcal{J}, B \neq A} \{j_B\} \]
\[ \Truth_1 = \Truth_0 \cup A \]
\noindent We observe now that if $\Truth = \Truth_0$ then $f(\xt) = 1$ and if $\Truth = \Truth_1$ then $f(\xt) = c^2$. Now since $\hat{s}$ does not depend on $\Truth \cap A$ given that $S \text{ does not verify } A$ we have that we 
can change $\Truth$ between $\Truth_0$ and $\Truth_1$ without changing the quantity $\Prob\left(\hat{s} \in Q \mid  S \text{ does not verify } A\right)$. Now 
\begin{itemize}
  \item[-] if $\Prob\left(\hat{s} \in [1, c] \mid  S \text{ does not verify } A\right) < 1/2$ then we set $\Truth = \Truth_1$ and 
  \item[-] if $\Prob\left(\hat{s} \in [c(c - 1), c^2] \mid  S \text{ does not verify } A\right) < 1/2$ then we set $\Truth = \Truth_1$. 
\end{itemize}

\noindent Observe that one of the two cases has to be true. In any of these we get that 
\[ \Prob\left(\hat{s} \in \left[\frac{1}{c}, c\right] \cdot f(\xt) \mid  S \text{ does not verify } A\right) < 1/2. \]

\noindent Hence we get that  
\[ \Prob\left(\hat{s} \in \left[\frac{1}{c}, c\right] \cdot f(\xt)\right) < 1/4 + \Prob\left(\hat{s} \in \left[\frac{1}{c}, c\right] \cdot f(\xt) \mid  S \text{ does not verify } A\right) < 3/4 \]
\noindent and therefore $S$ has to verify at least $n / 4 c^2$ records.
\end{proof}

\section{Proof of Theorem \ref{thm:wcont}} \label{sec:app:proofOfLipschitz}

 \begin{proof}
      We set $p_i = \frac{2w_i}{3 f(\xw) \eps}$ and we show that %if we choose to verify worker $i$ with probability $p_i$ then we 
    those values satisfy the LP \eqref{eq:neccLP1}. Thus, if we choose to verify record $i$ with probability $\min\{2p_i, 1\}$, we 
    get a valid $(\varepsilon,\delta)$-certification scheme.

    For any subset $S \subseteq \Workers$ by the  $\vec{w}$-Lipschitz property we get that
    \[ |f(\xw) - f(\vec{x}_{\Workers\setminus S})| \le \sum_{i \in S} w_i \Leftrightarrow \left| \frac{2}{3 \eps} - \frac{2f(\vec{x}_{\Workers\setminus S})}{3 \eps f(\xw)} \right| \le \sum_{i \in S} \frac{2w_i}{3 f(\xw) \eps}. \]

    \noindent Now if $\frac{f(\xw)}{f(\vec{x}_{\Workers\setminus S})} > \frac{1}{1 - \eps}$, we have
    $ \left| \frac{2}{3 \eps} - \frac{2f(\vec{x}_{\Workers\setminus S})}{3 \eps f(\xw)} \right| > \frac 2 3. $

    \noindent Also if $\frac{f(\xw)}{f(\vec{x}_{\Workers\setminus S})} < 1 - \eps$, we have
    $ \left| \frac{2}{3 \eps} - \frac{2f(\vec{x}_{\Workers\setminus S})}{3 \eps f(\xw)} \right| >
    \frac{2}{3 \eps (1-\eps)} - \frac{2}{3 \eps}  \ge \frac 2 3. $
    
    \noindent Therefore when $\frac{f(\xw)}{f(\vec{x}_{\Workers\setminus S})} \notin \left[1 - \eps, \frac{1}{1 - \eps}\right]$, we have
    $\sum_{i \in S} p_i = \sum_{i \in S} \frac{2w_i}{3 f(\xw) \eps} \ge \left| \frac{2}{3 \eps} - \frac{2f(\vec{x}_{\Workers\setminus S})}{3 \eps f(\xw)} \right| \ge \frac{2}{3}$.

    \noindent This means that LP \eqref{eq:neccLP1} is satisfied. % with $\delta = 1/2$.
    Now we can apply Theorem \ref{thm:optInstV} and we conclude that the certification scheme that verifies each record independently 
    with probability $\min\{2p_i, 1\}$, where $2p_i = \frac{4w_i}{3 f(\xw) \eps}$, verifies at most 
    $\frac{4\sum_{i \in \Workers} w_i}{3 f(\xw) \eps}$ records and has probability of success at least $2/3$. In order to get 
    probability of success $\delta$ we instead verify each record $i$ with probability $2 p_i \log(1/\delta)$ and the theorem follows.
  \end{proof}

\section{Complete Statement and applications of Theorem \ref{thm:sCorrection}} \label{sec:app:applicationsStrongCorrection}

  More precisely we are given an input $\xw = (x_1, x_2 , \cdots , x_n)$ of length $n$, where every $x_i$ belongs in some set 
$\Domain$. In this section, we will fix $\Domain = [\mathcal{D}]^d$ for some $\mathcal{D} = n^{O(1)}$ to be the discretized 
$d$-dimensional Euclidean space. Our goal is to compute the value of a symmetric function 
$f : \Domain^n \rightarrow \mathbb{R}_+$ with input $\vec x \in \Domain^n$. We assume that all $x_i$ are distinct and define 
$\Support \subseteq \Domain$ as the set $\Support = \{x_i : i \in \Workers\}$. Since we consider symmetric functions $f$, it 
is convenient to extend the definition of $f$ to sets $f(\Support) = f(x)$.

  The \emph{conditional sampling model} allows such queries of small description complexity to be performed. In particular, 
the algorithm is given access to an oracle $\Cond(C)$ that takes as input a function $C: \Domain \rightarrow \{0,1\}$ and 
returns a tuple $(i, x_i)$ with $C(x_i) = 1$ with $i$ chosen uniformly at random from the subset 
$\{ j \in [n] \mid C(x_j) = 1 \}$. If no such tuple exists the oracle returns $\bot$.

  The main result of this section is a reduction from any algorithm that uses conditional sampling to a strong correction
scheme.

\begin{theorem}
    An algorithm that uses $k$ conditional samples to compute a function $f$ can produce a strong correction scheme with 
  verification cost $k$.
\end{theorem}

\begin{proof}
    We will show how we can implement one conditional sample using only one verification. We take all the values of the
  records $x_1, \dots, x_n$ and we randomly shuffle them to get $x_{\pi_1}, \dots, x_{\pi_n}$. Then we take one by one 
  the records $x_{\pi_i}$ with this new order and we check if $C(x_{\pi_i}) = 1$. If yes then we verify $x_{\pi_i}$ and
  if it is valid we return it as the result of the conditional sampling oracle. If $\pi_i$ is invalid then we just ignore
  this records without any cost and we proceed with the next record. If we finish the records and we found no valid record 
  $x_{\pi_j}$ such that $C(x_{\pi_j}) = 1$, then we return $\bot$. It is easy to see that this procedure produces at every 
  step a verified conditional sample. Since the conditional sampling algorithm has only this access to the data we get that
  any guarantees of the conditional sampling immediately transfer to this corresponding strong correction scheme.
\end{proof}

  The above result gives a general framework for designing strong correction schemes for several computational and learning 
problems. We give some of these examples below that are based on the work of \cite{GouleakisTZ2017}. For other distributional 
learning tasks, one can use the conditional sampling algorithms of \cite{CanonneRS14} to get efficient strong correction 
schemes. Some applications of Theorem \ref{thm:sCorrection} can be found in Appendix \ref{sec:app:applicationsStrongCorrection}.
\paragraph{$k$-means Clustering}
  Let $\Domain$ be a metric space with distance metric $d : \Domain \times \Domain \rightarrow \reals$, i.e. $d(x, y)$ 
represents the distance between $x$ and $y$. Given a set of \textit{centers} $Ct$ we define the distance of a point $x$ from 
$Ct$ to be $d(x, Ct) = \min_{c \in Ct} d(x, c)$. Now given a set of $n$ input points $\Support \subseteq \Domain$ and a set of 
centers $Ct \subseteq \Omega$ we define the cost of $Ct$ for $\Support$ to be 
$d(\Support, Ct) = \sum_{x \in \Support} d(x, Ct)$. The $k$-means problem is the problem of minimizing the 
\textit{squared cost} $d^2(\Support, Ct) = \sum_{x \in \Support} d^2(x, Ct)$ over the choice of centers $Ct$ subject to the 
constraint $|Ct| = k$. We assume that the diameter of the metric space is $\Delta = \max_{x, y \in \Support} d(x, y)$. In this 
setting we assume that the records contain the points in the $d$-dimensional metric space.

\begin{corollary} \label{cor:kmeans}
    Let $x_1, x_2, \dots, x_n$ be the points in the $d$-dimensional metric space $\Domain$ stored in the records $\Workers$
  and $f(\xw)$ be the optimal $k$-means clustering of the points $\xw$. There exists a strong correction scheme with 
  $\tilde{O}(k^2 \log n \log ( k / \delta ))$ verifications that guarantees a constant approximation of the value optimal 
  clustering, with probability of failure at most $\delta$.
\end{corollary}

  The proof of this corollary is based on the Theorem \ref{thm:sCorrection} and the Theorem 2 from \cite{GouleakisTZ2017}.

\paragraph{Euclidean Minimum Spanning Tree}
  Given a set of points $\xw$ in $\reals^d$, the minimum spanning tree problem in Euclidean space ask to compute the a spanning
tree $T$ on the points minimizing the sum of weights of the edges. The weight of an edge between two points is equal to their 
Euclidean distance. We will focus on a simpler variant of the problem which is to compute just the weight of the best possible 
spanning tree, i.e. estimate the quantity
$\min_{\text{tree } T} \sum_{(x,x') \in T} \|x - x'\|_2$.

\begin{corollary} \label{cor:mst}
    Let $x_1, x_2, \dots, x_n$ be the points in $\reals^d$ stored in the records $\Workers$ and 
  $f(\xw) = \min_{\text{tree } T} \sum_{(x,x') \in T} \|x - x'\|_2$. There exists a strong correction scheme with 
  $\tilde O(d^3 \log^4 n / \eps^7 )\cdot \log(1/\delta)$ verifications that guarantees an $(1 + \eps)$-approximation of
  the weight of the minimum spanning tree, with probability of failure at most $\delta$.
\end{corollary}

  The proof of this corollary is based on the Theorem \ref{thm:sCorrection} and the Theorem 3 from \cite{GouleakisTZ2017}.

\textbf{Remark.} Observe that the value of the MST gives a 2-approximation of the metric TSP and the metric Steiner Tree 
problems. Hence Corollary \ref{cor:mst} implies efficient strong correction schemes that achieve constant approximation for 
those problems as well.
