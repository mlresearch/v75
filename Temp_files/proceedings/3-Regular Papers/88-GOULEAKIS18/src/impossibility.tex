% !TeX root = main.tex
\section{A Function with no Efficient Verification Scheme}
\label{sec:impos}

  In this section we show a counter example of a function $f$ that does not admit any efficient verification scheme, even for the certification procedure. The
takehome message of this section is that there are some functions $f$ that are very fragile in noise that migth exists in the input and these functions are not
appropriate for crowdsourcing environments. For this counter example $f$ will be the shortest path between two fixed points $a, b \in \Domain$.

  We assume that the workers report $\xw = (x_1, \dots, x_n)$ where $x_i \in \Domain$ and that there is an arbitrary function $\rho : \Domain^2 \to \reals$ such that
$\rho(u, v)$ describes the ``length'' of the edge $(u, v)$. We want to compute the minimum length path from $a$ to $b$ that can pass through any point in $\xw$. Notice
that $\rho(\cdot, \cdot)$ is not a metric function. Otherwise the answer would be that the shortest path is the direct edge from $a$ to $b$ with length $\rho(a, b)$.

  So formally we have that
  \[ f(\xw) = \min_{\text{$P$ path from $a$ to $b$ through $\xw$}} \sum_{(v_i, v_j) \in P} \rho(v_i, v_j) \]

  \noindent Let $x_0 = a$ and $x_{n + 1} = b$. Since $\rho$ is an arbitrary function it is obvious that we can construct the following instance
  \begin{equation*}
    \rho(x_i, x_j) = \left\{
    \begin{split}
      0 ~~~~~~&  j = i + 1 \\
      +\infty~~~ & \text{otherwise}
    \end{split}
    \right.
  \end{equation*}

  Therefore in this instance the shortest path is $P = \{(x_0, x_1), (x_1, x_2), \dots, (x_n, x_{n + 1})\}$ and it has cost $0$ therefore $f(\xw) = 0$. But even if one
of the reports $x_i$ are wrong then the cost of the shortest path becomes $+\infty$. Therefore for any arbitrary worker $i$, any verification scheme that does not
verify $i$ can be loose an infinitely bad approximation factor. This means that every worker has to be verified with probability $1$. This means that there is no
other verification scheme that the trivial which verifies every worker and has verification cost $n$.
  %We therefore conclude that this version of the shortest path problem is not appropriate fot crowdsourcing environments.