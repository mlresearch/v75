% !TeX root = main.tex
\section{Model and Preliminaries}
\label{sec:prelim}

%\begin{itemize}
%\item Let ${\cal N} = \{1,2,...,n\}$
%\item $n$ inputs given as a vector $\vec x = (x_1, x_2, ..., x_n)$
%\item Each input $x_i \in {\cal I}$
%\item Unknown set $\cal T \subseteq {\cal N}$ ($t = |\cal T|$) of real data, with query access to determine membership
%\item Let $\vec{x}_{\cal T} = (x_j)_{j\in {\cal T}}$ be a vector consisting only of the coordinates of $\vec{x}$ that are in ${\cal T}$.
%\item We are given an objective $f: {\cal I}^* \rightarrow \mathbb{R}_+$ and we want to compute $f( \vec x_{\cal T} )$
%\item Every input $x_j$ with $j \in {\cal T}$ is called \correct ~and the rest are called \wrong.
%\end{itemize}

\paragraph{Notation} For $m \in \nats$ we denote the set $\{1,\cdots,m\}$ by $[m]$. %We use $\tilde O(N)$ to denote $O(N \log^{O(1)} N)$ algorithms.
Let $\Workers = [n]$ be the set of all records of the data set and $\Truth\subseteq \Workers$ be the subset of records
that are valid. The set $\Truth$ is unknown to the algorithm. Suppose we are given an input $\xw = (x_1, x_2 , \cdots , x_n)$ 
of length $n$, where every $x_i$ belongs to some set $\Domain$. Let $\vec{x}_{\cal T} = (x_j)_{j\in {\cal T}}$ be a vector 
consisting only of the coordinates of $\vec{x}$ that are in $\Truth$. Our general goal is to approximate the value of a 
symmetric function $f : \Domain^* \rightarrow \mathbb{R}_+$ on input $\vec x_\Truth \in \Domain^*$. Finally, Every input $x_j$
with $j \in \Truth$
is called valid and the rest $\Workers \setminus \Truth$ are called invalid. We consider two different tasks; \emph{certification}
and \emph{correction}.

\smallskip
\noindent In the {\bf certification task}, we count the total number of verifications of records needed to test between
the following two hypotheses:

\begin{enumerate}[label=(H\arabic*)]
    \item \[ f(\xw) \in \left[1 - \eps, \frac{1}{1 - \eps} \right] \cdot f(\xt) \] \label{eq:verH0}
    \item there exists a record $i$ such that $i \notin {\cal T}$. \label{eq:verH1}
\end{enumerate}

\noindent We allow a small probability $\delta$ that the algorithm fails to find a witness, i.e.
\[ \Prob\left( f(\xw) \not\in \left[1-\eps, \frac 1 {1-\eps} \right] \cdot f(\xt) \wedge \text{ no invalid record found} \right) \le \delta \]

\smallskip
\noindent In the {\bf correction task},  %we are not only interested in verifying that the answer is correct but also
the goal is to always compute an approximation to the correct answer, even when the certification task fails.  
We consider two models for correction the {\bf weak correction} and the {\bf strong correction}.

\smallskip
\noindent In the {\bf weak correction} model after catching an invalid record we are allowed to restart the task and
therefore we do not count the number of verifications that we already used before catching the invalid record. So if we have 
the guarantee that a weak correction scheme uses $v(n , \eps)$ verifications and during the execution of the scheme we find
$k$ invalid records, then the total number of verifications used is at most $(k + 1) \cdot v(n, \eps)$.

\smallskip
\noindent In the {\bf strong correction} model instead of restarting every time we find an invalid record, we just ignore 
the data of this record and we also don't count them in the number of verifications. So if we have the
guarantee that a strong correction scheme uses $v(n , \eps)$ verifications and during the execution of the scheme we find $k$
invalid records, then the total number of samples used is at most $k + v(n, \eps)$.
