%!TEX root = ../submission.tex
\section{Preliminaries and Notation}

\subsection{Notation}
We will use the following notations and conventions throughout: For a distribution $\cD$ on $\R^d \times \R$ and function $f:\R^d \to \R$, we define $\err_\cD(f) = \E_{(x,y) \sim \cD}[ (f(x) - y)^2]$. For a vector $\ell \in \R^d$, we abuse notation and write $\err_\cD(\ell)$ for $\E_{(x,y) \sim \cD}[ (\iprod{\ell,x} - y)^2]$.  For a real-valued random variable $X$, and integer $k \geq 0$, we let $\|X\|_k = \E[X^k]^{1/k}$. 


\subsection{Distribution Families}
Our algorithmic results for a wide class of distributions that include Gaussian distributions and others such as log-concave and other product distributions. We next define the properties we need for the marginal distribution on examples to satisfy. 
\begin{definition}[Certifiable hypercontractivity]\label{def:hyperconc1}
For a function $C:[k] \to \R_+$, we say a distribution $D$ on $\R^d$ is $k$-certifiably $C$-hypercontractive if for every $r \leq k/2$, there's a degree $k$ sum of squares proof of the following inequality in variable $v$:
\[
\E_D \iprod{x,v}^{2 r} \leq \Paren{C(r) \E_{D} \iprod{x,v}^{2}}^{r}.
\] 
\end{definition}

Many natural distribution families satisfy certifiable hypercontractivity with reasonably growing functions $C$. For instance, Gaussian distributions, uniform distribution on Boolean hypercube satisfy the definitions with $C(r) = c r$ for a fixed constant $c$. %similarly, all log-concave distributions satisfy the definitions with $C(r) = c r^2$.
\Pnote{certifiable hypercontractivity of log-concave distributions is not known. It's true only under the KLS cooling conjecture so far.} More generally, all distributions that are affine transformations of isotropic distributions satisfying the Poincar\'{e} inequality \citep{DBLP:journals/corr/abs-1711-07465},  are also certifiably hypercontractive. In particular, this includes all strongly log-concave distributions.  Certifiable hypercontractivity also satisfies natural closure properties under simple operations such as affine transformations, taking bounded weight mixtures and taking products. We refer the reader to \citet{DBLP:journals/corr/abs-1711-11581} for a more detailed overview where certifiable hypercontractivity is referred to as certifiable subgaussianity. 



% \subsection{Generalization Bounds for Regression}

