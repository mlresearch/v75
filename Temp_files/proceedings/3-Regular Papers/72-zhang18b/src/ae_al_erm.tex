\documentclass[final,12pt]{colt2018} % Anonymized submission
%\usepackage{fullpage}
%\usepackage{amsthm}
\usepackage{amsmath}
\usepackage{amssymb}
\usepackage{algorithm}
\usepackage{algorithmic}
\usepackage[usenames,dvipsnames]{pstricks}
%\usepackage{epsfig}
\usepackage{pst-grad} % For gradients
\usepackage{pst-plot} % For axes
\usepackage{commath}
\usepackage{booktabs}

%\newtheorem{theorem}{Theorem}
%\newtheorem{lemma}{Lemma}
%\newtheorem{corollary}{Corollary}
\newtheorem{assumption}{Assumption}
\newtheorem{claim}{Claim}
%\newtheorem{definition}{Definition}

\renewcommand\algorithmicrequire{\textbf{input:}}
\renewcommand\algorithmicensure{\textbf{output:}}
\def\calF{\mathcal{F}}
\def\calG{\mathcal{G}}
\def\R{\mathbb{R}}
\def\E{\mathbb{E}}
\def\P{\mathbb{P}}
\def\calX{\mathcal{X}}
\def\calY{\mathcal{Y}}
\def\calH{\mathcal{H}}
\def\calO{\mathcal{O}}
\def\calA{\mathcal{A}}
\def\calW{\mathcal{W}}
\def\calZ{\mathcal{Z}}
\DeclareMathOperator{\argmin}{argmin}
\DeclareMathOperator{\B}{B}
\DeclareMathOperator{\HT}{P}
\DeclareMathOperator{\Alt}{Alt}
\DeclareMathOperator{\sign}{sign}
\DeclareMathOperator{\err}{err}
\DeclareMathOperator{\polylog}{polylog}
\DeclareMathOperator{\poly}{poly}
\def\Explore{\textsc{Explore}}
\def\Refine{\textsc{Refine}}

\def\cf{C_3}
\def\ct{C_5}
\def\ce{C_4}
\def\cs{C_6}
\def\cse{C_7}
\def\cfo{C_8}
\def\cn{C_9}

\title[Efficient active learning of sparse halfspaces]{Efficient active learning of sparse halfspaces}
\usepackage{times}

\coltauthor{\Name{Chicheng Zhang} \Email{chicheng.zhang@microsoft.com}
\\
\addr Microsoft Research \\
641 6th Avenue \\
New York, NY, 10011 \\
USA
}

\begin{document}

\maketitle

\begin{abstract}
We study the problem of efficient PAC active learning of homogeneous linear classifiers (halfspaces) in $\R^d$, where the goal is to learn
a halfspace with low error using as few label queries as possible.
Under the extra assumption that there is a $t$-sparse halfspace that
performs well on the data ($t \ll d$),
we would like our active learning algorithm to be {\em attribute efficient}, i.e. to have label requirements sublinear in $d$.
In this paper, we provide a computationally efficient algorithm that achieves this goal.
Under certain distributional assumptions on the data, our algorithm achieves a label complexity of $O(t \cdot \polylog(d, \frac 1 \epsilon))$.
In contrast, existing algorithms in this setting are either computationally inefficient, or subject to label requirements
polynomial in $d$ or $\frac 1 \epsilon$.
\end{abstract}
%In this setting, existing algorithms either achieve

%that the unlabeled distribution is log-concave, and the distribution satisfies

%While the non-sparse version of this problem is well studied, only a few works have made progress in
%this setting.
Online learning algorithms are a key tool in web search and content optimization, adaptively learning what users want to see. In a typical application, each time a user arrives, the algorithm chooses among various content presentation options (\eg news articles to display), the chosen content is presented to the user, and an outcome (\eg a click) is observed. Such algorithms must balance \emph{exploration} (making potentially suboptimal decisions now for the sake of acquiring information that will improve decisions in the future) and \emph{exploitation} (using information collected in the past to make better decisions now). Exploration could degrade the experience of a current user, but improves user experience in the long run. This exploration-exploitation tradeoff is commonly studied in the online learning framework of \emph{multi-armed bandits}~\citep{Bubeck-survey12}.

Concerns have been raised about whether exploration in such scenarios could be unfair, in the sense that some individuals or groups may experience too much of the downside of exploration without sufficient upside \citep{bird2016exploring}. We formally study these concerns in the \emph{linear contextual bandits} model~\citep{Langford-www10,chu2011contextual}, a standard variant of multi-armed bandits appropriate for content personalization scenarios.  We focus on \emph{externalities} arising due to exploration, that is, undesirable side effects that the presence of one party may impose on another.


We first examine the effects of exploration at a group level.  We introduce the notion of a \emph{group externality} in an online learning system, quantifying how much the presence of one population (which we dub the majority) impacts the rewards of another (the minority). We show that this impact can be negative, and that, in a particular precise sense, no algorithm can avoid it. This cannot be explained by the absence of suitably good policies since our adoption of the linear contextual bandits framework implies the existence of a feasible policy that is simultaneously optimal for everyone. Instead, the problem is inherent to the process of exploration. We come to a surprising conclusion that more data can sometimes lead to worse outcomes for the users of an explore-exploit-based system. \looseness=-1

We next turn to the effect of exploration at an individual level. We interpret exploration as a potential externality imposed on the current user by future users of the system. Indeed, it is only for the sake of the future users that the algorithm would forego the action that currently looks optimal. To avoid this externality, one may use the greedy algorithm that always chooses the action that appears optimal according to current estimates of the problem parameters. While this greedy algorithm performs poorly in the worst case,
it tends to work well in many applications and experiments.\footnote{Both positive and negative findings are folklore. One way to precisely state the negative result is that the greedy algorithm incurs constant per-round regret with constant probability; while results of this form have likely been known for decades,
\citet[Corollary A.2(b)]{competingBandits-itcs16}
proved this for a wide variety of scenarios. Very recently, the good empirical performance has been confirmed by state-of-art experiments in \citet{practicalCB-arxiv18}.}

In a new line of work, \citet{bastani2017exploiting} and \citet{kannan2018smoothed}
analyzed conditions under which inherent diversity in the data makes explicit exploration unnecessary.
\citet{kannan2018smoothed} proved that the greedy algorithm achieves a regret rate of
$\tilde{O}(\sqrt{T})$ in expectation over small perturbations of the context vectors (which ensure sufficient data diversity). This is the best rate that can be achieved in the worst case (\ie for all problem instances, without data diversity assumptions), but it leaves open the possibilities that (i) another algorithm may perform much better than the greedy algorithm on some problem instances, or (ii) the greedy algorithm may perform much better than worst case under the diversity conditions. We expand on this line of work. We prove that under the same diversity conditions, the greedy algorithm almost matches the best possible Bayesian regret rate of \emph{any} algorithm \emph{on the same problem instance}. This could be as low as $\polylog(T)$ for some instances, and, as we prove, at most $\tilde{O}(T^{1/3})$ whenever the diversity conditions hold.


Returning to group-level effects, we show that under the same diversity conditions, the negative group externalities imposed by the majority essentially vanish if one runs the greedy algorithm. Together, our results illustrate a sharp contrast between the high individual and group externalities that exist in the worst case, and the ability to remove all externalities if the data is sufficiently diverse.   \looseness=-1

\xhdr{Additional motivation.} Whether and when explicit exploration is necessary is an important concern in the study of the exploration-exploitation tradeoff. Fairness considerations aside, explicit exploration is expensive. It is wasteful and risky in the short term, it adds a layer of complexity to algorithm design \citep{Langford-nips07,monster-icml14}, and its adoption at scale tends to require substantial systems support and buy-in from management \citep{MWT-WhitePaper-2016,DS-arxiv}. A system based on the greedy algorithm would typically be cheaper to design and deploy.

Further, explicit exploration can run into incentive issues in applications such as recommender systems. Essentially, when it is up to the users which products or experiences to choose and the algorithm can only issue recommendations and ratings, an explore-exploit algorithm needs to provide incentives to explore for the sake of the future users \citep{Kremer-JPE14,Frazier-ec14,Che-13,ICexploration-ec15,Bimpikis-exploration-ms17}. Such incentive guarantees tend to come at the cost of decreased performance, and rely on assumptions about human behavior. The greedy algorithm avoids this problem as it is inherently consistent with the users' incentives.



\xhdr{Additional related work.}
Our research draws inspiration from the growing body of work on fairness in machine learning~\cite[e.g.,][]{dwork2012fairness,hardt2016equality,kleinberg2017inherent,chouldechova2017fair}.  Several other authors have studied fairness in the context of the contextual bandits framework.  Our work differs from the line of research on meritocratic fairness in online learning \citep{kearns2017meritocratic,liu2017calibrated,joseph2016fairness}, which considers the allocation of limited resources such as bank loans and requires that nobody should be passed over in favor of a less qualified applicant. We study a fundamentally different scenario in which there are no allocation constraints and we would like to serve each user the best content possible.  Our work also differs from that of \citet{celis2017fair}, who studied an alternative notion of fairness in the context of news recommendations. According to this notion, all users should have approximately the same probability of seeing a particular type of content (e.g., Republican-leaning articles), regardless of their individual preferences, in order to mitigate the possibility of discriminatory personalization.

The data diversity conditions in \citet{kannan2018smoothed} and this paper are inspired by the smoothed analysis framework of \citet{SmoothedAnalysis-jacm04}, who proved that the expected running time of the simplex algorithm is polynomial for perturbations of any initial problem instance (whereas the worst-case running time has long been known to be exponential). Such disparity implies that very bad problem instances are brittle. 
We find a similar disparity for the greedy algorithm in our setting.



\xhdr{Our results on group externalities.}  A typical goal in online learning is to minimize \emph{regret}, the (expected) difference between the cumulative reward that would have been obtained had the optimal policy been followed at every round and the cumulative reward obtained by the algorithm.  We define a corresponding notion of \emph{minority regret}, the portion of the regret experienced by the minority.  Since online learning algorithms update their behavior based on the history of their observations, minority regret is influenced by the entire population on which an algorithm is run.  If the minority regret is much higher when a particular algorithm is run on the full population than it is when the same algorithm is run on the minority alone, we can view the majority as imposing a negative externality on the minority; the minority population would achieve a higher cumulative reward if the majority were not present. Asking whether this can ever happen
amounts to asking whether access to more data points can ever lead an explore-exploit algorithm to make inferior decisions. One might think that more data should always lead to better decisions and therefore better outcomes for the users.
Surprisingly, we show that this is not the case, even with a standard algorithm.

Consider LinUCB~\citep{Langford-www10,chu2011contextual,abbasi2011improved}, a standard algorithm for linear contextual bandits that is based on the principle of ``optimism under uncertainty.''  We provide a specific problem instance on which, after observing $T$ users, LinUCB would have a minority regret of $\Omega(\sqrt T)$ if run on the full population, but only constant minority regret if run on the minority alone. While stylized, this example is motivated by the problem of providing driving directions to different populations of users, about which fairness concerns have been raised~\citep{bird2016exploring}. Further, the situation is reversed on a slight variation of this example: LinUCB obtains constant minority regret when run on the full population and $\Omega(\sqrt T)$ on the minority alone.  That is, group externalities can be large and positive in some cases, and large and negative in others.

Although these regret rates are specific to LinUCB, we show that this phenomenon is, in some sense, unavoidable. Consider the minority regret of LinUCB when run on the full population and the minority regret that LinUCB would incur if run on the minority alone. We know that one may be much smaller or larger than the other. We ask whether any algorithm could  achieve the minimum of the two on every problem instance. Using a variation of the same problem instance, we prove that this is impossible; in fact, no algorithm could simultaneously approximate both up to any $o(\sqrt{T})$ factor. In other words, an externality-free algorithm would sometimes ``leave money on the table."


In terms of techniques, we rely on the special structure of our example, which can be viewed as an instance of the sleeping bandits problem~\citep{SleepingBandits-ml10}. This simplifies the behavior and analysis of LinUCB, allowing us to obtain the $O(1)$ upper bounds.  The lower bounds are obtained using KL-divergence techniques to show that the two variants of our example are essentially indistinguishable, and an algorithm that performs well on one must obtain $\Omega(\sqrt{T})$ regret on the other. \looseness=-1


\xhdr{Our results on the greedy algorithm.} We consider a version of linear contextual bandits in which the latent weight vector $\theta$ is drawn from a known prior. In each round, an algorithm is presented several actions to choose from, each represented by a \emph{context vector}. The expected reward of an action is a linear product of $\theta$ and the corresponding context vector. The tuple of context vectors is drawn independently from a fixed distribution. In the spirit of smoothed analysis, we assume that this distribution has a small amount of jitter. Formally, the tuple of context vectors is drawn from some fixed distribution, and then a small \emph{perturbation}---small-variance Gaussian noise---is added independently to each coordinate of each context vector. This ensures arriving contexts are diverse. We are interested in Bayesian regret, i.e., regret in expectation over the Bayesian prior. Following the literature, we are primarily interested in the dependence on the time horizon $T$. \looseness=-1

We focus on a batched version of the greedy algorithm, in which new data arrives to the algorithm's optimization routine in small batches, rather than every round. This is well-motivated from a practical perspective---in high-volume applications data usually arrives to the ``learner" only after a substantial delay \citep{MWT-WhitePaper-2016,DS-arxiv}---and is essential for our analysis.

Our main result is that the greedy algorithm matches the Bayesian regret of any algorithm up to polylogarithmic factors, for each problem instance, fixing the Bayesian prior and the context distribution. We also prove that LinUCB achieves regret $\tilde{O}(T^{1/3})$ for each realization of $\theta$. This implies a worst-case Bayesian regret of $\tilde{O}(T^{1/3})$ for the greedy algorithm under the perturbation assumption. \looseness=-1

Our results hold for both natural versions of the batched greedy algorithm, Bayesian and frequentist, henceforth called \BayesGreedy and \FreqGreedy. In \BayesGreedy, the chosen action maximizes expected reward according to the Bayesian posterior. \FreqGreedy estimates $\theta$ using ordinary least squares regression and chooses the best action according to this estimate. The results for \FreqGreedy come with additive polylogarithmic factors, but are stronger in that the algorithm does not need to know the prior. Further, the $\tilde{O}(T^{1/3})$ regret bound for \FreqGreedy is approximately prior-independent, in the sense that it applies even to very concentrated priors such as independent Gaussians with standard deviation on the order of $T^{-2/3}$.

The key insight in our analysis of \BayesGreedy is that any (perturbed) data can be used to simulate any other data, with some discount factor. The analysis of \FreqGreedy requires an additional layer of complexity. We consider a hypothetical algorithm that receives the same data as \FreqGreedy, but chooses actions based on the Bayesian-greedy selection rule. We analyze this hypothetical algorithm using the same technique as \BayesGreedy, and then upper bound the difference in Bayesian regret between the hypothetical algorithm and \FreqGreedy.

Our analyses extend to group externalities and (Bayesian) minority regret. In particular, we circumvent the impossibility result mentioned above. We prove that both \BayesGreedy and \FreqGreedy match the Bayesian minority regret of any algorithm run on either the full population or the minority alone, up to polylogarithmic factors

\xhdr{Detailed comparison with prior work.} We substantially improve over the $\tilde{O}(\sqrt{T})$ worst-case regret bound from \citet{kannan2018smoothed}, at the cost of some additional assumptions. First, we consider Bayesian regret, whereas their regret bound is for each realization of $\theta$.%
\footnote{Equivalently, they allow point priors, whereas our priors must have variance $T^{-O(1)}$.} Second, they allow the context vectors to be chosen by an adversary before the perturbation is applied. Third, they extend their analysis to a somewhat more general model, in which there is a separate latent weight vector for every action (which amounts to a more restrictive model of perturbations). However, this extension relies on the greedy algorithm being initialized with a substantial amount of data. The results of \citet{kannan2018smoothed} do not appear to have implications on group externalities.

\citet{bastani2017exploiting} show that the greedy algorithm achieves logarithmic regret in an alternative linear contextual bandits setting that is incomparable to ours in several important ways.
They consider two-action instances where the actions share a common context vector in each round, but are parameterized by different latent vectors. They ensure data diversity via a strong assumption on the context distribution. This assumption does not follow from our perturbation conditions; among other things, it implies that each action is the best action in a constant fraction of rounds. Further, they assume a version of Tsybakov's \emph{margin condition}, which is known to substantially reduce regret rates in bandit problems \citep[\eg see][]{Zeevi-colt10}.


\section{Related work}

%al
\paragraph{Attribute efficient active learning of halfspaces.}
There is a rich body of theoretical literature on active learning of general concept classes
in the PAC setting~\citep{D11, H14}. For the problem of active halfspace learning, sharp distribution-dependent label complexity results are known,
in terms of e.g. the splitting index~\citep{D05}, or the disagreement coefficient~\citep{H07}.
Direct applications of these results (without taking advantage of sparsity assumptions)
yield algorithms with label complexities at least $\Omega(d \ln \frac 1 \epsilon)$~\citep{KMT93}.
To make these algorithms attribute efficient, a natural modification is to consider concept class
$\calH_t$, the set of $t$-sparse linear classifiers.
It is well known that $\calH_t$ has VC dimension
$O(t \ln d)$. In conjunction with existing results in the active learning
literature, this observation immediately yields attribute efficient active
learning algorithms. For example, when the unlabeled distribution is isotropic log-concave,
an application of~\cite{ZC14}'s algorithm with $\calH_t$ yields a label complexity
of $O(t \ln d \ln \frac 1 \epsilon)$ in the $t$-sparse realizable setting, and gives
$O(t \ln d \cdot (\ln \frac 1 \epsilon+\frac{\nu^2}{\epsilon^2}))$ and
$O(\frac{t\ln d}{(1-2\eta)^2} \ln \frac 1 \epsilon)$
label complexities in the $t$-sparse $\nu$-adversarial noise and $t$-sparse $\eta$-bounded noise settings.\footnote{To see this, note that the $\phi(\cdot,\cdot)$ function
defined in~\cite{ZC14} with respect to $\calH_t$ can be bounded as: $\phi(r,\xi) \leq O(r \ln \frac{r}{\xi})$, as $\calH_t$ is a subset of $\calH$. Theorem 4 of \cite{ZC14} now applies.}
However, these algorithms require solving
empirical 0-1 loss minimization subject to sparsity constraints, which is computationally intractable in general~\citep{N95}.
The only attribute and computationally efficient PAC active learning algorithms we are aware of are in~\cite{ABHZ16}.  Specifically, under the $t$-sparse $\Omega(\epsilon)$-adversarial noise setting, \cite{ABHZ16} gives an efficient algorithm with label complexity $\tilde{O}(\frac{t}{\epsilon^2}\polylog(d,\frac 1 \epsilon))$. Under the $t$-sparse $\eta$-bounded noise setting, ~\cite{ABHZ16} gives an efficient algorithm with label complexity $\tilde{O}((\frac t \epsilon)^{O(\frac 1 {(1-2\eta)^2})})$.


%It is implicit in~\cite{D05} that the Splitting algorithm therein with input hypothesis
%class $\calH_t$ results in an inefficient algorithm
%with $\tilde O(t (\ln d + \ln \frac 1 \epsilon))$ label complexity in the $t$-sparse realizable setting.
%~\cite{H14} proposes an general splitting index based algorithm that


%efficient al
%For example, ~\cite{DKM05,BBZ07} achieves a label complexity of $O(d \ln \frac 1 \epsilon)$ in the realizable setting;
%~\cite{ABL17} achieves a label complexity of $\tilde{O}(d\ln \frac 1 \epsilon)$ in the $\nu$-adversarial noise setting, where $\nu = \Omega(\epsilon)$;
%~\cite{ABHU15} achieves a label complexity of $\tilde{O}(d\ln \frac 1 \epsilon)$ in the $\eta$-bounded noise setting, where $\eta$ is at most a constant.

%It combines the technique of margin-based active learning, polynomial regression~\cite{KKMS08} and $\ell_1/\ell_\infty$ Rademacher complexity bounds, and gives efficient algorithms with label complexity polynomial in $t$ and $\ln d$.

%On the other hand, many computationally efficient active halfspace learning algorithms have been proposed in the literature, with different degrees of noise tolerance~\cite{DKM05,BBZ07, ABL17, ABHU15, ABHZ16, YZ17}.
%However, all of the algorithms above have a label complexity at least $\Omega(d)$ and is thus not attribute efficient.

The notion of attribute efficient learning algorithms is initially studied in the pioneering works of~\cite{L87,B90}.
\cite{L87} considers attribute efficient online
learning of linear classifiers, with an application to learning disjunctions that depends on only $t$ attributes.
The algorithm incurs a mistake bound of $O(t \ln d)$, which can be of substantially lower order than $O(d)$ when $t$ is small.
\cite{B90} considers an online learning model where the feature space is infinite dimensional,
and each instance shown has a bounded number of nonzero attributes.
%In addition, there is an underlying concept that relies on
%only $t$ of the attributes.
It gives efficient algorithms that learn $k$-CNFs and disjunctions
with finite mistake bounds in this setting.
\cite{S00, KS06, STT12} study attribute efficient learning of decision lists and analyzes the
tradeoff between running time and mistake bound.
\cite{LS07} shows that, if the unlabeled distribution is unconcentrated over $\{-1,1\}^d$, then there
is an algorithm that learns $t$-sparse linear classifiers with a sample complexity of $\poly(t, \ln d, 2^{O(\epsilon^{-2})})$. \cite{F07} gives algorithms for attribute efficient learning parity and DNFs
in the membership query model.


%\cite{LS07} proposes an efficient algorithm that learns halfspaces
%over $t$ variables using $\poly(t, \ln d))$ samples to achieve a constant error.
%Winnow is attribute efficient, in the sense that
%when the instances and the classifiers are all $\ell_\infty$ bounded by a constant, and there is
%a $t$-sparse linear separator that separates the examples by a margin, then the mistake bound is
%$O(t \ln d)$, which only has a logarithmic dependence on the dimension.


\paragraph{One-bit compressed sensing.} The line of work on one-bit compressed sensing~\citep{BB08} is closely related to our problem setup. In this setting,
there is a unknown $t$-sparse vector $u \in \R^d$, and the algorithm can make measurements of $u$ using vectors $x \in \R^d$ and receives (possibly noisy) values of $\sign(u \cdot x)$.
Note that different from standard compressed sensing~\citep{CT06,D06}, the measurement results of one-bit compressed sensing are {\em quantized} versions of $(u \cdot x)$'s (i.e. they lie in $\{-1,+1\}$ as opposed to $\R$).
The goal is to approximately recover $u$ up to scaling with a few (ideally, $O(t \ln d)$) measurements.
 In the non-adaptive setting, the measurement vector
$x$'s are chosen at the beginning, while in the adaptive setting, the measurement vector $x$'s can be chosen sequentially,
based on past observations.
The problem of adaptive one-bit compressed sensing is therefore equivalent to attribute efficient
active halfspace learning in the membership query model~\citep{A88}.
We remark that active learning in the PAC model is more challenging than in the membership model, in that the learner has to query the labels of the unlabeled examples it has drawn.

%The crucial difference between the membership query model and the PAC model
%is that, in the PAC model, the , and is thus more challenging from the
%viewpoint of algorithm design.
%in the data
%stream literature
%This algorithm, in conjunction with
%a set of $O(\frac k \epsilon)$ nonadaptive Gaussian measurements, results in an nonadaptive 1-bit compressed sensing algorithm with
%$O(k (\ln d + \frac 1 \epsilon))$ measurements
%This, in conjunction with efficient noise tolerant learning algorithm
%working in the support of $w^*$, gives an algorithm that uses $O(k (\ln d + \frac{1}{(1-2\eta)^2 \epsilon})$ measurements.

~\cite{JLBB13} gives an algorithm that has robust recovery guarantees, however it is based on computationally-intractable $\ell_0$ minimization. Inspired by the count sketch data structure~\citep{CCF02}, ~\cite{HB11} proposes an efficient procedure that recovers the support of $u$ using $O(t \ln d)$ queries, and has strong noise tolerance properties. In conjunction with efficient full-dimensional active halfspace learning algorithms~\citep{DKM05,ABL17,CHK17,YZ17}, this procedure
 yields efficient algorithms that have label complexities of $O(t (\ln d + \ln \frac 1 \epsilon ))$
(resp. $O(t (\ln d + \ln \frac 1 \epsilon))$, $O(\frac{t}{(1-2\eta)^2} (\ln d + \ln \frac 1 \epsilon ))$) in the $t$-sparse realizable setting (resp. $t$-sparse $\Omega(\epsilon)$-adversarial noise setting, $t$-sparse $\eta$-bounded noise setting).
~\cite{GNJN13, ABK17} gives upper and lower bounds for {\em universal} one-bit compressed sensing, that is, the same set of measurements can be used to approximately recover {\em any} underlying $t$-sparse signal. In this setting,~\cite{ABK17} shows that, perhaps surprisingly, the number of measurements necessary and sufficient for support recovery is $\tilde{\Theta}(t^2 \ln d)$, as opposed to $\Theta(t \ln d)$ in the non-universal setting.
 ~\cite{PV13a} proposes a linear programming based algorithm that works in the $t$-sparse realizable setting, and has a measurement complexity of $\tilde{O}(\frac{t}{\epsilon^5})$,
based on a new tool named random hyperplane tessellations. ~\cite{L16} gives a support recovery algorithm that tolerates bounded noise, %double check this result - why don't they have a 1-2eta dependency?
using $\alpha$-stable random projections.
\cite{PV13b} proposes a convex programming based algorithm that works in the $t$-sparse $\Omega(\epsilon^2)$-adversarial noise model,
and has a measurement complexity of $\tilde{O}(\frac{t}{\epsilon^{12}})$.


Works on one-bit compressed sensing under the symmetric noise condition has been studied in the literature~\citep{PV13b, ZYJ14, CB15, ZG15}. In this model, it is assumed that there is a known function $g$, such that for all $x$, $\E[y|x] = g(u \cdot x)$. This assumption captures some realistic scenarios, but is nevertheless strong: it requires any two examples that have the same projection on $u$ to have the same conditional label distribution. In contrast, the $t$-sparse adversarial noise and the $t$-sparse bounded noise conditions allow heterogeneous noise levels, even among examples that have the same projection on $u$.
In this setting, the state of the art result of \cite{ZYJ14} gives an nonadaptive algorithm with $O(\frac{t \ln d}{ \epsilon^2})$. It also proposes an adaptive algorithm that works in same setting, achieving a label complexity bound of $O(\min(\frac{t \ln d}{\epsilon^2}, \frac{t\sqrt{d} \ln d}{\epsilon}))$, which is sometimes lower than that of the nonadaptive algorithm.
The special case of Gaussian noise before quantization has been studied extensively, i.e. given $x$, the label $y$ is generated by the formula $y = \sign(u \cdot x + n)$, where $n$ is a Gaussian random variable. \cite{GNR10} shows that when $u$ has a large dynamic range (the absolute value of the ratio between $u$'s largest and smallest nonzero elements in magnitude), adaptive approaches require fewer measurements to identify the support of $u$ than nonadaptive approaches.

%The most popular noise model is
%the generalized linear model, i.e. . The analysis crucially relies on the parameter $\lambda:=\E_{x \sim N(0,1)}[x g(x)]$; for example, $\lambda = \sqrt{\frac 2 \pi}(1 - 2\eta)$ when every label is flipped with probability $\eta$.
%\cite{PV13b} proposes a convex programming based algorithm that has a measurement complexity of $O(\frac{t \ln d}{\lambda^2 \epsilon^4})$. \cite{ZYJ14} gives a one-pass algorithm that has a measurement complexity of $O(\frac{t \ln d}{\lambda^2 \epsilon^2})$; see also \cite{ZG15} for refinements of logarithmic factors. \cite{CB15, ALPV14} studies
%subgaussian measurements; the recovery error bounds has an non-diminishing term under this general setting.
%depending on the Gaussianity of the measurements.
% whereas it tends to zero if the measurement distribution is sufficiently close to Gaussian.

%In classical compressed sensing, adaptive measurements have proven useful to reduce the measurement requirement~\cite{HBCN12}, or to reduce the signal to noise ratio requirement~\cite{MN14}. Similar results lie in 1-bit compressed sensing. ; \cite{ZYJ14} gives an adaptive algorithm that works in the parametric noise model, giving a measurement complexity of $O(\min(\frac{t \ln d}{\lambda^2\epsilon^2}, \frac{s\sqrt{d} \ln d}{\lambda^2\epsilon}))$, improving over the $O(\frac{t \ln d}{\lambda^2 \epsilon^2})$ bound in the nonadaptive setting.

%Attribute efficient active learning in the membership query model can be seen as an instance of adaptive one-bit compressed sensing.


%\cite{SSSHZ14} generalizes the analysis of \cite{L87}, and provide sample complexity bounds
%of order $\frac{t \ln d}{\epsilon^2}$ for ERM based algorithms.


%It proposes the Winnow algorithm, that achieves a mistake bound of $O(t \ln d)$ when the data is
%separable by a $t$-disjunction.
%the mistake bound has only a logarithmic dependence on the dimension of the data.




%achieving a mistake bound of
%$O(\frac{\|w^*\|_1^2 X^2 \ln d}{\gamma^2})$, under the assumption that the data is $\ell_\infty$ bounded by $X$,
%and is separable by $w^*$ with margin $\gamma$. Winnow is attribute efficient, in the sense
%that if $w^*$ is $t$-sparse and


% A straightforward application of~\cite{D05} yields an computationally inefficient algorithm that achieve a label
%complexity of $O(t (\ln d + \ln \frac 1 \epsilon))$ in this setting. On the other hand, if computational efficiency is
%required, the best known label complexity bounds are much worse, for instance, $O(\frac t {\epsilon^2})$~\cite{ABHZ16} and
%$O(d \ln \frac 1 \epsilon)$~\cite{DKM05, BBZ07}. As we will see in this note, we develop a computationally efficient algorithm that has a near-optimal label complexity
%bound of $O( t (\ln d)^3 \ln \frac 1 \epsilon )$, advancing the state of the art.
We provide a detailed comparison between our work and the results most closely related to ours in Tables~\ref{tab:comp-r}, \ref{tab:comp-an}, and \ref{tab:comp-bn}.



\begin{table}[t]
\centering
\begin{tabular}{llll}
\toprule
Algorithm & Model & Label complexity & Efficient? \\
\midrule
\begin{tabular}{@{}c@{}}\cite{HB11}\\ with \cite{DKM05} \end{tabular} & MQ & $\tilde{O}(t (\ln d + \ln \frac 1 \epsilon))$ & Yes \\
\cite{D05} & PAC & $\tilde{O}(t (\ln d + \ln \frac 1 \epsilon))$ & No \\
\cite{ABHZ16} & PAC & $\tilde{O}(\frac{t}{\epsilon^2} \polylog(d,\frac 1 \epsilon) )$ & Yes \\
Our work & PAC & $\tilde{O}(t \polylog(d,\frac 1 \epsilon) )$ & Yes \\
\bottomrule
\end{tabular}
\caption{A comparison of algorithms for active learning of halfspaces in the $t$-sparse realizable setting (Definition~\ref{def:r}); all the PAC algorithms above work under isotropic log-concave distributions.}
\label{tab:comp-r}
\end{table}


\begin{table}[t]
\centering
\begin{tabular}{lllll}
\toprule
Algorithm & Model & Noise tolerance  & Label complexity & Efficient? \\
\midrule
\begin{tabular}{@{}c@{}}\cite{HB11}\\ with \cite{ABL17} \end{tabular}& MQ & $\nu = \Omega(\epsilon)$ & $\tilde{O}(t (\ln d + \ln \frac 1 \epsilon))$ & Yes \\
\cite{ZC14} & PAC & $\nu = \Omega(\epsilon)$ & $\tilde{O}(t \ln d \ln \frac 1 \epsilon))$ & No \\
\cite{PV13b} & PAC  & $\nu = \Omega(\epsilon^2)$ & $\tilde{O}(\frac{t\ln d}{\epsilon^{12}})$ & Yes \\
\cite{ABHZ16} & PAC & $\nu = \Omega(\epsilon)$ & $\tilde{O}(\frac{t}{\epsilon^2} \polylog(d,\frac 1 \epsilon) )$ & Yes \\
Our work & PAC & $\nu = \Omega(\epsilon)$ & $\tilde{O}(t \polylog(d,\frac 1 \epsilon) )$ & Yes \\
\bottomrule
\end{tabular}
\caption{A comparison of algorithms for active learning of halfspaces in the $t$-sparse $\nu$-adversarial noise setting (Definition~\ref{def:an});  all the PAC algorithms above work under isotropic log-concave distributions.}
\label{tab:comp-an}
\end{table}

\begin{table}[t]
\centering
\begin{tabular}{lllll}
\toprule
Algorithm & Model & Noise tolerance & Label complexity & Efficient? \\
\midrule
\begin{tabular}{@{}c@{}}\cite{HB11}\\ with \cite{CHK17} \end{tabular} & MQ & $\eta \in [0,\frac 1 2)$ & $\tilde{O}(\frac{t}{(1-2\eta)^2} (\ln d + \ln \frac 1 \epsilon))$ & Yes \\
\cite{ZC14} & PAC & $\eta \in [0,\frac 1 2)$ & $\tilde{O}(\frac{t}{(1-2\eta)^2} \ln d \ln \frac 1 \epsilon))$ & No \\
\cite{ABHZ16} & PAC & $\eta \in [0,\frac 1 2)$ & $\tilde{O}((\frac{t}{\epsilon})^{O(\frac{1}{(1-2\eta)^2})} )$ & Yes \\
Our work & PAC & $\eta \in [0, \Omega(1))$ & $\tilde{O}(t \polylog(d,\frac 1 \epsilon) )$ & Yes \\
\bottomrule
\end{tabular}
\caption{A comparison of algorithms for active learning of halfspaces in the $t$-sparse $\eta$-bounded noise setting (Definition~\ref{def:bn}); all the PAC algorithms above work under isotropic log-concave distributions.}
\label{tab:comp-bn}
\end{table}

\label{s:definitions}
In this section, we present basic definitions and notational conventions. Throughout, we let
$\lv v \rv_{2}$ denotes the Euclidean norm, for a vector $v \in \mathbb{R}^{d}$. 
\subsubsection{Assumptions on $f$}\label{ss:assumptions}
We make the following assumptions regarding the function $f$.
\begin{enumerate}[label=(A{\arabic*})]
\item The function $f$ is twice continuously-differentiable on $\mathbb{R}^d$ and has Lipschitz continuous gradients; that is, there exists a positive constant $L >0$ such that for all $x,y \in \mathbb{R}^{d}$ we have
\begin{align*}
\lVert \nabla f(x) - \nabla f(y) \rVert_2 \le L \lVert x-y \rVert_2.
\end{align*}
\item $f$ is $m$-strongly convex, that is, there exists a positive constant $m>0$ such that for all $x,y \in \mathbb{R}^d$,
\begin{align*}
f(y) \ge f(x) + \langle \nabla f(x),y-x \rangle + \frac{m}{2} \lVert x-y \rVert_2^2.
\end{align*}
\end{enumerate}
It is fairly easy to show that under these two assumptions the Hessian of $f$ is positive definite throughout its domain, with $m I_{d\times d}  \preceq  \nabla^2 f(x) \preceq L I_{d\times d}$. We define $\kappa = L/m$ as the condition number. Throughout the paper we denote the minimum of $f(x)$ by $x^*$. Finally, we assume that we have a gradient oracle $\nabla f(\cdot)$; that is, we have access to $\nabla f(x)$ for all $x \in \mathbb{R}^d$.
\subsubsection{Coupling and Wasserstein Distance}
Denote by $\mathcal{B}(\mathbb{R}^d)$ the Borel $\sigma$-field of $\mathbb{R}^d$. Given probability measures $\mu$ and $\nu$ on $(\mathbb{R}^d,\mathcal{B}(\mathbb{R}^d))$, we define a \emph{transference plan} $\zeta$ between $\mu$ and $\nu$ as a probability measure on $(\mathbb{R}^d \times \mathbb{R}^d,\mathcal{B}(\mathbb{R}^d\times \mathbb{R}^d))$ such that for all sets $A \in \mathcal{B}(\mathbb{R}^d)$, $\zeta(A\times \mathbb{R}^d) = \mu(A)$ and $\zeta( \mathbb{R}^d \times A) = \nu(A)$. We denote $\Gamma(\mu,\nu)$ as the set of all transference plans. A pair of random variables $(X,Y)$ is called a coupling if there exists a $\zeta \in \Gamma(\mu,\nu)$ such that $(X,Y)$ are distributed according to $\zeta$. (With some abuse of notation, we will also refer to $\zeta$ as the coupling.)

We define the Wasserstein distance of order two between a pair of probability measures as follows:
\begin{align*}
W_2(\mu,\nu) \triangleq \left(\inf_{\zeta\in\Gamma(\mu,\nu)} \int \lv x-y\rv_2^2 d\zeta(x,y) \right)^{1/2}.
\end{align*}
Finally we denote by $\Gamma_{opt}(\mu,\nu)$ the set of transference plans that achieve the infimum in the definition of the Wasserstein distance between $\mu$ and $\nu$ \citep[for more properties of $W_2(\cdot,\cdot)$ see][]{villani}. 

\subsubsection{Underdamped Langevin Diffusion} \label{ss:underdampedlangevindiffusionnotation}
Throughout the paper we use $B_t$ to denote standard Brownian motion~\citep{brownian}. Next we set up the notation specific to the continuous and discrete processes that we study in this paper.
\begin{enumerate}
%\item Let $f: \R^d \to R$ be a $m$ strongly convex and $L$ smooth function, and let $\kappa = m/L$ be the inverse of the condition number. 
%\item Given two distributions $p$ and $p'$, we denote by $\Gamma(p,p')$ the set of all couplings between $p$ and $p'$ (a coupling is a joint distribution whose marginals match its two arguments). 
\item Consider the exact underdamped Langevin diffusion defined by the SDE \eqref{e:exactlangevindiffusion}, with an initial condition $(x_0,v_0)\sim p_0$ for some distribution $p_0$ on $\R^{2d}$. Let $p_t$ denote the distribution of $(x_t,v_t)$ and let $\Phi_t$ denote the operator that maps from $p_0$ to $p_t$:
\begin{equation}\label{d:phi}
\Phi_t  p_0 = p_t.
\end{equation}
%\begin{enumerate}
%%\item Let $\Phi_s$ denote the (stochastic) map that maps a point $(x_0,v_0)$ to $(x_t,v_t)$ under \eqref{e:exactlangevindiffusion}.
%\item We let $p_t(x,v)$ denote the distribution of $(x_t, v_t)$ with an initial condition of the form $(x_0,v_0)\sim p_0$ for some given $p_0$ must be specified for $p_t$ to be defined.
%\item  We denote by 
%\begin{equation}\label{d:phi}
%\Phi_t  p_0
%\end{equation}
%the distribution $p_t$ of $(x_t,v_t)$ given the initial condition $ (x_0,v_0) \sim p_0$.
%%\item It can be shown that the invariant distribution $p_\infty(x,v)$ for \eqref{e:exactlangevindiffusion} is the desired stationary distribution $p^*(x,v) \propto e^{-(f(x) + \frac{1}{2u}\|v\|_2^2)}$, under mild conditions on $f(\cdot)$.
%\end{enumerate}
\item One step of the discrete underdamped Langevin diffusion is defined by the SDE
\begin{align}
\label{e:discretelangevindiffusion}
d\vt_t &= -\gamma \vt_t dt -u \nabla f(\xt_0) dt + (\sqrt{2\gamma u}) dB_t \\
d \xt_t &= \vt_t dt, \nonumber
\end{align}
with an initial condition $(\xt_0,\vt_0)\sim \pt_0$.
Let $\pt_t$ and $\Phit_t$ be defined analogously to $p_t$ and $\Phi_t$ for $(x_t,v_t)$.

\textbf{Note 1:} The discrete update differs from \eqref{e:exactlangevindiffusion} by using $\xt_0$ instead of $\xt_t$ in the drift of $\vt_s$.

\textbf{Note 2:} We will only be analyzing the solutions to \eqref{e:discretelangevindiffusion} for small $t$. Think of an integral solution of \eqref{e:discretelangevindiffusion} as a single step of the discrete Langevin MCMC.
\end{enumerate}
\subsubsection{Stationary Distributions}
Throughout the paper, we denote by $p^*$ the unique distribution which satisfies $p^*(x,v) \propto \exp{-(f(x)+\frac{1}{2u}\lv v\rv_2^2)}$.
It can be shown that $p^*$ is the unique invariant distribution of \eqref{e:exactlangevindiffusion} \citep[see Proposition 6.1 in][]{pav}. Let $g(x,v) = (x,x+v)$. We let $q^*$ be the distribution of $g(x,v)$ when $(x,v) \sim p^*$.


\section{Algorithm}
We present our main algorithm, namely Algorithm~\ref{alg:ae_al} in this section. We defer the exact settings of constants $c_1, c_2, c_3$ to Appendix~\ref{sec:params}.
Our algorithm uses the margin-based active learning framework, initially proposed by~\cite{BBZ07}.
Specifically, it proceeds in epochs, where at each epoch $k$, it draws a sample $S_k$ from distribution $D_X|_{B_k}$, queries their labels, and updates its iterate $w_k$ based on $S_k$. Due to technical reasons, at the first epoch ($k=0$), the sampling region $B_0$ and the constraint set $W_0$ are different from those in subsequent epochs. Throughout the process, the algorithm maintains the invariant that at each epoch $k$, $w_k$ is a  $t$-sparse unit vector.

At each epoch $k \geq 1$, the sampling region $B_k$ is a ``small-margin'' band $\{x: |w_{k-1} \cdot x| \leq b_k \}$, with bandwidth $b_k$ descreasing exponentially in $k$.
Then it performs constrained empirical hinge loss minimization over $S_k$, getting a linear classifier $w_k'$.
The constraint set $W_k$ is the intersection between an $\ell_1$ ball and an $\ell_2$ ball, centered at $w_{k-1}$ with different radii ($\rho_k$ and $r_k$). This is similar to the approach in~\cite{PV13b} for tackling the symmetric noise setting, where a linear optimization problem with a similar shaped constraint set is proposed. The construction of $W_k$'s is inspired by version space constructions in the PAC active learning literature~\citep{CAL94,BBL09,H14}.
Throughout the algorithm, we ensure $W_k$ to satisfy the following two properties with high probability: first, $u$ lie in all the $W_k$'s; second, the $W_k$'s are shrinking in size.\footnote{We refer the reader to Lemma~\ref{lem:induct} for a formal statement.}
In addition, the hinge loss used at epoch $k$ is parameterized by $\tau_k$, which also decreases exponentially in $k$.

Observe that $w_k'$ may not be a sparse vector; therefore, we perform a hard thresholding step (applying $\HT_t$), to ensure that our learned halfspace at the end of round $k$, is $t$-sparse. Hard thresholding has been widely used in the (unquantized) compressed sensing literature~\citep[See e.g.][]{BD09,GK09}, however its utility in one-bit compressed sensing is not yet well-understood. For example, ~\cite{JLBB13} proposes an algorithm named BIHT (binary iterative hard thresholding) that has strong empirical performance, but its convergence properties are unknown.
To the best of our knowledge, our work is the first that establishes convergence guarantees for iterative hard thresholding style algorithms for one-bit compressed sensing. We then perform a $\ell_2$ normalization step to ensure that our iterate $w_k$ is an unit vector, which has a scale comparable to $u$.

Finally, we remark that Algorithm~\ref{alg:ae_al} admits a computationally efficient implementation. First, the sampling regions $B_k$'s can be shown to have probability masses at least $\Omega(\epsilon)$ in $D_X$ for all $k$ in $\{0,1,\ldots,k_0\}$, which makes rejection sampling from $D_X|_{B_k}$ take $O(\frac 1 \epsilon)$ time per example.
Second, optimization problem~\eqref{eqn:opt} is convex, and can be approximately solved by e.g. stochastic gradient descent~\citep[See e.g.][Theorem 2]{SZ13} efficiently.

\begin{algorithm}[t]
  \caption{Attribute and computationally efficient active learning of halfspaces}
\begin{algorithmic}[1]
  \REQUIRE sparsity parameter $t$, target error $\epsilon$, failure probability $\delta$.
  \ENSURE  learned halfspace $\hat{w}$.
  \STATE Initialization: $k_0 \gets \lceil \log_2 \frac {1} {C_1 \epsilon} \rceil$, where $C_1$ is defined in Equation~\eqref{eqn:angdis}.
  \FOR{$k = 0, 1, 2 \ldots,k_0$}
  \STATE $S_k \gets $ sample $n_k = c_1 t (\ln d + \ln \frac{1}{\epsilon} + \ln\frac1{\delta_k})^3$ examples from $D_X|_{B_k}$ and query their labels, where
  \[ B_k := \begin{cases} \R^d, & k = 0, \\ \{x: |w_{k-1} \cdot x| \leq b_k \}, & k \geq 1, \end{cases}\]
	$\delta_k = \frac{\delta}{(k+1)(k+2)}$ and $b_k = c_2 \cdot 2^{-k}$.

  \STATE Solve the following optimization problem:
  \begin{equation}
    w_k' \gets \argmin_{w \in W_k} \sum_{(x,y) \in S_k} \ell_{\tau_k}(w, (x, y)),
    \label{eqn:opt}
  \end{equation}
  where
	%\B_2(0,1) \cap \B_1(0,\sqrt{t})
	%\B_2(w_{k-1},r_k) \cap \B_1(w_{k-1}, \rho_k)
  \[ W_k = \begin{cases} \{ w \in \R^d: \| w \|_2 \leq 1 \text{ and } \| w \|_1 \leq \sqrt{t} \}, & k = 0, \\ \{ w \in \R^d: \| w - w_{k-1} \|_2 \leq r_k \text{ and } \| w - w_{k-1} \|_1 \leq \rho_k \}, & k \geq 1, \end{cases}
  \]
  %\begin{cases}
  %\{w: \| w \|_2 \leq 1 \text{ and } \| w \|_1 \leq \sqrt{t} \}, & k = 0, \\
  %\{w: \| w - w_{k-1} \|_2 \leq r_k \text{ and } \| w - w_{k-1} \|_1 \leq \rho_k \}, & k \geq 1. \end{cases}
  $r_k = 2^{-k-3}$, $\rho_k = \sqrt{2t} \cdot 2^{-k-3}$, and $\tau_k = c_3 \cdot 2^{-k}$.
	\label{line:hlm}
  \STATE Let $w_k \gets \frac{\HT_t(w_k')}{\| \HT_t(w_k') \|_2}$.
	\label{line:ht}
  \ENDFOR
  \RETURN $w_{k_0}$.
\end{algorithmic}
\label{alg:ae_al}
\end{algorithm}

\section{Performance guarantees}

In this section, we prove Theorem~\ref{thm:main}, the main result of this paper. %for Algorithm~\ref{alg:ae_al}.
\begin{theorem}
  There exist numerical constants $\mu_1, \mu_2 \in (0, \frac 1 2)$ such that the following holds.
  Suppose $D_X$ is isotropic log-concave, and one of the following two conditions hold:
  \begin{enumerate}
    \item $D$ satisfies the $t$-sparse $\mu_1\epsilon$-adversarial noise condition;
    \item $D$ satisfies the $t$-sparse $\mu_2$-bounded noise condition.
  \end{enumerate}
	In addition, Algorithm~\ref{alg:ae_al} is run with sparsity parameter $t$, target error $\epsilon$ and failure probability $\delta$.
  Then, with probability $1-\delta$, the output halfspace $\hat{w}$ is such that
  $\err(h_{\hat{w}}) - \err(h^*) \leq \epsilon$,
  and the total number of label queries is $O( t \cdot (\ln d + \ln \frac 1 \epsilon)^3 \cdot \ln \frac 1 \epsilon )$.
  \label{thm:main}
\end{theorem}

As the $t$-sparse realizable setting is a special case of the $t$-sparse adversarial noise setting (by setting $\nu = 0$), Theorem~\ref{thm:main} immediately implies the
following corollary:
\begin{corollary}
	Suppose $D_X$ is isotropic log-concave, and the $t$-sparse realizable condition holds for $D$.
  In addition, Algorithm~\ref{alg:ae_al} is run with sparsity parameter $t$, target error $\epsilon$ and failure probability $\delta$.
  Then, with probability $1-\delta$, the output halfspace $\hat{w}$ is such that
  $\err(h_{\hat{w}}) - \err(h^*) \leq \epsilon$,
  and the total number of label queries is $O( t \cdot (\ln d + \ln \frac 1 \epsilon)^3 \cdot \ln \frac 1 \epsilon )$.
	\label{cor:main}
\end{corollary}
%We remark that although the requirement of $w_0$ seems strong, we can get an algorithm with the same label complexity
%unconditionally.
%Indeed, consider an arbitrary $t$-sparse unit vector $w_0 \in \R^d$; we run two copies of Algorithm~\ref{alg:ae_al}, initialized with $w_0$ and $-w_0$ respectively, getting classifiers $h_+$ and $h_-$. We can then run a standard hypothesis testing procedure to identify the classifier that has error guarantee $\epsilon$.
%This technique is well known in the literature; details can be found in (e.g.~\cite{ABL17}, Appendix B).

Theorem~\ref{thm:main} and Corollary~\ref{cor:main} imply that, under the respective noise conditions defined above, Algorithm~\ref{alg:ae_al}
has a label complexity of $O( t \polylog(d,  \frac 1 \epsilon) )$. To the best of our knowledge, this
is the first efficient PAC active learning algorithm that has a label complexity linear in $t$, and polylogarithmic
in $d$ and $\frac 1 \epsilon$. Previous works either need to sacrifice computational efficiency to achieve such guarantee~\citep{D05,ZC14}, or have label complexities polynomial in $d$ or $\frac 1 \epsilon$ ~\citep{ABL17, ABHZ16}. We remark that in the membership query model~\citep{A88, BB08}, efficient algorithms with $O( t \polylog(d,  \frac 1 \epsilon) )$ label complexities are implicit in the literature (e.g. by combining \cite{HB11}'s support recovery algorithm with efficient full-dimensional active halfspace learning algorithms~\citep{DKM05,ABL17,CHK17,YZ17}). In contrast, the focus of this paper is on the more challenging PAC setting, and it is unclear how to modify a membership query algorithm to make it work in the PAC setting.
%the membership query setting is harder than in the membership query setting.
%active learning in the PAC setting;


\subsection{Proof of Theorem~\ref{thm:main}}
Recall that $\delta_k = \frac{\delta}{(k+1)(k+2)}$; note that $\sum_{l=0}^{k_0} \delta_k \leq \delta$. To prove Theorem~\ref{thm:main}, we give exact settings of constants $\mu_1, \mu_2 \in (0, \frac 1 2)$ in Appendix~\ref{sec:params},
such that under either the $t$-sparse $\mu_1\epsilon$-adversarial noise condition or the $t$-sparse
$\mu_2$-bounded noise condition, the following lemma holds:
\begin{lemma}
	For every $k \in \{ 0,1,\ldots,k_0 \}$, there is an event $E_k$ with probability $1-\sum_{l=0}^k \delta_l$, on which $u$ is in $W_{k+1}$.
	%\B_1(w_k, \rho_{k+1}) \cap \B_2(w_k, r_{k+1})
	%$w_k$ is $t$-sparse, $\theta(w_k, u) \leq 2^{-k-3} \pi$ and $\| w_k - u \|_2 \leq 2^{-k-3} \pi$.
  %is t-sparse and
  %$\| w_{k-1} - u \|_2 \leq 2^{-k}$.
  \label{lem:induct}
\end{lemma}

The proof of Lemma~\ref{lem:induct} relies on the following two supporting lemmas. The first lemma (Lemma~\ref{lem:hlm}) shows that, $w_k'$ produced in the hinge loss minimization step (line~\ref{line:hlm}) has a small angle with $u$. Specifically, the upper bound on $\theta(w_k',u)$
is halved at each iteration $k$, with the help of constrained hinge loss minimization over a fresh set of $n_k = O(t \polylog(d,\frac1\epsilon))$ labeled examples. This relies on two ideas: first, as is standard
in the margin-based active learning framework~\citep[See e.g.][]{BBZ07,BL13}, it suffices to let $w_k'$ achieve a constant error with respect to the sampling distribution at epoch $k$; second, to ensure that the setting of $n_k$ ensures that $w_k'$ indeed has a constant error under the sampling distribution, we use a novel uniform concentration bound of hinge losses of $W_k$ over $S_k$ tighter than all prior works~\citep{ABL17,ABHZ16}. Thanks to our construction of $W_k$, our concentration bound of hinge losses is of order $\tilde{O}(\sqrt{\frac{t\ln d}{n_k}})$, which can be substantially tighter than
$\tilde{O}(\sqrt{\frac{d}{n_k}})$ used in~\citet{ABL17,HKY15} and $\tilde{O}(\sqrt{\frac{(t \ln d) \cdot 2^k}{n_k}})$ used in~\citet{ABHZ16}. We refer the reader to Appendix~\ref{sec:conc} for a formal statement.

 %in turn uses the generic results of~\cite{KST09} on the Rademacher complexity of $\ell_1$ bounded linear predictors.

%$w_{k-1}$ is t-sparse, $\theta(w_{k-1}, u) \leq 2^{-k-3} \pi$ and $\| w_{k-1} - u \|_2 \leq 2^{-k-3} \pi$
\begin{lemma}
For every $k \in \{ 0, 1,\ldots,k_0 \}$, if $u$ is in $W_k$, then with probability $1-\delta_k$, $\theta(w_k', u) \leq 2^{-k-8} \pi$.
\label{lem:hlm}
\end{lemma}

The second lemma (Lemma~\ref{lem:truncate}) shows that, performing a hard thresholding operation followed by $\ell_2$ normalization on $w_k'$ (line~\ref{line:ht}) yields a $t$-sparse unit vector $w_k$ that is close to $u$ in terms of both $\ell_1$ and $\ell_2$ distances. This ensures that $W_{k+1}$, the constraint set of the optimization problem at the next epoch, contains $u$. A key fact used in the proof of the lemma is that, the hard thresholding operator $\HT_t$ is effectively a $\ell_2$-projection onto the $\ell_0$ ball $\{w \in \R^d: \| w \|_0 \leq t \}$.


\begin{lemma}
For every $k \in \{ 0,1,\ldots,k_0 \}$, if $\theta(w_k', u) \leq 2^{-k-8} \pi$, then $u$ is in $W_{k+1}$.
\label{lem:truncate}
\end{lemma}
%then $w_k$ is $t$-sparse, $\| w_k - u \|_2 \leq 2^{-k-3} \pi$ and $\theta(w_k, u) \leq 2^{-k-3} \pi$

%The analysis of the first epoch (the base case $k=0$) is slightly different from the subsequent epochs, due to the usage of a different sampling region $B_1$ and constraint set $W_1$; this is taken care of by Lemma~\ref{lem:init} in the Appendix.

We are now ready to prove Lemma~\ref{lem:induct}.

\begin{proof}[Proof of Lemma~\ref{lem:induct}]
 We prove the lemma by induction.
\paragraph{Base case.} In the case of $k = 0$, observe that as $u$ has unit $\ell_2$ norm and $u$ is $t$-sparse, by Cauchy-Schwarz, $\| u \|_1 \leq \sqrt{t} \| u\|_2 = \sqrt{t}$. Therefore, $u$ belongs to the set $W_0$ deterministically.
Lemma~\ref{lem:hlm} with $k=0$ shows that there is an event $E_0$ with probability $1-\delta_0$, conditioned on which $\theta(w_0',u) \leq 2^{-8}\pi$. By Lemma~\ref{lem:truncate}, we get that $u$ is in $W_1$.
%$w_0$ is $t$-sparse, $\theta(w_0, w^*) \leq 2^{-3} \pi$ and $\| w_0 - w^* \|_2 \leq 2^{-3} \pi$.

% $w_{k-1}$ is $t$-sparse, $\theta(w_{k-1}, w^*) \leq 2^{-k-3} \pi$ and $\| w_{k-1} - w^* \|_2 \leq 2^{-k-3} \pi$
\paragraph{Inductive case.} For $k \geq 1$, suppose the inductive hypothesis holds. That is, there is an event $E_{k-1}$ with probability $1-\sum_{l=0}^{k-1} \delta_l$, such that
on $E_{k-1}$, $u$ is in $W_k$. By Lemma~\ref{lem:hlm}, there is an event $F_k$ such that $\P(F_k | E_{k-1}) \geq 1 - \delta_k$,
conditioned on which $\theta(w_k', u) \leq 2^{-k-8} \pi$.

%, with probability $1-\frac{\delta}{k(k+1)}$.

Define event $E_k:= E_{k-1} \cap F_k$. Observe that $\P(E_k) = \P(E_{k-1})\P(F_k | E_{k-1}) \geq 1-\sum_{l=0}^{k} \delta_l$.
%On event $E_k$, we have
Now, on event $E_k$, Lemma~\ref{lem:truncate} implies that $u$ is in $W_{k+1}$.
%$w_k$ is $t$-sparse, $\theta(w_k, w^*) \leq 2^{-k-3} \pi$ and $\| w_k - w^* \|_2 \leq 2^{-k-3} \pi$ holds.
This completes the induction.
\end{proof}

Theorem~\ref{thm:main} is now a direct consequence of Lemma~\ref{lem:induct}; we give its proof below.

\begin{proof}[Proof of Theorem~\ref{thm:main}]
From Lemma~\ref{lem:induct} and the fact that the output $\hat{w}$ is $w_{k_0}$, we have that with probability $1-\sum_{l=0}^{k_0}\delta_l \geq 1-\delta$, $u$ is in $W_{k_0+1}$. By the definition of $W_k$,
\[ \| u - w_{k_0} \|_2 \leq r_{k_0+1} = 2^{-k_0 - 4}. \]
By Lemma~\ref{lem:distangle} in the Appendix and the fact that $\|u\|_2 = 1$, we know that
$\theta(w_{k_0}, u) \leq \pi \| u - w_{k_0} \|_2 \leq 2^{-k_0 - 2} \leq \frac{C_1\epsilon}{2}$.
%$ \theta(\hat{w}, u) \leq 2^{-5} C_1 \epsilon.$
By the first inequality of Equation~\eqref{eqn:angdis}, we have that
$ \P_D (h_{w_{k_0}}(x) \neq h_{u}(x) ) \leq \frac \epsilon 2. $
Therefore, by triangle inequality and the fact that the output $\hat{w}$ is $w_{k_0}$,
\[ \err(h_{\hat{w}}) - \err(h_u) \leq \frac \epsilon 2. \]
We now consider two separate cases regarding the two different noise conditions:
\begin{enumerate}
\item In the $\mu_1 \epsilon$-adversarial noise setting, we know that $\err(h_u) \leq \mu_1 \epsilon \leq \frac \epsilon 2$.
Therefore,
\[ \err(h_{\hat{w}}) - \err(h^*) \leq \err(h_{\hat{w}}) \leq \err(h_u) + \frac \epsilon 2 \leq \frac \epsilon 2 + \frac \epsilon 2 \leq \epsilon. \]
\item In the $\mu_2$-bounded noise setting, as $h_u$ and $h^*$ are identical,
it immediately follows that $\err(h_{\hat{w}}) - \err(h^*) \leq \frac \epsilon 2 \leq \epsilon$.
\end{enumerate}

We now bound the label complexity of Algorithm~\ref{alg:ae_al}. The total number of labels queried is $\sum_{k=0}^{k_0} n_k$,
where $n_k \leq c_1 \cdot t (\ln d + \ln \frac 1 \epsilon + \ln \frac{k(k+1)}{\delta})^3$, and $k_0 = O(\ln\frac1\epsilon)$.
As a consequence, the total number of label queries is $O(t \cdot (\ln d + \ln \frac 1 \epsilon)^3 \cdot \ln \frac 1 \epsilon)$ in terms of $t, d$ and $\epsilon$.
The theorem follows.
\end{proof}


%Observe that Lemma 1 can be straightforwardly
%proved by induction in conjunction with the two claims.


%Now, denote by $S$ the support of $u$, $\hat{S}_k$ the set of coordinates that $P_t$ keeps on $\hat{w}_k'$.
%Additionally, for any vector $v$ and set $U \subset [d]$, denote by $v^U$ as the vector
%that agrees with $w$ on coordinates in $U$, and are 0's otherwise.
%We have that
%$\| \hat{w}_k'^{S^C} \|_2 = \| \hat{w}_k'^{S^C} - u^{S^C} \|_2 \leq \| \hat{w}_k' - u \|_2  \leq 2^{-k} / 4 $.
%By the optimality of $\hat{S}_k$,
%\[ \| \hat{w}_k'^{\hat{S}_k^C} \|_2 \leq \| \hat{w}_k'^{S^C} \|_2 \leq 2^{-k} / 4 \]
%Observe that $w_k = \hat{w}_k'^{\hat{S}_k^C}$, the above can be written as
%\[ \| \hat{w}_k' - w_k \|_2 \leq 2^{-k} / 4 \]
%Combining with Equation~\eqref{eqn:normalize_effect}, and triangle inequality, we have
%The lemma follows.
%\[ \| w_k - u \|_2 \leq \| \hat{w}_k' - w_k \|_2 + \| \hat{w}_k' - u \|_2 \leq 2^{-k} / 2. \]

\section{Conclusions and future work}

We give a computationally efficient PAC active halfspace learning algorithm that enjoys sharp attribute efficient label complexity bounds.
It combines the margin-based framework of~\cite{BBZ07,BL13} with iterative hard thresholding~\citep{BD09, GK09}.
The main novel technical component in our analysis is a uniform concentration bound of hinge losses over shrinking $\ell_1$ balls in the sampling regions.
We outline several promising directions of future research:
\begin{itemize}
\item Can we extend our algorithm to work under $\eta$-bounded noise, when $\eta$ is arbitrarily close to $\frac 1 2$? Recall that the results of \cite{ZC14}
imply a computationally inefficient algorithm with a label complexity of $O(\frac{t \ln d}{(1-2\eta)^2} \ln \frac 1 \epsilon)$ in this setting,
which state of the art computationally efficient algorithms~\citep[e.g.][]{ABHZ16} cannot achieve.
%A promising direction is to ``activize'' existing computationally and attribute efficient online halfspace learning algorithms, e.g.~\cite{L87, GLS01, G03}, as is done in the full-dimensional setting~\citep{DKM05, YZ17}.

\item Can we design attribute and computationally efficient active learning algorithms that work under broader distributions? Existing results in the active learning and one-bit compressed sensing literature have made substantial progress on settings when the unlabeled distribution is $\alpha$-stable~\citep{L16}, subgaussian~\citep{ALPV14, CB15}, or $s$-concave~\citep{BZ17}; an attribute and computationally efficient, statistically consistent recovery algorithm under any of the above settings would be a step forward.

%Recent work~\citep{L16} shows that when
%the unlabeled distribution is $\alpha$-stable for small $\alpha$, then the support of the target vector can be provably recovered. Can one approximately recover the target halfspace under this distribution?
%Can the result of \cite{L16} be extended to approximately
%recover $u$?

\item In one-bit compressed sensing, under the symmetric noise condition~\citep{PV13b}, algorithms with sample complexity polynomial in $\frac 1 \epsilon$ have been proposed~\citep{PV13b, ZYJ14, ZG15}.
Can we develop adaptive one-bit compressed sensing algorithms with $O(t \polylog(d,\frac 1 \epsilon))$ measurement complexity in this setting?
\end{itemize}


\section*{Acknowledgments}
I am grateful to Daniel Hsu for suggesting this research direction to me, and many insightful discussions along this line. I would also like to thank Pranjal Awasthi, Jie Shen and Hongyang Zhang for helpful initial conversations about the results in this paper. I thank the anonymous COLT reviewers for their thoughtful comments. Special thanks to Yue Liu, who provided unconditional support throughout this research project.

%\bibliographystyle{plain}
\bibliography{alsearch}

 %Oftentimes we assume that
 %$u^*$ is sparse, that is, there are at most $k$ nonzero elements ($k \ll d$) in vector $u$.
\newpage
\appendix

%\section*{Acknowledgement}

%Y. Fei and Y. Chen were partially supported by the National Science
%Foundation CRII award 1657420 and grant 1704828.


\appendix
%\appendixpage

\section{Additional notations}

We define the shorthand $\error\coloneqq\norm[\Yhat-\Ystar]1$. For
a matrix $\M$, we write $\norm[\M]{\infty}\coloneqq\max_{i,j}\left|M_{ij}\right|$
as its entry-wise $\ell_{\infty}$ norm, and $\opnorm{\M}$ as its
spectral norm (maximum singular value).  We let $\I$ and $\OneMat$
be the $\num\times\num$ identity matrix and all-one matrix, respectively.
For a real number $x$, $\left\lceil x\right\rceil $ denotes its
ceiling. We denote by $\clustset a\coloneqq\left\{ i\in\left[\num\right]:\labelstar(i)=a\right\} $
the set of indices of points in cluster $a$, and we define $\size\coloneqq\left|\clustset a\right|=\frac{\num}{\numclust}$. 

\section{Proof of Theorem \ref{thm:ip_sdp}\label{sec:proof_ip_sdp}}

\subsection{Initial steps}

We assume $\error>0$ since otherwise we are done. We can write $\Adj=\C+\C\t-2\H\H\t$,
where $\H$ is a matrix whose $i$-th row is the point $\h_{i}$ and
$\C$ is a matrix where the entries in the $i$-th row are identical
and equal to $\norm[\h_{i}]2^{2}$. Since the row-sum constraint in
the program (\ref{eq:SDP1}) ensures that the matrix $\Yhat-\Ystar$
has zero row sum, we have $\left\langle \Yhat-\Ystar,\C\right\rangle =\left\langle \Yhat-\Ystar,\C\t\right\rangle =0$
which implies $\left\langle \Yhat-\Ystar,\C+\C\t\right\rangle =0$.

Let $\G\coloneqq\H-\E\H$ be a matrix of entries in $\H$ with their
means removed. We can compute
\begin{align*}
\H\H\t & =\left(\G+\E\H\right)\left(\G+\E\H\right)\t\\
 & =\G\G\t+\G\left(\E\H\right)\t+\left(\E\H\right)\G\t+\left(\E\H\right)\left(\E\H\right)\t
\end{align*}
and 
\[
\E\H\H^{\top}=\E\G\G\t+\left(\E\H\right)\left(\E\H\right)\t.
\]
Therefore 
\[
\H\H\t-\E\H\H^{\top}=\left(\G\G\t-\E\G\G\t\right)+\G\left(\E\H\right)\t+\left(\E\H\right)\G\t.
\]
Let $\U\in\real^{\num\times\numclust}$ be the matrix of the left
singular vectors of $\Ystar$. For any $\M\in\real^{\num\times\num}$,
define the projection $\PT\left(\M\right)\coloneqq\U\U\t\M+\M\U\U\t-\U\U\t\M\U\U\t$
and its orthogonal complement $\PTperp\left(\M\right)\coloneqq\M-\PT\left(\M\right)$.
The fact that $\Yhat$ is optimal and $\Ystar$ is feasible to the
program (\ref{eq:SDP1}) implies 
\begin{align*}
0 & \leq-\frac{1}{2}\left\langle \Yhat-\Ystar,\Adj\right\rangle \\
 & =\left\langle \Yhat-\Ystar,\H\H\t-\E\H\H^{\top}\right\rangle +\left\langle \Yhat-\Ystar,\E\H\H^{\top}\right\rangle \\
 & =\left\langle \Yhat-\Ystar,\G\G\t-\E\G\G\t+\G\left(\E\H\right)\t+\left(\E\H\right)\G\t\right\rangle +\left\langle \Yhat-\Ystar,\E\H\H^{\top}\right\rangle \\
 & =\left\langle \Yhat-\Ystar,\PT\left(\G\G\t-\E\G\G\t\right)\right\rangle +\left\langle \Yhat-\Ystar,\PTperp\left(\G\G\t-\E\G\G\t\right)\right\rangle \\
 & \quad+2\left\langle \Yhat-\Ystar,\G\left(\E\H\right)\t\right\rangle +\left\langle \Yhat-\Ystar,\E\H\H^{\top}\right\rangle \\
 & \eqqcolon S_{1}+S_{2}+2S_{3}+S_{4}.
\end{align*}
We may control $S_{1}$, $S_{2}$ and $S_{4}$ using the following. 
\begin{prop}
\label{prop:S1} If $\snr^{2}\geq C\left(\sqrt{\frac{\numclust\vecdim}{\num}\log\left(\num\numclust\right)}+\sqrt{\frac{\numclust}{\num}}\log\left(\num\numclust\right)\right)$
for some universal constant $C>0$, then $S_{1}\leq\frac{1}{100}\minsep^{2}\error$
with probability at least $1-\left(2\num\right)^{-2}$.
\end{prop}

\begin{prop}
\label{prop:S2} If $\snr^{2}\geq C\numclust\left(\sqrt{\frac{\vecdim}{\num}}+1\right)$
for some universal constant $C>0$, then $S_{2}\leq\frac{1}{100}\minsep^{2}\error$
with probability at least $1-2e^{-\num}$.
\end{prop}

\begin{prop}
\label{prop:S4} We have $S_{4}=-\frac{1}{2}\sum_{a\ne b}T_{ab}\minsep_{ab}^{2}\le-\frac{1}{4}\minsep^{2}\error$
where $T_{ab}\coloneqq\sum_{i\in\clustset a,j\in\clustset b}\left(\Yhat-\Ystar\right)_{ij}$. 
\end{prop}
The proofs are given in Sections \ref{sec:proof_S1}, \ref{sec:proof_S2}
and \ref{sec:proof_S4}, respectively. Combining the above propositions,
we have $S_{1}+S_{2}\le-\frac{1}{2}S_{4}$ and therefore 
\begin{equation}
0\leq S_{3}+\frac{1}{4}S_{4}\eqqcolon S_{0}\label{eq:error_S3_bound}
\end{equation}
with probability at least $1-\left(2\num\right)^{-C'}-2e^{-\num}$
for some universal constant $C'>0$.

Let $\B\coloneqq\Yhat-\Ystar$. We have 
\begin{align*}
S_{3} & =\sum_{j}\sum_{a}\sum_{i\in C_{a}}B_{ji}\left\langle \Mean_{a},\g_{j}\right\rangle \\
 & =\size\sum_{j}\sum_{a}\left\langle \Mean_{a},\g_{j}\right\rangle \left(\frac{1}{\size}\sum_{i\in\clustset a}B_{ji}\right)\\
 & =\size\sum_{j}\sum_{a\ne\labelstar(j)}\left\langle \Mean_{a}-\Mean_{\labelstar(j)},\g_{j}\right\rangle \left(\frac{1}{\size}\sum_{i\in\clustset a}B_{ji}\right)
\end{align*}
where the last step holds since $\sum_{a\ne\labelstar(j)}\left(\sum_{i\in\clustset a}B_{ji}\right)=-\sum_{i\in\clustset a:a=\labelstar(j)}B_{ji}$
for each $j\in\left[\num\right]$ which follows from the row-sum constraint
of program (\ref{eq:SDP1}). By Proposition \ref{prop:S4}, we have
\begin{align*}
S_{4} & =-\size\sum_{j}\sum_{a\ne\labelstar(j)}\frac{1}{2}\minsep_{\labelstar(j),a}^{2}\left(\frac{1}{\size}\sum_{i\in\clustset a}B_{ji}\right).
\end{align*}
Therefore, we have 
\[
S_{0}=\size\sum_{j}\sum_{a\ne\labelstar(j)}\left(\left\langle \Mean_{a}-\Mean_{\labelstar(j)},\g_{j}\right\rangle -c\minsep_{\labelstar(j),a}^{2}\right)\left(\frac{1}{\size}\sum_{i\in\clustset a}B_{ji}\right)
\]
where $c=\frac{1}{8}$.

To control $S_{0}$, we define $\beta_{ja}\coloneqq\left\langle \Mean_{a}-\Mean_{\labelstar(j)},\g_{j}\right\rangle -c\minsep_{\labelstar(j),a}^{2}$
and consider the program 
\begin{align}
\max_{\X}\  & \sum_{j}\sum_{a\ne\labelstar(j)}\beta_{ja}X_{ja}\nonumber \\
\text{s.t.}\  & 0\leq X_{ja}\leq1,\qquad\forall a\ne\labelstar(j),j\in\left[\num\right]\nonumber \\
 & \sum_{a\ne\labelstar(j)}X_{ja}\leq1,\qquad\forall j\in\left[\num\right]\label{eq: int_opt}\\
 & \sum_{j}\sum_{a\ne\labelstar(j)}X_{ja}=R,\nonumber 
\end{align}
where $R\in(0,\num]$. Let us denote by $V(R)$ the optimal value
of the above program and we let $V(R)=-\infty$ if the program is
infeasible. The constraints of program (\ref{eq:SDP1}) implies that
$\frac{\error}{2\size}\in(0,\num]$ and 
\[
\sum_{j\in\left[\num\right]}\sum_{a\ne\labelstar(j)}\left(\sum_{i\in\clustset a}B_{ji}\right)=\frac{\error}{2}.
\]
Hence, by Equation (\ref{eq:error_S3_bound}), we have 
\begin{equation}
0\leq S_{0}\leq\size\cdot V\left(\frac{\error}{2\size}\right).\label{eq:basic_ineq_upper_bound_V}
\end{equation}


\subsection{Controlling $\protect\error$ by LP}

We show that $\error$ is upper bounded by the objective value of
an LP that is related to program (\ref{eq: int_opt}). If $\error=0$
then the conclusion of Theorem \ref{thm:ip_sdp} holds trivially.
For $\error>0$, we have the following cases:
\begin{enumerate}
\item If $\frac{\error}{2\size}\in(0,1]$, it follows from Equation (\ref{eq:basic_ineq_upper_bound_V})
that the error $\error$ must satisfy 
\[
0\le V\left(\frac{\error}{2\size}\right)=\beta^{*}\frac{\error}{2\size}\le\beta^{*}\left\lceil \frac{\error}{2\size}\right\rceil =V\left(\left\lceil \frac{\error}{2\size}\right\rceil \right)
\]
where $\beta^{*}\coloneqq\max_{j\in\left[\num\right],a\ne\labelstar(j)}\beta_{ja}$.
This implies 
\begin{align*}
\frac{\error}{2\size}\le\left\lceil \frac{\error}{2\size}\right\rceil  & \le\max\left\{ R\in\{0,1,.\ldots\}:V(R)\ge0\right\} .
\end{align*}
\item If $\frac{\error}{2\size}>1$, it follows from Equation (\ref{eq:basic_ineq_upper_bound_V})
that the error $\error$ must satisfy 
\[
0\le V\left(\frac{\error}{2\size}\right)\le\max\left\{ V\left(\left\lceil \frac{\error}{2\size}\right\rceil \right),V\left(\left\lfloor \frac{\error}{2\size}\right\rfloor \right)\right\} =\max\left\{ V\left(\left\lceil \frac{\error}{2\size}\right\rceil \right),V\left(\left\lceil \frac{\error}{2\size}\right\rceil -1\right)\right\} .
\]
In other words, we have
\begin{align*}
\frac{\error}{2\size}\le\left\lceil \frac{\error}{2\size}\right\rceil  & \le\max\left\{ R\in\{0,1,.\ldots\}:V(R)\vee V(R-1)\ge0\right\} \\
 & =1+\max\left\{ R\in\{0,1,.\ldots\}:V(R)\ge0\right\} .
\end{align*}
Note that $\left\lceil \frac{\error}{2\size}\right\rceil \ge2$, and
therefore we must have $1\le\max\left\{ R\in\{0,1,.\ldots\}:V(R)\ge0\right\} $.
This implies 
\[
\frac{\error}{2\size}\le2\max\left\{ R\in\{0,1,.\ldots\}:V(R)\ge0\right\} .
\]
\end{enumerate}
Consequently, we have 
\[
\frac{\error}{2\size}\le2\max\left\{ R\in\{0,1,.\ldots\}:V(R)\ge0\right\} .
\]


\subsection{Converting LP to IP}

We are now ready to formally establish a connection between the error
of the SDP (\ref{eq:SDP1}) and that of the Oracle IP (\ref{eq:oracleIP}),
by relating $\max\left\{ R\in\{0,1,.\ldots\}:V(R)\ge0\right\} $ to
the quantity (\ref{eq:IPerror}). Note that if $R\ge0$ is an integer,
then there exists an optimal solution $\left\{ w_{ja}\right\} $ of
program (\ref{eq: int_opt}) such that $w_{ja}\in\{0,1\}$ for all
$j\in[\num],a\in[\numclust]$. Therefore, if $R\in\{0,1,\ldots\}$
is an integer, then 
\begin{equation}
V(R)=\IP_{1}(R)\coloneqq\left\{ \begin{aligned}\max_{\X}\  & \sum_{j}\sum_{a\ne\labelstar(j)}\beta_{ja}X_{ja}\\
\text{s.t.}\  & X_{ja}\in\{0,1\},\qquad\forall a\ne\labelstar(j),j\in\left[\num\right]\\
 & \sum_{a\ne\labelstar(j)}X_{ja}\leq1,\qquad\forall j\in\left[\num\right]\\
 & \sum_{j}\sum_{a\ne\labelstar(j)}X_{ja}=R
\end{aligned}
\right\} .\label{eq:IP1}
\end{equation}
Combining the last two display equations we obtain that
\begin{align}
\frac{\error}{2\size} & \le2\max\left\{ R\in\{0,1,.\ldots\}:\IP_{1}(R)\ge0\right\} \nonumber \\
 & \overset{}{=}2\cdot\left\{ \begin{aligned}\max_{R,\X}\; & R\\
\text{s.t.}\; & R\in\{0,1,\ldots\}\\
 & \sum_{j}\sum_{a\ne\labelstar(j)}\beta_{ja}X_{ja}\ge0\\
 & X_{ja}\in\{0,1\},\qquad\forall a\ne\labelstar(j),j\in\left[\num\right]\\
 & \sum_{a\ne\labelstar(j)}X_{ja}\leq1,\qquad\forall j\in\left[\num\right]\\
 & \sum_{j}\sum_{a\ne\labelstar(j)}X_{ja}=R,
\end{aligned}
\right\} \nonumber \\
 & =2\cdot\IP_{2}\coloneqq2\cdot\left\{ \begin{aligned}\max_{\X}\; & \sum_{j}\sum_{a\ne\labelstar(j)}X_{ja}\\
\text{s.t.}\; & \sum_{j}\sum_{a\ne\labelstar(j)}\beta_{ja}X_{ja}\ge0\\
 & X_{ja}\in\{0,1\},\qquad\forall a\ne\labelstar(j),j\in\left[\num\right]\\
 & \sum_{a\ne\labelstar(j)}X_{ja}\leq1,\qquad\forall j\in\left[\num\right]
\end{aligned}
\right\} .\label{eq:error_bound2}
\end{align}

Let us reparameterize the integer program $\IP_{2}$ by a change of
variable. Recall that 
\[
\mathcal{F}\coloneqq\left\{ \F\in\{0,1\}^{\num\times\numclust}:\F\one_{\numclust}=\one_{\num}\right\} 
\]
is the set of all possible assignment matrices and $\F^{*}\in\mathcal{F}$
is the true assignment matrix; that is, $F_{ja}^{*}=\indic\left\{ a=\labelstar(j)\right\} $
for all $j\in[\num],a\in[\numclust]$. Consider any feasible solution
$\X$ of $\IP_{2}$; here for each $j\in[\num]$, we may fix $X_{j,\labelstar(j)}=-\sum_{a\neq\labelstar(j)}X_{ja}$
\textemdash{} doing so does not affect the feasibility and objective
value of $\X$ w.r.t. $\IP_{2}$. Define the new variable $\F\coloneqq\F^{*}+\X\in\mathcal{F}$.
The objective value and constraints of the old variable $\X$ can
be mapped to those of $\F$; in particular, we have 
\begin{align*}
\sum_{j}\sum_{a\ne\labelstar(j)}X_{ja} & =\sum_{j}\sum_{a\ne\labelstar(j)}(F_{ja}-F_{ja}^{*})=\frac{1}{2}\norm[\F-\F^{*}]1
\end{align*}
and
\begin{align*}
\left.\begin{array}{c}
X_{ja}\in\{0,1\},\forall a\ne\labelstar(j),j\in\left[\num\right]\\
\sum_{a\ne\labelstar(j)}X_{ja}\leq1,\forall j\in\left[\num\right]\\
X_{j,\labelstar(j)}=-\sum_{a\neq\labelstar(j)}X_{ja},\forall j\in[\num]
\end{array}\right\}  & \Longleftrightarrow\F\in\mathcal{F}
\end{align*}
and
\[
\sum_{j}\sum_{a\ne\labelstar(j)}\beta_{ja}X_{ja}\overset{(i)}{=}\sum_{j}\sum_{a}\beta_{ja}X_{ja}=\sum_{j}\sum_{a}\beta_{ja}F_{ja}-\sum_{j}\sum_{a}\beta_{ja}F_{ja}^{*}\overset{(ii)}{=}\sum_{j}\sum_{a}\beta_{ja}F_{ja},
\]
where steps $(i)$ and $(ii)$ both follow from the fact that $\beta_{j,\labelstar(j)}=0,\forall j.$
It follows that $\IP_{2}$ has the same optimal value as a corresponding
integer program in terms of $\X$; in particular, we have
\[
\IP_{2}=\IP_{3}\coloneqq\left\{ \begin{aligned}\max_{\F}\; & \frac{1}{2}\norm[\F-\F^{*}]1\\
\text{s.t.}\; & \sum_{j}\sum_{a}\beta_{ja}F_{ja}\ge0\\
 & \F\in\mathcal{F}
\end{aligned}
\right\} .
\]
Combining with equation (\ref{eq:error_bound2}), we see that the
error $\error$ satisfies
\begin{equation}
\frac{\error}{2\size}\le2\cdot\IP_{3}.\label{eq:error_bound3}
\end{equation}

We further simplify the first constraint in $\IP_{3}$. Recall that
$\bar{\h}_{i}\coloneqq\Mean_{\labelstar(i)}+(2c)^{-1}\g_{i}$ for
each $i\in[\num]$. Note that $\left\{ \bar{\h}_{i}\right\} $ can
be viewed as data points generated from the Sub-Gaussian Mixture Model
but with $(2c)^{-1}$ times the standard deviation. By definition
of $\beta_{ja}$, we have 
\begin{align*}
\beta_{ja} & =\left\langle \Mean_{a}-\Mean_{\labelstar(j)},\g_{j}\right\rangle -c\minsep_{\labelstar(j),a}^{2}\\
 & =c\left(2\left\langle \Mean_{a}-\Mean_{\labelstar(j)},(2c)^{-1}\g_{j}\right\rangle -\minsep_{\labelstar(j),a}^{2}\right)\\
 & =c\left(2\left\langle \Mean_{a}-\Mean_{\labelstar(j)},(2c)^{-1}\g_{j}\right\rangle -\norm[\Mean_{a}-\Mean_{\labelstar(j)}]2^{2}\right)\\
 & =c\left(2\left\langle \Mean_{a}-\Mean_{\labelstar(j)},(2c)^{-1}\g_{j}\right\rangle -\norm[\Mean_{a}-\Mean_{\labelstar(j)}]2^{2}-\norm[(2c)^{-1}\g_{j}]2^{2}+\norm[(2c)^{-1}\g_{j}]2^{2}\right)\\
 & =c\left(-\norm[\Mean_{\labelstar(j)}-\Mean_{a}+(2c)^{-1}\g_{j}]2^{2}+\norm[(2c)^{-1}\g_{j}]2^{2}\right)\\
 & =c\left(-\norm[\bar{\h}_{j}-\Mean_{a}]2^{2}+\norm[(2c)^{-1}\g_{j}]2^{2}\right).
\end{align*}
For any $\F\in\mathcal{F}$, we then have
\begin{align*}
\sum_{j}\sum_{a}\beta_{ja}F_{ja} & =c\sum_{j}\sum_{a}\left(-\norm[\bar{\h}_{j}-\Mean_{a}]2^{2}+\norm[(2c)^{-1}\g_{j}]2^{2}\right)F_{ja}\\
 & =c\left(-\sum_{j}\sum_{a}\norm[\bar{\h}_{j}-\Mean_{a}]2^{2}F_{ja}+\sum_{j}\norm[(2c)^{-1}\g_{j}]2^{2}\sum_{a}F_{ja}\right)\\
 & \overset{(i)}{=}c\left(-\sum_{j}\sum_{a}\norm[\bar{\h}_{j}-\Mean_{a}]2^{2}F_{ja}+\sum_{j}\norm[(2c)^{-1}\g_{j}]2^{2}\sum_{a}F_{ja}^{*}\right)\\
 & =c\left(-\sum_{j}\sum_{a}\norm[\bar{\h}_{j}-\Mean_{a}]2^{2}F_{ja}+\sum_{j}\sum_{a}\norm[(2c)^{-1}\g_{j}]2^{2}F_{ja}^{*}\right)\\
 & =c\left(-\sum_{j}\sum_{a}\norm[\bar{\h}_{j}-\Mean_{a}]2^{2}F_{ja}+\sum_{j}\sum_{a}\norm[\bar{\h}_{j}-\Mean_{\labelstar(j)}]2^{2}F_{ja}^{*}\right)\\
 & \overset{(ii)}{=}c\left(-\sum_{j}\sum_{a}\norm[\bar{\h}_{j}-\Mean_{a}]2^{2}F_{ja}+\sum_{j}\sum_{a}\norm[\bar{\h}_{j}-\Mean_{a}]2^{2}F_{ja}^{*}\right),
\end{align*}
where step $(i)$ holds because $\sum_{a}F_{ja}=1=\sum_{a}F_{ja}^{*},\forall j$,
and step $(ii)$ holds because $F_{ja}^{*}=1$ only if $a=\labelstar(j)$.
Again recall the shorthand
\[
\eta(\F)\coloneqq\sum_{j}\sum_{a}\norm[\bar{\h}_{j}-\Mean_{a}]2^{2}F_{ja}.
\]
We have the more compact expression
\begin{equation}
\sum_{j}\sum_{a}\beta_{ja}F_{ja}=c\left(\eta(\F^{*})-\eta(\F)\right)\label{eq:eta_func_equivalence}
\end{equation}
It follows that for any $\F\in\mathcal{F}$, the first constraint
in $\IP_{3}$ is satisfied if and only if 
\[
\eta(\F)\le\eta(\F^{*}).
\]
Combining with the (\ref{eq:error_bound3}), we obtain that
\[
\frac{\error}{2\size}\le2\cdot\IP_{3}=2\cdot\left\{ \begin{aligned}\max_{\F}\; & \frac{1}{2}\norm[\F-\F^{*}]1\\
\text{s.t.}\; & \eta(\F)\le\eta(\F^{*})\\
 & \F\in\mathcal{F}
\end{aligned}
\right\} .
\]
Rearranging terms, we have the bound
\begin{equation}
\error\le2\size\cdot\max\left\{ \norm[\F-\F^{*}]1:\eta(\F)\le\eta(\F^{*}),\F\in\mathcal{F}\right\} .\label{eq:error_bound3a}
\end{equation}
The result follows from the fact that $\norm[\Ystar]1=\num\size$
and $\norm[\F^{*}]1=\num$. 

\subsection{Proof of Proposition \ref{prop:S1} \label{sec:proof_S1}}

In this section we control $S_{1}$. We can further decompose $S_{1}$
as 
\begin{align*}
S_{1} & =\left\langle \Yhat-\Ystar,\U\U\t\left(\G\G\t-\E\G\G\t\right)\right\rangle +\left\langle \Yhat-\Ystar,\left(\G\G\t-\E\G\G\t\right)\U\U\t\right\rangle \\
 & \qquad-\left\langle \Yhat-\Ystar,\U\U\t\left(\G\G\t-\E\G\G\t\right)\U\U\t\right\rangle \\
 & \leq2\left|\left\langle \Yhat-\Ystar,\U\U\t\left(\G\G\t-\E\G\G\t\right)\right\rangle \right|+\left|\left\langle \Yhat-\Ystar,\U\U\t\left(\G\G\t-\E\G\G\t\right)\U\U\t\right\rangle \right|\\
 & \eqqcolon2T_{1}+T_{2}
\end{align*}
By the generalized Holder's inequality, we have 
\begin{align*}
T_{1} & \leq\error\cdot\norm[\U\U\t\left(\G\G\t-\E\G\G\t\right)]{\infty}
\end{align*}
and 
\begin{align*}
T_{2} & =\left|\left\langle \Yhat-\Ystar,\U\U\t\left(\G\G\t-\E\G\G\t\right)\U\U\t\right\rangle \right|\\
 & =\left|\left\langle \left(\Yhat-\Ystar\right)\U\U\t,\U\U\t\left(\G\G\t-\E\G\G\t\right)\right\rangle \right|\\
 & \leq\error\cdot\norm[\U\U\t\left(\G\G\t-\E\G\G\t\right)]{\infty}
\end{align*}
where the last inequality holds since 
\[
\norm[\left(\Yhat-\Ystar\right)\U\U\t]1\leq\norm[\Yhat-\Ystar]1=\error.
\]
Combining the above, we have 
\[
S_{1}\leq3\error\cdot\norm[\U\U\t\left(\G\G\t-\E\G\G\t\right)]{\infty}.
\]

Note that there are $m=\num\numclust$ distinct random variables in
$\U\U\t\left(\G\G\t-\E\G\G\t\right)$ and let us call them $X_{1},\ldots,X_{m}$.
For each $i$, we can see that $X_{i}$ is the average of $\size$
entries in $\G\G\t-\E\G\G\t$ and we let $\B_{i}$ be an $\num\times\num$
matrix with $\size$ entries equal to 1 and the others equal to 0
such that $\size X_{i}=\left\langle \B_{i},\G\G\t-\E\G\G\t\right\rangle $.
To proceed, we need the Hanson-Wright inequality (an extension of
Exercise 6.2.7 on pp.$\ $140 in \citet{vershynin2017high}).
\begin{lem}[Higher-dimensional Hanson-Wright inequality]
\emph{ \label{lem:hanson-wright} }Let $\x_{1},\ldots,\x_{N}$ be
independent, mean zero, sub-Gaussian random vectors in $\real^{M}$.
Let $\B$ be an $N\times N$ matrix. For every $t\geq0$ and some
universal constant $c>0$, we have 
\[
\P\left[\left|\sum_{i,j}B_{ij}\left\langle \x_{i},\x_{j}\right\rangle -\E\sum_{i,j}B_{ij}\left\langle \x_{i},\x_{j}\right\rangle \right|\geq t\right]\leq4\exp\left[-c\min\left(\frac{t^{2}}{K^{4}M\norm[\B]F^{2}},\frac{t}{K^{2}\norm[\B]{}}\right)\right]
\]
where $K\coloneqq\max_{i}\norm[\x_{i}]{\psi_{2}}.$ 
\end{lem}
The proof is given in Section \ref{sec:proof_hanson_wright}. Using
Lemma \ref{lem:hanson-wright}, we see that for any $t\ge0$ 
\[
\P\left\{ \size X_{i}\geq t\right\} =\P\left\{ \left\langle \B_{i},\G\G\t-\E\G\G\t\right\rangle \geq t\right\} \leq4\exp\left[-c\min\left(\frac{t^{2}}{K^{4}\vecdim\size},\frac{t}{K^{2}\sqrt{\size}}\right)\right].
\]
We can choose $t^{*}=DK^{2}\sqrt{\size}\left(\sqrt{\vecdim\log m}+\log m\right)$
with $K=\sgnorm$ and $D>0$ a universal constant. Apply the union
bound, we have 
\[
S_{1}\leq3\error\cdot\frac{1}{\size}\cdot t^{*}
\]
with probability at least $1-m\cdot\P\left\{ \size X\geq t\right\} \geq1-\exp\left(-C'\log m\right)=1-m^{-C'}$
where $C'>0$ is a universal constant. The result follows from the
condition of the proposition.

\subsection{Proof of Proposition \ref{prop:S2} \label{sec:proof_S2}}

In this section we control $S_{2}$. We have 
\begin{align*}
S_{2} & =\left\langle \PTperp\left(\Yhat-\Ystar\right),\G\G\t-\E\G\G\t\right\rangle \\
 & \leq\Tr\left[\PTperp\left(\Yhat-\Ystar\right)\right]\cdot\opnorm{\G\G\t-\E\G\G\t}\\
 & \le\frac{\error}{\size}\cdot\opnorm{\G\G\t-\E\G\G\t}.
\end{align*}
Let $\Var\left(g_{ij}\right)=\std^{2}$. We record a fact about the
sub-Gaussian property of columns of $\G$. 
\begin{fact}
\label{fact:satisfy_cond_gauss_choas_operator_norm_bound} Let $\x\in\real^{\num}$
be an arbitrary column of $\G$. We have 
\[
\norm[\left\langle \x,\w\right\rangle ]{\psi_{2}}\leq C\frac{\sgnorm}{\std}\sqrt{\E\left\langle \x,\w\right\rangle ^{2}}\qquad\text{for any }\w\in\real^{\num},
\]
where $C>0$ is a universal constant and $C\frac{\sgnorm}{\std}\ge1$.
\end{fact}
The proof is given in Section \ref{sec:proof_satisfy_cond}. Applying
Lemma \ref{lem:subg_cov_mat_bound} with $\rho_{0}=\frac{\sgnorm}{\std}$,
we have 
\[
\opnorm{\frac{1}{d}\G\G\t-\frac{1}{d}\E\G\G\t}\leq C_{1}\rho_{0}^{2}\left(\sqrt{\frac{2\num}{\vecdim}}+\frac{2\num}{\vecdim}\right)\opnorm{\frac{1}{d}\E\G\G\t}
\]
with probability at least $1-2e^{-\num}$. Here we let $m=\vecdim,u=\num$
and define $\x_{i}$ to be the $i$-th column of $\G$ and $\x$ to
be a vector independent of but identically distributed as each column of $\G$ (note that columns of $\G$ are identically
distributed). We also use the fact that $\E\x\x\t=\frac{1}{\vecdim}\E\G\G\t=\std^{2}\I$.
Multiplying $\vecdim$ on both sides of the above equation yields
\[
\opnorm{\G\G\t-\E\G\G\t}\leq C_{1}\left(\sqrt{\frac{2\num}{\vecdim}}+\frac{2\num}{\vecdim}\right)\vecdim\sgnorm^{2}.
\]
Hence, we have 
\[
S_{2}\le\frac{\error}{\size}\cdot C_{1}\left(\sqrt{\frac{2\num}{\vecdim}}+\frac{2\num}{\vecdim}\right)\vecdim\sgnorm^{2}=2C_{1}\error\numclust\left(\sqrt{\frac{\vecdim}{\num}}+1\right)\frac{\minsep^{2}}{\snr^{2}}
\]
The result follows from the condition of the proposition.

\subsection{Proof of Proposition \ref{prop:S4} \label{sec:proof_S4}}

We can compute 
\[
\left(\E\H\H\t\right)_{ij}=\begin{cases}
\vecdim\std^{2}+\norm[\Mean_{\labelstar(i)}]2^{2} & \text{if }i=j\\
\norm[\Mean_{\labelstar(i)}]2^{2} & \text{if }i\ne j\text{ and }\labelstar(i)=\labelstar(j)\\
\left\langle \Mean_{\labelstar(i)},\Mean_{\labelstar(j)}\right\rangle  & \text{otherwise}.
\end{cases}
\]
We partition the matrix $\Yhat-\Ystar$ into $\numclust^{2}$ of $\size\times\size$
blocks, and note that $T_{ab}$ denotes the sum of entries within
the $(a,b)$-th block. The constraints of program (\ref{eq:SDP1})
implies that 
\begin{enumerate}
\item $T_{aa}\leq0$ for each $a\in\left[\numclust\right]$ and $T_{ab}\geq0$
for each $a\ne b\in\left[\numclust\right]$;
\item $T_{ab}=T_{ba}$ for each $a,b\in\left[\numclust\right]$;
\item $-T_{aa}=\sum_{b\in\left[\numclust\right]:b\ne a}T_{ab}$ for each
$a\in\left[\numclust\right]$;
\item $-\sum_{a\in\left[\numclust\right]}T_{aa}+\sum_{a,b\in\left[\numclust\right]:a\ne b}T_{ab}=\error$
and thus $-\sum_{a\in\left[\numclust\right]}T_{aa}=\sum_{a,b\in\left[\numclust\right]:a\ne b}T_{ab}=\frac{\error}{2}$.
\end{enumerate}
Since $\Yhat-\Ystar$ has zero diagonal, we can write 
\begin{align*}
S_{4} & =\sum_{a\in\left[\numclust\right]}T_{aa}\norm[\Mean_{a}]2^{2}+2\sum_{a,b\in\left[\numclust\right]:a<b}T_{ab}\left\langle \Mean_{a},\Mean_{b}\right\rangle \\
 & =-\sum_{a,b\in\left[\numclust\right]:a<b}T_{ab}\minsep_{ab}^{2}\\
 & =-\frac{1}{2}\sum_{a,b\in\left[\numclust\right]:a\ne b}T_{ab}\minsep_{ab}^{2}\\
 & \leq-\frac{1}{2}\sum_{a,b\in\left[\numclust\right]:a\ne b}T_{ab}\minsep^{2}\\
 & =-\frac{1}{4}\minsep^{2}\error.
\end{align*}


\section{Proof of Theorem \ref{thm:ip_exp_rate}\label{sec:proof_ip_exp_rate}}

We define the shorthand 
\[
\iperror\coloneqq\max\left\{ \frac{1}{2}\norm[\F-\F^{*}]1:\eta(\F)\le\eta(\F^{*}),\F\in\mathcal{F}\right\} .
\]
It is not hard to see that $\iperror$ takes integer values in $[0,\num]$.
If $\iperror=0$ then we are done. We therefore focus on the case
$\iperror\in\left[\num\right]$.

Suppose $\iperror>3\num\numclust e^{-\snr^{2}/C_{0}^{2}}$ for a fixed
$C_{0}>D/c$. Note that 
\[
3\num\numclust e^{-\snr^{2}/C_{0}^{2}}\overset{(i)}{\le}\num\numclust\cdot\frac{1}{\numclust}\cdot e^{-\snr^{2}/\left(2C_{0}^{2}\right)}\le\num e^{-\snr^{2}/\left(2C_{0}^{2}\right)}<\num
\]
where step $(i)$ holds since we have assumed $\snr^{2}\ge\consts\numclust$
for some universal constant $\consts>0$. We record an important result
for our proof.
\begin{lem}
\label{lm:order_stats} Let $m\ge4$ and $g\ge1$ be integers. Let
$\X\in\real^{m\times g}$ be a matrix such that each $X_{ja}$ is
a sub-Gaussian random variable with its mean equal to $\lambda_{ja}$
and its sub-Gaussian norm no larger than $\rho_{ja}$, and each pair
$X_{ja}$ and $X_{ib}$ are independent for $j\ne i$ and $a,b\in\left[g\right]$.
Then for some universal constant $D>0$ and for any $\beta\in(0,m]$,
we have 
\begin{align*}
\sum_{j,a}X_{ja}M_{ja} & \le D\sqrt{\left\lceil \beta\right\rceil \left(\sum_{j,a}\rho_{ja}^{2}M_{ja}\right)\log\left(3mg/\beta\right)}+\sum_{j,a}\lambda_{ja}M_{ja},\\
 & \qquad\quad\forall\M\in\left\{ 0,1\right\} ^{m\times g}:\M\one_{g}\le\one_{m},\norm[\M]1=\left\lceil \beta\right\rceil ,
\end{align*}
with probability at least $1-\frac{1.5}{m}$.
\end{lem}
The proof is given in Section \ref{sec:proof_lm_order_stat}. Define
the set 
\[
\calM\coloneqq\left\{ \M\in\left\{ 0,1\right\} ^{\num\times\numclust}:\M\one_{\numclust}\le\one_{\num},\norm[\M]1=\iperror,M_{j,\labelstar(j)}=0\ \forall j\in\left[\num\right]\right\} .
\]
For any $\F$ feasible to $\IP_{3}$, we have 
\begin{align*}
0 & \le\frac{1}{c}\left(\eta(\F^{*})-\eta(\F)\right)\\
 & \overset{(i)}{=}\sum_{j\in[\num]}\sum_{a\in[\numclust]}\beta_{ja}F_{ja}\\
 & =\sum_{\left(j,a\right):F_{ja}=1,a\ne\labelstar(j)}\beta_{ja}\\
 & \le\max_{\M\in\calM}\sum_{j}\sum_{a\ne\labelstar(j)}\beta_{ja}M_{ja}\\
 & \overset{(ii)}{\le}\max_{\M\in\calM}\left[D\sqrt{\iperror\sgnorm^{2}\left(\sum_{j}\sum_{a\ne\labelstar(j)}\minsep_{\labelstar(j),a}^{2}M_{ja}\right)\log\left(3\num\left(\numclust-1\right)/\iperror\right)}-c\sum_{j}\sum_{a\ne\labelstar(j)}\minsep_{\labelstar(j),a}^{2}M_{ja}\right]\\
 & \le\max_{\M\in\calM}\left[D\sqrt{\iperror\sgnorm^{2}\left(\sum_{j}\sum_{a\ne\labelstar(j)}\minsep_{\labelstar(j),a}^{2}M_{ja}\right)\frac{\snr^{2}}{C_{0}^{2}}}-c\sum_{j}\sum_{a\ne\labelstar(j)}\minsep_{\labelstar(j),a}^{2}M_{ja}\right]\\
 & \le\left(\frac{D}{C_{0}}-c\right)\cdot\max_{\M\in\calM}\sum_{j}\sum_{a\ne\labelstar(j)}\minsep_{\labelstar(j),a}^{2}M_{ja}
\end{align*}
where step $(i)$ holds by Equation (\ref{eq:eta_func_equivalence}),
step $(ii)$ holds by Lemma \ref{lm:order_stats} with $g=\numclust-1$
since only $\numclust-1$ entries of $\left\{ \beta_{ja}\right\} $
are considered for each $j$ in the sum above $(ii)$, and the last
step holds since $\iperror\minsep^{2}\le\sum_{j}\sum_{a\ne\labelstar(j)}\minsep_{\labelstar(j),a}^{2}M_{ja}$.
Since $C_{0}>D/c$ and $\sum_{j}\sum_{a\ne\labelstar(j)}\minsep_{\labelstar(j),a}^{2}M_{ja}>0$,
the RHS above is negative, which is a contradiction. Hence, we must
have $\iperror\le3\num\numclust e^{-\snr^{2}/C_{0}^{2}}\le\num e^{-\snr^{2}/\left(2C_{0}^{2}\right)}$
and the result follows from the fact that $\norm[\F^{*}]1=\num$.

\section{Proof of technical results}

In this section we provide the proofs of the technical results used
in the proofs of our main theorems.

\subsection{Proof of Lemma \ref{lem:hanson-wright}\label{sec:proof_hanson_wright}}

We record the following lemma (Exercise 6.2.7 on pp.$\ $140 in \citet{vershynin2017high}).
\begin{lem}[Higher-dimensional Hanson-Wright inequality]
\emph{} \label{lem:hanson_wright_hdp} Let $\x_{1},\ldots,\x_{N}$
be independent, mean zero, sub-Gaussian random vectors in $\real^{M}$.
Let $\B=\left\{ B_{ij}\right\} $ be an $N\times N$ matrix. There
exists some universal constant $c>0$ such that for every $t\geq0$
\[
\P\left[\left|\sum_{i,j:i\ne j}^{N}B_{ij}\left\langle \x_{i},\x_{j}\right\rangle \right|\geq t\right]\leq2\exp\left[-c\min\left(\frac{t^{2}}{K^{4}M\norm[\B]F^{2}},\frac{t}{K^{2}\opnorm{\B}}\right)\right]
\]
where $K\coloneqq\max_{i}\norm[\x_{i}]{\psi_{2}}.$ 
\end{lem}
With this result, we only need to prove the same tail bound for $\P\left[\left|\sum_{i=1}^{N}B_{ii}\left(\norm[\x_{i}]2^{2}-\E\norm[\x_{i}]2^{2}\right)\right|\geq t\right]$.
To prove that, we cite another useful lemma (Theorem 2.8.2 on pp.$\ $36
in \citet{vershynin2017high}).
\begin{lem}[Bernstein's inequality for sub-exponential random variables]
\emph{}\label{lem:bernstein-subexp} Let $X_{1},\ldots,X_{N}$ be
independent, mean zero, sub-exponential random variables, and $\a\in\real^{N}$.
Then for every $t\geq0$, we have 
\[
\P\left[\left|\sum_{i=1}^{N}a_{i}X_{i}\right|\geq t\right]\leq2\exp\left[-c\min\left(\frac{t^{2}}{K_{1}^{2}\norm[\a]2^{2}},\frac{t}{K_{1}\norm[\a]{\infty}}\right)\right]
\]
where $K_{1}\coloneqq\max_{i}\norm[X_{i}]{\psi_{1}}$.
\end{lem}
Here, $\norm[\cdot]{\psi_{1}}$ denotes the sub-exponential norm;
see \citet{vershynin2017high} for more details. We work under the
premise of Lemma \ref{lem:hanson_wright_hdp}. Since $\x_{i}$ are
independent sub-Gaussian random vectors, each $\norm[\x_{i}]2^{2}-\E\norm[\x_{i}]2^{2}$
is the sum of $M$ independent, mean zero, sub-exponential random
variables with sub-exponential norm equal to $K^{2}$. Then Lemma
\ref{lem:bernstein-subexp} implies 
\[
\P\left[\left|\sum_{i=1}^{N}B_{ii}\left(\norm[\x_{i}]2^{2}-\E\norm[\x_{i}]2^{2}\right)\right|\geq t\right]\leq2\exp\left[-c\min\left(\frac{t^{2}}{K^{4}M\norm[\B]F^{2}},\frac{t}{K^{2}\opnorm{\B}}\right)\right]
\]
as required. 

\subsection{Proof of Fact \ref{fact:satisfy_cond_gauss_choas_operator_norm_bound}
\label{sec:proof_satisfy_cond}}

We prove the following equivalent statement 
\[
\norm[\left\langle \x,\w\right\rangle ]{\psi_{2}}^{2}\leq C\frac{\sgnorm^{2}}{\std^{2}}\E\left\langle \x,\w\right\rangle ^{2}\qquad\text{for any }\w\in\real^{\num},
\]
where $C>0$ is a universal constant and $C\frac{\sgnorm^{2}}{\std}\ge1.$
We first establish a relationship between $\sgnorm^{2}$ and $\Var\left(x_{1}\right)$:
Proposition 2.5.2 on pp. 24 of \citet{vershynin2017high} implies
that $\frac{C'\sgnorm^{2}}{\std^{2}}\ge\frac{1}{2}$ for some universal
constant $C'>0$. Hence, we have 
\begin{align*}
\norm[\left\langle \x,\w\right\rangle ]{\psi_{2}}^{2} & \overset{(i)}{\le}2C'\sum_{i\in\left[\num\right]}w_{i}^{2}\norm[x_{i}]{\psi_{2}}^{2}\\
 & =2C'\frac{\sgnorm^{2}}{\std^{2}}\sum_{i\in\left[\num\right]}w_{i}^{2}\std^{2}\\
 & \overset{(ii)}{=}2C'\frac{\sgnorm^{2}}{\std^{2}}\E\left\langle \x,\w\right\rangle ^{2},
\end{align*}
where $(i)$ holds according to Proposition 2.6.1 on pp. 28 of \citet{vershynin2017high},
and $(ii)$ holds since $x_{i}$ are i.i.d.$\ $and $\E x_{i}=0$.
Letting $C=2C'$ completes the proof.

\subsection{Proof of Lemma \ref{lm:order_stats} \label{sec:proof_lm_order_stat}}

We define 
\begin{align*}
L_{\M} & \coloneqq\sum_{j,a}\left(X_{ja}-\lambda_{ja}\right)M_{ja},\\
R_{\beta,\M} & \coloneqq D\sqrt{\left\lceil \beta\right\rceil \left(\sum_{j,a}\rho_{ja}^{2}M_{ja}\right)\log\left(3mg/\beta\right)},\\
\calM_{\beta} & \coloneqq\left\{ \M\in\left\{ 0,1\right\} ^{m\times g}:\M\one_{g}\le\one_{m},\norm[\M]1=\left\lceil \beta\right\rceil \right\} .
\end{align*}
To establish a uniform bound in $\beta$, we apply a discretization
argument to the possible values of $\beta$. Define the shorthand
$E\coloneqq(0,m]$. We can cover $E$ by the sub-intervals $E_{t}\coloneqq(t-1,t]$
for $t\in[m]$. For each $t\in[m]$ we define the probability

\begin{align*}
\alpha_{t} & \coloneqq\P\left\{ \exists\beta\in E_{t},\exists\M\in\calM_{\beta}:L_{\M}>R_{\beta,\M}\right\} .
\end{align*}
We bound each of these probabilities: 

\begin{align}
\alpha_{t} & \overset{(i)}{\le}\P\left\{ \exists\M\in\calM_{t}:L_{\M}>R_{t,\M}\right\} \nonumber \\
 & \leq\P\left\{ \bigcup_{\M\in\calM_{t}}\left\{ L_{\M}>R_{t,\M}\right\} \right\} \nonumber \\
 & \leq\sum_{\M\in\calM_{t}}\P\left\{ L_{\M}>R_{t,\M}\right\} ,\label{eq:double union bd on normal-1-1-2}
\end{align}
where step $(i)$ holds since $\beta\in E_{t}$ implies $\beta\le\left\lceil \beta\right\rceil =t$. 

Note that each $X_{ja}-\lambda_{ja}$ is an independent zero-mean
sub-Gaussian random variable and the squared sub-Gaussian norm of
$L_{\M}$ is at most $C_{\psi_{2}}\sum_{j,a}\rho_{ja}^{2}M_{ja}$
where $C_{\psi_{2}}>0$ is a universal constant. We apply Hoeffding
inequality (Lemma \ref{lem:hoeffding}) to bound the probability on
the RHS of (\ref{eq:double union bd on normal-1-1-2}): 
\begin{align*}
\P\left\{ L_{\M}>R_{t,\M}\right\}  & \leq\exp\left\{ -\frac{cD^{2}t\left(\sum_{j,a}\rho_{ja}^{2}M_{ja}\right)\log(3mg/t)}{C_{\psi_{2}}\sum_{j,a}\rho_{ja}^{2}M_{ja}}\right\} \\
 & \leq\exp\left\{ -4t\log(3mg/t)\right\} 
\end{align*}
where $c>0$ is a universal constant. Plugging this back to (\ref{eq:double union bd on normal-1-1-2}),
we have for each $t\in\left[m\right]$, 
\begin{align}
\alpha_{t} & \leq\sum_{\M\in\calM_{t}}\exp\left\{ -4t\log(3mg/t)\right\} \nonumber \\
 & =\binom{m}{t}g^{t}\exp\left\{ -4t\log(3mg/t)\right\} \nonumber \\
 & \leq\left(\frac{me}{t}\right)^{t}g^{t}\exp\left\{ -4t\log(3mg/t)\right\} \nonumber \\
 & \leq\exp\left\{ t\log(3mg/t)+t-4t\log(3mg/t)\right\} \nonumber \\
 & \leq\exp\left\{ -t\log(3mg/t)\right\} =\left(\frac{t}{3mg}\right)^{t},\label{eq:binom coeff bound-1-2}
\end{align}
where the last inequality follows from $t\leq t\log(3mg/t)$ for $t\in\left[m\right]$.
It follows that 
\begin{align*}
 & \quad\P\left\{ \exists\beta\in E,\exists\M\in\calM_{\beta}:L_{\M}>R_{\beta,\M}\right\} \\
 & \leq\P\left\{ \bigcup_{t=1}^{m}\left\{ \exists\beta\in E_{t},\exists\M\in\calM_{\beta}:L_{\M}>R_{\beta,\M}\right\} \right\} \\
 & \le\sum_{t=1}^{m}\alpha_{t}\\
 & \leq\sum_{t=1}^{m}\left(\frac{t}{3mg}\right)^{t}\eqqcolon P_{1}(m).
\end{align*}

It remains to show that $P_{1}(m)\leq\frac{1.5}{m}$. Since 
\begin{align*}
P_{1}(m) & \leq\sum_{t=1}^{m}\left(\frac{t}{3m}\right)^{t}\\
 & \le\frac{1}{3m}+\sum_{t=2}^{m}\left(\frac{t}{3m}\right)^{t}\\
 & \le\frac{1}{3m}+m\cdot\max_{t=2,3,\ldots,m}\left(\frac{t}{3m}\right)^{t},
\end{align*}
the proof is completed if for each integer $t=2,3,\ldots,m$, we can
show the bound $\left(\frac{t}{3m}\right)^{t}\leq\frac{1}{m^{2}}$,
or equivalently $f(t)\coloneqq t(\log3m-\log t)\geq2\log m.$ Since
$t\le m$, $f(t)$ has derivative 
\[
f'(t)=\log3m-\log t-1\ge\log3m-\log\left(\frac{3m}{3}\right)-1=\log3-1\ge0.
\]
Therefore, $f(t)$ is non-decreasing for $2\le t\le m$ and therefore
$f(t)\ge f(2)=2\log3m-2\log2\ge2\log m.$ Hence, $P_{1}(m)\le\frac{1.5}{m}$. 

\section{Proof of Theorem \ref{thm:cluster_error_rate}\label{sec:proof_cluster_error_rate}}

We only need to prove the first part of the theorem. The second part
follows immediately from the first part and Theorem \ref{cor:SDP_exp_rate}.

The proof follows similar lines as those of Theorem 17 and Lemma 18
in \citet{makarychev2016learning}. In the rest of the section, we
work under the context of Algorithms \ref{alg:apx_clustering} and
\ref{alg:est_clustering}. Recall that $\numclust'=\left|\left\{ B_{t}\right\} _{t\ge1}\right|$
and we let $\epsilon\coloneqq\norm[\Yhat-\Ystar]1/\norm[\Ystar]1$.
We have the following lemma.
\begin{lem}
\label{lem:apx_clustering} There exists a partial matching $\perm'$
between $\left[\numclust\right]$ and $\left[\numclust'\right]$ and
a universal constant $C>0$ such that 
\[
\left|\bigcup_{t=\perm'(a)}\clustset a\cap B_{t}\right|\ge\left(1-C\epsilon\right)\num.
\]
\end{lem}
The proof is given in Section \ref{sec:proof_apx_clustering}. The
next lemma concerns the quality of clustering by Algorithm \ref{alg:clustering}.
\begin{lem}
\label{lem:clustering} There exists a permutation $\perm$ on $\left[\numclust\right]$
and a universal constant $C>0$ such that 
\[
\left|\bigcup_{t=\perm(a)}\clustset a\cap U_{t}\right|\ge\left(1-C\epsilon\right)\num.
\]
\end{lem}
The proof is given in Section \ref{sec:proof_clustering}. The result
follows from combining the above lemmas and the fact that 
\[
\misrate(\LabelHat,\LabelStar)=1-\frac{1}{\num}\max_{\perm\in S_{\numclust}}\left|\bigcup_{t=\perm(a)}\clustset a\cap U_{t}\right|.
\]


\subsection{Proof of Lemma \ref{lem:apx_clustering}\label{sec:proof_apx_clustering}}

We define $\y_{a}$ to be an arbitrary row of $\Ystar$ whose index
is in $\clustset a$.
\begin{align*}
G_{a} & \coloneqq\left\{ i\in\clustset a:\norm[\Yhat_{i\bullet}-\y_{a}]1\leq\frac{\size}{8}\right\} ,\qquad\forall a\in\left[\numclust\right]\\
G & \coloneqq\bigcup_{a\in\left[\numclust\right]}G_{a},\\
H & \coloneqq\vertexset\backslash G.
\end{align*}

We construct a partial matching $\perm'$ between sets $\clustset a$
and $B_{t}$ by matching every cluster $\clustset a$ with the first
$B_{t}$ that intersects $G_{a}$, and we let $\perm'(a)=t$. Since
each $i\in\left[\num\right]$ belongs to some $B_{t}$, we are able
to match every $\clustset a$ with some $B_{t}$. The fact that we
cannot match two distinct clusters $\clustset a$ and $\clustset b$
with the same $B_{t}$ as well as the rest of the proof are given
by the following fact.
\begin{fact}
\label{fact:apx_clustering} We have 
\begin{enumerate}
\item For each $a\in\left[\numclust\right]$ and $t\in\left[\numclust'\right]$
such that $t=\perm'(a)$, we have $B_{t}\cap G_{b}=\emptyset$ for
any $b\in\left[\numclust\right]\backslash\left\{ a\right\} $ and
$B_{t}\subset G_{a}\cup H$;
\item For each $a\in\left[\numclust\right]$ and $t\in\left[\numclust'\right]$
such that $t=\perm'(a)$, we have 
\[
\left|B_{t}\cap\clustset a\right|\geq\left|G_{a}\right|-\left|B_{t}\cap H\right|.
\]
\item We have 
\[
\sum_{t=\perm'(a)}\left|B_{t}\cap\clustset a\right|\ge\left|\vertexset\right|-2\left|H\right|.
\]
\item There exists a universal constant $C>0$ such that $\left|H\right|\leq C\epsilon\num$.
\end{enumerate}
\end{fact}
The proof is given below.

\subsubsection{Proof of Fact \ref{fact:apx_clustering}\label{sec:proof_fact_apx_clustering}}
\begin{enumerate}
\item Suppose that there exist $B_{t}$ and $b\in\left[\numclust\right]$
such that $b\ne a$ and $B_{t}\cap G_{b}\ne\emptyset$. Let $u\in B_{t}\cap G_{a}$
and $v\in B_{t}\cap G_{b}$. Since $G_{a}$ and $G_{b}$ are disjoint,
we know that $u\ne v$. Let $w\in B_{t}$. Then we have 
\begin{align*}
\norm[\Yhat_{u\bullet}-\Yhat_{w\bullet}]1 & \leq\frac{\size}{4}\\
\norm[\Yhat_{v\bullet}-\Yhat_{w\bullet}]1 & \leq\frac{\size}{4}.
\end{align*}
Therefore 
\[
\norm[\Yhat_{u\bullet}-\Yhat_{v\bullet}]1\leq\norm[\Yhat_{u\bullet}-\Yhat_{w\bullet}]1+\norm[\Yhat_{v\bullet}-\Yhat_{w\bullet}]1\le\frac{\size}{2}.
\]
This implies 
\begin{align*}
\norm[\y_{a}-\y_{b}]1 & \leq\norm[\y_{a}-\Yhat_{u\bullet}]1+\norm[\Yhat_{u\bullet}-\Yhat_{v\bullet}]1+\norm[\y_{b}-\Yhat_{v\bullet}]1\\
 & \leq\frac{\size}{8}+\frac{\size}{2}+\frac{\size}{8}<\size,
\end{align*}
which is a contradiction to the fact that $\norm[\y_{a}-\y_{b}]1=2\size$.
To complete the proof, we note that for any $i\in B_{t}$ we have
either $i\in G_{a}$ or $i\in H$.
\item Fix $i\in G_{a}$ for some $a\in\left[\numclust\right]$. For any
$j\in G_{a}$ we have $j\in B(i)$ since 
\[
\norm[\Yhat_{i\bullet}-\Yhat_{j\bullet}]1\le\norm[\y_{a}-\Yhat_{i\bullet}]1+\norm[\y_{a}-\Yhat_{j\bullet}]1\le\frac{\size}{4}.
\]
Therefore, by definition 
\[
\left|B_{t}\right|\ge\left|B(i)\right|\ge\left|G_{a}\right|.
\]
We have 
\begin{align*}
\left|B_{t}\cap\clustset a\right| & \overset{(i)}{\ge}\left|B_{t}\cap G_{a}\right|\\
 & =\left|B_{t}\right|-\left|B_{t}\backslash G_{a}\right|\\
 & \overset{(ii)}{=}\left|B_{t}\right|-\left|B_{t}\cap H\right|\\
 & \ge\left|G_{a}\right|-\left|B_{t}\cap H\right|,
\end{align*}
where step $(i)$ holds since $G_{a}\subset\clustset a$ and step
$(ii)$ holds since $B_{t}\subset G_{a}\cup H$.
\item Summing the LHS of the above equation over $t=\perm'(a)$ gives 
\begin{align*}
\sum_{t=\perm'(a)}\left|B_{t}\cap\clustset a\right| & =\sum_{a\in\left[\numclust\right]}\left|G_{a}\right|-\sum_{t=\perm'(a)}\left|B_{t}\cap H\right|\\
 & \ge\sum_{a\in\left[\numclust\right]}\left|G_{a}\right|-\sum_{t\ge1}\left|B_{t}\cap H\right|\\
 & \overset{(i)}{=}\left|G\right|-\left|\vertexset\cap H\right|\\
 & =\left|\vertexset\right|-2\left|H\right|,
\end{align*}
where step $(i)$ holds since $B_{t}\cap H$ are disjoint and $\bigcup_{t\geq1}B_{t}=\vertexset$.
\item We have 
\[
\left|H\right|\cdot\frac{\size}{8}\leq\sum_{i\in H}\norm[\Yhat_{i\bullet}-\y_{\labelstar(i)}]1\leq\norm[\Yhat-\Ystar]1\leq\epsilon\norm[\Ystar]1=\epsilon\cdot\num\size
\]
where the last step follows from the fact that $\norm[\Ystar]1=\num\size$.
The result follows.
\end{enumerate}

\subsection{Proof of Lemma \ref{lem:clustering}\label{sec:proof_clustering}}

Let $\perm'$ be the partial matching between $\clustset a$ and $B_{t}$
from Lemma \ref{lem:apx_clustering}. Define $\perm(a)=\perm'(a)$
for $\perm'(a)\le\numclust$. If the resulting $\perm$ is a partial
permutation, we extend $\perm$ to a permutation defined on $\left[\numclust\right]$
in an arbitrary way. We may assume that $\left\{ U_{t}\right\} _{t\in\left[\numclust\right]}$
are $\left\{ B_{t}\right\} _{t\in\left[\numclust\right]}$ WLOG, and
that $U_{t}$ consists of $B_{t}$ and some elements from sets $B_{u}$
with $u>\numclust$. We have 
\begin{align*}
\left|\bigcup_{t=\perm(a)}\clustset a\cap U_{t}\right| & \ge\left|\bigcup_{t=\perm'(a)\le\numclust}\clustset a\cap B_{t}\right|\\
 & =\left|\bigcup_{t=\perm'(a)}\clustset a\cap B_{t}\right|-\left|\bigcup_{t=\perm'(a)>\numclust}\clustset a\cap B_{t}\right|\\
 & \ge\left(1-C'\epsilon\right)\num-\left|\bigcup_{t=\perm'(a)>\numclust}\clustset a\cap B_{t}\right|
\end{align*}
where $C'>0$ is a universal constant. Define 
\begin{align*}
T_{1} & \coloneqq\left\{ t>\numclust:t=\perm'(a)\text{ for some }a\in\left[\numclust\right]\right\} ,\\
T_{2} & \coloneqq\left\{ t\in\left[\numclust\right]:t\ne\perm'(a)\text{ for any }a\in\left[\numclust\right]\right\} .
\end{align*}
Note that $\left|T_{1}\right|=\left|T_{2}\right|$ and for any $t_{1}\in T_{1}$
and $t_{2}\in T_{2}$ we have $\left|B_{t_{1}}\right|\le\left|B_{t_{2}}\right|$.
Therefore, 
\begin{align*}
\left|\bigcup_{t=\perm'(a)>\numclust}\clustset a\cap B_{t}\right| & \le\left|\bigcup_{t\in T_{1}}B_{t}\right|\\
 & \le\left|\bigcup_{t\in T_{2}}B_{t}\right|\\
 & \le\left|\vertexset\right|-\left|\bigcup_{t=\perm'(a)}\clustset a\cap B_{t}\right|\\
 & =C'\epsilon\num.
\end{align*}
The result follows by setting $C\coloneqq2C'$.

\section{Proof of Theorem \ref{thm:mean_estimation_error}\label{sec:proof_mean_estimation_error}}

Let $\Var\left(g_{ij}\right)=\std^{2}$. For $a\in\left[\numclust\right]$,
define $\clustest a\coloneqq\left\{ i\in\left[\num\right]:\labelhat_{i}=a\right\} $
the estimated clusters encoded in $\LabelHat$, and recall that our
cluster center estimators are defined by $\Meanhat_{a}\coloneqq\size^{-1}\sum_{i\in\clustest a}\h_{i}$.
We assume $\left\{ \clustest a\right\} $ achieves the lowest clustering
error as given in Theorem \ref{thm:cluster_error_rate} WLOG. For
each $a\in\left[\numclust\right]$, we have 
\begin{align*}
\norm[\Meanhat_{a}-\Mean_{a}]2 & \le\norm[\frac{1}{\size}\sum_{i\in\clustest a}\h_{i}-\frac{1}{\size}\sum_{j\in\clustset a}\h_{j}]2+\norm[\frac{1}{\size}\sum_{j\in\clustset a}\h_{j}-\Mean_{a}]2\\
 & \eqqcolon Q_{1}+Q_{2}.
\end{align*}


\subsection{Controlling $Q_{1}$}

Define $\epsilon\coloneqq\misrate(\LabelHat,\LabelStar)$. We work
on the event that the result Theorem \ref{thm:cluster_error_rate}
is true. We have 
\[
Q_{1}=\frac{1}{\size}\norm[\sum_{i\in\clustest a\backslash\clustset a}\h_{i}-\sum_{j\in\clustset a\backslash\clustest a}\h_{j}]2
\]
Note that $\left|\clustest a\backslash\clustset a\right|=\left|\clustset a\backslash\clustest a\right|$
so we can pair each point in $\clustest a\backslash\clustset a$ with
a point in $\clustset a\backslash\clustest a$. Let us pair $i$th
point in $\clustest a\backslash\clustset a$ with $j(i)$th point
in $\clustset a\backslash\clustest a$, and define $\calM\coloneqq\left\{ \left(i,j(i)\right)\right\} $.
We have $\left|\calM\right|\le\num\epsilon$ and we can write 
\begin{align*}
Q_{1} & =\frac{1}{\size}\norm[\sum_{(i,j(i))\in\calM}\left(\h_{i}-\h_{j(i)}\right)]2\\
 & \le\frac{1}{\size}\sum_{(i,j(i))\in\calM}\norm[\h_{i}-\h_{j(i)}]2\\
 & \le\frac{1}{\size}\sum_{(i,j(i))\in\calM}\left(\minsep_{\labelstar(i),\labelstar(j(i))}+\norm[\g_{i}-\g_{j(i)}]2\right)\\
 & \le\frac{1}{\size}\sum_{(i,j(i))\in\calM}\left(C_{q}\minsep+\norm[\g_{i}-\g_{j(i)}]2\right),
\end{align*}
where the last step holds for some universal constant $C_{q}>0$ given
that $\max_{a,b\in\left[\numclust\right]}\minsep_{ab}\le C_{q}\minsep$.
By Theorem 3.1.1 on pp.$\ $41 of \citet{vershynin2017high}, $\frac{1}{\sqrt{2}\std}\norm[\g_{i}-\g_{j(i)}]2-\sqrt{\vecdim}$
is a sub-Gaussian random variable with sub-Gaussian norm at most $C_{\psi_{2}}\frac{\sgnorm^{2}}{\std^{2}}$
where $C_{\psi_{2}}>0$ is a universal constant. Then Lemma \ref{lem:hoeffding}
implies that 
\[
\P\left[\frac{1}{\sqrt{2}\std}\norm[\g_{i}-\g_{j(i)}]2-\sqrt{\vecdim}\ge C\frac{\sgnorm^{2}}{\std^{2}}\sqrt{\log\num}\right]\le\num^{-C'}
\]
for some universal constants $C,C'>2$. By the union bound and the
facts that $\left|\calM\right|\le\num$ and $\std\lesssim\sgnorm$,
we have 
\[
\max_{(i,j)\in\calM}\norm[\g_{i}-\g_{j(i)}]2\le C_{g}\left(\sgnorm\sqrt{2\vecdim}+C\sgnorm\sqrt{2\log\num}\right)
\]
with probability at least $1-n^{-C_{1}}$ where $C_{g},C_{1}>0$ are
universal constants. 

Therefore, we have 
\begin{align*}
Q_{1} & \le C_{0}\left(\minsep+\sgnorm\sqrt{\vecdim}+\sgnorm\sqrt{\log\num}\right)\cdot\numclust\exp\left[-\frac{\snr^{2}}{\conste}\right]\\
 & \le C_{0}\left(\minsep+\sgnorm\sqrt{\vecdim}+\sgnorm\sqrt{\log\num}\right)\cdot\exp\left[-\frac{\snr^{2}}{2\conste}\right]
\end{align*}
for some universal constant $C_{0},\conste>0$ with probability at
least $1-n^{-C_{1}}$, where the last step holds since $\snr^{2}\ge\numclust$.
The fact that $e^{x}\ge1+x>x$ for any $x$ implies 
\[
\exp\left[-\frac{\snr^{2}}{4\conste}\right]\le\frac{4\conste}{\snr^{2}}=\frac{\sgnorm}{\minsep}\cdot\frac{4\conste}{\snr}\le4\conste\frac{\sgnorm}{\minsep}
\]
where the last step holds since we have $\snr\ge1$ by the conditions
of Theorem \ref{thm:cluster_error_rate}. Hence, we have 
\begin{align*}
Q_{1} & \le C_{0}\sgnorm\left(4\conste+\sqrt{\vecdim}+\sqrt{\log\num}\right)\cdot\exp\left[-\frac{\snr^{2}}{4\conste}\right]\\
 & \leq C_{1}\sgnorm\left(1+\sqrt{\vecdim}+\sqrt{\log\num}\right)\cdot\exp\left[-\frac{\snr^{2}}{4\conste}\right]\\
 & \leq2C_{1}\sgnorm\left(\sqrt{\vecdim}+\sqrt{\log\num}\right)\cdot\exp\left[-\frac{\snr^{2}}{4\conste}\right]
\end{align*}
where $C_{1}>0$ is a universal constant.

\subsection{Controlling $Q_{2}$}

We have 
\[
Q_{2}=\norm[\frac{1}{\size}\sum_{j\in\clustset a}\g_{j}]2.
\]
We see that $\frac{1}{\size}\sum_{j\in\clustset a}g_{ji}$ has variance
$\frac{1}{\size}\std^{2}$. By Proposition 2.6.1 on pp.$\ $28 and
Theorem 3.1.1 on pp. 41 of \citet{vershynin2017high}, $\frac{\sqrt{\size}}{\std}\norm[\frac{1}{\size}\sum_{j\in\clustset a}\g_{j}]2-\sqrt{\vecdim}$
is a sub-Gaussian random variable with sub-Gaussian norm at most $C_{\psi_{2}}\frac{\sgnorm^{2}}{\std^{2}}$
where $C_{\psi_{2}}>0$ is a universal constant. Then Lemma \ref{lem:hoeffding}
implies that 
\[
\P\left[\frac{\sqrt{\size}}{\std}\norm[\frac{1}{\size}\sum_{j\in\clustset a}\g_{j}]2-\sqrt{\vecdim}\ge C\frac{\sgnorm^{2}}{\std^{2}}\sqrt{\log\num}\right]\le\num^{-C'}
\]
for some universal constants $C,C'>0$. Since $\std\lesssim\sgnorm$,
there exists a universal constant $C_{0}>0$ such that 
\[
Q_{2}\leq C_{0}\sgnorm\left(\sqrt{\frac{\numclust\vecdim}{\num}}+\sqrt{\frac{\numclust\log\num}{\num}}\right)
\]
with probability at least $1-\num^{-C'}$. 

\section{Technical lemmas}

The following lemma is Theorem 2.6.2 on pp.$\ $28 in \citet{vershynin2017high}.
\begin{lem}[General Hoeffding's inequality]
\emph{ \label{lem:hoeffding} }Let $X_{1},\ldots,X_{N}$ be independent,
mean zero, sub-Gaussian random variables. Then, for every $t\geq0$
we have 
\[
\P\left[\left|\sum_{i=1}^{N}X_{i}\right|\geq t\right]\leq2\exp\left[-\frac{ct^{2}}{\sum_{i=1}^{N}\norm[X_{i}]{\psi_{2}}^{2}}\right],
\]
where $c>0$ is a universal constant.
\end{lem}
The following lemma is Exercise 4.7.3 in \citet{vershynin2017high}.
\begin{lem}[Tail bound of covariance matrix of sub-Gaussians]
\emph{ }\label{lem:subg_cov_mat_bound} Let $\x$ be a sub-Gaussian
vector and let $\x_{1},\ldots,\x_{m}$ be independent samples of $\x$.
Let $m$ be a positive integer and define 
\begin{align*}
\boldsymbol{\Sigma} & \coloneqq\E\x\x\t,\\
\boldsymbol{\Sigma}_{m} & \coloneqq\frac{1}{m}\sum_{i=1}^{m}\x_{i}\x_{i}\t.
\end{align*}
Let $\rho_{0}\ge1$ be such that 
\[
\norm[\left\langle \x,\w\right\rangle ]{\psi_{2}}\leq\rho_{0}\sqrt{\E\left\langle \x,\w\right\rangle ^{2}}\qquad\text{for any }\w\in\real^{N}.
\]
For any $u\geq0$, we have for a universal constant $C>0$, 
\[
\opnorm{\boldsymbol{\Sigma}_{m}-\boldsymbol{\Sigma}}\leq C\rho_{0}^{2}\left(\sqrt{\frac{N+u}{m}}+\frac{N+u}{m}\right)\opnorm{\boldsymbol{\Sigma}}
\]
with probability at least $1-2e^{-u}$.
\end{lem}






\end{document}
