\section{Appendix setup}
\label{sec:setup}
We will first provide a note on the organization of the appendix and follow that up with introducing the notations.

\subsection{Organization}
\label{ssec:org}
\begin{itemize}
\item In subsection~\ref{ssec:notations}, we will recall notation from the main paper and introduce some new notation that will be used across the appendix.
\item In section~\ref{sec:tailAverageIterateCovariance}, we will write out expressions that characterize the generalization error of the proposed accelerated SGD method. In order to bound the generalization error, we require developing an understanding of two terms namely the bias error and the variance error.
\item In section~\ref{sec:commonLemmas}, we prove lemmas that will be used in subsequent sections to prove bounds on the bias and variance error.
\item In section~\ref{sec:biasContraction}, we will bound the bias error of the proposed accelerated stochastic gradient method. In particular, lemma~\ref{lem:main-bias} is the key lemma that provides a new potential function with which this paper achieves acceleration. Further, lemma~\ref{lem:bound-bias} is the lemma that bounds all the terms of the bias error.
\item In section~\ref{sec:varianceContraction}, we will bound the variance error of the proposed accelerated stochastic gradient method. In particular, lemma~\ref{lem:main-variance} is the key lemma that considers a stochastic process view of the proposed accelerated stochastic gradient method and provides a sharp bound on the covariance of the stationary distribution of the iterates. Furthermore, lemma~\ref{lem:bound-variance} bounds all terms of the variance error.
\item Section~\ref{sec:proofMainTheorem} presents the proof of Theorem~\ref{thm:main}. In particular, this section aggregates the result of lemma~\ref{lem:bound-bias} (which bounds all terms of the bias error) and lemma~\ref{lem:bound-variance} (which bounds all terms of the variance error) to present the guarantees of Algorithm~\ref{algo:TAASGD}.
\end{itemize}

\subsection{Notations}
\label{ssec:notations}

We begin by introducing $\M$, which is the fourth moment tensor of the input $\a\sim\D$, i.e.:
\begin{align*}
\M\defeq\Eover{\distr}{\a\otimes\a\otimes\a\otimes\a}
\end{align*}
Applying the fourth moment tensor $\M$ to any matrix $\S\in\R^{d\times d}$ produces another matrix in $\R^{d\times d}$ that is expressed as: 
\begin{align*}
\M\S \defeq \E{(\a\T\S\a)\a\a\T}. 
\end{align*}
With this definition in place, we recall $\infbound$ as the smallest number, such that $\M$ applied to the identity matrix $\eye$ satisfies:
\begin{align*}
\M\eye=\E{\twonorm{\a}^2\a\a\T}\preceq\infbound\ \Cov
\end{align*}
%This in particular implies $\E{\twonorm{\a}^2}\leq\infbound$. \praneeth{The above statement is not true. The implication is in the other direction.}
Moreover, we recall that the condition number of the distribution $\cnH = \infbound/\mu$, where $\mu$ is the smallest eigenvalue of $\Cov$. Furthermore, the definition of the statistical condition number $\cnS$ of the distribution follows by applying the fourth moment tensor $\M$ to $\Covinv$, i.e.:
\begin{align*}
\M\Covinv&=\E{(\a\T\Covinv\a)\cdot\a\a\T}\preceq\cnS\ \H
\end{align*}

We denote by $\tensor{A}_{\mathcal{L}}$ and $\tensor{A}_{\mathcal{R}}$ the left and right multiplication operator of any matrix $\A\in\R^{d\times d}$, i.e. for any matrix $\S\in\R^{d\times d}$, $\tensor{A}_{\mathcal{L}}\S=\A\S$ and $\tensor{A}_{\mathcal{R}}\S=\S\A$.

\underline{\bf Parameter choices:} In all of appendix we choose the parameters in Algorithm~\ref{algo:TAASGD} as
\begin{align*}
\alpha = \frac{\sqrt{\cnH\cnHh}}{\ctwo\sqrt{2\cone-\cone^2}+\sqrt{\cnH\cnHh}},\ \  \beta = \cthree\frac{\ctwo\sqrt{2\cone-\cone^2}}{\sqrt{\cnH\cnHh}},\ \  \gamma = \ctwo\frac{\sqrt{2\cone-\cone^2}}{\mu\sqrt{\cnH\cnHh}}, \ \ \delta=\frac{\cone}{\infbound}
\end{align*}
%where $\cone$ is an arbitrary constant satisfying $0 < \cone < \frac{1}{2}$. Note that we recover Theorems~\ref{thm:main} and~\ref{thm:par} by choosing $\cone = \frac{1}{8}$.
where $\cone$ is an arbitrary constant satisfying $0 < \cone < \frac{1}{2}$. Furthermore, we note that $\cthree=\frac{\ctwo\sqrt{2\cone-\cone^2}}{\cone}$, $\ctwo^2=\frac{\cfour}{2-\cone}$ and $\cfour< 1/6$. 
Note that we recover Theorem~\ref{thm:main} by choosing $\cone = 1/5, \ctwo = \sqrt{5}/9, \cthree=\sqrt{5}/3, \cfour = 1/9$. We denote 
\begin{align*}
c\defeq\alpha(1-\beta) \text{ and, } \g\defeq\alpha\delta+(1-\alpha)\gamma.
\end{align*}

Recall that $\xs$ denotes unique minimizer of $P(\x)$, i.e. $\xs=\arg\min_{\x \in \R^d} \Eover{\distr}{(b-\iprod{\x}{\a})^2}$. We track $\thetav_k=\begin{bmatrix}\x_k-\xs\\ \yt[k]-\xs\end{bmatrix}$. The following equation captures the updates of Algorithm~\ref{algo:TAASGD}:
\begin{align}
\label{eq:mainRec}
\thetav_{k+1}&=\begin{bmatrix}0&\eye-\delta\widehat{\H}_{k+1}\\-c\cdot\eye&(1+c)\cdot\eye-\g\cdot\widehat{\H}_{k+1}\end{bmatrix}\thetav_k
+\begin{bmatrix}\delta\cdot\epsilon_{k+1}\av_{k+1}\\\g\cdot\epsilon_{k+1}\av_{k+1}\end{bmatrix}\nonumber\\
&\defeq\Ah_{k+1}\thetav_{k}+\zetav_{k+1},
\end{align}
where, $\widehat{\H}_{k+1} \defeq \av_{k+1} \av_{k+1}^\top$, $\widehat{\A}_{k+1} \defeq \begin{bmatrix}0&\eye-\delta\widehat{\H}_{k+1}\\-c\cdot\eye&(1+c)\cdot\eye-\g\cdot\widehat{\H}_{k+1}\end{bmatrix}$ 
and $\zetav_{k+1} \defeq \begin{bmatrix}\delta\cdot\epsilon_{k+1}\av_{k+1}\\\g\cdot\epsilon_{k+1}\av_{k+1}\end{bmatrix}$.

\noindent Furthermore, we denote by $\phiv_k$ the expected covariance of $\thetav_k$, i.e.:
\begin{align*}
\phiv_k\defeq\E{\thetav_k\otimes\thetav_k}.
\end{align*}
Next, let $\mathcal{F}_k$ denote the filtration generated by samples $\{(\a_1,b_1),\cdots, (\a_k,b_k)\}$. Then,
\begin{align*}
\A&\eqdef \E{\Ah_{k+1}|\mathcal{F}_{k}}=\begin{bmatrix}
\zero & \eye - \delta \Cov \\ -c\eye & (1+c)\eye - \g\Cov
\end{bmatrix}.
\end{align*}
By iterated conditioning, we also have
\begin{align}\label{eqn:theta-det}
	\E{\thetav_{k+1}\middle \vert \mathcal{F}_{k}} = \A \thetav_k.
\end{align}
Without loss of generality, we assume that $\Cov$ is a diagonal matrix. We now note that we can rearrange the coordinates through an eigenvalue decomposition so that $\A$ becomes a block-diagonal matrix with $2\times2$ blocks. We denote the $j^{\textrm{th}}$ block by $\A_j$:
\begin{align*}
\A_j \eqdef \begin{bmatrix}
0 & 1 - \delta \lambda_j \\ -c & 1+c - \g \lambda_j
\end{bmatrix},
\end{align*}
where $\lambda_j$ denotes the $j^{\textrm{th}}$ eigenvalue of $\Cov$.
Next, 
\begin{align*}
\BT&\eqdef \E{\Ah_{k+1}\otimes\Ah_{k+1}|\mathcal{F}_{k}}, \mbox{ and }\\
\Sigh&\eqdef \E{\zetav_{k+1}\otimes\zetav_{k+1}|\mathcal{F}_k} = \begin{bmatrix}\delta^2&\delta\cdot \g\\\delta\cdot \g&\g^2\end{bmatrix}\otimes\Sig\preceq\sigma^2\cdot\begin{bmatrix}\delta^2&\delta\cdot \g\\\delta\cdot \g&\g^2\end{bmatrix}\otimes\H.
\end{align*}
Finally, we observe the following:
\begin{align*}
\E{(\A-\Ah_{k+1})\otimes(\A-\Ah_{k+1})|\mathcal{F}_k}&=\A\otimes\A-\E{\Ah_{k+1}\otimes\A|\mathcal{F}_k}\\&\quad\quad-\E{\Ah_{k+1}\otimes\A|\mathcal{F}_k}+\E{\Ah_{k+1}\otimes\Ah_{k+1}|\mathcal{F}_k}\\
&=-\A\otimes\A+\E{\Ah_{k+1}\otimes\Ah_{k+1}|\mathcal{F}_k}\\
\implies \E{\Ah_{k+1}\otimes\Ah_{k+1}|\mathcal{F}_k}&=\E{(\A-\Ah_{k+1})\otimes(\A-\Ah_{k+1})|\mathcal{F}_k}+\A\otimes\A
\end{align*}
We now define:
\begin{align*}
\RT&\defeq\E{(\A-\Ah_{k+1})\otimes(\A-\Ah_{k+1})|\mathcal{F}_k}, \mbox{ and }\\
\DT&\defeq\A\otimes\A.
\end{align*}
Thus implying the following relation between the operators $\BT,\DT$ and $\RT$:
\begin{align*}
\BT=\DT+\RT.
\end{align*}


\section{The Tail-Average Iterate: Covariance and bias-variance decomposition}
\label{sec:tailAverageIterateCovariance}
%\begin{proof}[Proof of Lemma~\ref{lem:average-covar}]
We begin by considering the first-order Markovian recursion as defined by equation~\ref{eq:mainRec}:
\begin{align*}
\thetav_j&=\Ah_j\thetav_{j-1}+\zetav_j.
\end{align*}
We refer by $\phiv_j$ the covariance of the $j^{\text{th}}$ iterate, i.e.:
\begin{align}
\label{eq:finalIterateCovariance}
\phiv_j \defeq \E{\thetav_j\otimes\thetav_j}
\end{align}
Consider a decomposition of $\thetat$ as $\thetat = \thetat^{\textrm{bias}} + \thetat^{\textrm{variance}}$, where $\thetat^{\textrm{bias}}$ and $\thetat^{\textrm{variance}}$ are defined as follows:
\begin{align}
	\thetat^{\textrm{bias}} \eqdef \Ahatj \thetat[j-1]^{\textrm{bias}} &;\qquad  \thetat[0]^{\textrm{bias}} \eqdef \thetat[0], \mbox{ and } \label{eq:biasRec}\\
	\thetat^{\textrm{variance}} \eqdef \Ahatj \thetat[j-1]^{\textrm{variance}} +\zetat &;\qquad  \thetat[0]^{\textrm{variance}} \eqdef \zero. \label{eq:varianceRec}
\end{align}
We note that 
\begin{align}
\E{\thetav_j^{\textrm{bias}}}=\A\E{\thetav_{j-1}^{\textrm{bias}}}\label{eq:condExpBias},\\
\E{\thetav_j^{\textrm{variance}}}=\A\E{\thetav_{j-1}^{\textrm{variance}}}\label{eq:condExpVar}.
\end{align}
Note equation~\ref{eq:condExpVar} follows using a conditional expectation argument with the fact that $\E{\zetav_k}=0\ \forall\ k$ owing to first order optimality conditions.

Before we prove the decomposition holds using an inductive argument, let us understand what the bias and variance sub-problem intuitively mean. 

Note that the {\em bias} sub-problem (defined by equation~\ref{eq:biasRec}) refers to running algorithm on the noiseless problem (i.e., where, $\zetav_{\Bigcdot}=0$ a.s.) by starting it at $\thetav_0^{\textrm{bias}}=\thetav_0$. The bias essentially measures the dependence of the generalization error on the excess risk of the initial point $\thetav_0$ and bears similarities to convergence rates studied in the context of offline optimization. 

The {\em variance} sub-problem (defined by equation~\ref{eq:varianceRec}) measures the dependence of the generalization error on the noise introduced during the course of optimization, and this is associated with the statistical aspects of the optimization problem. The variance can be understood as starting the algorithm at the solution ($\thetav_0^{\text{variance}}=0$) and running the optimization driven solely by noise. Note that the variance is associated with sharp statistical lower bounds which dictate its rate of decay as a function of the number of oracle calls $n$.

Now, we will prove that the decomposition $\thetat = \thetat^{\textrm{bias}} + \thetat^{\textrm{variance}}$ captures the recursion expressed in equation~\ref{eq:mainRec} through induction. For the base case $j=1$, we see that 
\begin{align*}
\thetav_1&=\Ah_1\thetav_0+\zetav_1\\
&=\underbrace{\Ah_1\thetav_0^{\textrm{bias}}}_{\because\ \thetav_0^{\textrm{bias}}=\thetav_0}+\underbrace{\Ah_1\thetav_0^{\textrm{variance}}}_{=0 ,\ \because\ \thetav_0^{\textrm{variance}}=0}+\zetav_1\\
&=\thetav_1^{\textrm{bias}}+\thetav_1^{\textrm{variance}}
\end{align*}
Now, for the inductive step, let us assume that the decomposition holds in the $j-1^{st}$ iteration, i.e. we assume $\thetav_{j-1}=\thetav_{j-1}^{\textrm{bias}}+\thetav_{j-1}^{\textrm{variance}}$. We will then prove that this relation holds in the $j^{th}$ iteration. Towards this, we will write the recursion:
\begin{align*}
\thetav_j&=\Ah_j\thetav_{j-1}+\zetav_j\\
&=\Ah_j(\thetav_{j-1}^{\textrm{bias}}+\thetav_{j-1}^{\textrm{variance}})+\zetav_j\quad\text{(using the inductive hypothesis)}\\
&=\Ah_j\thetav_{j-1}^{\textrm{bias}}+\Ah_j\thetav_{j-1}^{\textrm{variance}}+\zetav_j\\
&=\thetav_j^{\textrm{bias}}+\thetav_j^{\textrm{variance}}.
\end{align*}
This proves the decomposition holds through a straight forward inductive argument.
\iffalse
We note that $\thetat^{\textrm{bias}}$ is obtained by beginning at $\thetav_0^{\textrm{bias}}=\thetav_0$ and running the algorithm on the noiseless problem (i.e. where $\zetav_j=0$ a.s.). On the other hand, $\thetat^{\textrm{variance}}$ is obtained by beginning at the solution $\thetav^{\textrm{variance}}_0=0$, and allowing the noise $\zetav_j$ to drive the process. 
\fi

In a similar manner as $\thetat$, the tail-averaged iterate $\thetavb \eqdef \frac{1}{n-t} \sum_{j=t+1}^n \thetat[j]$ can also be written as $\thetavb = \thetavb^{\textrm{bias}} + \thetavb^{\textrm{variance}}$, where $\thetavb^{\textrm{bias}} \eqdef \frac{1}{n-t} \sum_{j=t+1}^n \thetat[j]^{\textrm{bias}}$ and $\thetavb^{\textrm{variance}} \eqdef \frac{1}{n-t} \sum_{j=t+1}^n \thetat[j]^{\textrm{variance}}$. Furthermore, the tail-averaged iterate $\thetavb$ and its bias and variance counterparts $\thetavb^{\text{bias}},\thetavb^{\text{variance}}$ are associated with their corresponding covariance matrices $\phivb_{t,n},\phivb_{t,n}^{\text{bias}},\phivb_{t,n}^{\text{variance}}$ respectively. Note that $\phivb_{t,n}$ can be upper bounded using Cauchy-Shwartz inequality as:
\begin{align}\label{eq:tailAvgCovarBound}
		\E{\thetavb \otimes \thetavb} &\preceq 2\cdot\bigg( \E{\thetavb^{\text{bias}} \otimes \thetavb^{\text{bias}}} + \E{\thetavb^{\text{variance}} \otimes \thetavb^{\text{variance}}}\bigg)\nonumber\\
\implies\phivb_{t,n}&\preceq 2\cdot(\phivb_{t,n}^{\text{bias}}+\phivb_{t,n}^{\text{variance}}).
\end{align}
The above inequality is referred to as the {\em bias-variance} decomposition and is well known from previous work~\cite{BachM13,FrostigGKS15,JainKKNS16}, and we re-derive this decomposition for the sake of completeness.
\iffalse
We begin by considering the first-order Markovian recursion as defined by equation~\ref{eq:mainRec}:
\begin{align*}
\thetav_k&=\Ah_k\thetav_{k-1}+\zetav_k.
\end{align*}
The equation above can be unrolled over $k$ steps to write the iterate $\thetav_k$ in terms of the initial conditions $\thetav_0$ and the noise introduced until the $k^{\text{th}}$ iteration, (i.e. $\zetav_j$, $j=1,2,..,k$), and this yields:
\begin{align*}
\thetav_k&=\big(\prod_{l=k}^1\Ah_l\big)\thetav_0 + \sum_{j=1}^k\big(\prod_{l=k}^{j+1}\Ah_l\big)\zetav_j
\end{align*}
Where, $\prod_{l=k}^{k+1}\Ah_l=\eye$. We refer by $\phiv_k$ the covariance of the $k^{\text{th}}$ iterate, i.e.:
\begin{align}
\label{eq:finalIterateCovariance}
\phiv_k \defeq \E{\thetav_k\otimes\thetav_k}
\end{align}
Now, we consider the tail-averaged iterate $\thetavb$:
\begin{align*}
\thetavb&\defeq\frac{1}{n-t}\sum_{k=t+1}^n\thetav_k\\
&=\frac{1}{n-t}\sum_{k=t+1}^n\bigg[ \big(\prod_{l=k}^1\Ah_l\big)\thetav_0 + \sum_{j=1}^k\big(\prod_{l=k}^{j+1}\Ah_l\big)\zetav_j\bigg]\\
&=\frac{1}{n-t}\bigg( \bigg[\sum_{k=t+1}^n\big(\prod_{l=k}^1\Ah_l\big)\bigg]\thetav_0 + \bigg[\sum_{k=t+1}^n\sum_{j=1}^k\big(\prod_{l=k}^{j+1}\Ah_l\big)\zetav_j\bigg]\bigg)
\end{align*}
Which now allows us to define the covariance of the tail-averaged iterate $\phivb_{t,n}$:
\begin{align}
\label{eq:tailAvgCovarBound}
\phivb_{t,n}&=\E{\thetavb\otimes\thetavb}\nonumber\\
&\preceq\frac{2}{(n-t)^2}\cdot\bigg(\E{ \otimes_2\bigg( \bigg[\sum_{k=t+1}^n\big(\prod_{l=k}^1\Ah_l\big)\bigg]\thetav_0\bigg) + \otimes_2\bigg(\bigg[\sum_{k=t+1}^n\sum_{j=1}^l\big(\prod_{l=k}^{j+1}\Ah_l\big)\zetav_j\bigg]\bigg)} \bigg)\nonumber\\
&\defeq 2\cdot(\phivb_{t,n}^{\text{bias}} + \phivb_{t,n}^{\text{variance}}),
\end{align}
where, $\phivb_{t,n}^{\text{bias}}=\E{\thetavb^{\textrm{bias}}\otimes\thetavb^{\textrm{bias}}}$, $\phivb_{t,n}^{\text{variance}}=\E{\thetavb^{\textrm{variance}}\otimes\thetavb^{\textrm{variance}}}$.
Note that we applied the Cauchy-Shwartz inequality for any pair of vectors $\v_1,\v_2$ so that, $\v_1\otimes\v_2+\v_2\otimes\v_1\preceq\v_1\otimes\v_1+\v_2\otimes\v_2$.  Note that equation~\ref{eq:tailAvgCovarBound} clearly indicates that the generalization error exhibits a {\em bias-variance} decomposition, which is described below.

The {\em bias} refers to the dependence of the generalization error on the starting point $\thetav_0$, and bears similarities to convergence rates studied in the context of offline optimization; this can be understood as running the optimization on the {\em noiseless} problem, i.e., where, $\zetav_{\Bigcdot}=0$ a.s. In particular, this sub-problem is characterized by the following first-order recursion:
\begin{align}
\label{eq:biasRec}
\thetav_k^{\text{bias}} = \Ah_k\thetav_{k-1}^{\text{bias}},\quad \thetav_0^{\text{bias}}=\thetav_0.
\end{align} 
The {\em variance} refers to the dependence of the generalization error on the noise introduced during the course of optimization, and this is associated with the statistical aspects of the optimization problem. The variance can be understood as starting the algorithm at the solution ($\thetav_0^{\text{variance}}=0$) and running the optimization driven solely by noise. Note that the variance is associated with sharp statistical lower bounds which dictate its rate of decay as a function of the number of oracle calls $n$.
\begin{align}
\label{eq:varianceRec}
\thetav_k^{\text{variance}} = \Ah_k\thetav_{k-1}^{\text{variance}} + \zetav_k, \quad \thetav_0^{\text{variance}}=0.
\end{align} 
\iffalse
We begin by considering the first-order Markovian recursion as defined by equation~\ref{eq:mainRec}:
\begin{align*}
\thetav_t&=\Ah_t\thetav_{t-1}+\zetav_t.
\end{align*}
By unrolling the recursion above, we have:
\begin{align*}
\thetav_t&=\prod_{j=t}^{1}\Ah_j\cdot\thetav_0+\sum_{l=1}^t\bigg(\prod_{j=t}^{l+1}\Ah_j\bigg)\zetav_l,
\end{align*}
where $\prod_{j=t}^{t+1}\Ah_j \defeq \eye$. Then, we begin by considering the covariance of the final iterate, i.e. $\phiv_t\defeq\E{\thetav_t\otimes\thetav_t}$, which leads us to the first sub-section, which is necessary in order to obtain an expression for the covariance of the tail-averaged iterate. 
\subsection{Covariance of the final iterate and the bias-variance decomposition}
\begin{align}
\label{eq:lpRec}
\phiv_t&=\E{\thetav_t\otimes\thetav_t}\nonumber\\
&=\E{\bigg(\prod_{j=t}^{1}\Ah_j\cdot\thetav_0+\sum_{l=1}^t\bigg(\prod_{j=t}^{l+1}\Ah_j\bigg)\zetav_l\bigg)\otimes\bigg(\prod_{j=t}^{1}\Ah_j\cdot\thetav_0+\sum_{l=1}^t\bigg(\prod_{j=t}^{l+1}\Ah_j\bigg)\zetav_l\bigg)}\nonumber\\
&\preceq2\cdot\E{\bigg(\prod_{j=t}^{1}\Ah_j\bigg)\cdot\thetav_0\otimes\thetav_0\cdot\bigg(\prod_{j=1}^{t}\Ah_j\T\bigg)}
+2\cdot\E{\bigg(\sum_{l=1}^t\big(\prod_{j=t}^{l+1}\Ah_j\big)\zetav_l\bigg)\otimes\bigg(\sum_{m=1}^t\big(\prod_{j'=t}^{m+1}\Ah_{j'}\big)\zetav_m\bigg)},
\end{align}
where the last step follows from the inequality $\u \otimes \v + \v \otimes \u \preceq \u \otimes \u + \v \otimes \v$. 

Equation~\ref{eq:lpRec} implies the covariance of the iterate $\thetav_t$ is sharply characterized by analyzing two sub-problems: (a) the first term of equation~\ref{eq:lpRec}, corresponding to the noiseless problem, also referred to as the {\em bias} risk, and (b) the second term of equation~\ref{eq:lpRec}, corresponding to the problem where we begin at the solution $\thetav^*$ and allow the noise to drive the random process, also referred to as the {\em variance} risk.

Owing to this decomposition, we will write out the covariance of the excess risk $\phiv_t$ in terms of the bias error covariance $\phiv_t^{\text{bias}}$ and the variance error covariance $\phiv_t^{\text{variance}}$:
\begin{align}
\label{eq:bvdecomp}
\phiv_t &\preceq 2\cdot\phiv_t^{\text{bias}} + 2\cdot\phiv_t^{\text{variance}}
\end{align}
where,
\begin{align}
\label{eq:biasLP}
\phiv_t^{\text{bias}}&\eqdef \E{\bigg(\prod_{j=t}^{1}\Ah_j\bigg)\cdot\thetav_0\otimes\thetav_0\cdot\bigg(\prod_{j=1}^{t}\Ah_j\T\bigg)}\nonumber\\
&=\E{\E{\Ah_t\bigg(\prod_{j=t-1}^{1}\Ah_j\bigg)\cdot\thetav_0\otimes\thetav_0\cdot\bigg(\prod_{j=1}^{t-1}\Ah_j\T\bigg)\Ah_t\T\bigg{\vert}\mathcal{F}_{t-1}}}\nonumber\\
&=\BT\cdot\E{\bigg(\prod_{j=t-1}^{1}\Ah_j\bigg)\cdot\thetav_0\otimes\thetav_0\cdot\bigg(\prod_{j=1}^{t-1}\Ah_j\T\bigg)}\nonumber\\
&=\BT^t\cdot\phiv_0.
\end{align}
Next, for the covariance of the variance error, we have:
\begin{align}
\label{eq:varianceLP}
\phiv_t^{\text{variance}}&\eqdef \E{\bigg(\sum_{l=1}^t\big(\prod_{j=t}^{l+1}\Ah_j\big)\zetav_l\bigg)\otimes\bigg(\sum_{m=1}^t\big(\prod_{j'=t}^{m+1}\Ah_{j'}\big)\zetav_m\bigg)}\nonumber\\
&=\sum_{l=1}^t\E{\big(\prod_{j=t}^{l+1}\Ah_j\big)\zetav_l\otimes\zetav_l\big(\prod_{j'=l+1}^{t}\Ah_{j'}\T\big)} \left(\mbox{cross terms are zero since } \E{\zetav_l\middle \vert \mathcal{F}_{l-1}} = 0\right)\nonumber\\
&=\sum_{l=1}^t\E{\E{\Ah_t\bigg(\prod_{j=t-1}^{l+1}\Ah_j\bigg)\cdot\zetav_l\otimes\zetav_l\cdot\bigg(\prod_{j=l+1}^{t-1}\Ah_j\T\bigg)\Ah_t\T\bigg{\vert}\mathcal{F}_{t-1}}}\nonumber\\
&=\sum_{l=1}^t\BT\cdot\E{\bigg(\prod_{j=t-1}^{l+1}\Ah_j\bigg)\cdot\zetav_l\otimes\zetav_l\cdot\bigg(\prod_{j=l+1}^{t-1}\Ah_j\T\bigg)}\nonumber\\
&=\sum_{l=1}^t\BT^{t-l}\cdot\E{\zetav_l\otimes\zetav_l}=\sum_{l=1}^t\BT^{t-l}\Sigh=(\eyeT-\BT)^{-1}(\eyeT-\BT^{l+1})\Sigh,
\end{align}
where the last step assumes that all the eigenvalues of $\BT$ are smaller than $1$, which follows from Lemma~\ref{lem:main-bias}.
%Where we note that the cross-terms involving noise $\zetav_l\otimes\zetav_k$ with $k\ne l$ turning out to be zero owing to straight forward use of the tower rule of conditional expectations and by using first order optimality conditions. %Furthermore, the conditional expectation arguments are straightforward given that the noise $\zetav_l\otimes\zetav_l$ are hit by updates from iteration $l+1$ to $t$.
\fi
\fi
We will now derive an expression for the covariance of the tail-averaged iterate and apply it to obtain the covariance of the bias ($\phivb_{t,n}^{\text{bias}}$) and variance ($\phivb_{t,n}^{\text{variance}}$) error of the tail-averaged iterate.
\subsection{The tail-averaged iterate and its covariance}
We begin by writing out an expression for the tail-averaged iterate $\thetavb$ as: 
\begin{align*}
\thetavb=\frac{1}{n-t}\sum_{j=t+1}^n\thetav_j
\end{align*}
To get the excess risk of the tail-averaged iterate $\thetavb$, we track its covariance $\phivb_{t,n}$:
\begin{align}
\label{eq:tailAvgCovar}
\phivb_{t,n}&=\E{\thetavb\otimes\thetavb}\nonumber\\
&=\frac{1}{(n-t)^2}\sum_{j,l=t+1}^n\E{\thetav_j\otimes\thetav_l}\nonumber\\
&=\frac{1}{(n-t)^2}\sum_j\left(\sum_{l=t+1}^{j-1}\E{\thetav_j\otimes\thetav_l}+\E{\thetav_j\otimes\thetav_j}+\sum_{l=j+1}^n\E{\thetav_j\otimes\thetav_l}\right)\nonumber\\
&=\frac{1}{(n-t)^2}\sum_j\left(\sum_{l=t+1}^{j-1}\A^{j-l}\E{\thetav_l\otimes\thetav_l}+\E{\thetav_j\otimes\thetav_j}+\sum_{l=j+1}^n\E{\thetav_j\otimes\thetav_j}(\A\T)^{l-j}\right) \left(\mbox{ from~\eqref{eqn:theta-det}}\right) \nonumber\\
&=\frac{1}{(n-t)^2}\bigg(\sum_{l=t+1}^n\sum_{j=l+1}^n\A^{j-l}\E{\thetav_l\otimes\thetav_l} + \sum_{j=t+1}^n\E{\thetav_j\otimes\thetav_j}+\sum_{j=t+1}^n\sum_{l=j+1}^n\E{\thetav_j\otimes\thetav_j}(\A\T)^{l-j}\bigg)\nonumber\\
&=\frac{1}{(n-t)^2}\bigg(\sum_{j=t+1}^n\sum_{l=j+1}^n\A^{l-j}\E{\thetav_j\otimes\thetav_j} + \sum_{j=t+1}^n\E{\thetav_j\otimes\thetav_j}+\sum_{j=t+1}^n\sum_{l=j+1}^n\E{\thetav_j\otimes\thetav_j}(\A\T)^{l-j}\bigg)\nonumber\\
&=\frac{1}{(n-t)^2}\bigg(\sum_{j=t+1}^n(\eye-\A)^{-1}(\A-\A^{n+1-j})\E{\thetav_j\otimes\thetav_j} + \sum_{j=t+1}^n\E{\thetav_j\otimes\thetav_j}\nonumber\\&\quad\quad\quad\quad\quad\quad +\sum_{j=t+1}^n\E{\thetav_j\otimes\thetav_j}(\eye-\A\T)^{-1}(\A\T-(\A\T)^{n+1-j})\bigg)\nonumber\\
&=\frac{1}{(n-t)^2}\sum_{j=t+1}^n\bigg( \eyeT + (\eyeT-\AL)^{-1}(\AL-\AL^{n+1-j}) + (\eyeT-\AR\T)^{-1}(\AR\T-(\AR\T)^{n+1-j}) \bigg)\E{\thetav_j\otimes\thetav_j}\nonumber\\
&=\frac{1}{(n-t)^2}\sum_{j=t+1}^n\bigg( \eyeT + (\eyeT-\AL)^{-1}(\AL-\AL^{n+1-j}) + (\eyeT-\AR\T)^{-1}(\AR\T-(\AR\T)^{n+1-j}) \bigg)\phiv_j.
\end{align}
%We can now use the bias-variance decomposition from equation~\ref{eq:bvdecomp} to express $\phiv_j\preceq2\cdot\phiv_j^{\text{bias}}+2\cdot\phiv_j^{\text{variance}}$. 
\iffalse
\praneeth{The below argument is not correct.}
By applying equation~\ref{eq:bvdecomp} to equation~\ref{eq:tailAvgCovar}, we have a bias-variance decomposition for the error of the tail-averaged iterate $\thetavb$. In particular, we have:
\begin{align}
\label{eq:tailAvgCovarBound}
\phivb_{t,n}\preceq 2\cdot\phivb_{t,n}^{\text{bias}}+2\cdot\phivb_{t,n}^{\text{variance}}
\end{align}
Where, we will obtain $\phivb_{t,n}^{\text{bias}}$ and $\phivb_{t,n}^{\text{variance}}$ (defined below) in the following subsections: %using equation~\ref{eq:biasLP} and equation~\ref{eq:varianceLP}, 
\fi
Note that the above recursion can be applied to obtain the covariance of the tail-averaged iterate for the bias ($\phivb_{t,n}^{\text{bias}}$) and variance ($\phivb_{t,n}^{\text{variance}}$) error, since the conditional expectation arguments employed in obtaining equation~\ref{eq:tailAvgCovar} are satisfied by both the recursion used in tracking the bias error (i.e. equation~\ref{eq:biasRec}) and the variance error (i.e. equation~\ref{eq:varianceRec}). This implies that,
\begin{align}
\label{eq:biasTailAvg1}
\phivb_{t,n}^{\text{bias}}&\defeq\frac{1}{(n-t)^2}\sum_{j=t+1}^n\bigg( \eyeT + (\eyeT-\AL)^{-1}(\AL-\AL^{n+1-j}) + (\eyeT-\AR\T)^{-1}(\AR\T-(\AR\T)^{n+1-j}) \bigg)\phiv_j^{\text{bias}}
\end{align}
\begin{align}
\label{eq:varianceTailAvg1}
\phivb_{t,n}^{\text{variance}}&\defeq\frac{1}{(n-t)^2}\sum_{j=t+1}^n\bigg( \eyeT + (\eyeT-\AL)^{-1}(\AL-\AL^{n+1-j}) + (\eyeT-\AR\T)^{-1}(\AR\T-(\AR\T)^{n+1-j}) \bigg)\phiv_j^{\text{variance}}
\end{align}
\subsection{Covariance of Bias error of the tail-averaged iterate}
\begin{proof}[Proof of Lemma~\ref{lem:average-covar-bias}]
To obtain the covariance of the bias error of the tail-averaged iterate, we first need to obtain $\phiv_j^{\text{bias}}$, which we will by unrolling the recursion of equation~\ref{eq:biasRec}:
\begin{align}
\label{eq:biasLP}
\thetav_k^{\text{bias}}&= \Ah_k\thetav_{k-1}^{\text{bias}}\nonumber\\
\implies \phiv_{k}^{\text{bias}} &=\E{\thetav_k^{\text{bias}}\otimes\thetav_k^{\text{bias}}}\nonumber\\
&=\E{\E{\thetav_k^{\text{bias}}\otimes\thetav_k^{\text{bias}}|\mathcal{F}_{k-1}}}\nonumber\\
&=\E{\E{\Ah_k\thetav_{k-1}^{\text{bias}}\otimes\thetav_{k-1}^{\text{bias}}\Ah_k\T|\mathcal{F}_{k-1}}}\nonumber\\
&=\BT\ \E{\thetav_{k-1}^{\text{bias}}\otimes\thetav_{k-1}^{\text{bias}}}=\BT\ \phiv_{k-1}^{\text{bias}}\nonumber\\
\implies\phiv_{k}^{\text{bias}}&=\BT^k\ \phiv_{0}^{\text{bias}}
\end{align}
Next, we recount the equation for the covariance of the bias of the tail-averaged iterate from equation~\ref{eq:biasTailAvg1}:
\begin{align*}
\phivb_{t,n}^{\text{bias}}&=\frac{1}{(n-t)^2}\sum_{j=t+1}^n\bigg( \eyeT + (\eyeT-\AL)^{-1}(\AL-\AL^{n+1-j}) + (\eyeT-\AR\T)^{-1}(\AR\T-(\AR\T)^{n+1-j}) \bigg)\phiv_j^{\text{bias}}
\end{align*}
Now, we substitute $\phiv_j^{\text{bias}}$ from equation~\ref{eq:biasLP}:
\begin{align}
\label{eq:biasTA}
\phivb_{t,n}^{\text{bias}}&=\frac{1}{(n-t)^2}\sum_{j=t+1}^n\bigg( \eyeT + (\eyeT-\AL)^{-1}(\AL-\AL^{n+1-j}) + (\eyeT-\AR\T)^{-1}(\AR\T-(\AR\T)^{n+1-j}) \bigg)\BT^j\phiv_0\nonumber\\
&=\frac{1}{(n-t)^2}\sum_{j=t+1}^n\bigg( \eyeT + (\eyeT-\AL)^{-1}\AL + (\eyeT-\AR\T)^{-1}\AR\T\bigg)\BT^j\phiv_0\nonumber\\
&\qquad-\frac{1}{(n-t)^2}\sum_{j=t+1}^n\bigg( (\eyeT-\AL)^{-1}\AL^{n+1-j} + (\eyeT-\AR\T)^{-1}(\AR\T)^{n+1-j} \bigg)\BT^j\phiv_0\nonumber\\
&=\underbrace{\frac{1}{(n-t)^2}\bigg( \eyeT + (\eyeT-\AL)^{-1}\AL + (\eyeT-\AR\T)^{-1}\AR\T\bigg)(\eyeT-\BT)^{-1}(\BT^{t+1}-\BT^{n+1})\phiv_0}_{\text{Leading order term}}\nonumber\\
&\qquad-\frac{1}{(n-t)^2}\sum_{j=t+1}^n\bigg( (\eyeT-\AL)^{-1}\AL^{n+1-j} + (\eyeT-\AR\T)^{-1}(\AR\T)^{n+1-j} \bigg)\BT^j\phiv_0.
\end{align}
There are two points to note here: (a) The second line consists of terms that constitute the lower-order terms of the bias. We will bound the summation by taking a supremum over $j$. (b) Note that the burn-in phase consisting of $t$ unaveraged iterations allows for a geometric decay of the bias, followed by the tail-averaged phase that allows for a sublinear rate of bias decay. 
\end{proof}
\subsection{Covariance of Variance error of the tail-averaged iterate}
\begin{proof}[Proof of Lemma~\ref{lem:average-covar-var}]
Before obtaining the covariance of the tail-averaged iterate, we note that  $\E{\thetav_j^{\text{variance}}}=0\ \forall\ j$. This can be easily seen since $\thetav_0^{\textrm{variance}}=0$ and $\E{\thetav_k^{\textrm{variance}}}=\A\E{\thetav_{k-1}^{\textrm{variance}}}$ (from equation~\ref{eq:condExpVar}).

Next, in order to obtain the covariance of the variance of the tail-averaged iterate, we first need to obtain $\phiv_j^{\text{variance}}$, and we will obtain this by unrolling the recursion of equation~\ref{eq:varianceRec}:
\begin{align}
\label{eq:varianceLP}
\thetav_k^{\text{variance}}&= \Ah_k\thetav_{k-1}^{\text{variance}}+\zetav_k\nonumber\\
\implies \phiv_{k}^{\text{variance}} &=\E{\thetav_k^{\text{variance}}\otimes\thetav_k^{\text{variance}}}\nonumber\\
&=\E{\E{\thetav_k^{\text{variance}}\otimes\thetav_k^{\text{variance}}|\mathcal{F}_{k-1}}}\nonumber\\
&=\E{\E{\Ah_k\thetav_{k-1}^{\text{variance}}\otimes\thetav_{k-1}^{\text{variance}}\Ah_k\T+\zetav_k\otimes\zetav_k|\mathcal{F}_{k-1}}}\nonumber\\
&=\BT\ \E{\thetav_{k-1}^{\text{variance}}\otimes\thetav_{k-1}^{\text{variance}}}+\Sigh=\BT\ \phiv_{k-1}^{\text{variance}}+\Sigh\nonumber\\
\implies\phiv_{k}^{\text{variance}}&=\sum_{j=0}^{k-1}\BT^j\ \Sigh\nonumber\\
&=(\eye-\BT)^{-1}(\eyeT-\BT^k)\Sigh
\end{align}
Note that the cross terms in the outer product computations vanish owing to the fact that $\E{\thetav_{k-1}^{\textrm{variance}}}=0\ \forall\ k$. We then recount the expression for the covariance of the variance error from equation~\ref{eq:varianceTailAvg1}:
\begin{align*}
\phivb_{t,n}^{\text{variance}}&=\frac{1}{(n-t)^2}\sum_{j=t+1}^n\bigg( \eyeT + (\eyeT-\AL)^{-1}(\AL-\AL^{n+1-j}) + (\eyeT-\AR\T)^{-1}(\AR\T-(\AR\T)^{n+1-j}) \bigg)\phiv_j^{\text{variance}}
\end{align*}
We will substitute the expression for $\phiv_j^{\text{variance}}$ from equation~\ref{eq:varianceLP}.
{\small
\begin{align}
\phivb_{t,n}^{\text{variance}}&=\frac{1}{(n-t)^2}\sum_{j=t+1}^n\bigg( \eyeT + (\eyeT-\AL)^{-1}(\AL-\AL^{n+1-j}) + (\eyeT-\AR\T)^{-1}(\AR\T-(\AR\T)^{n+1-j}) \bigg)(\eyeT-\BT)^{-1}(\eyeT-\BT^j)\Sigh\nonumber
\end{align}
}
Evaluating this summation, we have:
\begin{align}
\label{eq:varianceTA}
\phivb_{t,n}^{\text{variance}}&=\underbrace{\frac{1}{n-t}\big(\eyeT + (\eyeT-\AL)^{-1}\AL + (\eyeT-\AR\T)^{-1}\AR\T\big)(\eyeT-\BT)^{-1}\Sigh}_{\text{Leading order term}}\nonumber\\&-\frac{1}{(n-t)^2}\big((\eyeT-\AL)^{-2}(\AL-\AL^{n+1-t})+(\eyeT-\AR\T)^{-2}(\AR\T-(\AR\T)^{n+1-t})\big)(\eyeT-\BT)^{-1}\Sigh\nonumber\\&-\frac{1}{(n-t)^2}\big(\eyeT + (\eyeT-\AL)^{-1}\AL + (\eyeT-\AR\T)^{-1}\AR\T\big)(\eyeT-\BT)^{-2}(\BT^{t+1}-\BT^{n+1})\Sigh\nonumber\\&+\frac{1}{(n-t)^2}\sum_{j=t+1}^n\big((\eyeT-\AL)^{-1}\AL^{n+1-j}+(\eyeT-\AR\T)^{-1}(\AR\T)^{n+1-j}\big)(\eyeT-\BT)^{-1}\BT^j\Sigh
\end{align}

\end{proof}

Equations~\ref{eq:tailAvgCovarBound},~\ref{eq:biasTA},~\ref{eq:varianceTA} wrap up the proof of lemmas~\ref{lem:average-covar-bias},~\ref{lem:average-covar-var}.

The parameter error of the (tail-)averaged iterate can be obtained using a trace operator 
$\iprod{{\Bigcdot}}{{\Bigcdot}}$ to the tail-averaged iterate's covariance $\phivb_{t,n}$ with the matrix $\begin{bmatrix}\eye&0\\0&0\end{bmatrix}$, i.e. 
\begin{align*}
\|\bar{\x}_{t,n}-\xs\|_2^2=\iprod{\begin{bmatrix}\eye & 0\\ 0 & 0\end{bmatrix}}{\phivb_{t,n}}
\end{align*}
In order to obtain the function error, we note the following taylor expansion of the function $P(\Bigcdot)$ around the minimizer $\xs$:
\begin{align*}
P(\x)&=P(\xs) + \frac{1}{2}\ \|\x-\xs\|_{\nabla^2P(\xs)}^2\\
&=P(\xs) + \frac{1}{2}\ \|\x-\xs\|_{\Cov}^2
\end{align*}
This implies the excess risk can be obtained as:
\begin{align*}
P(\bar{\x}_{t,n})-P(\xs)&=\frac{1}{2}\cdot\iprod{\begin{bmatrix}\H & 0\\ 0 & 0\end{bmatrix}}{\phivb_{t,n}}\\
&\leq\iprod{\begin{bmatrix}\H & 0\\ 0 & 0\end{bmatrix}}{\phivb^{\text{bias}}_{t,n}}+\iprod{\begin{bmatrix}\H & 0\\ 0 & 0\end{bmatrix}}{\phivb^{\text{variance}}_{t,n}}
\end{align*}
\iffalse
We wrap up this section by noting that the function error or parameter error (respectively) can be obtained by applying the trace operator $\iprod{{\Bigcdot}}{{\Bigcdot}}$ to the tail-averaged iterate's covariance $\phivb_{t,n}$: $\iprod{\begin{bmatrix}\H & 0\\ 0 & 0\end{bmatrix}}{\phivb_{t,n}}$ or $\iprod{\begin{bmatrix}\eye & 0\\ 0 & 0\end{bmatrix}}{\phivb_{t,n}}$. 
\fi

\iffalse
We have brought out the leading order terms in the expansion through adding and subtracting terms in the summation.
\begin{align*}
\phivb_{t,N}&=\frac{1}{(n-t)^2}\bigg((\eye-\A)^{-1}\A\sum_{l=t+1}^n\E{\thetav_l\otimes\thetav_l} + \sum_{l=t+1}^n\E{\thetav_l\otimes\thetav_l^T}+\sum_{j=t+1}^n\E{\thetav_j\otimes\thetav_j}(\eye-\A^T)^{-1}\A^T\nonumber\\
&-\sum_{l=t+1}^n\sum_{j=n+1}^{\infty}\A^{j-l}\E{\thetav_l\otimes\thetav_l} -\sum_{j=t+1}^n\sum_{l=n+1}^{\infty}\E{\thetav_j\otimes\thetav_j}(\A^T)^{l-j}\bigg)\nonumber\\
&=\frac{1}{(n-t)^2}\bigg(\left(\eyeT+(\eyeT-\AL)^{-1}\AL+(\eyeT-\AR^T)^{-1}\AR^T\right)\cdot\sum_{l=t+1}^n\E{\thetav_l\otimes\thetav_l}\bigg)\nonumber\\
&-\frac{1}{(n-t)^2}\bigg(\sum_{l=t+1}^n\sum_{j=n+1}^{\infty}\A^{j-l}\E{\thetav_l\otimes\thetav_l} +\sum_{j=t+1}^n\sum_{l=n+1}^{\infty}\E{\thetav_j\otimes\thetav_j}(\A^T)^{l-j}\bigg)
\end{align*}
\fi
\iffalse
Furthermore, we note that these expressions can be converted to precise statements about the bias-variance decomposition. Namely, similar to~\cite{BachM13,DefossezB15,JainKKNS16}, we split the recursion involving $\E{\thetav_l\otimes\thetav_l}$ into two terms, one consisting of terms dependent on initial conditions $\thetav_0\otimes\thetav_0$ (otherwise referred to as the bias), and other terms that depend on the noise level of the problem (namely the variance). We loose atmost a factor of $2$ in the estimate of the generalization error through this decomposition.

\underline{\bf Bias Part}
For the bias part, we consider the noiseless problem, and that implies that
\begin{align*} \E{\thetav_l\otimes\thetav_l}=\B\E{\thetav_{l-1}\otimes\thetav_{l-1}}=\B^l\thetav_0\otimes\thetav_0
\end{align*}
Correspondingly, the bias covariance is:
\begin{align*}
\phivb^{\text{bias}}&=\frac{1}{(n-t)^2}\bigg(\left(\eyeT+(\eyeT-\AL)^{-1}\AL+(\eyeT-\AR^T)^{-1}\AR^T\right)\cdot\sum_{l=t+1}^n\E{\thetav_l\otimes\thetav_l}\bigg)\nonumber\\
&-\frac{1}{(n-t)^2}\bigg(\sum_{l=t+1}^n\sum_{j=n+1}^{\infty}\A^{j-l}\E{\thetav_l\otimes\thetav_l} +\sum_{j=t+1}^n\sum_{l=n+1}^{\infty}\E{\thetav_j\otimes\thetav_j}(\A^T)^{l-j}\bigg)\nonumber\\
&=\frac{1}{(n-t)^2}\bigg(\left(\eyeT+(\eyeT-\AL)^{-1}\AL+(\eyeT-\AR^T)^{-1}\AR^T\right)\cdot(\eyeT-\BT)^{-1}(\BT^{t+1}-\BT^{n+1})\thetav_0\otimes\thetav_0\bigg)\nonumber\\
&-\frac{1}{(n-t)^2}\bigg( (\eyeT-\AL)^{-1} (\eyeT-\AL^{-1}\BT)^{-1}(\AL^{n-t}\BT^{t+1}-\BT^{n+1})\thetav_0\otimes\thetav_0 \ \nonumber\\ &+ (\eyeT-\AR^T)^{-1}(\eyeT-(\AR^T)^{-1}\BT)^{-1}((\AR^T)^{n-t}\BT^{t+1}-\BT^{n+1})\thetav_0\otimes\thetav_0\bigg)
\end{align*}

\underline{\bf Variance Part}
For the variance part, we consider the problem as started from the solution i.e. $\thetav_0=0$ the noise drives the process. This implies that
\begin{align*} \E{\thetav_l\otimes\thetav_l}&=\B\E{\thetav_{l-1}\otimes\thetav_{l-1}}+\Sigh\nonumber\\
&=(\eyeT-\BT)^{-1}(\eyeT-\B^l)\Sigh
\end{align*}
Correspondingly, the covariance of the variance error is:
\begin{align*}
&\phivb^{\text{variance}}=\\
&\frac{1}{n-t}\cdot\bigg(\eyeT+(\eyeT-\AL)^{-1}\AL+(\eyeT-\AR\T)^{-1}\AR\T\bigg)(\eyeT-\BT)^{-1}\Sigh\nonumber\\
&-\frac{1}{(n-t)^2}\cdot\bigg(\eyeT+(\eyeT-\AL)^{-1}\AL+(\eyeT-\AR\T)^{-1}\AR\T\bigg)(\eyeT-\BT)^{-1}(\BT^{t+1}-\BT^{n+1})\Sigh\nonumber\\
&-\frac{1}{n-t}\cdot\bigg((\eyeT-\AL)^{-2}(\AL-\AL^{n-t})+(\eyeT-\AR\T)^{-2}(\AR\T-(\AR^T)^{n-t})\bigg)(\eyeT-\BT)^{-1}\Sigh\nonumber\\
&+\frac{1}{(n-t)^2}\cdot\bigg( (\eyeT-\AL)^{-1}(\eyeT-\AL^{-1}\BT)^{-1}(\AL^{n-t}\BT^{t+1}-\BT^{n+1}) \nonumber\\&+ (\eyeT-\AR\T)^{-1}(\eyeT-(\AR^T)^{-1}\BT)^{-1}((\AR^T)^{n-t}\BT^{t+1}-\BT^{n+1}) \bigg)(\eyeT-\BT)^{-1}\Sigh
\end{align*}
\fi
%\end{proof}


\section{Useful lemmas}
\label{sec:commonLemmas}
In this section, we will state and prove some useful lemmas that will be helpful in the later sections.
\begin{lemma}\label{lem:com3}
	\begin{align*}
	{\left(\Id - \A\T\right)}^{-1} \begin{bmatrix}
	\Cov & \zero \\ \zero & \zero
	\end{bmatrix} = \frac{1}{\g-c\delta}\begin{bmatrix}-(c\eye-\g\Cov)&0\\(\eye-\delta\Cov)&0\end{bmatrix}
	\end{align*}
\end{lemma}
\begin{proof}
Since we assumed that $\Cov$ is a diagonal matrix (with out loss of generality), we note that $\A$ is a block diagonal matrix after a rearrangement of the co-ordinates (via an eigenvalue decomposition).

In particular, by considering the $j^{\text{th}}$ block (denoted by $\A_j$ corresponding to the $j^{\textrm{th}}$ eigenvalue $\lambda_j$ of $\Cov$), we have:
\begin{align*}
\eye-\A_j\T=\begin{bmatrix} 1 & c \\ -(1-\delta\lambda_j) & -(c-\g\lambda_j)\end{bmatrix}
\end{align*}
Implying that the determinant $\Det{\eye-\A_j\T}=(\g-c\delta)\lambda_j$, using which:
\begin{align}
\label{eq:ATInv}
(\eye-\A_j\T)^{-1}&=\frac{1}{(\g-c\delta)\lambda_j}\begin{bmatrix}-(c-\g\lambda_j)&-c\\1-\delta\lambda_j&1\end{bmatrix}
\end{align}
Thus, 
\begin{align*}
(\eye-\A_j\T)^{-1}\begin{bmatrix}\lambda_j&0\\0&0\end{bmatrix}&=\frac{1}{\g-c\delta}\begin{bmatrix}-(c-\g\lambda_j)&0\\(1-\delta\lambda_j)&0 \end{bmatrix}
\end{align*}
Accumulating the results of each of the blocks and by rearranging the co-ordinates, the result follows.
\end{proof}

\begin{lemma}\label{lem:com1}
	\begin{align*}
			\inv{\left(\Id - \A\T\right)} \begin{bmatrix}
		\Cov & \zero \\ \zero & \zero
		\end{bmatrix} \inv{\left(\Id - \A \right)} = \frac{1}{(\g-c\delta)^2}\bigg(\otimes_2\begin{bmatrix} -(c\eye-\g\Cov)\Cov^{-1/2}\\(\eye-\delta\Cov)\Cov^{-1/2}\end{bmatrix}\bigg)
	\end{align*}
\end{lemma}
\begin{proof}

In a similar manner as in lemma~\ref{lem:com3}, we decompose the computation into each of the eigen-directions and subsequently re-arrange the results. In particular, we note:
\begin{align*}
(\eye-\A_j)^{-1}=\frac{1}{(\g-c\delta)\lambda_j}\begin{bmatrix}-(c-\g\lambda_j)&(1-\delta\lambda_j)\\-c&1\end{bmatrix}
\end{align*}
Multiplying the above with the result of lemma~\ref{lem:com3}, we have:
\begin{align*}
(\eye-\A_j\T)^{-1}\begin{bmatrix}\lambda_j&0\\0&0\end{bmatrix}(\eye-\A_j)^{-1}=\frac{1}{(\g-c\delta)^2}\bigg(\otimes_{2}\begin{bmatrix}-(c-\g\lambda_j)\lambda_j^{-1/2}\\(1-\delta\lambda_j)\lambda_j^{-1/2}\end{bmatrix}\bigg)
\end{align*}
From which the statement of the lemma follows through a simple re-arrangement.

\end{proof}

\begin{lemma}\label{lem:com2}
	\begin{align*}
	{\left(\Id - \A\T\right)}^{-2} \A\T \begin{bmatrix}
	\Cov & \zero \\ \zero & \zero
	\end{bmatrix} = \frac{1}{(\g-c\delta)^2}\begin{bmatrix}\Cov^{-1}(-c(1-c)\eye-c\g\Cov)(\eye-\delta\Cov)&0\\\Cov^{-1}((1-c)\eye-c\delta\Cov)(\eye-\delta\Cov)&0\end{bmatrix}
	\end{align*}
\end{lemma}
\begin{proof}
In a similar argument as in previous two lemmas, we analyze the expression in each eigendirection of $\Cov$ through a rearrangement of the co-ordinates. Utilizing the expression of $\eye-\A_j\T$ from equation~\ref{eq:ATInv}, we get:
\begin{align}
\label{eq:intermediateEqn}
(\eye-\A_j\T)^{-1}\A_j\T\begin{bmatrix}\lambda_j&0\\0&0\end{bmatrix}
=\frac{1}{(\g-c\delta)}\begin{bmatrix}-c(1-\delta\lambda_j)&0\\(1-\delta\lambda_j)&0\end{bmatrix}
\end{align}
thus implying:
\begin{align*}
(\eye-\A_j\T)^{-2}\A_j\T\begin{bmatrix}\lambda_j&0\\0&0\end{bmatrix}
=\frac{(1-\delta\lambda_j)}{(\g-c\delta)^2\lambda_j}\begin{bmatrix}-c(1-c)-c\g\lambda_j&0\\(1-c)-c\delta\lambda_j&0\end{bmatrix}
\end{align*}
Rearranging the co-ordinates, the statement of the lemma follows.
\end{proof}

\begin{lemma}\label{lem:eig-A}
	The matrix $\A$ satisfies the following properties:
	\begin{enumerate}
		\item	Eigenvalues $q$ of $\A$ satisfy $\abs{q} \leq \sqrt{\alpha}$, and
		\item	$ \twonorm{\A^k}\leq 3\sqrt{2} \cdot k \cdot \alpha^{\frac{k-1}{2}} \; \forall \; k \geq 1$.
	\end{enumerate}
\end{lemma}
\begin{proof}
	Since the matrix is block-diagonal with $2\times2$ blocks, after a rearranging the coordinates, we will restrict ourselves to bounding the eigenvalues and eigenvectors of each of these $2\times 2$ blocks. Combining the results for different blocks then proves the lemma. Recall that $\A_j = \begin{bmatrix}
	0 & 1 - \delta \lambda_j \\ -c & 1+c - \g \lambda_j
	\end{bmatrix}$.
	
	\textbf{Part I}: Let us first prove the statement about the eigenvalues of $\A$. There are two scenarios here:
	\begin{enumerate}
		\item \emph{Complex eigenvalues}: In this case, both eigenvalues of $\A_j$ have the same magnitude which is given by $\sqrt{\det(\A_j)} = \sqrt{c(1-\delta \lambda_j)}\leq \sqrt{c} \leq \sqrt{\alpha}$.
		\item	\emph{Real eigenvalues}: Let $q_1$ and $q_2$ be the two real eigenvalues of $\A_j$. We know that $q_1+q_2 = \trace{\A_j} = 1 + c - \g\lambda_j > 0$ and $q_1 \cdot q_2 = \det(\A_j) > 0$. This means that $q_1 > 0$ and $q_2 > 0$.
		Now, consider the matrix $\G_j \eqdef (1-\beta) \eye - \A_j = \begin{bmatrix}
		 (1-\beta)  & - 1 + \delta \lambda_j \\ c & -1+{ (1-\beta) (1-\alpha)} + \g \lambda_j
		\end{bmatrix}$. We see that $( (1-\beta) -q_1)( (1-\beta) -q_2) = \det(\G_j) =  (1-\beta) (1-\alpha)\left( (1-\beta) -1\right) +  (1-\beta)  \left(\g-\alpha \delta\right)\lambda_j = (1-\beta)\left(1-\alpha\right)\left(\gamma \lambda_j - \beta\right) \geq 0$. This means that there are two possibilities: either $q_1, q_2 \geq  (1-\beta) $ or $q_1, q_2 \leq  (1-\beta) $. If the second condition is true, then we are done. If not, if $q_1, q_2 \geq  (1-\beta) $, then $\max_i q_i = \frac{\det(\A_j)}{\min_i q_i} \leq \frac{c(1-\delta \lambda_j)}{ (1-\beta) }\leq \alpha (1-\delta \lambda_j)$. Since $\sqrt{\alpha} \geq \alpha \geq 1-\beta$, this proves the first part of the lemma.
	\end{enumerate}

	\textbf{Part II}: Let $\A_j = \V \Q \V\T$ be the Schur decomposition of $\A_j$ where $\Q = \begin{bmatrix}
	q_1 & q \\ 0 & q_2
	\end{bmatrix}$ is an upper triangular matrix with eigenvalues $q_1$ and $q_2$ of $\A_j$ on the diagonal and $\V$ is a unitary matrix i.e., $\V \V\T = \V\T \V = \Id$. We first observe that $\abs{q} \leq \twonorm{\Q} \stackrel{(\zeta_1)}{=} \twonorm{\A_j} \leq \frob{\A_j} \leq \sqrt{6}$, where $(\zeta_1)$ follows from the fact that $\V$ is a unitary matrix. $\V$ being unitary also implies that $\A_j^k = \V \Q^k \V\T$. On the other hand, a simple proof via induction tells us that
	\begin{align*}
		\Q^k = \begin{bmatrix}
		q_1^k & q \left(\sum_{\ell = 1}^{k-1} q_1^{\ell}q_2^{k-\ell}\right) \\ 0 & q_2^k
		\end{bmatrix}.
	\end{align*}
	So, we have $\twonorm{\A_j^k} = \twonorm{\Q^k} \leq \frob{\Q^k} \leq \sqrt{3}k \abs{q} \max\left(\abs{q_1}^{k-1}, \abs{q_2}^{k-1}\right) \leq 3\sqrt{2} \cdot k \cdot \alpha^{\frac{k-1}{2}}$, where we used $\abs{q} \leq \sqrt{6}$ and $\max\left(\abs{q_1},\abs{q_2}\right) \leq \sqrt{\alpha}$.
\end{proof}
Finally, we state and prove the following lemma which is a relation between left and right multiplication operators.
\begin{lemma}
	\label{lem:lhs-psd-lemma}
	Let $\A$ be any matrix with $\AL=\A\otimes\eye$ and $\AR=\eye\otimes\A$ representing its left and right multiplication operators. Then, the following expression holds:
	\begin{align*}
	\bigg(\eyeT + (\eyeT-\AL)^{-1}\AL + (\eyeT-\AR\T)^{-1}\AR\T\bigg)(\eyeT-\AL\AR\T)^{-1}&=(\eyeT-\AL)^{-1}(\eyeT-\AR\T)^{-1}
	\end{align*}
\end{lemma}
\begin{proof}
	Let us assume that $\A$ can be written in terms of its eigen decomposition as $\A = \V\Lambda\V^{-1}$.
	Then the first claim is that $\eyeT,\AL,\AR$ are diagonalized by the same basis consisting of the eigenvectors of $\A$, i.e. in particular, the matrix of eigenvectors of $\eyeT,\AL,\AR$ can be written as $\V\otimes\V$. In particular, this implies, $\forall\ i,j\in\{1,2,...,d\}\times\{1,2,...,d\}$, we have, applying $\vv_i\otimes\vv_j$ to the LHS, we have:
	\begin{align*}
	&\bigg(\eyeT + (\eyeT-\AL)^{-1}\AL + (\eyeT-\AR\T)^{-1}\AR\T\bigg)(\eyeT-\AL\AR\T)^{-1} \vv_i\otimes\vv_j\\
	&=(1-\lambda_i\lambda_j)^{-1}\bigg(\eyeT + (\eyeT-\AL)^{-1}\AL + (\eyeT-\AR\T)^{-1}\AR\T\bigg)\vv_i\otimes\vv_j\\
	&=(1+\lambda_i(1-\lambda_i)^{-1}+\lambda_j(1-\lambda_j)^{-1})\cdot(1-\lambda_i\lambda_j)^{-1}\vv_i\otimes\vv_j
	\end{align*}
	Applying $\vv_i\otimes\vv_j$ to the RHS, we have:
	\begin{align*}
	&(\eyeT-\AL)^{-1}(\eyeT-\AR\T)^{-1}\vv_i\otimes\vv_j\\
	&=(1-\lambda_i)^{-1}(1-\lambda_j)^{-1}\vv_i\otimes\vv_j
	\end{align*}
	The next claim is that for any scalars (real/complex) $x,y~\ne 1$, the following statement holds implying the statement of the lemma:
	\begin{align*}
	(1+(1-x)^{-1}x+(1-y)^{-1}y)\cdot(1-xy)^{-1}=(1-x)^{-1}(1-y)^{-1}
	\end{align*}
\end{proof}

\begin{lemma}\label{lem:G-bound}
	Recall the matrix $\G$ defined as $\G \eqdef \begin{bmatrix} \Id & \frac{-\alpha}{1-\alpha}\Id \\ \zero & \frac{1}{1-\alpha}\Id \end{bmatrix} \begin{bmatrix} \Id &\zero \\ \zero & {\mu}\inv{\Cov} \end{bmatrix} \begin{bmatrix} \Id &\zero \\ \frac{-\alpha}{1-\alpha}\Id & \frac{1}{1-\alpha}\Id \end{bmatrix}$. The condition number of $\G$, $\kappa(\G)$ satisfies $\kappa(\G)\leq \frac{4\cnH}{\sqrt{1-\alpha^2}}$.
\end{lemma}
\begin{proof}
	Since the above matrix is block-diagonal after a rearrangement of coordinates, it suffices to compute the smallest and largest singular values of each block. Let $\lambda_i$ be the $i^{\textrm{th}}$ eigenvalue of $\Cov$. Let $\C \eqdef \begin{bmatrix} 1 & 0 \\ \frac{-\alpha}{1-\alpha} & \frac{1}{1-\alpha} \end{bmatrix}$ and consider the matrix $\G_i \eqdef \C \begin{bmatrix} 1 & 0 \\ 0 & \frac{\mu}{\lambda_i} \end{bmatrix} \C \T$. The largest eigenvalue of $\G_i$ is at most $\singmax{\C}^2$, while the smallest eigenvalue, $\singmin{\G_i}$ is at least $\frac{\mu}{\lambda_i} \cdot \singmin{\C}^2$. We obtain the following bounds on $\singmin{\C}$ and $\singmax{\C}$.
	\begin{align*}
		\singmax{\C} &\leq \frob{\C} \leq \frac{2}{\sqrt{1-\alpha^2}} \quad (\because \; \alpha \leq 1 ) \\
		\singmin{\C} &\geq \frac{\sqrt{\det\left(\C \C\T\right)}}{\frob{\C}} \geq \frac{1}{2}, \\ &\qquad \left(\because \det\left(\C\C\T\right) = \singmax{\C}^2 \singmin{\C}^2\right)
	\end{align*}
	where we used the computation that $\det\left(\C \C\T\right) = \frac{1}{1-\alpha}$. This means that $\singmin{\G_i} \geq \frac{\mu}{2\lambda_i}$ and $\singmax{\G_i} \leq \frac{2}{\sqrt{1-\alpha^2}}$. Combining all the blocks, we see that the condition number of $\G$ is at most $\frac{4\cnH}{\sqrt{1-\alpha^2}}$, proving the lemma.
\end{proof}



\section{Lemmas and proofs for bias contraction}
\label{sec:biasContraction}
\begin{proof}[Proof of Lemma~\ref{lem:main-bias}]
Let $\v \eqdef \frac{1}{1-\alpha}\left(\y - \alpha \x\right)$ and consider the following update rules corresponding to the noiseless versions of the updates in Algorithm~\ref{algo:TAASGD}:
\begin{align*}
	\xplus &= \y - \delta \Hhat (\y-\xs) \\
	\z &= \beta \y + (1-\beta) \v \\
	\vplus &= \z - \gamma \Hhat (\y-\xs) \\
	\yplus &= \alpha \xplus + (1-\alpha) \vplus,
\end{align*}
where $\Hhat \eqdef \a \a\T$ where $\a$ is sampled from the marginal on $\distr$. We first note that 
\begin{align*}
\E{\otimes_2 \begin{bmatrix} \xplus-\xs \\ \yplus-\xs \end{bmatrix}} &= \E{\Ah \bigg(\otimes_2 \begin{bmatrix} \x-\xs \\ \y-\xs \end{bmatrix}\bigg)\Ah\T} \\
&=\BT\bigg(\otimes_2 \begin{bmatrix} \x-\xs \\ \y-\xs \end{bmatrix}\bigg) 
\end{align*}
Letting $\Gtilde \eqdef \begin{bmatrix} \Id &\zero \\ \frac{-\alpha}{1-\alpha}\Id & \frac{1}{1-\alpha}\Id \end{bmatrix}$, we can verify that $\begin{bmatrix} \x-\xs \\ \v-\xs \end{bmatrix} = \Gtilde \begin{bmatrix} \x-\xs \\ \y-\xs \end{bmatrix}$, similarly $\begin{bmatrix} \xplus-\xs \\ \vplus-\xs \end{bmatrix} = \Gtilde \begin{bmatrix} \xplus-\xs \\ \yplus-\xs \end{bmatrix}$. Recall that $\G \defeq \Gtilde \T \begin{bmatrix} \Id &\zero \\ \zero & {\mu}\inv{\Cov} \end{bmatrix} \Gtilde$. With this notation in place, we prove the statement below, and substitute the values of $\cone,\ctwo,\cthree$ to obtain the statement of the lemma:
\begin{align}
\label{eq:biasContraction}
\iprod{\begin{bmatrix}\eye&0\\0&\mu\cdot\Hinv\end{bmatrix}}{\otimes_{2}\bigg(\begin{bmatrix}\xplus-\xs\\\vplus-\xs\end{bmatrix}\bigg)}\leq\left(1-\cthree\frac{\ctwo\sqrt{2\cone-\cone^2}}{\sqrt{\cnH\cnHh}}\right)\cdot\iprod{\begin{bmatrix}\eye&0\\0&\mu\cdot\Hinv\end{bmatrix}}{\otimes_{2}\bigg(\begin{bmatrix}\x-\xs\\\v-\xs\end{bmatrix}\bigg)}
\end{align}
%Where, $\phiv_k,\hat{\A}_{k+1}$ represents the bias operator $\phiv_k^{\text{bias}}$, $\hat{\A}_{k+1}^{\text{bias}}$. Since we are dealing with the bias analysis, we understand error contraction on the noiseless problem.

To establish this result, let us define two quantities: $\e \eqdef {\twonorm{\x-\xs}^2}$, $\f \eqdef {\norm{\v-\xs}^2_{\Hinv}}$ and similarly, $\eplus \eqdef {\twonorm{\xplus-\xs}^2}$ and $\fplus \eqdef {\norm{\vplus-\xs}^2_{\Hinv}}$. The potential function we consider is $\e + \mu\cdot \f$. Recall that the parameters are chosen as:
%\praneeth{Fix the parameters below.}
\begin{align*}
\alpha = \frac{\sqrt{\cnH\cnHh}}{\ctwo\sqrt{2\cone-\cone^2}+\sqrt{\cnH\cnHh}},\ \  \beta = \cthree\frac{\ctwo\sqrt{2\cone-\cone^2}}{\sqrt{\cnH\cnHh}},\ \  \gamma = \ctwo\frac{\sqrt{2\cone-\cone^2}}{\mu\sqrt{\cnH\cnHh}}, \ \ \delta=\frac{\cone}{\infbound}
\end{align*}
with $\cone<1/2$, $\cthree=\frac{\ctwo\sqrt{2\cone-\cone^2}}{\cone}$, $\ctwo^2=\frac{\cfour}{2-\cone}$.
Consider $\eplus$ and employ the simple gradient descent bound:
\begin{align}
\label{eq:gd}
\eplus=\E{\twonorm{\xplus-\xs}^2}&=\E{\twonorm{\y-\delta\cdot\Hhat(\y-\xs)-\xs}^2}\nonumber\\
&=\E{\twonorm{\y-\xs}^2}-2\delta\cdot\E{\norm{\y-\xs}^2_{\H}}+\delta^2\E{\norm{\y-\xs}^2_{\M\eye}}\nonumber\\
&\leq\E{\twonorm{\y-\xs}^2}-2\delta\cdot\E{\norm{\y-\xs}^2_{\H}}+\infbound\delta^2\E{\norm{\y-\xs}^2_{\H}}\nonumber\\
&=\E{\twonorm{\y-\xs}^2}-\frac{2\cone-\cone^2}{\infbound}\E{\norm{\y-\xs}^2_{\H}}
\end{align}
Next, consider $\fplus$:
\begin{align}
\label{eq:1}
\fplus=\E{\norm{\vplus-\xs}^2_{\Hinv}}&=\E{\norm{\z-\gamma\Hhat(\y-\xs)-\xs}^2_{\Hinv}}\nonumber\\
&=\E{\norm{\z-\xs}^2_{\Hinv}}+\gamma^2\E{\norm{\y-\xs}^2_{\M\Hinv}}-2\gamma\E{\iprod{\z-\xs}{\y-\xs}}\nonumber\\
&\leq\E{\norm{\z-\xs}^2_{\Hinv}}+\gamma^2\cnHh\cdot\E{\norm{\y-\xs}^2_{\H}}-2\gamma\cdot\E{\iprod{\z-\xs}{\y-\xs}}
\end{align}
Where, we use the fact that $\M\Hinv\preceq\cnHh\H$, where $\cnHh$ is the {\em statistical} condition number.\linebreak
Consider $\E{\norm{\z-\xs}^2_{\Hinv}}$ and use convexity of the weighted $2-$norm to get:
\begin{align}
\label{eq:3}
\E{\norm{\z-\xs}^2_{\Hinv}}&\leq\beta\E{\norm{\y-\xs}^2_{\Hinv}}+(1-\beta)\E{\norm{\v-\xs}^2_{\Hinv}}\nonumber\\
&\leq\frac{\beta}{\mu} \E{\twonorm{\y-\xs}^2}+(1-\beta)\cdot \f
\end{align}
Next, consider $\E{\iprod{\z-\xs}{\y-\xs}}$, and first write $\z$ in terms of $\x$ and $\y$. This can be seen as two steps:
\begin{itemize}
\item $\v = \frac{1}{1-\alpha}\cdot\y-\frac{\alpha}{1-\alpha}\cdot\x$
\item $\z = \beta\y + (1-\beta)\v=\y + (1-\beta)(\v-\y)$. Then substituting $\v$ in terms of $\x$ and $\y$ as in the equation above, we get: $\z = \y + \left(\frac{\alpha\cdot(1-\beta)}{1-\alpha}\right)(\y-\x)$
\end{itemize}
Then, $\E{\iprod{\z-\xs}{\y-\xs}}$ can be written as:
\begin{align}
\label{eq:2}
\E{\iprod{\z-\xs}{\y-\xs}}&=\E{\twonorm{\y-\xs}^2}+\left(\frac{\alpha(1-\beta)}{1-\alpha}\right)\E{\iprod{\y-\x}{\y-\xs}}
\end{align}
Then, we note:
\begin{align*}
\E{\iprod{\y-\x}{\y-\xs}} &= \E{\twonorm{\y-\xs}^2}-\E{\iprod{\x-\xs}{\y-\xs}}\\
&\geq \E{\twonorm{\y-\xs}^2}-\frac{1}{2}\cdot\left(\E{\twonorm{\y-\xs}^2}+\E{\twonorm{\x-\xs}^2}\right) \\
&=\frac{1}{2}\cdot\left(\E{\twonorm{\y-\xs}^2}-\E{\twonorm{\x-\xs}^2}\right)
\end{align*}
Re-substituting in equation~\ref{eq:2}:
\begin{align}
\label{eq:4}
\E{\iprod{\z-\xs}{\y-\xs}}&\geq\left(1+\frac{1}{2}\cdot\frac{\alpha(1-\beta)}{1-\alpha}\right)\E{\twonorm{\y-\xs}^2}-\frac{1}{2}\cdot\frac{\alpha(1-\beta)}{1-\alpha}\E{\twonorm{\x-\xs}^2}\nonumber\\
&=\left(1+\frac{1}{2}\cdot\frac{\alpha(1-\beta)}{1-\alpha}\right)\E{\twonorm{\y-\xs}^2}-\frac{1}{2}\cdot\frac{\alpha(1-\beta)}{1-\alpha}\cdot \e
\end{align}
Substituting equations~\ref{eq:3},~\ref{eq:4} into equation~\ref{eq:1}, we get:
\begin{align*}
\mu\cdot \fplus&\leq \left(\beta-2\gamma\mu-\frac{\gamma\mu\alpha(1-\beta)}{1-\alpha}\right)\E{\twonorm{\y-\xs}^2}+\mu(1-\beta)\cdot \f \nonumber\\&+ \frac{\gamma\mu\alpha(1-\beta)}{1-\alpha}\cdot \e+\mu\gamma^2\cnHh\cdot\E{\norm{\y-\xs}^2_{\H}}
\end{align*}
Rewriting the guarantee on $\eplus$ as in equation~\ref{eq:gd}:
\begin{align*}
\eplus \leq \E{\twonorm{\y-\xs}^2}-\frac{2\cone-\cone^2}{\infbound}\cdot\E{\norm{\y-\xs}^2_{\H}}
\end{align*}
By considering $\eplus+\mu\cdot \fplus$, we see the following:
\begin{itemize}
\item The coefficient of $\E{\norm{\y-\xs}_{\H}^2}\leq 0$ by setting $\gamma = \ctwo\frac{\sqrt{2\cone-\cone^2}}{\mu\sqrt{\cnH\cnHh}}$, where, $0<\ctwo\leq1$, $\cnH = \frac{\infbound}{\mu}$.
\item Set $\frac{\gamma\mu\alpha}{1-\alpha}=1$ implying $\alpha = \frac{1}{1+\gamma\mu} = \frac{\sqrt{\cnH\cnHh}}{\ctwo\sqrt{2\cone-\cone^2}+\sqrt{\cnH\cnHh}}$
\end{itemize}
With these in place, we have the final result:
\begin{align*}
\eplus+\mu\cdot \fplus \leq (2\beta-2\gamma\mu)\E{\twonorm{\y-\xs}^2} + (1-\beta)\cdot(\e+\mu\cdot \f)
\end{align*}
In particular, setting $\beta=\cthree\gamma\mu=\cthree\frac{\ctwo\sqrt{2\cone-\cone^2}}{\sqrt{\cnH\cnHh}}$, we have a per-step contraction of $1-\beta$ which is precisely $1-\cthree\frac{\ctwo\sqrt{2\cone-\cone^2}}{\sqrt{\cnH\cnHh}}$, from which the claimed result naturally follows by substituting the values of $\cone,\ctwo,\cthree$.
\end{proof}

\begin{lemma}\label{lem:B-contraction}
%	For any vector $\thetav \in \R^{2d}$, we have:
	For any psd matrix $\Q \succeq 0$, we have:
	\begin{align*}
%		\twonorm{\B^k \thetav \thetav \T} \leq \frac{4\cnH}{\sqrt{1-\alpha^2}}\exp\left(\frac{-k \ctwo \cthree \sqrt{2\cone-\cone^2} }{\sqrt{\cnH\cnS}}\right) \twonorm{\thetav}^2.
		\twonorm{\B^k \Q} \leq \frac{4\cnH}{\sqrt{1-\alpha^2}}\bigg(1-\left(\frac{ \ctwo \cthree \sqrt{2\cone-\cone^2} }{\sqrt{\cnH\cnS}}\right)\bigg)^k \twonorm{\Q}.
	\end{align*}
\end{lemma}
\begin{proof}
	From Lemma~\ref{lem:main-bias}, we conclude that $\iprod{\G}{\B^k \Q} \leq \bigg(1-\left(\frac{ \ctwo \cthree \sqrt{2\cone-\cone^2} }{\sqrt{\cnH\cnS}}\right)\bigg)^k\iprod{\G}{\Q}$. This implies that $\twonorm{\B^k \Q} \leq \bigg(1-\left(\frac{ \ctwo \cthree \sqrt{2\cone-\cone^2} }{\sqrt{\cnH\cnS}}\right)\bigg)^k \twonorm{\Q} \kappa(\G)$. Plugging the bound on $\kappa(\G)$ from Lemma~\ref{lem:G-bound} proves the lemma.
\end{proof}
\begin{lemma}\label{lem:bias-bound}
	We have: %\rahul{write out for general psd matrices.}
	\begin{align*}
		&\left(\Id - \D\right) \inv{\left(\Id - \B \right)} \B^{t+1} \left(\Id - \B^{n-t}\right) \thetat[0]\thetat[0]\T \\ &\;\preceq \frac{4\cnH}{\sqrt{1-\alpha^2}} \exp\left(- t \ctwo \cthree \sqrt{2\cone - \cone^2}/\sqrt{\cnH\cnS}\right) \norm{\thetat[0]}^2 \left(\eye +  \frac{\sqrt{\cnH\cnS}}{\ctwo\cthree\sqrt{2\cone-\cone^2}} (\infbound/\sigma^2) \Sighat \right).
	\end{align*}
\end{lemma}
\begin{proof}
	The proof follows from Lemma~\ref{lem:main-bias}. Since $\B = \D + \Rc$, we have $\left(\eyeT - \D\right) \inv{\left(\eyeT - \B \right)} = \eyeT + \Rc \inv{\left(\eyeT - \B \right)}$. Since $\Rc, \B$ and $\inv{(\eyeT - \B)}$ are all PSD operators, we have
	\begin{align*}
		&\left(\eyeT - \D\right) \inv{\left(\eyeT - \B \right)} \B^{t+1} \left(\eyeT - \B^{n-t}\right) \thetat[0]\thetat[0]\T \\
		&= \left(\eyeT + \Rc \inv{\left(\eyeT - \B \right)}\right) \B^{t+1} \left(\eyeT - \B^{n-t}\right) \thetat[0]\thetat[0]\T \\
		&\preceq \underbrace{\B^{t+1} \thetat[0]\thetat[0]\T}_{\S_1\eqdef } + \underbrace{\Rc \inv{\left(\eyeT - \B \right)} \B^{t+1} \thetat[0]\thetat[0]\T}_{\S_2\eqdef }.
	\end{align*}
	Applying Lemma~\ref{lem:B-contraction} with $\Q=\thetav_0\thetav_0\T$ tells us that $\S_1 \preceq \frac{4\cnH}{\sqrt{1-\alpha^2}} \exp\left(-t \ctwo \cthree \sqrt{2\cone - \cone^2}/\sqrt{\cnH\cnS}\right) \twonorm{\thetat[0]}^2 \Id$.
%	Recall from Lemma~\ref{lem:main-bias} that $\iprod{\G}{\B \thetat[0]\thetat[0]\T} \leq \exp\left(-1/\sqrt{4 \cnH\cnS}\right) \iprod{\G}{\thetat[0]\thetat[0]\T}$. For $\S_1$, we have
%	\begin{align*}
%		&\iprod{\G}{\S_1} = \iprod{\G}{\B^{t+1} \thetat[0]\thetat[0]\T} \leq \exp\left(-t/\sqrt{4 \cnH\cnS}\right) \iprod{\G}{\thetat[0]\thetat[0]\T},
%	\end{align*}
%	giving us $\S_1 \preceq \kappa(\G) \exp\left(-t/\sqrt{4 \cnH\cnS}\right) \norm{\thetat[0]}^2 \eye$.
	For $\S_2$, we have
	\begin{align*}
		&\iprod{\G}{\inv{\left(\eyeT - \B \right)} \B^{t+1} \thetat[0]\thetat[0]\T} = \iprod{\G}{\sum_{j=t+1}^{\infty} \B^j \thetat[0]\thetat[0]\T} \\
		&\quad \leq \sum_{j=t+1}^{\infty} \bigg(1-\left(\frac{ \ctwo \cthree \sqrt{2\cone-\cone^2} }{\sqrt{\cnH\cnS}}\right)\bigg)^j \iprod{\G}{\thetat[0]\thetat[0]\T} \\
		&\quad \leq \frac{\sqrt{\cnH\cnS}}{\ctwo\cthree\sqrt{2\cone-\cone^2}} \exp\left(-t \ctwo \cthree \sqrt{2\cone - \cone^2}/\sqrt{4 \cnH\cnS}\right) \iprod{\G}{\thetat[0]\thetat[0]\T}.
	\end{align*}
	This implies
	\begin{align*}
	\inv{\left(\eyeT - \B \right)} \B^{t+1} \thetat[0]\thetat[0]\T \preceq  \kappa(\G) (\sqrt{\cnH\cnS}/(\ctwo\cthree\sqrt{2\cone-\cone^2}))  \exp\left(-t \ctwo \cthree \sqrt{2\cone - \cone^2}/\sqrt{4 \cnH\cnS}\right) \norm{\thetat[0]}^2 \eye,
	\end{align*}
	which tells us that 
	\begin{align*}
	\S_2 \preceq \kappa(\G) (\sqrt{\cnH\cnS}/(\ctwo\cthree\sqrt{2\cone-\cone^2}))  \exp\left(-t \ctwo \cthree \sqrt{2\cone - \cone^2}/\sqrt{4 \cnH\cnS}\right) \norm{\thetat[0]}^2 (\infbound/\sigma^2) \Sighat
	\end{align*} 
	Combining the bounds on $\S_1$ and $\S_2$, we obtain
	\begin{align*}
		&\left(\eyeT - \D\right) \inv{\left(\eyeT - \B \right)} \B^{t+1} \left(\eyeT - \B^{n-t}\right) \thetat[0]\thetat[0]\T \nonumber\\&\preceq \kappa(\G) \exp\left(- t \ctwo \cthree \sqrt{2\cone - \cone^2}/\sqrt{4 \cnH\cnS}\right) \norm{\thetat[0]}^2 \left(\eye + \frac{\sqrt{\cnH\cnS}}{\ctwo\cthree\sqrt{2\cone-\cone^2}} (\infbound/\sigma^2) \Sighat \right).
	\end{align*}
	Plugging the bound for $\kappa(\G)$ from Lemma~\ref{lem:G-bound} finishes the proof.
\end{proof}
\begin{corollary}\label{cor:bias-tail1}
For any psd matrix $\Q\succeq0$, we have:
\iffalse
	\begin{align*}
		\norm{\A^{n+1-j} \B^j \thetat[0] \thetat[0]\T} &\leq \frac{12\sqrt{2}(n+1-j)\cnH}{\sqrt{1-\alpha^2}} \alpha^{\frac{n-j}{2}} \left(1-\frac{ \ctwo \cthree \sqrt{2\cone-\cone^2} }{\sqrt{\cnH\cnS}}\right)^j \twonorm{\thetat[0]}^2\\
		&\leq\frac{12\sqrt{2}(n+1-j)\cnH}{\sqrt{1-\alpha^2}} \alpha^{\frac{n-j}{2}} \exp\left(\frac{-j \ctwo \cthree \sqrt{2\cone-\cone^2} }{\sqrt{\cnH\cnS}}\right) \twonorm{\thetat[0]}^2.
	\end{align*}
\fi
	\begin{align*}
		\norm{\A^{n+1-j} \B^j \Q} &\leq \frac{12\sqrt{2}(n+1-j)\cnH}{\sqrt{1-\alpha^2}} \alpha^{\frac{n-j}{2}} \left(1-\frac{ \ctwo \cthree \sqrt{2\cone-\cone^2} }{\sqrt{\cnH\cnS}}\right)^j \twonorm{\Q}\\
		&\leq\frac{12\sqrt{2}(n+1-j)\cnH}{\sqrt{1-\alpha^2}} \alpha^{\frac{n-j}{2}} \exp\left(\frac{-j \ctwo \cthree \sqrt{2\cone-\cone^2} }{\sqrt{\cnH\cnS}}\right) \twonorm{\Q}.
	\end{align*}
\end{corollary}
\begin{proof}
	This corollary follows directly from Lemmas~\ref{lem:eig-A} and~\ref{lem:B-contraction} and using the fact that $1-x\leq e^{-x}$
\end{proof}


The following lemma bounds the total error of $\thetavb^{\textrm{bias}}$.
\begin{lemma}\label{lem:bound-bias}
\iffalse
	\begin{align*}
	&\iprod{\begin{bmatrix}
			\Cov & \zero \\ \zero & \zero
		\end{bmatrix}}{\E{\thetavb^{\textrm{bias}} \otimes \thetavb^{\text{bias}}}} \\ &\leq \frac{1792}{(n-t)^2(\cone\cfour)^{5/4}}\cdot\frac{(\cnH\cnS)^{9/4}d}{\delta\cfour}\cdot\exp\bigg(-(t+1)\frac{\ctwo\cthree\sqrt{2\cone-\cone^2}}{\sqrt{\cnH\cnS}}\bigg)\norm{\thetat[0]}^2 +\\&\qquad\qquad \frac{5376}{(\cone\cfour)^{1/4}}\frac{(\cnH\cnS)^{5/4}d}{\delta\cfour}\exp\left(\frac{-n \ctwo \cthree \sqrt{2\cone-\cone^2} }{\sqrt{\cnH\cnS}}\right) \cdot \norm{\thetat[0]}^2.
	\end{align*}
	\fi
	\iffalse
	\begin{align*}
		&\iprod{\begin{bmatrix}
			\Cov & \zero \\ \zero & \zero
			\end{bmatrix} }{\E{\thetavb^{\textrm{bias}} \otimes \thetavb^{\text{bias}}}} \leq			\UC\cdot\frac{(\cnH\cnS)^{9/4}d}{\delta}\cdot\exp\bigg(-(t+1)\frac{\ctwo\cthree\sqrt{2\cone-\cone^2}}{\sqrt{\cnH\cnS}}\bigg)\norm{\thetat[0]}^2 \nonumber\\&\qquad\qquad\qquad+  \UC\cdot\frac{(\cnH\cnS)^{5/4}d}{\delta}(n-t)\exp\left(\frac{-n \ctwo \cthree \sqrt{2\cone-\cone^2} }{\sqrt{\cnH\cnS}}\right) \cdot \norm{\thetat[0]}^2
			\end{align*}	
			\fi
	\begin{align*}
		&\iprod{\begin{bmatrix}
			\Cov & \zero \\ \zero & \zero
			\end{bmatrix} }{\E{\thetavb^{\textrm{bias}} \otimes \thetavb^{\text{bias}}}} \leq		\UC\cdot\frac{(\cnH\cnS)^{9/4}d\cnH}{(n-t)^2}\cdot\exp\bigg(-(t+1)\frac{\ctwo\cthree\sqrt{2\cone-\cone^2}}{\sqrt{\cnH\cnS}}\bigg)\cdot \big(P(\x_0)-P(\xs)\big) \nonumber\\&\qquad\qquad\qquad+  \UC\cdot(\cnH\cnS)^{5/4}d\cnH\cdot\exp\left(\frac{-n \ctwo \cthree \sqrt{2\cone-\cone^2} }{\sqrt{\cnH\cnS}}\right) \cdot \big(P(\x_0)-P(\xs)\big)
			\end{align*}	
			Where, $\UC$ is a universal constant.
\end{lemma}
\begin{proof}
	Lemma~\ref{lem:average-covar-bias} tells us that
	\begin{align}
		\E{\thetavb^{\textrm{bias}} \otimes \thetavb^{\text{bias}}} &= \frac{1}{(n-t)^2} \bigg( \eyeT + (\eyeT-\AL)^{-1}\AL + (\eyeT-\AR\T)^{-1}\AR\T\bigg) (\eyeT-\BT)^{-1}(\BT^{t+1}-\BT^{n+1}) \left(\thetat[0]\otimes \thetat[0]\right) \nonumber \\	 &\quad -\frac{1}{(n-t)^2}\sum_{j=t+1}^n\bigg( (\eyeT-\AL)^{-1}\AL^{n+1-j} + (\eyeT-\AR\T)^{-1}(\AR\T)^{n+1-j} \bigg)\BT^j \thetat[0] \otimes \thetat[0] . \label{eqn:bias-main-1}
	\end{align}
	We now use lemmas in this section to bound inner product of the two terms in the above expression with $\begin{bmatrix}\H&0\\0&0\end{bmatrix}$, i.e. we seek to bound,
	\begin{align}
		&\iprod{\begin{bmatrix}\H&0\\0&0\end{bmatrix}}{\E{\thetavb^{\textrm{bias}} \otimes \thetavb^{\text{bias}}}}\nonumber\\ &=\iprod{\begin{bmatrix}\H&0\\0&0\end{bmatrix}}{\frac{1}{(n-t)^2} \bigg( \eyeT + (\eyeT-\AL)^{-1}\AL + (\eyeT-\AR\T)^{-1}\AR\T\bigg) (\eyeT-\BT)^{-1}(\BT^{t+1}-\BT^{n+1}) \left(\thetat[0]\otimes \thetat[0]\right)}\nonumber\\
&+\iprod{\begin{bmatrix}\H&0\\0&0\end{bmatrix}}{-\frac{1}{(n-t)^2}\sum_{j=t+1}^n\bigg( (\eyeT-\AL)^{-1}\AL^{n+1-j} + (\eyeT-\AR\T)^{-1}(\AR\T)^{n+1-j} \bigg)\BT^j \thetat[0] \otimes \thetat[0]}\label{eqn:bias-main}
	\end{align}
	For the first term of equation~\ref{eqn:bias-main}, we have
	\begin{align}
		&\iprod{\begin{bmatrix}
			\Cov & \zero \\ \zero & \zero
			\end{bmatrix}}{\bigg( \eyeT + (\eyeT-\AL)^{-1}\AL + (\eyeT-\AR\T)^{-1}\AR\T\bigg) (\eyeT-\BT)^{-1}(\BT^{t+1}-\BT^{n+1}) \left(\thetat[0]\otimes \thetat[0]\right)} \nonumber \\ 
		&=\left\langle{\begin{bmatrix}
			\Cov & \zero \\ \zero & \zero
			\end{bmatrix}},\bigg( \eyeT + (\eyeT-\AL)^{-1}\AL + (\eyeT-\AR\T)^{-1}\AR\T\bigg) \inv{\left(\eyeT-\AL\AR\T\right)} \left(\eyeT-\AL\AR\T\right)\right. \nonumber \\ &\qquad \qquad \qquad \qquad \qquad \qquad \qquad \qquad \qquad \qquad \qquad \qquad \left.  (\eyeT-\BT)^{-1}(\BT^{t+1}-\BT^{n+1}) \left(\thetat[0]\otimes \thetat[0]\right)\right \rangle \nonumber \\ 
		&=\left\langle{\begin{bmatrix}
			\Cov & \zero \\ \zero & \zero
			\end{bmatrix}},(\eyeT-\AL)^{-1} (\eyeT-\AR\T)^{-1} \left(\eyeT-\AL\AR\T\right) (\eyeT-\BT)^{-1}(\BT^{t+1}-\BT^{n+1}) \left(\thetat[0]\otimes \thetat[0]\right)\right \rangle\nonumber \\ &\qquad \qquad \qquad \qquad \qquad \qquad \qquad \qquad \qquad \qquad \qquad \qquad \qquad\qquad\qquad\qquad\qquad \left(\mbox{using Lemma~\ref{lem:lhs-psd-lemma}}\right)\nonumber \\ 
		&= \iprod{(\eye-\A\T)^{-1} \begin{bmatrix}
			\Cov & \zero \\ \zero & \zero
			\end{bmatrix} (\eye-\A)^{-1} }{\left(\eyeT-\DT\right) (\eyeT-\BT)^{-1}(\BT^{t+1}-\BT^{n+1}) \left(\thetat[0]\otimes \thetat[0]\right)} \nonumber \\
		&\leq \frac{1}{(\g-c\delta)^2}\frac{4\cnH}{\sqrt{1-\alpha^2}} \exp\left(-(t+1) \ctwo \cthree \sqrt{2\cone - \cone^2}/\sqrt{\cnH\cnS}\right) \norm{\thetat[0]}^2\nonumber\\&\qquad\qquad\qquad\qquad\qquad\qquad \iprod{\bigg(\otimes_2\begin{bmatrix} -(c\eye-\g\Cov)\Cov^{-1/2}\\(\eye-\delta\Cov)\Cov^{-1/2}\end{bmatrix}\bigg)}{\eye + 2 \sqrt{\cnH\cnS} (\infbound/\sigma^2) \Sighat}.\nonumber
	\end{align}
	The two terms above can be bounded as
	\begin{align*}
		\iprod{\bigg(\otimes_2\begin{bmatrix} -(c\eye-\g\Cov)\Cov^{-1/2}\\(\eye-\delta\Cov)\Cov^{-1/2}\end{bmatrix}\bigg)}{\eye} &\leq 7\cdot \trace{\inv{\Cov}} \leq \frac{7d}{\mu} \mbox{ and,} \\
		2 \sqrt{\cnH\cnS} (\infbound/\sigma^2) \iprod{\bigg(\otimes_2\begin{bmatrix} -(c\eye-\g\Cov)\Cov^{-1/2}\\(\eye-\delta\Cov)\Cov^{-1/2}\end{bmatrix}\bigg)}{\Sighat} &=  2 \sqrt{\cnH\cnS} \infbound (\g - c\delta)^2 d.
	\end{align*}
	Combining the above and noting the fact that $2 \sqrt{\cnH\cnS} \infbound (\g - c\delta)^2 d<\frac{7 d}{\mu}$, we have
	\begin{align}
		&\iprod{\begin{bmatrix}
			\Cov & \zero \\ \zero & \zero
			\end{bmatrix}}{\bigg( \eyeT + (\eyeT-\AL)^{-1}\AL + (\eyeT-\AR\T)^{-1}\AR\T\bigg) (\eyeT-\BT)^{-1}(\BT^{t+1}-\BT^{n+1}) \left(\thetat[0]\otimes \thetat[0]\right)} \nonumber\\ 
		&\qquad \leq \frac{56 \cnH d }{\sqrt{1-\alpha^2}} \cdot \frac{\norm{\thetat[0]}^2}{\mu \left(\g - c \delta\right)^2} \cdot \exp\left(-(t+1) \ctwo \cthree \sqrt{2\cone - \cone^2}/\sqrt{\cnH\cnS}\right).\label{eqn:bias-11}
	\end{align}
	We now note the following facts:
	\begin{align*}
		&\frac{1}{1-\alpha}=\frac{\ctwo\sqrt{2\cone-\cone^2}}{\sqrt{\cnH\cnS}+\ctwo\sqrt{2\cone-\cone^2}}\leq\frac{2}{\sqrt{\cone\cfour}}\cdot\sqrt{\cnH\cnS}\\
		&\frac{1}{\g-c\delta}\leq\frac{1}{\gamma(1-\alpha)}\leq\frac{\mu}{(1-\alpha)^2}\leq\frac{4\cnS}{\cfour\delta}
	\end{align*}
	This implies, equation~\ref{eqn:bias-11} can be bounded as:
	\begin{align}
		&\iprod{\begin{bmatrix}
			\Cov & \zero \\ \zero & \zero
			\end{bmatrix}}{\bigg( \eyeT + (\eyeT-\AL)^{-1}\AL + (\eyeT-\AR\T)^{-1}\AR\T\bigg) (\eyeT-\BT)^{-1}(\BT^{t+1}-\BT^{n+1}) \left(\thetat[0]\otimes \thetat[0]\right)} \nonumber\\ 
&\qquad\leq\frac{1792}{(\cone\cfour)^{5/4}}\cdot\frac{(\cnH\cnS)^{9/4}d}{\delta\cfour}\cdot\exp\bigg(-(t+1)\frac{\ctwo\cthree\sqrt{2\cone-\cone^2}}{\sqrt{\cnH\cnS}}\bigg)\norm{\thetat[0]}^2\nonumber\\
&\qquad\leq\frac{1792}{(\cone\cfour)^{5/4}}\cdot\frac{(\cnH\cnS)^{9/4}d\cnH}{\cone\cfour}\cdot\exp\bigg(-(t+1)\frac{\ctwo\cthree\sqrt{2\cone-\cone^2}}{\sqrt{\cnH\cnS}}\bigg)\mu\norm{\thetat[0]}^2\nonumber\\
&\qquad\leq\frac{3584}{(\cone\cfour)^{5/4}}\cdot\frac{(\cnH\cnS)^{9/4}d\cnH}{\cone\cfour}\cdot\exp\bigg(-(t+1)\frac{\ctwo\cthree\sqrt{2\cone-\cone^2}}{\sqrt{\cnH\cnS}}\bigg)\cdot\big(P(\x_0)-P(\xs)\big)\nonumber\\
&\qquad\leq\UC\cdot(\cnH\cnS)^{9/4}d\cnH\cdot\exp\bigg(-(t+1)\frac{\ctwo\cthree\sqrt{2\cone-\cone^2}}{\sqrt{\cnH\cnS}}\bigg)\cdot\big(P(\x_0)-P(\xs)\big).\label{eqn:bias-1}
	\end{align}
	Where, $\UC$ is a universal constant.
	
	Consider now a term in the summation in the second term of~\eqref{eqn:bias-main}.
	\begin{align}
		&\iprod{\begin{bmatrix}
			\Cov & \zero \\ \zero & \zero
			\end{bmatrix} }{\bigg( (\eyeT-\AL)^{-1}\AL^{n+1-j} + (\eyeT-\AR\T)^{-1}(\AR\T)^{n+1-j} \bigg)\BT^j \left({\thetat[0] \otimes \thetat[0]}\right)}  \nonumber \\
		&= \iprod{(\eye-\A\T)^{-1}\begin{bmatrix}
			\Cov & \zero \\ \zero & \zero
			\end{bmatrix} }{\A^{n+1-j} \BT^j \left({\thetat[0] \otimes \thetat[0]}\right)} \nonumber\\&+ \iprod{\begin{bmatrix}
			\Cov & \zero \\ \zero & \zero
			\end{bmatrix} (\eye-\A)^{-1}}{\bigg(\BT^j \left({\thetat[0] \otimes \thetat[0]}\right)\bigg) (\A\T)^{n+1-j} } \nonumber \\
		&\leq 4 d \norm{(\eye-\A\T)^{-1}\begin{bmatrix}
			\Cov & \zero \\ \zero & \zero
			\end{bmatrix} } \norm{\A^{n+1-j} \BT^j \left({\thetat[0] \otimes \thetat[0]}\right)} \nonumber \\
		&\leq \frac{4d}{\g-c\delta} \norm{\begin{bmatrix}-(c\eye-\g\Cov)&0\\(\eye-\delta\Cov)&0\end{bmatrix}} \cdot \frac{12\sqrt{2}(n+1-j)\cnH}{\sqrt{1-\alpha^2}} \alpha^{\frac{n-j}{2}} \exp\left(\frac{-j \ctwo \cthree \sqrt{2\cone-\cone^2} }{\sqrt{\cnH\cnS}}\right) \norm{\thetat[0]}^2  \nonumber \\ & \qquad \qquad \qquad \qquad \qquad \left(\mbox{Lemma~\ref{lem:com1} and Corollary~\ref{cor:bias-tail1}}\right) \nonumber \\
		&\leq \frac{672 (n-t) d \cnH}{(\g - c\delta)\sqrt{1-\alpha^2}} \cdot \exp\left(\frac{-n \ctwo \cthree \sqrt{2\cone-\cone^2} }{\sqrt{\cnH\cnS}}\right) \cdot \norm{\thetat[0]}^2\nonumber\\
		&\leq\frac{5376}{(\cone\cfour)^{1/4}}\frac{(\cnH\cnS)^{5/4}d}{\delta\cfour}(n-t)\exp\left(\frac{-n \ctwo \cthree \sqrt{2\cone-\cone^2} }{\sqrt{\cnH\cnS}}\right) \cdot \norm{\thetat[0]}^2\nonumber\\		
		&\leq\frac{5376}{(\cone\cfour)^{1/4}}\frac{(\cnH\cnS)^{5/4}d\cnH}{\cone\cfour}(n-t)\exp\left(\frac{-n \ctwo \cthree \sqrt{2\cone-\cone^2} }{\sqrt{\cnH\cnS}}\right) \cdot \mu\norm{\thetat[0]}^2\nonumber\\		
		&\leq\frac{10752}{(\cone\cfour)^{1/4}}\frac{(\cnH\cnS)^{5/4}d\cnH}{\cone\cfour}(n-t)\exp\left(\frac{-n \ctwo \cthree \sqrt{2\cone-\cone^2} }{\sqrt{\cnH\cnS}}\right) \cdot \big(P(\x_0)-P(\xs)\big)\nonumber\\		
		&\leq\UC\cdot(\cnH\cnS)^{5/4}d\cnH\cdot(n-t)\exp\left(\frac{-n \ctwo \cthree \sqrt{2\cone-\cone^2} }{\sqrt{\cnH\cnS}}\right) \cdot \big(P(\x_0)-P(\xs)\big).\label{eqn:bias-2}
	\end{align}
	Where, $\UC$ is a universal constant.
	Plugging~\eqref{eqn:bias-1} and~\eqref{eqn:bias-2} into~\eqref{eqn:bias-main}, we obtain
	\iffalse
	\begin{align*}
		&\iprod{\begin{bmatrix}
			\Cov & \zero \\ \zero & \zero
			\end{bmatrix} }{\E{\thetavb^{\textrm{bias}} \otimes \thetavb^{\text{bias}}}} \\ &\leq \frac{1792}{(n-t)^2(\cone\cfour)^{5/4}}\cdot\frac{(\cnH\cnS)^{9/4}d}{\delta\cfour}\cdot\exp\bigg(-(t+1)\frac{\ctwo\cthree\sqrt{2\cone-\cone^2}}{\sqrt{\cnH\cnS}}\bigg)\norm{\thetat[0]}^2 + \\ &\qquad\qquad\frac{5376}{(\cone\cfour)^{1/4}}\frac{(\cnH\cnS)^{5/4}d}{\delta\cfour}\exp\left(\frac{-n \ctwo \cthree \sqrt{2\cone-\cone^2} }{\sqrt{\cnH\cnS}}\right) \cdot \norm{\thetat[0]}^2
			\end{align*}
			\fi
	\begin{align*}
		&\iprod{\begin{bmatrix}
			\Cov & \zero \\ \zero & \zero
			\end{bmatrix} }{\E{\thetavb^{\textrm{bias}} \otimes \thetavb^{\text{bias}}}}\nonumber\\ &\leq			\UC\cdot\frac{(\cnH\cnS)^{9/4}d\cnH}{(n-t)^2}\cdot\exp\bigg(-(t+1)\frac{\ctwo\cthree\sqrt{2\cone-\cone^2}}{\sqrt{\cnH\cnS}}\bigg)\cdot \big(P(\x_0)-P(\xs)\big) \nonumber\\&\qquad\qquad\qquad+  \UC\cdot(\cnH\cnS)^{5/4}d\cnH\cdot\exp\left(\frac{-n \ctwo \cthree \sqrt{2\cone-\cone^2} }{\sqrt{\cnH\cnS}}\right) \cdot \big(P(\x_0)-P(\xs)\big)
			\end{align*}
	This proves the lemma.
%\pagebreak
\end{proof}

\section{Lemmas and proofs for Bounding variance error}
\label{sec:varianceContraction}
%\begin{proof}[Proof of Lemma~\ref{lem:main-variance}]
Before we prove lemma~\ref{lem:main-variance}, we recall old notation and introduce new notations that will be employed in these proofs.
\subsection{Notations}
We begin with by recalling that we track $\thetav_k=\begin{bmatrix}\x_k-\xs\\\y_k-\xs\end{bmatrix}$. Given $\thetav_k$, we recall the recursion governing the evolution of $\thetav_k$:
\begin{align}
\label{eq:simpleXYRec}
\thetav_{k+1}&=\begin{bmatrix}0&\eye-\delta\widehat{\H}_{k+1}\\-c\cdot\eye&(1+c)\eye-\g\cdot\widehat{\H}_{k+1}\end{bmatrix}\thetav_k+\begin{bmatrix}\delta\cdot\epsilon_{k+1}\av_{k+1}\\\g\cdot\epsilon_{k+1}\av_{k+1}\end{bmatrix}\nonumber\\
&=\widehat{\A}_{k+1}\thetav_{k}+\zetav_{k+1}
\end{align}
where, recall, $c=\alpha(1-\beta),\  \g=\alpha\delta+(1-\alpha)\gamma$, and $\widehat{\H}_{k+1}=\a_{k+1}\a_{k+1}\T$. Furthermore, we recall the following definitions, which will be heavily used in the following proofs:
\begin{align*}
\A&=\E{\Ah_{k+1}|\mathcal{F}_{k}}\\
\BT&=\E{\Ah_{k+1}\otimes\Ah_{k+1}|\mathcal{F}_{k}}\\
\Sigh&=\E{\zetav_{k+1}\otimes\zetav_{k+1}|\mathcal{F}_{k}}=\begin{bmatrix}\delta^2&\delta\cdot \g\\\delta\cdot \g&\g^2\end{bmatrix}\otimes\Sig\preceq\sigma^2\cdot\begin{bmatrix}\delta^2&\delta\cdot \g\\\delta\cdot \g&\g^2\end{bmatrix}\otimes\Cov
\end{align*}
\iffalse
Finally, we observe the following:
\begin{align*}
\E{(\A-\Ah_{k+1})\otimes(\A-\Ah_{k+1})|\mathcal{F}_k}&=\A\otimes\A-\E{\Ah_{k+1}\otimes\A|\mathcal{F}_k}-\E{\Ah_{k+1}\otimes\A|\mathcal{F}_k}+\E{\Ah_{k+1}\otimes\Ah_{k+1}|\mathcal{F}_k}\\
&=-\A\otimes\A+\E{\Ah_{k+1}\otimes\Ah_{k+1}|\mathcal{F}_k}\\
\implies \E{\Ah_{k+1}\otimes\Ah_{k+1}|\mathcal{F}_k}&=\E{(\A-\Ah_{k+1})\otimes(\A-\Ah_{k+1})|\mathcal{F}_k}+\A\otimes\A
\end{align*}
\fi
We recall:
\begin{align*}
\RT&=\E{(\A-\Ah_{k+1})\otimes(\A-\Ah_{k+1})|\mathcal{F}_k}\\
\DT&=\A\otimes\A
\end{align*}
And the operators $\BT,\DT,\RT$ being related by:
\begin{align*}
\BT=\DT+\RT
\end{align*}
Furthermore, in order to compute the steady state distribution with the fourth moment quantities in the mix, we need to rely on the following re-parameterization of the update matrix $\Ah$:
\begin{align*}
\Ah&=\begin{bmatrix}0&\eye-\delta\widehat{\H}\\-c\cdot\eye&(1+c)\cdot\eye-\g\cdot\widehat{\H}\end{bmatrix}\\
&=\begin{bmatrix}0&\eye\\-c\cdot\eye&(1+c)\cdot\eye\end{bmatrix}+\begin{bmatrix}0&-\delta\cdot\widehat{\H}\\0&-\g\cdot\widehat{\H}\end{bmatrix}\\
&\defeq \V_1+\Vh_2
\end{align*}
This implies in particular:
\begin{align*}
\Ah\otimes\Ah&=(\V_1+\Vh_2)\otimes(\V_1+\Vh_2)\\
&=\V_1\otimes\V_1 + \V_1\otimes\Vh_2 + \Vh_2\otimes\V_1 + \Vh_2\otimes\Vh_2
\end{align*}
Note in particular, the fourth moment part resides in the operator $\Vh_2\otimes\Vh_2$. Terms such as $\V_1\otimes\V_1$ are deterministic, or terms such as $\V_1\otimes\Vh_2$ or $\Vh_2\otimes\V_1$ contain second moment quantities. Furthermore, note that the operator $\BT=\E{\Ah\otimes\Ah}$ where the expectation is taken with respect to a single random draw from the distribution $\mathcal{D}$.

Considering the expectation of $\Ah\otimes\Ah$ with respect to a single draw from the distribution $\mathcal{D}$, we have:
\begin{align}
\BT=\E{\Ah\otimes\Ah}&=\V_1\otimes\V_1 + \E{\V_1\otimes\Vh_2} + \E{\Vh_2\otimes\V_1} + \E{\Vh_2\otimes\Vh_2}\nonumber\\
&=\V_1\otimes\V_1 + \V_1\otimes\V_2 + \V_2\otimes\V_1 + \E{\Vh_2\otimes\Vh_2},\nonumber
\end{align}
where $\V_2\defeq\E{\Vh_2} =\begin{bmatrix}0&-\delta\cdot\H\\0&-\g\cdot\H\end{bmatrix}$. 

Finally, we let $\text{nr}$ and $\text{dr}$ to denote the numerator and denominator respectively.

\subsection{An exact expression for the stationary distribution}
Note that a key term appearing in the expression for covariance of the variance equation~\eqref{eq:varianceTA} is $\inv{\left(\eyeT - \BT\right)}\Sighat$. This is in fact nothing but the covariance of the error when we run accelerated SGD forever starting at $\xs$ (i.e., at steady state). This can be seen by
%In order to compute the steady state distribution $\phiv_{\infty}$, we have to understand what it is in the first place, when we deal with bounding the variance error. In particular, we begin by noting that for the bounding the variance error, we start at the solution, i.e. $\thetav_0=0$, and analyze the stochastic process driven purely by noise. In particular, let us
considering the base variance recursion using equation~\eqref{eq:simpleXYRec}:
\begin{align*}
\thetav_{k}&=\Ah_{k}\thetav_{k-1}+\zetav_k\nonumber\\
\implies \phiv_k&\defeq\E{\thetav_k\otimes\thetav_k}\nonumber\\
&=\E{\E{\bigg(\Ah_{k}\thetav_{k-1}\otimes\thetav_{k-1}\Ah_{k}\T+\zetav_k\otimes\zetav_k\bigg)|\mathcal{F}_{k-1}}}\nonumber\\
&=\E{\E{\bigg(\Ah_{k}\thetav_{k-1}\otimes\thetav_{k-1}\Ah_{k}\T\bigg)|\mathcal{F}_{k-1}}}+\Sigh\nonumber\\
&=\BT\cdot\E{\thetav_{k-1}\otimes\thetav_{k-1}}+\Sigh\nonumber\\
&=\BT\cdot\phiv_{k-1}+\Sigh
\end{align*}
This recursion on the covariance operator $\phiv_{k}$ can be unrolled until the start i.e. $k=0$ to yield:
\begin{align}
\label{eq:steadyStateExp}
\phiv_k&=\BT^k\phiv_0 + \sum_{l=0}^{k-1} \BT^l\cdot\Sigh\nonumber\\
&=(\eyeT-\BT)^{-1}(\eyeT-\BT^{k})\Sigh\quad\quad\quad(\because\ \phiv_0=0)\nonumber\\
\implies\phiv_{\infty}&=\lim_{k\to\infty}\phiv_k=(\eyeT-\BT)^{-1}\Sigh
\end{align}

\iffalse
At this point, we will state the following lemma that presents a sharp bound on the steady state covariance of $\y-\xs$, i.e. $\E{(\y_{\infty}-\xs)\otimes(\y_{\infty}-\xs)}$:
\begin{lemma}
The Covariance of the steady state parameters of $\y_{\infty}-\xs$, i.e. $\E{(\y_{\infty}-\xs)\otimes(\y_{\infty}-\xs)}$ is upper bounded by:
\begin{align*}
\U_{22}\defeq\E{(\y_{\infty}-\xs)\otimes(\y_{\infty}-\xs)}=
\end{align*}
\end{lemma}
\fi

\subsection{Computing the steady state distribution}

We now proceed to compute the stationary distribution.
%Given the expression for the stationary distribution as in equation~\ref{eq:steadyStateExp}, we can now recall the following:
Recall that
\begin{align*}
\BT &= \V_1\otimes\V_1 + \V_1\otimes\V_2 + \V_2\otimes\V_1 + \E{\Vh_2\otimes\Vh_2}\nonumber\\
\implies \eyeT-\BT &= \big(\eyeT-\V_1\otimes\V_1 - \V_1\otimes\V_2 - \V_2\otimes\V_1\big)-\E{\Vh_2\otimes\Vh_2}
\end{align*}
Where the expectation is over a single sample drawn from the distribution $\mathcal{D}$.
This implies in particular,
\begin{align}
\label{eq:ibinv}
&(\eyeT-\BT)^{-1}=\bigg(\big(\eyeT-\V_1\otimes\V_1 - \V_1\otimes\V_2 - \V_2\otimes\V_1\big)-\E{\Vh_2\otimes\Vh_2}\bigg)^{-1}\nonumber\\
&=\sum_{k=0}^{\infty}\bigg(\big(\eyeT-\V_1\otimes\V_1 - \V_1\otimes\V_2 - \V_2\otimes\V_1\big)^{-1}\E{\Vh_2\otimes\Vh_2}\bigg)^k\nonumber\\&\qquad\qquad\qquad\qquad\qquad\qquad\cdot\big(\eyeT-\V_1\otimes\V_1 - \V_1\otimes\V_2 - \V_2\otimes\V_1\big)^{-1}
\end{align}
Since $\Sighat \preceq \sigma^2\cdot\begin{bmatrix}\delta^2&\delta\cdot \g\\\delta\cdot \g&\g^2\end{bmatrix}\otimes\Cov$, and $\inv{\left(\eyeT - \BT\right)}$ is a PSD operator, the steady state distribution $\phiv_\infty$ is bounded by:
\begin{align}
\label{eq:phivInftyBound}
\phiv_\infty&=(\eyeT-\BT)^{-1}\Sigh \preceq \sigma^2 (\eyeT-\BT)^{-1} \left(\begin{bmatrix}\delta^2&\delta\cdot \g\\\delta\cdot \g&\g^2\end{bmatrix}\otimes\Cov\right) \nonumber\\
&=\sigma^2 \sum_{k=0}^{\infty}\bigg(\big(\eyeT-\V_1\otimes\V_1 - \V_1\otimes\V_2 - \V_2\otimes\V_1\big)^{-1}\E{\Vh_2\otimes\Vh_2}\bigg)^k \cdot \nonumber \\
&\qquad \qquad \quad \big(\eyeT-\V_1\otimes\V_1 - \V_1\otimes\V_2 - \V_2\otimes\V_1\big)^{-1} \left(\begin{bmatrix}\delta^2&\delta\cdot \g\\\delta\cdot \g&\g^2\end{bmatrix}\otimes\Cov\right).
\end{align}
Note that the Taylor expansion above is guaranteed to be correct if the right hand side is finite. We will understand bounds on the steady state distribution by splitting the analysis into the following parts:
\begin{itemize}
\item Obtain $\U\eqdef\big(\eyeT-\V_1\otimes\V_1 - \V_1\otimes\V_2 - \V_2\otimes\V_1\big)^{-1} \left(\begin{bmatrix}\delta^2&\delta\cdot \g\\\delta\cdot \g&\g^2\end{bmatrix}\otimes\Cov\right)$ (in section~\ref{ssec:secMomentEffects}).
\item Obtain bounds on $\E{\Vh_2\otimes\Vh_2}\U$ (in section~\ref{ssec:fourthMomentEffects})
\item Combine the above to obtain bounds on $\phiv_{\infty}$ (lemma~\ref{lem:main-variance}).
\end{itemize}
Before deriving these bounds, we will present some reasoning behind the validity of the upper bounds that we derive on the stationary distribution $\phiv_{\infty}$:
\begin{align}
\label{eq:phivInftyUpperBound}
\phiv_\infty&=(\eyeT-\BT)^{-1}\Sigh\nonumber\\
&\preceq\sigma^2 \sum_{k=0}^{\infty}\bigg(\big(\eyeT-\V_1\otimes\V_1 - \V_1\otimes\V_2 - \V_2\otimes\V_1\big)^{-1}\E{\Vh_2\otimes\Vh_2}\bigg)^k\U\quad(***)\nonumber\\
&=\sigma^2\U+\sigma^2\sum_{k=1}^{\infty}\bigg(\big(\eyeT-\V_1\otimes\V_1 - \V_1\otimes\V_2 - \V_2\otimes\V_1\big)^{-1}\E{\Vh_2\otimes\Vh_2}\bigg)^k\U\nonumber\\
&=\sigma^2\U+\sigma^2\sum_{k=0}^{\infty}\bigg(\big(\eyeT-\V_1\otimes\V_1 - \V_1\otimes\V_2 - \V_2\otimes\V_1\big)^{-1}\E{\Vh_2\otimes\Vh_2}\bigg)^k\nonumber\\
&\qquad\qquad\qquad\qquad\qquad\cdot\big(\eyeT-\V_1\otimes\V_1 - \V_1\otimes\V_2 - \V_2\otimes\V_1\big)^{-1}\E{\Vh_2\otimes\Vh_2}\U\nonumber\\
&=\sigma^2\U+\sigma^2(\eyeT-\BT)^{-1}\cdot\E{\Vh_2\otimes\Vh_2}\U\qquad\qquad\qquad\qquad\qquad\qquad(\text{using equation}~\ref{eq:ibinv}),
\end{align}
with $(***)$ following through using equation~\ref{eq:phivInftyBound} and through the definition of $\U$.
Now, with this in place, we clearly see that since $(\eyeT-\BT)^{-1}$ and $\E{\Vh_2\otimes\Vh_2}$ are PSD operators, we can upper bound right hand side to create valid PSD upper bounds on $\phiv_{\infty}$. In particular, in section~\ref{ssec:secMomentEffects}, we derive with equality what $\U$ is, and follow that up with computation of an upper bound on $\E{\Vh_2\otimes\Vh_2}\U$ in section~\ref{ssec:fourthMomentEffects}. Combining this will enable us to present a valid PSD upper bound on $\phiv_{\infty}$ owing to equation~\ref{eq:phivInftyUpperBound}.

\subsubsection{Understanding the second moment effects}\label{ssec:secMomentEffects}
This part of the proof deals with deriving the solution to:
\begin{align*}
\U&=\big(\eyeT-\V_1\otimes\V_1 - \V_1\otimes\V_2 - \V_2\otimes\V_1\big)^{-1} \left(\begin{bmatrix}\delta^2&\delta\cdot \g \\\delta\cdot \g &\g ^2\end{bmatrix}\otimes\Cov\right)
\end{align*}
This is equivalent to solving the (linear) equation:
\begin{align}
\label{eq:linEq}
\big(\eyeT-\V_1\otimes\V_1 - \V_1\otimes\V_2 - \V_2\otimes\V_1\big)\cdot\U&= \left(\begin{bmatrix}\delta^2&\delta\cdot \g \\\delta\cdot \g &\g ^2\end{bmatrix}\otimes\Cov\right) \nonumber\\
\implies\U-\V_1\U\V_1\T-\V_1\U\V_2\T-\V_2\U\V_1\T&= \left(\begin{bmatrix}\delta^2&\delta\cdot \g \\\delta\cdot \g &\g ^2\end{bmatrix}\otimes\Cov\right)
\end{align}
Note that all the known matrices above i.e., $\V_1, \V_2$ and $\Cov$ are all diagonalizable with respect to $\Cov$, and thus, the solution of this system can be computed in each of the eigenspaces $(\lambda_j,\u_j)$ of $\Cov$. This implies, in reality, we deal with matrices $\U^{(j)}$, one corresponding to each eigenspace. However, for this section, we will neglect the superscript on $\U$, since it is clear from context for the purpose of this section.
\begin{align*}
\V_1\U\V_1\T&=\begin{bmatrix}0&1\\-c&1+c\end{bmatrix}\begin{bmatrix}u_{11}&u_{12}\\u_{12}&u_{22}\end{bmatrix}\begin{bmatrix}0&-c\\1&1+c\end{bmatrix}\\
&=\begin{bmatrix}u_{22}&-c u_{12}+(1+c)u_{22}\\-c u_{12}+(1+c)u_{22}&c^2u_{11}-2c(1+c)u_{12}+(1+c)^2u_{22}\end{bmatrix}
\end{align*}
Next,
\begin{align*}
\V_1\U\V_2\T&=\begin{bmatrix}0&1\\-c&1+c\end{bmatrix}\begin{bmatrix}u_{11}&u_{12}\\u_{12}&u_{22}\end{bmatrix}\begin{bmatrix}0 &0\\-\delta&-\g \end{bmatrix}\lambda_j\\
&=\begin{bmatrix}u_{12}&u_{22}\\-cu_{11}+(1+c)u_{12}&-cu_{12}+(1+c)u_{22}\end{bmatrix}\begin{bmatrix}0 &0\\-\delta&-\g \end{bmatrix}\lambda_j\\
&=\begin{bmatrix}-\delta u_{22}&-\g  u_{22}\\-\delta(-c u_{12}+(1+c)u_{22})&-\g (-c u_{12}+(1+c)u_{22})\end{bmatrix}\lambda_j
\end{align*}
It follows that:
\begin{align*}
\V_2\U\V_1\T&=(\V_1\U\V_2\T)\T\\
&=\begin{bmatrix}-\delta u_{22}&-\delta(-c u_{12}+(1+c)u_{22})\\-\g  u_{22}&-\g (-c u_{12}+(1+c)u_{22})\end{bmatrix}\lambda_j
\end{align*}
Given all these computations, comparing the $(1,1)$ term on both sides of equation~\ref{eq:linEq}, we get:
\begin{align}
\label{eq:t11}
u_{11}&-u_{22}+2\delta\lambda_j u_{22}=\delta^2\lambda_j\nonumber\\
u_{11}&=u_{22}(1-2\delta\lambda_j)+\delta^2\lambda_j
\end{align}
Next, comparing $(1,2)$ term on both sides of equation~\ref{eq:linEq}, we get:
\begin{align}
\label{eq:t12}
&u_{12}-(-c u_{12}+ (1+c) u_{22}) +\g  \lambda_j u_{22} + \delta\lambda_j (-c u_{12} + (1+c) u_{22})=\delta\ \g \lambda_j\nonumber\\
&u_{12}-(1-\delta\lambda_j)(-c u_{12} + (1+c) u_{22})+\g \lambda_ju_{22}=\delta\ \g \lambda_j\nonumber\\
&(1+c(1-\delta\lambda_j))\cdot u_{12}+(\g \lambda_j-(1+c)(1-\delta\lambda_j))\cdot u_{22}=\delta\ \g \lambda_j
\end{align}
Finally, comparing the $(2,2)$ term on both sides of equation~\ref{eq:linEq}, we get:
\begin{align}
\label{eq:t22}
&u_{22}-(c^2u_{11}-2c(1+c)u_{12}+(1+c)^2u_{22})+2\g \lambda_j(-cu_{12}+(1+c)u_{22})=\g ^2\lambda_j\nonumber\\
&\implies-c^2u_{11}+(2c(1+c)-2c\g \lambda_j)u_{12}+(1-(1+c)^2+2(1+c)\g \lambda_j)u_{22}=\g ^2\lambda_j\quad(\text{from equation}~\ref{eq:t11})\nonumber\\
&\implies-c^2(u_{22}(1-2\delta\lambda_j)+\delta^2\lambda_j)+(2c(1+c)-2c\g \lambda_j)u_{12}+(1-(1+c)^2+2(1+c)\g \lambda_j)u_{22}=\g ^2\lambda_j\nonumber\\
&\implies(2c(1+c)-2c\g \lambda_j)u_{12}+(1-(1+c)^2-c^2(1-2\delta\lambda_j)+2(1+c)\g \lambda_j)u_{22}=(\g ^2+c^2\delta^2)\lambda_j\nonumber\\
&\implies2c((1+c)-\g \lambda_j)u_{12}+2((1+c)(\g \lambda_j-c)+\delta\lambda_jc^2)u_{22}=(\g ^2+c^2\delta^2)\lambda_j
\end{align}
Now, we note that equations~\ref{eq:t12},~\ref{eq:t22} are linear systems in two variables $u_{12}$ and $u_{22}$. Denoting the system in the following manner,
\begin{align*}
a_{11} u_{12} + a_{12} u_{22} = b_1\\
a_{21} u_{12} + a_{22} u_{22} = b_2
\end{align*}
For analyzing the variance error, we require $u_{22},u_{12}$:
\begin{align*}
u_{22}=\frac{b_1a_{21}-b_2a_{11}}{a_{12}a_{21}-a_{11}a_{22}}, \ u_{12}=\frac{b_1a_{22}-b_2a_{12}}{a_{11}a_{22}-a_{12}a_{21}}
\end{align*}
Substituting the values from equations~\ref{eq:t12} and~\ref{eq:t22}, we get:
\begin{align}
\label{eq:u22-1}
u_{22}&=\frac{2c\g \delta\bigg(1+c-\g \lambda_j\bigg)-(\g ^2+c^2\delta^2)\bigg(1+c(1-\delta\lambda_j)\bigg)}{2c\bigg( \big(1+c-\g \lambda_j\big)\cdot \big(\lambda_j \g -(1+c)(1-\delta\lambda_j)\big) \bigg)-2\cdot\bigg(\big(1+c-c\delta\lambda_j\big)\cdot\big((1+c)(\g \lambda_j-c)+\delta\lambda_jc^2\big) \bigg)}\cdot\lambda_j
\end{align}
\begin{align}
\label{eq:u12-1}
u_{12}=\frac{2\g \delta\bigg((1+c)(\g \lambda_j-c)+\delta\lambda_jc^2\bigg)-(\g ^2+c^2\delta^2)\bigg(\lambda_j\g -(1+c)(1-\delta\lambda_j)\bigg)}{2\bigg(\big(1+c-c\delta\lambda_j\big)\cdot\big((1+c)(\g \lambda_j-c)+\delta\lambda_jc^2\big) \bigg)-2c\bigg( \big(1+c-\g \lambda_j\big)\cdot \big(\lambda_j \g -(1+c)(1-\delta\lambda_j)\big) \bigg)}\cdot\lambda_j
\end{align}
\underline{\bf Denominator of $u_{22}$}:
Let us consider the denominator of $u_{22}$ (from equation~\ref{eq:u22-1}) to write it in a concise manner.
\begin{align*}
\text{dr}(u_{22})=2 \bigg(\ \big(1+c-\g \lambda_j\big)\cdot k_1\ -\ \big(1+c-c\delta\lambda_j\big)\cdot k_2 \bigg)
\end{align*}
with 
\begin{align*}
k_1&=c\cdot\big(\lambda_j \g -(1+c)(1-\delta\lambda_j)\big)\\
&=\big(c\lambda_j \g -(c+c^2)(1-\delta\lambda_j)\big)\\
&=\big(c\g \lambda_j-c-c^2+c\delta\lambda_j+c^2\delta\lambda_j\big)\\
k_2&=\big((1+c)(\g \lambda_j-c)+\delta\lambda_jc^2\big)\\
&=\big(\g \lambda_j-c+c\g \lambda_j-c^2+\delta\lambda_jc^2\big)
\end{align*}
Plugging in expressions for $\g =\alpha\delta+(1-\alpha)\gamma$ and $c=\alpha(1-\beta)$, in $\text{dr}(u_{22})$ we get:
\begin{align}
\label{eq:dr-u22-int1}
\text{dr}(u_{22})=2\cdot\bigg(\ \big(1+c-\alpha\delta\lambda_j\big)(k_1-k_2)-\lambda_j\cdot\big((1-\alpha)\gamma k_1 + \alpha\beta\delta k_2\big)\ \bigg)
\end{align}
Next, considering $k_1-k_2$, we have:
\begin{align}
\label{eq:dr-u22-p1}
k_1-k_2&=c\lambda_j \g -c-c^2+c\delta\lambda_j+c^2\delta\lambda_j-\g \lambda_j+c-c \g \lambda_j+c^2-c^2\delta\lambda_j\nonumber\\
&=(c\delta-\g )\lambda_j\nonumber\\
&=-(\alpha\beta\delta+\gamma(1-\alpha))\lambda_j
\end{align}
Next, considering $\gamma(1-\alpha)k_1+\alpha\beta\delta\ k_2$, we have:
\begin{align*}
&\gamma(1-\alpha)k_1+\alpha\beta\delta\ k_2\\
&=\gamma(1-\alpha)(c\lambda_j \g -c-c^2+c^2\delta\lambda_j+c\delta\lambda_j)\\
&+\alpha\beta\delta(c\lambda_j \g -c-c^2+c^2\delta\lambda_j+\g \lambda_j)\\
&=(\alpha\beta\delta+(1-\alpha)\gamma)(c\lambda_j \g -c-c^2+c^2\delta\lambda_j)+\lambda_j\delta(c\gamma(1-\alpha)+\alpha\beta \g )
\end{align*}
Consider $c\gamma(1-\alpha)+\alpha\beta \g $:
\begin{align*}
c\gamma(1-\alpha)+\alpha\beta \g &=\alpha(1-\beta)\gamma(1-\alpha)+\alpha\beta(\alpha\delta+(1-\alpha)\gamma)\\
&=\alpha(1-\beta)\gamma(1-\alpha)+\alpha\beta\gamma(1-\alpha)+\alpha^2\beta\delta\\
&=\alpha\gamma(1-\alpha)+\alpha^2\beta\delta\\
&=\alpha(\alpha\beta\delta+(1-\alpha)\gamma)
\end{align*}
Re-substituting this in the expression for $\gamma(1-\alpha)k_1+\alpha\beta\delta k_2$, we have:
\begin{align}
\label{eq:dr-u22-p2}
\gamma(1-\alpha)k_1+\alpha\beta\delta\ k_2&=(\alpha\beta\delta+(1-\alpha)\gamma)(c\lambda_j \g -c-c^2+c^2\delta\lambda_j)+\lambda_j\delta(c\gamma(1-\alpha)+\alpha\beta \g )\nonumber\\
&=(\alpha\beta\delta+(1-\alpha)\gamma)(c\lambda_j \g -c-c^2+c^2\delta\lambda_j)+\alpha\lambda_j\delta(\alpha\beta\delta+(1-\alpha)\gamma)\nonumber\\
&=(\alpha\beta\delta+(1-\alpha)\gamma)(c\lambda_j \g -c-c^2+c^2\delta\lambda_j+\alpha\lambda_j\delta)
\end{align}
Substituting equations~\ref{eq:dr-u22-p1},~\ref{eq:dr-u22-p2} into equation~\ref{eq:dr-u22-int1}, we have:
\begin{align}
\label{eq:dr-u22}
\text{dr}(u_{22})&=-2\lambda_j(\alpha\beta\delta+\gamma(1-\alpha))\cdot(1+c-\alpha\delta\lambda_j+c\lambda_j \g -c-c^2+c^2\delta\lambda_j+\alpha\delta\lambda_j)\nonumber\\
&=-2\lambda_j(\alpha\beta\delta+\gamma(1-\alpha))\cdot(1-c^2+c\lambda_j(\g +c\delta))
\end{align}
We note that the denominator of $u_{12}$ (in equation~\ref{eq:u12-1}) is just the negative of the denominator of $u_{22}$ as represented in equation~\ref{eq:dr-u22}.

\underline{\bf Numerator of $u_{22}$}:
We begin by writing out the numerator of $u_{22}$ (from equation~\ref{eq:u22-1}):
\begin{align}
\label{eq:nr-u22-int}
\text{nr}(u_{22})&=\lambda_j\cdot\bigg(2c\g \delta\big(1+c-\g \lambda_j\big)-(\g ^2+c^2\delta^2)\big(1+c(1-\delta\lambda_j)\big)\bigg)\nonumber\\
&=\lambda_j\cdot\bigg(2c\g \delta\big(1+c-\alpha\delta\lambda_j-\gamma(1-\alpha)\lambda_j\big)-(\g ^2+c^2\delta^2)\big(1+c-\alpha\delta\lambda_j+\alpha\beta\delta\lambda_j\big)\bigg)\nonumber\\
&=\lambda_j\cdot\bigg( -(1+c-\alpha\delta\lambda_j)(\g -c\delta)^2-\lambda_j\cdot\big(2c\g \delta\gamma(1-\alpha)+(\g ^2+(c\delta)^2)\alpha\beta\delta\big)\bigg)
\end{align}
We now consider $2c\g \delta\gamma(1-\alpha)+(\g ^2+(c\delta)^2)\alpha\beta\delta$:
\begin{align}
\label{eq:nr-u22-p1}
&2c\g \delta\gamma(1-\alpha)+(\g ^2+(c\delta)^2)\alpha\beta\delta\nonumber\\
&=2c\g \delta\cdot(\gamma(1-\alpha)+\alpha\beta\delta)+(\g ^2+(c\delta)^2-2c\g \delta)\alpha\beta\delta\nonumber\\
&=2c\g \delta(\g -c\delta)+(\g -c\delta)^2\alpha\beta\delta
\end{align}
Substituting equation~\ref{eq:nr-u22-p1} into equation~\ref{eq:nr-u22-int} and grouping common terms, we obtain:
\begin{align}
\label{eq:nr-u22}
\text{nr}(u_{22})&=\lambda_j\cdot\bigg( -(1+c-\alpha\delta\lambda_j)(\g -c\delta)^2-\lambda_j\cdot\big(2c\g \delta(\g -c\delta)+(\g -c\delta)^2\alpha\beta\delta\big)\bigg)\nonumber\\
&=\lambda_j\cdot\bigg( -(1+c-c\delta\lambda_j)(\g -c\delta)^2-\lambda_j\cdot\big(2c\g \delta(\g -c\delta)\big)\bigg)\nonumber\\
&=-\lambda_j\cdot\bigg( (1+c-c\delta\lambda_j)(\g -c\delta)^2+2c\g \delta\lambda_j(\g -c\delta)\bigg)
\end{align}
With this, we can write out the exact expression for $u_{22}$:
\begin{align}
\label{eq:u22}
u_{22}&=\frac{\big(1+c-c\delta\lambda_j\big)(\g -c\delta)+2c\g \delta\lambda_j}{2\cdot(1-c^2+c\lambda_j\cdot(\g +c\delta))}
\end{align}

\underline{\bf Numerator of $u_{12}$}:
We begin by rewriting the numerator of $u_{12}$ (from equation~\ref{eq:u12-1}):
\begin{align}
\label{eq:nr-u12-start}
\text{nr}(u_{12})=\lambda_j\cdot\bigg(2\g \delta\big((1+c)(\g \lambda_j-c)+\delta\lambda_jc^2\big)-(\g ^2+c^2\delta^2)\big(\lambda_j\g -(1+c)(1-\delta\lambda_j)\big)\bigg)
\end{align}
We split the simplification into two parts: one depending on $(1+c)$ and the other part representing terms that don't contain $(1+c)$. In particular, we consider the terms that do not carry a coefficient of $(1+c)$:
\begin{align}
\label{eq:nr-u12-p1}
&2\g \delta^2\lambda_j c^2-(\g ^2+c^2\delta^2)\cdot(\g \lambda_j)\nonumber\\
&=\g \lambda_j\cdot(2\delta^2c^2-\g ^2-\delta^2c^2)\nonumber\\
&=-\g \lambda_j\cdot(\g ^2-(c\delta)^2)
\end{align}
Next, we consider the other term containing the $(1+c)$ part:
\begin{align}
\label{eq:nr-u12-p2}
&(1+c)\cdot\bigg(2\g \delta\cdot(\g \lambda_j-c)\ +\ (\g ^2+(c\delta)^2)\cdot(1-\delta\lambda_j)\bigg)\nonumber\\
&=(1+c)\cdot\bigg(2\g ^2\delta\lambda_j-2\g \delta c+\g ^2+(c\delta)^2-\g ^2\delta\lambda_j-c^2\delta^3\lambda_j\bigg)\nonumber\\
&=(1+c)\cdot\bigg(\ (\g -c\delta)^2 + \delta\lambda_j\ (\g ^2-(c\delta)^2)\ \bigg)
\end{align}
Substituting equations~\ref{eq:nr-u12-p1},~\ref{eq:nr-u12-p2} into equation~\ref{eq:nr-u12-start}, we get:
\begin{align}
\label{eq:nr-u12}
\text{nr}(u_{12})&=\lambda_j\cdot\big((1+c)\delta\lambda_j(\g ^2-(c\delta)^2)+(1+c)(\g -c\delta)^2-\g \lambda_j(\g ^2-(c\delta)^2)\big)\nonumber\\
&=\lambda_j\cdot\big((1+c)(\g -c\delta)^2+\lambda_j\big((1+c)\delta-\g \big)\cdot(\g ^2-(c\delta)^2)\big)\nonumber\\
&=\lambda_j\cdot\big((1+c)(\g -c\delta)^2+\lambda_j\big(\delta-(\g -c\delta)\big)\cdot(\g ^2-(c\delta)^2)\big)\nonumber\\
&=\lambda_j\cdot\big((1+c)(\g -c\delta)^2+\delta\lambda_j\cdot(\g ^2-(c\delta)^2)-\lambda_j(\g +c\delta)(\g -c\delta)^2\big)\nonumber\\
&=\lambda_j\cdot\big((1+c-\lambda_j\cdot(\g +c\delta))\cdot(\g -c\delta)^2+\delta\lambda_j\cdot(\g ^2-(c\delta)^2)\big)
\end{align}
With which, we can now write out the expression for $u_{12}$:
\begin{align}
\label{eq:u12}
u_{12}&=\frac{\big(1+c-\lambda_j(\g +c\delta)\big)(\g -c\delta)+\delta\lambda_j(\g +c\delta)}{2\cdot(1-c^2+c\lambda_j\cdot(\g +c\delta))}
\end{align}
\underline{\bf Obtaining $u_{11}$:}
We revisit equation~\ref{eq:t11} and substitute $u_{22}$ from equation~\ref{eq:u22}:
\begin{align*}
u_{11}&=u_{22}(1-2\delta\lambda_j)+\delta^2\lambda_j\\
&=\frac{\big(1+c-c\delta\lambda_j\big)(\g -c\delta)+2c\g \delta\lambda_j}{2\cdot(1-c^2+c\lambda_j\cdot(\g +c\delta))}\cdot(1-2\delta\lambda_j)+\delta^2\lambda_j
\end{align*}
From which, we consider the numerator of $u_{11}$ and begin simplifying it:
\begin{align}
\label{eq:nr-u11}
\text{nr}(u_{11})&=(1+c-c\delta\lambda_j)(\g -c\delta)(1-2\delta\lambda_j)+2c\g \delta\lambda_j(1-2\delta\lambda_j)+2\delta^2\lambda_j(1-c^2+c\lambda_j(\g +c\delta))\nonumber\\
&=(1+c-c\delta\lambda_j)(\g -c\delta)(1-2\delta\lambda_j)+2\delta^2\lambda_j + 2c\delta\lambda_j(\g -c\delta)(1-\delta\lambda_j)\nonumber\\
&=(1+c+c\delta\lambda_j)(\g -c\delta)(1-\delta\lambda_j)+2\delta^2\lambda_j-\delta\lambda_j(1+c-c\delta\lambda_j)(\g -c\delta)\nonumber\\
&=(1+c+c\delta\lambda_j)(\g -c\delta)-2\delta\lambda_j(\g -c\delta)(1+c)+2\delta^2\lambda_j\nonumber\\
&=(1+c-c\delta\lambda_j)(\g -c\delta)-2\delta\lambda_j(\g -c\delta)+2\delta^2\lambda_j
\end{align}
This implies,
\begin{align}
\label{eq:u11}
u_{11} = \frac{(1+c-c\delta\lambda_j)(\g -c\delta)-2\delta\lambda_j(\g -c\delta)+2\delta^2\lambda_j}{2\cdot(1-c^2+c\lambda_j\cdot(\g +c\delta))}
\end{align}
\underline{\bf Obtaining a bound on $\U_{22}$}

For obtaining a PSD upper bound on $\U_{22}$, we will write out a sharp bound of $u_{22}$ in each eigen space:
\iffalse
\begin{align*}
u_{22}&=\frac{\big(1+c-c\delta\lambda_j\big)(\g -c\delta)+2c\g \delta\lambda_j}{2\cdot(1-c^2+c\lambda_j\cdot(\g +c\delta))}\nonumber\\
&\leq \frac{1}{2\cnHh\lambda_j} + \frac{\delta}{2}
\end{align*}
Implying, $\U_{22}\preceq\frac{1}{2\cnHh}\cdot\Hinv+\frac{\delta}{2}\cdot\eye$.
\fi
\begin{align*}
u_{22}&=\frac{\big(1+c-c\lambda_j\delta\big)(\g -c\delta)+2c\g \delta\lambda_j}{2\cdot(1-c^2+c\lambda_j\cdot(\g +c\delta))}\nonumber\\
&=\frac{\big(1-c^2+c\lambda_j(\g +c\delta)+\g\lambda_j+(1+c)(c-\lambda_j(\g +c\delta))\big)(\g -c\delta)+2c\g \delta\lambda_j}{2\cdot(1-c^2+c\lambda_j\cdot(\g +c\delta))}\nonumber\\
&=\frac{\g -c\delta}{2}+\frac{\g\lambda_j(\g-c\delta)}{2\cdot(1-c^2+c\lambda_j\cdot(\g +c\delta))}+\frac{(1+c)(c-\lambda_j(\g +c\delta))(\g -c\delta)+2c\g \delta\lambda_j}{2\cdot(1-c^2+c\lambda_j\cdot(\g +c\delta))}\nonumber\\
&\leq\frac{\g -c\delta}{2}+\frac{\g\lambda_j(\g-c\delta)}{2\cdot(c\lambda_j\cdot(\g +c\delta))}+\frac{(1+c)(c-\lambda_j(\g +c\delta))(\g -c\delta)+2c\g \delta\lambda_j}{2\cdot(1-c^2+c\lambda_j\cdot(\g +c\delta))}\nonumber\\
&\leq\frac{\g -c\delta}{2}\cdot\frac{1+c}{c}+\frac{(1+c)(c-\lambda_j(\g +c\delta))(\g -c\delta)+2c\g \delta\lambda_j}{2\cdot(1-c^2+c\lambda_j\cdot(\g +c\delta))}\nonumber
\end{align*}
Let us consider bounding the numerator of the $2^{\text{nd}}$ term:
\begin{align*}
&(1+c)(c-\lambda_j(\g +c\delta))(\g -c\delta)+2c\g \delta\lambda_j\nonumber\\
&=c(1+c)(\g -c\delta)-(1+c)\lambda_j(\g +c\delta)(\g -c\delta)+2c\g \delta\lambda_j\nonumber\\
&=c(1+c)(\g -c\delta)-(1+c)\lambda_j(\g -c\delta)^2-2c\delta\lambda_j(1+c)(\g -c\delta)+2c\g \delta\lambda_j\nonumber\\
&=c(1+c)(\g -c\delta)-(1+c)\lambda_j(\g -c\delta)^2-2c\delta\lambda_j(1+c)(\g -c\delta)+2c(\g -c\delta)\delta\lambda_j+2c^2\delta^2\lambda_j\nonumber\\
&=c(1+c)(\g -c\delta)+2c^2\delta^2\lambda_j-(1+c)\lambda_j(\g -c\delta)^2-2c^2\delta\lambda_j(\g -c\delta)\nonumber\\
&\leq c(1+c)(\g -c\delta)+2c^2\delta^2\lambda_j
\end{align*}
Implying,
\begin{align*}
u_{22}&\leq\frac{\g -c\delta}{2}\cdot\frac{1+c}{c}+\frac{c(1+c)(\g -c\delta)+2c^2\delta^2\lambda_j}{2\cdot(1-c^2+c\lambda_j\cdot(\g +c\delta))}\nonumber\\
&\leq\frac{\g -c\delta}{2}\cdot\frac{1+c}{c}+\frac{c(1+c)(\g -c\delta)}{2\cdot(1-c^2+c\lambda_j\cdot(\g +c\delta))}+\frac{c^2\delta^2\lambda_j}{(1-c^2+c\lambda_j\cdot(\g +c\delta))}
\end{align*}
We will first upper bound the third term:
\begin{align*}
\frac{c^2\delta^2\lambda_j}{(1-c^2+c\lambda_j\cdot(\g +c\delta))}&\leq\frac{c\delta^2}{(\g +c\delta)}\nonumber\\
&=\frac{c\delta^2}{(\g -c\delta+2c\delta)}\nonumber\\
&\leq\frac{c\delta^2}{2c\delta}=\frac{\delta}{2}
\end{align*}
This implies,
\begin{align*}
u_{22}&\leq\frac{\g -c\delta}{2}\cdot\frac{1+c}{c}+\frac{\delta}{2}+\frac{c(1+c)(\g -c\delta)}{2\cdot(1-c^2+c\lambda_j\cdot(\g +c\delta))}\nonumber\\
&=\frac{\g -c\delta}{2}\cdot\frac{1+c}{c}+\frac{\delta}{2}+\frac{c^2(\g -c\delta)}{1-c^2+c\lambda_j\cdot(\g +c\delta)}+\frac{c(1-c)(\g -c\delta)}{2\cdot(1-c^2+c\lambda_j\cdot(\g +c\delta))}\nonumber\\
&\leq\frac{\g -c\delta}{2}\cdot\frac{1+c}{c}+\frac{\delta}{2}+\frac{c^2(\g -c\delta)}{1-c^2+c\lambda_j\cdot(\g +c\delta)}+\frac{c(1-c)(\g -c\delta)}{2\cdot(1-c^2)}\nonumber\\
&=\frac{\g -c\delta}{2}\cdot\frac{1+c}{c}+\frac{\delta}{2}+\frac{c^2(\g -c\delta)}{1-c^2+c\lambda_j\cdot(\g +c\delta)}+\frac{c(\g -c\delta)}{2\cdot(1+c)}\nonumber\\
&= \frac{\g -c\delta}{2}\cdot\bigg(\frac{1+c}{c}+\frac{c}{1+c}\bigg)+\frac{\delta}{2}+\frac{c^2(\g -c\delta)}{1-c^2+c\lambda_j\cdot(\g +c\delta)}\nonumber\\
&\leq \frac{\g -c\delta}{2}\cdot\frac{3}{c}+\frac{\delta}{2}+\frac{c^2(\g -c\delta)}{1-c^2+c\lambda_j\cdot(\g +c\delta)}\nonumber\\
&\leq \frac{\g -c\delta}{2}\cdot\frac{3}{c}+\frac{\delta}{2}+\frac{c(\g -c\delta)}{\lambda_j\cdot(\g +c\delta)}\nonumber\\
&= \frac{\g -c\delta}{2}\cdot\frac{3}{c}+\frac{\delta}{2}+\frac{c(\g -c\delta)}{\lambda_j\cdot(\g -c\delta+2c\delta)}\nonumber\\
&\leq \frac{\g -c\delta}{2}\cdot\frac{3}{c}+\frac{\delta}{2}+\frac{\g -c\delta}{2\lambda_j\delta}\nonumber\\
&\leq \frac{4}{c}\cdot\frac{\g -c\delta}{2\delta\lambda_j}+\frac{\delta}{2}
\end{align*}
Let us consider bounding $\frac{\g -c\delta}{2\delta\lambda_j}$ :
\begin{align*}
\frac{\g -c\delta}{2\delta\lambda_j}&=\frac{\alpha\beta\delta+\gamma(1-\alpha)}{2\delta\lambda_j}
\end{align*}
Substituting the values for $\alpha,\beta,\gamma,\delta$ applying $\frac{1}{1+\gamma\mu}\leq 1$, $c_3=\frac{c_2\sqrt{2c_1-c_1^2}}{c_1}$ and, $c_2^2=\frac{c_4}{2-c_1}$ with $0<c_4<1/6$ we get:
\begin{align*}
\frac{\g -c\delta}{2\delta\lambda_j}&\leq\bigg( \frac{c_3c_2\sqrt{2c_1-c_1^2}}{2}\sqrt{\frac{\cnS}{\cnH}} + \frac{c_2^2(2c_1-c_1^2)}{2c_1} \bigg)\cdot\frac{1}{\lambda_j\cnS}\nonumber\\
&\leq\bigg( \frac{c_3c_2\sqrt{2c_1-c_1^2}}{2} + \frac{c_2^2(2c_1-c_1^2)}{2c_1} \bigg)\cdot\frac{1}{\lambda_j\cnS}\nonumber\\
&=c_2^2(2-c_1)\cdot\frac{1}{\cnS\lambda_j}=c_4\cdot\frac{1}{\lambda_j\cnS}
\end{align*}
Which implies the bound on $u_{22}$:
\begin{align*}
u_{22}\leq\frac{4}{c}\cdot\frac{c_4}{\lambda_j\cnS}+\frac{\delta}{2}
\end{align*}
Now, consider the following bound on $1/c$:
\begin{align}
\label{eq:oneOverc}
\frac{1}{c}&=\frac{1}{\alpha(1-\beta)}\nonumber\\
&=1+\frac{(1+\cthree)\ctwo\sqrt{2\cone-\cone^2}}{\sqrt{\cnH\cnS}-\ctwo\cthree\sqrt{2\cone-\cone^2}}\nonumber\\
&\leq1+\frac{(1+\cthree)\ctwo\sqrt{2\cone-\cone^2}}{1-\ctwo\cthree\sqrt{2\cone-\cone^2}}\nonumber\\
&=1+\frac{\sqrt{\cone\cfour}+\cfour}{1-\cfour}\nonumber\\
&=\frac{1+\sqrt{\cone\cfour}}{1-\cfour}
\end{align}
Substituting values of $\cone$, $\cfour$ we have: $1/c\leq1.5$. This implies the following bound on $u_{22}$:
\begin{align}
\label{eq:u22b}
u_{22}\leq6\cdot\frac{c_4}{\lambda_j\cnS}+\frac{\delta}{2}
\end{align}
Alternatively, this implies that $\U_{22}$ can be upper bounded in a psd sense as:
\begin{align*}
\U_{22}\preceq \frac{6 c_4}{\cnS}\cdot\Hinv + \frac{\delta}{2}\cdot\eye
\end{align*}
\subsubsection{Understanding fourth moment effects}\label{ssec:fourthMomentEffects}
We wish to obtain a bound on:
\begin{align*}
\E{\Vh_2\otimes\Vh_2}\U&=\E{\Vh_2\U\Vh_2\T}\\
&=\begin{bmatrix}\delta^2&\delta\cdot \g \\\delta\cdot \g &\g ^2\end{bmatrix}\otimes\M\U_{22}
\end{align*}
We need to understand $\M\U_{22}$. 
\begin{align}
\label{eq:MU22}
\M\U_{22}&\preceq\frac{6c_4}{\cnHh}\cdot\M\Hinv+\frac{\delta}{2}\cdot\M\eye\nonumber\\
&\preceq(6c_4+\frac{\delta\infbound}{2})\cdot\H\nonumber\\
&=s\cdot\H
\end{align}
where, $s\eqdef (6c_4+\frac{\delta\infbound}{2})=23/30\leq \frac{4}{5}$. This implies (along with the fact that for any PSD matrices $\A,\mat{B},\C$, if $\A\preceq\mat{B}$, then, $\A\otimes\C\preceq\mat{B}\otimes\C$)),
\begin{align}
\label{eq:fourthMomentAccBound}
\E{\Vh_2\otimes\Vh_2}\U&\preceq s\cdot\begin{bmatrix}\delta^2&\delta\cdot \g \\\delta\cdot \g &\g ^2\end{bmatrix}\otimes\H \preceq \frac{4}{5}\cdot\begin{bmatrix}\delta^2&\delta\cdot \g \\\delta\cdot \g &\g ^2\end{bmatrix}\otimes\H.
\end{align}

%\subsubsection{Obtaining a bound on $\phiv_{\infty}$}

This will lead us to obtaining a PSD upper bound on $\phiv_{\infty}$, i.e., the proof of lemma~\ref{lem:main-variance}

\begin{proof}
[Proof of lemma~\ref{lem:main-variance}]
We begin by recounting the expression for the steady state covariance operator $\phiv_{\infty}$ and applying results derived from previous subsections:
\begin{align}
\label{eq:stationaryDistBound}
\phiv_{\infty}&=(\eyeT-\BT)^{-1}\Sigh\nonumber\\
&\preceq\sigma^2\U+\sigma^2(\eyeT-\BT)^{-1}\cdot\E{\Vh_2\otimes\Vh_2}\U\quad(\text{from equation}~\ref{eq:phivInftyUpperBound})\nonumber\\
&\preceq\sigma^2\U+\frac{4}{5}\sigma^2(\eyeT-\BT)^{-1}\bigg(\begin{bmatrix}\delta^2&\delta\cdot \g \\\delta\cdot \g &\g ^2\end{bmatrix}\otimes\H\bigg)\quad(\text{from equation}~\ref{eq:fourthMomentAccBound})\nonumber\\
&=\sigma^2\U+\frac{4}{5}(\eyeT-\BT)^{-1}\Sigh\nonumber\\
&=\sigma^2\U+\frac{4}{5}\cdot\phiv_{\infty}\nonumber\\
\implies\phiv_{\infty}&\preceq5\sigma^2\U.
\end{align}
\iffalse
\begin{align}
\label{eq:stationaryDistBound}
\phiv_\infty&=(\eyeT-\BT)^{-1}\Sigh \preceq \sigma^2 (\eyeT-\BT)^{-1} \left(\begin{bmatrix}\delta^2&\delta\cdot \g \\\delta\cdot \g &\g ^2\end{bmatrix}\otimes\Cov\right) \nonumber\\
&=\sigma^2 \sum_{k=0}^{\infty}\bigg(\big(\eyeT-\V_1\otimes\V_1 - \V_1\otimes\V_2 - \V_2\otimes\V_1\big)^{-1}\E{\Vh_2\otimes\Vh_2}\bigg)^k\cdot \nonumber\\ &\qquad \qquad \quad \big(\eyeT-\V_1\otimes\V_1 - \V_1\otimes\V_2 - \V_2\otimes\V_1\big)^{-1}\left(\begin{bmatrix}\delta^2&\delta\cdot \g \\\delta\cdot \g &\g ^2\end{bmatrix}\otimes\Cov\right)\nonumber\\
&\preceq\sigma^2\bigg(\sum_{k=0}^{\infty} (4/5)^{-k}\bigg)\cdot\U=5\sigma^2 \U.
\end{align}
\fi
Now, given the upper bound provided by equation~\ref{eq:stationaryDistBound}, we can now obtain a (mildly) looser upper PSD bound on $\U$ that is more interpretable, and this is by providing an upper bound on $\U_{11}$ and $\U_{22}$ by considering their magnitude along each eigen direction of $\H$. In particular, let us consider the max of $u_{11}$ and $u_{22}$ along the $j^{th}$ eigen direction (as implied by equations~\ref{eq:u11},~\ref{eq:u22}):
\begin{align*}
\max(u_{11},u_{22}) &= \frac{(1+c-c\delta\lambda_j)(\g -c\delta)+2\delta^2\lambda_j}{2\cdot(1-c^2+c\lambda_j\cdot(\g +c\delta))}\\
&=\frac{(1+c-c\delta\lambda_j)(\g -c\delta)+2\delta^2\lambda_j}{2\cdot(1-c^2+c\lambda_j\cdot(\g +c\delta))}\\
&=\frac{(1+c-c\delta\lambda_j)(\g -c\delta)+2c\g\lambda_j-2c\g\lambda_j+2\delta^2\lambda_j}{2\cdot(1-c^2+c\lambda_j\cdot(\g +c\delta))}\\
&=u_{22}+\frac{-2c\g\lambda_j+2\delta^2\lambda_j}{2\cdot(1-c^2+c\lambda_j\cdot(\g +c\delta))}\\
&\leq \frac{6\cfour}{\cnS\lambda_j} + \frac{\delta}{2} + \frac{\delta^2\lambda_j-c\g\lambda_j}{(1-c^2+c\lambda_j\cdot(\g +c\delta))}\quad\text{(using equation~\ref{eq:u22b})}
\end{align*}
This implies, we can now consider upper bounding the term in the equation above and this will yield us the result:
\begin{align*}
\frac{\delta^2\lambda_j-c\g\lambda_j}{(1-c^2+c\lambda_j\cdot(\g +c\delta))}&\leq\frac{\delta^2\lambda_j-c\g\lambda_j}{c\lambda_j\cdot(\g +c\delta)}\\
&\leq\frac{\delta^2\lambda_j-c\g\lambda_j}{2c^2\delta\lambda_j}\\
&=\frac{\delta^2\lambda_j-c(\alpha\delta+\gamma(1-\alpha))\lambda_j}{2c^2\delta\lambda_j}\\
&\leq\frac{\delta^2\lambda_j-c\alpha\delta\lambda_j}{2c^2\delta\lambda_j}=\frac{1-c\alpha}{c^2}\cdot\frac{\delta}{2}\\
&=\big(\frac{1-c}{c^2}+\frac{1-\alpha}{c}\big)\cdot\frac{\delta}{2}\\
&=\big(\frac{(1+\cthree)(1-\alpha)}{c^2}+\frac{1-\alpha}{c}\big)\cdot\frac{\delta}{2}\\
&=\frac{1-\alpha}{c}\big(\frac{(1+\cthree)}{c}+1\big)\cdot\frac{\delta}{2}\\
&\leq3\frac{1-\alpha}{c}\cdot\frac{1}{c}\cdot\frac{\delta}{2}\\
&\leq3\frac{1-\alpha}{c}\cdot\frac{1+\sqrt{\cone\cfour}}{1-\cfour}\cdot\frac{\delta}{2}\\
&=3\cdot\frac{\cone\cthree}{\sqrt{\cnH\cnS}-\cone\cthree^2}\cdot\frac{1+\sqrt{\cone\cfour}}{1-\cfour}\cdot\frac{\delta}{2}\\
&\leq3\cdot\frac{\cone\cthree}{1-\cone\cthree^2}\cdot\frac{1+\sqrt{\cone\cfour}}{1-\cfour}\cdot\frac{\delta}{2}\\
&\leq (2/3) \frac{\delta}{2}
\end{align*}
Plugging this into the bound for $\max{u_{11},u_{22}}$, we get:
\begin{align*}
\max(u_{11},u_{22}) &\leq \frac{6\cfour}{\cnS\lambda_j} + (5/3)\frac{\delta}{2}=(2/3)\frac{1}{\cnS\lambda_j}+(5/3)\frac{\delta}{2}
\end{align*}
This implies the bound written out in the lemma, that is,
\begin{align*}
\U\preceq\begin{bmatrix}1&0\\0&1\end{bmatrix}\otimes\bigg(\frac{2}{3}\big(\frac{1}{\cnS}\Hinv\big)+\frac{5}{6}\cdot\big(\delta\ \eye\big)\bigg)
\end{align*}
\end{proof}
%\praneeth{Replace $c_{taylor}$ with explicit constants.}

\begin{lemma}\label{lem:var-main-1}
\begin{align*}
&\iprod{\begin{bmatrix}\H&0\\0&0\end{bmatrix}}{\bigg(\eyeT+(\eyeT-\AL)^{-1}\AL+(\eyeT-\AR\T)^{-1}\AR\T\bigg)\cdot\E{\thetav_{l}\otimes\thetav_{l}}}\leq\nonumber\\&\iprod{\begin{bmatrix}\H&0\\0&0\end{bmatrix}}{\bigg(\eyeT+(\eyeT-\AL)^{-1}\AL+(\eyeT-\AR\T)^{-1}\AR\T\bigg)\cdot\E{\thetav_{\infty}\otimes\thetav_{\infty}}}\leq 5  \sigma^2 d.
\end{align*}
Where, $d$ is the dimension of the problem.
\label{lem:leadingOrderVar}
\end{lemma}
%It is unclear about how to upper bound the quantity in the lemma above given that the operator multiplying $\thetav_l\otimes\thetav_l$ may not even be PSD. In order to ensure we create valid upper bounds, we appeal to the following sequence of lemmas enroute to proving Lemma~\ref{lem:leadingOrderVar}.

Before proving Lemma~\ref{lem:var-main-1}, we note that the sequence of expected covariances of the centered parameters $\E{\thetav_l\otimes\thetav_l}$ when initialized at the zero covariance (as in the case of variance analysis) only grows (in a psd sense) as a function of time and settles at the steady state covariance.
\begin{lemma}
Let $\thetav_0=0$. Then, by running the stochastic process defined using the recursion as in equation~\ref{eq:simpleXYRec}, the covariance of the resulting process is monotonically increasing until reaching the stationary covariance $\E{\thetav_{\infty}\otimes\thetav_{\infty}}$.
\end{lemma}
\begin{proof}
As long as the process does not diverge (as defined by spectral norm bounds of the expected update $\BT=\E{\hat{\A}\otimes\hat{\A}}$ being less than $1$), the first-order Markovian process converges geometrically to its unique stationary distribution $\thetav_{\infty}\otimes\thetav_{\infty}$.
In particular,
\begin{align*}
\E{\thetav_l\otimes\thetav_l}&=\BT\E{\thetav_{l-1}\otimes\thetav_{l-1}}+\Sigh\\
&=(\sum_{k=0}^{l-1}\BT^k)\Sigh
\end{align*}
Thus implying the fact that
\begin{align*}
\E{\thetav_l\otimes\thetav_l} = \E{\thetav_{l-1}\otimes\thetav_{l-1}}+\BT^{l-1}\Sigh
\end{align*}
Owing to the PSD'ness of the operators in the equation above, the lemma concludes with the claim that $\E{\thetav_l\otimes\thetav_l}\succeq\E{\thetav_{l-1}\otimes\thetav_{l-1}}$
\end{proof}

Given these lemmas, we are now in a position to prove lemma~\ref{lem:leadingOrderVar}.
\begin{proof}[Proof of Lemma~\ref{lem:leadingOrderVar}]

\begin{align}
\label{eq:simpVarMain}
&\iprod{\begin{bmatrix}\H&0\\0&0\end{bmatrix}}{\bigg(\eyeT+(\eyeT-\AL)^{-1}\AL+(\eyeT-\AR\T)^{-1}\AR\T\bigg)
\cdot\E{\thetav_{l}\otimes\thetav_{l}}}\nonumber\\
&=\iprod{\begin{bmatrix}\H&0\\0&0\end{bmatrix}}{\bigg(\eyeT+(\eyeT-\AL)^{-1}\AL+(\eyeT-\AR\T)^{-1}\AR\T\bigg)(\eyeT-\AL\AR\T)^{-1}(\eyeT-\AL\AR\T)\cdot\E{\thetav_{l}\otimes\thetav_{l}}}\nonumber\\
&=\iprod{\begin{bmatrix}\H&0\\0&0\end{bmatrix}}{\bigg((\eyeT-\AL)^{-1}(\eyeT-\AR\T)^{-1}\bigg)(\eyeT-\AL\AR\T)\cdot\E{\thetav_{l}\otimes\thetav_{l}}}\quad \left(\mbox{using Lemma~\ref{lem:lhs-psd-lemma}}\right)\nonumber\\
&=\iprod{\bigg((\eyeT-\AL\T)^{-1}(\eyeT-\AR)^{-1}\bigg)\begin{bmatrix}\H&0\\0&0\end{bmatrix}}{(\eyeT-\AL\AR\T)\cdot\E{\thetav_{l}\otimes\thetav_{l}}}\nonumber\\
&=\iprod{(\eye-\A\T)^{-1}\begin{bmatrix}\H&0\\0&0\end{bmatrix}(\eye-\A)^{-1}}{(\eyeT-\AL\AR\T)\cdot\E{\thetav_{l}\otimes\thetav_{l}}}\nonumber\\
&=\iprod{(\eye-\A\T)^{-1}\begin{bmatrix}\H&0\\0&0\end{bmatrix}(\eye-\A)^{-1}}{(\eyeT-\DT)\cdot\E{\thetav_{l}\otimes\thetav_{l}}}\nonumber\\
&=\frac{1}{(\g -c\delta)^2}\iprod{\bigg(\otimes_2\begin{bmatrix} -(c\eye-\g \Cov)\Cov^{-1/2}\\(\eye-\delta\Cov)\Cov^{-1/2}\end{bmatrix}\bigg)}{(\eyeT-\DT)\cdot\E{\thetav_l\otimes\thetav_l}}\quad\text{(using lemma~\ref{lem:com1})}\nonumber\\
&=\frac{1}{(\g -c\delta)^2}\iprod{\bigg(\otimes_2\begin{bmatrix} -(c\eye-\g \Cov)\Cov^{-1/2}\\(\eye-\delta\Cov)\Cov^{-1/2}\end{bmatrix}\bigg)}{(\eyeT-\DT)(\eyeT-\BT)^{-1}(\eyeT-\BT^l)\Sigh}\nonumber\\
&=\frac{1}{(\g -c\delta)^2}\iprod{\bigg(\otimes_2\begin{bmatrix} -(c\eye-\g \Cov)\Cov^{-1/2}\\(\eye-\delta\Cov)\Cov^{-1/2}\end{bmatrix}\bigg)}{(\eyeT-\BT+\RT)(\eyeT-\BT)^{-1}(\eyeT-\BT^l)\Sigh}\nonumber\\
&=\frac{1}{(\g -c\delta)^2}\iprod{\bigg(\otimes_2\begin{bmatrix} -(c\eye-\g\Cov)\Cov^{-1/2}\\(\eye-\delta\Cov)\Cov^{-1/2}\end{bmatrix}\bigg)}{\Sigh-\BT^l\Sigh+\RT(\eyeT-\BT)^{-1}\Sigh-\RT(\eyeT-\BT)^{-1}\BT^l\Sigh}\nonumber\\
&\leq\frac{1}{(\g -c\delta)^2}\iprod{\bigg(\otimes_2\begin{bmatrix} -(c\eye-\g \Cov)\Cov^{-1/2}\\(\eye-\delta\Cov)\Cov^{-1/2}\end{bmatrix}\bigg)}{\Sigh+\sigma^2\RT\cdot(5 \U)}
\end{align}
So, we need to understand $\RT\U$:
\begin{align*}
\RT\U&=\mathbb{E}\bigg( \begin{bmatrix}0 & \delta\cdot(\H-\av\av\T)\\0 & \g \cdot(\H-\av\av\T)\end{bmatrix}\U\begin{bmatrix}0 & 0\\\delta\cdot(\H-\av\av\T) & \g \cdot(\H-\av\av\T)\end{bmatrix} \bigg)\\
&=\begin{bmatrix}\delta^2&\delta\cdot \g \\\delta\cdot \g &\g ^2\end{bmatrix}\otimes \E{(\H-\av\av\T)\U_{22}(\H-\av\av\T)}\\
&=\begin{bmatrix}\delta^2&\delta\cdot \g \\\delta\cdot \g &\g ^2\end{bmatrix}\otimes \big(\M-\HL\HR\big)\U_{22}\\
&\preceq \begin{bmatrix}\delta^2&\delta\cdot \g \\\delta\cdot \g &\g ^2\end{bmatrix}\otimes \M\U_{22}\\
&\preceq \frac{4}{5}\cdot\begin{bmatrix}\delta^2&\delta\cdot \g \\\delta\cdot \g &\g ^2\end{bmatrix}\otimes \H\quad(\text{from equation}~\ref{eq:MU22}).
\end{align*}
\iffalse
This implies,
\begin{align*}
\RT\phiv_{\infty}&\preceq 5\sigma^2 \RT\U \preceq 4\sigma^2\begin{bmatrix}\delta^2&\delta\cdot \g \\\delta\cdot \g &\g ^2\end{bmatrix}\otimes \H.
\end{align*}
\fi
Then,
\begin{align}
&\iprod{\begin{bmatrix}\H&0\\0&0\end{bmatrix}}{\bigg(\eyeT+(\eyeT-\AL)^{-1}\AL+(\eyeT-\AR\T)^{-1}\AR\T\bigg)\cdot\thetav_{l}\otimes\thetav_{l}}\nonumber\\
&\leq\frac{1}{(\g -c\delta)^2}\iprod{\bigg(\otimes_2\begin{bmatrix} -(c\eye-\g \Cov)\Cov^{-1/2}\\(\eye-\delta\Cov)\Cov^{-1/2}\end{bmatrix}\bigg)}{\Sigh+\sigma^2\RT\cdot(5\U)}\qquad\qquad(\text{from equation}~\ref{eq:simpVarMain})\nonumber\\
&\leq\frac{5\sigma^2}{(\g -c\delta)^2} \cdot \iprod{\bigg(\otimes_2\begin{bmatrix} -(c\eye-\g \Cov)\Cov^{-1/2}\\(\eye-\delta\Cov)\Cov^{-1/2}\end{bmatrix}\bigg)}{\begin{bmatrix}\delta^2&\delta\cdot \g \\\delta\cdot \g &\g ^2\end{bmatrix}\otimes \H}\nonumber\\
&=\frac{5}{(\g -c\delta)^2} \cdot d\ \sigma^2\cdot (\g-c\delta)^2\nonumber\\
&=5\sigma^2d.
\end{align}
\end{proof}

%\end{proof}
%\praneeth{Update the statements and proofs below.}
\iffalse
\begin{lemma}\label{lem:variance-main-1}
	\begin{align*}
		\twonorm{\left(\Id - \D\right)\inv{\left(\Id - \B\right)} \B^j \Sighat} \leq \frac{4 \widetilde{c}\cnH^2 \sigma^2}{\sqrt{1-\alpha^2}} \exp\left(-j \sqrt{2\cone - \cone^2}/\sqrt{\cnH\cnS}\right).
	\end{align*}
\end{lemma}
\begin{proof}
	Since $\B= \D + \RT$, we see that $(\Id - \D)\inv{(\Id - \B)} = (\Id - \B + \Rc)\inv{(\Id - \B)} = \Id + \Rc \inv{(\Id - \B)}$. This means that
	\begin{align*}
		\left(\Id - \D\right)\inv{\left(\Id - \B\right)} \B^j \Sighat = \B^j \Sighat + \Rc \B^j \inv{(\Id - \B)} \Sighat.
	\end{align*}
	From Lemma~\ref{lem:main-variance}, we see that $\inv{(\Id - \B)} \Sighat \preceq \widetilde{c} \begin{bmatrix}
	\Uhat_{22} & \zero \\ \zero & \Uhat_{22}
	\end{bmatrix}$ where $\Uhat_{22} \preceq \frac{\sigma^2}{4 \cnS} \Covinv + \frac{\delta \sigma^2}{2} \Id$. This implies that 
	\begin{align*}
	\Rc \B^j \inv{(\Id - \B)} \Sighat \preceq \B^{j+1} \inv{(\Id - \B)} \Sighat
	\end{align*}
	 and hence 
	\begin{align*}
	\twonorm{\Rc \B^j \inv{(\Id - \B)} \Sighat \preceq \B^{j+1} \inv{(\Id - \B)} \Sighat} &\leq \widetilde{c} \twonorm{\B^{j+1} \begin{bmatrix}
		\Uhat_{22} & \zero \\ \zero & \Uhat_{22}
		\end{bmatrix}} \nonumber\\
		&\leq \frac{4 \widetilde{c}\cnH^2\sigma^2}{\sqrt{1-\alpha^2}} \exp\left(-j \sqrt{2\cone - \cone^2}/\sqrt{\cnH\cnS}\right)
		\end{align*}
		where we used Lemma~\ref{lem:B-contraction} in the last step. A similar argument for $\B^j \Sighat$ finishes the proof.
\end{proof}
\begin{lemma}\label{lem:variance-tail-1}
	\begin{align*}
		\norm{\A^{n+1-j}\inv{\left(\Id - \B\right)} \B^j \Sighat} \leq \frac{12\sqrt{2} \widetilde{c} n \cnH^2 \sigma^2}{\sqrt{1-\alpha^2}} \exp\left(-j \sqrt{2\cone - \cone^2}/\sqrt{\cnH\cnS}\right)
	\end{align*}
\end{lemma}
\begin{proof}
	The lemma follows by combining the proof of Lemma~\ref{lem:variance-main-1} and statement of Lemma~\ref{lem:eig-A}.
\end{proof}
\fi
%\rahul{Update proofs below based on step sizes.}
\begin{lemma}
\label{lem:var1N2bound}
\iffalse
\begin{align*}
&\bigg\vert\iprod{\begin{bmatrix}\Cov&0\\0&0\end{bmatrix}}{ \bigg((\eyeT-\AL)^{-2}\AL+(\eyeT-\AR\T)^{-2}\AR\T\bigg)\phiv_{\infty}}\bigg\vert\\&\leq4\sigma^2 \cdot d \cdot\bigg(\ \frac{2}{\cfour}\cdot\bigg(1+\big(\frac{1+\sqrt{\cone\cfour}}{1-\cfour}\big)^2\bigg) + 3 \cdot \frac{1+\sqrt{\cone\cfour}}{1-\cfour} \cdot \frac{1+\sqrt{2}+\sqrt{\cfour/\cone}}{\cfour} \cdot (\sqrt{2}+\sqrt{\cfour/\cone}) \ \bigg)\sqrt{\cnH\cnS}
\end{align*}
\fi
\begin{align*}
&\bigg\vert\iprod{\begin{bmatrix}\Cov&0\\0&0\end{bmatrix}}{ \bigg((\eyeT-\AL)^{-2}\AL+(\eyeT-\AR\T)^{-2}\AR\T\bigg)\phiv_{\infty}}\bigg\vert\leq\UC\cdot\sigma^2 d\sqrt{\cnH\cnS}
\end{align*}
Where, $\UC$ is a universal constant.
\end{lemma}
\begin{proof}
We begin by noting the following while considering the left side of the above expression:
\begin{align*}
&\iprod{\begin{bmatrix}\Cov&0\\0&0\end{bmatrix}}{ \bigg((\eyeT-\AR\T)^{-2}\AR\T+(\eyeT-\AL)^{-2}\AL\bigg)\phiv_{\infty}}\\
&=\iprod{\begin{bmatrix}\Cov&0\\0&0\end{bmatrix}\A(\eye-\A)^{-2}+(\eye-\A\T)^{-2}\A\T\begin{bmatrix}\Cov&0\\0&0\end{bmatrix}}{\phiv_{\infty}}
\end{align*}
The inner product above is a sum of two terms, so let us consider the first of the terms:
\begin{align*}
&\iprod{\begin{bmatrix}\Cov&0\\0&0\end{bmatrix}\A(\eye-\A)^{-2}}{\phiv_{\infty}}\\
&=\text{Tr}\bigg((\eye-\A\T)^{-2}\A\T\begin{bmatrix}\H^{1/2}\\0\end{bmatrix}\begin{bmatrix}\H^{1/2}&0\end{bmatrix}\phivi \bigg)\\
&=\text{Tr}\bigg(\bigg(\begin{bmatrix}\H^{1/2}\\0\end{bmatrix}\T\phivi (\eye-\A\T)^{-2}\A\T\begin{bmatrix}\H^{1/2}\\0\end{bmatrix}\bigg)\bigg) \\
&=\sum_{j=1}^{d}\text{Tr}\bigg(\bigg(\begin{bmatrix}\lambda_j^{1/2}\\0\end{bmatrix}\T (\phivi)_j (\eye-\A_j\T)^{-2}\A_j\T\begin{bmatrix}\lambda_j^{1/2}\\0\end{bmatrix}\bigg)\bigg) \\
&=\sum_{j=1}^{d}\text{Tr}\bigg(\bigg(\begin{bmatrix}\lambda_j^{1/2}\\0\end{bmatrix}\T(\phivih)_j\bigg)\cdot\bigg((\phivih)_j\T(\eye-\A_j\T)^{-2}\A_j\T\begin{bmatrix}\lambda_j^{1/2}\\0\end{bmatrix}\bigg)\bigg),
\end{align*}
where $(\phivi)_j$ is the $2\times 2$ block of $\phivi$ corresponding to the $j^{\textrm{th}}$ eigensubspace of $\Cov$,  $(\phivih)_j$ denotes the $2\times 2d$ submatrix (i.e., $2$ rows) of $\phivih$ corresponding to the $j^{\textrm{th}}$ eigensubspace and $\A_j$ denotes the $j^{\textrm{th}}$ diagonal block of $\A$. Note that $(\phivih)_j (\phivih)_j \T = (\phivi)_j$.
It is very easy to observe that the second term in the dot product can be written in a similar manner, i.e.:
\begin{align*}
&\iprod{(\eye-\A\T)^{-2}\A\T\begin{bmatrix}\Cov&0\\0&0\end{bmatrix}}{\phiv_{\infty}}\\
&=\sum_{j=1}^{d}\text{Tr}\bigg(\bigg((\phivih)_j\T\begin{bmatrix}\lambda_j^{1/2}\\0\end{bmatrix}\bigg)\cdot\bigg(\begin{bmatrix}\lambda_j^{1/2}\\0\end{bmatrix}\T\A_j(\eye-\A_j)^{-2} (\phivih)_j \bigg)\bigg)
%&=\text{Tr}\bigg(\bigg(\phivih\begin{bmatrix}\H^{1/2}\\0\end{bmatrix}\bigg)\cdot\bigg(\begin{bmatrix}\H^{1/2}\\0\end{bmatrix}\T\A(\eye-\A)^{-2}\phivih\bigg)\bigg)
\end{align*}
So, essentially, the expression in the left side of the lemma can be upper bounded by using cauchy-shwartz inequality:
%\begin{align}
%\label{eq:lotpmain1}
%&\text{Tr}\bigg(\bigg(\begin{bmatrix}\H^{1/2}\\0\end{bmatrix}\T\phivih\bigg)\cdot\bigg(\phivih(\eye-\A\T)^{-2}\A\T\begin{bmatrix}\H^{1/2}\\0\end{bmatrix}\bigg)\bigg)+\text{Tr}\bigg(\bigg(\phivih\begin{bmatrix}\H^{1/2}\\0\end{bmatrix}\bigg)\cdot\bigg(\begin{bmatrix}\H^{1/2}\\0\end{bmatrix}\T\A(\eye-\A)^{-2}\phivih\bigg)\bigg)\nonumber\\
%&\quad\quad\quad\quad\leq2\adanorm{\begin{bmatrix}\H^{1/2}\\0\end{bmatrix}}_{\phivi}\cdot\adanorm{(\eye-\A\T)^{-2}\A\T\begin{bmatrix}\H^{1/2}\\0\end{bmatrix}}_{\phivi}
%&\quad\quad\quad\quad\leq2\adanorm{\begin{bmatrix}\H^{1/2}\\0\end{bmatrix}}_{\phivi}\cdot\adanorm{(\eye-\A\T)^{-2}\A\T\begin{bmatrix}\H^{1/2}\\0\end{bmatrix}}_{\phivi}
%&\quad\quad\quad\quad\leq2\adanorm{\begin{bmatrix}\H^{1/2}\\0\end{bmatrix}}_{\phivi}\cdot\adanorm{(\eye-\A\T)^{-2}\A\T\begin{bmatrix}\H^{1/2}\\0\end{bmatrix}}_{\phivi}
%\end{align}
\begin{align}
\label{eq:lotpmain1}
&\text{Tr}\bigg(\bigg(\begin{bmatrix}\lambda_j^{1/2}\\0\end{bmatrix}\T (\phivih)_j \bigg)\cdot\bigg((\phivih)_j \T (\eye-\A_j\T)^{-2}\A_j\T\begin{bmatrix}\lambda_j^{1/2}\\0\end{bmatrix}\bigg)\bigg)\nonumber\\&+\text{Tr}\bigg(\bigg((\phivih)_j \T \begin{bmatrix}\lambda_j^{1/2}\\0\end{bmatrix}\bigg)\cdot\bigg(\begin{bmatrix}\lambda_j^{1/2}\\0\end{bmatrix}\T\A_j(\eye-\A_j)^{-2}(\phivih)_j \bigg)\bigg)\nonumber\\
%&\quad\quad\quad\quad\leq2\adanorm{\begin{bmatrix}\lambda_j^{1/2}\\0\end{bmatrix}}_{\phivi}\cdot\adanorm{(\eye-\A\T)^{-2}\A\T\begin{bmatrix}\lambda_j^{1/2}\\0\end{bmatrix}}_{(\phivi)_j}
%&\quad\quad\quad\quad\leq2\adanorm{\begin{bmatrix}\lambda_j^{1/2}\\0\end{bmatrix}}_{\phivi}\cdot\adanorm{(\eye-\A\T)^{-2}\A\T\begin{bmatrix}\lambda_j^{1/2}\\0\end{bmatrix}}_{(\phivi)_j}
&\quad\quad\quad\quad\leq2\adanorm{\begin{bmatrix}\lambda_j^{1/2}\\0\end{bmatrix}}_{(\phivi)_j}\cdot\adanorm{(\eye-\A_j\T)^{-2}\A_j\T\begin{bmatrix}\lambda_j^{1/2}\\0\end{bmatrix}}_{(\phivi)_j}
\end{align}
The advantage with the above expression is that we can now begin to employ psd upper bounds on the covariance of the steady state distribution $\phivi$ and provide upper bounds on the expression on the right hand side. In particular, we employ the following bound provided by the taylor expansion that gives us an upper bound on $\phivi$:
\begin{align*}
\phivi\defeq\begin{bmatrix}\hat{\U}_{11}&\hat{\U}_{12} \\ \hat{\U}_{12}\T&\hat{\U}_{22}\end{bmatrix} \preceq 5\sigma^2 \U = 5\sigma^2 \begin{bmatrix} \U_{11} & \U_{12}\\\U_{12}\T&\U_{22}\end{bmatrix}\quad(\text{using equation~\ref{eq:stationaryDistBound}})
\end{align*}
This implies in particular that $(\phivi)_j \preceq 5 \sigma^2 \U_j$ for every $j\in[d]$ and hence, for any vector $\adanorm{\a}_{(\phivi)_j}\leq \sqrt{5\sigma^2}\adanorm{\a}_{\U_j}$. The important property of the matrix $\U$ that serves as a PSD upper bound is that it is diagonalizable using the basis of $\Cov$, thus allowing us to bound the computations in each of the eigen directions of $\Cov$. %But, before using bounds on $\phivi$, we will consider the true behavior, and use bounds towards the end of the proofs. 
%\begin{align}
%\label{eq:parta}
%\adanorm{\begin{bmatrix}\H^{1/2}\\0\end{bmatrix}}_{\phivi}&=\sqrt{\Cov^{1/2}\hat{U}_{11}\Cov^{1/2}}
%\end{align}
%Next, 
\begin{align}
\label{eq:p1}
&\adanorm{(\eye-\A_j\T)^{-2}\A_j\T\begin{bmatrix}\lambda_j^{1/2}\\0\end{bmatrix}}_{(\phivi)_j}\nonumber\\
=&\sqrt{\begin{bmatrix}\lambda_j^{1/2}&0\end{bmatrix}\A_j(\eye-\A_j)^{-2}(\phivi)_j(\eye-\A_j\T)^{-2}\A_j\T\begin{bmatrix}\lambda_j^{1/2}\\0\end{bmatrix}}\nonumber\\
\leq&\sqrt{5\sigma^2\begin{bmatrix}\lambda_j^{1/2}&0\end{bmatrix}\A_j(\eye-\A_j)^{-2}\U_j(\eye-\A_j\T)^{-2}\A_j\T\begin{bmatrix}\lambda_j^{1/2}\\0\end{bmatrix}}\nonumber\\
=&\sqrt{5\sigma^2}\adanorm{(\eye-\A_j\T)^{-2}\A_j\T\begin{bmatrix}\lambda_j^{1/2}\\0\end{bmatrix}}_{\U_j}
\end{align}
\iffalse
Let us consider $\begin{bmatrix}\H^{1/2}&0\end{bmatrix}\A(\eye-\A)^{-2}$:
\begin{align*}
\begin{bmatrix}\H^{1/2}&0\end{bmatrix}\A&=\begin{bmatrix}0&\H^{1/2}(\eye-\delta\H)\end{bmatrix}
\end{align*}
Furthermore, 
\begin{align*}
(\eye-\A)^{-1}&=\frac{1}{\g -c\delta}\cdot\begin{bmatrix}-\Hinv (c\eye-\g \H)&\Hinv(\eye-\delta\H)\\-c\Hinv&\Hinv\end{bmatrix}
\end{align*}
Implying,
\begin{align*}
&\begin{bmatrix}\H^{1/2}&0\end{bmatrix}\A(\eye-\A)^{-1}=\frac{1}{\g -c\delta}\begin{bmatrix}-c\H^{-1/2}(\eye-\delta\H)&\H^{-1/2}(\eye-\delta\H) \end{bmatrix}\\
\implies&\begin{bmatrix}\H^{1/2}&0\end{bmatrix}\A(\eye-\A)^{-2}=\frac{1}{(\g -c\delta)^2}\begin{bmatrix}-c\H^{-3/2}(\eye-\delta\H)((1-c)\eye+\g \H)&\H^{-3/2}(\eye-\delta\H)((1-c)\eye+c\delta\H)\end{bmatrix}\\
&=\frac{1}{(\g -c\delta)^2}\cdot\bigg( \begin{bmatrix}-c&1\end{bmatrix}\otimes\big(\H^{-3/2}(\eye-\delta\H)((1-c)\eye+c\delta\H)\big)\ -\ c(\g -c\delta)\begin{bmatrix}1&0\end{bmatrix}\otimes\big(\H^{-1/2}(\eye-\delta\H) \big)   \bigg).
\end{align*}
\fi
%Since we know that $\U$, $\A$ diagonalizes with $\H$, equation~\ref{eq:p1} can be analyzed in each of the eigen directions $(\lambda_j,\u_j)$ of $\H$ followed by adding their contributions up. 
So, let us consider $\begin{bmatrix}\lambda_j^{1/2}&0\end{bmatrix}\A_j(\eye-\A_j)^{-2}$ and write out the following series of equations:
\begin{align*}
\begin{bmatrix}\lambda_j^{1/2}&0\end{bmatrix}\A_j&=\begin{bmatrix}0&\sqrt{\lambda_j}(1-\delta\lambda_j)\end{bmatrix}\\
\eye-\A_j&=\begin{bmatrix}1&-(1-\delta\lambda_j)\\c&-(c-\g \lambda_j)\end{bmatrix}\\
\det(\eye-\A_j)&=(\g -c\delta)\lambda_j\\
(\eye-\A_j)^{-1}&=\frac{1}{(\g -c\delta)\lambda_j}\begin{bmatrix}-(c-\g \lambda_j)&1-\delta\lambda_j\\-c&1\end{bmatrix}\\
\implies\begin{bmatrix}\lambda_j^{1/2}&0\end{bmatrix}\A_j(\eye-\A_j)^{-1}&=\frac{\sqrt{\lambda_j}(1-\delta\lambda_j)}{(\g -c\delta)\lambda_j}\begin{bmatrix}-c&1\end{bmatrix}\\
\implies\begin{bmatrix}\lambda_j^{1/2}&0\end{bmatrix}\A_j(\eye-\A_j)^{-2}&=\frac{\sqrt{\lambda_j}(1-\delta\lambda_j)}{((\g -c\delta)\lambda_j)^2}\begin{bmatrix}-c(1-c+\g \lambda_j)&1-c+c\delta\lambda_j\end{bmatrix}\\
&=\frac{\sqrt{\lambda_j}(1-\delta\lambda_j)}{((\g -c\delta)\lambda_j)^2}\cdot\bigg((1-c+c\delta\lambda_j)\begin{bmatrix}-c&1\end{bmatrix} - c\lambda_j(\g -c\delta)\begin{bmatrix}1&0\end{bmatrix}   \bigg)
\end{align*}
This implies,
\begin{align}
\label{eq:lotp2}
\adanorm{(\eye-\A_j\T)^{-2}\A_j\T\begin{bmatrix}\lambda_j^{1/2}\\0\end{bmatrix}}_{\U_j}\leq\frac{\sqrt{\lambda_j}(1-\delta\lambda_j)}{((\g -c\delta)\lambda_j)^2}\cdot(1-c+c\delta\lambda_j)\adanorm{\begin{bmatrix}-c\\1\end{bmatrix}}_{\U_j}+\frac{c\sqrt{\lambda_j}(1-\delta\lambda_j)}{((\g -c\delta)\lambda_j)}\adanorm{\begin{bmatrix}1\\0\end{bmatrix}}_{\U_j}
\end{align}
Next, let us consider $\adanorm{\begin{bmatrix}-c\\1\end{bmatrix}}^2_{\U_j}$:
\begin{align*}
\adanorm{\begin{bmatrix}-c\\1\end{bmatrix}}^2_{\U_j}&=c^2u_{11}+u_{22}-2c\cdot u_{12}
\end{align*}
Note that $u_{11},u_{12},u_{22}$ share the same denominator, so let us evaluate the numerator $\text{nr}(c^2 u_{11}-2cu_{12}+u_{22})$. For this, we have, from equations~\ref{eq:u11},~\ref{eq:u12},~\ref{eq:u22} respectively:
%Before that, we will evaluate $\text{nr}(U_{11})$:
\iffalse
\begin{align*}
\text{nr}(U_{11})&=\text{nr}(U_{22})(1-2\delta\lambda_j)+\delta^2\lambda_j\text{dr}(U_{22})\\
&=(1+c-c\delta\lambda_j)(\g -c\delta)(1-2\delta\lambda_j)+2c\g \delta\lambda_j(1-2\delta\lambda_j)+2\delta^2\lambda_j(1-c^2+c\lambda_j(\g +c\delta))\\
&=(1+c-c\delta\lambda_j)(\g -c\delta)(1-2\delta\lambda_j)+2\delta^2\lambda_j + 2c\delta\lambda_j(\g -c\delta)(1-\delta\lambda_j)\\
&=(1+c+c\delta\lambda_j)(\g -c\delta)(1-\delta\lambda_j)+2\delta^2\lambda_j-\delta\lambda_j(1+c-c\delta\lambda_j)(\g -c\delta)\\
&=(1+c+c\delta\lambda_j)(\g -c\delta)-2\delta\lambda_j(\g -c\delta)(1+c)+2\delta^2\lambda_j\\
&=(1+c-c\delta\lambda_j)(\g -c\delta)-2\delta\lambda_j(\g -c\delta)+2\delta^2\lambda_j
\end{align*}
\fi
Furthermore, 
\begin{align*}
\text{nr}(u_{11})&=(1+c-c\delta\lambda_j)(\g -c\delta)-2\delta\lambda_j(\g -c\delta)+2\delta^2\lambda_j\\
\text{nr}(u_{12})&=(1+c-\lambda_j(\g +c\delta))(\g -c\delta)+\delta\lambda_j(\g +c\delta)\\
\text{nr}(u_{22})&=(1+c-c\delta\lambda_j)(\g -c\delta)+2c\g \delta\lambda_j
\end{align*}
Combining these, we have:
\begin{align*}
&c^2\text{nr}(u_{11})-2c\cdot\text{nr}(u_{12})+\text{nr}(u_{22})\\
=&\big((1+c-c\delta\lambda_j)(1-c)^2+2c\g \lambda_j\big)(\g -c\delta)-2c^2\delta\lambda_j(\g -c\delta)\\
=&\big((1+c-c\delta\lambda_j)(1-c)^2(\g -c\delta)\big)+2c\lambda_j(\g -c\delta)^2
\end{align*}
Implying,
\begin{align*}
\adanorm{\begin{bmatrix}-c\\1\end{bmatrix}}^2_{\U_j}&=\frac{(1+c-c\delta\lambda_j)(1-c)^2(\g -c\delta)+2c\lambda_j(\g -c\delta)^2}{1-c^2+c\lambda_j(\g +c\delta)}
\end{align*}
In a very similar manner,
\begin{align*}
\adanorm{\begin{bmatrix}1\\0\end{bmatrix}}^2_{\U_j}&=u_{11}\\
&=\frac{(1+c-c\delta\lambda_j)(\g -c\delta)-2\delta\lambda_j(\g -c\delta)+2\delta^2\lambda_j}{1-c^2+c\lambda_j(\g +c\delta)}
\end{align*}
This implies, plugging into equation~\ref{eq:lotp2}
\begin{align}
\label{eq:lotp3}
&\adanorm{(\eye-\A_j\T)^{-2}\A_j\T\begin{bmatrix}\lambda_j^{1/2}\\0\end{bmatrix}}_{\U_j}\nonumber\\&\leq\frac{\sqrt{\lambda_j}(1-\delta\lambda_j)}{((\g -c\delta)\lambda_j)^2}\cdot(1-c+c\delta\lambda_j)\sqrt{\frac{(1+c-c\delta\lambda_j)(1-c)^2(\g -c\delta)+2c\lambda_j(\g -c\delta)^2}{1-c^2+c\lambda_j(\g +c\delta)}}\nonumber\\&+\frac{c\sqrt{\lambda_j}(1-\delta\lambda_j)}{((\g -c\delta)\lambda_j)}\sqrt{\frac{(1+c-c\delta\lambda_j)(\g -c\delta)-2\delta\lambda_j(\g -c\delta)+2\delta^2\lambda_j}{1-c^2+c\lambda_j(\g +c\delta)}}
\end{align}
Finally, we need, 
\begin{align*}
\adanorm{\begin{bmatrix}\H^{1/2}\\0\end{bmatrix}}_{\phivi}\leq\sqrt{5\sigma^2}\adanorm{\begin{bmatrix}\H^{1/2}\\0\end{bmatrix}}_{\U}
\end{align*}
Again, this can be analyzed in each of the eigen directions $(\lambda_j,\u_j)$ of $\H$ to yield:
\begin{align}
\label{eq:lotp4}
\adanorm{\begin{bmatrix}\lambda_j^{1/2}\\0\end{bmatrix}}_{\U_j}&=\sqrt{\lambda_ju_{11}}\nonumber\\
&=\sqrt{\lambda_j\cdot\frac{(1+c-c\delta\lambda_j)(\g -c\delta)-2\delta\lambda_j(\g -c\delta)+2\delta^2\lambda_j}{1-c^2+c\lambda_j(\g +c\delta)}}
\end{align}
Now, we require to bound the product of equation~\ref{eq:lotp3} and~\ref{eq:lotp4}:
\begin{align}
\label{eq:lotp5}
\adanorm{(\eye-\A_j\T)^{-2}\A_j\T\begin{bmatrix}\lambda_j^{1/2}\\0\end{bmatrix}}_{\U_j}\cdot\adanorm{\begin{bmatrix}\lambda_j^{1/2}\\0\end{bmatrix}}_{\U_j}=T_1+T_2
\end{align}
Where, 
\begin{align*}
T_1&=\frac{\lambda_j(1-\delta\lambda_j)}{((\g -c\delta)\lambda_j)^2}\cdot(1-c+c\delta\lambda_j)\bigg(\sqrt{\frac{(1+c-c\delta\lambda_j)(1-c)^2(\g -c\delta)+2c\lambda_j(\g -c\delta)^2}{1-c^2+c\lambda_j(\g +c\delta)}}\bigg)\nonumber\\&\cdot\bigg(\sqrt{\frac{(1+c-c\delta\lambda_j)(\g -c\delta)-2\delta\lambda_j(\g -c\delta)+2\delta^2\lambda_j}{1-c^2+c\lambda_j(\g +c\delta)}}\bigg)\nonumber
\end{align*}
And,
\begin{align*}
T_2&=\frac{c(1-\delta\lambda_j)}{\g -c\delta}\cdot\bigg(\frac{(1+c-c\delta\lambda_j)(\g -c\delta)-2\delta\lambda_j(\g -c\delta)+2\delta^2\lambda_j}{1-c^2+c\lambda_j(\g +c\delta)}\bigg)
\end{align*}
We begin by considering $T_1$:
\begin{align}
\label{eq:lotp51}
T_1&=\frac{\lambda_j(1-\delta\lambda_j)}{((\g -c\delta)\lambda_j)^2}\cdot(1-c+c\delta\lambda_j)\bigg(\sqrt{\frac{(1+c-c\delta\lambda_j)(1-c)^2(\g -c\delta)+2c\lambda_j(\g -c\delta)^2}{1-c^2+c\lambda_j(\g +c\delta)}}\bigg)\nonumber\\&\cdot\bigg(\sqrt{\frac{(1+c-c\delta\lambda_j)(\g -c\delta)-2\delta\lambda_j(\g -c\delta)+2\delta^2\lambda_j}{1-c^2+c\lambda_j(\g +c\delta)}}\bigg)\nonumber\\
&=\bigg(\frac{\lambda_j(1-\delta\lambda_j)}{((\g -c\delta)\lambda_j)^2}\bigg)\cdot\bigg(\frac{1-c+c\delta\lambda_j}{1-c^2+c\lambda_j(\g +c\delta)}\bigg)\cdot\nonumber\\&\bigg(\sqrt{(1+c-c\delta\lambda_j)(\g -c\delta)-2\delta\lambda_j(\g -c\delta)+2\delta^2\lambda_j}\cdot\sqrt{(1+c-c\delta\lambda_j)(1-c)^2(\g -c\delta)+2c\lambda_j(\g -c\delta)^2}\bigg)\nonumber\\
&\leq\bigg(\frac{\lambda_j}{((\g -c\delta)\lambda_j)^2}\bigg)\cdot\bigg(\frac{1-c+c\delta\lambda_j}{1-c^2+c\lambda_j(\g +c\delta)}\bigg)\cdot\nonumber\\&\bigg(\sqrt{(1+c-c\delta\lambda_j)(\g -c\delta)+2\delta^2\lambda_j}\cdot\sqrt{(1+c-c\delta\lambda_j)(1-c)^2(\g -c\delta)+2c\lambda_j(\g -c\delta)^2}\bigg)
\end{align}
We will consider the four terms within the square root and bound them separately:
\begin{align*}
T_{1}^{11}&=\frac{(1+c-c\delta\lambda_j)(1-c)}{(\g -c\delta)\lambda_j}\\
&\leq\frac{2(1-c)}{\lambda_j\cdot(\g -c\delta)}\leq\frac{2(1+\cthree)}{\lambda_j\gamma}\\
&\leq\frac{2(1+\cthree)}{\ctwo\sqrt{2c_1-c_1^2}}\sqrt{\cnH\cnS}
\end{align*}
Next,
\begin{align*}
T_{1}^{21}&=\frac{\sqrt{2\delta^2\lambda_j}\sqrt{(1+c-c\delta\lambda_j)(1-c)^2(\g -c\delta)}}{(\g -c\delta)^2\lambda_j}\\
&\leq\frac{2\delta(1-c)}{\sqrt{(\g -c\delta)^3\lambda_j}}=\frac{2\delta}{\sqrt{(\g -c\delta)\lambda_j}}\frac{1-c}{\g -c\delta}\\
&=\frac{2(1+\cthree)\delta}{\gamma}\cdot\frac{1}{\sqrt{(\g -c\delta)\lambda_j}}\\
&\leq\frac{2(1+\cthree)\delta}{\gamma}\cdot\frac{1}{\sqrt{\gamma(1-\alpha)\mu}}\\
&\leq\frac{2\sqrt{2}(1+\cthree)}{\ctwo^2(2-c_1)}\cdot\cnS
\end{align*}
Next, 
\begin{align*}
T_1^{12}&=\frac{\sqrt{(1+c-c\delta\lambda_j)(\g -c\delta)^3\cdot 2c\lambda_j}}{(\g -c\delta)^2\lambda_j}\\
&\leq\frac{2\sqrt{2}}{\ctwo\sqrt{2c_1-c_1^2}}\cdot\sqrt{\cnH\cnS}
\end{align*}
Finally, 
\begin{align*}
T_1^{22}&=\frac{\sqrt{2\delta^2\lambda_j\cdot2\ c\lambda_j(\g -c\delta)^2}}{(\g -c\delta)^2\lambda_j}\\
&\leq\frac{2\delta}{\g -c\delta}\leq\frac{4}{\ctwo^2(2-c_1)}\cdot\cnS
\end{align*}
Implying,
\begin{align*}
T_1&\leq\bigg(\frac{1-c+c\delta\lambda_j}{1-c^2+c\lambda_j(\g +c\delta)}\bigg)\cdot(T_1^{11}+T_1^{12}+T_1^{21}+T_1^{22})\\
&\leq\bigg(\frac{1-c+c\delta\lambda_j}{1-c^2+c\lambda_j(\g +c\delta)}\bigg)\cdot2\cdot(1+\sqrt{2}+\cthree)\bigg(\frac{\sqrt{\cnH\cnS}}{\ctwo\sqrt{2\cone-\cone^2}}+\sqrt{2}\frac{\cnS}{\ctwo^2(2-\cone)}\bigg)\\
&\leq\bigg(\frac{1}{1+c}+\frac{1}{2c}\bigg)\cdot2\cdot(1+\sqrt{2}+\cthree)\bigg(\frac{\sqrt{\cnH\cnS}}{\ctwo\sqrt{2\cone-\cone^2}}+\sqrt{2}\frac{\cnS}{\ctwo^2(2-\cone)}\bigg)\\
&=\bigg(\frac{1}{1+c}+\frac{1}{2c}\bigg)\cdot2\cdot(1+\sqrt{2}+\cthree)\bigg(\frac{\sqrt{\cnH\cnS}}{\sqrt{\cone\cfour}}+\frac{\sqrt{2}\cnS}{\cfour}\bigg)\\
&\leq\frac{3}{c}\cdot(1+\sqrt{2}+\cthree)\bigg(\frac{\sqrt{\cnH\cnS}}{\sqrt{\cone\cfour}}+\frac{\sqrt{2}\cnS}{\cfour}\bigg)
%&\leq\frac{144}{\sqrt{2c_1-c_1^2}}\sqrt{\cnH\cnS}
\end{align*}
\iffalse
We will now consider bounding $\frac{1}{c}$:
\begin{align*}
\frac{1}{c}&=\frac{1}{\alpha(1-\beta)}\\
&=1+\frac{(1+\cthree)\ctwo\sqrt{2\cone-\cone^2}}{\sqrt{\cnH\cnS}-\ctwo\cthree\sqrt{2\cone-\cone^2}}\\
&\leq1+\frac{(1+\cthree)\ctwo\sqrt{2\cone-\cone^2}}{1-\ctwo\cthree\sqrt{2\cone-\cone^2}}\\
&=1+\frac{\sqrt{\cone\cfour}+\cfour}{1-\cfour}\\
&=\frac{1+\sqrt{\cone\cfour}}{1-\cfour}
\end{align*}
\fi
Recall the bound on 1/c from equation~\ref{eq:oneOverc}:
\begin{align*}
\frac{1}{c}\leq\frac{1+\sqrt{\cone\cfour}}{1-\cfour}
\end{align*}
Implying,
\begin{align}
\label{eq:t1final}
T_1&\leq\frac{3}{c}\cdot(1+\sqrt{2}+\cthree)\bigg(\frac{\sqrt{\cnH\cnS}}{\sqrt{\cone\cfour}}+\frac{\sqrt{2}\cnS}{\cfour}\bigg)\nonumber\\
&\leq\frac{3}{c}\cdot(1+\sqrt{2}+\cthree)\bigg(\frac{1}{\sqrt{\cone\cfour}}+\frac{\sqrt{2}}{\cfour}\bigg)\sqrt{\cnH\cnS}\nonumber\\
&\leq3(1+\sqrt{2}+\cthree)\bigg(\frac{1}{\sqrt{\cone\cfour}}+\frac{\sqrt{2}}{\cfour}\bigg)\cdot\frac{1+\sqrt{\cone\cfour}}{1-\cfour}\sqrt{\cnH\cnS}\nonumber\\
&\leq3(1+\sqrt{2}+\sqrt{(\cfour/\cone)})\bigg(\frac{1}{\sqrt{\cone\cfour}}+\frac{\sqrt{2}}{\cfour}\bigg)\cdot\frac{1+\sqrt{\cone\cfour}}{1-\cfour}\sqrt{\cnH\cnS}
\end{align}
Next, we consider $T_{2}$:
\begin{align*}
T_2&=\frac{c(1-\delta\lambda_j)}{\g -c\delta}\cdot\bigg(\frac{(1+c-c\delta\lambda_j)(\g -c\delta)-2\delta\lambda_j(\g -c\delta)+2\delta^2\lambda_j}{1-c^2+c\lambda_j(\g +c\delta)}\bigg)\\
&\leq\bigg(\frac{(1+c-c\delta\lambda_j)(\g -c\delta)-2\delta\lambda_j(\g -c\delta)+2\delta^2\lambda_j}{(\g -c\delta)\cdot(1-c^2+c\lambda_j(\g +c\delta))}\bigg)\\
&\leq\bigg(\frac{(1+c-c\delta\lambda_j)(\g -c\delta)+2\delta^2\lambda_j}{(\g -c\delta)\cdot(1-c^2+c\lambda_j(\g +c\delta))}\bigg)
\end{align*}
We split $T_2$ into two parts:
\begin{align*}
T_2^{1}&=\frac{(1+c-c\delta\lambda_j)}{(1-c^2+c\lambda_j(\g +c\delta))}\\
&\leq\frac{1}{1-c}=\frac{1}{1-\alpha+\alpha\beta}\\
&=\frac{1}{(1+\cthree)(1-\alpha)}\\
&\leq\frac{2\sqrt{\cnH\cnS}}{(1+\cthree)\ctwo\sqrt{2c_1-c_1^2}}\\
&\leq\frac{2\sqrt{\cnH\cnS}}{(1+\sqrt{\cfour/\cone})\sqrt{\cone\cfour}}\\
&=\frac{2\sqrt{\cnH\cnS}}{\sqrt{\cone\cfour}+\cfour}
\end{align*}
Then,
\begin{align*}
T_2^{2}&=\frac{2\delta^2\lambda_j}{(\g -c\delta)(1-c^2+c\lambda_j(\g +c\delta))}\\
&\leq\frac{\delta^2\lambda_j}{\gamma(1-\alpha)c^2\lambda_j\delta}=\frac{\delta}{c^2\gamma(1-\alpha)}\\
&=\frac{2\cnS}{\cfour}\cdot\frac{1}{c^2}
\end{align*}
Implying,
\begin{align}
\label{eq:t2final}
T_2&\leq2\cdot\bigg(\frac{\sqrt{\cnH\cnS}}{\cfour+\sqrt{\cone\cfour}}+\frac{\cnS}{c^2\cfour}\bigg)\nonumber\\
&\leq2\cdot\bigg(\frac{1}{\sqrt{\cone\cfour}+\cfour}+\big(\frac{1+\sqrt{\cone\cfour}}{1-\cfour}\big)^2\cdot\frac{1}{\cfour}\bigg)\sqrt{\cnH\cnS}\nonumber\\
&\leq\frac{2}{\cfour}\cdot\bigg(1+\big(\frac{1+\sqrt{\cone\cfour}}{1-\cfour}\big)^2\bigg)\sqrt{\cnH\cnS}
\end{align}
We add $T_1$ and $T_2$ and revisit equation~\ref{eq:lotp5}:
\begin{align}
\label{eq:perDirectionBound}
&\adanorm{(\eye-\A_j\T)^{-2}\A_j\T\begin{bmatrix}\lambda_j^{1/2}\\0\end{bmatrix}}_{\U_j}\cdot\adanorm{\begin{bmatrix}\lambda_j^{1/2}\\0\end{bmatrix}}_{\U_j}\nonumber\\
&=T_1+T_2\nonumber\\
&\leq\bigg(\ \frac{2}{\cfour}\cdot\bigg(1+\big(\frac{1+\sqrt{\cone\cfour}}{1-\cfour}\big)^2\bigg) + 3 \cdot \frac{1+\sqrt{\cone\cfour}}{1-\cfour} \cdot \frac{1+\sqrt{2}+\sqrt{\cfour/\cone}}{\cfour} \cdot (\sqrt{2}+\sqrt{\cfour/\cone}) \ \bigg)\sqrt{\cnH\cnS}
\end{align}
Then, we revisit equation~\ref{eq:lotpmain1}:
\begin{align}
\label{eq:lotpmain2}
&\bigg(\begin{bmatrix}\H^{1/2}\\0\end{bmatrix}\T\phivih\bigg)\cdot\bigg(\phivih(\eye-\A\T)^{-2}\A\T\begin{bmatrix}\H^{1/2}\\0\end{bmatrix}\bigg)+\bigg(\phivih\begin{bmatrix}\H^{1/2}\\0\end{bmatrix}\bigg)\cdot\bigg(\begin{bmatrix}\H^{1/2}\\0\end{bmatrix}\T\A(\eye-\A)^{-2}\phivih\bigg)\nonumber\\
&\leq2\sum_{j=1}^{d} \adanorm{\begin{bmatrix}\lambda_j^{1/2}\\0\end{bmatrix}}_{(\phivi)_j}\cdot\adanorm{(\eye-\A_j\T)^{-2}\A_j\T\begin{bmatrix}\lambda_j^{1/2}\\0\end{bmatrix}}_{(\phivi)_j} \nonumber\\
&\leq10\sigma^2 \sum_{j=1}^{d} \adanorm{\begin{bmatrix}\lambda_j^{1/2}\\0\end{bmatrix}}_{\U_j}\cdot\adanorm{(\eye-\A_j\T)^{-2}\A_j\T\begin{bmatrix}\lambda_j^{1/2}\\0\end{bmatrix}}_{\U_j}\quad(\text{using equation~\ref{eq:stationaryDistBound}})\nonumber\\
&\leq10\sigma^2 \cdot d \cdot\bigg(\ \frac{2}{\cfour}\cdot\bigg(1+\big(\frac{1+\sqrt{\cone\cfour}}{1-\cfour}\big)^2\bigg) + 3 \cdot \frac{1+\sqrt{\cone\cfour}}{1-\cfour} \cdot \frac{1+\sqrt{2}+\sqrt{\cfour/\cone}}{\cfour} \cdot (\sqrt{2}+\sqrt{\cfour/\cone}) \ \bigg)\sqrt{\cnH\cnS}\nonumber\\
&\leq \UC \sigma^2 d \sqrt{\cnH\cnS}
\end{align}
Where the equation in the penultimate line is obtained by summing over all eigen directions the bound implied by equation~\ref{eq:perDirectionBound}, and $\UC$ is a universal constant.
%\rahul{might have to change slightly the constants permeating from changes to the taylor expansion part}
\end{proof}

\begin{lemma}\label{lem:bound-variance}
\iffalse
	\begin{align*}
	&\iprod{\begin{bmatrix}
		\Cov & \zero \\ \zero & \zero
		\end{bmatrix}}{\E{\thetavb^{\textrm{variance}} \otimes \thetavb^{\text{variance}}}}\leq 5\frac{\sigma^2d}{n-t} + 6912\cdot\sigma^2d\cdot\frac{(\cnH\cnS)^{7/4}}{\cthree\cfour(\cone\cthree)^{3/2}}\exp^{-(n+1)\cdot\frac{\ctwo\cthree\sqrt{2\cone-\cone^2}}{\sqrt{\cnH\cnS}}}\\ &+ 4\cdot\frac{\sigma^2 d}{(n-t)^2} \cdot\bigg(\ \frac{2}{\cfour}\cdot\bigg(1+\big(\frac{1+\sqrt{\cone\cfour}}{1-\cfour}\big)^2\bigg) + 3 \cdot \frac{1+\sqrt{\cone\cfour}}{1-\cfour} \cdot \frac{1+\sqrt{2}+\sqrt{\cfour/\cone}}{\cfour} \cdot (\sqrt{2}+\sqrt{\cfour/\cone}) \ \bigg)\sqrt{\cnH\cnS} \\ &+ 41472\frac{\sigma^2d}{n-t}(\cnH\cnS)^{11/4}\alpha^{(n-t-1)/2}\frac{1}{\cthree\cfour^2(\cone\cthree)^{3/2}} + 41472\cdot\frac{\sigma^2d}{(n-t)^2}\cdot\frac{1}{\cfour^2(\cone\cthree^2)^3}\cdot\exp\bigg({-(n+1)\frac{\cone\cthree^2}{\sqrt{\cnH\cnS}}}\bigg)\cdot(\cnH\cnS)^{7/2}\cnS
	\end{align*}
	\fi
		\begin{align*}
	&\iprod{\begin{bmatrix}
		\Cov & \zero \\ \zero & \zero
		\end{bmatrix}}{\E{\thetavb^{\textrm{variance}} \otimes \thetavb^{\text{variance}}}}\leq 5\frac{\sigma^2d}{n-t} +\UC\cdot\frac{\sigma^2 d}{(n-t)^2} \cdot\sqrt{\cnH\cnS}  \\ &+  \UC\cdot\frac{\sigma^2d}{n-t}(\cnH\cnS)^{11/4}\exp\bigg(-\frac{(n-t-1)\ctwo\sqrt{2\cone-\cone^2}}{4\sqrt{\cnH\cnS}}\bigg) \\&+ \UC\cdot\frac{\sigma^2d}{(n-t)^2}\cdot\exp\bigg({-(n+1)\frac{\cone\cthree^2}{\sqrt{\cnH\cnS}}}\bigg)\cdot(\cnH\cnS)^{7/2}\cnS+\UC\cdot\sigma^2d\cdot(\cnH\cnS)^{7/4}\exp\bigg({-(n+1)\cdot\frac{\ctwo\cthree\sqrt{2\cone-\cone^2}}{\sqrt{\cnH\cnS}}}\bigg)%\UC\cdot\frac{\sigma^2d}{n-t}(\cnH\cnS)^{11/4}\alpha^{(n-t-1)/2}
	\end{align*}
	where, $\UC$ is a universal constant.
\end{lemma}
\begin{proof}
We begin by recounting the expression for the covariance of the variance error of the tail-averaged iterate $\thetavb^{\textrm{variance}}$ from equation~\ref{eq:varianceTA}:
\begin{align*}
\E{\thetavb^{\text{variance}}\otimes\thetavb^{\text{variance}}}
&=\underbrace{\frac{1}{n-t}\big(\eyeT + (\eyeT-\AL)^{-1}\AL + (\eyeT-\AR\T)^{-1}\AR\T\big)(\eyeT-\BT)^{-1}\Sigh}_{\Y_1\eqdef}\nonumber\\&\underbrace{-\frac{1}{(n-t)^2}\big((\eyeT-\AL)^{-2}\AL+(\eyeT-\AR\T)^{-2}\AR\T\big)(\eyeT-\BT)^{-1}\Sigh}_{\Y_2\eqdef}\nonumber\\&\underbrace{+\frac{1}{(n-t)^2}\big((\eyeT-\AL)^{-2}\AL^{n+1-t}+(\eyeT-\AR\T)^{-2}(\AR\T)^{n+1-t}\big)(\eyeT-\BT)^{-1}\Sigh}_{\Y_3\eqdef}\nonumber\\&\underbrace{-\frac{1}{(n-t)^2}\big(\eyeT + (\eyeT-\AL)^{-1}\AL + (\eyeT-\AR\T)^{-1}\AR\T\big)(\eyeT-\BT)^{-2}(\BT^{t+1}-\BT^{n+1})\Sigh}_{\Y_4\eqdef}\nonumber\\&\underbrace{+\frac{1}{(n-t)^2}\sum_{j=t+1}^n\big((\eyeT-\AL)^{-1}\AL^{n+1-j}+(\eyeT-\AR\T)^{-1}(\AR\T)^{n+1-j}\big)(\eyeT-\BT)^{-1}\BT^j\Sigh}_{\Y_5\eqdef}
\end{align*}
The goal is to bound $\iprod{\begin{bmatrix}\H&0\\0&0\end{bmatrix}}{\Y_i}$, for $i=1,..,5$.

For the case of $\Y_1$, combining the fact that $\E{\thetav_{\infty}\otimes\thetav_{\infty}}=(\eyeT-\BT)^{-1}\Sigh$ and lemma~\ref{lem:var-main-1}, we get:
\begin{align}
\label{eq:e1}
\iprod{\begin{bmatrix}\H&0\\0&0\end{bmatrix}}{\Y_1}&=\frac{1}{n-t}\iprod{\begin{bmatrix}\H&0\\0&0\end{bmatrix}}{\big(\eyeT + (\eyeT-\AL)^{-1}\AL + (\eyeT-\AR\T)^{-1}\AR\T\big)(\eyeT-\BT)^{-1}\Sigh}\nonumber\\
&=\frac{1}{n-t}\iprod{\begin{bmatrix}\H&0\\0&0\end{bmatrix}}{\big(\eyeT + (\eyeT-\AL)^{-1}\AL + (\eyeT-\AR\T)^{-1}\AR\T\big)\E{\thetav_\infty\otimes\thetav_\infty}}\nonumber\\
&\leq 5\frac{\sigma^2d}{n-t}
\end{align}

For the case of $\Y_2$, we employ the result from lemma~\ref{lem:var1N2bound}, and this gives us:

\begin{align}
\label{eq:e2}
&\bigg\vert\iprod{\begin{bmatrix}\H&0\\0&0\end{bmatrix}}{\Y_2}\bigg\vert\leq\frac{\UC\cdot\sigma^2 d\sqrt{\cnH\cnS}}{(n-t)^2}%4\cdot\frac{\sigma^2 d}{(n-t)^2} \cdot\bigg(\ \frac{2}{\cfour}\cdot\bigg(1+\big(\frac{1+\sqrt{\cone\cfour}}{1-\cfour}\big)^2\bigg) + 3 \cdot \frac{1+\sqrt{\cone\cfour}}{1-\cfour} \cdot \frac{1+\sqrt{2}+\sqrt{\cfour/\cone}}{\cfour} \cdot (\sqrt{2}+\sqrt{\cfour/\cone}) \ \bigg)\sqrt{\cnH\cnS}
\end{align}

For $i=3$, we have:
\begin{align}
\label{eq:e31}
&\iprod{\begin{bmatrix}\H&0\\0&0\end{bmatrix}}{\Y_3}=\frac{1}{(n-t)^2}\iprod{\begin{bmatrix}\H&0\\0&0\end{bmatrix}}{\big((\eyeT-\AL)^{-2}\AL^{n+1-t}+(\eyeT-\AR\T)^{-2}(\AR\T)^{n+1-t}\big)(\eyeT-\BT)^{-1}\Sigh}\nonumber\\
&=\frac{1}{(n-t)^2}\bigg(\iprod{(\eye-\A\T)^{-2}\A\T\begin{bmatrix}\H&0\\0&0\end{bmatrix}}{\A^{n-t}(\eyeT-\BT)^{-1}\Sigh}+\iprod{\begin{bmatrix}\H&0\\0&0\end{bmatrix}\A(\eye-\A)^{-2}}{(\eyeT-\BT)^{-1}\Sigh\ (\A\T)^{n-t}}\bigg)\nonumber\\
&=\frac{4d}{(n-t)^2}\cdot\|(\eye-\A\T)^{-2}\A\T\begin{bmatrix}\H&0\\0&0\end{bmatrix}\|\cdot\|\A^{n-t}(\eyeT-\BT)^{-1}\Sigh\|
\end{align}
We will consider bounding $\|\A^{n-t}(\eyeT-\BT)^{-1}\Sigh\|$:
\begin{align}
\label{eq:e311}
\|\A^{n-t}(\eyeT-\BT)^{-1}\Sigh\|&\leq\sum_{i=0}^{\infty}\|\A^{n-t}\BT^i\Sigh\|\nonumber\\
&\leq\frac{12\sqrt{2}}{\sqrt{1-\alpha^2}}\cnH(n-t)\alpha^{(n-t-1)/2}\bigg(\sum_{i}\big(1-\frac{\ctwo\cthree\sqrt{2\cone-\cone^2}}{\sqrt{\cnH\cnS}}\big)^i\bigg)\|\Sigh\|\nonumber\\&\qquad\qquad\qquad\qquad\qquad\qquad\qquad\qquad\qquad\qquad\quad(\text{using corollary}~\ref{cor:bias-tail1})\nonumber\\
&=\frac{12\sqrt{2}}{\sqrt{1-\alpha^2}}\cnH(n-t)\alpha^{(n-t-1)/2}\cdot\frac{\sqrt{\cnH\cnS}}{\ctwo\cthree\sqrt{2\cone-\cone^2}}\cdot\|\Sigh\|\nonumber\\
&=\frac{12\sqrt{2}\sigma^2}{\sqrt{1-\alpha^2}}\cnH(n-t)\alpha^{(n-t-1)/2}\cdot\frac{\sqrt{\cnH\cnS}}{\ctwo\cthree\sqrt{2\cone-\cone^2}}\cdot(\g+c\delta)^2\|\H\|\nonumber\\
&\leq\frac{108\sqrt{2}\sigma^2}{\sqrt{1-\alpha^2}}\cnH(n-t)\alpha^{(n-t-1)/2}\cdot\frac{\sqrt{\cnH\cnS}}{\ctwo\cthree\sqrt{2\cone-\cone^2}}\cdot\delta^2\|\H\|
\end{align}
We also upper bound $\alpha$ as:
\begin{align}
\label{eq:alpBound}
\alpha&=1-\frac{\ctwo\sqrt{2\cone-\cone^2}}{\sqrt{\cnH\cnS}+\ctwo\sqrt{2\cone-\cone^2}}\nonumber\\
&\leq1-\frac{\ctwo\sqrt{2\cone-\cone^2}}{2\sqrt{\cnH\cnS}}\nonumber\\
&=e^{-\frac{\ctwo\sqrt{2\cone-\cone^2}}{2\sqrt{\cnH\cnS}}}%=e^{-\frac{1}{6\sqrt{5}\cdot\sqrt{\cnH\cnS}}}\nonumber\\
%&=e^{-\frac{1}{15\sqrt{\cnH\cnS}}}.
\end{align}
Furthermore, for $\|(\eye-\A\T)^{-2}\A\T\begin{bmatrix}\H&0\\0&0\end{bmatrix}\|$, we consider a bound in each eigendirection $j$ and accumulate the results subsequently:
\begin{align*}
&\|(\eye-\A_j\T)^{-2}\A_j\T\begin{bmatrix}\lambda_j&0\\0&0\end{bmatrix}\|\\&\leq\frac{1}{(\g-c\delta)^2}\cdot\frac{1-\delta\lambda_j}{\lambda_j}\cdot\sqrt{(1+c^2)(1-c)^2+c^2\lambda_j^2(\g^2+\delta^2)}\nonumber\\&\qquad\qquad\qquad\qquad\qquad\qquad\qquad\qquad\qquad\qquad\quad(\text{using lemma}~\ref{lem:com2})\\
&\leq\frac{\sqrt{7}}{(\g-c\delta)^2}\cdot\frac{1}{\lambda_j}\\
&\leq\frac{\sqrt{7}}{(\gamma(1-\alpha))^2}\cdot\frac{1}{\lambda_j}\\
&\leq\frac{48(\cnH\cnS)^2}{(\cone\cfour)^2}\frac{\mu^2}{\lambda_j}=\frac{48\cnS^2}{(\delta\cfour)^2}\frac{1}{\lambda_j}\\
\implies\|(\eye-\A\T)^{-2}\A\T\begin{bmatrix}\H&0\\0&0\end{bmatrix}\|&\leq\frac{48\cnS^2}{(\delta\cfour)^2}\cdot\frac{1}{\mu}
\end{align*}
Plugging this into equation~\ref{eq:e31}, we obtain:
\iffalse
\begin{align}
\label{eq:e3}
&\iprod{\begin{bmatrix}\H&0\\0&0\end{bmatrix}}{\Y_3}\leq41472\frac{\sigma^2d}{n-t}(\cnH\cnS)^{11/4}\alpha^{(n-t-1)/2}\frac{1}{\cthree\cfour^2(\cone\cthree)^{3/2}}
\end{align}
\fi
\begin{align}
\label{eq:e3}
\iprod{\begin{bmatrix}\H&0\\0&0\end{bmatrix}}{\Y_3}&\leq41472\frac{\sigma^2d}{n-t}(\cnH\cnS)^{11/4}\alpha^{(n-t-1)/2}\frac{1}{\cthree\cfour^2(\cone\cthree)^{3/2}}\nonumber\\
&\leq\UC\frac{\sigma^2d}{n-t}(\cnH\cnS)^{11/4}\alpha^{(n-t-1)/2}\nonumber\\%\frac{1}{\cthree\cfour^2(\cone\cthree)^{3/2}}
&\leq\UC\frac{\sigma^2d}{n-t}(\cnH\cnS)^{11/4}\exp^{-\frac{(n-t-1)\ctwo\sqrt{2\cone-\cone^2}}{4\sqrt{\cnH\cnS}}}
\end{align}
Next, let us consider $\Y_4$:
\begin{align}
\label{eq:e41}
&\iprod{\begin{bmatrix}\H&0\\0&0\end{bmatrix}}{\Y_4}\nonumber\\&=-\frac{1}{(n-t)^2}\iprod{\begin{bmatrix}\H&0\\0&0\end{bmatrix}}{\big(\eyeT + (\eyeT-\AL)^{-1}\AL + (\eyeT-\AR\T)^{-1}\AR\T\big)(\eyeT-\BT)^{-2}(\BT^{t+1}-\BT^{n+1})\Sigh}\nonumber\\
&=-\frac{1}{(n-t)^2}\iprod{(\eye-\A\T)^{-1}\begin{bmatrix}\H&0\\0&0\end{bmatrix}(\eye-\A)^{-1}}{(\eyeT-\DT)(\eyeT-\BT)^{-2}(\BT^{t+1}-\BT^{n+1})\Sigh}\nonumber\\
&=-\frac{1}{(\g-c\delta)^2(n-t)^2}\iprod{\bigg(\otimes_{2}\begin{bmatrix}-(c\eye-\g\Cov)\Cov^{-1/2}\\(\eye-\delta\Cov)\Cov^{-1/2}\end{bmatrix}\bigg)}{(\eyeT-\BT+\RT)(\eyeT-\BT)^{-2}(\BT^{t+1}-\BT^{n+1})\Sigh}\nonumber\\&\qquad\qquad\qquad\qquad\qquad\qquad\qquad\qquad\qquad\qquad\qquad\qquad\qquad\qquad\qquad(\text{using lemma}~\ref{lem:com1})\nonumber\\
&\leq\frac{1}{(\g-c\delta)^2(n-t)^2}\iprod{\bigg(\otimes_{2}\begin{bmatrix}-(c\eye-\g\Cov)\Cov^{-1/2}\\(\eye-\delta\Cov)\Cov^{-1/2}\end{bmatrix}\bigg)}{(\eyeT-\BT+\RT)(\eyeT-\BT)^{-2}\BT^{n+1}\Sigh}\nonumber\\
&\leq\frac{1}{(\g-c\delta)^2(n-t)^2}\cdot\bigg(\iprod{\bigg(\otimes_{2}\begin{bmatrix}-(c\eye-\g\Cov)\Cov^{-1/2}\\(\eye-\delta\Cov)\Cov^{-1/2}\end{bmatrix}\bigg)}{(\eyeT-\BT)^{-1}\BT^{n+1}\Sigh}\nonumber\\&\qquad \qquad\qquad \qquad \qquad  +\iprod{\RT\T\bigg(\otimes_{2}\begin{bmatrix}-(c\eye-\g\Cov)\Cov^{-1/2}\\(\eye-\delta\Cov)\Cov^{-1/2}\end{bmatrix}\bigg)}{(\eyeT-\BT)^{-2}\BT^{n+1}\Sigh}\bigg)\nonumber\\
&=\frac{1}{(\g-c\delta)^2(n-t)^2}\cdot\bigg(\iprod{\bigg(\otimes_{2}\begin{bmatrix}-(c\eye-\g\Cov)\Cov^{-1/2}\\(\eye-\delta\Cov)\Cov^{-1/2}\end{bmatrix}\bigg)}{(\eyeT-\BT)^{-1}\BT^{n+1}\Sigh}\nonumber\\&\qquad \qquad\qquad \qquad \qquad  +\iprod{\otimes_2\begin{bmatrix}\delta\\\g\end{bmatrix}\otimes\bigg(\M-\HL\HR\bigg)(\eye-\delta\H)\H^{-1}(\eye-\delta\H)}{(\eyeT-\BT)^{-2}\BT^{n+1}\Sigh}\bigg)\nonumber\\
&\leq\frac{1}{(\g-c\delta)^2(n-t)^2}\cdot\bigg(\iprod{\bigg(\otimes_{2}\begin{bmatrix}-(c\eye-\g\Cov)\Cov^{-1/2}\\(\eye-\delta\Cov)\Cov^{-1/2}\end{bmatrix}\bigg)}{(\eyeT-\BT)^{-1}\BT^{n+1}\Sigh}\nonumber \nonumber\\&\qquad\qquad\qquad\qquad\qquad\qquad\qquad\qquad\qquad\qquad+\cnS\cdot\iprod{(\Sigh/\sigma^2)}{(\eyeT-\BT)^{-2}\BT^{n+1}\Sigh}\bigg)
\end{align}

To bound $\|\otimes_{2}\begin{bmatrix}-(c\eye-\g\Cov)\Cov^{-1/2}\\(\eye-\delta\Cov)\Cov^{-1/2}\end{bmatrix}
\|$, we will consider a bound along each eigendirection and accumulate the results:
\begin{align*}
\|\otimes_{2}\begin{bmatrix}-(c-\g\lambda_j)\lambda_j^{-1/2}\\(1-\delta\lambda_j)\lambda_j^{-1/2}\end{bmatrix}\|&\leq\frac{(c-\g\lambda_j)^2+(1-\delta\lambda_j)^2}{\lambda_j}\\
&\leq2\cdot\frac{(1+c^2)+(\g^2+\delta^2)\lambda_j^2}{\lambda_j}\\
&\leq2\cdot\frac{2+5\delta^2\lambda_j^2}{\lambda_j}\leq\frac{14}{\lambda_j}\\
\implies\|\otimes_{2}\begin{bmatrix}-(c\eye-\g\Cov)\Cov^{-1/2}\\(\eye-\delta\Cov)\Cov^{-1/2}\end{bmatrix}
\|&\leq\frac{14}{\mu}
\end{align*}

Next, we bound $\|\BT^k(\eyeT-\BT)^{-1}\Sigh\|$ (as a consequence of lemma~\ref{lem:B-contraction} with $\Q=\Sigh$):
\begin{align*}
\|\BT^k(\eyeT-\BT)^{-1}\Sigh\|&\leq\frac{1}{\lambda_{\min}(\G)}\|\G\T\BT^k(\eyeT-\BT)^{-1}\Sigh\|\\
&\leq\frac{1}{\lambda_{\min}(\G)}\sum_{l=k}^{\infty}\|\G\T\B^k\Sigh\|\\
&\leq\frac{\sqrt{\cnH\cnS}}{\ctwo\cthree\sqrt{2\cone-\cone^2}}\kappa(\G)\exp({-k\frac{\ctwo\cthree\sqrt{2\cone-\cone^2}}{\sqrt{\cnH\cnS}}})\|\Sigh\|\\
&\leq\frac{4\sigma^2\cnH}{\sqrt{1-\alpha^2}}\cdot\frac{\sqrt{\cnH\cnS}}{\cone\cthree^2}\exp({-k\frac{\cone\cthree^2}{\sqrt{\cnH\cnS}}})\cdot 9\delta^2\|\H\|_2\\
&\leq\frac{36\sigma^2\cnH}{\sqrt{1-\alpha^2}}\cdot\frac{\sqrt{\cnH\cnS}}{\cone\cthree^2}\exp({-k\frac{\cone\cthree^2}{\sqrt{\cnH\cnS}}})\cdot \delta
\end{align*}

This implies, 
\begin{align*}
\iprod{\bigg(\otimes_{2}\begin{bmatrix}-(c\eye-\g\Cov)\Cov^{-1/2}\\(\eye-\delta\Cov)\Cov^{-1/2}\end{bmatrix}\bigg)}{(\eyeT-\BT)^{-1}\BT^{n+1}\Sigh}\leq\nonumber\\504\cdot\frac{\cnH}{\sqrt{1-\alpha^2}}\cdot\frac{\sqrt{\cnH\cnS}}{\cone\cthree^2}\exp\bigg({-(n+1)\frac{\cone\cthree^2}{\sqrt{\cnH\cnS}}}\bigg)\cdot \frac{\delta}{\mu}\cdot \sigma^2d
\end{align*}

Furthermore, 
\begin{align*}
\frac{\cnS}{\sigma^2}\iprod{\Sigh}{(\eyeT-\BT)^{-2}\BT^{n+1}\Sigh}&=\frac{\cnS}{\sigma^2}\iprod{(\eyeT-\BT)^{-1}\BT^{(n+1)/2}\Sigh}{(\eyeT-\BT)^{-1}\BT^{(n+1)/2}\Sigh}\\
&\leq\frac{\cnS}{\sigma^2}\|(\eyeT-\BT)^{-1}\BT^{(n+1)/2}\Sigh\|^2\cdot d\\
&\leq1296\frac{\sigma^2d}{1-\alpha^2}\bigg(\cnH\frac{\sqrt{\cnH\cnS}}{\cone\cthree^2}\bigg)^2\delta^2\cnS\exp({-(n+1)\frac{\cone\cthree^2}{\sqrt{\cnH\cnS}}})
\end{align*}

This implies that,
\begin{align}
\label{eq:e42}
&\iprod{\bigg(\otimes_{2}\begin{bmatrix}-(c\eye-\g\Cov)\Cov^{-1/2}\\(\eye-\delta\Cov)\Cov^{-1/2}\end{bmatrix}\bigg)}{(\eyeT-\BT)^{-1}\BT^{n+1}\Sigh}+\frac{\cnS}{\sigma^2}\iprod{\Sigh}{(\eyeT-\BT)^{-2}\BT^{n+1}\Sigh}\nonumber\\&\qquad\qquad\leq2592\frac{\sigma^2d}{1-\alpha^2}\bigg(\cnH\frac{\sqrt{\cnH\cnS}}{\cone\cthree^2}\bigg)^2\delta^2\cnS\exp({-(n+1)\frac{\cone\cthree^2}{\sqrt{\cnH\cnS}}})\nonumber\\
&\qquad\qquad\leq2592\cdot\sigma^2d\cdot\bigg(\frac{\sqrt{\cnH\cnS}}{\cone\cthree^2}\bigg)^3\exp({-(n+1)\frac{\cone\cthree^2}{\sqrt{\cnH\cnS}}})\cdot\delta^2\cnH^2\cnS
\end{align}

Finally, we also note the following:
\begin{align*}
\frac{1}{(\g-c\delta)}\leq\frac{1}{(\gamma(1-\alpha))}\leq\frac{\mu}{(1-\alpha)^2}\leq\frac{4\cnS}{\delta\cfour}
\end{align*}

Plugging equation~\ref{eq:e42} into equation~\ref{eq:e41}, we get:
\begin{align}
\label{eq:e4}
\iprod{\begin{bmatrix}\H&0\\0&0\end{bmatrix}}{\Y_4}&=2592\cdot\frac{\sigma^2d}{(n-t)^2(\g-c\delta)^2}\cdot\bigg(\frac{\sqrt{\cnH\cnS}}{\cone\cthree^2}\bigg)^3\exp({-(n+1)\frac{\cone\cthree^2}{\sqrt{\cnH\cnS}}})\cdot\delta^2\cnH^2\cnS\nonumber\\
&\leq41472\cdot\frac{\sigma^2d}{(n-t)^2}\cdot\frac{1}{\cfour^2}\cdot\bigg(\frac{\sqrt{\cnH\cnS}}{\cone\cthree^2}\bigg)^3\exp({-(n+1)\frac{\cone\cthree^2}{\sqrt{\cnH\cnS}}})\cdot\cnH^2\cnS^3\nonumber\nonumber\\
&=41472\cdot\frac{\sigma^2d}{(n-t)^2}\cdot\frac{1}{\cfour^2(\cone\cthree^2)^3}\cdot\exp\bigg({-(n+1)\frac{\cone\cthree^2}{\sqrt{\cnH\cnS}}}\bigg)\cdot(\cnH\cnS)^{7/2}\cnS\nonumber\\
&\leq\UC\cdot\frac{\sigma^2d}{(n-t)^2}\cdot\exp\bigg({-(n+1)\frac{\cone\cthree^2}{\sqrt{\cnH\cnS}}}\bigg)\cdot(\cnH\cnS)^{7/2}\cnS
\end{align}

Next, we consider $\Y_5$:
\begin{align}
\label{eq:e51}
&\iprod{\begin{bmatrix}\H&0\\0&0\end{bmatrix}}{\Y_5}\nonumber\\&=\frac{1}{(n-t)^2}\sum_{j=t+1}^n\iprod{\begin{bmatrix}\H&0\\0&0\end{bmatrix}}{\big((\eyeT-\AL)^{-1}\AL^{n+1-j}+(\eyeT-\AR\T)^{-1}(\AR\T)^{n+1-j}\big)(\eyeT-\BT)^{-1}\BT^j\Sigh}\nonumber\\
&=\frac{1}{(n-t)^2}\sum_{j=t+1}^n\bigg(\iprod{(\eyeT-\A\T)^{-1}\A\T\begin{bmatrix}\H&0\\0&0\end{bmatrix}}{\A^{n-j}(\eyeT-\BT)^{-1}\BT^j\Sigh}\nonumber\\&\qquad\qquad\qquad\qquad\qquad+\iprod{\begin{bmatrix}\H&0\\0&0\end{bmatrix}\A(\eyeT-\A)^{-1}}{(\eyeT-\BT)^{-1}\BT^j\Sigh(\A\T)^{n-j}}\bigg)\nonumber\\
&\leq\frac{4d}{(n-t)^2}\sum_{j=t+1}^n\|(\eye-\A\T)^{-1}\A\T\begin{bmatrix}\H&0\\0&0\end{bmatrix}\|\cdot\|\A^{n-j}(\eyeT-\BT)^{-1}\BT^j\Sigh\|
\end{align}
In a manner similar to bounding $\|\A^{n-t}(\eyeT-\BT)^{-1}\Sigh\|$ as in equation~\ref{eq:e311}, we can bound $\|\A^{n-j}(\eyeT-\BT)^{-1}\BT^j\Sigh\|$ as:
\begin{align*}
\|\A^{n-j}(\eyeT-\BT)^{-1}\BT^j\Sigh\|&\leq\frac{108\sqrt{2}\sigma^2}{\sqrt{1-\alpha^2}}\cnH(n-j)\alpha^{(n-j-1)/2}\cdot\frac{\sqrt{\cnH\cnS}}{\ctwo\cthree\sqrt{2\cone-\cone^2}}\cdot\exp^{-(\frac{j\ctwo\cthree\sqrt{2\cone-\cone^2}}{\sqrt{\cnH\cnS}})}\cdot\delta^2\|\H\|
\end{align*}
Furthermore, we will consider the bound $\|(\eye-\A\T)^{-1}\A\T\begin{bmatrix}\H&0\\0&0\end{bmatrix}\|$ along one eigen direction (by employing equation~\ref{eq:intermediateEqn}) and collect the results:
\begin{align*}
\|(\eye-\A_j\T)^{-1}\A_j\T\begin{bmatrix}\lambda_j&0\\0&0\end{bmatrix}\|&\leq\frac{1+c^2}{\g-c\delta}\leq\frac{2}{\g-c\delta}\\
&\leq\frac{2}{\gamma(1-\alpha)}\leq\frac{4\cnS}{\delta\cfour}\\
\implies\|(\eye-\A\T)^{-1}\A\T\begin{bmatrix}\Cov&0\\0&0\end{bmatrix}\|&\leq\frac{4\cnS}{\delta\cfour}
\end{align*}
Plugging this into equation~\ref{eq:e51}, and upper bounding the sum by $(n-t)$ times the largest term of the series:
\begin{align}
\label{eq:e5}
\iprod{\begin{bmatrix}\H&0\\0&0\end{bmatrix}}{\Y_5}&\leq6912\cdot\sigma^2d\cdot\frac{(\cnH\cnS)^{7/4}}{\cthree\cfour(\cone\cthree)^{3/2}}\exp^{-(n+1)\cdot\frac{\ctwo\cthree\sqrt{2\cone-\cone^2}}{\sqrt{\cnH\cnS}}}\nonumber\\
&\leq\UC\cdot\sigma^2d\cdot(\cnH\cnS)^{7/4}\cdot\exp^{-(n+1)\cdot\frac{\ctwo\cthree\sqrt{2\cone-\cone^2}}{\sqrt{\cnH\cnS}}}
\end{align}
Summing up equations~\ref{eq:e1},~\ref{eq:e2},~\ref{eq:e3},~\ref{eq:e4},~\ref{eq:e5}, the statement of the lemma follows.
\end{proof}


\section{Proof of Theorem~\ref{thm:main}}\label{sec:proofMainTheorem}
\begin{proof}[Proof of Theorem~\ref{thm:main}]
	The proof of the theorem follows through various lemmas that have been proven in the appendix:
	\begin{itemize}
	\item Section~\ref{sec:tailAverageIterateCovariance} provides the bias-variance decomposition and provides an exact tensor expression governing the covariance of the bias error (through lemma~\ref{lem:average-covar-bias})and the variance error (lemma~\ref{lem:average-covar-var}).
	\item Section~\ref{sec:biasContraction} provides a scalar bound of the bias error through lemma~\ref{lem:bound-bias}. The technical contribution of this section (which introduces a new potential function) is in lemma~\ref{lem:main-bias}. 
	\item Section~\ref{sec:varianceContraction} provides a scalar bound of the variance error through lemma~\ref{lem:bound-variance}. The key technical contribution of this section is in the introduction of a stochastic process viewpoint of the proposed accelerated stochastic gradient method through lemmas~\ref{lem:main-variance},~\ref{lem:var-main-1}. These lemmas provide a tight characterization of the stationary distribution of the covariance of the iterates of the accelerated method. Lemma~\ref{lem:var1N2bound} is necessary to show the sharp burn-in (up to log factors), beyond which the leading order term of the error is up to constants the statistically optimal error rate $\mathcal{O}(\sigma^2 d/n)$.
	\end{itemize}
	Combining the results of these lemmas, we obtain the following guarantee of algorithm~\ref{algo:TAASGD}:
	
%	In particular, Lemma~\ref{lem:average-covar} gives the covariance matrix of the tail-averaged iterate. Combining this expression with the error estimates given by Lemma~\ref{lem:bias-bound} and Corollary~\ref{cor:bias-tail1} for bias, and Lemmas~\ref{lem:var-main-1},~\ref{lem:variance-main-1},~\ref{lem:variance-tail-1} and~\ref{lem:var1N2bound} for variance tell us that
\iffalse
	\begin{align*}
		\E{P(\xtilde)}-P(\xs) &\leq 10^7\cdot\frac{(\cnH\cnS)^{9/4}d}{(n-t)^2\delta}\cdot\exp\bigg(-\frac{t+1}{9\sqrt{\cnH\cnS}}\bigg)\norm{\thetat[0]}^2 +10^6\cdot\frac{(\cnH\cnS)^{5/4}d}{\delta}\exp\left(\frac{-n }{9\sqrt{\cnH\cnS}}\right) \cdot \norm{\thetat[0]}^2 + \\&5\frac{\sigma^2d}{n-t}+ 1360\cdot\frac{\sigma^2 d}{(n-t)^2} \sqrt{\cnH\cnS} + 10^6\cdot\sigma^2d\cdot(\cnH\cnS)^{7/4}\cdot\exp\bigg(\frac{-(n+1)}{9\sqrt{\cnH\cnS}}\bigg) \\ &+ 10^8\cdot\frac{\sigma^2d}{n-t}(\cnH\cnS)^{11/4}\alpha^{(n-t-1)/2} +10^{10}\cdot\frac{\sigma^2d}{(n-t)^2}\cdot\exp\bigg({-\frac{(n+1)}{9\sqrt{\cnH\cnS}}}\bigg)\cdot(\cnH\cnS)^{7/2}\cnS
	\end{align*}
	\fi
	\begin{align*}
		\E{P(\bar{\x}_{t,n})}-P(\xs) &\leq \UC\cdot\frac{(\cnH\cnS)^{9/4}d\cnH}{(n-t)^2}\cdot\exp\bigg(-\frac{t+1}{9\sqrt{\cnH\cnS}}\bigg)\cdot\big(P(\x_0)-P(\xs)\big) \\&+\UC\cdot(\cnH\cnS)^{5/4}d\cnH\cdot\exp\left(\frac{-n }{9\sqrt{\cnH\cnS}}\right) \cdot \big(P(\x_0)-P(\xs)\big) + 5\frac{\sigma^2d}{n-t}\\&+ \UC\cdot\frac{\sigma^2 d}{(n-t)^2} \sqrt{\cnH\cnS} + \UC\cdot\sigma^2d\cdot(\cnH\cnS)^{7/4}\cdot\exp\bigg(\frac{-(n+1)}{9\sqrt{\cnH\cnS}}\bigg) \\ &+ \UC\cdot\frac{\sigma^2d}{n-t}(\cnH\cnS)^{11/4}\exp\bigg(-\frac{(n-t-1)}{30\sqrt{\cnH\cnS}}\bigg) \\&+\UC\cdot\frac{\sigma^2d}{(n-t)^2}\cdot\exp\bigg({-\frac{(n+1)}{9\sqrt{\cnH\cnS}}}\bigg)\cdot(\cnH\cnS)^{7/2}\cnS
	\end{align*}	
	Where, $\UC$ is a universal constant.

%	proving the theorem.
\end{proof}

%\section{Simulations}\label{sec:simulations}
%
%\begin{figure}[t!] % "[t!]" placement specifier just for this example
%\begin{subfigure}{0.33\textwidth}
%\includegraphics[width=\linewidth,height=4cm]{Gaussian-bias.png}
%\caption{Bias Risk} \label{fig:1a}
%\end{subfigure}%\hspace*{\fill}
%\begin{subfigure}{0.33\textwidth}
%\includegraphics[width=\linewidth,height=4cm]{Gaussian-variance.png}
%\caption{Variance Risk} \label{fig:1b}
%\end{subfigure}
%\begin{subfigure}{0.33\textwidth}
%\includegraphics[width=\linewidth,height=4cm]{Gaussian-total.png}
%\caption{Total Risk} \label{fig:1c}
%\end{subfigure}
%\caption{Comparing Accelerated SGD with SGD for the Gaussian distribution. The start of tail-averaging is indicated with a vertical black line. The bias (figure~\ref{fig:1a}) and total risk (figure~\ref{fig:1c}) indicate the superior behavior of accelerated SGD compared to SGD, while the variance (figure~\ref{fig:1b}) of accelerated SGD is a constant factor worse off SGD.} \label{fig:Gaussian}
%\end{figure}
%\begin{figure}[t!] % "[t!]" placement specifier just for this example
%\begin{subfigure}{0.33\textwidth}
%\includegraphics[width=\linewidth,height=4cm]{probability-bias.png}
%\caption{Bias Risk} \label{fig:2a}
%\end{subfigure}%\hspace*{\fill}
%\begin{subfigure}{0.33\textwidth}
%\includegraphics[width=\linewidth,height=4cm]{probability-variance.png}
%\caption{Variance Risk} \label{fig:2b}
%\end{subfigure}
%\begin{subfigure}{0.33\textwidth}
%\includegraphics[width=\linewidth,height=4cm]{probability-total.png}
%\caption{Total Risk} \label{fig:2c}
%\end{subfigure}
%\caption{Comparing Accelerated SGD with SGD for the equal probability discrete distribution. The start of tail-averaging is shown using a vertical black line. The bias (figure~\ref{fig:2a}), variance (figure~\ref{fig:2b}) and total error (figure~\ref{fig:2c}) plots corroborate with theory in that Acceleration does not offer any improvement over SGD for this worst-case example.}\label{fig:Discrete}
%\end{figure}
%\iffalse
%\begin{figure*}[t!]
%	\begin{tabular}{ccc}\hspace*{-10pt}
%		\includegraphics[width=.33\textwidth]{figures1/Gaussian-bias.png}&\hspace*{-20pt}
%		\includegraphics[width=.33\textwidth]{figures1/Gaussian-variance.png}&\hspace*{-20pt}
%		\includegraphics[width=.33\textwidth]{figures1/Gaussian-total.png}\\
%		(a)&(b)&(c)\label{fig1}
%	\end{tabular}
%\caption{Plots (a),(b),(c) represent respectively the behavior of the bias, variance and the total error for the case when the inputs are sampled from a Gaussian Distribution. Note that tail-averaging begins after the point indicated by the black vertical line. We note that for the Gaussian case, $\cnS<<\cnH$ thus indicating huge wins for TASGD (blue) over SGD (red) in terms of the rate of killing the initial error (plot(a)). The variance of TASGD (plot (b)) matches the behavior indicated by the theory bound in that it is worse compared to classic tail-averaged SGD by a constant factor. The total risk (plot (c)) indicates a significantly lower generalization error of TASGD compared to vanilla tail-averaged SGD indicating vastly superior sample complexity.}
%\end{figure*}
%
%\begin{figure*}[t!]\label{fig2}
%	\begin{tabular}{ccc}\hspace*{-10pt}
%		\includegraphics[width=.33\textwidth]{figures1/probability-bias.png}&\hspace*{-20pt}
%		\includegraphics[width=.33\textwidth]{figures1/probability-variance.png}&\hspace*{-20pt}
%		\includegraphics[width=.33\textwidth]{figures1/probability-total.png}\\
%		(d)&(e)&(f)
%	\end{tabular}
%\caption{Plots (d),(e),(f) represent respectively the behavior of the bias, variance and the total error for the case when the inputs are sampled from a Discrete Distribution with equal probabilities of observing each direction. Note that tail-averaging begins after the point indicated by the black vertical line. In this case, $\cnS=\cnH$ thus indicating TASGD (blue) does not improve upon classic tail-averaged SGD (red) in terms of the rate of killing the initial error (plot(d)), as well as in the variance error and total generalization error, thus corroborating with bounds indicated by theory. }
%\end{figure*}
%\fi
%We present an empirical study of the performance of the proposed accelerated SGD method by considering two data distributions, namely, (a) the Gaussian distribution, and  (b) the discrete distribution with equal probability of observing each direction. Recall that our bound for initial error (bias) indicates advantage of using accelerated SGD over SGD only if the condition number $\cnH$ is significantly larger than the statistical condition number $\cnS$ (see Section~\ref{sec:prob}), which implies a significantly faster rate of initial error decay for the Gaussian case and no improvement for the discrete case.
%
%{\bf Methodology:} In each experiment, we generate $n=\cnH$ samples of $(\a_i, b_i)$ pairs, where $\a_i\in \R^{50}$ is sampled from ${\cal D}$ while $b_i$ is sampled from ${\cal N}(\iprod{\a_i}{\vec{x}^*}, \sigma^2)$ with $\sigma=10$. The distribution ${\cal D}$ has second moment $\H=\E{\a_i\otimes \a_i}$ with smallest eigenvalue $\mu=10^{-4}$ while $\|\H\|_2=1$. We run both SGD and  accelerated SGD for a total of $\cnH$ iterations, and begin the tail-averaging phase after $\cnH/10$ iterations. We report excess risk on log scale averaged over $100$ trials where $\x^*$ is fixed, but the optimization is run with freshly sampled data points in each trial. 
%
%We first consider the Gaussian distribution with diagonal covariance. Note that the statistical condition number $\cnS$ for Gaussian distribution is always bounded by $\cnS\leq3d$ irrespective of $\mu$. The condition number $\cnH=\frac{R^2}{\mu}$ in this case is significantly bigger compared to $\sqrt{\cnH\cnS}$, implying significantly faster bias decay for accelerated SGD compared to SGD, which is indeed the behavior observed in figure~\ref{fig:1a}. Moreover, Figure~\ref{fig:1b} presents evidence that the variance of accelerated SGD is not off by more than constant factors off the vanilla SGD. Finally, figure~\ref{fig:1c} lends credence to the claim that accelerated SGD provides a much stronger rate in competing on the overall generalization error compared to classic tail-averaged SGD, thus indicating significantly improved sample complexity.
%
%Next, we consider a distribution over $1$-sparse vectors with exactly same magnitude. That is, the non-zero coordinate $j$ is sampled with probability $p_j$ and then the corresponding $\a_i$ is given by $\vec{e}_j$, the $j$-th canonical basis vector. In this case, $\cnH=\cnS=\frac{1}{p_{min}}$. This implies that the bias decay for accelerated SGD matches that of classic SGD. Figure~\ref{fig:2a},~\ref{fig:2b},~\ref{fig:2c} corroborates this claim that the behavior of accelerated SGD nearly degenerates to classic SGD when considering bias, variance and total error, thus indicating that there exists worst case distributions where acceleration with stochastic oracles does not offer any improvement over classic SGD.
%\iffalse
%However, if each $\a_i$ is sampled from $1$-sparse vectors with different magnitude, i.e., $\a_i = \alpha_j \vec{e}_j$ if $j$-th coordinate is sampled uniformly with probability $p_i=\frac{1}{d}$. Then, $\cnH=\frac{1}{\mu}$ as $R=1$ but $\cnS\leq d$, again indicating significant difference in $\cnH$ and $\cnS$ and hence between bias error bounds for SGD and TASGD. Figure~\ref{fig:fig1} (d) verifies our claim as bias of TASGD decays significantly more rapidly than that of SGD. 
%\fi


\iffalse
\section{Proof of Theorem~\ref{thm:par}}\label{sec:thm-par}
\begin{algorithm}[t]
	\caption{ cde}%$\MTASGD(\set{(\ai[1],\bi[1]),\cdots,(\ai[n],\bi[n])},$ $\xt[0], m, t, \alpha, \beta, \gamma, \delta )$ \\ Mini-batched, tail-averaged accelerated stochastic gradient descent. Here $\widehat{\nabla}_k P(\x)$ denotes the gradient at $\x$ due to the $k^{\textrm{th}}$ sample $(\ai[k],\bi[k])$ i.e., $\widehat{\nabla}_k P(\x) \eqdef \left(\iprod{\ai[k]}{\x} - \bi[k]\right)\ai[k]$. }\label{algo:MBTAASGD}
	\begin{algorithmic}[1]
		\INPUT Samples $(\ai[1],\bi[1]),\cdots,(\ai[n],\bi[n])$, Initial point $\xt[0]$, Mini-batch size $m$, Non-averaging phase $t$, Parameters $\alpha, \beta, \gamma, \delta$
		\STATE $\vt[0] = \xt[0]$
		\FOR{$j = 1, \cdots n/m$}
		\STATE $\yt[j-1] \leftarrow \alpha \xt[j] + (1-\alpha) \vt[j]$
		\STATE $\zt[j-1] \leftarrow \beta \yt[j] + (1-\beta) \vt[j]$
		\STATE $\xt[j] \leftarrow \yt[j-1] - \delta \frac{1}{m} \sum_{k=(j-1)*m+1}^{j*m} \widehat{\nabla}_k P(\x)$
		\STATE $\vt[j] \leftarrow \zt[j-1] - \gamma \frac{1}{m} \sum_{k=(j-1)*m+1}^{j*m} \widehat{\nabla}_k P(\x)$
		\ENDFOR
		\OUTPUT $\frac{1}{n-t}\sum_{j=t+1}^{n} \xt[j]$
	\end{algorithmic}
\end{algorithm}

\begin{proof}[Proof of Theorem~\ref{thm:par}]
	The proof follows easily by combining Theorem~\ref{thm:main} with the ideas of~\cite{JainKKNS16}. The analysis can be broken down into three parts.
	
	\textbf{Initial phase}: In this phase, Algorithm~\ref{algo:PASGD} runs Algorithm~\ref{algo:MBTAASGD} with a mini-batch size of $m = \max\left(\frac{\infbound}{\twonorm{\Cov}}, \cnS\right)$. For this mini-batch size, recall that we have condition number $\cnH_m \leq \frac{m-1}{m}\kappa(\Cov) + \frac{1}{m} \cnH \leq 2 \kappa(\Cov)$, statistical condition number $\cnS_m \leq \frac{m-1}{m} + \frac{1}{m} \cnS \leq 2$ and noise level $\sigma_m^2 = \sigma^2 / m$. Applying Theorem~\ref{thm:main} for this setting tells us that after every invocation of Algorithm~\ref{algo:MBTAASGD}, the initial error decays by a factor of $2$. So the error of the iterate $\xt[j]$ at the end of the initial phase is at most $L(\xt[j]) \leq 8 \sigma^2 d/m$. If $n < 2qm \log \frac{E_0 m}{\sigma^2}$, note that the iterate after initial phase already satisfies the theorem's guarantee.
	
	\textbf{Middle phase}: In this phase, again the condition number and statistical condition numbers are bounded by $2\kappa(\Cov)$ and $2$ respectively. We show that in every iteration $j$, after the function call to Algorithm~\ref{algo:MBTAASGD}, the following invariant is maintained:
	\begin{align}
		L(\xt[j]) \leq \frac{8 \sigma^2 d}{m_0*2^{j-1}},\label{eqn:geomdecay}
	\end{align}
	where $m_0$ is the initial mini-batch size.
	The base case for $j=1$ holds by the guarantee from the initial phase. Suppose now that the above condition holds for some $j$ and we wish to establish it for $j+1$. From Theorem~\ref{thm:main}, we see that:
	\begin{align*}
		L(\xt[j+1]) &\leq \exp\left(\frac{-\sqrt{7}q}{4\sqrt{\kappa(\Cov)}}\right) L(\xt[j]) + \frac{4 \sigma^2 d}{m_0 2^{j}},
	\end{align*}
	where we used the fact that the variance with a mini-batch size of $m_0*2^j$ in $j+1^{\textrm{th}}$ iteration is $\frac{\sigma^2}{m_0 2^j}$ Substituting the value of number of iterations $q$ and the hypothesis on $L(\xt[j])$ finish the induction step.
	
	\textbf{Final phase}: In the final phase, we again directly apply Theorem~\ref{thm:main} to conclude that the loss of output $\x$ can be bounded by
	\begin{align*}
		L(\x) \leq \exp\left(\frac{-\sqrt{7}q}{4\sqrt{\kappa(\Cov)}}\right) L(\xt[j]) + \frac{4 \sigma^2 d}{n/2} \leq \frac{9 \sigma^2 d}{n} \leq \exp\left(-n/qm\right) L(\xt[0]) + \frac{9 \sigma^2 d}{n},
	\end{align*}
	where we used~\eqref{eqn:geomdecay} in the second last step.
	This proves the theorem.
\end{proof}
\fi