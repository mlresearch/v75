\documentclass[final,12pt]{colt2018} % Anonymized submission
% \documentclass{colt2017} % Include author names

% The following packages will be automatically loaded:
% amsmath, amssymb, natbib, graphicx, url, algorithm2e

\title[Lower Bounds for Higher-Order Convex Optimization]{Lower Bounds for Higher-Order Convex Optimization}
\usepackage{times}
\usepackage{amssymb,amsfonts}
\usepackage{multirow}
\usepackage{hhline}
\usepackage{makecell}
 % Use \Name{Author Name} to specify the name.
 % If the surname contains spaces, enclose the surname
 % in braces, e.g. \Name{John {Smith Jones}} similarly
 % if the name has a "von" part, e.g \Name{Jane {de Winter}}.
 % If the first letter in the forenames is a diacritic
 % enclose the diacritic in braces, e.g. \Name{{\'E}louise Smith}

 %Two authors with the same address
  \coltauthor{\Name{Naman Agarwal} \Email{namana@cs.princeton.edu}\\
  \addr Department of Computer Science, Princeton University, Princeton, NJ \\
   \Name{Elad Hazan} \Email{ehazan@cs.princeton.edu}\\
   \addr Department of Computer Science, Princeton University, Princeton, NJ\\
   \addr Google Brain}

 % Three or more authors with the same address:
 % \coltauthor{\Name{Author Name1} \Email{an1@sample.com}\\
 %  \Name{Author Name2} \Email{an2@sample.com}\\
 %  \Name{Author Name3} \Email{an3@sample.com}\\
 %  \addr Address}

%\def\pinv{\dag}
%\def\ltwo#1{\left\|{#1}\right\|_2}
%\def\sqrtm#1{{#1}^{1/2}}
\def\half{\frac{1}{2}}
%\def\outerproduct#1#2{{#1}{#2}^{\!\!\!\top}}
%\def\xdomain{\mathcal{X}}

\DeclareMathOperator*{\argmin}{argmin}
\DeclareMathOperator*{\argmax}{argmax}
%%%%%% Author Notes %%%%%%%%
\newcommand{\na}[1]{$\ll$\textsf{\color{red} Naman : #1}$\gg$}
\newcommand{\elad}[1]{$\ll$\textsf{\color{blue} Elad : #1}$\gg$}
\newcommand{\bab}[1]{$\ll$\textsf{\color{green} Brian : #1}$\gg$}
\newcommand{\tm}[1]{$\ll$\textsf{\color{blue} Tengyu : #1}$\gg$}
\newcommand{\zeyuan}[1]{$\ll$\textsf{\color{purple} Zeyuan: #1}$\gg$}

\def\X{{\mathcal X}}
\def\H{{\mathcal H}}
\def\Y{{\mathcal Y}}
\def\D{{\mathcal D}}
\def\K{{\mathcal K}}
\def\L{{\mathcal L}}
\def\F{{\mathcal F}}
\def\mP{{\mathcal P}}
\def\reals{{\mathbb R}}
\def\R{{\mathcal R}}
\def\fhat{{\ensuremath{\hat{f}}}}
\def\mH{\mathcal{H}}
\newcommand{\at}{\makeatletter @\makeatother}

\DeclareMathOperator{\clip}{clip}
\def\norm#1{\mathopen\| #1 \mathclose\|}% use instead of $\|x\|$

\newcommand{\proj}{\mathop{\prod}}

\newcommand{\err}{\mathop{\mbox{\rm error}}}
\newcommand{\rank}{\mathop{\mbox{\rm rank}}}
\newcommand{\ellipsoid}{{\mathcal E}}
\newcommand{\sround}{\mathop{\mbox{\rm SmartRound}}}
\newcommand{\round}{\mathop{\mbox{\rm round}}}
\newcommand{\myspan}{\mathop{\mbox{\rm span}}}
\newcommand{\edges}{\mathop{\mbox{\rm edges}}}
\newcommand{\conv}{\mathop{\mbox{\rm conv}}}
\newcommand{\sign}{\mathop{\mbox{\rm sign}}}

\newcommand{\ignore}[1]{}
\newcommand{\equaldef}{\stackrel{\text{\tiny def}}{=}}
\newcommand{\equaltri}{\triangleq}
\newcommand{\tr}{\ensuremath{{\scriptscriptstyle\mathsf{T}}}}
\def\trace{{\bf Tr}}
\def\dist{{\bf P}}
\def\mydist{{\bf Q}}
\def\reals{{\mathbb R}}
\newcommand{\E}{\mathop{\mbox{\bf E}}}
\newcommand\ball{\mathbb{B}}
\newcommand\var{\mbox{VAR}}
\def\bzero{\mathbf{0}}

\def\Exp{\mathop{\mbox{E}}}
\def\mP{{\mathcal P}}
\def\risk{{\mbox{Risk}}}
\def\ent{{\mbox{ent}}}
\def\mG{{\mathcal G}}
\def\mA{{\mathcal A}}
\def\V{{\mathcal V}}
\def\lhat{\hat{\ell}}
\def\Lhat{\hat{L}}
\def\ltil{\tilde{\ell}}
\def\Ltil{\tilde{L}}

\def\Ahat{\hat{\mathbf{A}}}

\def\bold0{\mathbf{0}}
\def\boldone{\mathbf{1}}
\def\boldp{\mathbf{p}}
\def\boldq{\mathbf{q}}
\def\boldr{\mathbf{r}}
\def\boldA{\mathbf{A}}
\def\boldI{\mathbf{I}}
\newcommand\mycases[4] {{
\left\{
\begin{array}{ll}
    {#1} & {#2} \\\\
    {#3} & {#4}
\end{array}
\right. }}
\newcommand\mythreecases[6] {{
\left\{
\begin{array}{ll}
    {#1} & {#2} \\\\
    {#3} & {#4} \\\\
    {#5} & {#6}
\end{array}
\right. }}
\def\ba{\mathbf{a}}
\def\bb{\mathbf{b}}
\def\bc{\mathbf{c}}
\def\bB{\mathbf{B}}
\def\bC{\mathbf{C}}
\def\be{\mathbf{e}}
\def\etil{\tilde{\mathbf{e}}}
\def\bh{\mathbf{h}}
\def\bk{\mathbf{k}}
\def\bu{\mathbf{u}}
\def\bx{\mathbf{x}}
\def\w{\mathbf{w}}
\def\by{\mathbf{y}}
\def\bz{\mathbf{z}}
\def\bp{\mathbf{p}}
\def\bq{\mathbf{q}}
\def\br{\mathbf{r}}
\def\bu{\mathbf{u}}
\def\bv{\mathbf{v}}

\def\bw{\mathbf{w}}
\def\ba{\mathbf{a}}
\def\bA{\mathbf{A}}
\def\bM{\mathbf{M}}
\def\bS{\mathbf{S}}
\def\bG{\mathbf{G}}
\def\bH{\mathbf{H}}
\def\bI{\mathbf{I}}
\def\bJ{\mathbf{J}}
\def\bP{\mathbf{P}}
\def\bQ{\mathbf{Q}}
\def\bV{\mathbf{V}}
\def\bone{\mathbf{1}}


\def\xhat{\hat{\mathbf{x}}}
\def\xbar{\bar{\mathbf{x}}}
\def\vol{\mbox{vol}}
\def\trace{{\bf Tr}}
\def\Exp{\mathop{\mbox{E}}}
\newcommand{\val}{\mathop{\mbox{\rm val}}}
\renewcommand{\deg}{\mathop{\mbox{\rm deg}}}
\newcommand{\eps}{\varepsilon}

% Macros for this chapter
\def\bone{\mathbf{1}}
\newcommand{\sphere}{\ensuremath{\mathbb {S}}}
\newcommand{\simplex}{\ensuremath{\Delta}}
\newcommand{\diag}{\mbox{diag}}
%\newcommand{\K}{\ensuremath{\mathcal K}}
\newcommand{\Ktil}{\ensuremath{\mathcal{\tilde{K}}}}
\def\mA{{\mathcal A}}
\newcommand{\x}{\ensuremath{\mathbf x}}
\newcommand{\vn}{\ensuremath{\mathbf n}}
\newcommand{\rv}[1][t]{\ensuremath{\mathbf r_{#1}}}
\newcommand{\y}{\ensuremath{\mathbf y}}
\newcommand{\z}{\ensuremath{\mathbf z}}
\newcommand{\h}{\ensuremath{\mathbf h}}
\newcommand{\xtil}[1][t]{\ensuremath{\mathbf {\tilde{x}}_{#1}}}
\newcommand{\xv}[1][t]{\ensuremath{\mathbf x_{#1}}}
\newcommand{\xvh}[1][t]{\ensuremath{\mathbf {\hat x}}_{#1}}
\newcommand{\ftildeh}[1][t]{\ensuremath{\mathbf {\hat f}}_{#1}}
\newcommand{\xvbar}[1][t]{\ensuremath{\bar{\mathbf x}_{#1}}}
\newcommand{\xstar}{\ensuremath{\mathbf x^{*}}}
\newcommand{\xstarr}{\ensuremath{\mathbf x^{\star}}}
\newcommand{\yv}[1][t]{\ensuremath{\mathbf y_{#1}}}
\newcommand{\zv}[1][t]{\ensuremath{\mathbf z_{#1}}}
\newcommand{\fv}[1][t]{\ensuremath{\mathbf f_{#1}}}
\newcommand{\gv}[1][t]{\ensuremath{\mathbf g_{#1}}}
\newcommand{\qv}[1][t]{\ensuremath{\mathbf q_{#1}}}
\newcommand{\pv}[1][t]{\ensuremath{\mathbf p_{#1}}}
\newcommand{\ftilde}[1][t]{\ensuremath{\mathbf {\tilde f}_{#1}}}
\newcommand{\gtilde}[1][t]{\ensuremath{\mathbf {\tilde g}_{#1}}}
\newcommand{\ev}[1][i]{\ensuremath{\mathbf e_{#1}}}
\newcommand{\uv}{\ensuremath{\mathbf u}}
\newcommand{\vv}{\ensuremath{\mathbf v}}
%\newcommand{\A}[1][t]{\ensuremath{\mathbf A_{#1}}}
\newcommand{\Ftilde}[1][t]{\ensuremath{\mathbf {\tilde F}_{#1}}}
\newcommand{\Gtilde}[1][t]{\ensuremath{\mathbf {\tilde G}_{#1}}}
\newcommand{\hessum}[1][t]{\ensuremath{\boldsymbol{H}_{1:#1}}}
\newcommand{\reg}[1][t]{\ensuremath{\lambda_{#1}}}
\newcommand{\regsum}[1][t]{\ensuremath{\boldsymbol{\lambda}_{1:#1}}}
%\newcommand{\F}{\ensuremath{\mathcal F}}
\newcommand{\ntil}{\tilde{\nabla}}
\newcommand{\Ftil}{\tilde{F}}
\newcommand{\ftil}{\tilde{f}}
\newcommand{\rhotil}{\tilde{\rho}}
\newcommand{\gtil}{\tilde{g}}
\newcommand{\mutil}{\tilde{\mu}}
\newcommand{\Atil}{\tilde{A}}
\newcommand{\Qtil}{\tilde{Q}}
\newcommand{\vtil}{\tilde{v}}
\newcommand\super[1] {{^{(#1)}}}
\newcommand\pr{\mbox{\bf Pr}}
\newcommand\av{\mbox{\bf E}}
\def\regret{\ensuremath{\mbox{Regret}}}
\def\bb{\mathbf{b}}
%\def\bB{\mathbf{B}}
%\def\bB{{B}}
%\def\bC{\mathbf{C}}
\def\bC{{C}}
\def\be{\mathbf{e}}
\def\bu{\mathbf{u}}
\def\bx{\mathbf{x}}
\def\by{\mathbf{y}}
\def\bz{\mathbf{z}}
\def\bp{\mathbf{p}}
\def\bdelta{\mathbf{\delta}}
\def\bq{\mathbf{q}}
\def\br{\mathbf{r}}
\def\bv{\mathbf{v}}
\def\ba{\mathbf{a}}
\def\bg{\mathbf{g}}
\def\bs{\mathbf{s}}
\newcommand{\dm}{\delta_{\times}}
\newcommand{\cA}{\mathcal{A}}
%\def\bA{\mathbf{A}}
%\def\bA{{A}}
\def\bI{\mathbf{I}}
\def\bJ{\mathbf{J}}
\def\bP{\mathbf{P}}
\def\bV{\mathbf{V}}
\def\sgn{\text{sgn}}
\def\eps{\varepsilon}
\def\epsilon{\varepsilon}
\def\veps{\varepsilon}
\newcommand{\etal}{ \textit{et al.}\xspace}
\def\tsum{{\textstyle \sum}}
\def\universal{{\sc Universal}\xspace}
\def\ogd{{\sc Online Gradient Descent}\xspace}
\def\R{\ensuremath{\mathcal R}}
\def\lmid{\lambda_{\text{mid}}}

\def\hestinv{\tilde{\nabla}^{-2}}
\def\hest{\tilde{\nabla}^{2}}
\def\hessinv{\nabla^{-2}}
\def\hess{\nabla^2}
\def\grad{\nabla}
\newcommand{\defeq}{\triangleq}

\newcommand*\xor{\mathbin{\oplus}}
\newcommand{\innerprod}[2]{\langle #1, #2 \rangle}
\newcommand{\matnorm}[2]{\| #1 \|_{#2}}
\newcommand{\invmatnorm}[2]{\| #1 \|_{#2^{-1}}}


%specific to non convex second order paper
\newcommand{\hlinverse}{(\bH + \lambda \bI)^{-1}}
\newcommand{\hlpinverse}{(\bH + \lambda \bI)^{+}}
\newcommand{\hlinversea}[1]{(\bH + #1 \bI)^{-1}}
\newcommand{\hlpinversea}[1]{(\bH + #1 \bI)^{+}}
\newcommand{\ceil}[1]{\left\lceil #1 \right\rceil}

\newcommand{\Tg}{\mathbb{T}_g}
\newcommand{\Th}{\mathbb{T}_h}
\newcommand{\Thh}{\mathbb{T}_{h,1}}
\newcommand{\Time}{\mathbb{T}_{\mathsf{inverse}}}

\newtheorem{innercustomthm}{Theorem}
\newenvironment{stheorem}[1]
  {\renewcommand\theinnercustomthm{#1}\innercustomthm}
  {\endinnercustomthm}
\newtheorem{innercustomlemma}{Lemma}
\newenvironment{slemma}[1]
  {\renewcommand\theinnercustomlemma{#1}\innercustomlemma}
  {\endinnercustomlemma}
%
\newtheorem{ctheorem}{Theorem}
\newtheorem{clemma}{Main Lemma}

% \newtheorem{theorem}{Theorem}[section]
% %\newtheorem*{theorem*}{Theorem}

\newtheorem{claim}[theorem]{Claim}
% \newtheorem{subclaim}{Claim}[theorem]
% \newtheorem{proposition}[theorem]{Proposition}
% \newtheorem{lemma}[theorem]{Lemma}
% \newtheorem{corollary}[theorem]{Corollary}
% \newtheorem{conjecture}[theorem]{Conjecture}
\newtheorem{observation}[theorem]{Observation}
% \newtheorem{fact}[theorem]{Fact}


% \newtheorem{definition}[theorem]{Definition}
% \newtheorem{construction}[theorem]{Construction}
% \newtheorem{example}[theorem]{Example}
% \newtheorem{algorithm1}[theorem]{Algorithm}
% \newtheorem{protocol}[theorem]{Protocol}
% \newtheorem{assumption}[theorem]{Assumption}

%\newtheorem*{rep@theorem}{\rep@title}
\newcommand{\newreptheorem}[2]{%
\newenvironment{rep#1}[1]{%
 \def\rep@title{#2 \ref{##1}}%
 \begin{rep@theorem}}%
 {\end{rep@theorem}}}


\newreptheorem{theorem}{Theorem}
\newreptheorem{lemma}{Lemma}
\newreptheorem{proposition}{Proposition}
\newreptheorem{claim}{Claim}
\newreptheorem{corollary}{Corollary}
\newreptheorem{mainlemma}{Main Lemma}



% \theoremstyle{remark}
% \newtheorem{remark}[theorem]{Remark}

%%%%%%%%%%%%%%%%%%%%%%%%%%%%%%%%%%%%%%%%%%%%%%%%%%%%%%%%%%%%%%%%%%%%%%%%%%%%%%%%
% REFERNCE MACROS

% \newcommand{\namedref}[2]{\mbox{\hyperref[#2]{#1~\ref*{#2}}}}

% \newcommand{\chapterref}[1]{\namedref{Chapter}{#1}}
% \newcommand{\sectionref}[1]{\namedref{Section}{#1}}
% \newcommand{\appendixref}[1]{\namedref{Appendix}{#1}}
% \newcommand{\theoremref}[1]{\namedref{Theorem}{#1}}
% \newcommand{\factref}[1]{\namedref{Fact}{#1}}
% \newcommand{\remarkref}[1]{\namedref{Remark}{#1}}
% \newcommand{\definitionref}[1]{\namedref{Definition}{#1}}
% \newcommand{\figureref}[1]{\namedref{Figure}{#1}}
% \newcommand{\figurerefb}[2]{\mbox{\hyperref[#1]{Figure~\ref*{#1}#2}}}
% \newcommand{\tableref}[1]{\namedref{Table}{#1}}
% \newcommand{\lemmaref}[1]{\namedref{Lemma}{#1}}
% \newcommand{\mainlemmaref}[1]{\namedref{Main Lemma}{#1}}
% \newcommand{\claimref}[1]{\namedref{Claim}{#1}}
% \newcommand{\propositionref}[1]{\namedref{Proposition}{#1}}
% \newcommand{\corollaryref}[1]{\namedref{Corollary}{#1}}
% \newcommand{\constructionref}[1]{\namedref{Construction}{#1}}
% \newcommand{\itemref}[1]{\namedref{Item}{#1}}
% \newcommand{\propertyref}[1]{\namedref{Property}{#1}}
% \newcommand{\protocolref}[1]{\namedref{Protocol}{#1}}
% \newcommand{\algorithmref}[1]{\namedref{Algorithm}{#1}}
% \newcommand{\assumptionref}[1]{\namedref{Assumption}{#1}}
% \newcommand{\stepref}[1]{\namedref{Step}{#1}}
% \newcommand{\paramref}[1]{\namedref{Parameter}{#1}}
% \newcommand{\footnoteref}[1]{\namedref{Footnote}{#1}}
% \newcommand{\equationref}[1]{\mbox{\hyperref[#1]{(\ref*{#1})}}}
% \renewcommand{\eqref}{\equationref}
% %\newcommand{\equationref}[1]{\namedref{Equation}{#1}}
% \newcommand{\exampleref}[1]{\namedref{Example}{#1}}
%\newcommand{\lineref}[1]{\namedref{Line}{#1}}
%%%%%%%%%%%%%%%%%%%%%%%%%%%%%%%%%%%%%%%%%%%%%%%%%%%%%%%%%%%%%%%%%%%%%%%%%%%%%%%%

\numberwithin{equation}{section}


\newcommand{\da}{\text{\ding{172}}\xspace}
\newcommand{\db}{\text{\ding{173}}\xspace}
\newcommand{\dc}{\text{\ding{174}}\xspace}
\newcommand{\dd}{\text{\ding{175}}\xspace}
\newcommand{\de}{\text{\ding{176}}\xspace}
\newcommand{\df}{\text{\ding{177}}\xspace}
\newcommand{\dg}{\text{\ding{178}}\xspace}



%%% Paper specific Macros

\def\smooth#1{#1_{\delta,\Gamma}}
\def\deltat{\delta_t}
\def\hardf{f^{\dagger}}
\def\tf{\tilde{f}}
\def\smallq{\frac{1}{20T^{1.5}}}
\def\alg{\mathrm{ALG}}


\newcommand{\naedit}[1]{{\color{red}#1}}



\begin{document}

\maketitle

\begin{abstract}
State-of-the-art methods in mathematical optimization employ higher-order derivative information. We explore the limitations of higher-order optimization and prove that even for convex optimization, a polynomial dependence on the approximation guarantee and higher-order smoothness parameters is necessary. This refutes the hope that higher-order smoothness and higher-order derivatives can lead to dimension free polynomial time algorithms for convex optimization. As a special case, we show Nesterov's accelerated cubic regularization method and higher-order methods to be nearly tight. 
\end{abstract}

\begin{keywords}
Convex Optimization, Second-Order Optimization, Higher-Order Optimization, Newton's method, Lower Bounds.
\end{keywords}

\section{Introduction}
%This paper explores whether linearly-converging \footnote{methods for which the dependence on the additive error $\epsilon$ is logarithmic} convex optimization methods  necessarily require polynomial dependence on dimension. 

Linearly-converging\footnote{methods for which the dependence on the additive error $\epsilon$ is logarithmic}  convex optimization algorithms  fall into two categories.   The first are methods whose iteration complexity scales with the dimension. This includes the ellipsoid \citep{khachiyan1980polynomial,grotschel2012geometric}, cutting plane \citep{vaidya1989new,lee2015faster} and random-walk \citep{bertsimas2004solving,kalai2006simulated,lovasz2006fast} based methods. They solve convex optimization in the general membership oracle model. 

The other category of linearly-converging algorithms is iterative derivative-based methods. These achieve fast dimension-free iteration rates for certain types of convex functions, namely those that are strongly convex and smooth. Indeed gradient descent and further extensions have therefore been extremely successful for optimization especially in machine learning.

More recently state-of-the-art optimization for machine learning has shifted from gradient based methods, namely stochastic gradient descent and its derivatives  \citep{adagrad,svrg}, to methods that are based on higher moments.  Notably, the fastest theoretical running times for both convex \citep{LiSSA2016,xu2016sub,bollapragada2016exact} and non-convex \citep{Agarwal:2017:FAL:3055399.3055464,CarmonAGD} optimization are attained by algorithms that either explicitly or implicitly exploit second-order information and third-order smoothness. 

Of particular interest is Newton's method, due to recent efficient implementations that run in near-linear time in the input representation. The hope was that Newton's method or higher-order methods can achieve logarithmic in error iteration complexity which is independent of the dimensionality, which is  extremely high for many large-scale applications, and without requiring strong convexity.  %Table \ref{table:main} gives existing optimization algorithms and their iteration complexity below: currently all poly-time methods require either dimension dependence or condition number (which requires strong convexity) dependence on the number of iterations. 






\begin{table}[h]

\label{table:main}
\begin{center}
\begin{tabular}{ |c|c|c|c|c|} 
 
 \hline
 \textbf{Method} & \thead{\textbf{Dim.} \\ \textbf{Free}} & \textbf{Order} & \textbf{Assumptions} & \textbf{Upper Bound}\\
 \hline   % \\ Cutting plane \cite{lee2015faster} 
 \makecell{Ellipsoid \\ \citep{grotschel2012geometric} \\ \citep{vaidya1989new} \\ Random Walks \\ \citep{kalai2006simulated}  \\ Interior Point \\ \citep{nesterov1994interior}\\\citep{abernethy2015faster} }  & No & NA & None & $poly\left(d, \log(1/\epsilon)\right)$\\
 \hline
  \multirow{2}{*}{Gradient Descent} & \multirow{2}{*}{Yes} & \multirow{2}{*}{k=1} & Bounded $L_2$ & $O\left(\left(\frac{L_2}{\epsilon}\right)^{1/2}\right)$ \\
\hhline{~~~--} \citep{NesterovBook} & & & $\lambda_{\min} > 0$  & $O\left(\left(\frac{L_2}{\lambda_{\min}}\right)^{1/2}  \log(\frac{1}{\epsilon})\right)$  \\
\hline
\multirow{2}{*}{Newton's method } & \multirow{2}{*}{Yes} & \multirow{2}{*}{k=2} & Bounded $L_3$ & $O\left(\left(\frac{L_3}{\epsilon}\right)^{2/7}\right)$ \\
\hhline{~~~--} \citep{monteiro2013accelerated} & & & $\lambda_{\min} > 0$  & $\tilde{O}\left(\left(\frac{L_3}{\lambda_{\min}}\right)^{\frac{2}{7}}+ \log\log(\frac{1}{\epsilon})\right)$  \\ 
 \hline
 \makecell{Higher-Order \\ \citep{baeshigherorder}} & Yes & $k > 2$ & Bounded $L_k$ & $O\left(\left(\frac{L_{k+1}}{\epsilon}\right)^{1/(k+1)}\right)$ \\ 
 \hline
\end{tabular}
\end{center}
\caption{Table summarizing known results for convex optimization. }
\end{table}



Table \ref{table:main} surveys the known algorithms for convex optimization and their iteration complexity.  Polynomial time algorithms admit {\it linear convergence}, i.e. $O(\log \frac{1}{\eps})$, and invariably depend on the dimension. Dimension free polynomial-time algorithms require both an upper bound on the smoothness and a lower bound on the strong convexity.  

The main question we set to answer is {\bf whether there exists linearly-converging iterative algorithms for (high-order) smooth functions that are not strongly convex?}

In this paper we show that unfortunately, these hopes cannot be attained without stronger assumptions on the underlying optimization problem. In particular Theorem \ref{thm:mainthminit} shows that even if the functions are $k^{th}$-order smooth, for arbitrarily large $k$, and the iterative algorithm uses $k^{th}$-order derivative information, the answer is negative. To the best of our knowledge, our results are the first lower bound for $k^{th}$-order optimization for $k \geq 2$ that include higher-order smoothness.\footnote{After the writing of the first manuscript we were made aware of the work by \cite{shamir2017oracle} which provides lower bounds for these settings as well.}

\subsection{Statement of Result}

We consider the problem of $k^{th}$-order optimization. We model a $k^{th}$-order algorithm as follows. Given a $k$-times differentiable function $f: \reals^d \rightarrow \reals$, at every iteration $i$, the algorithms outputs a point $x_i$ and receives as input the tuple $[ f(x_i), \nabla f(x_i), \nabla^2 f(x_i) \ldots \nabla^k f(x_i)]$, i.e. the value of the function and its $k$ derivatives at $x_i$.\footnote{An iteration is equivalent to an oracle call to $k^{th}$-order derivatives in this model.} The goal of the algorithm is to output a point $x_T$ such that 
\[ f(x_T) - \min_{x \in \reals^d} f(x) \leq \epsilon. \]

For the $k^{th}$-order derivatives to be informative, one needs to bound their rate of change, or equivalently the Lipschitz constant of its derivative. This is called $k^{th}$-order smoothness, and we denote it by $L_{k+1}$. In particular we assume that
$$ \| \nabla^k f(x) - \nabla^k f(y) \| \leq L_{k+1} \| x-y \|,$$
where $\|\cdot\|$ is defined as the induced operator norm with respect to the Euclidean norm. Our main theorem shows the limitation of $k^{th}$-order iterative optimization algorithms: 

\begin{theorem}
\label{thm:mainthminit}
For every number $L_{k+1}$ and $k^{th}$-order algorithm $\alg$ (deterministic or randomized), there exists a number $\epsilon_0(L_{k+1})$ such that for all $\epsilon \leq \epsilon_0(L_{k+1})$, there exists a $k$-differentiable convex function $f:B_d \rightarrow \reals$ with $k^{th}$-order smoothness coefficient $L_{k+1}$ such that $\alg$ cannot output a point $x_T$ such that
\[f(x_T) \leq \min_{x \in B_d} f(x) + \epsilon,\]
 in number of iterations $T$ fewer than
\[ c_k\left(\frac{L_{k+1}}{\epsilon}\right)^{\Omega(1/k) },\]
where $c_k$ is a constant depending on $k$ and $B_d$ is defined to be the unit ball in $d$ dimensions.
\end{theorem}

The above lower bound is known to be tight up to constants in the exponent:  for $k > 2$,~\cite{baeshigherorder} proves an upper bound of $O\left( \left(\frac{L_{k+1}}{\epsilon} \right)^{\frac{1}{k+1}}\right)$. %In comparison Theorem \ref{thm:mainthminit} proves a lower bound of $\Omega\left(\left(\frac{L_{k+1}}{\epsilon}\right)^{2/(5k + 1)}\right)$. 


Although the bound is stated for constrained optimization over the unit ball, it can be extended to an unconstrained setting via the addition of an appropriate scaled multiple of $\|x\|^2$. We leave this adaptation for a full version of this paper. Further as is common with lower bounds the underlying dimension $d$ is assumed to be large enough and differs for the deterministic vs randomized version. Theorems~\ref{thm:mainthm} and~\ref{thm:mainthmrandomized} make the dependence precise.

%\paragraph{Comparison to existing bounds. } 
Table \ref{table:main2} surveys the known lower bounds for derivative based optimization. For the case of $k=2$, the most efficient methods known are the cubic regularization technique proposed by~\cite{nesterov2008cubic} and an accelerated hybrid proximal extra-gradient method proposed by~\cite{monteiro2013accelerated}. The best known upper bound in this setting is $O\left(\frac{L_3}{\epsilon}\right)^{2/7}$~\citep{monteiro2013accelerated}. We show a lower bound of $\Omega\left(\left( \frac{L_3}{\epsilon}\right)^{2/11}\right)$ (c.f. Theorem \ref{thm:mainthmprecise}) demonstrating that the upper bound is nearly tight.
\begin{table}
  \begin{center}
  \begin{tabular}{ |c|c|}
  \hline
  Oracle & Lower bound \\
  \hline
  First-Order Oracle & $\Omega\left( \left(\frac{L_2}{\epsilon}\right)^{1/2} \right)$  \citep{Nemirovsky1978}\\
  \hline
  Second-Order Oracle & $\Omega\left( \left(\frac{L_3}{\epsilon}\right)^{2/11} \right)$ (This paper  - c.f. Theorem \ref{thm:mainthmprecise}) \\
  \hline
  $k^{th}$-Order Oracle & $\Omega\left( \left(\frac{L_{k+1}}{\epsilon}\right)^{2/(5k + 1)} \right)$ (This paper - c.f. Theorem \ref{thm:mainthmprecise}) \\
  \hline
  \end{tabular}
  \caption{Lower bounds for higher order oracles}
  \label{table:main2}
  \end{center}

\end{table}


% We further conjecture that our bounds can be improved in the following manner. 
% \begin{conjecture}
% \label{remark:conjecture}
%   We conjecture that the lower bound via the construction in  Thoerem \ref{thm:mainthminit} can be improved to \[O\left(\left(\frac{L_{k+1}}{\epsilon}\right)^{2/(3k+1)}\right)\]
% \end{conjecture}
% Examining the conjecture above we believe that the exponent over $T$ in Equation \eqref{eqn:condlk} (c.f. Theorem \ref{thm:mainthm}) can be improved from $2.5$ to $1.5$ which will imply the above conjecture. Note that the conjecture would make the result tight with respect to known upper bound results for $k=1,2$. However this still leaves a gap in the known upper bound vs the conjectured lower bound for $k > 2$. In this setting we belive that the bound $O\left( \left(\frac{L_{k+1}}{\epsilon_k} \right)^{\frac{1}{k+1}}\right)$ \cite{baeshigherorder} is in fact not tight and can be improved. 

%  Moreover we further show nearly tight lower bounds for general $k$ order optimization which come close to matching the best known upper bounds established by \cite{baeshigherorder}. Table \ref{table:main} summarizes known results for higher-order convex optimization and where our new lower bounds fit. 




% \begin{tikzpicture}
% \begin{axis}
% \addplot+ [fill] {x^2+2} \closedcycle;
% \end{axis}
% \end{tikzpicture}

% \begin{tikzpicture}
% \begin{axis}
% \addplot+ [name path=A,domain=0:5, fill, style=
% {pattern color=gray!50,
% pattern=north east lines}] {x^2} \closedcycle;
% \end{axis}
% \end{tikzpicture}




 


\subsection{Related work.}
The literature on convex optimization is too vast to survey; the reader is referred to \cite{boyd,NesterovBook}. 

Lower bounds for convex optimization were studied extensively in the seminal work of \cite{Nemirovsky1978}. In particular, tight first-order optimization lower bounds were established assuming first-order smoothness.(Also see \cite{NesterovBook} for a concise presentation of the lower bound). In a recent work, \cite{arjevanisecondorder} presented a lower bound when given access to second-order derivatives. However a key component (as remarked by the authors themselves) missing from the bound established by \cite{arjevanisecondorder} was that the constructed function was not third-order smooth. Indeed the lower bound established by \cite{arjevanisecondorder} can  be overcome when the function is third-order smooth (ref. \cite{nesterov2008cubic}). The upper bounds for higher-order oracles (assuming appropriate smoothness) was established by \cite{baeshigherorder}. 

Higher order smoothness has been leveraged recently in the context of non-convex optimization \citep{Agarwal:2017:FAL:3055399.3055464,CarmonAGD,allen2017natasha}. In a surprising new discovery, \cite{carmon2017convex} show that assuming higher-order smoothness the bounds for first-order optimization can be improved without having explicit access to higher-order oracles. This is a property observed in our lower bound too. Indeed as shown in the proof the higher order derivatives at the points queried by the algorithm are always 0. For further details regarding first-order lower bounds for various different settings we refer the reader to \cite{agarwal2009information,woodworth2016tight,arjevanisecondorder,arjevani2015lower} and the references therein. The work of \cite{guzman2015lower} studies a different kind of smoothing namely inf-convolution to obtain first-order smoothness in arbitrary norms, however it does not provide guarantees with higher-order smoothness.

In parallel and independently, Arjevani et al. \cite{shamir2017oracle} also obtain lower bounds for deterministic higher-order optimization.  In comparison, their lower bound is stronger in terms of the exponent than the ones proved in this paper, and matches the upper bound for $k=2$.  However, our construction and proof are simple (based on the well known technique of ball smoothing) and our bounds hold for randomized algorithms as well, as opposed to their deterministic lower bounds. 

\subsection{Overview of Techniques}

Our lower bound is inspired by the lower bound presented in \cite{ClarksonHW2012}. In particular we construct the function as a piece-wise linear convex function defined by $f(x) = \max_i \{ a_i^Tx \}$ with carefully constructed vectors $a_i$ and restricting the domain to be the unit ball. The key idea here is that querying a point reveals information about at most one hyperplane. The optimal point however can be shown to require information about all the hyperplanes.

Unfortunately the above function is not differentiable. We now smooth the function by the ball smoothing operator (defined in Definition \ref{def:smoothingoperator}) which averages the function in a small Euclidean ball around a point. We show (c.f. Corollary \ref{cor:maincor}) that iterative application of the smoothing operator ensures $k$-differentiability as well as boundedness of the $k^{th}$-order derivative.

Two key issues arise due to smoothing described above. Firstly although the smoothing operator leaves the function unchanged around regions far away from the intersection of the hyperplanes, it is not the case for points lying near the intersection. Indeed querying a point near the intersection of the hyperplanes can potentially lead to leak of information about multiple hyperplanes at once. To avoid this, we carefully shift the linear hyperplanes making them affine and then arguing that this shifting indeed forces sufficient gap between the points queried by the algorithm and the intersections leaving sufficient room for smoothing. 

Secondly such a smoothing is well known to introduce a dependence on the dimension $d$ in the smoothness coefficients. Our key insight here is that for the class of functions being considered for the lower bound (c.f. Definition \ref{defn:gammainvariance}) smoothing can be achieved without a dependence on the dimension(c.f. Theorem \ref{thm:smoothingmain}). This is essential to achieving dimension free lower bounds and we believe this characterization can be of intrinsic interest.

\subsection{Organization of the paper}

We begin by providing requisite notation and definitions for the smoothing operator and proving the relevant lemmas regarding smoothing in Section \ref{sec:prelimssmooth}. Section \ref{sec:mainthmstatement} provides a quantitative statement of our main theorem. In Section \ref{sec:construction} we provide the construction of our hard function. In Section \ref{sec:deterministic} we state and prove our main theorem (Theorem \ref{thm:mainthm}) showing the lower bound against deterministic algorithms. We also prove Theorem \ref{thm:mainthminit} based on Theorem \ref{thm:mainthm} in this Section. In Section \ref{sec:randomized} we state and prove Theorem \ref{thm:mainthmrandomized}, showing the lower bound against randomized algorithms.

\section{Preliminaries}
\label{sec:prelimssmooth}
\subsection{Notation}

We use $B_d$ to refer to the $d$-dimensional $\ell_2$ unit ball. We suppress the $d$ from the notation when it is clear from the context. Let $\Gamma$ be an $r$-dimensional linear subspace of $\reals^d$. We denote by $M_{\Gamma}$, an $r \times d$ matrix which contains an orthonormal basis of $\Gamma$ as rows. Let $\Gamma^{\perp}$ denote the orthogonal complement of $\Gamma$. Given a vector $v$ and a subspace $\Gamma$, let $v \perp \Gamma$ denote the perpendicular component of $v$ w.r.t $\Gamma$. We now define the notion of a $\Gamma$-invariant function. 

\begin{definition}[$\Gamma$-invariance]
\label{defn:gammainvariance}
Let $\Gamma$ be an $r$ dimensional linear subspace of $\reals^d$. A function $f:\reals^d \rightarrow \reals$ is said to be $\Gamma$-invariant if for all $x \in \reals^d$ and $y$ belonging to the subspace $\Gamma^{\perp}$, i.e. $M_{\Gamma} y = 0$, we have that
    \[ f(x) = f(x + y)\]
   Equivalently there exists a function $g:\reals^r \rightarrow \reals$ such that for all $x$, $f(x) = g(M_{\Gamma} x)$.
\end{definition}
\noindent A function $f: \reals^d \rightarrow \reals$ is defined to be $c$-Lipschitz with respect to a norm $\|\cdot\|$ if it satisfies
\[ f(x) - f(y) \leq c\|x-y\| \] 
 Lipschitzness for the rest of the paper will be measured in the $\ell_2$ norm. 
\subsection{Smoothing}
In this section we define the smoothing operator and derive the requisite properties.
\begin{definition}[Smoothing operator]
\label{def:smoothingoperator}
Given an $r$-dimensional subspace $\Gamma \in \reals^d$ and a parameter $\delta > 0$, define the operator $S_{\delta,\Gamma}:(\reals^d \rightarrow \reals) \rightarrow (\reals^d \rightarrow \reals)$ (referred henceforth as the smoothing operator) as 
  \[ S_{\delta, \Gamma}f(x) \defeq \mathbb{E}_{v \in \Gamma, \|v\| \leq 1}[ f(x + \delta v)],\]
  where the expectation is over sampling a unit vector from the subspace $\Gamma$ uniformly randomly. 
  

  As a shorthand we define $f_{\delta, \Gamma} \defeq S_{\delta, \Gamma}f$. Further for any $t \in \mathbb{N}$ define $S_{\delta,\Gamma}^t f \defeq S_{\delta,\Gamma}(... S_{\delta,\Gamma}( f )) $ i.e. the smoothing operator applied on $f$ iteratively $t$ times.
\end{definition}
When $\Gamma = \reals^d$ we suppress the notation from $f_{\delta, \Gamma}$ to $f_{\delta}$. Following is the main lemma we prove regarding the smoothing operator. 
\begin{lemma}
  \label{thm:smoothingmain}
  Let $\Gamma $ be an $r$-dimensional linear subspace of $\reals^d$ and $f : \reals^d \rightarrow \reals$ be $\Gamma$-invariant and $G$-Lipschitz. Let $\smooth{f} \defeq S_{\delta, \Gamma} f$ be the smoothing of $f$. Then we have the following properties.
  \begin{enumerate}
    \item $\smooth{f}$ is differentiable and also $G$-Lipschitz and $\Gamma$-invariant.
    \item $\nabla \smooth{f}$ is $\frac{rG}{\delta}$-Lipschitz.
    \item $\forall \;x: |f_{\delta,\Gamma}(x) - f(x)| \leq \delta G$.
  \end{enumerate}
  \end{lemma}

Following is a corollary of the above lemma. 
\begin{corollary}
\label{cor:maincor}
  Given a $G$-Lipschitz continuous function $f$ and an $r$-dimensional subspace $\Gamma$ such that $f$ is $\Gamma$-invariant, we have that the function $S_{\delta,\Gamma}^k f$ is $k$-times differentiable $\forall\;k$. Moreover we have that for any $x,y$
  \[ \forall i \in [k] \;\; \| \nabla^{i} S_{\delta,\Gamma}^k f(x) - \nabla^{i} S_{\delta,\Gamma}^k f(y)\| \leq \left(\frac{r}{\delta} \right)^iG\|x-y\|,\] 
  \[| S_{\delta,\Gamma}^k f(x) - f(x)| \leq G\delta k.\]
\end{corollary}
The proofs of Lemma \ref{thm:smoothingmain} and Corollary \ref{cor:maincor} are included in the appendix.

\section{Main Theorem Statement}
\label{sec:mainthmstatement}
The main result we prove in the paper is given by the following theorem. The theorem is a restatement of Theorem \ref{thm:mainthminit}. 

\begin{theorem}
\label{thm:mainthmprecise}
For every number $L_{k+1}$ and $k^{th}$-order algorithm $\alg$ (deterministic or randomized), there exists an $\epsilon_0(L_{k+1})$ such that for all $\epsilon \leq \epsilon_0(L_{k+1})$, there exists a $k$-differentiable convex function $f \in B_d \rightarrow \reals$ with $k^{th}$-order smoothness coefficient $L_{k+1}$ such that $\alg$ cannot output a point $x_T$ such that
\[f(x_T) \leq \min_{x \in B_d} f(x) + \epsilon,\]
 in number of iterations $T$ fewer than
\[ c_k\left(\frac{L_{k+1}}{\epsilon}\right)^{\frac{2}{5k + 1} }.\]
where $c_k$ is a constant depending on $k$.
\end{theorem}
The rest of the paper is dedicated to the proof of the above theorem. 


\section{Construction of the hard function}

\label{sec:construction}

In this section we describe the construction of our hard function $\hardf$. Our construction is inspired by the information-theoretic hard instance based on zero-sum games proposed by \cite{ClarksonHW2012}. The construction of the function will be characterized by a sequence of vectors $X^{1 \rightarrow r} = \{x_1 \ldots x_r\}$, $x_i \in B_d$ and parameters $k,\gamma, \delta, m$. We assume $d > m \geq r$. To make the dependence explicit we denote the hard function as  \[\hardf(X^{1 \rightarrow r},\gamma,k,\delta, m):B_d \rightarrow \reals.\]

For brevity in the rest of the section we suppress $X^{1 \rightarrow r}, \gamma, k, \delta, m$ from the notation, however all the quantities defined in the section depend on them. To define $\hardf$ we will define auxiliary vectors $\{a_1 \ldots a_r\}$ and auxiliary functions $f,\tilde{f}$. 

Given a sequence of vectors $\{x_1, x_2, \ldots x_r\}, x_i \in B_d$, let $X_i$ for $i \leq r$ be defined as the subspace spanned by the vectors $\{x_1 \ldots x_{i}\}$. Further inductively define vectors $\{a_1 \ldots a_r\}$ as follows.

\noindent If $x_i \notin X_{i-1}$, define  
\[a_i \defeq \frac{\hat{a}_i}{\| \hat{a}_i\|}\text{ where } \hat{a}_i \defeq x_i \perp X_{i-1}.\]
If indeed $x_i \in X_i$, then $a_i$ is defined to be an arbitrary unit vector in the orthogonal component $X_{i-1}^{\perp}$. Further define an auxiliary function 
\[f(x) \defeq \max_{i \in [r]} f_i(x) \text{ where } f_i(x) \defeq a_i^T x.\]
Given the parameter $\gamma$, now define the following functions 
\[ \tilde{f}(x) \defeq \max_{i \in [r]} \tilde{f}_i(x) \text{ where } \tilde{f}_i(x) \defeq f_i(x) + \left(1 - \frac{i}{m} \right)\gamma \defeq a_i^T x + \left(1 - \frac{i}{m} \right)\gamma.\]
With these definitions in place we can now define the hard function parametrized by $k, \delta$. Let $A_r$ be the subspace spanned by $\{a_1 \ldots a_r\}$
\begin{equation}
  \label{eqn:hardfdef}
  \hardf(X^{1 \rightarrow r}, k, \gamma, \delta, m) \defeq S^k_{\delta, A_r} \;\tilde{f}(X^{1 \rightarrow r}, \gamma, m),
\end{equation}
i.e. $\hardf$ is constructed by smoothing $\tilde{f}$ $k$-times with respect to the parameters $\delta, A_r$. We now collect some important observations regarding the function $\hardf$. 

\begin{observation}
  $\hardf$ is convex and continuous. Moreover it is 1-Lipschitz and is invariant with the respect to the $r$-dimensional subspace $A_r$.
\end{observation}

Note that $\tilde{f}$ is a $\max$ of linear functions and hence convex. Since smoothing preserves convexity we have that $\hardf$ is convex. 1-Lipschitzness follows by noting that by definition $\|a_i\| = 1$ and it can be seen that $\tilde{f}$ is $A_r$-invariant and therefore by Theorem \ref{thm:smoothingmain} we get that $\hardf$ is $A_r$-invariant. 


\begin{observation}
  $\hardf$ is $k$-differentiable with the Lipschitz constants $L_{i+1} \leq \left(\frac{r}{\delta}\right)^i$ for all $i \leq k$.
\end{observation}
\noindent Above is a direct consequence of Corollary \ref{cor:maincor} and the fact that $\tilde{f}$ is 1-Lipschitz and invariant with respect to the $r$-dimensional subspace $A_r$. 
Corollary \ref{cor:maincor} also implies that 
\begin{equation}
  \label{eqn:errorbound}
  \forall x \;\;|\hardf(x) - \tf(x)| \leq k\delta. 
\end{equation}
Setting $\hat{x} \defeq -\sum_{i=1}^{r} \frac{a_i}{\sqrt{r}}$, we get that $f(\hat{x}) = \frac{-1}{\sqrt{r}}$. Therefore the following inequality follows from Equation \eqref{eqn:errorbound} and by noting that $\|f(x) - \tf(x)\|_{\infty} \leq \gamma$ 

\begin{equation}
\label{eqn:funcminbound}
\min_{x \in B_d} \hardf(x) \leq \hardf(\hat{x}) \leq \frac{-1}{\sqrt{r}} + \gamma + k\delta
\end{equation}
% Further noting that $f(x) \in [-1,1] \;\;\;\forall x \in B_d$ which implies that $\tf(x) \in [-1, 1+\gamma]$ we get that 
% \begin{equation}
% \label{eqn:funcmaxbound}
% \hardf(x) \in [-1, 1 + \gamma] \;\;\;\;\forall x \in B_d
% \end{equation}
% It is also easy to note that
% \begin{equation}
% \label{eqn:funcmaxlowerbound}
%   \max_{x \in B_d} \hardf(x) \geq 1 + \gamma - k \delta
% \end{equation}
The following lemma provides a characterization of the derivatives of $\hardf$ at the points $x_i$. 

\begin{lemma}
\label{lemma:derivatives}
  Given a sequence of vectors $\{x_1 \ldots x_r\}$ and parameters $\delta, \gamma, r, m$, let $\{g_1 \ldots g_r\}$ be a sequence of functions defined as
    \[ \forall\;i\;\;g_i \defeq \hardf(X^{1 \rightarrow i}, k,\gamma, \delta, m).\] 
  If the parameters are such that $2k\delta \leq \frac{\gamma}{m}$ then we have that 
  \[ \forall\;i\in[r] \;\forall j \in [k]  \;\; g_i(x_i) = g_r(x_i), \; \nabla^j g_i(x_i) = \nabla^j g_r(x_i).\]

\end{lemma}

\begin{proof}
  

We will first note the following about the smoothing operator $S^k_{\delta}$. At any point $x$, all the $k$ derivatives and the function value of $S^k_{\delta} f$ for any function $f$ depend only on the value of the function $f$ in a ball of radius at most $k \delta$ around the point $x$. Consider the function $g_r$ and $g_i$ for any $i \in [r]$. Note that by definition of the functions $g_i$, for any $x$ such that \[
\argmax_{j \in [r]} \;\;a_j^Tx + \left(1 - \frac{j}{m}\right)\gamma \leq i\]
we have that $g_i(x) = g_r(x)$. Therefore to prove the lemma it is sufficient to show that
\[ \forall \;i, x \in \|x - x_i\| \leq k\delta\;\;\;\; \argmax_{j \in [r]} \;\;a_j^Tx + \left(1 - \frac{j}{m}\right)\gamma \leq i.\]
Let us first note the following facts. By construction we have that $\forall j > i, a_j^Tx_i = 0$. This immediately implies that  
\begin{equation}
\max_{j > i} \;\; a_j^Tx_i + \left(1 - \frac{j}{m}\right)\gamma =\left(1 - \frac{i+1}{m} \right) \gamma.
\end{equation}
Further using the fact that $\|a_j\| \leq 1$, $\forall j \in [r]$ we have that 
\begin{equation}
\forall x \;\;\text{s.t.}\;\;\|x - x_i\| \leq k\delta \;\;\text{we have}\;\; \max_{j > i} \;\; a_j^Tx + \left(1 - \frac{j}{m}\right)\gamma \leq\left(1 - \frac{i+1}{m} \right) \gamma + k\delta.
\end{equation}
Further note that by construction $a_i^Tx_i \geq 0$ which implies $a_i^Tx + \left(1 - \frac{i}{m}\right)\gamma \geq \left(1 - \frac{i}{m} \right)\gamma$. Again using the fact that $\|a_j\| \leq 1$, $\forall j \in [r]$ we have that 
\begin{equation}
\label{eqn:templabel1}
\forall x \;\;\text{s.t.}\;\;\|x - x_i\| \leq k\delta \;\;\text{we have}\;\; \max_{j \leq i} \;\; a_j^Tx + \left(1 - \frac{j}{m}\right) \geq \left(1 - \frac{i}{m} \right) \gamma - k\delta.
\end{equation}
The above equations in particular imply that as long as $2k\delta < \frac{\gamma}{m}$ , we have that  
\begin{equation}
\forall x \;\;\text{s.t.}\;\;\|x - x_i\| \leq k\delta \;\; \argmax_{j \in [r]} a_j^Tx + \left(1 - \frac{j}{m}\right) \leq i 
\end{equation}
which as we argued before is sufficient to prove the lemma.
\end{proof}



\section{Main Theorem and Proof}
\label{sec:deterministic}
The following theorem (Theorem \ref{thm:mainthm}) proves the existence of the required hard function. Theorem \ref{thm:mainthmprecise} for the deterministic case is a simple derivation which we provide after the theorem statement.  

\begin{theorem}
\label{thm:mainthm}
  For any integer $k$, any $T > 5k$, and $d > T$ and any $k^{th}$-order deterministic algorithm, there exists a convex function $\hardf: B_d \rightarrow \reals$ for every $d > T$, such that for $T$ steps of the algorithm every point $y \in B_d$ queried by the algorithm is such that 
  \[\hardf(y) \geq \min_{x \in B_d}\hardf(x) + \frac{1}{2\sqrt{T}}.\] 
  Moreover the function is guaranteed to be $k$-differentiable with Lipschitz constants $L_{i+1}$ bounded as 
  \begin{equation}
  \label{eqn:condlk}
    \forall \;i \leq k \;\; L_{i+1} \leq (10k)^iT^{2.5i}.
  \end{equation}

\end{theorem}
We first prove Theorem \ref{thm:mainthmprecise} in the deterministic case using Theorem \ref{thm:mainthm}. 
\begin{proof}[Proof of Theorem \ref{thm:mainthmprecise} Deterministic case]
Given an algorithm $\alg$ and numbers $L_{k+1},k$ define $\epsilon_0(L_{k+1},k) \defeq L_{k+1}/(10k)^k$. For any $\epsilon \leq \epsilon_0$ pick a number $\mathcal{T}$ such that
\[\epsilon = \frac{L_{k+1}}{2(10k)^k\mathcal{T}^{(2.5k + 0.5)}}.\]
Let $\hardf$ be the function constructed in Theorem \ref{thm:mainthm} for parameters $k, \mathcal{T}, \alg$ and define the hard function $h:B_d \rightarrow \reals$
\[ h(x) \defeq \frac{L_{k+1}}{(10k)^k\mathcal{T}^{2.5k}}\hardf(x).\]
Note that by the guarantee in Equation \eqref{eqn:condlk} we get that $h(x)$ is $k^{th}$-order smooth with coefficient at most $L_{k+1}$. Note that since this is a scaling of the original hard function $\hardf$ the lower bound applies directly and therefore $\alg$ cannot achieve accuracy 
\[ \frac{L_{k+1}}{2(10k)^k\mathcal{T}^{2.5k}\sqrt{\mathcal{T}}} \defeq \epsilon,\]
in less than $\mathcal{T} = c_{k}\left(\frac{L_{k+1}}{\epsilon}\right)^{\frac{2}{5k+1}}$ iterations where $c_k$ is a constant only depending on $k$. This finishes the proof of the theorem.  
\end{proof}
We now provide the proof of Theorem \ref{thm:mainthm}.
\begin{proof}[Proof of Theorem \ref{thm:mainthm}]

Define the following parameters $\gamma \defeq \frac{1}{3\sqrt{T}}$ and $\delta_T \defeq \frac{\gamma}{3kT}$.

Consider a deterministic algorithm $\alg$. Since $\alg$ is deterministic let the first point played by the algorithm be fixed to be $x_1$. We now define a series of functions $\hardf_i$ inductively for all $i = \{1, \ldots T\}$ as follows
\begin{equation}
  \label{eqn:Xdefn}
  X^{1 \rightarrow i} \defeq \{x_1 \ldots x_i\} \quad\quad\quad \hardf_i \defeq \hardf(X^{1 \rightarrow i}, \gamma, k, \delta_T, T)
\end{equation}
\begin{equation}
  \label{eqn:inductinpidefn}
  Inp^x_i \defeq \{ \hardf_i(x_i), \nabla \hardf_i(x_i) \ldots \nabla^k \hardf_i(x_i)\} \quad\quad\quad x_{i+1} \defeq \alg(Inp^x_1, \ldots Inp^x_i)
\end{equation}
The above definitions \textit{simulate} the deterministic algorithm $\alg$ with respect to changing functions $\hardf_i$. $Inp_{i}^x$ is the input the algorithm will receive if it queried point $x_i$ and the function was $\hardf_i$. $x_{i+1}$ is the next point the algorithm $\alg$ will query on round $i+1$ given the inputs $\{Inp_1^x \ldots Inp_i^x\}$ over the previous rounds. Note that thus far these quantities are tools defined for analysis. Since $\alg$ is deterministic these quantities are all deterministic and well defined. We will now prove that the function $\hardf_T$ defined in the series above satisfies the properties required by the Theorem \ref{thm:mainthm}. 
\\
\\
% \noindent \textbf{Boundedness} Firstly note that Equation \eqref{eqn:funcmaxbound} and Equation \eqref{eqn:funcmaxlowerbound} immediately gives us that 
% \[\forall i \in [T]\;\;\forall x \in B_d \;\; \hardf_i(x) \in [-1/1 + \gamma - k\delta_T,1] \in [-1,1] \]
\noindent \textbf{Bounded Lipschitz Constants} Using Corollary \ref{cor:maincor}, the fact that $\hardf$ has Lipschitz constant bounded by 1 and that $\hardf_T$ is invariant with respect to a $T$ dimensional subspace, we get that the function $\hardf_T$ has higher order Lipschitz constants bounded above as
\[ \forall i \leq k \;\;\; L_{i+1} \leq \left(\frac{T}{\delta_T}\right)^i \leq \left(10kT^{2.5}\right)^i.\] 
\noindent \textbf{Suboptimality}

Let $\{y_0 \ldots y_T\}$ be the points queried by the algorithm $\alg$ when executed on $\hardf_T$. We need to show that 
\begin{equation}
  \label{eqn:suboptimalitygoal}
  \forall i \in [1 \ldots T] \qquad \hardf_T(y_i) \geq \min_{x \in B_d} \hardf_T(x) + \frac{1}{2\sqrt{T}}.
\end{equation}
Equation \ref{eqn:suboptimalitygoal} follows as a direct consequence of the following two claims. 

\begin{claim}
\label{claim:consistency}
  We have that for all $i \in [1,T]$, $y_i = x_i$ where $x_i$ is defined by Equation \eqref{eqn:inductinpidefn}.
\end{claim}

\begin{claim}
\label{claim:xsuboptimality}
We have that
\[  \forall i \in [1 \ldots T] \qquad \hardf_T(x_i) \geq \min_{x \in B_d} \hardf_T(x) + \frac{1}{2\sqrt{T}}.\]
\end{claim} 

To remind the reader, $x_i$ (Equation \eqref{eqn:inductinpidefn}) are variables which were defined by simulating the algorithm on a changing function where as $y_i$ are the points played by the algorithm $\alg$ when run on $\hardf_T$. Claim \ref{claim:consistency} shows that even though $\hardf_T$ was constructed using $x_i$ the outputs produced by the algorithm does not change. 

Claim \ref{claim:consistency} and Claim \ref{claim:xsuboptimality} derive Equation \ref{eqn:suboptimalitygoal} in a straightforward manner thus finishing the proof of Theorem \ref{thm:mainthm}.

\end{proof}

We now provide the proofs of Claim \ref{claim:consistency} and Claim \ref{claim:xsuboptimality}. 

\begin{proof}[Proof of Claim \ref{claim:consistency}]
Note that since the algorithm is deterministic $y_1$ is fixed and $y_i$ for $i \geq 2$ is defined inductively as follows. 
\begin{equation}
\label{eqn:inductivecaseproof}
  Inp^y_i \defeq \{ \hardf_T(y_i), \nabla \hardf_T(y_i), \ldots \nabla^k \hardf_T(y_i)\} \quad \quad \quad y_{i+1} = \alg(Inp^y_1, \ldots Inp^y_T)
\end{equation}

We will prove the claim via strong induction. The base case $x_1 = y_1$ is immediate because $\alg$ is deterministic and therefore the first point queried by it is always the same. 

Assume now that the claim holds for all $j \leq i$.  
Since by definition $2k\delta_T \leq \gamma/T$, we can see as a direct consequence of Lemma \ref{lemma:derivatives}, that
\begin{equation}
\label{eqn:inputequalities}
  \{\forall j \leq i \;\; x_j = y_j \} \Rightarrow \{ \forall j \leq i \;\; Inp_i^{y_i} = Inp_i^{x_i} \}  
  \end{equation}
where $Inp_i^x$ is as defined in Equation \eqref{eqn:inductinpidefn}. Note that $Inp_i^{x_i}$ is the set of derivatives of $\hardf_i$ at $x_i$ and $Inp_i^{y_i}$ is the set of derivatives of $\hardf_T$ at $y_i$. Also by definition we have that 
\[ \{ \forall j \leq i \;\; Inp_i^{y_i} = Inp_i^{x_i} \} \Rightarrow \{x_{i+1} = y_{i+1}\}.\]
Putting the above two together we have that $\{\forall j \leq i \;\; x_j = y_j \} \Rightarrow \{x_{i+1} = y_{i+1}\}$
which finishes the induction. 
\end{proof}

 
\begin{proof}[Proof of Claim \ref{claim:xsuboptimality}]
  Using Lemma \ref{lemma:derivatives} we have that $\hardf_i(x_i) = \hardf_T(x_i)$. Further Equation \eqref{eqn:templabel1} implies that 
  \[\hardf_i(x_i) \geq \left(1 - \frac{i}{T}\right)\gamma - k\delta_T.\]Now using \eqref{eqn:funcminbound} using we get that every point in $\{x_1 \ldots x_T\}$ is such that 
\[ \hardf_T(x_i) - \min_{x \in B} \hardf_T(x) \geq  \left( \frac{1}{\sqrt{T}} - \frac{i\gamma}{T} - 2k\delta_T \right)\geq \frac{1}{2\sqrt{T}}.\] 
The above follows by the choice of parameters and $T$ being large enough. This finishes the proof of Claim \ref{claim:xsuboptimality}.
\end{proof}

\section{Lower Bounds against Randomized Algorithms}

\label{sec:randomized}



In this section we prove the version of Theorem \ref{thm:mainthm} for randomized algorithms. The key idea underlying the proof remains the same. However since we cannot \textit{simulate} the algorithm anymore we choose the vectors $\{a_i\}$ forming the subspace randomly from $\reals^d$ for a large enough $d$. This ensures that no algorithm with few queries can discover the subspace in which the function is non-invariant with reasonable probability. Naturally the dimension required for Theorem \ref{thm:mainthm} now is larger than the tight $d > T$ we achieved as in the case of deterministic algorithms.

The proof of Theorem \ref{thm:mainthmprecise} for randomized algorithms follows in exactly the same way as the proof for the deterministic case using Theorem \ref{thm:mainthm}. 

\begin{theorem}
\label{thm:mainthmrandomized}
  For any integer $k$, any $T > 5k$, $\delta \in [0,1]$,  and any $k$-order (potentially randomized algorithm), there exists a $k$-differentiable convex function $\hardf: B_d \rightarrow \reals$ for $d = \Omega(T^3\log(T^2/\delta))$, such that with probability at least $1 - \delta$ (over the randomness of the algorithm) for $T$ steps of the algorithm every point $y$ queried by the algorithm is such that 
   \[\hardf(y) \geq \min_{x \in B_d}\hardf(x) + \frac{1}{2\sqrt{T}}.\] 
   Moreover the function $\hardf$ is guaranteed to be $k$-differentiable with Lipschitz constants $L_i$ bounded as 
  \[ \forall \;i \leq k \;\; L_{i+1} \leq (20kT^{2.5})^i.\]
\end{theorem}
\noindent Due to space constraints the proof of Theorem \ref{thm:mainthmrandomized} is included in the appendix.

\section{Conclusion}

We have considered the problem of achieving dimension free polynomial time algorithms for minimizing convex functions where the function is guaranteed to be $k$-differentiable and $k^{th}$-order smooth and the algorithm is allowed to have access to $k$ derivatives at every iteration. We showed an oracle complexity lower bound for convex optimization under these conditions demonstrating that the number of points queried by any deterministic/randomized algorithm should have at least an inverse polynomial dependence on the desired error. This rules out linearly-converging derivative-based algorithms even under these assumptions. 

While we provide precise guarantees for the dependence on the exponent, our bounds are weaker than those proved independently and concurrently by \cite{shamir2017oracle} (which only applies to deterministic algorithms). We believe that our construction (or potentially a similar one) might be able to achieve the improved bounds and leave this direction as immediate future work. Furthermore we remark that while the known upper and lower bounds are tight for first and second order optimization, they are not tight for $k > 2$ and this is an intriguing open question. 







% Acknowledgments---Will not appear in anonymized version
\acks{The authors would like to acknowledge and thank Ohad Shamir for providing insightful comments on the first draft of this manuscript and Brian Bullins and Gopi Sivakanth for helpful suggestions. The authors are supported by NSF grant 1523815.}


\bibliography{references}
\pagebreak


\appendix
\appendix

\section{Proof of Lemma \ref{thm:general_instantaneous}}
\begin{proof}{\textbf{of Lemma \ref{thm:general_instantaneous}}.}
We first state a useful property used in typical OMD analysis. Let $\Omega$ be a convex compact set in $\mathbb{R}^K$, $\psi$ be a convex function on $\Omega$, 
$w'$ be an arbitrary point in $\Omega$, and $x \in \mathbb{R}^K$.
If $w^*=\argmin_{w\in \Omega}\{\inn{w,x}+D_{\psi}(w,w')\}$, then for any $u \in \Omega$,
\begin{align*}
\inn{w^*-u, x}\leq D_{\psi}(u,w')-D_\psi(u,w^*)-D_{\psi}(w^*,w'). 
\end{align*}
This is by the first-order optimality condition of $w^*$ and direct calculations. Applying this to update rule~\eqref{eqn:update_rule_2} we have
\begin{align}
\inn{w_{t+1}^\p-u, \hat{\ell}_t+ a_t} \leq D_{\psi_t}(u,w_{t}^\p)-D_{\psi_t}(u,w_{t+1}^\p)-D_{\psi_t}(w_{t+1}^\p, w_{t}^\p); \label{eqn:apply1}
\end{align}
while applying it to update rule~\eqref{eqn:update_rule_1} and picking $u=w_{t+1}^\p$ we have
\begin{align}
\inn{w_t-w_{t+1}^\p, m_t} \leq D_{\psi_t}(w_{t+1}^\p, w_t^\p)-D_{\psi_t}(w_{t+1}^\p, w_t)-D_{\psi_t}(w_t, w_t^\p).\label{eqn:apply2} 
\end{align}
Now we bound the instantaneous regret as follows:
\begin{align}
&\inn{w_t-u, \hat{\ell}_t}\nonumber \\
&=\inn{w_t-u, \hat{\ell}_t+ a_t}-\inn{w_t, a_t}+\inn{u,  a_t}\nonumber \\
&=\inn{w_t-w_{t+1}^\p, \hat{\ell}_t+a_t}-\inn{w_t, a_t}+\inn{w_{t+1}^\p-u, \hat{\ell}_t+a_t}+\inn{u,   a_t}\nonumber \\
&=\inn{w_t-w_{t+1}^\p, \hat{\ell}_t+a_t-m_t}-\inn{w_t,a_t}+\inn{w_{t+1}^\p-u, \hat{\ell}_t+ a_t}+\inn{w_t-w_{t+1}^\p, m_t}+\inn{u,   a_t} \nonumber \\
&\leq D_{\psi_t}(u,w_{t}^\p)-D_{\psi_t}(u,w_{t+1}^\p)-D_{\psi_t}(w_{t+1}^\p, w_t)-D_{\psi_t}(w_t, w_t^\p)+\inn{u, a_t}, \label{eqn:regret_decomposition}
\end{align}
where last inequality is by the condition $\inn{w_t-w_{t+1}^\p, \hat{\ell}_t+a_t-m_t}-\inn{w_t,a_t}\leq 0$, Eq.~\eqref{eqn:apply1}, and Eq.~\eqref{eqn:apply2}.
\end{proof}

\section{Lemmas for Log-barrier OMD}
\label{section:all_kinds_of_lemmas}

In this section we establish some useful lemmas for update rules~\eqref{eqn:update_rule_1} and~\eqref{eqn:update_rule_2} with log-barrier regularizer,
which are used in the proofs of other theorems.
We start with some definitions.

\begin{definition}
\label{definition:norm}
For any $h \in \mathbb{R}^K$, define norm $\norm{h}_{t,w}=\sqrt{h^\top \nabla^2 \psi_t(w) h}=\sqrt{\sum_{i=1}^K \frac{1}{\eta_{t,i}}\frac{h_i^2}{w_i^2}}$ and its dual norm $\norm{h}_{t,w}^*=\sqrt{h^\top \nabla^{-2} \psi_t(w) h}=\sqrt{\sum_{i=1}^K \eta_{t,i}w_i^2 h_i^2}$.
For some radius $r > 0$, define ellipsoid $\mathcal{E}_{t,w}(r)=\left\{u \in \mathbb{R}^K : \norm{u-w}_{t,w}\leq r \right\}$ . 
\end{definition}

\begin{lemma}
\label{lemma:norm_close}
If $w^\p \in \mathcal{E}_{t,w}(1)$ and $\eta_{t,i}\leq \frac{1}{81}$ for all $i$, then $w_i^\p\in \left[ \frac{1}{2}w_i, \frac{3}{2}w_i \right]$ for all $i$, and also $ 0.9\norm{h}_{t,w} \leq \norm{h}_{t,w^\p} \leq 1.2\norm{h}_{t,w}$ for any $h\in \mathbb{R}^K$. 
\end{lemma}
\begin{proof}
$w^\p\in \mathcal{E}_{t,w}(1)$ implies $\sum_{i=1}^K \frac{1}{\eta_{t,i}}\frac{(w^\p_i-w_i)^2}{w_i^2}\leq 1$. Thus for every $i$, we have $\frac{\abs{w_i^\p-w_i}}{w_i}\leq \sqrt{\eta_{t,i}}\leq \frac{1}{9}$, implying $w_i^\p\in \left[ \frac{8}{9}w_i, \frac{10}{9}w_i \right]\subset\left[ \frac{1}{2}w_i, \frac{3}{2}w_i \right]$. 
Therefore, $\norm{h}_{t,w^\p}
=\sqrt{\sum_{i=1}^K \frac{1}{\eta_{t,i}} \frac{h_i^2}{w^{\p 2}_i}}
\geq \sqrt{\sum_{i=1}^K \frac{1}{\eta_{t,i}}\frac{h_i^2}{\left(\frac{10}{9}w_i\right)^2}}
=0.9\norm{h}_{t,w}$. 
Similarly, we have $\norm{h}_{t,w^\p}\leq 1.2\norm{h}_{t,w}$. 
\end{proof}

\begin{lemma}
\label{lemma:stability}
Let $w_t, w_{t+1}^\p$ follow \eqref{eqn:update_rule_1} and \eqref{eqn:update_rule_2} where $\psi_t$ is the log-barrier with $\eta_{t,i}\leq \frac{1}{81}$ for all $i$. If $\norm{\hat{\ell}_t-m_t+a_t}^*_{t,w_t}\leq \frac{1}{3}$, then $w_{t+1}^\p \in \mathcal{E}_{t,w_t}(1)$. 
\end{lemma}

\begin{proof}
Define $F_{t}(w)=\inn{w, m_t}+D_{\psi_t}(w, w_t^\p)$ and $F_{t+1}^\p(w)=\inn{w, \hat{\ell}_t+a_t}+D_{\psi_t}(w, w_t^\p)$. Then by definition we have $w_t=\argmin_{w\in\Omega}F_{t}(w)$ and $w_{t+1}^\p=\argmin_{w\in\Omega}F_{t+1}^\p(w)$. To show $w_{t+1}^\p\in \mathcal{E}_{t,w_t}(1)$, it suffices to show that for all $u$ on the boundary of $\mathcal{E}_{t,w_t}(1)$, $F^\p_{t+1}(u)\geq F^\p_{t+1}(w_t)$. 

Indeed, using Taylor's theorem, for any $u\in \partial \mathcal{E}_{t,w_t}(1)$, there is an $\xi$ on the line segment between $w_t$ and $u$ such that (let $h\triangleq u-w_t$)
\begin{align*}
F^\p_{t+1}(u)&=F^\p_{t+1}(w_t)+\nabla F^{\p}_{t+1} (w_t)^\top h+ \frac{1}{2}h^\top\nabla^2 F^\p_{t+1}(\xi)h \\
&=F^\p_{t+1}(w_t)+ (\hat{\ell}_t-m_t+a_t)^\top h +\nabla F_t (w_t)^\top h+ \frac{1}{2}h^\top\nabla^2 \psi_t(\xi)h \\
&\geq F^\p_{t+1}(w_t)+ (\hat{\ell}_t-m_t+a_t)^\top h + \frac{1}{2}\norm{h}_{t,\xi}^2 \tag{by the optimality of $w_t$}\\
&\geq F^\p_{t+1}(w_t)+ (\hat{\ell}_t-m_t+a_t)^\top h + \frac{1}{2}\times0.9^2\norm{h}_{t,w_t}^2 \tag{by Lemma \ref{lemma:norm_close}} \\
&\geq F^\p_{t+1}(w_t)- \norm{\hat{\ell}_t-m_t+a_t}^*_{t,w_t} \norm{h}_{t,w_t} + \frac{1}{3}\norm{h}_{t,w_t}^2 \\
&=F^\p_{t+1}(w_t)- \norm{\hat{\ell}_t-m_t+a_t}^*_{t,w_t} + \frac{1}{3} \tag{$\norm{h}_{t,w_t}=1$}\\
&\geq F^\p_{t+1}(w_t). \tag{by the assumption}
\end{align*}
\end{proof}

\begin{lemma}
\label{lemma:stability_under_condition}
Let $w_t, w_{t+1}^\p$ follow \eqref{eqn:update_rule_1} and \eqref{eqn:update_rule_2} where $\psi_t$ is the log-barrier with $\eta_{t,i}\leq \frac{1}{81}$ for all $i$. If $\norm{\hat{\ell}_t-m_t+a_t}^*_{t,w_t}\leq \frac{1}{3}$, then $\norm{w_{t+1}^\p-w_t}_{t,w_t}\leq 3\norm{\hat{\ell}_t-m_t+a_t}_{t,w_t}^*$. 
\end{lemma}
\begin{proof}
Define $F_t(w)$ and $F_{t+1}^\p(w)$ to be the same as in Lemma \ref{lemma:stability}. Then we have 
\begin{align}
F_{t+1}^\p(w_t)-F_{t+1}^\p(w_{t+1}^\p)&=(w_t-w_{t+1}^\p)^\top(\hat{\ell}_t-m_t+a_t) + F_t(w_t)-F_t(w_{t+1}^\p) \nonumber \\
&\leq (w_t-w_{t+1}^\p)^\top(\hat{\ell}_t-m_t+a_t) \nonumber \tag{optimality of $w_t$}\\
&\leq \norm{w_t-w_{t+1}^\p}_{t,w_t}\norm{\hat{\ell}_t-m_t+a_t}_{t,w_t}^*. \label{eqn:direction1}
\end{align}
On the other hand, for some $\xi$ on the line segment between $w_t$ and $w_{t+1}^\p$, we have by Taylor's theorem and the optimality of $w_{t+1}^\p$,
\begin{align}
F_{t+1}^\p(w_t)-F_{t+1}^\p(w_{t+1}^\p)&=\nabla F_{t+1}^\p(w_{t+1}^\p)^\top (w_t-w_{t+1}^\p) + \frac{1}{2}(w_t-w_{t+1}^\p)^\top \nabla^2 F_{t+1}^\p(\xi)(w_t-w_{t+1}^\p) \nonumber \\
&\geq \frac{1}{2}\norm{w_t-w_{t+1}^\p}_{t,\xi}^2 .
\label{eqn:direction2}
\end{align}
Since the condition in Lemma \ref{lemma:stability} holds, $w_{t+1}^\p\in \mathcal{E}_{t,w_t}(1)$, and thus $\xi\in \mathcal{E}_{t,w_t}(1)$. Using again Lemma \ref{lemma:norm_close}, we have 
\begin{align}
\frac{1}{2}\norm{w_t-w_{t+1}^\p}_{t,\xi}^2 \geq \frac{1}{3}\norm{w_t-w_{t+1}^\p}_{t,w_t}^2\label{eqn:direction3}.
\end{align}
Combining \eqref{eqn:direction1}, \eqref{eqn:direction2}, and \eqref{eqn:direction3}, we have $\norm{w_t-w_{t+1}^\p}_{t,w_t}\norm{\hat{\ell}_t-m_t+a_t}_{t,w_t}^* \geq \frac{1}{3}\norm{w_t-w_{t+1}^\p}_{t,w_t}^2$, which leads to the stated inequality. 
\end{proof}

\begin{lemma}
\label{lemma:condition_automatic_hold}
%Let $w_t, w_{t+1}^\p$ follow \eqref{eqn:update_rule_1} and \eqref{eqn:update_rule_2}. 
When the three conditions in Theorem \ref{lemma:MAB_condition} hold, we have $\norm{\hat{\ell}_t-m_t+a_t}^{*}_{t,w_t}\leq \frac{1}{3}$ for either $a_{t,i}=6\eta_{t,i}w_{t,i}(\hat{\ell}_{t,i}-m_{t,i})^2$ or $a_{t,i}=0$.  
\end{lemma}
\begin{proof}
For $a_{t,i}=6\eta_{t,i}w_{t,i}(\hat{\ell}_{t,i}-m_{t,i})^2$, we have
\begin{align*}
\norm{\hat{\ell}_t-m_t+a_t}^{*2}_{t,w_t}
&= \sum_{i=1}^K\eta_{t,i}w_{t,i}^2\big(\hat{\ell}_{t,i}-m_{t,i} + 6\eta_{t,i}w_{t,i}(\hat{\ell}_{t,i}-m_{t,i})^2\big)^2 \\
&=\sum_{i=1}^K\eta_{t,i}w_{t,i}^2(\hat{\ell}_{t,i}-m_{t,i})^2+12\eta_{t,i}^2w_{t,i}^3(\hat{\ell}_{t,i}-m_{t,i})^3 +36\eta_{t,i}^3w_{t,i}^4(\hat{\ell}_{t,i}-m_{t,i})^4\\
&\leq \sum_{i=1}^K \eta_{t,i}w_{t,i}^2(\hat{\ell}_{t,i}-m_{t,i})^2(1+36\eta_{t,i}+324\eta_{t,i}^2) \tag{condition (ii)}\\
&\leq 2\sum_{i=1}^K \eta_{t,i}w_{t,i}^2(\hat{\ell}_{t,i}-m_{t,i})^2 \tag{condition (i)}\\
&\leq 2\times \frac{1}{18}=\frac{1}{9}.\tag{condition (iii)}
\end{align*}
For $a_{t,i}=0$, we have
\begin{align*}
\norm{\hat{\ell}_t-m_t+a_t}^{*2}_{t,w_t}=\norm{\hat{\ell}_t-m_t}^{*2}_{t,w_t}=\sum_{i=1}^K \eta_{t,i}w_{t,i}^2(\hat{\ell}_{t,i}-m_{t,i})^2\leq \frac{1}{18} < \frac{1}{9}. \tag{condition (iii)}
\end{align*}
%When $a_{t,i}=0$,
%\begin{align*}
%\norm{\hat{\ell}_t-m_t+a_t}_{t,w_t}^{*2}= \sum_{i=1}^K \eta_{t,i}w_{t,i}^2(\hat{\ell}_{t,i}-m_{t,i})^2 \leq \frac{1}{18}<\frac{1}{9}. \text{\ \ \ \ \ \ \ \ \ \ \ \ \ \ \ \ \ \ \ \ \ \ \ (condition(iii))} 
%\end{align*}
\end{proof}

\begin{lemma}
\label{lemma:2times_bound}
If the three conditions in Theorem \ref{lemma:MAB_condition} hold, \textsc{Broad-OMD} (with either Option I or II)
satisfies $\frac{1}{2}w_{t,i}\leq w^\p_{t+1,i}\leq \frac{3}{2}w_{t,i}$.
%Let $w_t, w_{t+1}^\p$ follow \eqref{eqn:update_rule_1} and \eqref{eqn:update_rule_2}. If the three conditions in Theorem \ref{lemma:MAB_condition} hold, then $\frac{1}{2}w_{t,i}\leq w^\p_{t+1,i}\leq \frac{3}{2}w_{t,i}$, for either $a_{t,i}=6\eta_{t,i}w_{t,i}(\hat{\ell}_{t,i}-m_{t,i})^2$ or $a_{t,i}=0$.
\end{lemma}
\begin{proof}
This is a direct application of Lemmas \ref{lemma:condition_automatic_hold},  \ref{lemma:stability}, and \ref{lemma:norm_close}.
%It suffices to prove $w_{t+1}^\p \in \mathcal{E}_{t,w_t}(1)$, because if it is true, then
%\begin{align*}
%\norm{w_{t+1}^\p-w_t}_{t,w_t}=\sqrt{\sum_{i=1}^K\frac{1}{\eta_{t,i}}\frac{(w^\p_{t+1,i}-w_{t,i})^2}{w_{t,i}^2}}\leq 1, 
%\end{align*}
%which implies $\frac{\abs{w_{t+1,i}^\p-w_{t,i}}}{w_{t,i}}\leq \sqrt{\eta_{t,i}}\leq \frac{1}{2}$. Thus, $w_{t+1,i}^\p \in [\frac{1}{2}w_{t,i}, \frac{3}{2}w_{t,i}]$.

%Since we assume the three conditions in Theorem \ref{lemma:MAB_condition} hold, $w_{t+1}^\p \in \mathcal{E}_{t,w_t}(1)$ can be proved by applying Lemma \ref{lemma:condition_automatic_hold} and Lemma \ref{lemma:stability} back to back.

\end{proof}

\begin{lemma}
\label{lemma:2times_bound_another}
For the MAB problem, if the three conditions in Theorem \ref{lemma:MAB_condition} hold, \textsc{Broad-OMD} (with either Option I or II)
satisfies $\frac{1}{2}w_{t,i}\leq w^\p_{t,i}\leq \frac{3}{2}w_{t,i}$.
\end{lemma}
\begin{proof}
It suffices to prove $w_{t}^\p \in \mathcal{E}_{t,w_t}(1)$ by Lemma~\ref{lemma:norm_close}.
Since we assume that the three conditions in Theorem \ref{lemma:MAB_condition} hold and $w_t\in \Delta_K$, we have $\norm{m_t}_{t,w_t}^*=\sqrt{\sum_{i=1}^K \eta_{t,i}w_{t,i}^2m_{t,i}^2}\leq \sqrt{\frac{1}{162}\sum_{i=1}^K w_{t,i}^2}\leq \sqrt{\frac{1}{162}}< \frac{1}{3}$. This implies $w_{t}^\p \in \mathcal{E}_{t,w_t}(1)$ by a similar arguments as in the proof of Lemma~\ref{lemma:stability} (one only needs to replace $F_{t+1}^\p(w)$ there by $G(w)\triangleq D_{\psi_t}(w,w_t^\p)$ and note that $w_t^\p=\argmin_{w\in \Delta_K}G(w)$).
\end{proof}


%\begin{lemma}
%\label{lemma:stability_game}
%For MAB problems, if the three conditions in Theorem \ref{lemma:MAB_condition} hold, then \textsc{Broad-OMD} with $a_{t,i}=\mathbf{0}$ and fixed learning rate $\eta$ guarantees $\norm{w_{t+1}-w_t}_1 = \mathcal{O}(\eta)$ for all $t$.  
%\end{lemma}
%\begin{proof}
%By Lemma \ref{lemma:condition_automatic_hold} and \ref{lemma:stability_under_condition}, we have $\norm{w_t-w_{t+1}^\p}_{t,w_t}\leq 3\norm{\hat{\ell}_t-m_t}_{t,w_t}^*$, which implies
%\begin{align*}
%\frac{1}{\eta}\frac{(w_{t,j}-w_{t+1,j}^\p)^2}{w_{t,j}^2}\leq \sum_{i=1}^K\frac{1}{\eta}\frac{(w_{t,i}-w_{t+1,i}^\p)^2}{w_{t,i}^2}\leq 3\eta\sum_{i=1}^K  w_{t,i}^2(\hat{\ell}_{t,i}-m_{t,i})^2\leq 3\eta\times 9. 
%\end{align*}
%Therefore, $\abs{w_{t,j}-w_{t+1,j}^\p} = \mathcal{O}(\eta w_{t,j})$. We can use similar techniques in Lemma \ref{lemma:condition_automatic_hold} and \ref{lemma:stability_under_condition} to prove $\norm{w_{t+1}-w_{t+1}^\p}_{t,w_{t+1}^\p}\leq 3\norm{m_{t+1}}_{t,w_{t+1}^\p}^*$, and thus  $\abs{w_{t+1,j}^\p-w_{t+1,j}}=\mathcal{O}(\eta w_{t+1,j}^\p)=\mathcal{O}(\eta w_{t,j})$. Thus, $\abs{w_{t,j}-w_{t+1,j}}=\mathcal{O}(\eta w_{t,j})$, which implies $\norm{w_t-w_{t+1}}_1=\mathcal{O}(\eta)$. 
%\end{proof}

\section{Proof of Theorem \ref{lemma:MAB_condition} and Corollary~\ref{cor:clear_corollary}}
%To prove Theorem \ref{lemma:MAB_condition}, we need some definitions and lemmas established in Section~\ref{section:all_kinds_of_lemmas}. 

\begin{proof}{\textbf{of Theorem \ref{lemma:MAB_condition}}.}
We first prove Eq.~\eqref{eqn:condition1} holds: by Lemmas \ref{lemma:condition_automatic_hold} %, we have $\norm{\hat{\ell}_t-m_t+a_t}^{*}_{t,w_t}\leq \frac{1}{3}$. Then by
and \ref{lemma:stability_under_condition}, we have
\begin{align*}
\inn{w_t-w_{t+1}^\p, \hat{\ell}_t-m_t+ a_t}
&\leq \norm{w_t-w_{t+1}^\p}_{t,w_t}\norm{\hat{\ell}_t-m_t+a_t}_{t,w_t}^*\\
&\leq 3\norm{\hat{\ell}_t-m_t+a_t}_{t,w_t}^{*2}\\
&\leq 3\sum_{i=1}^K \eta_{t,i}w_{t,i}^2(\hat{\ell}_{t,i}-m_{t,i})^2(1+36\eta_{t,i}+324\eta_{t,i}^2) \\
&\leq 6\sum_{i=1}^K \eta_{t,i}w_{t,i}^2(\hat{\ell}_{t,i}-m_{t,i})^2 = \inn{w_t, a_t},
\end{align*}
where the last two inequalities are by the same calculations done in the proof of Lemma~\ref{lemma:condition_automatic_hold}.
%where $C=6$. With our choice of $C$ and $\eta_{t,i}$, we have $3(1+6\eta_{t,i}C+9\eta_{t,i}^2C^2)\leq 3\left(1+6\times\frac{6}{162}+9\times \left(\frac{6}{162}\right)^2\right)\leq C$. Therefore, the last expression is further bounded by $\sum_{i=1}^K C\eta_{t,i}w_{t,i}^2(\hat{\ell}_{t,i}-m_{t,i})^2$, which is equal to $\inn{w_t, a_t}$. 

Since Eq.~\eqref{eqn:condition1} holds, using Lemma~\ref{thm:general_instantaneous} we have (ignoring non-positive terms $-A_t$'s),
\begin{align}
\sum_{t=1}^T\inn{w_t-u, \hat{\ell}_t}&\leq \sum_{t=1}^T\left(D_{\psi_t}(u,w_t^\p)-D_{\psi_t}(u,w^\p_{t+1})\right)+\sum_{t=1}^T\inn{u,a_t}\nonumber \\
&\leq D_{\psi_1}(u, w_1^\p) + \sum_{t=1}^{T}\left( D_{\psi_{t+1}}(u, w^\p_{t+1})-D_{\psi_{t}}(u, w^\p_{t+1}) \right)+\sum_{t=1}^T\inn{u,a_t}.\label{eqn:some_intermediate}
\end{align}
In the last inequality, we add a term $D_{\psi_{T+1}}(u, w_{T+1}^\p) \geq 0$ artificially. As mentioned, $\psi_{T+1}$, defined in terms of $\eta_{T+1,i}$, never appears in the \textsc{Broad-OMD} algorithm. We can simply pick any $\eta_{T+1,i} > 0$ for all $i$ here. This is just to simplify some analysis later. 

The first term in \eqref{eqn:some_intermediate} can be bounded by the optimality of $w_1^\p$:
\begin{align*}
D_{\psi_1}(u, w_1^\p)&=\psi_1(u)-\psi_1(w_1^\p)-\inn{\nabla\psi_1(w_1^\p), u-w_1^\p}\\
&\leq \psi_1(u)-\psi_1(w_1^\p)=\sum_{i=1}^K \frac{1}{\eta_{1,i}}\ln\frac{w_{1,i}^\p}{u_i}.
\end{align*}
%where the inequality is because $w_1^\p$ is the minimizer of $\psi_1$. 
The second term, by definition, is
\begin{align*}
\sum_{t=1}^{T}\sum_{i=1}^K \left(\frac{1}{\eta_{t+1,i}}-\frac{1}{\eta_{t,i}}\right) h\left(\frac{u_i}{w_{t+1,i}^\p}\right). 
\end{align*}
Plugging the above two terms into \eqref{eqn:some_intermediate} finishes the proof.
\end{proof}

\begin{proof}{\textbf{of Corollary~\ref{cor:clear_corollary}}.}
We first check the three conditions in Theorem~\ref{lemma:MAB_condition} under our choice of $\eta_{t,i}$ and $\hat{\ell}_{t,i}$: $\eta_{t,i}=\eta=\frac{1}{162K_0}\leq \frac{1}{162}$; $w_{t,i}\abs{\hat{\ell}_{t,i}-m_{t,i}}=\abs{\ell_{t,i}-m_{t,i}}\mathbbm{1}\{i\in b_t\} \leq 2<3$; 
$\sum_{i=1}^K \eta_{t,i}w_{t,i}^2(\hat{\ell}_{t,i}-m_{t,i})^2=\frac{1}{162K_0}\sum_{i=1}^K (\ell_{t,i}-m_{t,i})^2\mathbbm{1}\{i\in b_t\} \leq \frac{4}{162} < \frac{1}{18}$. 
%(in the MAB case, $\sum_{i=1}^K \eta_{t,i}w_{t,i}^2(\hat{\ell}_{t,i}-m_{t,i})^2=\eta_{t,i_t}(\ell_{t,i_t}-m_{t,i_t})^2\leq \frac{1}{162}\times 3^2=\frac{1}{18}$). 
Applying Theorem~\ref{lemma:MAB_condition} we then have 
\begin{align*}
\sum_{t=1}^T \inn{w_t-u, \hat{\ell}_t}\leq \sum_{i=1}^K  \frac{\ln\frac{w^\p_{1,i}}{u_i}}{\eta}  +\sum_{t=1}^T \inn{u,a_t}.
\end{align*}
As mentioned, if we let $u=b^*$, then $\ln \frac{w_{1,i}^\p}{u_i}$
becomes infinity for those $i\notin b^*$. Instead, we let $u=\left(1-\frac{1}{T}\right)b^* + \frac{1}{T}w_1^\p$. With this choice of $u$, we have $\frac{w_{1,i}^\p}{u_i}\leq \frac{w_{1,i}^\p}{\frac{1}{T}w_{1,i}^\p}=T$. Plugging $u$ into the above inequality and rearranging, we get 
\begin{align}
\sum_{t=1}^T \inn{w_t-b^*, \hat{\ell}_t}\leq \frac{K\ln T}{\eta}+\sum_{t=1}^T \inn{b^*,a_t}+B,  \label{eqn:sb_corollary}
\end{align}
where $B\triangleq \frac{1}{T}\sum_{t=1}^T \inn{-b^*+w_1^\p, \hat{\ell}_t+a_t}$. 

Now note that $\mathbb{E}_{b_t}[a_{t,i}]=6\eta (\ell_{t,i}-m_{t,i})^2=\mathcal{O}(\eta)$ and $\mathbb{E}_{b_t}[\hat{\ell}_{t,i}]=\ell_{t,i}=\mathcal{O}(1)$ for all $i$. Thus, $\mathbb{E}[B]=\mathbb{E}\left[\frac{1}{T}\sum_{t=1}^T \inn{-b^*+w_1^\p, \mathbb{E}_{b_t}[\hat{\ell}_t+a_t]}\right] \leq \mathbb{E}\left[\frac{1}{T}\sum_{t=1}^T \norm{-b^*+w_1^\p}_1 \norm{\mathbb{E}_{b_t}[\hat{\ell}_t+a_t]}_\infty\right] = \mathcal{O}(K_0)$. Taking expectation on both sides of \eqref{eqn:sb_corollary}, we have 
\begin{align*}
\mathbb{E}\left[\sum_{t=1}^T b_t^\top \ell_t  - \sum_{t=1}^T b^{*\top} \ell_t \right] \leq \frac{K\ln T}{\eta} + 6\eta\mathbb{E}\left[\sum_{t=1}^T \sum_{i\in b^*}^K (\ell_{t,i}-m_{t,i})^2\right] + \mathcal{O}(K_0). 
\end{align*}
%One can verify that in the MAB case, the last term can be $\mathcal{O}(1)$ because now $b^*, w_1^\p \in \Delta_K$.  
\end{proof}

\section{Proof of Theorem \ref{cor:variance_bound}}
\begin{proof}{\textbf{of Theorem \ref{cor:variance_bound}}.}
As in \cite{hazan2011better}, 
for the rounds we perform uniform sampling we do not update $w_t^\p$. 
Let $\mathcal{S}$ be the set of rounds of uniform sampling. %Then in all other rounds the learner is essentially running an untouched \textsc{Broad-OMD}. Therefore, we can use Corollary \ref{cor:clear_corollary} to bound the regret. By Corollary \ref{cor:clear_corollary}, 
Then for the other rounds we can apply Corollary \ref{cor:clear_corollary} to arrive at
\begin{align}
\mathbb{E}\left[\sum_{t\in [T]\backslash \mathcal{S}} \ell_{t,i_t}-\ell_{t,i^*} \right]\leq \frac{K\ln T}{\eta} + 6\eta \mathbb{E}\left[\sum_{t\in [T]\backslash \mathcal{S}}(\ell_{t,i^*}-\tilde{\mu}_{t-1,i^*})^2\right] + \mathcal{O}(1). \label{eqn:regret_bound:a_t_neq_0_another} 
\end{align}
The second term can be bounded as follows: 
\begin{align}
&\mathbb{E}\left[\sum_{t\in [T]\backslash \mathcal{S}} (\ell_{t,i^*}-\tilde{\mu}_{t-1,i^*})^2\right]\leq \mathbb{E}\left[\sum_{t=2}^T (\ell_{t,i^*}-\tilde{\mu}_{t-1,i^*})^2\right] \nonumber \\
&\leq 3\sum_{t=2}^T(\ell_{t,i^*}-\mu_{t,i^*})^2+3\sum_{t=2}^T(\mu_{t,i^*}-\mu_{t-1,i^*})^2 + 3\mathbb{E}\left[\sum_{t=2}^T(\mu_{t-1,i^*}-\tilde{\mu}_{t-1,i^*})^2\right].\label{eqn:decompose_three}
\end{align}
The first and the third terms in \eqref{eqn:decompose_three} can be bounded using Lemma 10 and 11 of \citep{hazan2011better} respectively, and they are both of order $\mathcal{O}(Q_{T,i^*}+1)$ if we pick $M=\Theta(\ln T)$. The second term in \eqref{eqn:decompose_three} can be bounded by a constant by Lemma \ref{lemma:second_Q_term}. Thus second term in \eqref{eqn:regret_bound:a_t_neq_0_another}  can be bounded by $\mathcal{O}\left(\eta (Q_{T,i^*}+1)\right)$. Finally, note that $\mathbb{E}\left[\sum_{t=1}^T \ell_{t,i_t}-\ell_{t,i^*} \right]\leq\mathbb{E}\left[\sum_{t\in [T]\backslash \mathcal{S}} \ell_{t,i_t}-\ell_{t,i^*} \right]+2\mathbb{E}[\abs{\mathcal{S}}]$ and that $\mathbb{E}[\abs{\mathcal{S}}]=\mathcal{O}\left(\sum_{t=1}^T \frac{MK}{t}\right)=\mathcal{O}\left(MK\ln T\right)=\mathcal{O}\left(K(\ln T)^2\right)$. Combining everything, we get 
\begin{align*}
\mathbb{E}\left[\sum_{t=1}^T \ell_{t,i_t}-\ell_{t,i^*} \right]=\mathcal{O}\left( \frac{K\ln T}{\eta} + \eta Q_{T,i^*} + K(\ln T)^2\right).
\end{align*}
\end{proof}

\begin{lemma}
\label{lemma:second_Q_term}
For any $i$, $\sum_{t=2}^T (\mu_{t,i}-\mu_{t-1,i})^2=\mathcal{O}(1)$. 
\end{lemma}
\begin{proof}
By definition, \sloppy$\absolute{\mu_{t,i}-\mu_{t-1,i}}=\absolute{\frac{1}{t}\sum_{s=1}^t \ell_{s,i}-\frac{1}{t-1}\sum_{s=1}^{t-1} \ell_{s,i}}=\absolute{\frac{1}{t}\ell_{t,i}-\frac{1}{t(t-1)}\sum_{s=1}^{t-1}\ell_{s,i}}\leq \absolute{\frac{1}{t}\ell_{t,i}}+\absolute{\frac{1}{t(t-1)}\sum_{s=1}^{t-1}\ell_{s,i}}\leq \frac{2}{t}$. Therefore, $\sum_{t=2}^T (\mu_{t,i}-\mu_{t-1,i})^2\leq \sum_{t=2}^T \frac{4}{t^2}=\mathcal{O}(1)$. 
\end{proof}

\section{Proof of Theorem \ref{thm:path_length}}
We first state a useful lemma.
\begin{lemma}
\label{lemma:bound_ni}
Let $n_i$ be such that $\eta_{T+1,i}=\kappa^{n_i}\eta_{1,i}$, i.e., the number of times the learning rate of arm $i$ changes in \textsc{Broad-OMD+}. Then $n_i\leq \log_2 T$, and $\eta_{t,i}\leq 5\eta_{1,i}$ for all $t,i$.  
\end{lemma}
\begin{proof}
Let $t_1, t_2, \ldots, t_{n_i}\in [T]$ be the rounds the learning rate for arm $i$ changes (i.e., $\eta_{t+1,i}=\kappa \eta_{t,i}$ for $t=t_1, \ldots, t_{n_i}$). 
By the algorithm, we have 
\begin{align*}
KT\geq \frac{1}{\bar{w}_{t_{n_i},i}}>\rho_{t_{n_i},i}>2\rho_{t_{n_i-1},i}>\cdots>2^{n_i-1}\rho_{t_1,i}=2^{n_i}K. 
\end{align*}
Therefore, $n_i\leq \log_2 T$. And we have $\eta_{t,i}\leq \kappa^{\log_2 T}\eta_{1,i}=e^{\frac{\log_2 T}{\ln T}}\eta_{1,i}\leq 5\eta_{1,i}$.
\end{proof}




\begin{proof}{\textbf{of Theorem \ref{thm:path_length}}.}
Again, we verify the three conditions stated in Theorem \ref{lemma:MAB_condition}. By Lemma \ref{lemma:bound_ni}, $\eta_{t,i}\leq 5\eta\leq 5\times\frac{1}{810}=\frac{1}{162}$; also, $w_{t,j}\absolute{\hat{\ell}_{t,j}-m_{t,j}}=w_{t,j}\absolute{\frac{(\ell_{t,j}-m_{t,j})\mathbbm{1}\{i_t=j\}}{\bar{w}_{t,j}}}\leq w_{t,j}\absolute{\frac{2}{w_{t,j}\left(1-\frac{1}{T}\right)}}\leq 3$ because we assume $T\geq 3$; finally, $
\sum_{j=1}^K \eta_{t,j}w_{t,j}^2(\hat{\ell}_{t,j}-m_{t,j})^2=\eta_{t,i_t}w_{t,i_t}^2(\hat{\ell}_{t,i_t}-m_{t,i_t})^2\leq \frac{1}{162}\times 3^2=\frac{1}{18}$.

Let $\tau_j$ denote the last round the learning rate for arm $j$ is updated, that is, $\tau_j\triangleq \max\{t\in [T]: \eta_{t+1,j}=\kappa\eta_{t,j} \}$. 
We assume that the learning rate is updated at least once so that $\tau_j$ is well defined, otherwise one can verify that the bound is trivial.
For any arm $i$ to compete with, let 
$u=\left(1-\frac{1}{T}\right)\mathbf{e}_{i}+\frac{1}{T}w_1^\p
=\left(1-\frac{1}{T}\right)\mathbf{e}_{i}+\frac{1}{KT}\mathbf{1}$, which guarantees $\frac{w_{1,i}^\p}{u_i}\leq T$. Applying Theorem \ref{lemma:MAB_condition}, with $B\triangleq \frac{1}{T}\sum_{t=1}^T \inn{-\mathbf{e}_{i}+w^\p_{1}, \hat{\ell}_t+a_t}$ we have
\begin{align}
\sum_{t=1}^T\inn{w_t, \hat{\ell}_t}-\hat{\ell}_{t,i}&\leq \frac{K\ln T}{\eta} + \sum_{t=1}^{T}\sum_{j=1}^K\left(\frac{1}{\eta_{t+1,j}}-\frac{1}{\eta_{t,j}}\right)h\left(\frac{u_{j}}{w_{t+1,j}^\p}\right)+\sum_{t=1}^T a_{t,i}+B\nonumber \\
&\leq \frac{K\ln T}{\eta} + \left(\frac{1}{\eta_{\tau_i+1,i}}-\frac{1}{\eta_{\tau_i,i}}\right)h\left(\frac{u_{i}}{w_{\tau_i+1,i}^\p}\right)+\sum_{t=1}^T a_{t,i}+B\nonumber \\
&\leq \frac{K\ln T}{\eta} + \frac{1-\kappa}{\eta_{\tau_i+1,i}}h\left(\frac{u_{i}}{w_{\tau_i+1,i}^\p}\right)+\sum_{t=1}^T a_{t,i}+B\nonumber \\
&\leq \frac{K\ln T}{\eta} - \frac{1}{5\eta \ln T}h\left(\frac{u_{i}}{w_{\tau_i+1,i}^\p}\right)+\sum_{t=1}^T a_{t,i}+B,  \label{eqn:quasi_regret_bound1}
\end{align}
where the last inequality is by Lemma~\ref{lemma:bound_ni} and the fact $\kappa-1 \geq \frac{1}{\ln T}$. Now we bound the second and the third term in \eqref{eqn:quasi_regret_bound1} separately. 
\begin{enumerate}
\item For the second term,  by Lemma \ref{lemma:2times_bound} and $T \geq 3$ we have
\begin{align*}
\frac{u_{i}}{w^\p_{\tau_i+1,i}} \geq \frac{1-\frac{1}{T}}{ \frac{3}{2}w_{\tau_i, i} }\geq \frac{\left(1-\frac{1}{T}\right)^2}{\frac{3}{2}\bar{w}_{\tau_i,i}} =\frac{\left(1-\frac{1}{T}\right)^2}{\frac{3}{2}}\times \frac{\rho_{T+1,i}}{2}\geq \frac{\rho_{T+1,i}}{8} \geq \frac{4K}{8} \geq 1.
\end{align*}
Noting that $h(y)$ is an increasing function when $y\geq 1$, we thus have
\begin{align}
h\left(\frac{u_{i}}{w^\p_{\tau_i+1,i}}\right)\geq h\left(\frac{\rho_{T+1,i}}{8}\right)
=\frac{\rho_{T+1,i}}{8}-1-\ln\left(\frac{\rho_{T+1,i}}{8}\right)\geq \frac{\rho_{T+1,i}}{8}-1-\ln\left(\frac{KT}{4}\right). \label{eqn:path_length_second_term}
\end{align}

\item For the third term, we proceed as
\begin{align}
\sum_{t=1}^T a_{t,i} &= 6\sum_{t=1}^T \eta_{t,i}w_{t,i}(\hat{\ell}_{t,i}-m_{t,i})^2\leq 90\eta \sum_{t=1}^T \abs{\hat{\ell}_{t,i}-m_{t,i}}  \nonumber \\
&\leq 90\eta\left(\max_{t\in[T]}\frac{1}{\bar{w}_{t,i}}\right) \sum_{t=1}^{T}  \abs{\ell_{t,i}-\ell_{t-1,i}} \leq 90\eta\rho_{T+1,i} V_{T,i}, \label{eqn:path_length_third_term}
\end{align}
where in the first inequality, we use $w_{t,i}\abs{\hat{\ell}_{t,i}-m_{t,i}}\leq 3$ and $\eta_{t,i}\leq 5\eta$; in the second inequality, we do a similar calculation as in Eq.~\eqref{eqn:path_length_trick} (only replacing $w_{t,i}$ by $\bar{w}_{t,i}$); and in the last inequality, we use the fact $\frac{1}{\bar{w}_{t,i}}\leq \rho_{T+1,i}$ for all $t\in [T]$
by the algorithm.
\end{enumerate}
Combining Eq.~\eqref{eqn:path_length_second_term} and Eq.~\eqref{eqn:path_length_third_term} and using the fact $\frac{1+\ln\left(\frac{KT}{4}\right)}{5\ln T}\leq K\ln T$, we continue from Eq.~\eqref{eqn:quasi_regret_bound1} to arrive at
\begin{align}
\sum_{t=1}^T \inn{w_t, \hat{\ell}_t}-\hat{\ell}_{t,i}\leq \frac{2K\ln T}{\eta}+ \rho_{T+1,i}\left( \frac{-1}{40\eta\ln T} +90\eta V_{T,i} \right) +B,  \label{eqn:quasi_regret_bound2}
\end{align}
We are almost done here, but note that the left-hand side of \eqref{eqn:quasi_regret_bound2} is not the desired regret. What we would like to bound is
\begin{align}
\sum_{t=1}^T \inn{\bar{w}_t, \hat{\ell}_t} - \sum_{t=1}^T \hat{\ell}_{t,i}=\sum_{t=1}^T \inn{\bar{w}_t-w_t, \hat{\ell}_t}+ \sum_{t=1}^T\left(\inn{w_t, \hat{\ell}_t}-\hat{\ell}_{t,i}\right), \label{eqn:quasi_regret_bound3}
\end{align}
where the second summation on the right-hand side is bounded by Eq.~\eqref{eqn:quasi_regret_bound2}.
The first term can be written as $\sum_{t=1}^T \inn{-\frac{1}{T}w_t+\frac{1}{KT}\mathbf{1}, \hat{\ell}_t}$. Note that$
\frac{1}{T}\sum_{t=1}^T\inn{-w_t, \hat{\ell}_t}\leq \frac{1}{T}\sum_{t=1}^T\abs{\inn{w_t,\hat{\ell}_t-m_t}}+\frac{1}{T}\sum_{t=1}^T\abs{\inn{w_t, m_t}} \leq 3 + 1=4$, and
$
\mathbb{E}\left[\frac{1}{T}\sum_{t=1}^T \inn{\frac{1}{K}\mathbf{1},\hat{\ell}_{t}}\right]=\frac{1}{T}\sum_{t=1}^T \inn{\frac{1}{K}\mathbf{1},\ell_t}\leq 1. $
Therefore, taking expectation on both sides of \eqref{eqn:quasi_regret_bound3}, we get 
\begin{align*}
\mathbb{E}\left[\sum_{t=1}^T \ell_{t,i_t} \right] - \sum_{t=1}^T \ell_{t,i} \leq \frac{2K\ln T}{\eta} + \mathbb{E}[\rho_{T+1,i}]\left( \frac{-1}{40\eta\ln T} +90\eta V_{T,i} \right) + \mathcal{O}(1),   
\end{align*}
because $\mathbb{E}[B]$ is also $\mathcal{O}(1)$ as proved in Corollary~\ref{cor:clear_corollary}. 
\end{proof}

\section{Proofs of Lemma \ref{lemma:simple_lemma} and Theorem \ref{lemma:second_order_regret_bound}}

\begin{proof}{\textbf{of Lemma \ref{lemma:simple_lemma}}.}
By the same arguments as in the proof of Lemma~\ref{thm:general_instantaneous}, we have
\begin{align*}
\inn{w_{t+1}^\p-u, \hat{\ell}_t} \leq D_{\psi_t}(u,w_{t}^\p)-D_{\psi_t}(u,w_{t+1}^\p)-D_{\psi_t}(w_{t+1}^\p, w_{t}^\p); 
\end{align*}
and 
\begin{align*}
\inn{w_t-w_{t+1}^\p, m_t} \leq D_{\psi_t}(w_{t+1}^\p, w_t^\p)-D_{\psi_t}(w_{t+1}^\p, w_t)-D_{\psi_t}(w_t, w_t^\p).
\end{align*}
Therefore, by expanding the instantaneous regret, we have
\begin{align*}
&\inn{w_t-u, \hat{\ell}_t}\nonumber \\
&=\inn{w_t-w_{t+1}^\p, \hat{\ell}_t-m_t}+\inn{w_{t+1}^\p-u, \hat{\ell}_t}+\inn{w_t-w_{t+1}^\p, m_t} \nonumber \\
&\leq \inn{w_t-w_{t+1}^\p, \hat{\ell}_t-m_t} + D_{\psi_t}(u,w_{t}^\p)-D_{\psi_t}(u,w_{t+1}^\p)-D_{\psi_t}(w_{t+1}^\p, w_t)-D_{\psi_t}(w_t, w_t^\p). 
\end{align*}
\end{proof}
\begin{proof}{\textbf{of Theorem \ref{lemma:second_order_regret_bound}}.}
Applying Lemma \ref{lemma:simple_lemma}, we have 
\begin{align*}
\sum_{t=1}^T\inn{w_t-u, \hat{\ell}_t} &\leq \sum_{t=1}^T \left(D_{\psi_t}(u,w_{t}^\p)-D_{\psi_t}(u,w_{t+1}^\p)+\inn{w_t-w_{t+1}^\p, \hat{\ell}_t-m_t}-A_t\right) \\
&\leq \sum_{i=1}^K \frac{\ln \frac{w_{1,i}^\p}{u_i}}{\eta} +\sum_{t=1}^T \inn{w_t-w_{t+1}^\p, \hat{\ell}_t-m_t}-A_t .
\end{align*}
%The proof of the inequality
%\begin{align*}
%\sum_{t=1}^T D_{\psi_t}(u,w_{t}^\p)-D_{\psi_t}(u,w_{t+1}^\p) \leq \sum_{i=1}^K \left( \frac{\ln \frac{w_{1,i}^\p}{u_i}}{\eta_{1,i}} + \sum_{t=1}^T \left(\frac{1}{\eta_{t+1,i}}-\frac{1}{\eta_{t,i}}\right)h\left(\frac{u_i}{w_{t+1,i}^\p}\right) \right)
%\end{align*}
%is the same as in Lemma \ref{lemma:MAB_condition}. By the non-decreasing learning rate assumption and the fact that $h(\cdot)$ is positive, we can discard $\sum_{t=1}^T \left(\frac{1}{\eta_{t+1,i}}-\frac{1}{\eta_{t,i}}\right)h\left(\frac{u_i}{w_{t+1,i}^\p}\right)$. For the other term, we have 
For the second term, using Lemma \ref{lemma:condition_automatic_hold} and \ref{lemma:stability_under_condition} we bound $\inn{w_t-w_{t+1}^\p, \hat{\ell}_t-m_t}$ by
\begin{align*}
\norm{w_t-w_{t+1}^\p}_{t,w_t}\norm{\hat{\ell}_t-m_t}_{t,w_t}^* 
\leq 3\norm{\hat{\ell}_t-m_t}_{t,w_t}^{*2} = 3\eta\sum_{i=1}^K w_{t,i}^2(\hat{\ell}_{t,i}-m_{t,i})^2
%&= 3\sum_{i=1}^K \eta_{t,i}(\ell_{t,i}-m_{t,i})^2\mathbbm{1}\{i\in b_t\}=3\sum_{i=1}^K \eta_{t,i}b_{t,i}(\ell_{t,i}-m_{t,i})^2
\end{align*}

Finally we lower bound $A_t$ for the MAB case. Note $h(y)=y-1-\ln y\geq \frac{(y-1)^2}{6}$ for $y\in [\frac{1}{2},2]$. By Lemma~\ref{lemma:2times_bound} and \ref{lemma:2times_bound_another}, $\frac{w_{t+1,i}^\p}{w_{t,i}}$ and $\frac{w_{t,i}}{w_{t,i}^\p}$ both belong to $[\frac{1}{2},2]$. Therefore, 
\begin{align*}
A_t&=D_{\psi_t}(w_{t+1}^\p, w_t)+D_{\psi_t}(w_t, w_t^\p)=\frac{1}{\eta} \sum_{i=1}^K \left(h\left(\frac{w_{t+1,i}^\p}{w_{t,i}}\right) +h\left(\frac{w_{t,i}}{w_{t,i}^\p}\right)\right) \\
&\geq \frac{1}{6\eta} \sum_{i=1}^K \left( \frac{(w_{t+1,i}^\p-w_{t,i})^2}{w_{t,i}^2} + \frac{(w_{t,i}-w_{t,i}^\p)^2}{w_{t,i}^{\p 2}} \right) \\
&\geq \frac{1}{24\eta} \sum_{i=1}^K \left( \frac{(w_{t+1,i}^\p-w_{t,i})^2}{w_{t,i}^2} + \frac{(w_{t,i}-w_{t,i}^\p)^2}{w_{t-1,i}^2} \right), 
\end{align*}
and 
\begin{align*}
\sum_{t=1}^T A_t &\geq \frac{1}{24\eta}\sum_{t=2}^{T}\sum_{i=1}^K\frac{(w_{t,i}^\p-w_{t-1,i})^2}{w_{t-1,i}^2}+\sum_{t=2}^T \sum_{i=1}^K \frac{(w_{t,i}-w_{t,i}^\p)^2}{w_{t-1,i}^2}\geq \frac{1}{48\eta}\sum_{t=2}^T \sum_{i=1}^K \frac{(w_{t,i}-w_{t-1,i})^2}{w_{t-1,i}^2}. 
\end{align*}
\end{proof}

\section{Doubling Trick}
\label{app:doubling_trick}

\begin{algorithm}[t]
\DontPrintSemicolon
\caption{Doubling trick for \textsc{Broad-OMD} with $a_t=\mathbf{0}$}
\label{alg:doubling}
\textbf{Initialize}: $\eta=\frac{1}{162K_0}, T_0=0, t=1.$\\
\For{$\beta=0, 1, \ldots$}{
   $w_{t}^\p=\argmin_{w\in \Omega}\psi_1(w)$ (restart \textsc{Broad-OMD}). \\
   \While{$t\leq T$}{
      Update $w_t$, sample $b_t\sim w_t$, and update $w_{t+1}^\p$ as in \textsc{Broad-OMD} with Option II. \\
      \If{$\sum_{s=T_\beta+1}^{t} \sum_{i=1}^K w_{s,i}^2(\hat{\ell}_{s,i}-m_{s,i})^2 \geq \frac{K\ln T}{3\eta^2}$}{
          $\eta \leftarrow \eta/2$, $T_{\beta+1} \leftarrow t$, $t\leftarrow t+1$. \\
          \textbf{break}.
      }
      $t\leftarrow t+1$. 
   }
}
\end{algorithm}

We include the version of our algorithm with the doubling trick in Algorithm~\ref{alg:doubling}.
For simplicity we still assume the time horizon $T$ is known; the extension to unknown horizon is straightforward. 

\begin{proof}{\textbf{of Theorem \ref{thm:doubling_trick_theorem}}.}
Let $u=\left(1-\frac{1}{T}\right)b^*+\frac{1}{T}w_1^\p$ so that $\ln \frac{w^\p_{1,i}}{u_i} \leq \ln T$.
At some epoch $\beta$, by Theorem~\ref{lemma:second_order_regret_bound}, the break condition, and condition (iii) we have with $\eta_\beta\triangleq\frac{2^{-\beta}}{162K_0}$,
\begin{align*}
\sum_{t= T_\beta+1}^{T_{\beta+1}}\inn{w_t-u, \hat{\ell}_t} &\leq \frac{K\ln T}{\eta_\beta} + 
3\eta_\beta\sum_{t=T_\beta+1}^{T_{\beta+1}}\sum_{i=1}^K w_{t,i}^2(\hat{\ell}_{t,i}-m_{t,i})^2 \\
& \leq  \frac{2K\ln T}{\eta_\beta} + 3\eta_\beta \sum_{i=1}^K w_{T_{\beta+1},i}^2(\hat{\ell}_{T_{\beta+1},i}-m_{T_{\beta+1},i})^2 
= \mathcal{O}\left(\frac{K\ln T}{\eta_\beta}\right).
\end{align*}

Suppose that at time $T$, the algorithm is at epoch $\beta=\beta^*$. Then we have
\[
\sum_{t=1}^T\inn{w_t-u, \hat{\ell}_t}\leq\sum_{\beta=0}^{\beta^*}\mathcal{O}\left( \frac{K\ln T}{\eta_\beta} \right)\leq \sum_{\beta=0}^{\beta^*}\mathcal{O}\left( 2^\beta K_0 K\ln T \right)\leq\mathcal{O}\left(2^{\beta^*}K_0 K\ln T\right).
\]
It remains to bound $\beta^*$.
If $\beta^*=0$ (no restart ever happened), then trivially $\sum_{t= 1}^{T}\inn{w_t-u, \hat{\ell}_t}=\mathcal{O}(K_0 K\ln T)$. 
Otherwise, because epoch $\beta^*-1$ finishes, we have 
\begin{align*}
\sum_{t= T_{\beta^*-1}+1}^{T_{\beta^*}}\sum_{i=1}^K w_{t,i}^2(\hat{\ell}_{t,i}-m_{t,i})^2 \geq \frac{K\ln T}{3(\eta_{\beta^*-1})^2} = \Omega(2^{2\beta^*}K_0^2K\ln T).  
\end{align*}
Combining them, we have 
\begin{align}
\sum_{t=1}^T\inn{w_t-u, \hat{\ell}_t} &\leq \mathcal{O}\left(2^{\beta^*}K_0K\ln T\right)
\leq \mathcal{O}\left( \sqrt{(K\ln T)\sum_{t= T_{\beta^*-1}+1}^{T_{\beta^*}}\sum_{i=1}^K w_{t,i}^2(\hat{\ell}_{t,i}-m_{t,i})^2} \right) \nonumber \\
&\leq \mathcal{O}\left( \sqrt{(K\ln T)\sum_{t= 1}^{T}\sum_{i=1}^K w_{t,i}^2(\hat{\ell}_{t,i}-m_{t,i})^2} \right), \label{eqn:doubling_bound_1}
\end{align}
Combining both cases we have 
\begin{align}
\sum_{t=1}^T\inn{w_t-u, \hat{\ell}_t} \leq \mathcal{O}\left( \sqrt{K\ln T\sum_{t= 1}^{T}\sum_{i=1}^K w_{t,i}^2(\hat{\ell}_{t,i}-m_{t,i})^2} + K_0K\ln T\right).
\end{align}
Now substituting $u$ by its definition and taking expectations, with $B\triangleq \frac{1}{T} \sum_{t=1}^T \inn{-b^*+w_1^\p, \hat{\ell}_{t}}$ we arrive at 
\begin{align*}
\mathbb{E}\left[ \sum_{t=1}^T \inn{b_t-b^*, \ell_t} \right]
&\leq \mathcal{O}\left(\mathbb{E}\left[\sqrt{K\ln T\sum_{t= 1}^{T}\sum_{i=1}^K w_{t,i}^2(\hat{\ell}_{t,i}-m_{t,i})^2}\right]+K_0K\ln T \right)+\mathbb{E}[B] \\
&\leq \mathcal{O}\left(  \sqrt{K\ln T\mathbb{E}\left[\sum_{t= 1}^{T}\sum_{i=1}^K w_{t,i}^2(\hat{\ell}_{t,i}-m_{t,i})^2\right]}+K_0K\ln T \right), 
\end{align*}
where the last inequality uses the fact $\mathbb{E}[B]=\mathcal{O}(K)$ and Jensen's inequality.
\end{proof}


\section{Proofs of Corollary \ref{cor:path_length_bound_1} and Theorem \ref{thm:fast_convergence_theorem}}

\begin{proof}{\textbf{of Corollary~\ref{cor:path_length_bound_1}}.}
We first verify the three conditions in Theorem~\ref{lemma:second_order_regret_bound}: $\eta \leq \frac{1}{162}$ by assumption; $w_{t,i}\absolute{\hat{\ell}_{t,i}-m_{t,i}}=\absolute{(\ell_{t,i}-\ell_{\alpha_i(t),i})\mathbbm{1}\{i_t=i\}}\leq 2<3$; $\eta \sum_{i=1}^K w_{t,i}^2(\hat{\ell}_{t,i}-m_{t,i})^2=\eta w_{t,i_t}^2(\hat{\ell}_{t,i_t}-m_{t,i_t})^2\leq \frac{9}{162}=\frac{1}{18}$. Let $u=\left(1-\frac{1}{T}\right)\mathbf{e}_{i^*} + \frac{1}{T}w_1^\p$, which guarantees $\frac{w_{1,i}^\p}{u_i}\leq T$. By Theorem~\ref{lemma:second_order_regret_bound} and some rearrangement, we have
\begin{align*}
\sum_{t=1}^T \inn{w_t-\mathbf{e}_{i^*}, \hat{\ell}_t}\leq \frac{K\ln T}{\eta}  +3\eta\sum_{t=1}^T\sum_{i=1}^K  w_{t,i}^2(\hat{\ell}_{t,i}-m_{t,i})^2-\sum_{t=1}^T A_t+B, 
\end{align*}
where $B\triangleq \frac{1}{T}\sum_{t=1}^T \inn{-\mathbf{e}_{i^*}+w_1^\p, \hat{\ell}_t}$. To get the stated bound, just note that $\mathbb{E}[B]=\mathcal{O}(1)$,  and replace $\sum_{t=1}^T\sum_{i=1}^K w_{t,i}^2(\hat{\ell}_{t,i}-m_{t,i})^2$ by the upper bound at \eqref{eqn:path_length_calculation_1} and $A_t$ by the lower bound in Theorem~\ref{lemma:second_order_regret_bound}.
\end{proof}

\section{Omitted Details in Section~\ref{subsection:games}}
\label{appendix:game}
Although the generalization to multi-player games is straightforward, for simplicity we only consider two-player zero-sum games.

We first describe the protocol of the game. The game is defined by an unknown matrix $G\in[-1,1]^{M\times N}$
where entry $G(i,j)$ specifies the loss (or reward) for Player 1 (or Player 2) if Player 1 picks row $i$ while Player 2 picks column $j$.
The players play the game repeatedly for $T$ rounds.
At round $t$, Player 1 randomly picks a row $i_t \sim x_t$ for some $x_t \in \Delta_M$
while Player 2 randomly picks a column $j_t \sim y_t$ for some $y_t \in \Delta_N$.
In~\citep{syrgkanis2015fast}, the feedbacks they receive are the vectors $Gy_t$ and $x_t^\top G$ respectively.
As a natural extension to the bandit setting, we consider a setting where the feedbacks are the scalar values $\mathbf{e}_{i_t}^\top Gy_t$
and $x_t^\top G\mathbf{e}_{j_t}$ respectively, that is, the expected loss/reward for the players' own realized actions (over the opponent's randomness). 

It is clear that each player is essentially facing an MAB problem and thus can employ an MAB algorithm.
Specifically, if both players apply Exp3 for example, their expected average strategies converge to a Nash equilibrium at rate $1/\sqrt{T}$.
However, if instead Player 1 applies \textsc{Broad-OMD} configured as in Corollary~\ref{cor:path_length_bound_1},
then her regret has a path-length term that can be bounded as follows:
\begin{align*}
\sum_{i=1}^K \sum_{t=2}^T\left| \mathbf{e}_{i}^\top Gy_t -  \mathbf{e}_{i}^\top Gy_{t-1}\right|
\leq \sum_{i=1}^K \sum_{t=2}^T\left\| \mathbf{e}_{i}^\top G \right\|_\infty \|y_t - y_{t-1}\|_1 \leq K \sum_{t=2}^T \|y_t - y_{t-1}\|_1,
\end{align*}
which is closely related to the negative regret term in Corollary~\ref{cor:path_length_bound_1}
for Player 2 if she also employs the same \textsc{Broad-OMD}.
The cancellation of these terms then lead to faster convergence rate.

\begin{theorem}
\label{thm:fast_convergence_theorem}
For the setting described above, if both players run \textsc{Broad-OMD} configured as in Corollary~\ref{cor:path_length_bound_1} except that $\eta_{t,i}=\eta= (M+N)^{-\frac{1}{4}}T^{-\frac{1}{4}}$, then their expected average strategies converge to Nash equilibriums at the rate of $\tilde{\mathcal{O}}\left((M+N)^{\frac{5}{4}}/T^{\frac{3}{4}}\right)$, that is,
\begin{align*}
\max_{y\in \Delta_N} \mathbb{E}[\bar{x}]^\top Gy \leq \text{\rm Val} + \tilde{\mathcal{O}}((M+N)^{\frac{5}{4}}/T^{\frac{3}{4}}) \quad\text{and}\quad
\min_{x\in \Delta_M}x^\top G\mathbb{E}[\bar{y}] \geq \text{\rm Val} - \tilde{\mathcal{O}}((M+N)^{\frac{5}{4}}/T^{\frac{3}{4}}),
\end{align*}
where $\bar{x}=\frac{1}{T}\sum_{t=1}^T x_t, \bar{y}=\frac{1}{T}\sum_{t=1}^T y_t$ and 
$\text{\rm Val}= \min\limits_{x\in \Delta_M}\max\limits_{y\in \Delta_N} x^\top Gy = \max\limits_{y\in \Delta_N}\min\limits_{x\in \Delta_M} x^\top Gy$.
\end{theorem}

\begin{proof}
As mentioned, Player 1's $V_{T,i}$ is 
\begin{align*}
\sum_{t=1}^T \abs{\ell_{t,i}-\ell_{t-1,i}}=&\sum_{t=1}^T \abs{\mathbf{e}_i^\top Gy_t-\mathbf{e}_i^\top Gy_{t-1}}\leq \sum_{t=1}^T \norm{\mathbf{e}_i^\top G}_\infty \norm{y_t-y_{t-1}}_1\leq \sum_{t=1}^T \norm{y_t-y_{t-1}}_1
\end{align*}
due to the assumption $|G(i,j)|\leq 1$. Therefore, by Corollary \ref{cor:path_length_bound_1}, Player 1's (pseudo) regret is
\begin{align*}
&\max_{x \in \Delta_M} \mathbbm{E}\left[\sum_{t=1}^T x_t^\top G y_t - \sum_{t=1}^T x^\top G y_t \right] \\
&\leq \mathcal{O}\left(\frac{M\ln T}{\eta}\right) + \mathbb{E}\left[6\eta M \sum_{t=1}^T \norm{y_t-y_{t-1}}_1 -\frac{1}{48\eta}\sum_{t=2}^T \sum_{i=1}^M \frac{(x_{t,i}-x_{t-1,i})^2}{x_{t-1,i}^2} \right],%\label{eqn:game_regret_1}
\end{align*}
while Player 2's (pseudo) regret is
\begin{align*}
&\max_{y \in \Delta_N} \mathbbm{E}\left[\sum_{t=1}^T x_T^\top G y - \sum_{t=1}^T x_t^\top G y_t \right] \\
&\leq \mathcal{O}\left(\frac{N\ln T}{\eta}\right) + \mathbb{E}\left[6\eta N \sum_{t=1}^T \norm{x_t-x_{t-1}}_1 -\frac{1}{48\eta}\sum_{t=2}^T \sum_{i=1}^N \frac{(y_{t,i}-y_{t-1,i})^2}{y_{t-1,i}^2} \right].%\label{eqn:game_regret_2}
\end{align*}
%Picking $\eta=\Theta\left( T^{-\frac{1}{3}}(\ln T)^{\frac{1}{3}} \right)$, we can bound the right-hand side of \eqref{eqn:game_regret_1} by $\mathcal{O}(M(\ln T)^{\frac{2}{3}}T^{-\frac{2}{3}})=\tilde{\mathcal{O}}(MT^{-\frac{2}{3}})$. With a similar analysis, we have 
Summing up the above two bounds, and using the following fact (by the inequality $a - b \leq \frac{a^2}{4b}$):
\begin{align*}
\sum_{i=1}^N \left(6\eta M \abs{y_{t,i}-y_{t-1,i}}- \frac{(y_{t,i}-y_{t-1,i})^2}{48\eta y_{t-1,i}^2}\right)\leq 432\eta^3 M^2 \sum_{i=1}^N y_{t-1,i}^2 \leq 432\eta^3 M^2, 
\end{align*}
we get 
\begin{align*}
\max_{y \in \Delta_N}\mathbb{E}[\bar{x}]^\top Gy - \min_{x \in \Delta_M} x^\top G \mathbb{E}[\bar{y}] 
 =\mathcal{O}\left(\frac{(M+N)\ln T}{T\eta} + \eta^3 (M^2+N^2) \right).
\end{align*}
With $\eta=\tilde{\Theta}\left( (M+N)^{-\frac{1}{4}}T^{-\frac{1}{4}}\right)$ the above bound becomes $\tilde{\mathcal{O}}\left((M+N)^{\frac{5}{4}}T^{-\frac{3}{4}}\right)$.
Rearranging then gives
%Note that $\mathbb{E}\left[ \frac{1}{T}\left(\text{Reg}_T^1+\text{Reg}_T^2\right) \right]=\mathbb{E}\left[ \max_{y}\bar{x}^\top Gy -\min_{x}x^\top G\bar{y} \right]$. Using 
%By von Neumann’s minimax theorem, we then have
%\begin{align*}
%\max_{y\in \Delta_N}\min_{x\in \Delta_M}x^\top Gy= \min_{x\in \Delta_M}\max_{y\in \Delta_N} x^\top Gy 
%&\leq \max_{y \in \Delta_N}\mathbb{E}[\bar{x}]^\top Gy \leq \min_{x \in \Delta_M} x^\top G \mathbb{E}[\bar{y}]  + \tilde{\mathcal{O}}((M+N)^{\frac{5}{4}}T^{-\frac{3}{4}}) \\
%&\leq \min_{x\in \Delta_M}\max_{y\in \Delta_N} x^\top Gy + \tilde{\mathcal{O}}((M+N)^{\frac{5}{4}}T^{-\frac{3}{4}}).
%\end{align*}
%This implies that in expectation the two players' average strategies $(\bar{x},\bar{y})$ converges to nearly optimal ones in the rate of $\tilde{\mathcal{O}}((M+N)^{\frac{5}{4}}T^{-\frac{3}{4}})$, because 
\begin{align*}
\max_{y\in\Delta_N} \mathbb{E}[\bar{x}]^\top Gy 
&\leq \min_{x \in \Delta_M} x^\top G \mathbb{E}[\bar{y}]  + \tilde{\mathcal{O}}((M+N)^{\frac{5}{4}}T^{-\frac{3}{4}}), \\
&\leq \min_{x\in \Delta_M}\max_{y\in \Delta_N} x^\top Gy + \tilde{\mathcal{O}}((M+N)^{\frac{5}{4}}T^{-\frac{3}{4}})
= \text{\rm Val} + \tilde{\mathcal{O}}((M+N)^{\frac{5}{4}}T^{-\frac{3}{4}}), 
\end{align*}
and similarly
\begin{align*}
\min_{x\in \Delta_M} x^\top G\mathbb{E}[\bar{y}]
&\geq \max_{y\in\Delta_N} \mathbb{E}[\bar{x}]^\top Gy - \tilde{\mathcal{O}}((M+N)^{\frac{5}{4}}T^{-\frac{3}{4}}) \\
&\geq \max_{y\in \Delta_N}\min_{x\in \Delta_M}x^\top Gy - \tilde{\mathcal{O}}((M+N)^{\frac{5}{4}}T^{-\frac{3}{4}})
= \text{\rm Val} - \tilde{\mathcal{O}}((M+N)^{\frac{5}{4}}T^{-\frac{3}{4}}), 
\end{align*}
completing the proof.
\end{proof}

%\begin{proof}{\textbf{of Theorem~\ref{theorem:better_for_stable}}.}
%Similar to the proof of Theorem~\ref{thm:fast_convergence_theorem}, we have by Corollary \ref{cor:path_length_bound_1}, 
%\begin{align*}
%\mathbbm{E}\left[ \text{Reg}_T^1 \right] &=\mathcal{O}\left(\frac{M\ln T}{\eta} + \eta M \mathbb{E}\left[\sum_{t=1}^T \norm{y_t-y_{t-1}}_1\right]  \right).
%\end{align*}
%By our assumption, the bounded is further bounded by $\tilde{\mathcal{O}}\left( \frac{M}{\eta}+\eta MT\kappa \right)$. Selecting the optimal $\eta$ we get the desired bound. 
%\end{proof}

As shown by the theorem, we obtain convergence rate faster than $1/\sqrt{T}$,
but still slower than the $1/T$ rate compared to the full-information setup of~\citep{rakhlin2013optimization, syrgkanis2015fast},
due to the fact that we only have first-order instead of second-order path-length bound.

Note that~\citet{rakhlin2013optimization} also studies two-player zero-sum games with bandit feedback
but with an unnatural restriction that in each round the players play the same strategy for four times.
\citet{foster2016learning} greatly weakened the restriction, but their algorithm only converges to some approximation of Val.
For further comparisons, the readers are referred to the comparisons to~\citep{syrgkanis2015fast}
in \citep{foster2016learning}.
We also point out that the question raised in \citep{rakhlin2013optimization} remains open: if the players only receive the realized loss/reward $\mathbf{e}_{i_t}^\top G\mathbf{e}_{j_t}$ as feedback (a more natural setup), can the convergence rate to Val be faster than $1/\sqrt{T}$?

\section{Proof of Theorem \ref{thm:best of both}}
%\begin{lemma}
%\label{lemma:same_point}
%Consider the MAB problem ($\Omega=\Delta_K$). In the update rule~\eqref{eqn:update_rule_1}, if $m_t=r\mathbf{1}$ for some $r\in [-1,1]$, then $w_t=w_t^\p$. 
%\end{lemma}
%\begin{proof}
%Because $\psi_t$ is strictly convex in $\Delta_K$, $D_{\psi_t}(w, w_t^\p)>0$ for all $w\in \Delta_K$ and $w\neq w_t^\p$. Now if $w_t\neq w_t^\p$, then we have $D_{\psi_t}(w_t, w_t^\p)>0$. On the other hand, by $w_t$'s optimality, we have $\inn{w_t, r\mathbf{1}}+D_{\psi_t}(w_t,w_t^\p)\leq \inn{w_t^\p, r\mathbf{1}}+D_{\psi_t}(w_t^\p,w_t^\p)=\inn{w_t^\p, r\mathbf{1}}$. Therefore, $D_{\psi_t}(w_t,w_t^\p)\leq \inn{w_t^\p-w_t, r\mathbf{1}}=0$ because $\inn{w_t, \mathbf{1}}=\inn{w_t^\p, \mathbf{1}}=1$. This contradicts to $D_{\psi_t}(w_t,w_t^\p)>0$. Hence, $w_t=w_t^\p$.
%\end{proof}


\begin{proof}{\textbf{of Theorem \ref{thm:best of both}}.}
We first verify conditions (ii) and (iii) in Theorem~\ref{thm:doubling_trick_theorem} hold for $\hat{\ell}_{t,i}=\frac{\ell_{t,i}\mathbbm{1}\{i_t=i\}}{w_{t,i}}$ and $m_{t,i}=\ell_{t,i_t}$. Indeed, condition (ii) holds since $w_{t,i}\abs{\hat{\ell}_{t,i}-m_{t,i}}=\abs{\ell_{t,i}\mathbbm{1}\{i_t=i\}-w_{t,i}\ell_{t,i_t}}\leq 2<3$.
Other the other hand, condition (iii) also holds because
\begin{align*}
\eta\sum_{i=1}^K w_{t,i}^2(\hat{\ell}_{t,i}-m_{t,i})^2
&=\eta\sum_{i=1}^K  (\ell_{t,i}\mathbbm{1}\{i_t=i\}-w_{t,i}\ell_{t,i_t})^2\\
&=\eta\sum_{i=1}^K (\ell_{t,i}^2\mathbbm{1}\{i_t=i\}-2\ell_{t,i}w_{t,i}\ell_{t,i_t}\mathbbm{1}\{i_t=i\}+w_{t,i}^2\ell_{t,i_t}^2)\\
&\leq \frac{1}{162}\left(\ell_{t,i_t}^2-2w_{t,i_t}\ell_{t,i_t}^2 + \left(\sum_{i=1}^K w_{t,i}^2\right)\ell_{t,i_t}^2\right)\\
&\leq \frac{1}{162}\left(1+0+1\right) < \frac{1}{18}. 
\end{align*}
Thus, %by Theorem \ref{lemma:second_order_regret_bound}, we have the bound 
%\begin{align}
%\sum_{t=1}^T \inn{w_t-\mathbf{e}_{i^*}, \hat{\ell}_t}\leq\frac{K\ln T}{\eta} +3\eta\sum_{t=1}^T\sum_{i=1}^K w_{t,i}^2(\hat{\ell}_{t,i}-\ell_{t,i_t})^2 + B,  \label{eqn:squint_like}
%\end{align}
%where $B\triangleq \frac{1}{T}\sum_{t=1}^T \inn{-\mathbf{e}_{i^*}+w_1^\p, \hat{\ell}_t}$, 
%and 
by Theorem~\ref{thm:doubling_trick_theorem}, we have
\begin{align}
\mathbb{E}\left[\sum_{t=1}^T \ell_{t,i_t}-\ell_{t,i^*}\right]=\mathcal{O}\left( \sqrt{(K\ln T) \mathbb{E}\left[\sum_{t=1}^T \sum_{i=1}^K w_{t,i}^2(\hat{\ell}_{t,i}-\ell_{t,i_t})^2\right]} + K\ln T \right). \label{eqn:squint_like_further}
\end{align}

Now we consider the stochastic setting. In this case, we further take expectations over $\ell_1, \ldots, \ell_T$ on both sides of \eqref{eqn:squint_like_further}. The left-hand side of \eqref{eqn:squint_like_further} can be lower bounded by
\begin{align}
\mathbb{E}\left[\sum_{t=1}^T \ell_{t,i_t}-\ell_{t,i^*}\right]
&=\mathbb{E}\left[\sum_{t=1}^T \ell_{t,i_t}-\min_{j}\sum_{t=1}^T\ell_{t,j}\right]
\geq \mathbb{E}\left[\sum_{t=1}^T \ell_{t,i_t}-\sum_{t=1}^T\ell_{t,a^*}\right] \nonumber\\
&=\mathbb{E}\left[\sum_{t=1}^T \sum_{i=1}^K w_{t,i}(\ell_{t,i}-\ell_{t,a^*})\right]
\geq \mathbb{E}\left[\sum_{t=1}^T \sum_{i\neq a^*} w_{t,i}\Delta\right]=\Delta\mathbb{E}\left[ \sum_{t=1}^T (1-w_{t,a^*}) \right]. \label{eqn:best_one_direction}
\end{align}
On the other hand, 
\begin{align}
&\mathbb{E}_{i_t\sim w_t}\left[\sum_{i=1}^Kw_{t,i}^2(\hat{\ell}_{t,i}-\ell_{t,i_t})^2\right]
=\mathbb{E}_{i_t\sim w_t}\left[\sum_{i=1}^Kw_{t,i}^2\left(\frac{\ell_{t,i}\mathbbm{1}\{i_t=i\}}{w_{t,i}}-\ell_{t,i_t}\right)^2\right] \nonumber \\
&=\mathbb{E}_{i_t\sim w_t}\left[\sum_{i=1}^K\left(\ell_{t,i}\mathbbm{1}\{i_t=i\}-w_{t,i}\ell_{t,i_t}\right)^2\right] \nonumber \\
&=\sum_{i=1}^K \left( w_{t,i}\left( \ell_{t,i}-w_{t,i}\ell_{t,i} \right)^2 +\sum_{j\neq i} w_{t,j}(w_{t,i}\ell_{t,j})^2 \right) \nonumber \\
&\leq \sum_{i=1}^K \left( w_{t,i}\left( 1-w_{t,i}\right)^2 +\sum_{j\neq i} w_{t,j}w_{t,i}^2 \right)=\sum_{i=1}^K w_{t,i}(1-w_{t,i}) \nonumber \\
&\leq (1-w_{t,a^*}) + \sum_{i\neq a^*} w_{t,i} = 2(1-w_{t,a^*}). \label{eqn:regret_expansion_best_both}
\end{align}
Therefore, the first term on the right-hand side of \eqref{eqn:squint_like_further} can be upper bounded by
\begin{align}
\sqrt{(K\ln T)\mathbb{E}\left[\sum_{t=1}^T \sum_{i=1}^K w_{t,i}^2(\hat{\ell}_{t,i}-\ell_{t,i_t})^2\right]}\leq \sqrt{(K\ln T) \mathbb{E}\left[\sum_{t=1}^T 2(1-w_{t,a^*}) \right]}. \label{eqn:best_another_direction}
\end{align}
Let $H=\mathbb{E}\left[\sum_{t=1}^T (1-w_{t,a^*}) \right]$. Combining \eqref{eqn:best_one_direction}, \eqref{eqn:best_another_direction}, and \eqref{eqn:squint_like_further}, we have 
\begin{align*}
H\Delta \leq \mathcal{O}\left(\sqrt{(K\ln T)H} + K\ln T\right), 
\end{align*}
which implies $H=\mathcal{O}\left(\frac{K\ln T}{\Delta^2}\right)$. Therefore, the expected regret is upper bounded by 
\[\mathcal{O}\left( \sqrt{(K\ln T)H}+K\ln T \right) = \mathcal{O}\left(\frac{K\ln T}{\Delta}\right) .\]

For the adversarial setting, we continue from an intermediate step of \eqref{eqn:regret_expansion_best_both}: 
\begin{align*}
&\mathbb{E}_{i_t\sim w_t}\left[\sum_{i=1}^Kw_{t,i}^2(\hat{\ell}_{t,i}-\ell_{t,i_t})^2\right]
=\sum_{i=1}^K \left( w_{t,i}(1-w_{t,i})^2\ell_{t,i}^2 +\sum_{j\neq i} w_{t,j}w_{t,i}^2\ell_{t,j}^2 \right) \\
&\leq \sum_{i=1}^K w_{t,i}\ell_{t,i}^2 + \sum_{j=1}^K \sum_{i\neq j} w_{t,j}w_{t,i}^2\ell_{t,j}^2\leq \sum_{i=1}^K w_{t,i}\ell_{t,i}^2 + \sum_{j=1}^K w_{t,j}\ell_{t,j}^2 = 2\mathbb{E}_{i_t\sim w_t}\left[\ell_{t,i_t}^2\right]
\end{align*}
%Combining this with \eqref{eqn:squint_like_further}, we get 
%\begin{align}
%\mathbb{E}[\text{Reg}_T] = \mathcal{O}\left( \sqrt{KL\ln T}+K\ln T\right). \label{eqn:best_bound_general}
%\end{align}
Assuming $\ell_{t,i}\in [0,1]$, we thus have $\ell_{t,i_t}^2\leq \ell_{t,i_t}$ and  
\begin{align*}
\mathbb{E}\left[ \sum_{t=1}^T \ell_{t,i_t} \right] - \sum_{t=1}^T \ell_{t,i^*} = \mathcal{O}\left( \sqrt{(K\ln T)\mathbb{E}\left[ \sum_{t=1}^T \ell_{t,i_t} \right]} + K\ln T\right). 
\end{align*}
Solving for $\sqrt{\mathbb{E}\left[ \sum_{t=1}^T\ell_{t,i_t} \right]}$ and rearranging then give
\begin{align*}
\mathbb{E}\left[ \sum_{t=1}^T \ell_{t,i_t} \right] - \sum_{t=1}^T \ell_{t,i^*} = \mathcal{O}\left( \sqrt{(K\ln T)\sum_{t=1}^T \ell_{t,i^*} } + K\ln T\right)=\mathcal{O}\left(\sqrt{KL_{T,i^*}\ln T}+K\ln T\right). 
\end{align*}
\end{proof}


\end{document}
