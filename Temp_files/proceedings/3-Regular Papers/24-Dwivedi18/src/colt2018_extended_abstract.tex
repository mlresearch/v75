\documentclass[final, 12pt]{colt2018} % Anonymized submission
% \documentclass{colt2017} % Include author names

% The following packages will be automatically loaded:
% amsmath, amssymb, natbib, graphicx, url, algorithm2e

\title[MALA]{Log-concave sampling: Metropolis-Hastings algorithms are fast!}

% \usepackage{amsthm}
% \usepackage{fullpage}
\usepackage{graphics}
\usepackage{graphicx}
\usepackage{times}
% Top and bottom rules for tables
\usepackage{booktabs}

% for adjust width
\usepackage{chngpage}

% to label enumerate
\usepackage{enumitem}
% for captions
% \usepackage{caption}
% \usepackage{subcaption}


% for mathmakebox
\usepackage{mathtools}
% macros
%!TEX root = LWM.tex
% Macros
\DeclareMathOperator{\BigOm}{\mathcal{O}}
\DeclareMathOperator{\BigOmtil}{\widetilde{\mathcal{O}}}

\newcommand{\BigOh}[1]{\BigOm\left({#1}\right)}
\newcommand{\BigOhTil}[1]{\BigOmtil\left({#1}\right)}

\DeclareMathOperator{\BigTm}{\Theta}
\newcommand{\BigTheta}[1]{\BigTm\left({#1}\right)}
\DeclareMathOperator{\BigWm}{\Omega}
\newcommand{\BigOmega}[1]{\BigWm\left({#1}\right)}
\DeclareMathOperator{\LittleOm}{\mathrm{o}}
\newcommand{\LittleOh}[1]{\LittleOm\left({#1}\right)}
\DeclareMathOperator{\LittleWm}{\omega}
\newcommand{\LittleOmega}[1]{\LittleWm\left({#1}\right)}


\newcommand{\vol}{\mathrm{vol}}
\newcommand{\adj}{*}

\newcommand{\Gambar}{\overline{\Gamma}}
\newcommand{\Gamk}{\Gamma_k}
\newcommand{\Gamsb}{\Gamma_{\mathrm{sb}}}


\newcommand{\sphereGamin}{\calS_{\Gamin}}
\newcommand{\normGamax}{\|\Gamax^{1/2}(\cdot)\|_2}

\newcommand{\Gamin}{\mathtt{\Gamma}_{\min}}
\newcommand{\Gamax}{\mathtt{\Gamma}_{\max}}

\newcommand{\calV}{\mathcal{V}}
\newcommand{\calL}{\mathcal{L}}
\newcommand{\calE}{\mathcal{E}}

\newcommand{\calN}{\mathcal{N}}
\newcommand{\calF}{\mathcal{F}}
\newcommand{\calT}{\mathcal{T}}

\newcommand{\calY}{\mathcal{Y}}
\newcommand{\calS}{\mathcal{S}}
\newcommand{\calU}{\mathcal{U}}
\newcommand{\Alt}{\mathrm{Alt}}
\newcommand{\kl}{\mathrm{kl}}
\newcommand{\calA}{\mathcal{A}}
\newcommand{\calP}{\mathcal{P}}
\newcommand{\calB}{\mathcal{B}}
\newcommand{\op}{\mathrm{op}}

\newcommand{\Xtil}{\widetilde{X}}
\newcommand{\calFtil}{\widetilde{\mathcal{F}}}

\newcommand{\calG}{\mathcal{G}}
\newcommand{\eff}{\mathrm{eff}}

\newcommand{\calH}{\mathcal{H}}

\newcommand{\ArgTop}{\mathrm{ArgTop}}
\newcommand{\Top}{\mathrm{Top}}
\newcommand{\TopEps}{\mathrm{Top}^{\epsilon}}
\newcommand{\ArgTopEps}{\mathrm{ArgTop}^{\epsilon}}


\newcommand{\boldeps}{\boldsymbol{\epsilon}}
\newcommand{\itlam}{\overline{\lambda}}
\newcommand{\itlamin}{\itlam_{\min}}
\newcommand{\itlamax}{\lambda_{+}}
\newcommand{\itdelta}{\mathtt{\Delta}}

\newcommand{\diag}{\mathrm{diag}}
\newcommand{\Gramm}{\Gamma}
\newcommand{\GrammB}{\Gamma^{B_*}}
\newcommand{\sigmaU}{\sigma_u}

\newcommand{\KL}{\mathrm{KL}}
\newcommand{\Sym}{\mathrm{Sym}}
\newcommand{\Alg}{\mathrm{Alg}}
\newcommand{\Sim}{\mathrm{Sim}}
\newcommand{\MAT}{\mathsf{MAT}}
\newcommand{\VEC}{\mathsf{VEC}}
\newcommand{\ealphasmoothv}{\mathcal{E}_{\mathrm{smooth}}(v,\alpha)}
\newcommand{\ealphasmoothvi}{\mathcal{E}_{\mathrm{smooth}}(v_i,\alpha)}
\newcommand{\ealphasmooth}{\mathcal{E}_{\mathrm{smooth}}(\alpha)}

\newcommand{\matU}{\mathbf{U}}
\newcommand{\matV}{\mathbf{V}}
\newcommand{\matSig}{\mathbf{\Sigma}}

\newcommand{\matutil}{\widetilde{\matu}}

\newcommand{\matX}{\mathbf{X}}
\newcommand{\mateps}{\mathbf{E}}
\newcommand{\cond}{\mathrm{cond}}
\newcommand{\kmrk}{k_{\mathrm{mrk}}}

\newcommand{\istar}{i^*}
\newcommand{\im}{\mathrm{im }}
\newcommand{\sysid}{\mathtt{SysID }}
\newcommand{\OLS}{\mathsf{OLS }}

\newcommand{\Z}{\mathbb{Z}}
\newcommand{\N}{\mathbb{N}}
\newcommand{\Ast}{A_{*}}
\newcommand{\Bst}{B_{*}}
\newcommand{\ALS}{\widehat{A}}
\newcommand{\BLS}{\widehat{B}}
\renewcommand{\ast}{a_{*}}
\newcommand{\als}{\widehat{a}}
\newcommand{\iidsim}{\overset{\mathrm{idd}}{\sim}}
\newcommand{\Bern}{\mathrm{Bernoulli}}

\newcommand{\Od}{\mathrm{O}(d)}
\newcommand{\SOd}{\mathrm{SO}(d)}
\newcommand{\Skew}{\mathrm{Skew}}
\newcommand{\eig}{\rho}

\newcommand{\block}{B}
\newcommand{\Noise}{\mathbf{E}}
\newcommand{\noiseb}{W}
\newcommand{\noise}{\eta}
\newcommand{\direc}{w}
\newcommand{\C}{\mathbb{C}}

\newcommand{\R}{\mathbb{R}}
\newcommand{\I}{\mathbb{I}}
\newcommand{\Exp}{\mathbb{E}}
\newcommand{\Q}{\mathbb{Q}}
\newcommand{\sign}{\mathrm{sign\ }}
\newcommand{\abs}{\mathrm{abs\ }}
\newcommand{\eps}{\varepsilon}
\newcommand{\Var}{\mathrm{Var}}
\newcommand{\tr}{\operatorname{tr}}
\newcommand{\Cov}{\mathrm{Cov}}
\newcommand{\zo}{\{0, 1\}}
\newcommand{\SAT}{\mathit{SAT}}
\renewcommand{\P}{\mathbf{P}}
\renewcommand{\implies}{~\mathrm{implies}~}
\newcommand{\iid}{\stackrel{\mathclap{\text{\scriptsize{ \tiny i.i.d.}}}}{\sim}}

\newcommand{\NP}{\mathbf{NP}}
\newcommand{\coNP}{\co{NP}}
\newcommand{\co}[1]{\mathbf{co#1}}
\renewcommand{\Pr}{\mathbb{P}}

\newtheorem{thm}{Theorem}[section]
\newtheorem{claim}[thm]{Claim}
\newtheorem{lem}[thm]{Lemma}
\newtheorem{cor}[thm]{Corollary}
\newtheorem{asm}{Assumption}


\newtheorem{prop}[thm]{Proposition}

%\theoremstyle{definition}
\newtheorem{defn}{Definition}[section]
\newtheorem{prob}{Problem}[section]
\newtheorem{fact}{Fact}[section]

\newtheorem{conj}{Conjecture}[section]
\newtheorem{exmp}{Example}[section]
\newtheorem{exc}{Exercise}[section]


%\theoremstyle{remark}
\newtheorem*{rem}{Remark}
\newtheorem*{note}{Note}
\usepackage{mathtools}
\DeclarePairedDelimiter\ceil{\lceil}{\rceil}
\DeclarePairedDelimiter\floor{\lfloor}{\rfloor}

\makeatletter
\makeatother
\renewcommand{\theequation}{\arabic{section}.\arabic{equation}}

\algnewcommand\algorithmicinput{\textbf{INPUT:}}
\algnewcommand\INPUT{\item[\algorithmicinput]}
\algnewcommand\algorithmicoutput{\textbf{OUTPUT:}}
\algnewcommand\OUTPUT{\item[\algorithmicoutput]}


% Norms
\newcommand{\opnormbig}[1]{\left\lVert #1 \right\rVert_{\mathrm{op}}}

\newcommand{\opnorm}[1]{\lVert #1 \rVert_{\mathrm{op}}}
\newcommand{\fronorm}[1]{\lVert #1 \rVert_{\mathrm{F}}}

%%% Local Variables:
%%% mode: plain-tex
%%% TeX-master: t
%%% End:

 % Use \Name{Author Name} to specify the name.
 % If the surname contains spaces, enclose the surname
 % in braces, e.g. \Name{John {Smith Jones}} similarly
 % if the name has a "von" part, e.g \Name{Jane {de Winter}}.
 % If the first letter in the forenames is a diacritic
 % enclose the diacritic in braces, e.g. \Name{{\'E}louise Smith}

 % Two authors with the same address
  % \coltauthor{\Name{Author Name1} \Email{abc@sample.com}\and
  %  \Name{Author Name2} \Email{xyz@sample.com}\\
  %  \addr Address}

 % Three or more authors with the same address:
 % \coltauthor{\Name{Author Name1} \Email{an1@sample.com}\\
 %  \Name{Author Name2} \Email{an2@sample.com}\\
 %  \Name{Author Name3} \Email{an3@sample.com}\\
 %  \addr Address}


 % Authors with different addresses:
 \coltauthor{\Name{Raaz Dwivedi$^\diamond$} \Email{raaz.rsk@berkeley.edu}\\
 \addr Department of Electrical Engineering and Computer Sciences, UC Berkeley
 \AND
 \Name{Yuansi Chen$^\diamond$} \Email{yuansi.chen@berkeley.edu}\\
 \addr Department of Statistics, UC Berkeley
 \AND
 \Name{Martin Wainwright} \Email{wainwrig@berkeley.edu} \\
 \Name{Bin Yu} \Email{binyu@berkeley.edu}\\
 \addr Department EECS and Department of Statistics, UC Berkeley
 }

\begin{document}

\maketitle

\begin{abstract}
We consider the problem of sampling from a strongly log-concave
density in $\mathbb{R}^d$, and prove a non-asymptotic upper bound on
the mixing time of the Metropolis-adjusted Langevin algorithm (MALA).
The method draws samples by running a Markov chain obtained from the
discretization of an appropriate Langevin diffusion, combined with an
accept-reject step to ensure the correct stationary distribution.
Relative to known guarantees for the unadjusted Langevin algorithm
(ULA), our bounds reveal that the use of an accept-reject step in MALA
leads to an exponentially improved dependence on the error-tolerance.
Concretely, in order to obtain samples with TV error at most $\delta$
for a density with condition number $\kappa$, we show that MALA
requires $\mathcal{O} \big(\kappa d \log(1/\delta) \big)$ steps, as
compared to the $\mathcal{O} \big(\kappa^2 d/\delta^2 \big)$ steps
established in past work on ULA.  We also demonstrate the gains of
MALA over ULA for weakly log-concave densities.  Furthermore, we
derive mixing time bounds for a zeroth-order method Metropolized
random walk (MRW) and show that it mixes $\mathcal{O}(\kappa d)$
slower than MALA.
\end{abstract}
\let\thefootnote\relax\footnotetext{$\diamond$Raaz Dwivedi and Yuansi
Chen contributed equally to this work.\\
Extended abstract. Full version appears as [\url{https://arxiv.org/abs/1801.02309}, v2]}

\begin{keywords}
MCMC, sampling, random walk, Metropolis-adjusted Langevin algorithm, convergence
\end{keywords}

\section{Main Results}
\label{sec:main_results}
Recent decades have witnessed great success of Markov Chain Monte
Carlo (MCMC) algorithms suited for generating random samples; for
instance, see the handbook~\citep{brooks2011handbook} and references
therein. In a broad sense, these methods are based on two steps.  The
first step is to construct a Markov chain whose stationary
distribution is either equal to the target distribution or close to it
in a suitable metric.  Given this chain, the second step is to draw
samples by simulating the chain for a certain number of steps.

Many algorithms have been proposed and studied for sampling from
probability distributions with a density on a continuous space.  Two
broad categories of these methods are \emph{zeroth-order methods} and
\emph{first-order methods}. On one hand, a zeroth-order method is
based on querying the density of the distribution (up to a
proportionality constant) at a point in each iteration.  By contrast,
a first-order method also makes use of gradient information about the
density.  A few popular examples of zeroth order algorithms include
Metropolized random walk
(MRW)~\citep{mengersen1996rates,roberts1996geometric}, Ball
Walk~\citep{lovasz1990ballwalk,dyer1991random,lovasz1993random} and the
Hit-and-run algorithm~\citep{belisle1993hit, kannan1995isoperimetric,
  lovasz1999hit,lovasz2006hit, lovasz2007geometry}.  A number of
first-order methods are based on the Langevin diffusion.  Algorithms
related to the Langevin diffusion include the Metropolis adjusted
Langevin Algorithm (MALA)~\citep{roberts1996exponential,
  roberts2002langevin,bou2012nonasymptotic}, the unadjusted Langevin
algorithm (ULA)~\citep{parisi1981correlation,
  grenander1994representations,
  roberts1996exponential,dalalyan2016theoretical}, underdamped
Langevin MCMC~\citep{cheng2017underdamped}, Riemannian
MALA~\citep{xifara2014langevin},
Proximal-MALA~\citep{pereyra2016proximal, durmus2016efficient},
Metropolis adjusted Langevin truncated
algorithm~\citep{roberts1996exponential}, Hamiltonian Monte
carlo~\citep{neal2011mcmc} and Projected ULA~\citep{bubeck2015sampling}.
There is now a rich body of work on these methods, and we do not
attempt to provide a comprehensive summary in this paper.  More
details can be found in the survey~\citep{roberts2004general}, which
covers MCMC algorithms for general distributions, and the
survey~\citep{vempala2005geometric}, which focuses on random walks for
compactly supported distributions.

In this paper, we study sampling algorithms for sampling from a
log-concave distribution equipped with a density.  A log-concave
density takes the form
\begin{align}
\label{eq:target}
  \target\parenth{x} =
  \frac{e^{-\myfun(x)}}{\displaystyle\int_{\realdim} e^{-\myfun(y)}
    dy} \mbox{for all $x \in \realdim$,}
\end{align}
where $\myfun$ is a convex function on $\realdim$.  Up to an additive
constant, the function $-\myfun$ corresponds to the log-likelihood
defined by the density.  Standard examples of log-concave
distributions include the normal distribution, exponential
distribution and Laplace distribution.

Some recent work has provided non-asymptotic bounds on the mixing
times of Langevin type algorithms for sampling from a log-concave
density.  The mixing time corresponds to the number of steps, as
function of both the problem dimension $d$ and the error tolerance
$\threshold$, to obtain a sample from a distribution that is
$\threshold$-close to the target distribution in total variation
distance. It is known that both the ULA
updates~\citep{dalalyan2016theoretical,
  durmus2016high,cheng2017convergence} as well as underdamped Langevin
MCMC~\citep{cheng2017underdamped} have mixing times that scale
polynomially in the dimension $\dims$, as well the inverse of the
error tolerance $1/\threshold$.

Both the ULA and underdamped-Langevin MCMC methods are based on
evaluations of the gradient~$\gradf$, along with the addition of
Gaussian noise. \citet{durmus2016high} show that
for an appropriate decaying step size schedule, the ULA algorithm
converges to the right stationary distribution. However, their
results, albeit non-asymptotic, are hard to quantify. In the sequel,
we limit our discussion to Langevin algorithms based on constant step
sizes, for which there are a number of explicit quantitative bounds on
the mixing time.  When one uses a fixed step size for these
algorithms, an important issue is that the resulting random walks are
asymptotically biased: due to the lack of Metropolis-Hastings
correction step, the algorithms \emph{will not} converge to the
stationary distribution if run for a large number of steps.
Furthermore, if the step size is not chosen carefully the chains may
become transient~\citep{roberts1996exponential}.  Thus, typical theory
is based on running such a chain for a pre-specified number of steps,
depending on the tolerance, dimension and other problem parameters.


In contrast, the Metropolis-Hastings step that underlies the MALA
algorithm ensures that the resulting random walk has the correct
stationary distribution. \citet{roberts1996exponential} derived sufficient
conditions for exponential convergence of the Langevin
diffusion and its discretizations, with and without Metropolis-adjustment.
However, they considered the distributions with $\myfun(x) = \enorm{x}^\alpha$
and proved geometric convergence of ULA and MALA under some specific conditions.
In a more general setting, \citet{bou2012nonasymptotic} and \citet{eberle2014error}
derived non-asymptotic mixing time bounds for MALA.
However, all these bounds are non-explicit in the case of logconcave sampling, and so makes it
difficult to extract an explicit dependence in terms of the dimension
$\usedim$ and error tolerance~$\threshold$. In particular, \citet{eberle2014error} made significant contribution to establishing the accept-reject rate of MALA by assuming differentiability of the target function to fourth order, but its final mixing rate is only explicit when the sampling domain is contained in a ball with constant radius.  A precise characterization of this
dependence is needed if one wants to make quantitative comparisons
with other algorithms, including ULA and other Langevin-type schemes.
With this context, one of the main contributions of our paper is to
provide an explicit upper bound on the mixing time of the MALA
algorithm.

%%%%%%%%%%%%%%%%%%%%%%%%%%%%%%%%%%%%%%%%%%%%%%%%%%%%%%%%%%%%%%%%%%%%%%%%%%
% Our contributions
This work contains two main results,
both having to do with the mixing times of MCMC methods for
sampling. As described above, our first and primary contribution is an
explicit analysis of the mixing time of Metropolis adjusted Langevin
Algorithm (MALA).  A second contribution is to use similar techniques
to analyze a zeroth-order method called Metropolized random walk
(MRW) and derive a explicit non-asymptotic mixing time bound for it.
Unlike the ULA, these methods make use of the
Metropolis-hastings accept-reject step and consequently converge to
the target distributions in the limit of infinite steps.  Here we
provide explicit non-asymptotic mixing time bounds for MALA and MRW
and show that MALA converges significantly faster than ULA.
In particular, we show that if the density is strongly log-concave and
smooth, the $\threshold$-mixing time for MALA scales as $\condition
\dims\log(1/\threshold)$
which is significantly faster than ULA's convergence rate of order
$\condition^2\dims/\threshold^2$. We also show that
MRW mixes $\order{\condition\dims}$ slowly when compared to MALA.
Furthermore, if the density is
weakly log-concave, we show that MALA converges in $\calo\parenth{
  \dims^2/\threshold^{1.5}
}$ time in comparison to the $\calo\parenth{\dims^3/\threshold^{4}}$
mixing time for ULA. These results are summarized in Table~\ref{tab:mixing_times}.


\begin{table}
    \centering
    % \begin{adjustwidth}{-.2in}{-.3in}
    {\renewcommand{\arraystretch}{1.8}
    \begin{tabular}{ccc}
        \toprule
        \small {\bf Random walk} & {\bf Strongly log-concave}
        & {\bf Weakly log-concave}
        \\ \midrule
        %
        ULA~\citep{cheng2017convergence}
        &$\mathcal{O}\parenth{\displaystyle\frac{\dims\condition^2\log((\log \warmparam)/\threshold)}{\threshold^2}}$
        & $\displaystyle\tilde{\mathcal{O}}\parenth{\frac{\dims\smoothness^{2}}{\threshold^{6}}}$
        \\[2mm]
        ULA~\citep{dalalyan2016theoretical}
        &$\mathcal{O}\parenth{\displaystyle\frac{\dims\condition^2 \log^2(\warmparam/\threshold)}{\threshold^2}}$
        & $\displaystyle\tilde{\mathcal{O}}\parenth{\frac{\dims^3\smoothness^{2}}{\threshold^{4}}}$
        \\[2mm]
        %
        MRW
        &$\mathcal{O}\parenth{\displaystyle\dims^2 \condition^2 \log\parenth{\frac\warmparam\threshold}}$
        & $\displaystyle\tilde{\mathcal{O}}\parenth{\frac{\dims^4\,\smoothness^{2.5}}{\threshold^{1.5}}}$
        \\[2mm]
%
        MALA
        &$\mathcal{O}\parenth{\displaystyle \max\braces{\dims\condition, \dims^{0.5}\condition^{1.5}}
                \log\parenth{\frac\warmparam\threshold}}$
        & $\displaystyle\tilde{\mathcal{O}}\parenth{\frac{\dims^2\, \smoothness^{1.5}}{\threshold^{1.5}}}$
        \\[2ex]
        \bottomrule
        \hline
    \end{tabular}
    }
    % \end{adjustwidth}
    \caption{Scalings of upper bounds on $\threshold$-mixing time for different random walks in
      $\realdim$ with target $\target \propto e^{-\myfun}$.
      In the second column, we consider smooth and strongly log-concave densities, and report the bounds from a $\warmparam$-warm start for densities such that
      $\scparam \Ind_\dims \preceq \hessf(x) \preceq
      \smoothness\Ind_\dims$ for any $x \in \realdim$ and use $\condition
      \defn \smoothness/\scparam$ to denote the condition number of the density. The big-O notation hides universal constants.
      In the last column, we summarize the scaling for weakly log-concave smooth densities: $0 \preceq \hessf(x) \preceq
      \smoothness\Ind_\dims$ for all $x \in \realdim$. For this case, the $\tilde\calo$ notation is used to track scaling only with respect to $\dims, \threshold$ and $\smoothness$ and ignore dependence on the starting distribution and a few other parameters.}
    \label{tab:mixing_times}
\end{table}

% Acknowledgments---Will not appear in anonymized version
\acks{This work was supported by Office of Naval Research grant DOD
ONR-N00014 to MJW, and by ARO W911NF1710005, NSF-DMS 1613002 and the
Center for Science of Information (CSoI), US NSF Science and
Technology Center, under grant agreement CCF-0939370 to BY.  In
addition, MJW was partially supported by National Science Foundation
grant NSF-DMS-1612948, and RD was partially supported by the Berkeley
Fellowship.}


\bibliography{mala_bib}

\end{document}


