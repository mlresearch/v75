%\documentclass[11pt]{article}
\documentclass[final,12pt]{colt2018}

\usepackage[T1]{fontenc}

%\usepackage{geometry}
% \usepackage[letterpaper, left=1in, right=1in, top=1in, bottom=1in]{geometry}

%\usepackage[dvipsnames]{xcolor}
%\usepackage[colorlinks=true, linkcolor=blue, citecolor=blue]{hyperref}
\usepackage{microtype} 

\usepackage{algorithm}
\usepackage{algpseudocode}

%\usepackage{parskip}

% \usepackage{natbib}
% \bibliographystyle{plainnat}
% \bibpunct{(}{)}{;}{a}{,}{,}

% \usepackage{amsthm}
\usepackage{mathtools}
\usepackage{amsmath}
\usepackage{bbm}
\usepackage{amsfonts}
\usepackage{amssymb}
\let\vec\undefined
\usepackage{MnSymbol} %Actually conflicts with amssymb and others

\usepackage{xpatch}

%%% theorems

% \theoremstyle{definition}  %Sets style of subsequent newtheorems to 'definition'
% \newtheorem{exercise}{Exercise}
% \newtheorem{claim}{Claim}
% \newtheorem{lemma}{Lemma}
\newtheorem{assumption}{Assumption}
% \newtheorem{conjecture}{Conjecture}
% \newtheorem{corollary}{Corollary}
% \newtheorem{proposition}{Proposition}
% \newtheorem{fact}{Fact}
% %\newtheorem{model}{Model}
% %\newtheorem{problem}{Problem}
% \newtheorem{assumption}{Assumption}
% \newtheorem{problem}{Problem}
% \newtheorem{model}{Model}
% \theoremstyle{plain}
% \newtheorem{remark}{Remark}
% \newtheorem{example}{Example}
% \newtheorem{theorem}{Theorem}
% \newtheorem{definition}{Definition}

\xpatchcmd{\proof}{\itshape}{\normalfont\proofnameformat}{}{}
\newcommand{\proofnameformat}{\bfseries}

\newcommand{\notimplies}{\nRightarrow}

%%% prettyref

\usepackage{prettyref}
\newcommand{\pref}[1]{\prettyref{#1}}
\newcommand{\pfref}[1]{Proof of \prettyref{#1}}
\newcommand{\savehyperref}[2]{\texorpdfstring{\hyperref[#1]{#2}}{#2}}
\newrefformat{eq}{\savehyperref{#1}{\textup{(\ref*{#1})}}}
\newrefformat{eqn}{\savehyperref{#1}{Equation~\ref*{#1}}}
\newrefformat{lem}{\savehyperref{#1}{Lemma~\ref*{#1}}}
\newrefformat{def}{\savehyperref{#1}{Definition~\ref*{#1}}}
\newrefformat{thm}{\savehyperref{#1}{Theorem~\ref*{#1}}}
\newrefformat{corr}{\savehyperref{#1}{Corollary~\ref*{#1}}}
\newrefformat{sec}{\savehyperref{#1}{Section~\ref*{#1}}}
\newrefformat{app}{\savehyperref{#1}{Appendix~\ref*{#1}}}
\newrefformat{ass}{\savehyperref{#1}{Assumption~\ref*{#1}}}
\newrefformat{ex}{\savehyperref{#1}{Example~\ref*{#1}}}
\newrefformat{fig}{\savehyperref{#1}{Figure~\ref*{#1}}}
\newrefformat{alg}{\savehyperref{#1}{Algorithm~\ref*{#1}}}
\newrefformat{rem}{\savehyperref{#1}{Remark~\ref*{#1}}}
\newrefformat{conj}{\savehyperref{#1}{Conjecture~\ref*{#1}}}
\newrefformat{prop}{\savehyperref{#1}{Proposition~\ref*{#1}}}
\newrefformat{proto}{\savehyperref{#1}{Protocol~\ref*{#1}}}
\newrefformat{prob}{\savehyperref{#1}{Problem~\ref*{#1}}}
\newrefformat{claim}{\savehyperref{#1}{Claim~\ref*{#1}}}

% Math delimiters
\DeclarePairedDelimiter{\abs}{\lvert}{\rvert} %
\DeclarePairedDelimiter{\brk}{[}{]}
\DeclarePairedDelimiter{\crl}{\{}{\}}
\DeclarePairedDelimiter{\prn}{(}{)}
\DeclarePairedDelimiter{\nrm}{\|}{\|}
\DeclarePairedDelimiter{\tri}{\langle}{\rangle}
\DeclarePairedDelimiter{\dtri}{\llangle}{\rrangle}

\DeclarePairedDelimiter{\ceil}{\lceil}{\rceil}
\DeclarePairedDelimiter{\floor}{\lfloor}{\rfloor}


% \DeclareMathOperator{\E}{\mathbb{E}} %expecation
\let\Pr\undefined
\let\P\undefined
\DeclareMathOperator*{\En}{\mathbb{E}}
\DeclareMathOperator{\Enn}{\mathbb{E}}
\DeclareMathOperator*{\Eh}{\widehat{\mathbb{E}}}
\DeclareMathOperator{\P}{P}
\DeclareMathOperator{\Pr}{Pr}

% Arg<x>
\DeclareMathOperator*{\argmin}{arg\,min} % * Places subscript directly under operator
\DeclareMathOperator*{\argmax}{arg\,max}             
\DeclareMathOperator*{\arginf}{arg\,inf} 
\DeclareMathOperator*{\argsup}{arg\,sup} 

\DeclareMathOperator*{\smax}{smax_{\eta}}
\DeclareMathOperator*{\smin}{smin_{\eta}}

% one-off macros
\newcommand{\ls}{\ell}
\newcommand{\ind}{\mathbbm{1}}    %Indicator
\newcommand{\pmo}{\crl*{\pm{}1}}
\newcommand{\eps}{\epsilon}
\newcommand{\veps}{\varepsilon}

\newcommand{\ldef}{\vcentcolon=}
\newcommand{\rdef}{=\vcentcolon}

% styles
\newcommand{\mc}[1]{\mathcal{#1}}
\newcommand{\wt}[1]{\widetilde{#1}}
\newcommand{\wh}[1]{\widehat{#1}}


% Special letters: blackboard, mathcal, widehat
% djhsu magic
\def\ddefloop#1{\ifx\ddefloop#1\else\ddef{#1}\expandafter\ddefloop\fi}
\def\ddef#1{\expandafter\def\csname bb#1\endcsname{\ensuremath{\mathbb{#1}}}}
\ddefloop ABCDEFGHIJKLMNOPQRSTUVWXYZ\ddefloop
\def\ddefloop#1{\ifx\ddefloop#1\else\ddef{#1}\expandafter\ddefloop\fi}
\def\ddef#1{\expandafter\def\csname b#1\endcsname{\ensuremath{\mathbf{#1}}}}
\ddefloop ABCDEFGHIJKLMNOPQRSTUVWXYZ\ddefloop
\def\ddef#1{\expandafter\def\csname c#1\endcsname{\ensuremath{\mathcal{#1}}}}
\ddefloop ABCDEFGHIJKLMNOPQRSTUVWXYZ\ddefloop
\def\ddef#1{\expandafter\def\csname h#1\endcsname{\ensuremath{\widehat{#1}}}}
\ddefloop ABCDEFGHIJKLMNOPQRSTUVWXYZ\ddefloop
\def\ddef#1{\expandafter\def\csname hc#1\endcsname{\ensuremath{\widehat{\mathcal{#1}}}}}
\ddefloop ABCDEFGHIJKLMNOPQRSTUVWXYZ\ddefloop
\def\ddef#1{\expandafter\def\csname t#1\endcsname{\ensuremath{\widetilde{#1}}}}
\ddefloop ABCDEFGHIJKLMNOPQRSTUVWXYZ\ddefloop
\def\ddef#1{\expandafter\def\csname tc#1\endcsname{\ensuremath{\widetilde{\mathcal{#1}}}}}
\ddefloop ABCDEFGHIJKLMNOPQRSTUVWXYZ\ddefloop

% Names
\newcommand{\Holder}{H{\"o}lder}
%!TEX root = LWM.tex
% Macros
\DeclareMathOperator{\BigOm}{\mathcal{O}}
\DeclareMathOperator{\BigOmtil}{\widetilde{\mathcal{O}}}

\newcommand{\BigOh}[1]{\BigOm\left({#1}\right)}
\newcommand{\BigOhTil}[1]{\BigOmtil\left({#1}\right)}

\DeclareMathOperator{\BigTm}{\Theta}
\newcommand{\BigTheta}[1]{\BigTm\left({#1}\right)}
\DeclareMathOperator{\BigWm}{\Omega}
\newcommand{\BigOmega}[1]{\BigWm\left({#1}\right)}
\DeclareMathOperator{\LittleOm}{\mathrm{o}}
\newcommand{\LittleOh}[1]{\LittleOm\left({#1}\right)}
\DeclareMathOperator{\LittleWm}{\omega}
\newcommand{\LittleOmega}[1]{\LittleWm\left({#1}\right)}


\newcommand{\vol}{\mathrm{vol}}
\newcommand{\adj}{*}

\newcommand{\Gambar}{\overline{\Gamma}}
\newcommand{\Gamk}{\Gamma_k}
\newcommand{\Gamsb}{\Gamma_{\mathrm{sb}}}


\newcommand{\sphereGamin}{\calS_{\Gamin}}
\newcommand{\normGamax}{\|\Gamax^{1/2}(\cdot)\|_2}

\newcommand{\Gamin}{\mathtt{\Gamma}_{\min}}
\newcommand{\Gamax}{\mathtt{\Gamma}_{\max}}

\newcommand{\calV}{\mathcal{V}}
\newcommand{\calL}{\mathcal{L}}
\newcommand{\calE}{\mathcal{E}}

\newcommand{\calN}{\mathcal{N}}
\newcommand{\calF}{\mathcal{F}}
\newcommand{\calT}{\mathcal{T}}

\newcommand{\calY}{\mathcal{Y}}
\newcommand{\calS}{\mathcal{S}}
\newcommand{\calU}{\mathcal{U}}
\newcommand{\Alt}{\mathrm{Alt}}
\newcommand{\kl}{\mathrm{kl}}
\newcommand{\calA}{\mathcal{A}}
\newcommand{\calP}{\mathcal{P}}
\newcommand{\calB}{\mathcal{B}}
\newcommand{\op}{\mathrm{op}}

\newcommand{\Xtil}{\widetilde{X}}
\newcommand{\calFtil}{\widetilde{\mathcal{F}}}

\newcommand{\calG}{\mathcal{G}}
\newcommand{\eff}{\mathrm{eff}}

\newcommand{\calH}{\mathcal{H}}

\newcommand{\ArgTop}{\mathrm{ArgTop}}
\newcommand{\Top}{\mathrm{Top}}
\newcommand{\TopEps}{\mathrm{Top}^{\epsilon}}
\newcommand{\ArgTopEps}{\mathrm{ArgTop}^{\epsilon}}


\newcommand{\boldeps}{\boldsymbol{\epsilon}}
\newcommand{\itlam}{\overline{\lambda}}
\newcommand{\itlamin}{\itlam_{\min}}
\newcommand{\itlamax}{\lambda_{+}}
\newcommand{\itdelta}{\mathtt{\Delta}}

\newcommand{\diag}{\mathrm{diag}}
\newcommand{\Gramm}{\Gamma}
\newcommand{\GrammB}{\Gamma^{B_*}}
\newcommand{\sigmaU}{\sigma_u}

\newcommand{\KL}{\mathrm{KL}}
\newcommand{\Sym}{\mathrm{Sym}}
\newcommand{\Alg}{\mathrm{Alg}}
\newcommand{\Sim}{\mathrm{Sim}}
\newcommand{\MAT}{\mathsf{MAT}}
\newcommand{\VEC}{\mathsf{VEC}}
\newcommand{\ealphasmoothv}{\mathcal{E}_{\mathrm{smooth}}(v,\alpha)}
\newcommand{\ealphasmoothvi}{\mathcal{E}_{\mathrm{smooth}}(v_i,\alpha)}
\newcommand{\ealphasmooth}{\mathcal{E}_{\mathrm{smooth}}(\alpha)}

\newcommand{\matU}{\mathbf{U}}
\newcommand{\matV}{\mathbf{V}}
\newcommand{\matSig}{\mathbf{\Sigma}}

\newcommand{\matutil}{\widetilde{\matu}}

\newcommand{\matX}{\mathbf{X}}
\newcommand{\mateps}{\mathbf{E}}
\newcommand{\cond}{\mathrm{cond}}
\newcommand{\kmrk}{k_{\mathrm{mrk}}}

\newcommand{\istar}{i^*}
\newcommand{\im}{\mathrm{im }}
\newcommand{\sysid}{\mathtt{SysID }}
\newcommand{\OLS}{\mathsf{OLS }}

\newcommand{\Z}{\mathbb{Z}}
\newcommand{\N}{\mathbb{N}}
\newcommand{\Ast}{A_{*}}
\newcommand{\Bst}{B_{*}}
\newcommand{\ALS}{\widehat{A}}
\newcommand{\BLS}{\widehat{B}}
\renewcommand{\ast}{a_{*}}
\newcommand{\als}{\widehat{a}}
\newcommand{\iidsim}{\overset{\mathrm{idd}}{\sim}}
\newcommand{\Bern}{\mathrm{Bernoulli}}

\newcommand{\Od}{\mathrm{O}(d)}
\newcommand{\SOd}{\mathrm{SO}(d)}
\newcommand{\Skew}{\mathrm{Skew}}
\newcommand{\eig}{\rho}

\newcommand{\block}{B}
\newcommand{\Noise}{\mathbf{E}}
\newcommand{\noiseb}{W}
\newcommand{\noise}{\eta}
\newcommand{\direc}{w}
\newcommand{\C}{\mathbb{C}}

\newcommand{\R}{\mathbb{R}}
\newcommand{\I}{\mathbb{I}}
\newcommand{\Exp}{\mathbb{E}}
\newcommand{\Q}{\mathbb{Q}}
\newcommand{\sign}{\mathrm{sign\ }}
\newcommand{\abs}{\mathrm{abs\ }}
\newcommand{\eps}{\varepsilon}
\newcommand{\Var}{\mathrm{Var}}
\newcommand{\tr}{\operatorname{tr}}
\newcommand{\Cov}{\mathrm{Cov}}
\newcommand{\zo}{\{0, 1\}}
\newcommand{\SAT}{\mathit{SAT}}
\renewcommand{\P}{\mathbf{P}}
\renewcommand{\implies}{~\mathrm{implies}~}
\newcommand{\iid}{\stackrel{\mathclap{\text{\scriptsize{ \tiny i.i.d.}}}}{\sim}}

\newcommand{\NP}{\mathbf{NP}}
\newcommand{\coNP}{\co{NP}}
\newcommand{\co}[1]{\mathbf{co#1}}
\renewcommand{\Pr}{\mathbb{P}}

\newtheorem{thm}{Theorem}[section]
\newtheorem{claim}[thm]{Claim}
\newtheorem{lem}[thm]{Lemma}
\newtheorem{cor}[thm]{Corollary}
\newtheorem{asm}{Assumption}


\newtheorem{prop}[thm]{Proposition}

%\theoremstyle{definition}
\newtheorem{defn}{Definition}[section]
\newtheorem{prob}{Problem}[section]
\newtheorem{fact}{Fact}[section]

\newtheorem{conj}{Conjecture}[section]
\newtheorem{exmp}{Example}[section]
\newtheorem{exc}{Exercise}[section]


%\theoremstyle{remark}
\newtheorem*{rem}{Remark}
\newtheorem*{note}{Note}
\usepackage{mathtools}
\DeclarePairedDelimiter\ceil{\lceil}{\rceil}
\DeclarePairedDelimiter\floor{\lfloor}{\rfloor}

\makeatletter
\makeatother
\renewcommand{\theequation}{\arabic{section}.\arabic{equation}}

\algnewcommand\algorithmicinput{\textbf{INPUT:}}
\algnewcommand\INPUT{\item[\algorithmicinput]}
\algnewcommand\algorithmicoutput{\textbf{OUTPUT:}}
\algnewcommand\OUTPUT{\item[\algorithmicoutput]}


% Norms
\newcommand{\opnormbig}[1]{\left\lVert #1 \right\rVert_{\mathrm{op}}}

\newcommand{\opnorm}[1]{\lVert #1 \rVert_{\mathrm{op}}}
\newcommand{\fronorm}[1]{\lVert #1 \rVert_{\mathrm{F}}}

%%% Local Variables:
%%% mode: plain-tex
%%% TeX-master: t
%%% End:


%\usepackage[suppress]{color-edits}
%\usepackage{color-edits}
%\addauthor{df}{red}
%\addauthor{ks}{blue}
%\usepackage{showlabels}

%%%% Misc macros

% \newcommand{\pt}{\partial_{t}}
%\newcommand{\dl}{\partial}
\newcommand{\reg}{\mathrm{Reg}_n}
\renewcommand{\trn}{\top}
\newcommand{\dl}{\delta}
\newcommand{\sym}{\mathbb{S}}
\newcommand{\Bfun}{Burkholder }

\newcommand{\predt}{\wt{y}}

\newcommand{\propone}{$1^{o}$}
\newcommand{\proptwo}{$2^{o}$}
\newcommand{\propthree}{$3^{o}$}
\newcommand{\propthreep}{$3'$}

%%%% document info %%%%%%%

\title[Online Learning: Sufficient Statistics and the Burkholder Method]{Online Learning: Sufficient Statistics and the Burkholder Method}
\usepackage{times}

\date{}

%\author{
%  Dylan J. Foster\thanks{Cornell University}
%  \and
%  Alexander Rakhlin\thanks{MIT}
%  \and
%  Karthik Sridharan\footnotemark[1]
%  }
  \coltauthor{\Name{Dylan J. Foster}\Email{djfoster@cs.cornell.edu}\\
\addr Cornell University
\AND
\Name{Alexander Rakhlin}\Email{rakhlin@mit.edu}\\
\addr Massachusetts Institute of Technology
\AND
\Name{Karthik Sridharan}\Email{sridharan@cs.cornell.edu}\\
\addr Cornell University
  }

\begin{document}

\maketitle

\begin{abstract}
We uncover a fairly general principle in online learning: If a regret inequality can be (approximately) expressed as a function of certain ``sufficient statistics'' for the data sequence, then there exists a special \emph{Burkholder function} that 1) can be used algorithmically to achieve the regret bound and 2) only depends on these sufficient statistics, not the entire data sequence,  so that the online strategy is only required to keep the sufficient statistics in memory. This characterization is achieved by bringing the full power of the \emph{Burkholder Method}---originally developed for certifying probabilistic martingale inequalities---to bear on the online learning setting.

To demonstrate the scope and effectiveness of the Burkholder method, we develop a novel online strategy for matrix prediction that attains a regret bound corresponding to the variance term in matrix concentration inequalities. We also present a linear-time/space prediction strategy for parameter-free supervised learning with linear classes and general smooth norms. 

\end{abstract}

\section{Introduction}
Online learning algorithms are a key tool in web search and content optimization, adaptively learning what users want to see. In a typical application, each time a user arrives, the algorithm chooses among various content presentation options (\eg news articles to display), the chosen content is presented to the user, and an outcome (\eg a click) is observed. Such algorithms must balance \emph{exploration} (making potentially suboptimal decisions now for the sake of acquiring information that will improve decisions in the future) and \emph{exploitation} (using information collected in the past to make better decisions now). Exploration could degrade the experience of a current user, but improves user experience in the long run. This exploration-exploitation tradeoff is commonly studied in the online learning framework of \emph{multi-armed bandits}~\citep{Bubeck-survey12}.

Concerns have been raised about whether exploration in such scenarios could be unfair, in the sense that some individuals or groups may experience too much of the downside of exploration without sufficient upside \citep{bird2016exploring}. We formally study these concerns in the \emph{linear contextual bandits} model~\citep{Langford-www10,chu2011contextual}, a standard variant of multi-armed bandits appropriate for content personalization scenarios.  We focus on \emph{externalities} arising due to exploration, that is, undesirable side effects that the presence of one party may impose on another.


We first examine the effects of exploration at a group level.  We introduce the notion of a \emph{group externality} in an online learning system, quantifying how much the presence of one population (which we dub the majority) impacts the rewards of another (the minority). We show that this impact can be negative, and that, in a particular precise sense, no algorithm can avoid it. This cannot be explained by the absence of suitably good policies since our adoption of the linear contextual bandits framework implies the existence of a feasible policy that is simultaneously optimal for everyone. Instead, the problem is inherent to the process of exploration. We come to a surprising conclusion that more data can sometimes lead to worse outcomes for the users of an explore-exploit-based system. \looseness=-1

We next turn to the effect of exploration at an individual level. We interpret exploration as a potential externality imposed on the current user by future users of the system. Indeed, it is only for the sake of the future users that the algorithm would forego the action that currently looks optimal. To avoid this externality, one may use the greedy algorithm that always chooses the action that appears optimal according to current estimates of the problem parameters. While this greedy algorithm performs poorly in the worst case,
it tends to work well in many applications and experiments.\footnote{Both positive and negative findings are folklore. One way to precisely state the negative result is that the greedy algorithm incurs constant per-round regret with constant probability; while results of this form have likely been known for decades,
\citet[Corollary A.2(b)]{competingBandits-itcs16}
proved this for a wide variety of scenarios. Very recently, the good empirical performance has been confirmed by state-of-art experiments in \citet{practicalCB-arxiv18}.}

In a new line of work, \citet{bastani2017exploiting} and \citet{kannan2018smoothed}
analyzed conditions under which inherent diversity in the data makes explicit exploration unnecessary.
\citet{kannan2018smoothed} proved that the greedy algorithm achieves a regret rate of
$\tilde{O}(\sqrt{T})$ in expectation over small perturbations of the context vectors (which ensure sufficient data diversity). This is the best rate that can be achieved in the worst case (\ie for all problem instances, without data diversity assumptions), but it leaves open the possibilities that (i) another algorithm may perform much better than the greedy algorithm on some problem instances, or (ii) the greedy algorithm may perform much better than worst case under the diversity conditions. We expand on this line of work. We prove that under the same diversity conditions, the greedy algorithm almost matches the best possible Bayesian regret rate of \emph{any} algorithm \emph{on the same problem instance}. This could be as low as $\polylog(T)$ for some instances, and, as we prove, at most $\tilde{O}(T^{1/3})$ whenever the diversity conditions hold.


Returning to group-level effects, we show that under the same diversity conditions, the negative group externalities imposed by the majority essentially vanish if one runs the greedy algorithm. Together, our results illustrate a sharp contrast between the high individual and group externalities that exist in the worst case, and the ability to remove all externalities if the data is sufficiently diverse.   \looseness=-1

\xhdr{Additional motivation.} Whether and when explicit exploration is necessary is an important concern in the study of the exploration-exploitation tradeoff. Fairness considerations aside, explicit exploration is expensive. It is wasteful and risky in the short term, it adds a layer of complexity to algorithm design \citep{Langford-nips07,monster-icml14}, and its adoption at scale tends to require substantial systems support and buy-in from management \citep{MWT-WhitePaper-2016,DS-arxiv}. A system based on the greedy algorithm would typically be cheaper to design and deploy.

Further, explicit exploration can run into incentive issues in applications such as recommender systems. Essentially, when it is up to the users which products or experiences to choose and the algorithm can only issue recommendations and ratings, an explore-exploit algorithm needs to provide incentives to explore for the sake of the future users \citep{Kremer-JPE14,Frazier-ec14,Che-13,ICexploration-ec15,Bimpikis-exploration-ms17}. Such incentive guarantees tend to come at the cost of decreased performance, and rely on assumptions about human behavior. The greedy algorithm avoids this problem as it is inherently consistent with the users' incentives.



\xhdr{Additional related work.}
Our research draws inspiration from the growing body of work on fairness in machine learning~\cite[e.g.,][]{dwork2012fairness,hardt2016equality,kleinberg2017inherent,chouldechova2017fair}.  Several other authors have studied fairness in the context of the contextual bandits framework.  Our work differs from the line of research on meritocratic fairness in online learning \citep{kearns2017meritocratic,liu2017calibrated,joseph2016fairness}, which considers the allocation of limited resources such as bank loans and requires that nobody should be passed over in favor of a less qualified applicant. We study a fundamentally different scenario in which there are no allocation constraints and we would like to serve each user the best content possible.  Our work also differs from that of \citet{celis2017fair}, who studied an alternative notion of fairness in the context of news recommendations. According to this notion, all users should have approximately the same probability of seeing a particular type of content (e.g., Republican-leaning articles), regardless of their individual preferences, in order to mitigate the possibility of discriminatory personalization.

The data diversity conditions in \citet{kannan2018smoothed} and this paper are inspired by the smoothed analysis framework of \citet{SmoothedAnalysis-jacm04}, who proved that the expected running time of the simplex algorithm is polynomial for perturbations of any initial problem instance (whereas the worst-case running time has long been known to be exponential). Such disparity implies that very bad problem instances are brittle. 
We find a similar disparity for the greedy algorithm in our setting.



\xhdr{Our results on group externalities.}  A typical goal in online learning is to minimize \emph{regret}, the (expected) difference between the cumulative reward that would have been obtained had the optimal policy been followed at every round and the cumulative reward obtained by the algorithm.  We define a corresponding notion of \emph{minority regret}, the portion of the regret experienced by the minority.  Since online learning algorithms update their behavior based on the history of their observations, minority regret is influenced by the entire population on which an algorithm is run.  If the minority regret is much higher when a particular algorithm is run on the full population than it is when the same algorithm is run on the minority alone, we can view the majority as imposing a negative externality on the minority; the minority population would achieve a higher cumulative reward if the majority were not present. Asking whether this can ever happen
amounts to asking whether access to more data points can ever lead an explore-exploit algorithm to make inferior decisions. One might think that more data should always lead to better decisions and therefore better outcomes for the users.
Surprisingly, we show that this is not the case, even with a standard algorithm.

Consider LinUCB~\citep{Langford-www10,chu2011contextual,abbasi2011improved}, a standard algorithm for linear contextual bandits that is based on the principle of ``optimism under uncertainty.''  We provide a specific problem instance on which, after observing $T$ users, LinUCB would have a minority regret of $\Omega(\sqrt T)$ if run on the full population, but only constant minority regret if run on the minority alone. While stylized, this example is motivated by the problem of providing driving directions to different populations of users, about which fairness concerns have been raised~\citep{bird2016exploring}. Further, the situation is reversed on a slight variation of this example: LinUCB obtains constant minority regret when run on the full population and $\Omega(\sqrt T)$ on the minority alone.  That is, group externalities can be large and positive in some cases, and large and negative in others.

Although these regret rates are specific to LinUCB, we show that this phenomenon is, in some sense, unavoidable. Consider the minority regret of LinUCB when run on the full population and the minority regret that LinUCB would incur if run on the minority alone. We know that one may be much smaller or larger than the other. We ask whether any algorithm could  achieve the minimum of the two on every problem instance. Using a variation of the same problem instance, we prove that this is impossible; in fact, no algorithm could simultaneously approximate both up to any $o(\sqrt{T})$ factor. In other words, an externality-free algorithm would sometimes ``leave money on the table."


In terms of techniques, we rely on the special structure of our example, which can be viewed as an instance of the sleeping bandits problem~\citep{SleepingBandits-ml10}. This simplifies the behavior and analysis of LinUCB, allowing us to obtain the $O(1)$ upper bounds.  The lower bounds are obtained using KL-divergence techniques to show that the two variants of our example are essentially indistinguishable, and an algorithm that performs well on one must obtain $\Omega(\sqrt{T})$ regret on the other. \looseness=-1


\xhdr{Our results on the greedy algorithm.} We consider a version of linear contextual bandits in which the latent weight vector $\theta$ is drawn from a known prior. In each round, an algorithm is presented several actions to choose from, each represented by a \emph{context vector}. The expected reward of an action is a linear product of $\theta$ and the corresponding context vector. The tuple of context vectors is drawn independently from a fixed distribution. In the spirit of smoothed analysis, we assume that this distribution has a small amount of jitter. Formally, the tuple of context vectors is drawn from some fixed distribution, and then a small \emph{perturbation}---small-variance Gaussian noise---is added independently to each coordinate of each context vector. This ensures arriving contexts are diverse. We are interested in Bayesian regret, i.e., regret in expectation over the Bayesian prior. Following the literature, we are primarily interested in the dependence on the time horizon $T$. \looseness=-1

We focus on a batched version of the greedy algorithm, in which new data arrives to the algorithm's optimization routine in small batches, rather than every round. This is well-motivated from a practical perspective---in high-volume applications data usually arrives to the ``learner" only after a substantial delay \citep{MWT-WhitePaper-2016,DS-arxiv}---and is essential for our analysis.

Our main result is that the greedy algorithm matches the Bayesian regret of any algorithm up to polylogarithmic factors, for each problem instance, fixing the Bayesian prior and the context distribution. We also prove that LinUCB achieves regret $\tilde{O}(T^{1/3})$ for each realization of $\theta$. This implies a worst-case Bayesian regret of $\tilde{O}(T^{1/3})$ for the greedy algorithm under the perturbation assumption. \looseness=-1

Our results hold for both natural versions of the batched greedy algorithm, Bayesian and frequentist, henceforth called \BayesGreedy and \FreqGreedy. In \BayesGreedy, the chosen action maximizes expected reward according to the Bayesian posterior. \FreqGreedy estimates $\theta$ using ordinary least squares regression and chooses the best action according to this estimate. The results for \FreqGreedy come with additive polylogarithmic factors, but are stronger in that the algorithm does not need to know the prior. Further, the $\tilde{O}(T^{1/3})$ regret bound for \FreqGreedy is approximately prior-independent, in the sense that it applies even to very concentrated priors such as independent Gaussians with standard deviation on the order of $T^{-2/3}$.

The key insight in our analysis of \BayesGreedy is that any (perturbed) data can be used to simulate any other data, with some discount factor. The analysis of \FreqGreedy requires an additional layer of complexity. We consider a hypothetical algorithm that receives the same data as \FreqGreedy, but chooses actions based on the Bayesian-greedy selection rule. We analyze this hypothetical algorithm using the same technique as \BayesGreedy, and then upper bound the difference in Bayesian regret between the hypothetical algorithm and \FreqGreedy.

Our analyses extend to group externalities and (Bayesian) minority regret. In particular, we circumvent the impossibility result mentioned above. We prove that both \BayesGreedy and \FreqGreedy match the Bayesian minority regret of any algorithm run on either the full population or the minority alone, up to polylogarithmic factors

\xhdr{Detailed comparison with prior work.} We substantially improve over the $\tilde{O}(\sqrt{T})$ worst-case regret bound from \citet{kannan2018smoothed}, at the cost of some additional assumptions. First, we consider Bayesian regret, whereas their regret bound is for each realization of $\theta$.%
\footnote{Equivalently, they allow point priors, whereas our priors must have variance $T^{-O(1)}$.} Second, they allow the context vectors to be chosen by an adversary before the perturbation is applied. Third, they extend their analysis to a somewhat more general model, in which there is a separate latent weight vector for every action (which amounts to a more restrictive model of perturbations). However, this extension relies on the greedy algorithm being initialized with a substantial amount of data. The results of \citet{kannan2018smoothed} do not appear to have implications on group externalities.

\citet{bastani2017exploiting} show that the greedy algorithm achieves logarithmic regret in an alternative linear contextual bandits setting that is incomparable to ours in several important ways.
They consider two-action instances where the actions share a common context vector in each round, but are parameterized by different latent vectors. They ensure data diversity via a strong assumption on the context distribution. This assumption does not follow from our perturbation conditions; among other things, it implies that each action is the best action in a constant fraction of rounds. Further, they assume a version of Tsybakov's \emph{margin condition}, which is known to substantially reduce regret rates in bandit problems \citep[\eg see][]{Zeevi-colt10}.





\section{Example: Matrix Prediction}
\label{sec:matrix}
% !TEX root = paper.tex

In this section we focus on linear matrix prediction problems. The side information $x_t$ is now matrix-valued, and we shall denote it by a capital letter $X_t\in\bbR^{d_1\times{}d_2}$. Our goal is to achieve a regret inequality as in \pref{eq:phi_comp_adap} with a class $\F=\crl*{X\mapsto\tri*{W,X}\mid{} W\in\cW}$, where $\cW=\crl*{W\in\bbR^{d_1\times{}d_2}\mid{}\nrm*{W}_{\Sigma}\leq{}r}$. Here $\tri*{A,B}=\Tr(AB^{\trn})$ is the standard matrix inner product and $\nrm*{\cdot}_{\Sigma}$ denotes the nuclear norm. We also let $\nrm*{\cdot}_{\sigma}$ denote the spectral norm. As before, the loss $\loss$ is assumed to be $L$-Lipschitz and regret against a matrix $W\in\cW$ is given by
$
  \reg(W)\ldef \sum_{t=1}^{n}\loss(\pred_t, y_t) - \loss(\tri*{W,X_t}, y_t).
$

In a search for an adaptive bound on regret, we inspect the adaptive bound \pref{eq:adaptive} for the vector case. The direct analogue for matrices would be a bound proportional to $\left(\sum_{t=1}^n \norm{X_t}^2_\sigma \right)^{1/2}$, and indeed such a bound is possible with Matrix Exponential Weights \cite[Theorem 13]{HazKalSha12}.\footnote{With more work it is possible to obtain a bound of $\left(\max_{t}\nrm*{X_t}_{\sigma}\cdot\nrm*{\sum_{t=1}^nX_t}_\sigma \right)^{1/2}$; this is still weaker than our result, and seems to only be possible when the constraint set and $X_t$s are restricted to be positive-semidefinite.} However, the matrix version of Khintchine inequality, as well as matrix deviation inequalities, involve---for the case of random centered self-adjoint matrices---the tighter quantity $\norm{\sum_{t=1}^n X_t^2}^{1/2}_\sigma$ (see \citep{tropp2012user,mackey2014matrix}). Given the correspondence between online regret bounds and martingale inequalities, one may wonder if there is an algorithm that achieves this adaptive bound. We shall exhibit such a method using our approach, and the reader might already guess that $\sum_{t=1}^n X_t^2$ should be part of the sufficient statistic for the online algorithm. This is indeed the case, though we present results for general non-square matrices.

Let $\sym^{d}$ denote the set of symmetric matrices in $\bbR^{d\times{}d}$, $\sym^{d}_{+}$ denote the set of positive-semidefinite matrices, and $\sym^{d}_{++}$ denote the set of positive-definite matrices. 
For $X\in\sym^{d}$ we let $\lambda(X)\in\bbR^{d}$ denote its eigenvalues arranged in decreasing order, so that $\lambda_1(X)$ is the largest eigenvalue. For any matrix $X\in\bbR^{d_1\times{}d_2}$ we define its Hermitian dilation $\cH(X)\in\sym^{d_1+d_2}$ and  $\cM(X) \in\sym^{d_1+d_2}$ as:
\begin{equation}
  \cH(X) = \left(
    \begin{array}{ll}
      0 & X\\
      X^{\trn} & 0
    \end{array}
    \right) ~~~~~~~~~\cM(X) = \cH(X)^{2} = \left(
    \begin{array}{ll}
      XX^{\trn} & 0\\
      0 & X^{\trn}X
    \end{array}
  \right).
  \end{equation}
  It is well-known that for any matrix $X$, $\lambda_{1}(\cH(X)) = \nrm*{X}_{\sigma}.$

With these definitions in place, the desired adaptive regret bound takes the form
\begin{equation}
\cA_{\eta}(X_{1},\ldots,X_{n}) = \frac{\eta{}rL^2}{2}\nrm*{\sum_{t=1}^{n}\cM(X_t)}_{\sigma} + \frac{c}{\eta}
\end{equation}
for some fixed $\eta>0$ and constant $c>0$. The sufficient statistic takes values in  $\cT=\reals\times \sym^{d_1+d_2}\times \sym^{d_1+d_2}_{+}$ and incorporates the matrix variance terms $\cM(X_t)$.



\begin{proposition}
  \label{prop:matrix_sufficient}
  The pair $(\suff, V)$ defined via $\suff(X_t,\pred_t,\delta_t) = \left( \delta_t\cdot\pred_t, \delta_t\cdot \cH(X_t), \cM(X_t) \right)\in\reals\times \sym^{d_1+d_2}\times \sym^{d_1+d_2}_{+}$ and
\begin{equation}
\label{eq:matrix_v}
V(a, H, M) = a + r\lambda_1\prn*{H -{\textstyle\frac{1}{2}}\eta{}L^2 M} - \frac{c}{\eta},
\end{equation}
form a sufficient statistic pair for the adaptive regret bound $\cA_{\eta}$.
\end{proposition}
  
Now that we proposed a sufficient statistic, \pref{lem:suff_to_martingale} and \pref{lem:equivalence_burkholder} give a specific form for a martingale inequality and a construction for the special function (if the martingale inequality holds). Since the function constructed in the proof of \pref{lem:equivalence_burkholder} may not be efficiently computable, we embark on a search for a function that \emph{can} be evaluated efficiently. The next theorem presents such a Burkholder function. The proof rests on Lieb's Concavity Theorem \citep{lieb1973convex}, which states that for any fixed $A\in\sym^{d}$, the function $X \mapsto\Tr\,\exp\prn*{A + \log{}X}$ is concave over $\sym^{d}_{++}$.
  

  \begin{theorem}
    \label{thm:matrix_burkholder}
    Define $\burk:\reals\times \sym^{d_1+d_2}\times \sym^{d_1+d_2}_{+}\to{}\bbR$ via
    \[
      \burk(a, H, M) = a+ \frac{r}{\eta}\log\,\Tr\,\exp\prn*{\eta{}H - {\textstyle\frac{1}{2}}\eta^{2}L^{2}M} - \frac{c}{\eta}.
    \]
    Then $\burk$ is a Burkholder function, for the pair $(\suff, V)$ in \pref{eq:matrix_v} when $c\geq{}r\log(d_1 + d_2)$.

  \end{theorem}
  This Burkholder function construction immediately implies both existence of a prediction strategy (via \pref{lem:universal_algo}) and that a probabilistic inequality for matrix-values martingales holds. We will present both in detail. The matrix prediction strategy granted by the Burkholder algorithm is particularly simple due to extra convexity properties of $\burk$; see \pref{app:efficient}.
  
  \begin{corollary}[Matrix prediction algorithm]
    \label{corr:matrix_strategy}
    Suppose that $\cY=\brk*{-B, B}$ for some $B>0$. Then the deterministic strategy
    \begin{equation}
      \label{eq:matrix_pred}      
      \pred_{t} = \mathrm{proj}_{\brk*{-B, B}}\prn*{-\frac{r}{L\eta}\En_{\sigma\in\pmo}\brk*{
        \sigma\log\,\Tr\,\exp\,\prn*{\eta\sigma{}L\cH(X_t) + \eta{}\sum_{s=1}^{t-1}\dl_{s}\cH(X_s) - {\textstyle\frac{1}{2}}\eta^{2}L^{2}\sum_{s=1}^{t}\cM(X_s)}
        }}
      \end{equation}
       leads to a regret bound of
      \[
        \sum_{t=1}^{n}\loss(\pred_t, y_t) - \inf_{W\in\cW}\sum_{t=1}^{n}\loss(\tri*{W,X_t}, y_t) \leq{}
        {\textstyle\frac{1}{2}}\eta{}L^2 r\nrm*{\sum_{t=1}^{n}\cM(X_t)}_{\sigma} + \frac{r\log(d_1 + d_2)}{\eta}.
      \]
  \end{corollary}

  Since this regret bound is monotonically increasing with time, it is easy to tune $\eta$ to obtain a fully adaptive strategy.
  \begin{proposition}
    Let $R=\max_{t}\nrm*{X_t}_{\sigma}$ be known. By tuning $\eta$ through the standard doubling trick, we arrive at a regret bound of  
    \begin{align*}
      &\sum_{t=1}^{n}\loss(\pred_t, y_t) - \inf_{W\in\cW}\sum_{t=1}^{n}\loss(\tri*{W,X_t}, y_t) 
    \\ &\leq{}
      O\prn*{r\sqrt{\max\crl*{\nrm*{\sum_{t=1}^{n}X_{t}X_{t}^{\trn}}_{\sigma}, \nrm*{\sum_{t=1}^{n}X_{t}^{\trn}X_{t}}_{\sigma}}\log(d_1+d_2)} + Rr\log(n)}.
    \end{align*}
  \end{proposition}
	
	Let us briefly discuss the result. First, the computation in \pref{eq:matrix_pred} involves an SVD, and does not scale with $t$ since the method only keeps in memory the cumulative statistics. The regret bound gives a \emph{sequence-optimal} rate for the problem of \emph{Online Matrix Completion}, where each $X_t$ is an indicator $e_{i_t}e_{j_t}^{\top}$ corresponding to---for example---a user-movie pair for which the learner must predict a score. Here the regret bound obtained by \pref{eq:matrix_pred} interpolates between the worst-case configuration of the entries $(i_t,j_t)$ and ``spread-out'' (e.g. uniform) sampling of the entries. The result improves on \citep{foster2017zigzag}, which showed that this type of bound is possible by invoking the UMD inequality for Schatten norms but did not provide an efficient algorithm. See that paper for further discussion of the setting and problem.
	

  We now deliver on the second promise, namely a probabilistic martingale inequality. This inequality is stated for $\bbR^{d_1+d_2}$-valued Paley-Walsh martingale difference sequences $(\eps_t\mb{X}_t(\eps))_{t\leq{}n}$, where each $\mb{X}_{t}(\eps) = \mb{X}_{t}(\eps_1,\ldots,\eps_{t-1})$ is predictable with respect to $\cF_{t-1}=\sigma(\eps_{1},\ldots,\eps_{t-1})$ for Rademacher random variables $\eps_{1},\ldots,\eps_{n}$.

  \begin{corollary}[Martingale Matrix Square Function Inequality]
    \label{corr:matrix_square}
    For all Paley-Walsh martingale difference sequences $(\eps_t\mb{X}_{t}(\eps))_{t\leq{}n}$ it holds that
    \begin{equation}
      \label{eq:matrix_square}
    \En_{\eps}\nrm*{\sum_{t=1}^{n}\eps_t\mb{X}_{t}(\eps)}_{\sigma}
    \leq{} \sqrt{2\En_{\eps}\max\crl*{\nrm*{\sum_{t=1}^{n}\mb{X}_{t}(\eps)\mb{X}_{t}(\eps)^{\trn}}_{\sigma}, \nrm*{\sum_{t=1}^{n}\mb{X}_{t}(\eps)^{\trn}\mb{X}_{t}(\eps)}_{\sigma}}\log(d_1+d_2)}.
    \end{equation}
  \end{corollary}
    In the special case where $\mb{X}_t(\eps)=X_t$ is a fixed sequence, this square function inequality \pref{eq:matrix_square} recovers the Matrix Khintchine inequality \citep{mackey2014matrix}, including constants.  A similar martingale inequality can be obtained from the Matrix Freedman/Bennett inequalities of \cite{tropp2011freedman}, but this will depend on almost sure bounds on spectral norms of $(\mb{X}_{t}(\eps))_{t\leq{}n}$.

%%% Local Variables:
%%% mode: latex
%%% TeX-master: "paper"
%%% End:


\section{Further Examples}
\label{sec:further}
% !TEX root = paper.tex

\subsection{ZigZag Algorithm and the UMD Property}

\cite{Pisier75} used martingale techniques to provide a characterization of super-reflexive Banach spaces as those admitting an equivalent uniformly convex norm. As already described in \pref{ex:smoothness}, the essential ingredient of this analysis is a construction of a function $\burk$ with the desired restricted concavity property (which turns out to be equivalent to uniform smoothness) for the martingale inequality \pref{eq:smoothness_martingale_ineq}. The corresponding notion in the world of online learning is that of an adaptive gradient (or mirror) descent. 

\cite{burkholder1981geometrical} provided a geometrical characterization of UMD spaces, and a key ingredient of the approach was to establish existence of (and sometimes to compute in closed form) the function $\burk$ with corresponding geometric properties ($\zeta$-convexity, which is equivalent to  ``zigzag concavity'' \citep{osekowski2012sharp}). As shown in \citep{foster2017zigzag}, in the online learning world the corresponding adaptive regret bound is that of empirical Rademacher averages:
$$\sum_{t=1}^n\loss(\pred_t,y_t) - \min_{\norm{w}\leq 1}\sum_{t=1}^n \loss(\inner{w,x_t},y_t) - C\En\norm{\sum_{t=1}^n \epsilon_t \delta_t x_t}.$$
By linearizing the loss, it suffices to use the sufficient statistic $\suff(x_t,\pred_t,\delta_t)= (\delta_t\pred_t, \delta_t x_t, \epsilon_t x_t)$ where $(\epsilon_t)$ is taken to be a sequence drawn by the algorithm.
The corresponding martingale inequality is
\begin{align}
	\label{eq:umd_martingale_ineq}
	\En\left[ \norm{\sum_{t=1}^n \epsilon_t\bx_t(\epsilon)}^p - C\norm{\sum_{t=1}^n \epsilon_t'\bx_t(\epsilon)}^p\right] \leq 0,
\end{align}
where the process in the subtracted term is decoupled and $p>1$ is arbitrary. We refer the reader to \citep{foster2017zigzag} for more details. 

We would like to emphasize that both smoothness/strong convexity (as in Pisier's work) and the UMD property (as in Burkholder's work) are two distinct notions with distinct sets of sufficient statistics. Since the fundamental works of Pisier and Burkholder, the so-called ``Burkholder method'' has been employed to prove a wide range of martingale inequalities and discover the corresponding geometric properties of the special function \citep{osekowski2012sharp,hytonen2016analysis}. The goal of this paper is to present a unifying approach for working with arbitrary sufficient statistics in online learning, and to show that the Burkholder approach is in fact \emph{algorithmic}. 


\subsection{AdaGrad and Square Function Inequalities}
\label{sec:adagrad}

The Burkholder method can be used to recover efficient algorithms that obtain regret bounds in the vein of diagonal AdaGrad and full-matrix AdaGrad \citep{duchi2011adaptive}, with optimal constants. We thank Adam Os\k{e}kowski for suggesting this example to us \citep{osekowski2017personal}.

Define a function $\burk_{\textrm{square}}(x, y):\bbR^{d}\times{}\bbR_{+}\to\bbR$  \citep{osekowski2005two,osekowski2012sharp} via
\[
\burk_{\textrm{square}}(x, y) = \left\{
\begin{array}{ll}
-\sqrt{2y^{2} - \nrm*{x}_{2}^{2}},\quad& y\geq{}\nrm*{x}_{2}. \\
\nrm*{x}_{2}-2y,\quad& y<\nrm*{x}_{2}.
\end{array}
\right.
\]
$\burk_{\textrm{square}}$ satisfies three properties as in \pref{lem:equivalence_burkholder}: \textbf{1.} $\burk_{\textrm{square}}(x,y)\geq{}\nrm*{x}_{2}-2y$, \textbf{2.} $\burk_{\textrm{square}}(x,\nrm*{x}_{2})\leq{}0$, and \textbf{3.} $\burk_{\textrm{square}}(x+d,\sqrt{y^2 + \nrm*{d}_{2}^{2}})\leq{} \burk_{\textrm{square}}(x,y) + \tri*{\partial_{x}\burk_{\textrm{square}}(x,y), d}$. This function consequently leads to two algorithms in the style of AdaGrad \citep{duchi2011adaptive} but with optimal constants, and which we now sketch.

The first regret bound is for $\ls_{2}$ classes, as in full-matrix AdaGrad, and has the form
\[
\sum_{t=1}^n\loss(\pred_t,y_t) - \min_{\norm{w}_{2}\leq 1}\sum_{t=1}^n \loss(\inner{w,x_t},y_t) - 2L\sqrt{\sum_{t=1}^{n}\nrm*{x_t}^{2}_{2}}\leq{}0.
\]

The associated martingale inequality is 
$
\En\norm{\sum_{t=1}^n \epsilon_t\bx_t(\epsilon)}_{2} \leq{} 2\En\sqrt{\sum_{t=1}^{n}\nrm*{\bx_{t}(\eps)}^{2}_{2}},
$
which was shown to be optimal in \cite{osekowski2005two}.\footnote{Note that the expectation is outside the square root, so this is stronger than the ubiquitous inequality $\En\norm{\sum_{t=1}^n \epsilon_t\bx_t(\epsilon)}_{2} \leq{} \sqrt{\En\sum_{t=1}^{n}\nrm*{\bx_{t}(\eps)}^{2}_{2}}$.} The second regret bound is for $\ls_{\infty}$ classes, as in diagonal AdaGrad, and has the form
\[
\sum_{t=1}^n\loss(\pred_t,y_t) - \min_{\norm{w}_{\infty}\leq 1}\sum_{t=1}^n \loss(\inner{w,x_t},y_t) - 2L\nrm*{\prn*{\sum_{t=1}^{n}x_{t}^{2}}^{1/2}}_{1}\leq{}0,
\]
where $x_{t}^{2}$ denotes the element-wise square. This is obtained by applying the scalar version of $\burk_{\textrm{square}}$ coordinate-wise. The associated martingale inequality is 
$
\En\norm{\sum_{t=1}^n \epsilon_t\bx_t(\epsilon)}_{1} \leq{} 2\En\nrm*{\prn*{\sum_{t=1}^{n}\bx_{t}(\eps)^{2}}^{1/2}}_{1}.
$
Both regret bounds require no prior knowledge of the range of $(x_t)_{t\leq{}n}$.

\subsection{Strongly Convex Losses}
\label{sec:square_loss}
In this section we take $\cF=\crl*{x\mapsto{}\tri*{w,x}\mid{}w\in\bbR^{d}}$ and equip this space with a regularizer $\Phi(w) = \frac{1}{2}\nrm*{w}_{2}^{2}$. We assume that the loss $\ls(\yh, y)$ is $\rho$-strongly convex and $L$-Lipschitz. We adopt the shorthand $z_t=(x_t, -\yh_t)$, and our goal is to obtain a data- and comparator- dependent regret bound of the form

\[
\cA_{\lambda}(w; z_{1},\ldots,z_{n}) = \Phi((w,1)) + c\log\,\det\prn*{\rho{}\sum_{t=1}^{n}z_tz_t^{\trn} + \lambda{}I} - c\log\,\det(\lambda{}I).
\]
for some $c>0$. Here we recover the classical Vovk-Azoury-Warmuth-type bound for strongly convex losses \citep{Vovk98,AzouryWarmuth01}. This example is important because it shows that the Burkholder method in full generality can both obtain fast rates for curved losses and obtain bounds that jointly depend on the comparator and data; the UMD-type Burkholder functions used in \cite{foster2017zigzag} do not obtain such results. The right sufficient statistic for this problem should be familiar: In addition to storing a sum of gradients, we also store the empirical covariance $\sum_{t=1}^{n}z_tz_t^{\trn}$. We introduce one last piece of notation: For $A\succeq{}0$, $\Psi_{A}(w)=\frac{1}{2}\tri*{w,Aw}$.

\begin{proposition}
  \label{prop:square_loss_sufficient}
   The sufficient statistic $\suff(x_t,\pred_t,\delta_t)= \left( \delta_{t}z_t, z_tz_t^{\trn} \right)\in\bbR^{d+1}\times{}\sym^{d+1}_{+}$ and
\begin{equation}
\label{eq:vovk_azoury_warmuth_v}
V(x, A) = \Psi^{\star}_{\rho{}A + \lambda{}I}\prn*{x} - c\log\prn*{\det(\rho{}A + \lambda{}I)/\det(\lambda{}I)}
\end{equation}
forms a sufficient statistic pair for the adaptive regret bound $\cA_{\lambda}$.
\end{proposition}

\begin{theorem}
  \label{thm:square_loss_burkholder}
      For the sufficient statistic pair $(\suff, V)$ in \pref{prop:square_loss_sufficient}, $\burk=V$ is a Burkholder function whenever $c\geq{}L^{2}/\rho$. 
\end{theorem}
Note that for this setting the natural choice for $V$ turned out to be a Burkholder function itself.





\section*{Discussion}
Due to space constraints the following additional results have been deferred to the appendix: Discussion of further directions (\pref{app:discussion}), algorithms for parameter-free online learning in Banach spaces (\pref{app:linear_loss}), and necessary conditions for existence of Burkholder functions (\pref{app:necessary}).

\section*{Acknowledgements}
We thank Adam Os\k{e}kowski  for helpful discussions and for suggesting the example in \pref{sec:adagrad}.  DF is supported in part by the NDSEG fellowship. Research was supported in part by the NSF under grants no. CDS\&E-MSS 1521529 and 1521544. KS additionally acknowledges support from NSF CAREER Award 1750575 and a Sloan Research Fellowship.

\bibliography{refs}

\appendix

\section{Further Directions}
\label{app:discussion}
% !TEX root = paper.tex

The core techniques developed in this paper suggest a number of promising future directions and natural extensions.

\paragraph{Finding sufficient statistics} This paper gives multiple examples of Burkholder function constructions and sufficient statistics. If one wishes to find sufficient statistics for an adaptive bound $\cA$ of interest, a basic rule of thumb is to consider a single input instance (instead of all $n$ data points) and determine---say---a polynomial expansion or expansion in another basis for the terms in $\mathrm{Reg}_{n}-\cA$ involving the instance. This gives a coarse sketch of which statistics are necessary. 

As an example, take the standard square loss with linear predictors as the benchmark class and suppose we are interested in a non-adaptive bound. Following the heuristic above, we need to find an expansion for terms of the form ``$(\hat{y} - y)^2 - (\tri*{w,x} - y)^2 - ~\mathsf{constant}$''. Expanding this expression out, we find that $\hat{y}^2$, $y \cdot x$ and $x x^\top$ are all required to write the expression explicitly. In fact, for this square loss example, the weighted sum of the $x_t$s and the sum of the outer products $\sum_t x_t x_t^\top$ turn out to be sufficient statistics as well.  

For the examples in this paper, we exclusively considered benchmark classes $\F$ that were linear, which appears to have made the search for sufficient statistics easier. However, even when one considers a class $\F$ of non-linear functions, the approach of trying to expand the desired regret inequality (which now involves nonlinear $f \in \F$) around a given instance $x$ in terms of some basis may still help to obtain an adequate sufficient statistics. Furthermore, one may enlarge the class $\F$ to make the sufficient statistic search easier. For instance, if we want to learn the class of boolean decision trees of depth $d$, we can exploit that the class can be represented by polynomials of degree $d$ by using the discrete Fourier coefficients of the input instances up to degree $d$ as a sufficient statistic. In summary, for non-linear classes one may still search for sufficient statistics and Burkholder functions by expressing nonlinearities (approximately) via linear combinations of higher-order terms. 

\paragraph{Toward plug-and-play online learning}
A natural next step is to automatize the search for sufficient statistics and \Bfun functions. Suppose that the sufficient statistic pair $(\suff, V)$ is fixed and all that remains is to find a Burkholder function $\burk$. If $V$ can be written as a polynomial of degree over sufficient statistic space $\cT$, a natural approach is to restrict the search to Burkholder functions $\burk$ that are themselves polynomials and relax the inequalities \propone/\proptwo/\propthree{} to sum-of-squares constraints \citep{barak2014sum}. We can then jointly search for a function $\burk$ and a degree-$d$ sum-of-squares proof that this function satisfies the three properties in polynomial time once the degree of $\burk$ is fixed. As a specific example, the problem of finding the zig-zag concave Burkholder function for $\ell_p$ norms explored in \cite{foster2017zigzag} has a sufficient statistic $V$ that is a polynomial of degree $p$ when $p\geq{}2$ is an integer. 

This approach is sound in that it will never incorrectly return a function $\burk$ that does not satisfy the three properties, but may not be complete a-priori. An interesting direction is therefore to explore whether there are conditions under which this system can indeed be made complete.

\paragraph{Generalized/non-additive sufficient statistics} The restriction in \pref{def:sufficiency} that sufficient statistics combine additively can be relaxed. A more general form is as follows. First, define a \emph{representation space} $\cT$. The function $\suff$ now takes the form:
\[
  \suff: \cX\times{}\cY\times{}\brk*{-L, L}\times{}\cT \to \cT.
\]
The restricted concavity condition for $\burk$ under this definition becomes
\[
\forall{}z, \tau:\quad\sup_{\En\brk*{\alpha}=0}\En_{\alpha}\burk\prn*{\suff\prn*{z, \alpha, \tau}} \leq{} \burk(\tau).
\]
Properties \propone{} and \proptwo{} of \pref{lem:equivalence_burkholder} remain the same. This generalized notion of a sufficient statistic allows us to move beyond additive updates---$\suff$ can multiply $z$ with elements of $\cT$, for example---but still restricts storage to the space $\cT$ and is fully compatible with the Burkholder method and general algorithm framework. The generalizations of the equivalence theorem (\pref{lem:equivalence_burkholder}) and the Burkholder algorithm (\pref{lem:universal_algo}) for this notion of sufficient statistic hold as well.

\section{Fast and Easy Parameter-Free Online Learning}
\label{app:linear_loss}
% !TEX root = paper.tex

So far all of our examples have concerned adaptive bounds $\cA$ that adapt to the data sequence $x_{1},\ldots,x_{n}$, not the comparator $f$. In this section we will show that the framework of Burkholder functions and sufficient statistics readily encompasses comparator-dependent norms by giving a new family of algorithms for the problem of \emph{parameter-free online learning} \citep{mcmahan2014unconstrained}. The setup is as follows: We equip the subset $\cX\subseteq{}\bbR^{d}$ with a norm $\nrm*{\cdot}$ and assume that $\nrm*{x_{t}}\leq{}1$ for all $t$.\footnote{The result extends verbatim to the general Banach space case; this is only to simplify presentation.} Recall that $\norm{\cdot}_{\star}$ will denote the dual norm. Rather than constraining the benchmark class to a compact set, we set $\cW=\bbR^{d}$ and set $\cF=\crl*{x\mapsto\tri*{w,x}\mid{}w\in\cW}$. We assume smoothness of the norm: letting $\Psi(x) = \frac{1}{2}\nrm*{x}^{2}$, it holds that\footnote{Our analysis extends to the general case where we instead have  $\frac{1}{2}\nrm*{x}^{2}\leq{}\Psi(x)$ for some $\Psi\neq{}\frac{1}{2}\nrm*{\cdot}^{2}$ and the same smoothness inequality holds, which is needed for settings such as $\ls_1$/$\ls_{\infty}$.}
  \[
    \Psi(x+y) \leq{} \Psi(x) + \tri*{\grad{}\Psi(x), y} + \frac{\beta}{2}\nrm*{y}^{2}.
  \]


To ease notational burden, we will assume the loss is $1$-Lipschitz in this section. We will efficiently obtain a regret bound of the form
\begin{equation}
  \label{eq:regret_param}
\reg(w) \leq{} \cA(w) \ldef \nrm*{w}_{\star}\sqrt{2\beta{}n\log\prn*{\sqrt{\beta}n\nrm*{w}_{\star} + 1}} + 1 \quad\forall{}w\in\bbR^{d}
\end{equation}
for any such smooth norm. We begin by stating a sufficient statistic representation for the problem. This is based on a familiar potential which has appeared in previous works on parameter-free online learning (e.g. \citep{mcmahan2014unconstrained}) in Hilbert spaces; we extend it to any smooth norm, then use it in the Burkholder method to provide \emph{the first linear time/linear space algorithm for parameter-free learning with general smooth norms in online supervised learning}.\footnote{Since the original submission of this paper, the independent work of \citep{cutkosky2018blackbox} has provided an algorithm with a similar regret guarantee and computational efficiency.}

\begin{proposition}
  \label{prop:param_sufficient} Suppose we are interested in an adaptive regret bound of
\[
\cA(w) = \nrm*{w}_{\star}\sqrt{2an\log\prn*{\frac{\sqrt{an}\nrm*{w}_{\star}}{\gamma} + 1}} + c
\]
for constants $a,\gamma,c>0$. Then $\suff(x_t,\pred_t,\delta_t) = \left( \delta_t\cdot\pred_t, \delta_t\cdot x_t\right)\in\reals\times \X$ and the function
\begin{equation}
\label{eq:param_v}
V(b, x) = b + \gamma\exp\prn*{\frac{\nrm*{x}^{2}}{2an}} - c,
\end{equation}
yield a sufficient statistic pair for the regret bound $\cA$.

\end{proposition}

Because the regret bound we provide is not horizon independent unlike previous examples, it will be convenient to allow time-indexed Burkholder functions $(\burk_{t})_{t\leq{}n}$. This indexing is of purely notational convenience, as time-dependent Burkholder functions fit squarely into the algorithmic framework of \pref{lem:universal_algo} by enlarging $\cX$ to $\cX\times{}\brk*{n}$. Nonetheless, we recap the analogous properties for time-dependent Burkholder functions in the proof of the following theorem.

\begin{theorem}
  \label{thm:param_free}
  Suppose $c=1$, $a=\beta$, and $\gamma=1/\sqrt{n}$ in \pref{eq:param_v}. Then
  \[
    \burk_{t}(b, x) \ldef b + \frac{1}{\sqrt{n}}\exp\prn*{\frac{\nrm*{x}^{2}}{2\beta{}t} + \frac{1}{2}\sum_{s=t+1}^{n}\frac{1}{s}} - 1, 
  \]
  is a family of time-varying Burkholder functions satisfying $1^o$, $2^o$, and $3'$.
\end{theorem}

This Burkholder function immediately yields both a prediction strategy achieving \pref{eq:regret_param} and a simple probabilistic martingale inequality. We will now state them both. Because $(\burk_t)_{t\leq{}n}$ satisfy additional convexity properties, the strategy is especially efficient (per \pref{app:efficient} and \pref{lem:det_strat3}).

\begin{corollary}
Suppose that $\cY=\brk*{-B, B}$ for some $B>0$. Then the deterministic prediction strategy
  \[
    \yh_{t} = \mathrm{proj}_{\brk*{-B, B}}\prn*{-\frac{1}{\sqrt{n}}\En_{\sigma\in\pmo}\brk*{\sigma\cdot\exp\prn*{\frac{\nrm*{\sum_{s=1}^{t-1}\dl_{s}x_s + \sigma{}x_t}^{2}}{2\beta{}t} + \frac{1}{2}\sum_{s=t+1}^{n}\frac{1}{s}}}}
  \]
  leads to a regret bound of
  \[
    \sum_{t=1}^{n}\ls(\yh_t, y_t) - \sum_{t=1}^{n}\ls(\tri*{w,x_t}, y_t) \leq{} \nrm*{w}_{\star}\sqrt{2\beta{}n\log\prn*{\sqrt{\beta}n\nrm*{w}_{\star} + 1}} + 1 \quad\forall{}w\in\bbR^{d}.
  \]
\end{corollary}

The Burkholder function family stated above and \pref{lem:equivalence_burkholder} certify that $\sup\En\brk*{V}\leq{}0$. One special case of this martingale inequality is the following mgf bound for vector-valued martingales under smooth norms.

\begin{corollary}
  Let $\bx_{t}(\eps) \ldef \bx_{t}(\eps_1,\ldots,\eps_{t-1})$ be adapted to the filtration $\cF_{t-1}=\sigma(\eps_{1},\ldots,\eps_{t-1})$ for Rademacher random variables $\eps_{1},\ldots,\eps_{n}$, and let $\nrm*{\bx_t}\leq{}1$ almost surely, where $\nrm*{\cdot}$ is a $\beta$-smooth norm. Then it holds that
  \[
    \En_{\eps}\exp\prn*{\frac{\nrm*{\sum_{t=1}^{n}\eps_{t}\bx_{t}(\eps)}^{2}}{2\beta{}n}} \leq{} \sqrt{n}.
  \]

\end{corollary}

\noindent\textbf{Related work~~} Parameter-free online learning is a very active area of research, but essentially all results in this area that we are aware of \citep{mcmahan2013minimax,mcmahan2014unconstrained,orabona2014simultaneous,orabona2016coin,cutkosky2016online, cutkosky2017online} only provide regret bounds of the form \pref{eq:regret_param} in the special case where $\nrm*{\cdot}$ is a Hilbert space. The only exception is \citep{foster2017parameter} which gives an algorithm for smooth norms $\nrm*{\cdot}$, but has time $\mathrm{poly}(n)$ per step.
Our Burkholder-based algorithm has running time $O(d)$ per step and only $O(d)$ memory.\footnote{Technically our algorithm only applies to the online supervised learning setting, whereas the algorithm of \cite{foster2017parameter} applies to the OCO setting.} The key ingredient to achieving this improvement was to examine a known potential through the lens of the Burkholder method. We hope that this approach can lead to similarly useful improvements by applying the Burkholder method to construct more sophisticated potentials as in, e.g. \citep{orabona2016coin, cutkosky2017online}, particularly to achieve regret bounds that adapt jointly to the model and to data.

%%% Local Variables:
%%% mode: latex
%%% TeX-master: "paper"
%%% End:


\section{Necessary Conditions}
\label{app:necessary}
% !TEX root = paper.tex

We now state a simple, yet powerful result that characterizes when existence of a Burkholder function for a sufficient statistic representation pair $(\suff,V)$ is not only sufficient, but \emph{necessary} to obtain a particular regret bound.
\begin{proposition}
\label{prop:lb}
Let $\delta=(\delta_1,\ldots,\delta_n)$ be a $[-L,L]$-valued martingale difference sequence over filtration $\F_{t-1}=\sigma(\delta_1,\ldots,\delta_{t-1})$ and let $\bz=(\bz_1,\ldots,\bz_n)$ be a sequence of functions $\bz_t: [-L,L]^{t-1} \to \X \times \Y$, each viewed as a predictable process with respect to $\F_{t-1}$. Suppose for every such $(\delta, \bz)$ pair there exists a randomized adversary strategy $(x_t, y_t)$ that guarantees, for every learner strategy $(\yh_t)_{t\leq{}n}$,
\begin{equation}
\label{eq:lb}
\En\sup_{f\in\cF}\brk*{\sum_{t=1}^n\loss(\pred_t,y_t)-\loss(f(x_t),y_t) - \cA(f; x_{1},\ldots,x_{n})}
\geq{} \En\left[  V\left(\sum_{t=1}^n \suff(\bz_t, \delta_t) \right) \right].
\end{equation}
Then, if there exists a strategy $(\yh_t)_{t\leq{}n}$ that achieves the regret bound $\cA(f;\xr[n])$, this implies that
\[
\sup_{\delta, \mb{z}}\En\left[  V\left(\sum_{t=1}^n \suff(\bz_t, \delta_t) \right) \right]\leq{}0.\footnote{In the more general case, if \pref{eq:lb} holds up to additive slack $\Delta$, the corresponding condition is $\sup\En\brk*{V}\leq{}\Delta$.}
\]
Consequently, the regret bound $\cA(f;\xr[n])$ is achievable only if there exists a Burkholder function $\burk:\cT\to\bbR$ that satisfies properties \propone/\proptwo/\propthree of \pref{lem:equivalence_burkholder}. 

When $\alpha \mapsto V(\tau+\suff(z,\alpha))$ is convex for any $z\in\X\times\Y,\tau\in\T$, we only require the preceeding inequalities to hold for $\delta_t=\epsilon_t \cdot L$, $\forall{}t=1,\ldots,n$, where $\epsilon_t$s are independent Rademacher random variables. In this case achievability of the regret bound $\cA(f;\xr[n])$ only implies existence of a Burkholder function $\burk$ satisfying property \propthreep{}, not \propthree{}.
\end{proposition}
\paragraph{Linear Classes}
At first glance the conditions of \pref{prop:lb} may seem fairly restrictive, but it is fairly straightforward to instantiate for all the examples in this paper. Consider the following linear setting: Take $\cX\subseteq{}\bbR^{d}$, $\cY$ arbitrary, and let $\cF$ be a linear class of the form $\crl*{x\mapsto{}\tri*{w,x}\mid{}w\in\cW}$, where $\sup_{x\in\cX,w\in\cW}\tri*{w,x}\leq{}1$ and $\cW$ is symmetric. Pick an arbitrary vector space $\overline{\cT}$, let $\overline{\suff}:\cX\to\overline{\cT}$ be an any featurization of the input space, and let $F:\overline{\cT}\to\bbR$ be an arbitrary function. Our goal will be to achieve a regret bound of the form 
\begin{equation}
\label{eq:suff_example}
\sum_{t=1}^{n}\ls(\yh_t, y_t) - \inf_{f\in\cF}\sum_{t=1}^n \loss(f(x_t),y_t)\leq{} \cA(\xr[n]) \ldef F\prn*{\sum_{t=1}^{n}\overline{\suff}(x_t)}.
\end{equation}
Let us first consider a natural choice of $V$ for the upper bound in this setting. Linearizing and using symmetry of $\cW$, we have
\[
\sum_{t=1}^{n}\ls(\yh_t, y_t) - \inf_{f\in\cF}\sum_{t=1}^n \loss(f(x_t),y_t) - \cA(\xr[n])
\leq{} \sum_{t=1}^{n}\yh_t\cdot\dl_t + \sup_{w\in\cW}\tri*{w,\sum_{t=1}^{n}\dl_{t}x_t} - F\prn*{\sum_{t=1}^{n}\overline{\suff}(x_t)}.
\]
This means that if we choose a sufficient statistic $\suff: (x_t, \yh_t, \dl_{t})\mapsto{} (\yh_t\dl_t, x_t\dl_t, \overline{\suff}(x_t))\in{}\bbR\times{}\bbR^{d}\times{}\overline{\cT}$ and choose $V(a, x, \overline{\tau})=a + \sup_{w\in\cW}\tri*{w,x} - F(\overline{\tau})$, then it holds that
\[
\sum_{t=1}^{n}\ls(\yh_t, y_t) - \inf_{f\in\cF}\sum_{t=1}^n \loss(f(x_t),y_t) - \cA(\xr[n]) \leq{} V\prn*{\sum_{t=1}^{n}\suff(x_t, \yh_t, \dl_t)}.
\]
Noting that $\alpha\mapsto{}V(\tau + \suff(x, \yh, \alpha))$ is convex, \pref{lem:suff_to_martingale} implies that a sufficient condition to achieve the regret bound for any convex $1$-Lipschitz loss is that
\begin{equation}
\label{eq:v_regret}
\sup_{\mb{z}}\En_{\eps}\brk*{V\prn*{\sum_{t=1}^{n}\suff(\mb{z}_t, \eps_t)}}\leq{}0,
\end{equation}
where $\mb{z}$ is any $\cX\times{}\cY$-valued predictable process with respect to the Rademacher sequence $\eps_{1},\ldots,\eps_{n}$. 

By specializing to the absolute loss $\ls(\yh, y)=\abs*{\yh-y}$ and choosing an adversary that plays $y_{t}$ to be Rademacher random variables and $x_{t}$ to be any predictable sequence, it can be shown that \pref{eq:v_regret} is also \emph{necessary}; this is proven formally in the appendix. As a corollary, we derive the following result.
\begin{proposition}
\label{prop:necessary}
There exists a Burkholder function $\burk$ for the pair $(\suff, V)$ \emph{if and only if} the regret bound \pref{eq:suff_example} is achievable.
\end{proposition}
 Consider the matrix prediction setting of \pref{sec:matrix} for the special case of $L=1$ and $r=1$. This setting fits into the linear class framework above by taking $\cW$ to be the nuclear norm ball in $\bbR^{d_1\times{}d_2}$ and setting $\overline{\suff}(X)=\cM(X)$ for any matrix $X\in\bbR^{d_1\times{}d_2}$. For this setting \pref{prop:necessary} implies the following equivalence.
\begin{example}[Matrix Prediction]
The following are equivalent:
\begin{enumerate}
\item The regret bound
\[
\sum_{t=1}^{n}\ls(\yh_t, y_t) - \inf_{W\;:\;\nrm*{W}_{\Sigma}\leq{}1}\sum_{t=1}^n \loss(\tri*{W,X_t},y_t) \leq{} \frac{\eta{}}{2}\nrm*{\sum_{t=1}^{n}\cM(X_t)}_{\sigma} + \frac{c}{\eta}
\]
is achievable.
\item The martingale inequality
\[
\En_{\eps}\nrm*{\sum_{t=1}^{n}\eps_t\mb{X}_{t}(\eps)}_{\sigma} \leq{} \frac{\eta{}}{2}\En_{\eps}\nrm*{\sum_{t=1}^{n}\cM(\mb{X}_t(\eps))}_{\sigma} + \frac{c}{\eta}
\]
holds for all $\bbR^{d_1\times{}d_2}$-valued predictable processes $\mb{X}$.
\item There exists a Burkholder function for the sufficient statistic pair $(\suff, V)$ in \pref{eq:v_regret}.
\end{enumerate}

\end{example}




\section{Proofs}
\label{app:proofs}

%\section*{Acknowledgement}

%Y. Fei and Y. Chen were partially supported by the National Science
%Foundation CRII award 1657420 and grant 1704828.


\appendix
%\appendixpage

\section{Additional notations}

We define the shorthand $\error\coloneqq\norm[\Yhat-\Ystar]1$. For
a matrix $\M$, we write $\norm[\M]{\infty}\coloneqq\max_{i,j}\left|M_{ij}\right|$
as its entry-wise $\ell_{\infty}$ norm, and $\opnorm{\M}$ as its
spectral norm (maximum singular value).  We let $\I$ and $\OneMat$
be the $\num\times\num$ identity matrix and all-one matrix, respectively.
For a real number $x$, $\left\lceil x\right\rceil $ denotes its
ceiling. We denote by $\clustset a\coloneqq\left\{ i\in\left[\num\right]:\labelstar(i)=a\right\} $
the set of indices of points in cluster $a$, and we define $\size\coloneqq\left|\clustset a\right|=\frac{\num}{\numclust}$. 

\section{Proof of Theorem \ref{thm:ip_sdp}\label{sec:proof_ip_sdp}}

\subsection{Initial steps}

We assume $\error>0$ since otherwise we are done. We can write $\Adj=\C+\C\t-2\H\H\t$,
where $\H$ is a matrix whose $i$-th row is the point $\h_{i}$ and
$\C$ is a matrix where the entries in the $i$-th row are identical
and equal to $\norm[\h_{i}]2^{2}$. Since the row-sum constraint in
the program (\ref{eq:SDP1}) ensures that the matrix $\Yhat-\Ystar$
has zero row sum, we have $\left\langle \Yhat-\Ystar,\C\right\rangle =\left\langle \Yhat-\Ystar,\C\t\right\rangle =0$
which implies $\left\langle \Yhat-\Ystar,\C+\C\t\right\rangle =0$.

Let $\G\coloneqq\H-\E\H$ be a matrix of entries in $\H$ with their
means removed. We can compute
\begin{align*}
\H\H\t & =\left(\G+\E\H\right)\left(\G+\E\H\right)\t\\
 & =\G\G\t+\G\left(\E\H\right)\t+\left(\E\H\right)\G\t+\left(\E\H\right)\left(\E\H\right)\t
\end{align*}
and 
\[
\E\H\H^{\top}=\E\G\G\t+\left(\E\H\right)\left(\E\H\right)\t.
\]
Therefore 
\[
\H\H\t-\E\H\H^{\top}=\left(\G\G\t-\E\G\G\t\right)+\G\left(\E\H\right)\t+\left(\E\H\right)\G\t.
\]
Let $\U\in\real^{\num\times\numclust}$ be the matrix of the left
singular vectors of $\Ystar$. For any $\M\in\real^{\num\times\num}$,
define the projection $\PT\left(\M\right)\coloneqq\U\U\t\M+\M\U\U\t-\U\U\t\M\U\U\t$
and its orthogonal complement $\PTperp\left(\M\right)\coloneqq\M-\PT\left(\M\right)$.
The fact that $\Yhat$ is optimal and $\Ystar$ is feasible to the
program (\ref{eq:SDP1}) implies 
\begin{align*}
0 & \leq-\frac{1}{2}\left\langle \Yhat-\Ystar,\Adj\right\rangle \\
 & =\left\langle \Yhat-\Ystar,\H\H\t-\E\H\H^{\top}\right\rangle +\left\langle \Yhat-\Ystar,\E\H\H^{\top}\right\rangle \\
 & =\left\langle \Yhat-\Ystar,\G\G\t-\E\G\G\t+\G\left(\E\H\right)\t+\left(\E\H\right)\G\t\right\rangle +\left\langle \Yhat-\Ystar,\E\H\H^{\top}\right\rangle \\
 & =\left\langle \Yhat-\Ystar,\PT\left(\G\G\t-\E\G\G\t\right)\right\rangle +\left\langle \Yhat-\Ystar,\PTperp\left(\G\G\t-\E\G\G\t\right)\right\rangle \\
 & \quad+2\left\langle \Yhat-\Ystar,\G\left(\E\H\right)\t\right\rangle +\left\langle \Yhat-\Ystar,\E\H\H^{\top}\right\rangle \\
 & \eqqcolon S_{1}+S_{2}+2S_{3}+S_{4}.
\end{align*}
We may control $S_{1}$, $S_{2}$ and $S_{4}$ using the following. 
\begin{prop}
\label{prop:S1} If $\snr^{2}\geq C\left(\sqrt{\frac{\numclust\vecdim}{\num}\log\left(\num\numclust\right)}+\sqrt{\frac{\numclust}{\num}}\log\left(\num\numclust\right)\right)$
for some universal constant $C>0$, then $S_{1}\leq\frac{1}{100}\minsep^{2}\error$
with probability at least $1-\left(2\num\right)^{-2}$.
\end{prop}

\begin{prop}
\label{prop:S2} If $\snr^{2}\geq C\numclust\left(\sqrt{\frac{\vecdim}{\num}}+1\right)$
for some universal constant $C>0$, then $S_{2}\leq\frac{1}{100}\minsep^{2}\error$
with probability at least $1-2e^{-\num}$.
\end{prop}

\begin{prop}
\label{prop:S4} We have $S_{4}=-\frac{1}{2}\sum_{a\ne b}T_{ab}\minsep_{ab}^{2}\le-\frac{1}{4}\minsep^{2}\error$
where $T_{ab}\coloneqq\sum_{i\in\clustset a,j\in\clustset b}\left(\Yhat-\Ystar\right)_{ij}$. 
\end{prop}
The proofs are given in Sections \ref{sec:proof_S1}, \ref{sec:proof_S2}
and \ref{sec:proof_S4}, respectively. Combining the above propositions,
we have $S_{1}+S_{2}\le-\frac{1}{2}S_{4}$ and therefore 
\begin{equation}
0\leq S_{3}+\frac{1}{4}S_{4}\eqqcolon S_{0}\label{eq:error_S3_bound}
\end{equation}
with probability at least $1-\left(2\num\right)^{-C'}-2e^{-\num}$
for some universal constant $C'>0$.

Let $\B\coloneqq\Yhat-\Ystar$. We have 
\begin{align*}
S_{3} & =\sum_{j}\sum_{a}\sum_{i\in C_{a}}B_{ji}\left\langle \Mean_{a},\g_{j}\right\rangle \\
 & =\size\sum_{j}\sum_{a}\left\langle \Mean_{a},\g_{j}\right\rangle \left(\frac{1}{\size}\sum_{i\in\clustset a}B_{ji}\right)\\
 & =\size\sum_{j}\sum_{a\ne\labelstar(j)}\left\langle \Mean_{a}-\Mean_{\labelstar(j)},\g_{j}\right\rangle \left(\frac{1}{\size}\sum_{i\in\clustset a}B_{ji}\right)
\end{align*}
where the last step holds since $\sum_{a\ne\labelstar(j)}\left(\sum_{i\in\clustset a}B_{ji}\right)=-\sum_{i\in\clustset a:a=\labelstar(j)}B_{ji}$
for each $j\in\left[\num\right]$ which follows from the row-sum constraint
of program (\ref{eq:SDP1}). By Proposition \ref{prop:S4}, we have
\begin{align*}
S_{4} & =-\size\sum_{j}\sum_{a\ne\labelstar(j)}\frac{1}{2}\minsep_{\labelstar(j),a}^{2}\left(\frac{1}{\size}\sum_{i\in\clustset a}B_{ji}\right).
\end{align*}
Therefore, we have 
\[
S_{0}=\size\sum_{j}\sum_{a\ne\labelstar(j)}\left(\left\langle \Mean_{a}-\Mean_{\labelstar(j)},\g_{j}\right\rangle -c\minsep_{\labelstar(j),a}^{2}\right)\left(\frac{1}{\size}\sum_{i\in\clustset a}B_{ji}\right)
\]
where $c=\frac{1}{8}$.

To control $S_{0}$, we define $\beta_{ja}\coloneqq\left\langle \Mean_{a}-\Mean_{\labelstar(j)},\g_{j}\right\rangle -c\minsep_{\labelstar(j),a}^{2}$
and consider the program 
\begin{align}
\max_{\X}\  & \sum_{j}\sum_{a\ne\labelstar(j)}\beta_{ja}X_{ja}\nonumber \\
\text{s.t.}\  & 0\leq X_{ja}\leq1,\qquad\forall a\ne\labelstar(j),j\in\left[\num\right]\nonumber \\
 & \sum_{a\ne\labelstar(j)}X_{ja}\leq1,\qquad\forall j\in\left[\num\right]\label{eq: int_opt}\\
 & \sum_{j}\sum_{a\ne\labelstar(j)}X_{ja}=R,\nonumber 
\end{align}
where $R\in(0,\num]$. Let us denote by $V(R)$ the optimal value
of the above program and we let $V(R)=-\infty$ if the program is
infeasible. The constraints of program (\ref{eq:SDP1}) implies that
$\frac{\error}{2\size}\in(0,\num]$ and 
\[
\sum_{j\in\left[\num\right]}\sum_{a\ne\labelstar(j)}\left(\sum_{i\in\clustset a}B_{ji}\right)=\frac{\error}{2}.
\]
Hence, by Equation (\ref{eq:error_S3_bound}), we have 
\begin{equation}
0\leq S_{0}\leq\size\cdot V\left(\frac{\error}{2\size}\right).\label{eq:basic_ineq_upper_bound_V}
\end{equation}


\subsection{Controlling $\protect\error$ by LP}

We show that $\error$ is upper bounded by the objective value of
an LP that is related to program (\ref{eq: int_opt}). If $\error=0$
then the conclusion of Theorem \ref{thm:ip_sdp} holds trivially.
For $\error>0$, we have the following cases:
\begin{enumerate}
\item If $\frac{\error}{2\size}\in(0,1]$, it follows from Equation (\ref{eq:basic_ineq_upper_bound_V})
that the error $\error$ must satisfy 
\[
0\le V\left(\frac{\error}{2\size}\right)=\beta^{*}\frac{\error}{2\size}\le\beta^{*}\left\lceil \frac{\error}{2\size}\right\rceil =V\left(\left\lceil \frac{\error}{2\size}\right\rceil \right)
\]
where $\beta^{*}\coloneqq\max_{j\in\left[\num\right],a\ne\labelstar(j)}\beta_{ja}$.
This implies 
\begin{align*}
\frac{\error}{2\size}\le\left\lceil \frac{\error}{2\size}\right\rceil  & \le\max\left\{ R\in\{0,1,.\ldots\}:V(R)\ge0\right\} .
\end{align*}
\item If $\frac{\error}{2\size}>1$, it follows from Equation (\ref{eq:basic_ineq_upper_bound_V})
that the error $\error$ must satisfy 
\[
0\le V\left(\frac{\error}{2\size}\right)\le\max\left\{ V\left(\left\lceil \frac{\error}{2\size}\right\rceil \right),V\left(\left\lfloor \frac{\error}{2\size}\right\rfloor \right)\right\} =\max\left\{ V\left(\left\lceil \frac{\error}{2\size}\right\rceil \right),V\left(\left\lceil \frac{\error}{2\size}\right\rceil -1\right)\right\} .
\]
In other words, we have
\begin{align*}
\frac{\error}{2\size}\le\left\lceil \frac{\error}{2\size}\right\rceil  & \le\max\left\{ R\in\{0,1,.\ldots\}:V(R)\vee V(R-1)\ge0\right\} \\
 & =1+\max\left\{ R\in\{0,1,.\ldots\}:V(R)\ge0\right\} .
\end{align*}
Note that $\left\lceil \frac{\error}{2\size}\right\rceil \ge2$, and
therefore we must have $1\le\max\left\{ R\in\{0,1,.\ldots\}:V(R)\ge0\right\} $.
This implies 
\[
\frac{\error}{2\size}\le2\max\left\{ R\in\{0,1,.\ldots\}:V(R)\ge0\right\} .
\]
\end{enumerate}
Consequently, we have 
\[
\frac{\error}{2\size}\le2\max\left\{ R\in\{0,1,.\ldots\}:V(R)\ge0\right\} .
\]


\subsection{Converting LP to IP}

We are now ready to formally establish a connection between the error
of the SDP (\ref{eq:SDP1}) and that of the Oracle IP (\ref{eq:oracleIP}),
by relating $\max\left\{ R\in\{0,1,.\ldots\}:V(R)\ge0\right\} $ to
the quantity (\ref{eq:IPerror}). Note that if $R\ge0$ is an integer,
then there exists an optimal solution $\left\{ w_{ja}\right\} $ of
program (\ref{eq: int_opt}) such that $w_{ja}\in\{0,1\}$ for all
$j\in[\num],a\in[\numclust]$. Therefore, if $R\in\{0,1,\ldots\}$
is an integer, then 
\begin{equation}
V(R)=\IP_{1}(R)\coloneqq\left\{ \begin{aligned}\max_{\X}\  & \sum_{j}\sum_{a\ne\labelstar(j)}\beta_{ja}X_{ja}\\
\text{s.t.}\  & X_{ja}\in\{0,1\},\qquad\forall a\ne\labelstar(j),j\in\left[\num\right]\\
 & \sum_{a\ne\labelstar(j)}X_{ja}\leq1,\qquad\forall j\in\left[\num\right]\\
 & \sum_{j}\sum_{a\ne\labelstar(j)}X_{ja}=R
\end{aligned}
\right\} .\label{eq:IP1}
\end{equation}
Combining the last two display equations we obtain that
\begin{align}
\frac{\error}{2\size} & \le2\max\left\{ R\in\{0,1,.\ldots\}:\IP_{1}(R)\ge0\right\} \nonumber \\
 & \overset{}{=}2\cdot\left\{ \begin{aligned}\max_{R,\X}\; & R\\
\text{s.t.}\; & R\in\{0,1,\ldots\}\\
 & \sum_{j}\sum_{a\ne\labelstar(j)}\beta_{ja}X_{ja}\ge0\\
 & X_{ja}\in\{0,1\},\qquad\forall a\ne\labelstar(j),j\in\left[\num\right]\\
 & \sum_{a\ne\labelstar(j)}X_{ja}\leq1,\qquad\forall j\in\left[\num\right]\\
 & \sum_{j}\sum_{a\ne\labelstar(j)}X_{ja}=R,
\end{aligned}
\right\} \nonumber \\
 & =2\cdot\IP_{2}\coloneqq2\cdot\left\{ \begin{aligned}\max_{\X}\; & \sum_{j}\sum_{a\ne\labelstar(j)}X_{ja}\\
\text{s.t.}\; & \sum_{j}\sum_{a\ne\labelstar(j)}\beta_{ja}X_{ja}\ge0\\
 & X_{ja}\in\{0,1\},\qquad\forall a\ne\labelstar(j),j\in\left[\num\right]\\
 & \sum_{a\ne\labelstar(j)}X_{ja}\leq1,\qquad\forall j\in\left[\num\right]
\end{aligned}
\right\} .\label{eq:error_bound2}
\end{align}

Let us reparameterize the integer program $\IP_{2}$ by a change of
variable. Recall that 
\[
\mathcal{F}\coloneqq\left\{ \F\in\{0,1\}^{\num\times\numclust}:\F\one_{\numclust}=\one_{\num}\right\} 
\]
is the set of all possible assignment matrices and $\F^{*}\in\mathcal{F}$
is the true assignment matrix; that is, $F_{ja}^{*}=\indic\left\{ a=\labelstar(j)\right\} $
for all $j\in[\num],a\in[\numclust]$. Consider any feasible solution
$\X$ of $\IP_{2}$; here for each $j\in[\num]$, we may fix $X_{j,\labelstar(j)}=-\sum_{a\neq\labelstar(j)}X_{ja}$
\textemdash{} doing so does not affect the feasibility and objective
value of $\X$ w.r.t. $\IP_{2}$. Define the new variable $\F\coloneqq\F^{*}+\X\in\mathcal{F}$.
The objective value and constraints of the old variable $\X$ can
be mapped to those of $\F$; in particular, we have 
\begin{align*}
\sum_{j}\sum_{a\ne\labelstar(j)}X_{ja} & =\sum_{j}\sum_{a\ne\labelstar(j)}(F_{ja}-F_{ja}^{*})=\frac{1}{2}\norm[\F-\F^{*}]1
\end{align*}
and
\begin{align*}
\left.\begin{array}{c}
X_{ja}\in\{0,1\},\forall a\ne\labelstar(j),j\in\left[\num\right]\\
\sum_{a\ne\labelstar(j)}X_{ja}\leq1,\forall j\in\left[\num\right]\\
X_{j,\labelstar(j)}=-\sum_{a\neq\labelstar(j)}X_{ja},\forall j\in[\num]
\end{array}\right\}  & \Longleftrightarrow\F\in\mathcal{F}
\end{align*}
and
\[
\sum_{j}\sum_{a\ne\labelstar(j)}\beta_{ja}X_{ja}\overset{(i)}{=}\sum_{j}\sum_{a}\beta_{ja}X_{ja}=\sum_{j}\sum_{a}\beta_{ja}F_{ja}-\sum_{j}\sum_{a}\beta_{ja}F_{ja}^{*}\overset{(ii)}{=}\sum_{j}\sum_{a}\beta_{ja}F_{ja},
\]
where steps $(i)$ and $(ii)$ both follow from the fact that $\beta_{j,\labelstar(j)}=0,\forall j.$
It follows that $\IP_{2}$ has the same optimal value as a corresponding
integer program in terms of $\X$; in particular, we have
\[
\IP_{2}=\IP_{3}\coloneqq\left\{ \begin{aligned}\max_{\F}\; & \frac{1}{2}\norm[\F-\F^{*}]1\\
\text{s.t.}\; & \sum_{j}\sum_{a}\beta_{ja}F_{ja}\ge0\\
 & \F\in\mathcal{F}
\end{aligned}
\right\} .
\]
Combining with equation (\ref{eq:error_bound2}), we see that the
error $\error$ satisfies
\begin{equation}
\frac{\error}{2\size}\le2\cdot\IP_{3}.\label{eq:error_bound3}
\end{equation}

We further simplify the first constraint in $\IP_{3}$. Recall that
$\bar{\h}_{i}\coloneqq\Mean_{\labelstar(i)}+(2c)^{-1}\g_{i}$ for
each $i\in[\num]$. Note that $\left\{ \bar{\h}_{i}\right\} $ can
be viewed as data points generated from the Sub-Gaussian Mixture Model
but with $(2c)^{-1}$ times the standard deviation. By definition
of $\beta_{ja}$, we have 
\begin{align*}
\beta_{ja} & =\left\langle \Mean_{a}-\Mean_{\labelstar(j)},\g_{j}\right\rangle -c\minsep_{\labelstar(j),a}^{2}\\
 & =c\left(2\left\langle \Mean_{a}-\Mean_{\labelstar(j)},(2c)^{-1}\g_{j}\right\rangle -\minsep_{\labelstar(j),a}^{2}\right)\\
 & =c\left(2\left\langle \Mean_{a}-\Mean_{\labelstar(j)},(2c)^{-1}\g_{j}\right\rangle -\norm[\Mean_{a}-\Mean_{\labelstar(j)}]2^{2}\right)\\
 & =c\left(2\left\langle \Mean_{a}-\Mean_{\labelstar(j)},(2c)^{-1}\g_{j}\right\rangle -\norm[\Mean_{a}-\Mean_{\labelstar(j)}]2^{2}-\norm[(2c)^{-1}\g_{j}]2^{2}+\norm[(2c)^{-1}\g_{j}]2^{2}\right)\\
 & =c\left(-\norm[\Mean_{\labelstar(j)}-\Mean_{a}+(2c)^{-1}\g_{j}]2^{2}+\norm[(2c)^{-1}\g_{j}]2^{2}\right)\\
 & =c\left(-\norm[\bar{\h}_{j}-\Mean_{a}]2^{2}+\norm[(2c)^{-1}\g_{j}]2^{2}\right).
\end{align*}
For any $\F\in\mathcal{F}$, we then have
\begin{align*}
\sum_{j}\sum_{a}\beta_{ja}F_{ja} & =c\sum_{j}\sum_{a}\left(-\norm[\bar{\h}_{j}-\Mean_{a}]2^{2}+\norm[(2c)^{-1}\g_{j}]2^{2}\right)F_{ja}\\
 & =c\left(-\sum_{j}\sum_{a}\norm[\bar{\h}_{j}-\Mean_{a}]2^{2}F_{ja}+\sum_{j}\norm[(2c)^{-1}\g_{j}]2^{2}\sum_{a}F_{ja}\right)\\
 & \overset{(i)}{=}c\left(-\sum_{j}\sum_{a}\norm[\bar{\h}_{j}-\Mean_{a}]2^{2}F_{ja}+\sum_{j}\norm[(2c)^{-1}\g_{j}]2^{2}\sum_{a}F_{ja}^{*}\right)\\
 & =c\left(-\sum_{j}\sum_{a}\norm[\bar{\h}_{j}-\Mean_{a}]2^{2}F_{ja}+\sum_{j}\sum_{a}\norm[(2c)^{-1}\g_{j}]2^{2}F_{ja}^{*}\right)\\
 & =c\left(-\sum_{j}\sum_{a}\norm[\bar{\h}_{j}-\Mean_{a}]2^{2}F_{ja}+\sum_{j}\sum_{a}\norm[\bar{\h}_{j}-\Mean_{\labelstar(j)}]2^{2}F_{ja}^{*}\right)\\
 & \overset{(ii)}{=}c\left(-\sum_{j}\sum_{a}\norm[\bar{\h}_{j}-\Mean_{a}]2^{2}F_{ja}+\sum_{j}\sum_{a}\norm[\bar{\h}_{j}-\Mean_{a}]2^{2}F_{ja}^{*}\right),
\end{align*}
where step $(i)$ holds because $\sum_{a}F_{ja}=1=\sum_{a}F_{ja}^{*},\forall j$,
and step $(ii)$ holds because $F_{ja}^{*}=1$ only if $a=\labelstar(j)$.
Again recall the shorthand
\[
\eta(\F)\coloneqq\sum_{j}\sum_{a}\norm[\bar{\h}_{j}-\Mean_{a}]2^{2}F_{ja}.
\]
We have the more compact expression
\begin{equation}
\sum_{j}\sum_{a}\beta_{ja}F_{ja}=c\left(\eta(\F^{*})-\eta(\F)\right)\label{eq:eta_func_equivalence}
\end{equation}
It follows that for any $\F\in\mathcal{F}$, the first constraint
in $\IP_{3}$ is satisfied if and only if 
\[
\eta(\F)\le\eta(\F^{*}).
\]
Combining with the (\ref{eq:error_bound3}), we obtain that
\[
\frac{\error}{2\size}\le2\cdot\IP_{3}=2\cdot\left\{ \begin{aligned}\max_{\F}\; & \frac{1}{2}\norm[\F-\F^{*}]1\\
\text{s.t.}\; & \eta(\F)\le\eta(\F^{*})\\
 & \F\in\mathcal{F}
\end{aligned}
\right\} .
\]
Rearranging terms, we have the bound
\begin{equation}
\error\le2\size\cdot\max\left\{ \norm[\F-\F^{*}]1:\eta(\F)\le\eta(\F^{*}),\F\in\mathcal{F}\right\} .\label{eq:error_bound3a}
\end{equation}
The result follows from the fact that $\norm[\Ystar]1=\num\size$
and $\norm[\F^{*}]1=\num$. 

\subsection{Proof of Proposition \ref{prop:S1} \label{sec:proof_S1}}

In this section we control $S_{1}$. We can further decompose $S_{1}$
as 
\begin{align*}
S_{1} & =\left\langle \Yhat-\Ystar,\U\U\t\left(\G\G\t-\E\G\G\t\right)\right\rangle +\left\langle \Yhat-\Ystar,\left(\G\G\t-\E\G\G\t\right)\U\U\t\right\rangle \\
 & \qquad-\left\langle \Yhat-\Ystar,\U\U\t\left(\G\G\t-\E\G\G\t\right)\U\U\t\right\rangle \\
 & \leq2\left|\left\langle \Yhat-\Ystar,\U\U\t\left(\G\G\t-\E\G\G\t\right)\right\rangle \right|+\left|\left\langle \Yhat-\Ystar,\U\U\t\left(\G\G\t-\E\G\G\t\right)\U\U\t\right\rangle \right|\\
 & \eqqcolon2T_{1}+T_{2}
\end{align*}
By the generalized Holder's inequality, we have 
\begin{align*}
T_{1} & \leq\error\cdot\norm[\U\U\t\left(\G\G\t-\E\G\G\t\right)]{\infty}
\end{align*}
and 
\begin{align*}
T_{2} & =\left|\left\langle \Yhat-\Ystar,\U\U\t\left(\G\G\t-\E\G\G\t\right)\U\U\t\right\rangle \right|\\
 & =\left|\left\langle \left(\Yhat-\Ystar\right)\U\U\t,\U\U\t\left(\G\G\t-\E\G\G\t\right)\right\rangle \right|\\
 & \leq\error\cdot\norm[\U\U\t\left(\G\G\t-\E\G\G\t\right)]{\infty}
\end{align*}
where the last inequality holds since 
\[
\norm[\left(\Yhat-\Ystar\right)\U\U\t]1\leq\norm[\Yhat-\Ystar]1=\error.
\]
Combining the above, we have 
\[
S_{1}\leq3\error\cdot\norm[\U\U\t\left(\G\G\t-\E\G\G\t\right)]{\infty}.
\]

Note that there are $m=\num\numclust$ distinct random variables in
$\U\U\t\left(\G\G\t-\E\G\G\t\right)$ and let us call them $X_{1},\ldots,X_{m}$.
For each $i$, we can see that $X_{i}$ is the average of $\size$
entries in $\G\G\t-\E\G\G\t$ and we let $\B_{i}$ be an $\num\times\num$
matrix with $\size$ entries equal to 1 and the others equal to 0
such that $\size X_{i}=\left\langle \B_{i},\G\G\t-\E\G\G\t\right\rangle $.
To proceed, we need the Hanson-Wright inequality (an extension of
Exercise 6.2.7 on pp.$\ $140 in \citet{vershynin2017high}).
\begin{lem}[Higher-dimensional Hanson-Wright inequality]
\emph{ \label{lem:hanson-wright} }Let $\x_{1},\ldots,\x_{N}$ be
independent, mean zero, sub-Gaussian random vectors in $\real^{M}$.
Let $\B$ be an $N\times N$ matrix. For every $t\geq0$ and some
universal constant $c>0$, we have 
\[
\P\left[\left|\sum_{i,j}B_{ij}\left\langle \x_{i},\x_{j}\right\rangle -\E\sum_{i,j}B_{ij}\left\langle \x_{i},\x_{j}\right\rangle \right|\geq t\right]\leq4\exp\left[-c\min\left(\frac{t^{2}}{K^{4}M\norm[\B]F^{2}},\frac{t}{K^{2}\norm[\B]{}}\right)\right]
\]
where $K\coloneqq\max_{i}\norm[\x_{i}]{\psi_{2}}.$ 
\end{lem}
The proof is given in Section \ref{sec:proof_hanson_wright}. Using
Lemma \ref{lem:hanson-wright}, we see that for any $t\ge0$ 
\[
\P\left\{ \size X_{i}\geq t\right\} =\P\left\{ \left\langle \B_{i},\G\G\t-\E\G\G\t\right\rangle \geq t\right\} \leq4\exp\left[-c\min\left(\frac{t^{2}}{K^{4}\vecdim\size},\frac{t}{K^{2}\sqrt{\size}}\right)\right].
\]
We can choose $t^{*}=DK^{2}\sqrt{\size}\left(\sqrt{\vecdim\log m}+\log m\right)$
with $K=\sgnorm$ and $D>0$ a universal constant. Apply the union
bound, we have 
\[
S_{1}\leq3\error\cdot\frac{1}{\size}\cdot t^{*}
\]
with probability at least $1-m\cdot\P\left\{ \size X\geq t\right\} \geq1-\exp\left(-C'\log m\right)=1-m^{-C'}$
where $C'>0$ is a universal constant. The result follows from the
condition of the proposition.

\subsection{Proof of Proposition \ref{prop:S2} \label{sec:proof_S2}}

In this section we control $S_{2}$. We have 
\begin{align*}
S_{2} & =\left\langle \PTperp\left(\Yhat-\Ystar\right),\G\G\t-\E\G\G\t\right\rangle \\
 & \leq\Tr\left[\PTperp\left(\Yhat-\Ystar\right)\right]\cdot\opnorm{\G\G\t-\E\G\G\t}\\
 & \le\frac{\error}{\size}\cdot\opnorm{\G\G\t-\E\G\G\t}.
\end{align*}
Let $\Var\left(g_{ij}\right)=\std^{2}$. We record a fact about the
sub-Gaussian property of columns of $\G$. 
\begin{fact}
\label{fact:satisfy_cond_gauss_choas_operator_norm_bound} Let $\x\in\real^{\num}$
be an arbitrary column of $\G$. We have 
\[
\norm[\left\langle \x,\w\right\rangle ]{\psi_{2}}\leq C\frac{\sgnorm}{\std}\sqrt{\E\left\langle \x,\w\right\rangle ^{2}}\qquad\text{for any }\w\in\real^{\num},
\]
where $C>0$ is a universal constant and $C\frac{\sgnorm}{\std}\ge1$.
\end{fact}
The proof is given in Section \ref{sec:proof_satisfy_cond}. Applying
Lemma \ref{lem:subg_cov_mat_bound} with $\rho_{0}=\frac{\sgnorm}{\std}$,
we have 
\[
\opnorm{\frac{1}{d}\G\G\t-\frac{1}{d}\E\G\G\t}\leq C_{1}\rho_{0}^{2}\left(\sqrt{\frac{2\num}{\vecdim}}+\frac{2\num}{\vecdim}\right)\opnorm{\frac{1}{d}\E\G\G\t}
\]
with probability at least $1-2e^{-\num}$. Here we let $m=\vecdim,u=\num$
and define $\x_{i}$ to be the $i$-th column of $\G$ and $\x$ to
be a vector independent of but identically distributed as each column of $\G$ (note that columns of $\G$ are identically
distributed). We also use the fact that $\E\x\x\t=\frac{1}{\vecdim}\E\G\G\t=\std^{2}\I$.
Multiplying $\vecdim$ on both sides of the above equation yields
\[
\opnorm{\G\G\t-\E\G\G\t}\leq C_{1}\left(\sqrt{\frac{2\num}{\vecdim}}+\frac{2\num}{\vecdim}\right)\vecdim\sgnorm^{2}.
\]
Hence, we have 
\[
S_{2}\le\frac{\error}{\size}\cdot C_{1}\left(\sqrt{\frac{2\num}{\vecdim}}+\frac{2\num}{\vecdim}\right)\vecdim\sgnorm^{2}=2C_{1}\error\numclust\left(\sqrt{\frac{\vecdim}{\num}}+1\right)\frac{\minsep^{2}}{\snr^{2}}
\]
The result follows from the condition of the proposition.

\subsection{Proof of Proposition \ref{prop:S4} \label{sec:proof_S4}}

We can compute 
\[
\left(\E\H\H\t\right)_{ij}=\begin{cases}
\vecdim\std^{2}+\norm[\Mean_{\labelstar(i)}]2^{2} & \text{if }i=j\\
\norm[\Mean_{\labelstar(i)}]2^{2} & \text{if }i\ne j\text{ and }\labelstar(i)=\labelstar(j)\\
\left\langle \Mean_{\labelstar(i)},\Mean_{\labelstar(j)}\right\rangle  & \text{otherwise}.
\end{cases}
\]
We partition the matrix $\Yhat-\Ystar$ into $\numclust^{2}$ of $\size\times\size$
blocks, and note that $T_{ab}$ denotes the sum of entries within
the $(a,b)$-th block. The constraints of program (\ref{eq:SDP1})
implies that 
\begin{enumerate}
\item $T_{aa}\leq0$ for each $a\in\left[\numclust\right]$ and $T_{ab}\geq0$
for each $a\ne b\in\left[\numclust\right]$;
\item $T_{ab}=T_{ba}$ for each $a,b\in\left[\numclust\right]$;
\item $-T_{aa}=\sum_{b\in\left[\numclust\right]:b\ne a}T_{ab}$ for each
$a\in\left[\numclust\right]$;
\item $-\sum_{a\in\left[\numclust\right]}T_{aa}+\sum_{a,b\in\left[\numclust\right]:a\ne b}T_{ab}=\error$
and thus $-\sum_{a\in\left[\numclust\right]}T_{aa}=\sum_{a,b\in\left[\numclust\right]:a\ne b}T_{ab}=\frac{\error}{2}$.
\end{enumerate}
Since $\Yhat-\Ystar$ has zero diagonal, we can write 
\begin{align*}
S_{4} & =\sum_{a\in\left[\numclust\right]}T_{aa}\norm[\Mean_{a}]2^{2}+2\sum_{a,b\in\left[\numclust\right]:a<b}T_{ab}\left\langle \Mean_{a},\Mean_{b}\right\rangle \\
 & =-\sum_{a,b\in\left[\numclust\right]:a<b}T_{ab}\minsep_{ab}^{2}\\
 & =-\frac{1}{2}\sum_{a,b\in\left[\numclust\right]:a\ne b}T_{ab}\minsep_{ab}^{2}\\
 & \leq-\frac{1}{2}\sum_{a,b\in\left[\numclust\right]:a\ne b}T_{ab}\minsep^{2}\\
 & =-\frac{1}{4}\minsep^{2}\error.
\end{align*}


\section{Proof of Theorem \ref{thm:ip_exp_rate}\label{sec:proof_ip_exp_rate}}

We define the shorthand 
\[
\iperror\coloneqq\max\left\{ \frac{1}{2}\norm[\F-\F^{*}]1:\eta(\F)\le\eta(\F^{*}),\F\in\mathcal{F}\right\} .
\]
It is not hard to see that $\iperror$ takes integer values in $[0,\num]$.
If $\iperror=0$ then we are done. We therefore focus on the case
$\iperror\in\left[\num\right]$.

Suppose $\iperror>3\num\numclust e^{-\snr^{2}/C_{0}^{2}}$ for a fixed
$C_{0}>D/c$. Note that 
\[
3\num\numclust e^{-\snr^{2}/C_{0}^{2}}\overset{(i)}{\le}\num\numclust\cdot\frac{1}{\numclust}\cdot e^{-\snr^{2}/\left(2C_{0}^{2}\right)}\le\num e^{-\snr^{2}/\left(2C_{0}^{2}\right)}<\num
\]
where step $(i)$ holds since we have assumed $\snr^{2}\ge\consts\numclust$
for some universal constant $\consts>0$. We record an important result
for our proof.
\begin{lem}
\label{lm:order_stats} Let $m\ge4$ and $g\ge1$ be integers. Let
$\X\in\real^{m\times g}$ be a matrix such that each $X_{ja}$ is
a sub-Gaussian random variable with its mean equal to $\lambda_{ja}$
and its sub-Gaussian norm no larger than $\rho_{ja}$, and each pair
$X_{ja}$ and $X_{ib}$ are independent for $j\ne i$ and $a,b\in\left[g\right]$.
Then for some universal constant $D>0$ and for any $\beta\in(0,m]$,
we have 
\begin{align*}
\sum_{j,a}X_{ja}M_{ja} & \le D\sqrt{\left\lceil \beta\right\rceil \left(\sum_{j,a}\rho_{ja}^{2}M_{ja}\right)\log\left(3mg/\beta\right)}+\sum_{j,a}\lambda_{ja}M_{ja},\\
 & \qquad\quad\forall\M\in\left\{ 0,1\right\} ^{m\times g}:\M\one_{g}\le\one_{m},\norm[\M]1=\left\lceil \beta\right\rceil ,
\end{align*}
with probability at least $1-\frac{1.5}{m}$.
\end{lem}
The proof is given in Section \ref{sec:proof_lm_order_stat}. Define
the set 
\[
\calM\coloneqq\left\{ \M\in\left\{ 0,1\right\} ^{\num\times\numclust}:\M\one_{\numclust}\le\one_{\num},\norm[\M]1=\iperror,M_{j,\labelstar(j)}=0\ \forall j\in\left[\num\right]\right\} .
\]
For any $\F$ feasible to $\IP_{3}$, we have 
\begin{align*}
0 & \le\frac{1}{c}\left(\eta(\F^{*})-\eta(\F)\right)\\
 & \overset{(i)}{=}\sum_{j\in[\num]}\sum_{a\in[\numclust]}\beta_{ja}F_{ja}\\
 & =\sum_{\left(j,a\right):F_{ja}=1,a\ne\labelstar(j)}\beta_{ja}\\
 & \le\max_{\M\in\calM}\sum_{j}\sum_{a\ne\labelstar(j)}\beta_{ja}M_{ja}\\
 & \overset{(ii)}{\le}\max_{\M\in\calM}\left[D\sqrt{\iperror\sgnorm^{2}\left(\sum_{j}\sum_{a\ne\labelstar(j)}\minsep_{\labelstar(j),a}^{2}M_{ja}\right)\log\left(3\num\left(\numclust-1\right)/\iperror\right)}-c\sum_{j}\sum_{a\ne\labelstar(j)}\minsep_{\labelstar(j),a}^{2}M_{ja}\right]\\
 & \le\max_{\M\in\calM}\left[D\sqrt{\iperror\sgnorm^{2}\left(\sum_{j}\sum_{a\ne\labelstar(j)}\minsep_{\labelstar(j),a}^{2}M_{ja}\right)\frac{\snr^{2}}{C_{0}^{2}}}-c\sum_{j}\sum_{a\ne\labelstar(j)}\minsep_{\labelstar(j),a}^{2}M_{ja}\right]\\
 & \le\left(\frac{D}{C_{0}}-c\right)\cdot\max_{\M\in\calM}\sum_{j}\sum_{a\ne\labelstar(j)}\minsep_{\labelstar(j),a}^{2}M_{ja}
\end{align*}
where step $(i)$ holds by Equation (\ref{eq:eta_func_equivalence}),
step $(ii)$ holds by Lemma \ref{lm:order_stats} with $g=\numclust-1$
since only $\numclust-1$ entries of $\left\{ \beta_{ja}\right\} $
are considered for each $j$ in the sum above $(ii)$, and the last
step holds since $\iperror\minsep^{2}\le\sum_{j}\sum_{a\ne\labelstar(j)}\minsep_{\labelstar(j),a}^{2}M_{ja}$.
Since $C_{0}>D/c$ and $\sum_{j}\sum_{a\ne\labelstar(j)}\minsep_{\labelstar(j),a}^{2}M_{ja}>0$,
the RHS above is negative, which is a contradiction. Hence, we must
have $\iperror\le3\num\numclust e^{-\snr^{2}/C_{0}^{2}}\le\num e^{-\snr^{2}/\left(2C_{0}^{2}\right)}$
and the result follows from the fact that $\norm[\F^{*}]1=\num$.

\section{Proof of technical results}

In this section we provide the proofs of the technical results used
in the proofs of our main theorems.

\subsection{Proof of Lemma \ref{lem:hanson-wright}\label{sec:proof_hanson_wright}}

We record the following lemma (Exercise 6.2.7 on pp.$\ $140 in \citet{vershynin2017high}).
\begin{lem}[Higher-dimensional Hanson-Wright inequality]
\emph{} \label{lem:hanson_wright_hdp} Let $\x_{1},\ldots,\x_{N}$
be independent, mean zero, sub-Gaussian random vectors in $\real^{M}$.
Let $\B=\left\{ B_{ij}\right\} $ be an $N\times N$ matrix. There
exists some universal constant $c>0$ such that for every $t\geq0$
\[
\P\left[\left|\sum_{i,j:i\ne j}^{N}B_{ij}\left\langle \x_{i},\x_{j}\right\rangle \right|\geq t\right]\leq2\exp\left[-c\min\left(\frac{t^{2}}{K^{4}M\norm[\B]F^{2}},\frac{t}{K^{2}\opnorm{\B}}\right)\right]
\]
where $K\coloneqq\max_{i}\norm[\x_{i}]{\psi_{2}}.$ 
\end{lem}
With this result, we only need to prove the same tail bound for $\P\left[\left|\sum_{i=1}^{N}B_{ii}\left(\norm[\x_{i}]2^{2}-\E\norm[\x_{i}]2^{2}\right)\right|\geq t\right]$.
To prove that, we cite another useful lemma (Theorem 2.8.2 on pp.$\ $36
in \citet{vershynin2017high}).
\begin{lem}[Bernstein's inequality for sub-exponential random variables]
\emph{}\label{lem:bernstein-subexp} Let $X_{1},\ldots,X_{N}$ be
independent, mean zero, sub-exponential random variables, and $\a\in\real^{N}$.
Then for every $t\geq0$, we have 
\[
\P\left[\left|\sum_{i=1}^{N}a_{i}X_{i}\right|\geq t\right]\leq2\exp\left[-c\min\left(\frac{t^{2}}{K_{1}^{2}\norm[\a]2^{2}},\frac{t}{K_{1}\norm[\a]{\infty}}\right)\right]
\]
where $K_{1}\coloneqq\max_{i}\norm[X_{i}]{\psi_{1}}$.
\end{lem}
Here, $\norm[\cdot]{\psi_{1}}$ denotes the sub-exponential norm;
see \citet{vershynin2017high} for more details. We work under the
premise of Lemma \ref{lem:hanson_wright_hdp}. Since $\x_{i}$ are
independent sub-Gaussian random vectors, each $\norm[\x_{i}]2^{2}-\E\norm[\x_{i}]2^{2}$
is the sum of $M$ independent, mean zero, sub-exponential random
variables with sub-exponential norm equal to $K^{2}$. Then Lemma
\ref{lem:bernstein-subexp} implies 
\[
\P\left[\left|\sum_{i=1}^{N}B_{ii}\left(\norm[\x_{i}]2^{2}-\E\norm[\x_{i}]2^{2}\right)\right|\geq t\right]\leq2\exp\left[-c\min\left(\frac{t^{2}}{K^{4}M\norm[\B]F^{2}},\frac{t}{K^{2}\opnorm{\B}}\right)\right]
\]
as required. 

\subsection{Proof of Fact \ref{fact:satisfy_cond_gauss_choas_operator_norm_bound}
\label{sec:proof_satisfy_cond}}

We prove the following equivalent statement 
\[
\norm[\left\langle \x,\w\right\rangle ]{\psi_{2}}^{2}\leq C\frac{\sgnorm^{2}}{\std^{2}}\E\left\langle \x,\w\right\rangle ^{2}\qquad\text{for any }\w\in\real^{\num},
\]
where $C>0$ is a universal constant and $C\frac{\sgnorm^{2}}{\std}\ge1.$
We first establish a relationship between $\sgnorm^{2}$ and $\Var\left(x_{1}\right)$:
Proposition 2.5.2 on pp. 24 of \citet{vershynin2017high} implies
that $\frac{C'\sgnorm^{2}}{\std^{2}}\ge\frac{1}{2}$ for some universal
constant $C'>0$. Hence, we have 
\begin{align*}
\norm[\left\langle \x,\w\right\rangle ]{\psi_{2}}^{2} & \overset{(i)}{\le}2C'\sum_{i\in\left[\num\right]}w_{i}^{2}\norm[x_{i}]{\psi_{2}}^{2}\\
 & =2C'\frac{\sgnorm^{2}}{\std^{2}}\sum_{i\in\left[\num\right]}w_{i}^{2}\std^{2}\\
 & \overset{(ii)}{=}2C'\frac{\sgnorm^{2}}{\std^{2}}\E\left\langle \x,\w\right\rangle ^{2},
\end{align*}
where $(i)$ holds according to Proposition 2.6.1 on pp. 28 of \citet{vershynin2017high},
and $(ii)$ holds since $x_{i}$ are i.i.d.$\ $and $\E x_{i}=0$.
Letting $C=2C'$ completes the proof.

\subsection{Proof of Lemma \ref{lm:order_stats} \label{sec:proof_lm_order_stat}}

We define 
\begin{align*}
L_{\M} & \coloneqq\sum_{j,a}\left(X_{ja}-\lambda_{ja}\right)M_{ja},\\
R_{\beta,\M} & \coloneqq D\sqrt{\left\lceil \beta\right\rceil \left(\sum_{j,a}\rho_{ja}^{2}M_{ja}\right)\log\left(3mg/\beta\right)},\\
\calM_{\beta} & \coloneqq\left\{ \M\in\left\{ 0,1\right\} ^{m\times g}:\M\one_{g}\le\one_{m},\norm[\M]1=\left\lceil \beta\right\rceil \right\} .
\end{align*}
To establish a uniform bound in $\beta$, we apply a discretization
argument to the possible values of $\beta$. Define the shorthand
$E\coloneqq(0,m]$. We can cover $E$ by the sub-intervals $E_{t}\coloneqq(t-1,t]$
for $t\in[m]$. For each $t\in[m]$ we define the probability

\begin{align*}
\alpha_{t} & \coloneqq\P\left\{ \exists\beta\in E_{t},\exists\M\in\calM_{\beta}:L_{\M}>R_{\beta,\M}\right\} .
\end{align*}
We bound each of these probabilities: 

\begin{align}
\alpha_{t} & \overset{(i)}{\le}\P\left\{ \exists\M\in\calM_{t}:L_{\M}>R_{t,\M}\right\} \nonumber \\
 & \leq\P\left\{ \bigcup_{\M\in\calM_{t}}\left\{ L_{\M}>R_{t,\M}\right\} \right\} \nonumber \\
 & \leq\sum_{\M\in\calM_{t}}\P\left\{ L_{\M}>R_{t,\M}\right\} ,\label{eq:double union bd on normal-1-1-2}
\end{align}
where step $(i)$ holds since $\beta\in E_{t}$ implies $\beta\le\left\lceil \beta\right\rceil =t$. 

Note that each $X_{ja}-\lambda_{ja}$ is an independent zero-mean
sub-Gaussian random variable and the squared sub-Gaussian norm of
$L_{\M}$ is at most $C_{\psi_{2}}\sum_{j,a}\rho_{ja}^{2}M_{ja}$
where $C_{\psi_{2}}>0$ is a universal constant. We apply Hoeffding
inequality (Lemma \ref{lem:hoeffding}) to bound the probability on
the RHS of (\ref{eq:double union bd on normal-1-1-2}): 
\begin{align*}
\P\left\{ L_{\M}>R_{t,\M}\right\}  & \leq\exp\left\{ -\frac{cD^{2}t\left(\sum_{j,a}\rho_{ja}^{2}M_{ja}\right)\log(3mg/t)}{C_{\psi_{2}}\sum_{j,a}\rho_{ja}^{2}M_{ja}}\right\} \\
 & \leq\exp\left\{ -4t\log(3mg/t)\right\} 
\end{align*}
where $c>0$ is a universal constant. Plugging this back to (\ref{eq:double union bd on normal-1-1-2}),
we have for each $t\in\left[m\right]$, 
\begin{align}
\alpha_{t} & \leq\sum_{\M\in\calM_{t}}\exp\left\{ -4t\log(3mg/t)\right\} \nonumber \\
 & =\binom{m}{t}g^{t}\exp\left\{ -4t\log(3mg/t)\right\} \nonumber \\
 & \leq\left(\frac{me}{t}\right)^{t}g^{t}\exp\left\{ -4t\log(3mg/t)\right\} \nonumber \\
 & \leq\exp\left\{ t\log(3mg/t)+t-4t\log(3mg/t)\right\} \nonumber \\
 & \leq\exp\left\{ -t\log(3mg/t)\right\} =\left(\frac{t}{3mg}\right)^{t},\label{eq:binom coeff bound-1-2}
\end{align}
where the last inequality follows from $t\leq t\log(3mg/t)$ for $t\in\left[m\right]$.
It follows that 
\begin{align*}
 & \quad\P\left\{ \exists\beta\in E,\exists\M\in\calM_{\beta}:L_{\M}>R_{\beta,\M}\right\} \\
 & \leq\P\left\{ \bigcup_{t=1}^{m}\left\{ \exists\beta\in E_{t},\exists\M\in\calM_{\beta}:L_{\M}>R_{\beta,\M}\right\} \right\} \\
 & \le\sum_{t=1}^{m}\alpha_{t}\\
 & \leq\sum_{t=1}^{m}\left(\frac{t}{3mg}\right)^{t}\eqqcolon P_{1}(m).
\end{align*}

It remains to show that $P_{1}(m)\leq\frac{1.5}{m}$. Since 
\begin{align*}
P_{1}(m) & \leq\sum_{t=1}^{m}\left(\frac{t}{3m}\right)^{t}\\
 & \le\frac{1}{3m}+\sum_{t=2}^{m}\left(\frac{t}{3m}\right)^{t}\\
 & \le\frac{1}{3m}+m\cdot\max_{t=2,3,\ldots,m}\left(\frac{t}{3m}\right)^{t},
\end{align*}
the proof is completed if for each integer $t=2,3,\ldots,m$, we can
show the bound $\left(\frac{t}{3m}\right)^{t}\leq\frac{1}{m^{2}}$,
or equivalently $f(t)\coloneqq t(\log3m-\log t)\geq2\log m.$ Since
$t\le m$, $f(t)$ has derivative 
\[
f'(t)=\log3m-\log t-1\ge\log3m-\log\left(\frac{3m}{3}\right)-1=\log3-1\ge0.
\]
Therefore, $f(t)$ is non-decreasing for $2\le t\le m$ and therefore
$f(t)\ge f(2)=2\log3m-2\log2\ge2\log m.$ Hence, $P_{1}(m)\le\frac{1.5}{m}$. 

\section{Proof of Theorem \ref{thm:cluster_error_rate}\label{sec:proof_cluster_error_rate}}

We only need to prove the first part of the theorem. The second part
follows immediately from the first part and Theorem \ref{cor:SDP_exp_rate}.

The proof follows similar lines as those of Theorem 17 and Lemma 18
in \citet{makarychev2016learning}. In the rest of the section, we
work under the context of Algorithms \ref{alg:apx_clustering} and
\ref{alg:est_clustering}. Recall that $\numclust'=\left|\left\{ B_{t}\right\} _{t\ge1}\right|$
and we let $\epsilon\coloneqq\norm[\Yhat-\Ystar]1/\norm[\Ystar]1$.
We have the following lemma.
\begin{lem}
\label{lem:apx_clustering} There exists a partial matching $\perm'$
between $\left[\numclust\right]$ and $\left[\numclust'\right]$ and
a universal constant $C>0$ such that 
\[
\left|\bigcup_{t=\perm'(a)}\clustset a\cap B_{t}\right|\ge\left(1-C\epsilon\right)\num.
\]
\end{lem}
The proof is given in Section \ref{sec:proof_apx_clustering}. The
next lemma concerns the quality of clustering by Algorithm \ref{alg:clustering}.
\begin{lem}
\label{lem:clustering} There exists a permutation $\perm$ on $\left[\numclust\right]$
and a universal constant $C>0$ such that 
\[
\left|\bigcup_{t=\perm(a)}\clustset a\cap U_{t}\right|\ge\left(1-C\epsilon\right)\num.
\]
\end{lem}
The proof is given in Section \ref{sec:proof_clustering}. The result
follows from combining the above lemmas and the fact that 
\[
\misrate(\LabelHat,\LabelStar)=1-\frac{1}{\num}\max_{\perm\in S_{\numclust}}\left|\bigcup_{t=\perm(a)}\clustset a\cap U_{t}\right|.
\]


\subsection{Proof of Lemma \ref{lem:apx_clustering}\label{sec:proof_apx_clustering}}

We define $\y_{a}$ to be an arbitrary row of $\Ystar$ whose index
is in $\clustset a$.
\begin{align*}
G_{a} & \coloneqq\left\{ i\in\clustset a:\norm[\Yhat_{i\bullet}-\y_{a}]1\leq\frac{\size}{8}\right\} ,\qquad\forall a\in\left[\numclust\right]\\
G & \coloneqq\bigcup_{a\in\left[\numclust\right]}G_{a},\\
H & \coloneqq\vertexset\backslash G.
\end{align*}

We construct a partial matching $\perm'$ between sets $\clustset a$
and $B_{t}$ by matching every cluster $\clustset a$ with the first
$B_{t}$ that intersects $G_{a}$, and we let $\perm'(a)=t$. Since
each $i\in\left[\num\right]$ belongs to some $B_{t}$, we are able
to match every $\clustset a$ with some $B_{t}$. The fact that we
cannot match two distinct clusters $\clustset a$ and $\clustset b$
with the same $B_{t}$ as well as the rest of the proof are given
by the following fact.
\begin{fact}
\label{fact:apx_clustering} We have 
\begin{enumerate}
\item For each $a\in\left[\numclust\right]$ and $t\in\left[\numclust'\right]$
such that $t=\perm'(a)$, we have $B_{t}\cap G_{b}=\emptyset$ for
any $b\in\left[\numclust\right]\backslash\left\{ a\right\} $ and
$B_{t}\subset G_{a}\cup H$;
\item For each $a\in\left[\numclust\right]$ and $t\in\left[\numclust'\right]$
such that $t=\perm'(a)$, we have 
\[
\left|B_{t}\cap\clustset a\right|\geq\left|G_{a}\right|-\left|B_{t}\cap H\right|.
\]
\item We have 
\[
\sum_{t=\perm'(a)}\left|B_{t}\cap\clustset a\right|\ge\left|\vertexset\right|-2\left|H\right|.
\]
\item There exists a universal constant $C>0$ such that $\left|H\right|\leq C\epsilon\num$.
\end{enumerate}
\end{fact}
The proof is given below.

\subsubsection{Proof of Fact \ref{fact:apx_clustering}\label{sec:proof_fact_apx_clustering}}
\begin{enumerate}
\item Suppose that there exist $B_{t}$ and $b\in\left[\numclust\right]$
such that $b\ne a$ and $B_{t}\cap G_{b}\ne\emptyset$. Let $u\in B_{t}\cap G_{a}$
and $v\in B_{t}\cap G_{b}$. Since $G_{a}$ and $G_{b}$ are disjoint,
we know that $u\ne v$. Let $w\in B_{t}$. Then we have 
\begin{align*}
\norm[\Yhat_{u\bullet}-\Yhat_{w\bullet}]1 & \leq\frac{\size}{4}\\
\norm[\Yhat_{v\bullet}-\Yhat_{w\bullet}]1 & \leq\frac{\size}{4}.
\end{align*}
Therefore 
\[
\norm[\Yhat_{u\bullet}-\Yhat_{v\bullet}]1\leq\norm[\Yhat_{u\bullet}-\Yhat_{w\bullet}]1+\norm[\Yhat_{v\bullet}-\Yhat_{w\bullet}]1\le\frac{\size}{2}.
\]
This implies 
\begin{align*}
\norm[\y_{a}-\y_{b}]1 & \leq\norm[\y_{a}-\Yhat_{u\bullet}]1+\norm[\Yhat_{u\bullet}-\Yhat_{v\bullet}]1+\norm[\y_{b}-\Yhat_{v\bullet}]1\\
 & \leq\frac{\size}{8}+\frac{\size}{2}+\frac{\size}{8}<\size,
\end{align*}
which is a contradiction to the fact that $\norm[\y_{a}-\y_{b}]1=2\size$.
To complete the proof, we note that for any $i\in B_{t}$ we have
either $i\in G_{a}$ or $i\in H$.
\item Fix $i\in G_{a}$ for some $a\in\left[\numclust\right]$. For any
$j\in G_{a}$ we have $j\in B(i)$ since 
\[
\norm[\Yhat_{i\bullet}-\Yhat_{j\bullet}]1\le\norm[\y_{a}-\Yhat_{i\bullet}]1+\norm[\y_{a}-\Yhat_{j\bullet}]1\le\frac{\size}{4}.
\]
Therefore, by definition 
\[
\left|B_{t}\right|\ge\left|B(i)\right|\ge\left|G_{a}\right|.
\]
We have 
\begin{align*}
\left|B_{t}\cap\clustset a\right| & \overset{(i)}{\ge}\left|B_{t}\cap G_{a}\right|\\
 & =\left|B_{t}\right|-\left|B_{t}\backslash G_{a}\right|\\
 & \overset{(ii)}{=}\left|B_{t}\right|-\left|B_{t}\cap H\right|\\
 & \ge\left|G_{a}\right|-\left|B_{t}\cap H\right|,
\end{align*}
where step $(i)$ holds since $G_{a}\subset\clustset a$ and step
$(ii)$ holds since $B_{t}\subset G_{a}\cup H$.
\item Summing the LHS of the above equation over $t=\perm'(a)$ gives 
\begin{align*}
\sum_{t=\perm'(a)}\left|B_{t}\cap\clustset a\right| & =\sum_{a\in\left[\numclust\right]}\left|G_{a}\right|-\sum_{t=\perm'(a)}\left|B_{t}\cap H\right|\\
 & \ge\sum_{a\in\left[\numclust\right]}\left|G_{a}\right|-\sum_{t\ge1}\left|B_{t}\cap H\right|\\
 & \overset{(i)}{=}\left|G\right|-\left|\vertexset\cap H\right|\\
 & =\left|\vertexset\right|-2\left|H\right|,
\end{align*}
where step $(i)$ holds since $B_{t}\cap H$ are disjoint and $\bigcup_{t\geq1}B_{t}=\vertexset$.
\item We have 
\[
\left|H\right|\cdot\frac{\size}{8}\leq\sum_{i\in H}\norm[\Yhat_{i\bullet}-\y_{\labelstar(i)}]1\leq\norm[\Yhat-\Ystar]1\leq\epsilon\norm[\Ystar]1=\epsilon\cdot\num\size
\]
where the last step follows from the fact that $\norm[\Ystar]1=\num\size$.
The result follows.
\end{enumerate}

\subsection{Proof of Lemma \ref{lem:clustering}\label{sec:proof_clustering}}

Let $\perm'$ be the partial matching between $\clustset a$ and $B_{t}$
from Lemma \ref{lem:apx_clustering}. Define $\perm(a)=\perm'(a)$
for $\perm'(a)\le\numclust$. If the resulting $\perm$ is a partial
permutation, we extend $\perm$ to a permutation defined on $\left[\numclust\right]$
in an arbitrary way. We may assume that $\left\{ U_{t}\right\} _{t\in\left[\numclust\right]}$
are $\left\{ B_{t}\right\} _{t\in\left[\numclust\right]}$ WLOG, and
that $U_{t}$ consists of $B_{t}$ and some elements from sets $B_{u}$
with $u>\numclust$. We have 
\begin{align*}
\left|\bigcup_{t=\perm(a)}\clustset a\cap U_{t}\right| & \ge\left|\bigcup_{t=\perm'(a)\le\numclust}\clustset a\cap B_{t}\right|\\
 & =\left|\bigcup_{t=\perm'(a)}\clustset a\cap B_{t}\right|-\left|\bigcup_{t=\perm'(a)>\numclust}\clustset a\cap B_{t}\right|\\
 & \ge\left(1-C'\epsilon\right)\num-\left|\bigcup_{t=\perm'(a)>\numclust}\clustset a\cap B_{t}\right|
\end{align*}
where $C'>0$ is a universal constant. Define 
\begin{align*}
T_{1} & \coloneqq\left\{ t>\numclust:t=\perm'(a)\text{ for some }a\in\left[\numclust\right]\right\} ,\\
T_{2} & \coloneqq\left\{ t\in\left[\numclust\right]:t\ne\perm'(a)\text{ for any }a\in\left[\numclust\right]\right\} .
\end{align*}
Note that $\left|T_{1}\right|=\left|T_{2}\right|$ and for any $t_{1}\in T_{1}$
and $t_{2}\in T_{2}$ we have $\left|B_{t_{1}}\right|\le\left|B_{t_{2}}\right|$.
Therefore, 
\begin{align*}
\left|\bigcup_{t=\perm'(a)>\numclust}\clustset a\cap B_{t}\right| & \le\left|\bigcup_{t\in T_{1}}B_{t}\right|\\
 & \le\left|\bigcup_{t\in T_{2}}B_{t}\right|\\
 & \le\left|\vertexset\right|-\left|\bigcup_{t=\perm'(a)}\clustset a\cap B_{t}\right|\\
 & =C'\epsilon\num.
\end{align*}
The result follows by setting $C\coloneqq2C'$.

\section{Proof of Theorem \ref{thm:mean_estimation_error}\label{sec:proof_mean_estimation_error}}

Let $\Var\left(g_{ij}\right)=\std^{2}$. For $a\in\left[\numclust\right]$,
define $\clustest a\coloneqq\left\{ i\in\left[\num\right]:\labelhat_{i}=a\right\} $
the estimated clusters encoded in $\LabelHat$, and recall that our
cluster center estimators are defined by $\Meanhat_{a}\coloneqq\size^{-1}\sum_{i\in\clustest a}\h_{i}$.
We assume $\left\{ \clustest a\right\} $ achieves the lowest clustering
error as given in Theorem \ref{thm:cluster_error_rate} WLOG. For
each $a\in\left[\numclust\right]$, we have 
\begin{align*}
\norm[\Meanhat_{a}-\Mean_{a}]2 & \le\norm[\frac{1}{\size}\sum_{i\in\clustest a}\h_{i}-\frac{1}{\size}\sum_{j\in\clustset a}\h_{j}]2+\norm[\frac{1}{\size}\sum_{j\in\clustset a}\h_{j}-\Mean_{a}]2\\
 & \eqqcolon Q_{1}+Q_{2}.
\end{align*}


\subsection{Controlling $Q_{1}$}

Define $\epsilon\coloneqq\misrate(\LabelHat,\LabelStar)$. We work
on the event that the result Theorem \ref{thm:cluster_error_rate}
is true. We have 
\[
Q_{1}=\frac{1}{\size}\norm[\sum_{i\in\clustest a\backslash\clustset a}\h_{i}-\sum_{j\in\clustset a\backslash\clustest a}\h_{j}]2
\]
Note that $\left|\clustest a\backslash\clustset a\right|=\left|\clustset a\backslash\clustest a\right|$
so we can pair each point in $\clustest a\backslash\clustset a$ with
a point in $\clustset a\backslash\clustest a$. Let us pair $i$th
point in $\clustest a\backslash\clustset a$ with $j(i)$th point
in $\clustset a\backslash\clustest a$, and define $\calM\coloneqq\left\{ \left(i,j(i)\right)\right\} $.
We have $\left|\calM\right|\le\num\epsilon$ and we can write 
\begin{align*}
Q_{1} & =\frac{1}{\size}\norm[\sum_{(i,j(i))\in\calM}\left(\h_{i}-\h_{j(i)}\right)]2\\
 & \le\frac{1}{\size}\sum_{(i,j(i))\in\calM}\norm[\h_{i}-\h_{j(i)}]2\\
 & \le\frac{1}{\size}\sum_{(i,j(i))\in\calM}\left(\minsep_{\labelstar(i),\labelstar(j(i))}+\norm[\g_{i}-\g_{j(i)}]2\right)\\
 & \le\frac{1}{\size}\sum_{(i,j(i))\in\calM}\left(C_{q}\minsep+\norm[\g_{i}-\g_{j(i)}]2\right),
\end{align*}
where the last step holds for some universal constant $C_{q}>0$ given
that $\max_{a,b\in\left[\numclust\right]}\minsep_{ab}\le C_{q}\minsep$.
By Theorem 3.1.1 on pp.$\ $41 of \citet{vershynin2017high}, $\frac{1}{\sqrt{2}\std}\norm[\g_{i}-\g_{j(i)}]2-\sqrt{\vecdim}$
is a sub-Gaussian random variable with sub-Gaussian norm at most $C_{\psi_{2}}\frac{\sgnorm^{2}}{\std^{2}}$
where $C_{\psi_{2}}>0$ is a universal constant. Then Lemma \ref{lem:hoeffding}
implies that 
\[
\P\left[\frac{1}{\sqrt{2}\std}\norm[\g_{i}-\g_{j(i)}]2-\sqrt{\vecdim}\ge C\frac{\sgnorm^{2}}{\std^{2}}\sqrt{\log\num}\right]\le\num^{-C'}
\]
for some universal constants $C,C'>2$. By the union bound and the
facts that $\left|\calM\right|\le\num$ and $\std\lesssim\sgnorm$,
we have 
\[
\max_{(i,j)\in\calM}\norm[\g_{i}-\g_{j(i)}]2\le C_{g}\left(\sgnorm\sqrt{2\vecdim}+C\sgnorm\sqrt{2\log\num}\right)
\]
with probability at least $1-n^{-C_{1}}$ where $C_{g},C_{1}>0$ are
universal constants. 

Therefore, we have 
\begin{align*}
Q_{1} & \le C_{0}\left(\minsep+\sgnorm\sqrt{\vecdim}+\sgnorm\sqrt{\log\num}\right)\cdot\numclust\exp\left[-\frac{\snr^{2}}{\conste}\right]\\
 & \le C_{0}\left(\minsep+\sgnorm\sqrt{\vecdim}+\sgnorm\sqrt{\log\num}\right)\cdot\exp\left[-\frac{\snr^{2}}{2\conste}\right]
\end{align*}
for some universal constant $C_{0},\conste>0$ with probability at
least $1-n^{-C_{1}}$, where the last step holds since $\snr^{2}\ge\numclust$.
The fact that $e^{x}\ge1+x>x$ for any $x$ implies 
\[
\exp\left[-\frac{\snr^{2}}{4\conste}\right]\le\frac{4\conste}{\snr^{2}}=\frac{\sgnorm}{\minsep}\cdot\frac{4\conste}{\snr}\le4\conste\frac{\sgnorm}{\minsep}
\]
where the last step holds since we have $\snr\ge1$ by the conditions
of Theorem \ref{thm:cluster_error_rate}. Hence, we have 
\begin{align*}
Q_{1} & \le C_{0}\sgnorm\left(4\conste+\sqrt{\vecdim}+\sqrt{\log\num}\right)\cdot\exp\left[-\frac{\snr^{2}}{4\conste}\right]\\
 & \leq C_{1}\sgnorm\left(1+\sqrt{\vecdim}+\sqrt{\log\num}\right)\cdot\exp\left[-\frac{\snr^{2}}{4\conste}\right]\\
 & \leq2C_{1}\sgnorm\left(\sqrt{\vecdim}+\sqrt{\log\num}\right)\cdot\exp\left[-\frac{\snr^{2}}{4\conste}\right]
\end{align*}
where $C_{1}>0$ is a universal constant.

\subsection{Controlling $Q_{2}$}

We have 
\[
Q_{2}=\norm[\frac{1}{\size}\sum_{j\in\clustset a}\g_{j}]2.
\]
We see that $\frac{1}{\size}\sum_{j\in\clustset a}g_{ji}$ has variance
$\frac{1}{\size}\std^{2}$. By Proposition 2.6.1 on pp.$\ $28 and
Theorem 3.1.1 on pp. 41 of \citet{vershynin2017high}, $\frac{\sqrt{\size}}{\std}\norm[\frac{1}{\size}\sum_{j\in\clustset a}\g_{j}]2-\sqrt{\vecdim}$
is a sub-Gaussian random variable with sub-Gaussian norm at most $C_{\psi_{2}}\frac{\sgnorm^{2}}{\std^{2}}$
where $C_{\psi_{2}}>0$ is a universal constant. Then Lemma \ref{lem:hoeffding}
implies that 
\[
\P\left[\frac{\sqrt{\size}}{\std}\norm[\frac{1}{\size}\sum_{j\in\clustset a}\g_{j}]2-\sqrt{\vecdim}\ge C\frac{\sgnorm^{2}}{\std^{2}}\sqrt{\log\num}\right]\le\num^{-C'}
\]
for some universal constants $C,C'>0$. Since $\std\lesssim\sgnorm$,
there exists a universal constant $C_{0}>0$ such that 
\[
Q_{2}\leq C_{0}\sgnorm\left(\sqrt{\frac{\numclust\vecdim}{\num}}+\sqrt{\frac{\numclust\log\num}{\num}}\right)
\]
with probability at least $1-\num^{-C'}$. 

\section{Technical lemmas}

The following lemma is Theorem 2.6.2 on pp.$\ $28 in \citet{vershynin2017high}.
\begin{lem}[General Hoeffding's inequality]
\emph{ \label{lem:hoeffding} }Let $X_{1},\ldots,X_{N}$ be independent,
mean zero, sub-Gaussian random variables. Then, for every $t\geq0$
we have 
\[
\P\left[\left|\sum_{i=1}^{N}X_{i}\right|\geq t\right]\leq2\exp\left[-\frac{ct^{2}}{\sum_{i=1}^{N}\norm[X_{i}]{\psi_{2}}^{2}}\right],
\]
where $c>0$ is a universal constant.
\end{lem}
The following lemma is Exercise 4.7.3 in \citet{vershynin2017high}.
\begin{lem}[Tail bound of covariance matrix of sub-Gaussians]
\emph{ }\label{lem:subg_cov_mat_bound} Let $\x$ be a sub-Gaussian
vector and let $\x_{1},\ldots,\x_{m}$ be independent samples of $\x$.
Let $m$ be a positive integer and define 
\begin{align*}
\boldsymbol{\Sigma} & \coloneqq\E\x\x\t,\\
\boldsymbol{\Sigma}_{m} & \coloneqq\frac{1}{m}\sum_{i=1}^{m}\x_{i}\x_{i}\t.
\end{align*}
Let $\rho_{0}\ge1$ be such that 
\[
\norm[\left\langle \x,\w\right\rangle ]{\psi_{2}}\leq\rho_{0}\sqrt{\E\left\langle \x,\w\right\rangle ^{2}}\qquad\text{for any }\w\in\real^{N}.
\]
For any $u\geq0$, we have for a universal constant $C>0$, 
\[
\opnorm{\boldsymbol{\Sigma}_{m}-\boldsymbol{\Sigma}}\leq C\rho_{0}^{2}\left(\sqrt{\frac{N+u}{m}}+\frac{N+u}{m}\right)\opnorm{\boldsymbol{\Sigma}}
\]
with probability at least $1-2e^{-u}$.
\end{lem}


% !TEX root = paper.tex


\subsection{Proofs from \pref{sec:matrix}}



\begin{proof}[\pfref{prop:matrix_sufficient}]

  Recall that $\cA_{\eta}(X_{1},\ldots,X_{n}) = \frac{\eta{}rL^{2}}{2}\nrm*{\sum_{t=1}^{n}\cM(X_t)}_{\sigma} + \frac{c}{\eta}$. Linearizing the loss with the adaptive bound as in \pref{eq:phi_comp_adap}, 
\begin{align*}
	&\sum_{t=1}^{n}\ls(\pred_t, y_t) - \inf_{W\in\cW}\ls(\tri*{W,X_t}, y_t) - \cA_{\eta}(X_{1},\ldots,X_{n}) \\
    &\leq \sup_{W\in\cW}\crl*{
       \sum_{t=1}^{n}\partial\ls(\pred_t, y_t)(\pred_t - \tri*{W,X_t}) - \cA_{\eta}(X_{1},\ldots,X_{n})
      } \\
    &=
      \sum_{t=1}^{n}\partial\ls(\pred_t, y_t)\pred_t  + r\nrm*{\sum_{t=1}^{n}\partial\ls(\pred_t, y_t)X_t}_{\sigma}- \cA_{\eta}(X_{1},\ldots,X_{n}) .
  \end{align*}
	We now abbreviate $\partial\ls(\pred_t, y_t) = \dl_{t}$ and expand out $\cA_{\eta}$, yielding
\begin{align*}
        &\sum_{t=1}^{n}\dl_t\cdot\pred_t  + r\nrm*{\sum_{t=1}^{n}\dl_tX_t}_{\sigma} - \frac{\eta{}rL^{2}}{2}\nrm*{\sum_{t=1}^{n}\cM(X_t)}_{\sigma} - \frac{c}{\eta}.
\end{align*}
Using the fact that $\lambda_{1}(\cH(X)) = \nrm*{X}_{\sigma}$, linearity of $\cH$, and that $\cM(X_t)$ is positive semidefinite, we write this as
\begin{align*}
      &\sum_{t=1}^{n}\dl_t\cdot\pred_t  + r\lambda_{1}\prn*{\sum_{t=1}^{n}\dl_t\cH(X_t)} - r\lambda_{1}\prn*{\frac{\eta{}L^2}{2}\sum_{t=1}^{n}\cM(X_t)} - \frac{c}{\eta}
\end{align*}
Sub-additivity of $\lambda_{1}$ gives a further upper bound of
\begin{align*}    
        \sum_{t=1}^{n}\dl_t\cdot\pred_t  + r\lambda_{1}\prn*{\sum_{t=1}^{n}\dl_t\cH(X_t)-\frac{\eta{}L^{2}}{2}\sum_{t=1}^{n}\cM(X_t)} - \frac{c}{\eta}
\end{align*}
Then $\suff(X_t,\pred_t,\delta_t) = \left( \delta_t\cdot\pred_t, \delta_t\cdot \cH(X_t), \cM(X_t) \right)\in\reals\times \sym^{d_1+d_2}\times \sym^{d_1+d_2}_{+}$ is a sufficient statistic. Namely, writing
\[
V(a, H, M) = a + r\lambda_1\prn*{H -\frac{\eta{L^{2}}}{2}M} - \frac{c}{\eta},
\]
our calculation shows that
\[
\sup_{W\in\cW}\crl*{\reg(W) - \cA(X_{1},\ldots,X_{n})
      }
      \leq{} V\prn*{
      \sum_{t=1}^{n}\suff(X_t,\pred_t,\delta_t)
      }.
\]
  \end{proof}

  \begin{proof}[\pfref{thm:matrix_burkholder}]

    Recall that
    \[
      \burk(a, H, M) = a+ \frac{r}{\eta}\log\,\Tr\,\exp\prn*{\eta{}H - \frac{\eta^{2}L^{2}}{2}M} - \frac{c}{\eta}
    \]
We will show that $\burk$ satisfies the three properties of \pref{lem:equivalence_burkholder}. For property \propone{}, we have
\[
\burk(0) = \frac{r}{\eta}\log\prn*{\Tr\prn*{\exp\prn*{0}}} - \frac{c}{\eta} = \frac{r\log(d_1 + d_2)}{\eta} - \frac{c}{\eta}.
\]
Thus, $\burk(0)\leq{}0$ as soon as $c\geq{}r\log(d_1+d_2)$.

For property \proptwo{}, it suffices to show that $\lambda_{1}(H-\frac{\eta{}L^{2}}{2}M) \leq{} \frac{1}{\eta}\log\,\Tr\,\exp\prn*{\eta{}H - \frac{\eta^{2}L^{2}}{2}M}$. To this end, we have
\begin{align*}
  \lambda_{1}\left(H-\frac{\eta{}L^{2}}{2}M \right)  =   \frac{1}{\eta}\log \lambda_{1}\prn*{\exp\prn*{\eta{}H-\frac{\eta^2L^{2}}{2}M}} \leq{} \frac{1}{\eta}\log\,\Tr\,\exp\prn*{\eta{}H-\frac{\eta^2L^{2}}{2}M
  },
\end{align*}
where the equality is well-defined because the matrix under consideration is symmetric and the inequality follows because $e^{A}$ is positive semidefinite for any symmetric matrix $A$.

For the third property, observe that the mapping $\alpha\mapsto{}V(\tau + \suff(z,\alpha))$ is convex (e.g. \citep{lewis1996convex}). Consequently, by \pref{lem:equivalence_burkholder}, it suffices only to prove property $3'$, i.e. that the restricted concavity condition holds only for Rademacher random variables.

Fix $\tau\in\cT$ and $z=(X, \pred)\in\cX\times{}\cY$, and let $\eps$ be a Rademacher random variable. Writing
\[
\tau = (\tau_{1}, \tau_{2}, \tau_{3})\in\reals\times \sym^{d_1+d_2}\times \sym^{d_1+d_2}_{+},
\]
we have
\begin{align*}
 &\En_{\eps}\left[\burk(\tau + \suff(z,\eps{}L))\right] \\
 &= \En_{\eps}\brk*{
   \tau_{1} + \pred\eps{}L + \frac{r}{\eta}\log\,\Tr\,\exp\prn*{\eta{}\tau_{2} - \frac{\eta^{2}L^{2}}{2}\tau_{3} + \eta\eps{}L\cH(X) - \frac{\eta^{2}L^{2}}{2}\cM(X)}
   } - \frac{c}{\eta}. \\
   &= \frac{r}{\eta}\En_{\eps}\brk*{
     \log\,\Tr\,\exp\prn*{\eta{}\tau_{2} - \frac{\eta^{2}L^{2}}{2}\tau_{3} + \eta\eps{}L\cH(X) - \frac{\eta^{2}L^{2}}{2}\cM(X)}
   } +      \tau_{1} - \frac{c}{\eta}.
\end{align*}
Focusing on the log-trace-exponential term, observe that
\begin{align*}
  &  \En_{\eps}\brk*{
  \log\,\Tr\,\exp\prn*{\eta{}\tau_{2} - \frac{\eta^{2}L^{2}}{2}\tau_{3} + \eta\eps{}L\cH(X) - \frac{\eta^{2}L^{2}}{2}\cM(X)}
} \\
&=  \En_{\eps}\brk*{
       \log\,\Tr\,\exp\prn*{\eta{}\tau_{2} - \frac{\eta^{2}L^{2}}{2}\tau_{3} + \log\prn*{\exp\prn*{\eta\eps{}L\cH(X)}} - \frac{\eta^{2}L^{2}}{2}\cM(X)}
       }.
\end{align*}
Since $\exp\prn*{\eta\eps{}L\cH(X)}$ is positive definite and $\eta{}\tau_{2} - \frac{\eta^{2}L^{2}}{2}\tau_{3} -\frac{\eta^{2}L^{2}}{2}\cM(X)$ is symmetric (by assumption), we can apply Lieb's Concavity Theorem to upper bound this by
\begin{align*}	   
  & \log\,\Tr\,\exp\prn*{\eta{}\tau_{2} - \frac{\eta^{2}L^2}{2}\tau_{3} + \log\prn*{\En_{\eps}\exp\prn*{\eta\eps{}L\cH(X)}} - \frac{\eta^{2}L^2}{2}\cM(X)}.
\end{align*}
The Rademacher matrix mgf bound \citep{tropp2012user} now yields
\[
\log\prn*{\En_{\eps}\exp\prn*{\eta\eps{}L\cH(X)}} \preceq \log\prn*{\exp\prn*{\eta^{2}L^{2}\cM(X)}/2} = \eta^{2}L^{2}\cM(X)/2.
\]
Since $A\preceq{}B$ implies $\Tr{}e^{A}\leq{}\Tr{}e^{B}$, this implies that
\[
\En_{\eps}\brk*{
  \log\,\Tr\,\exp\prn*{\eta{}\tau_{2} - \frac{\eta^{2}L^{2}}{2}\tau_{3} + \eta\eps{}L\cH(X) - \frac{\eta^{2}L^{2}}{2}\cM(X)}
}
\leq{}
  \log\,\Tr\,\exp\prn*{\eta{}\tau_{2} - \frac{\eta^{2}L^{2}}{2}\tau_{3}}
\]
Combining everything we proved so far, this implies
\[
  \En_{\eps}\left[U(\tau + \suff(z,\eps{}L))\right]
  \leq{} \tau_{1} + \frac{r}{\eta}  \log\,\Tr\,\exp\prn*{\eta{}\tau_{2} - \frac{\eta^{2}L^2}{2}\tau_{3}}  - \frac{c}{\eta} = U(\tau).
\]

\end{proof}

\begin{proof}[\pfref{corr:matrix_strategy}]
  The Burkholder function $\burk$ satisfies the conditions of \pref{lem:det_strat3}. Direct calculation shows that the strategy in \pref{lem:det_strat3} matches the strategy in the statement of the corollary.
\end{proof}

\begin{proof}[\pfref{corr:matrix_square}]
  We invoke the Burkholder function $\burk$ from \pref{thm:matrix_burkholder} for the special case $r=1$ and $c=\log(d_1 + d_2)$, and $L=1$. In particular, its existence per \pref{lem:equivalence_burkholder} implies (for the corresponding $V$, here denoted $V_{\eta}$ to refer to the $V$ given for a fixed value of $\eta$)
  \[
    \inf_{\eta}\sup_{\mb{z},\mb{p}, n}\En\brk*{V_{\eta}\prn*{\sum_{t=1}^{n}\suff(\mb{z}_t, \dl_{t})}}\leq{}0
  \]
  We use this inequality only for the special case where $\dl_{t}=\eps_{t}$ and $\mb{z}_t=(\mb{X}_{t}(\eps), 0)$. For this special case, the inequality implies
    \[
      \inf_{\eta}\sup_{\mb{X}, n}\En\brk*{\nrm*{\sum_{t=1}^{n}\eps_t\mb{X}_{t}(\eps)} - \frac{\eta}{2}\nrm*{\sum_{t=1}^{n}\cM(\mb{X}_{t}(\eps))} - \frac{\log(d_1 + d_2)}
        {\eta} }\leq{}0.
    \]
    For any fixed martingale $(\mb{X}_{t}(\eps))_{t\leq{}n}$, this implies
    \begin{align*}
      \En\nrm*{\sum_{t=1}^{n}\eps_t\mb{X}_{t}(\eps)} &\leq{} \inf_{\eta>0}\crl*{\frac{\eta}{2}\En\nrm*{\sum_{t=1}^{n}\cM(\mb{X}_{t}(\eps))} + \frac{\log(d_1 + d_2)}{\eta} } \\ &= \sqrt{2\En\nrm*{\sum_{t=1}^{n}\cM(\mb{X}_{t}(\eps))}\log(d_1 + d_2)}.
    \end{align*}
	To conclude, observe that for any sequence $(X_t)$ we have  
	  \[
	    \nrm*{\sum_{t=1}^{n}\cM(X_t)}_{\sigma} \leq{}  \max\crl*{\nrm*{\sum_{t=1}^{n}X_tX_t^{\trn}}_{\sigma}, \nrm*{\sum_{t=1}^{n}X_t^{\trn}X_t}_{\sigma}}.
	  \]
	  Indeed, $\sum_{t=1}^{n}\cM(X_t)  = \left(
	  \begin{array}{ll}
	  \sum_{t=1}^{n}X_tX_t^{\trn}& 0\\
	  0 & \sum_{t=1}^{n}X_t^{\trn}X_t
	  \end{array}
	  \right)$ and the spectral norm of a block-diagonal matrix is always obtained by the spectral norm of one of its blocks.
    


\end{proof}

\subsection{Proofs from \pref{sec:further}}
\label{app:square}

\begin{proof}[\textbf{Sketch of proofs for claims from \pref{sec:adagrad}}]

For the $\ls_{2}$ result we have
\begin{align*}
& \sum_{t=1}^n\loss(\pred_t,y_t) - \min_{\norm{w}_{2}\leq 1}\sum_{t=1}^n \loss(\inner{w,x_t},y_t) - 2L\sqrt{\sum_{t=1}^{n}\nrm*{x_t}^{2}_{2}} \\
& \leq{} \sup_{\norm{w}_{2}\leq 1} \crl*{\sum_{t=1}^n\partial\loss(\pred_t,y_t)(\yh_t, - \tri*{w, x_t}) } - 2L\sqrt{\sum_{t=1}^{n}\nrm*{x_t}^{2}_{2}} \\ 
& = \sum_{t=1}^n\partial\loss(\pred_t,y_t)\yh_t + \nrm*{\sum_{t=1}^{n}\partial\loss(\pred_t,y_t)x_t}_{2} - 2L\sqrt{\sum_{t=1}^{n}\nrm*{x_t}^{2}_{2}} \\
& \leq{} \sum_{t=1}^n\partial\loss(\pred_t,y_t)\yh_t + \burk_{\textrm{square}}\prn*{\sum_{t=1}^{n}\partial\loss(\pred_t,y_t)x_t, L\sqrt{\sum_{t=1}^{n}\nrm*{x_t}^{2}_{2}}}. 
\end{align*}
The path from here to a Burkholder function in the sense of \pref{lem:equivalence_burkholder} is clear given the three properties of $\burk_{\textrm{square}}$ stated in the main body.

For the $\ls_{\infty}$ result, the quantity
\begin{align*}
& \sum_{t=1}^n\loss(\pred_t,y_t) - \min_{\norm{w}_{\infty}\leq 1}\sum_{t=1}^n \loss(\inner{w,x_t},y_t) - 2L\nrm*{\prn*{\sum_{t=1}^{n}x_{t}^{2}}^{1/2}}_{1} 
\end{align*}
can be upper bounded by
\begin{align*}
&\sup_{\norm{w}_{\infty}\leq 1} \crl*{\sum_{t=1}^n\partial\loss(\pred_t,y_t)(\yh_t, - \tri*{w, x_t}) } - 2L\nrm*{\prn*{\sum_{t=1}^{n}x_{t}^{2}}^{1/2}}_{1} \\ 
& = \sum_{t=1}^n\partial\loss(\pred_t,y_t)\yh_t + \nrm*{\sum_{t=1}^{n}\partial\loss(\pred_t,y_t)x_t}_{1} - 2L\nrm*{\prn*{\sum_{t=1}^{n}x_{t}^{2}}^{1/2}}_{1} \\
& \leq{} \sum_{t=1}^n\partial\loss(\pred_t,y_t)\yh_t + \sum_{i=1}^{d}\burk_{\textrm{square}}\prn*{\sum_{t=1}^{n}\partial\loss(\pred_t,y_t)x_t[i], L\sqrt{\sum_{t=1}^{n}\prn*{x_t[i]}^{2}_{2}}},
\end{align*}
where $x_{t}[i]$ refers to the $i$th coordinate of $x_t$. Once again, the three properties of $\burk_{\textrm{square}}$ directly lead to a valid Burkholder function $\burk$.
\end{proof}

\begin{proof}[\pfref{prop:square_loss_sufficient}]  
  Let $A_{n}=\rho\sum_{t=1}^{n}z_tz_t^{\trn}+\lambda{}I$ and $A_{0}=\lambda{}I$. Recall that $\Psi_{A}(w) = \frac{1}{2}\tri*{w,Aw}$. We begin by rewriting the desired regret bound as
  \[
    \cA(w;z_{1},\ldots,z_{n}) = \lambda\Phi((w,1)) + c\log\prn*{\det(A_n)/\det(A_{0})}
  \]
  for a constant $c>0$ to be determined. With this definition, we have
  \begin{align*}
    &\sup_{w\in\bbR^{d}}\crl*{
    \reg(w) - \cA(w;z_{1},\ldots,z_{n})
      } \\
    &=\sup_{w\in\bbR^{d}}\crl*{
      \sum_{t=1}^{n}\ls(\yh_t, y_t) - \sum_{t=1}^{n}\ls(\tri*{w,x_t}, y_t)  - \lambda\Phi((w,1))
      } - c\log\prn*{\det(A_n)/\det(A_{0})}
      \intertext{Using strong convexity of $\ls$:}
    &=\sup_{w\in\bbR^{d}}\crl*{
      \sum_{t=1}^{n}\partial{}\ls(\yh_t, y_t)(\yh_t - \tri*{w,x_t}) - \frac{\rho}{2}\prn*{\yh_t - \tri*{w,x_t}}^{2}  - \lambda\Phi((w,1))
      } - c\log\prn*{\det(A_n)/\det(A_{0})}\\
    &=\sup_{w\in\bbR^{d}}\crl*{
      \sum_{t=1}^{n}\partial{}\ls(\yh_t, y_t)(-\tri*{(w,1),z_t}) - \frac{\rho}{2}\prn*{\tri*{(w,1),z_t}}^{2}  - \lambda\Phi((w,1))
      } - c\log\prn*{\det(A_n)/\det(A_{0})}
      \intertext{We now move to an upper bound by allowing the final coordinate of $(w,1)$ to act as a free parameter.}
    &\leq{}\sup_{w\in\bbR^{d+1}}\crl*{
      \sum_{t=1}^{n}\partial{}\ls(\yh_t, y_t)\tri*{w,z_t} - \frac{\rho}{2}\tri*{w,z_t}^{2}  - \lambda\Phi(w)
      } - c\log\prn*{\det(A_n)/\det(A_{0})}
      \intertext{We can rewrite this as}
    &\leq{}\sup_{w\in\bbR^{d+1}}\crl*{
      \tri*{w,\sum_{t=1}^{n}\partial{}\ls(\yh_t, y_t)z_t} - \Psi_{\rho\Sigma_{n}}(w)  - \lambda\Phi(w)
      } - c\log\prn*{\det(A_n)/\det(A_{0})}\\
     &=\sup_{w\in\bbR^{d+1}}\crl*{
      \tri*{w,\sum_{t=1}^{n}\partial{}\ls(\yh_t, y_t)z_t} - \Psi_{A_{n}}(w)
      } - c\log\prn*{\det(A_n)/\det(A_{0})}\\
    &=\Psi_{A_n}^{\star}\prn*{
      \sum_{t=1}^{n}\partial{}\ls(\yh_t, y_t)z_t}
      - c\log\prn*{\det(A_n)/\det(A_{0})}.
  \end{align*}
  
This establishes that $\suff(x_t,\pred_t,\delta_t) = \left( \delta_{t}z_t, z_tz_t^{\trn} \right)\in\bbR^{d+1}\times{}\sym^{d+1}_{+}$ is a sufficient statistic. This is because we can write
\[
V(x, A) = \Psi^{\star}_{\rho{}A + \lambda{}I}\prn*{x} - c\log\prn*{\det(\rho{}A + \lambda{}I)/\det(A_0)}.
\]
and we just proved that
\[
\sup_{w\in\bbR^{d}}\crl*{\reg(w) - \cA(x_{1},\ldots,x_{n})
      }
      \leq{} V\prn*{
      \sum_{t=1}^{n}\suff(x_t,\pred_t,\delta_t)
      }.
\]

\end{proof}

\begin{proof}[\pfref{thm:square_loss_burkholder}]
Recall that we have defined \[\burk(x, A) = V(x,A) =\Psi^{\star}_{A}\prn*{x} - c\log\prn*{\det(A)/\det(A_0)}.\] We verify the properties from \pref{lem:equivalence_burkholder}. Property \proptwo{} is immediate, and for property \propone{} we have
\[
\burk(0) = \Psi^{\star}_{0 + \lambda{}I}(0) - c\log(\det(A_0)/\det(A_0)) = 0.
\]
We proceed to prove property \propthree{}. Fix $\tau=(\tau_1, \tau_2)\in\cT=\bbR^{d+1}\times{}\sym_{+}^{d+1}$ and a mean-zero distribution $p$ over $\brk*{-L, L}$. Then we have
\begin{align*}
\En_{\alpha\sim{}p}\burk(\tau + \suff(z, \alpha))
 &= \En_{\alpha\sim{}p}\brk*{
 \Psi^{\star}_{\rho(\tau_{2}+zz^{\trn})+\lambda{}I}(\tau_{1} + \alpha{}z) - c\log(\det(\rho(\tau_{2}+zz^{\trn})+\lambda{}I)/\det(A_0))
 } \\
 &= \En_{\alpha\sim{}p}\brk*{
 \Psi^{\star}_{\rho(\tau_{2}+zz^{\trn})+\lambda{}I}(\tau_{1} + \alpha{}z)} - c\log(\det(\rho(\tau_{2}+zz^{\trn})+\lambda{}I)/\det(A_0)).
\end{align*}
Let $A=\rho(\tau_{2}+zz^{\trn})+\lambda{}I$ and $B=\rho\tau_{2}+\lambda{}I$. Then since $\Psi^{\star}$ is a squared Euclidean norm and $\alpha$ is mean-zero:
\[
\En_{\alpha\sim{}p}\brk*{\Psi^{\star}_{A}(\tau_{1} + \alpha{}z)} \leq{} \Psi^{\star}_{A}(\tau_{1}) + \En_{\alpha\sim{}p}\brk*{\alpha^{2}\tri*{z, A^{-1}z}} \leq{} \Psi^{\star}_{A}(\tau_{1}) + L^{2}\brk*{\alpha^{2}\tri*{z, A^{-1}z}}.
\]
Also note that since $B\preceq{}A$, $\Psi^{\star}_{A}(\tau_1) \leq{} \Psi^{\star}_{B}(\tau_1)$.

To conclude, observe that we just established
\begin{align*}
\En_{\alpha\sim{}p}\burk(\tau + \suff(z, \alpha)) &\leq{} \Psi_{B}^{\star}(\tau_1) + L^{2}\tri*{z, A^{-1}z} - c\log(\det(A)/\det(A_0)).
\intertext{Using a standard argument (e.g. from \cite{PLG}) and using that $A=B+\rho{}zz^{\trn}$:}
&\leq{} \Psi_{B}^{\star}(\tau_1) + \frac{L^{2}}{\rho}\log(\det(A)/\det(B)) - c\log(\det(A)/\det(A_0)).
\intertext{For $c\geq{}L^{2}/\rho$, this is bounded by}
&\leq{} \Psi_{B}^{\star}(\tau_1) - c\log(\det(B)/\det(A_0)) \\
&= \burk(\tau).
\end{align*}

\end{proof}

\subsection{Proofs from \pref{app:linear_loss}}

\begin{proof}[\pfref{prop:param_sufficient}]
  We define a potential function that will eventually be used in the construction of the Burkholder function $\burk$ we provide for $V$. As discussed in the main body, a variant of this potential was first introduced by \cite{mcmahan2014unconstrained} for the special case of Hilbert spaces. Let $\Psi(x) = \frac{1}{2}\nrm*{x}^{2}$ (not necessarily a Hilbert space norm) and define
  \[
    F_{n}(x) = \gamma\exp\prn*{\frac{\Psi(x)}{an}}.
  \]
  From \cite[Lemma 14]{mcmahan2014unconstrained}, along with the additional fact that $(f(\nrm*{\cdot}))^{\star} = f^{\star}(\nrm*{\cdot}_{\star})$ for general dual norm pairs, it holds that
  \[
    F_{n}^{\star}(w) \leq{} \nrm*{w}_{\star}\sqrt{2an\log\prn*{\frac{\sqrt{an}\nrm*{w}_{\star}}{\gamma} + 1}}.
  \]
  This is all we need to establish the result. We proceed as follows
\begin{align*}
&\sup_{w\in\bbR^{d}}\crl*{
    \reg(w) - \cA(w)
      } \\
      &=
      \sup_{w\in\bbR^{d}}\crl*{
    \sum_{t=1}^{n}\ls(\pred_t, y_t) - \ls(\tri*{w,x_t}, y_t) - \cA(w)
      } \\
     &\leq{}
      \sup_{w\in\bbR^{d}}\crl*{
    \sum_{t=1}^{n}\partial\ls(\pred_t, y_t)(\pred_t - \tri*{w,x_t}) - \cA(w)
      } \\
     &=
\sum_{t=1}^{n}\partial\ls(\pred_t, y_t)\cdot\pred_t  +  \sup_{w\in\bbR^{d}}\crl*{
       \tri*{w,\sum_{t=1}^{n}\partial\ls(\pred_t, y_t)x_t} - \cA(w)
       }
       \intertext{Using the inequality for the potential $F^{\star}_n$ stated above:}
      &\leq{}
        \sum_{t=1}^{n}\partial\ls(\pred_t, y_t)\cdot\pred_t  +  
        F^{\star}_n\prn*{\sum_{t=1}^{n}\partial\ls(\pred_t, y_t)x_t}
        - c
\end{align*}
It follows that $\suff(x_t,\pred_t,\delta_t) = \left( \delta_t\cdot\pred_t, \delta_t\cdot x_t \right)\in\reals\times \X$ is a sufficient statistic. This is because we can write
\[
V(b, x) = b + F^{\star}_n(x) - c.
\]
and we have just shown that
\[
\sup_{w}\crl*{\reg(w) - \cA(x_{1},\ldots,x_{n})
      }
      \leq{} V\prn*{
      \sum_{t=1}^{n}\suff(x_t,\pred_t,\delta_t)
      }.
\]

  
\end{proof}

\begin{proof}[\pfref{thm:param_free}]
    Since $\burk$ depends on time, we generalize the properties of \pref{lem:equivalence_burkholder} to
  \begin{enumerate}
  \item[$1^o$] $\burk_{0}(0) \le 0$
  \item[$2^o$] For any $\tau \in \T$, $\burk_{n}(\tau) \ge V(\tau)$
  \item[$3^o$] For any $\tau \in \T$, $z \in \X \times \Y$, and any mean-zero distribution $p$ on $[-L,L]$, and any $t\geq{}1$
    \begin{align}
      \En_{\alpha\sim p}\left[\burk_{t}(\tau + \suff(z,\alpha))\right] \le \burk_{t-1}(\tau) 
    \end{align}    
  \item[$3'$] For any $\tau \in \T$, $z \in \X \times \Y$, and any $t\geq{}1$,
    $$
    \forall \tau \in \T, z \in \X \times \Y,~~~ \En_\epsilon \burk_{t}(\tau + \suff(z,\epsilon L))  \le \burk_{t-1}(\tau),
    $$
    where $\epsilon$ is a Rademacher random variable. 
  \end{enumerate}

  Recall that for simplicity we assume $L=1$ and $\X$ is a unit ball: $\nrm*{x}\leq{}1$. Let $\Psi(x) = \frac{1}{2}\nrm*{x}^{2}$, where we have assumed that $\beta$-smoothness of $\Psi$:
  \[
    \Psi(x+y) \leq{} \Psi(x) + \tri*{\grad{}\Psi(x), y} + \frac{\beta}{2}\nrm*{y}^{2}.
  \]
  
  Define a family of potentials
  \[
    F_{t}(x) = \gamma\exp\prn*{\frac{\Psi(x)}{at} + \frac{1}{2}\sum_{s=t+1}^{n}\frac{1}{s}}
  \]
  and $F_{0} = \gamma\exp\prn*{\frac{1}{2}\sum_{t=1}^{n}\frac{1}{t}}$. Note that $F_{n}$ here is the same as in the proof of \pref{prop:param_sufficient}.
  
  Observe that
  \[
    \burk_{t}(b, x) = b + F^{\star}_t(x) - c, 
  \]
  where $F_{t}^{\star}$ is as defined as in the proof of \pref{prop:param_sufficient}. We proceed to establish the three properties of $\burk$ from \pref{lem:equivalence_burkholder}. Property $2^o$ holds since $V=\burk_{n}$. We will show property \propthreep{} first, then conclude with property \propone{}. Note that $\alpha\mapsto{}\burk_{t}(\tau + \suff(z,\alpha))$ is convex with respect to $\alpha$, and so it indeed suffices to show property \propthreep{}.

  Fix an element $\tau=(\tau_1, \tau_2)\in\bbR\times{}\cX=\cT$ of the sufficient statistic space. At time $n$ we have
  \[
    \En_{\eps}\left[\burk_{n}(\tau + \suff(z,\eps))\right] = \En_{\eps}\brk*{\tau_{1} +  \eps\cdot\pred + F_{n}(\tau_{2} + \eps{}x_n)} - c = \tau_{1} + \En_{\eps}\brk*{F_{n}(\tau_{2} + \eps{}x_n)} - c.
  \]
  To handle $F_n$, begin by using smoothness of $\Psi$:
  \begin{align*}
\En_{\eps}\brk*{F_{n}(\tau_{2} + \eps{}x_n)} =  \En_{\eps}\exp\prn*{\frac{\Psi(\tau_{2} + \eps{}x)}{an}} &\leq{}
                                                               \En_{\eps}\exp\prn*{\frac{\Psi(\tau_{2}) + \eps\tri*{\grad\Psi(\tau_2), x} + \frac{\beta}{2}\nrm*{x}^{2}}{an}} 
\end{align*}
Using the standard Rademacher mgf bound, $\En_{\eps}e^{\lambda{}\eps}\leq{}e^{\lambda^{2}/2}$, we upper bound the above quantity by
\begin{align*}
\exp\prn*{\frac{\Psi(\tau_{2}) + \frac{\beta}{2}\nrm*{x}^{2}}{an} + \frac{\tri*{\grad\Psi(\tau_2), x}^{2}}{2(an)^{2}}}\leq{}    \exp\prn*{\frac{\Psi(\tau_{2}) + \frac{\beta}{2}\nrm*{x}^{2}}{an} + \frac{\nrm*{\grad\Psi(\tau_2)}_{\star}^{2}\nrm*{x}^{2}}{2(an)^{2}}}.
\end{align*}
Using the assumption $\nrm*{x}\leq{}1$, we obtain an upper bound of
\begin{align*}
&\exp\prn*{\frac{\Psi(\tau_{2}) + \frac{\beta}{2}}{an} + \frac{\nrm*{\grad\Psi(\tau_2)}_{\star}^{2}}{2(an)^{2}}}.
\end{align*}
We now use a basic fact from convex analysis, namely that any $\beta$-smooth convex function $f$, $\frac{1}{2\beta}\nrm*{\grad{}f(x) - \grad{}f(y)}_{\star}^{2} \leq{} f(x) - f(y) - \tri*{\grad{}f(y), x-y}$ . This yields an upper bound
\begin{align*}
&\exp\prn*{\frac{\Psi(\tau_{2}) + \frac{\beta}{2}}{an} + \frac{\beta\Psi(\tau_2)}{(an)^{2}}}
\end{align*}
Setting $a=\beta$, this is equal to
\begin{align*} 
	\exp\prn*{\frac{1}{\beta}\prn*{\frac{1}{n} + \frac{1}{n^2}}\Psi(\tau_2) + \frac{1}{2n}}.
\end{align*}
  As a last step, observe that $\frac{1}{n} + \frac{1}{n^{2}} \leq{} \frac{1}{n-1}$. Indeed,
  \[
    \frac{1}{n} + \frac{1}{n^{2}} = \frac{1}{n}\prn*{1 + \frac{1}{n}} = \frac{1}{n-1}\frac{n-1}{n}\prn*{1+\frac{1}{n}}
    = \frac{1}{n-1}\prn*{1-\frac{1}{n}}\prn*{1+\frac{1}{n}}
    = \frac{1}{n-1}\prn*{1 - \frac{1}{n^{2}}}\leq{} \frac{1}{n-1}.
  \]
  Therefore, we have established that
  \[
    \En_{\eps}\brk*{F_{n}(\tau_{2} + \eps{}x_n)} \leq{} \exp\prn*{ \frac{\Psi(\tau_2)}{\beta(n-1)} + \frac{1}{2n}} = F_{n-1}(\tau_2),
  \]
  and in particular $\En_{\eps}\burk_{n}(\tau + \suff(z,\eps))\leq{}\burk_{n-1}(\tau)$.
  In fact, by folding the terms $\frac{1}{2}\sum_{s=t+1}^{n}\frac{1}{s}$---which do not depend on data---into a multiplicative constant, this argument yields, for any $t$ and any $\nrm*{x}\leq{}1$,
  \[
    \En_{\eps}\brk*{F_{t}(\tau + \eps{}x)} \leq{} F_{t-1}(\tau).
  \]
  Thus, for each $t\geq{}2$ we have
  \[
    \En_{\eps}\left[\burk_{t}(\tau + \suff(z,\eps))\right] = \En_{\eps}\brk*{\tau_{1} +  \eps\cdot\pred + F_{n}(\tau_{2} + \eps{}x)} - c \leq{} \burk_{t-1}(\tau).
  \]
  
The argument also yields (by removing unnecessary steps):
  \[
    \En_{\eps}\brk*{F_{1}(0 + \eps{}x)} \leq{} \gamma\exp\prn*{\frac{1}{2}\sum_{t=1}^{n}\frac{1}{t}} = F_{0}.
  \]
  This means that
  \[
    \burk_{0}(0) = \gamma\exp\prn*{\frac{1}{2}\sum_{t=1}^{n}\frac{1}{t}} - c \leq{} \gamma\exp\prn*{\log(n)/2} - c.
  \]
  We will set $\gamma=\frac{1}{\sqrt{n}}$ and $c=1$, which yields $\burk_{0}(0) \leq{} 0$.
  
\end{proof}

\subsection{Proofs from \pref{app:necessary}}

\begin{proof}[\pfref{prop:necessary}]
Recall that the regret inequality of interest is
\[
\sum_{t=1}^{n}\ls(\yh_t, y_t) - \inf_{f\in\cF}\sum_{t=1}^n \loss(f(x_t),y_t) - F\prn*{\sum_{t=1}^{n}\overline{\suff}(x_t)} \leq{} 0.
\]
As sketched in the \pref{app:necessary}, \pref{lem:suff_to_martingale} shows that this is implied by
\begin{equation}
\sup_{\mb{z}}\En_{\eps}\brk*{V\prn*{\sum_{t=1}^{n}\suff(\mb{z}_t, \eps_t)}}\leq{}0,
\end{equation}
so the remainder of this proof will focus on the opposite direction. Suppose that $\ls(\yh,y)\ldef\abs*{\yh-y}$ is the absolute loss. We fix a Rademacher sequence $\eps_{1},\ldots,\eps_{n}$ and a tree $\x$ with $\x_{t}(\eps)=\x_{t}(\eps_{1},\ldots,\eps_{t-1})$. As a lower bound, consider a randomized adversary that plays $y_{t}=\eps_{t}$ and $x_{t} = \x_{t}(\eps)$. In this case the expected value of the regret inequality is
\[
\En_{\eps}\brk*{
\sum_{t=1}^{n}\ls(\yh_t, \eps_t) - \inf_{f\in\cF}\sum_{t=1}^n \loss(f(\x_t(\eps)),\eps_t) - F\prn*{\sum_{t=1}^{n}\overline{\suff}(\x_t(\eps))}
}.
\]
Observe that for any $\eps\in\pmo$ we have $\ls(\yh, \eps) = \abs*{1-\yh\eps}\geq{}1-\yh\eps$. Since the range of each $f\in\cF$ lies in $\brk*{-1,1}$, we have $\ls(f(x), \eps)=1-f(x)\eps$ exactly. The expected value of the regret inequality is therefore lower bounded by
\begin{align*}
&\En_{\eps}\brk*{
\sum_{t=1}^{n}(1-\yh_t\eps_t) - \inf_{f\in\cF}\sum_{t=1}^n (1-f(\x_t(\eps))\eps_t) - F\prn*{\sum_{t=1}^{n}\overline{\suff}(\x_t(\eps))}
} \\
&= \En_{\eps}\brk*{
- \inf_{f\in\cF}\sum_{t=1}^n (1-f(\x_t(\eps))\eps_t) - F\prn*{\sum_{t=1}^{n}\overline{\suff}(\x_t(\eps))}
} \\
&= \En_{\eps}\brk*{
\sup_{w\in\cW}\tri*{w, \sum_{t=1}^n\eps_{t}\x_{t}(\eps)} - F\prn*{\sum_{t=1}^{n}\overline{\suff}(\x_t(\eps))}
} \\
&= \En_{\eps}\brk*{
V\prn*{\sum_{t=1}^{n}\suff(\x_{t}(\eps), 0, \eps_t)
}}.
\end{align*}

For the final step, let $\wt{\y}$ be an arbitrary $\cY$-valued tree $\wt{\y}_{t}(\eps) = \wt{\y}_{t}(\eps_1,\ldots,\eps_{t-1})$. Using the explicit form for $V$, we have
\begin{align*}
\En_{\eps}\brk*{
V\prn*{\sum_{t=1}^{n}\suff(\x_{t}(\eps), \wt{\y}_{t}(\eps), \eps_t)
}}
&= \En_{\eps}\brk*{
\sum_{t=1}^{n}\eps_{t}\wt{\y}_{t}(\eps) + 
\sup_{w\in\cW}\tri*{w, \sum_{t=1}^n\eps_{t}\x_{t}(\eps)} - F\prn*{\sum_{t=1}^{n}\overline{\suff}(\x_t(\eps))}
} \\
&= \En_{\eps}\brk*{
0 + 
\sup_{w\in\cW}\tri*{w, \sum_{t=1}^n\eps_{t}\x_{t}(\eps)} - F\prn*{\sum_{t=1}^{n}\overline{\suff}(\x_t(\eps))}
} \\
& = \En_{\eps}\brk*{
V\prn*{\sum_{t=1}^{n}\suff(\x_{t}(\eps), 0, \eps_t)
}}.
\end{align*}

Since the argument above holds for any trees $\x$ and $\wt{\y}$, we conclude that the regret inequality implies that
\[
\sup_{\mb{z}}\En_{\eps}\brk*{V\prn*{\sum_{t=1}^{n}\suff(\mb{z}_t, \eps_t)}}\leq{}0.
\]
for all $\cX\times{}\cY$-valued trees.

\end{proof}

%%% Local Variables:
%%% mode: latex
%%% TeX-master: "paper"
%%% End:




\section{Burkholder Algorithm Implementation}
\label{app:efficient}

% !TEX root = paper.tex

\subsection{Generic Implementation}
In this section we assume that $\cY=\brk*{-B, B}$ for $B>0$ for simplicity. The only assumption we make on the form of $\burk$ is Lipschitzness and boundedness.

\begin{assumption}
The are constants $K_t$ and $H_t$ such that the mapping
\[
\yh\mapsto{}\burk\Big(\zeta_{t-1} + \suff(x_t,\pred,\partial \loss(\pred,y_t))\Big)
\]
is $K_t$-Lipschitz and bounded in magnitude by $H_t$ for any $y_t\in\cY$, $x_t\in\cX$, and $\zeta_{t-1}$ of the form $\zeta_{t}=\sum_{s=1}^{t}\suff(x_s, \pred_s, \partial(\pred_s, y_s))$.
\end{assumption}

Consider the following strategy:
\begin{itemize}
\item Fix precision $\veps_{1}>0$ and set $N=\ceil*{2B/\veps_{1}}$.
\item Define control points $z_{i} = \min\crl*{-B + \veps_{1}\cdot{}i, B}$ for $0\leq{}i\leq{}N$.
\item Let $\widehat{\mu}_{t}$ be a solution to the convex program
\begin{equation}
\label{eq:burkholder_approx}
\min_{\mu\in\Delta_{N}}\sup_{y\in\cY}\sum_{i=1}^{N}\mu_{i}\burk\Big(\zeta_{t-1} + \suff(x_t,z_{i},\partial \loss(z_{i},y))\Big)
\end{equation}
up to additive precision $\veps_{2}$.
\item Sample $\pred_{t}\sim{}\wh{\mu}_{t}$.
\end{itemize}

\begin{proposition}
\label{prop:burkholder_efficient}
Given a Burkholder function $\burk$, the strategy above guarantees
\[
\En\brk*{\sum_{t=1}^n \loss(\pred_t,y_t)} - \phi(x_1,y_1,\ldots,x_n,y_n) \leq{} \veps_{1}\sum_{t=1}^{n}K_t + \veps_{2}n.
\]
That is, the regret inequality \pref{eq:def_phi_regret} is obtained up to additive slack controlled by $\veps_{1}$ and $\veps_{2}$.
\end{proposition}
Before proving the theorem, let us discuss the computational prospects of implementing this strategy. First, suppose $K_t=K$ and $H_{t}=H\;\forall{}t\leq{}n$. To obtain the regret inequality up to constant error it suffices to take $\veps_{1}=1/Kn$ and $\veps_{2}=1/n$. In this case, we have $N=O(BKn)$. 

Now we must approximately solve \pref{eq:burkholder_approx}, which is a standard finite-dimensional convex non-smooth optimization problem. There are many possible solvers; we will choose Mirror Descent (e.g. \citep{nemirovskii1983problem,nesterov1998introductory,ben2001lectures}) for simplicity. Let $G(\mu)=\sup_{y\in\cY}\sum_{i=1}^{N}\mu_{i}\burk\Big(\zeta_{t-1} + \suff(x_t,z_{i},\partial \loss(z_{i},y))\Big)$. Our constraint set is $\ls_1$-bounded, and the boundedness assumption on $\burk$ implies that $G$ is $H$-Lipschitz with respect to the $\ls_{\infty}$ norm. In this case, Mirror Descent with the entropic regularizer (a.k.a. multiplicative weights) guarantees an $\veps$-approximate minimizer for $G(\mu)$ after $O\prn*{H\log(N)/\veps^{2}}$ update steps, each of which requires one evaluation of the subgradient of this function.

Evaluating the subgradient of $G(\mu)$ requires computing a supremum over $y\in\cY$. If $\burk\Big(\zeta_{t-1} + \suff(x_t,z_{i},\partial \loss(z_{i},y))\Big)$ is convex with respect to $y$, then the supremum is obtained in $\crl*{\pm{}B}$ and so can be checked in time $O(N)$. In this case, since each Mirror Descent update takes time $O(N)$, the total complexity of the algorithm is $O(BHKn^{3}\log(BKn))$.

If the supremum over $y\in\cY$ does not have a closed form, we can compute an approximate subgradient by taking a grid over the range $\brk*{-B,B}$ with spacing $\veps'$ and computing the $\argmax$ over this grid by brute force. If a $O(\veps)$-precision solution to the convex program is required, then it suffices to set $\veps'=\veps/K$ and use the approximate subgradients in the Mirror Descent scheme above. The approximate subgradient computation time is $O(KN/\veps)$ in this case, since we evaluate $\sum_{i=1}^{N}\mu_{i}\burk\Big(\zeta_{t-1} + \suff(x_t,z_{i},\partial \loss(z_{i},y))\Big)$ once per candiate $y$. The final time complexity is then $O(BHK^2n^{4}\log(BKn))$.

Lastly, we remark that if we replace Mirror Descent with Mirror Prox for saddle points \citep{nemirovski2004prox}, the dependence on $n$ in running time for the two cases above can be improved to $O(n^{2})$ and $O(n^{3})$ respectively.

The runtime can improved further if a regret bound of order $O(\sqrt{n})$ is sufficient, as this requires less precision.

\begin{proof}[\pfref{prop:burkholder_efficient}]

To begin, observe that since $\wh{\mu}_{t}$ is an approximate solution to \pref{eq:burkholder_approx}, it holds that
\[
\sup_{y\in\cY}\sum_{i=1}^{N}\wh{\mu}_{i}\burk\Big(\zeta_{t-1} + \suff(x_t,z_{i},\partial \loss(z_{i},y_t))\Big)
\leq{}
\inf_{\mu\in\Delta_{N}}\sup_{y\in\cY}\sum_{i=1}^{N}\mu_{i}\burk\Big(\zeta_{t-1} + \suff(x_t,z_{i},\partial \loss(z_{i},y_t))\Big)
+ \veps_{2}.
\]
The remainder of the proof will show that the right-hand-side above can be bounded as
\begin{align*}
&\inf_{\mu\in\Delta_{N}}\sup_{y\in\cY}\sum_{i=1}^{N}\mu_{i}\burk\Big(\zeta_{t-1} + \suff(x_t,z_{i},\partial \loss(z_{i},y_t))\Big) \\
&\leq{} \inf_{q\in\Delta_{\cY}}\sup_{y\in\cY}\En_{\yh\sim{}q}\burk\Big(\zeta_{t-1} + \suff(x_t,\pred,\partial \loss(\pred,y))\Big)
+ K_{t}\veps_{1} \\
&\leq{} \burk(\zeta_{t-1})
+ K_{t}\veps_{1},
\end{align*}
where the second inequality follows from property \propthree{} of $\burk$ and was shown in the proof of \pref{lem:universal_algo}.

The first inequality can be seen as follows. Let $q\in\Delta_{\cY}$ and $y\in\cY$ be fixed. Let $F(z)\ldef{}\burk(\zeta_{t-1}, \suff(x_t, z, \partial\ls(z, y)))$. Since $q$ is a Borel probability measure and $F$ is continuous and bounded, $F$ is integrable with respect to $q$:
\[
\En_{\yh\sim{}q}\burk(\zeta_{t-1}, \suff(x_t, \yh, \partial\ls(\yh, y))) = \int_{\brk*{-B, B}}F(z)dq(z).
\]
Define $\cI_{1}=\brk*{z_0, z_1}$ and $\cI_{i}=(z_{i-1}, z_i]$ for $2\leq{}N$. Then $\crl*{\cI_i}$ form a partition of $\brk*{-B,B}$ and the integral can be approximated as
\begin{align*}
\int_{\brk*{-B, B}}F(z)dq(z) &= \sum_{i=1}^{N}\int_{\cI_{i}}F(z)dq(z) \\
&\geq{} \sum_{i=1}^{N}\int_{\cI_{i}}F(z_i)dq(z) - \sum_{i=1}^{N}\int_{\cI_{i}}\abs*{F(z_i)-F(z)}dq(z) \\
&= \sum_{i=1}^{N}q(\cI_i)F(z_i) - \sum_{i=1}^{N}\int_{\cI_{i}}\abs*{F(z_i)-F(z)}dq(z)\\
&\geq{} \sum_{i=1}^{N}q(\cI_i)F(z_i) - \sum_{i=1}^{N}\int_{\cI_{i}}K_{t}\veps_1dq(z)\\
&= \sum_{i=1}^{N}q(\cI_i)F(z_i) - K_{t}\veps_1\sum_{i=1}^{N}q(\cI_i) \\
&= \sum_{i=1}^{N}q(\cI_i)F(z_i) - K_{t}\veps_1.
\end{align*}
Since this holds for any $q\in\Delta_{\cY}$ and $y\in\cY$, we have
\begin{align*}
&\inf_{q\in\Delta_{\cY}}\sup_{y\in\cY}\En_{\yh\sim{}q}\burk\Big(\zeta_{t-1} + \suff(x_t,\pred,\partial \loss(\pred,y))\Big) \\
&\geq{} \inf_{q\in\Delta_{\cY}}\sup_{y\in\cY}\sum_{i=1}^{n}q(\cI_i)\burk\Big(\zeta_{t-1} + \suff(x_t,z_i,\partial \loss(z_i,y))\Big) - K_t\veps_1 \\
&= \inf_{\mu\in\Delta_{N}}\sup_{y\in\cY}\sum_{i=1}^{n}\mu_i\burk\Big(\zeta_{t-1} + \suff(x_t,z_i,\partial \loss(z_i,y))\Big) - K_t\veps_1.
\end{align*}

\end{proof}

\subsection{Faster Implementation under Specific Structure}

In the remainder of this section of the appendix we show how to implement the Burkholder algorithm for certain special cases that enable admit especially simple strategies.

\begin{lemma}
  \label{lem:det_strat2}
  Suppose that the map
  \[
    \pred \mapsto \burk(\tau + \suff((x,\yh) , \partial(\pred, y)))
  \]
  is convex for all $y$. Then the strategy
\begin{align}
	\label{eq:det_strat2}
  \pred_{t}=\argmin_{\pred\in\cY}\sup_{y\in\cY} ~\burk\left(\sum_{j=1}^{t-1} \zeta_{t-1} + \suff(x_t,\yh,\partial \loss(\pred,y))\right)
\end{align}
achieves the value of the game in \pref{lem:universal_algo}.
\end{lemma}
\begin{proof}[\pfref{lem:det_strat2}]
  This follows by reduction to the general case:
  \begin{align*}
	  \inf_{\pred\in\cY}\sup_{y\in\cY} \burk\left(\zeta_{t-1} + \suff(x_t,\pred,\partial \loss(\pred,y))\right)
    &= \inf_{q\in\Delta_{\cY}}\sup_{y\in\cY} \burk\left(\zeta_{t-1} + \suff(x_t,\En_{\pred\sim{}q}\brk*{\pred},\partial \loss(\En_{\pred\sim{}q}\brk*{\pred},y))\right) \\
    &\leq \inf_{q\in\Delta_{\cY}}\sup_{y\in\cY} \En_{\yh\sim{}q}\burk\left(\zeta_{t-1} + \suff(x_t,\yh,\partial \loss(\yh,y))\right).
  \end{align*}
  The strategy in \pref{eq:det_strat2} is the minimax strategy for second expression above. The final expression is precisely the value of the Burkholder algorithm, which is controlled when $\burk$ is a Burkholder function via \pref{lem:universal_algo}.
\end{proof}
  
  \begin{lemma}
  \label{lem:det_strat3}
  Suppose that $\cY=\brk*{-B, B}$ for some $B>0$. Further suppose that we can write 
  \[
	\burk(\tau + \suff((x,\yh) , \dl)) = \yh\cdot\dl + F(\tau, x, \dl),
  \]
  where $\dl\mapsto{}F(\tau, x, \dl)$ is convex for all $\tau,x$.
Then the prediction strategy
  \begin{align}
	\label{eq:det_strat3}
  \pred_{t}= \mathrm{proj}_{\brk*{-B,B}}\prn*{-\frac{1}{L}\En_{\sigma\in\pmo}\brk*{
  \sigma{}F(\zeta_{t-1}, x_{t}, L\sigma)
  }},
  \end{align}
achieves the value of the game in \pref{lem:universal_algo}.
\end{lemma}

\begin{proof}[\pfref{lem:det_strat3}]
Let $\wt{y}_t$ denote the unprojected version of $\pred_t$:
\[
  \wt{y}_{t}= - \frac{1}{L}\En_{\sigma\in\pmo}\brk*{
  \sigma{}F(\zeta_{t-1}, x_{t}, L\sigma)
  }.
\]
We prove the lemma by inducting backwards. Let $t\in\brk*{n}$ be fixed. We first claim that
\begin{align*}
\sup_{y\in\cY} \burk\left(\zeta_{t-1} + \suff(x_t,\pred_t,\partial \loss(\pred_t,y))\right)
& = 
\sup_{y\in\cY}\brk*{\pred_t\cdot\partial \loss(\pred_t,y) + F(\zeta_{t-1}, x_t, \partial \loss(\pred_t,y))} \\ 
& \leq 
\sup_{y\in\cY}\brk*{\predt_t\cdot\partial \loss(\pred_t,y) + F(\zeta_{t-1}, x_t, \partial \loss(\pred_t,y))}.
\end{align*}
This holds by the assumption that $\argmin_{\yh\in\bbR}\ls(\yh, y)$ is obtained in $\brk*{-B, B}$ for any $y$. The assumption implies that for any $y$, $\partial\ls(\yh, y)\geq{}0$ for $\pred\geq{}B$ and $\partial\ls(\yh, y)\leq{}0$ for $\pred\leq{}-B$. If $\yh_{t}\neq{}\predt_t$, then either $\yh_{t}=B$ and $\predt_{t}>B$, so that $\partial\ls(\yh_{t}, y)\yh_{t}\leq{}\partial\ls(\yh_{t}, y)\predt_{t}$, or similarly $\yh_{t}=-B$ and $\predt_{t}<-B$, which also implies $\partial\ls(\yh_{t}, y)\yh_{t}\leq{}\partial\ls(\yh_{t}, y)\predt_{t}$.

Now, by the convexity assumption of the lemma, it holds that
\begin{align*}
\sup_{y\in\cY}\brk*{\predt_t\cdot\partial \loss(\pred_t,y) + F(\zeta_{t-1}, x_t, \partial \loss(\pred_t,y))}
&\leq{} \sup_{\delta\in\brk*{-L, L}}\brk*{\predt_t\cdot\delta + F(\zeta_{t-1}, x_t, \delta)} \\
&= \max_{\sigma\in\pmo}\brk*{\predt_t\cdot{}L\sigma + F(\zeta_{t-1}, x_t, L\sigma)}.
\end{align*}

The choice of $\predt_t$ guarantees that $\predt_t\cdot{}L\cdot(1) + F(\zeta_{t-1}, x_t, L\cdot(1)) = \predt_t\cdot{}L\cdot{}(-1) + F(\zeta_{t-1}, x_t, L\cdot(-1))$; this can be seen by rearranging this equality and solving for $\predt_t$. This means that we can take $\sigma=1$ to obtain the maximum in the expression above. Substituting in the value of $\predt_{t}$ then yields
\[
\max_{\sigma\in\pmo}\brk*{\predt_t\cdot{}L\sigma + F(\zeta_{t-1}, x_t, L\sigma)}
=\predt_t\cdot{}L\cdot(1) + F(\zeta_{t-1}, x_t, L\cdot(1)) = \En_{\sigma\in\pmo}\brk*{F(\zeta_{t-1}, x_t, \sigma{}L)}.
\]
Finally, we use property \propthreep{} of $\burk$ and the explicit form for $\burk$ assumed in the lemma statement to proceed back to time $t-1$:
\begin{align*}
\En_{\sigma\in\pmo}\brk*{F(\zeta_{t-1}, x_{t}, \sigma{}L)} &= \En_{\sigma\in\pmo}\brk*{\yh_{t}\sigma{}L+F(\zeta_{t-1}, x_{t}, \sigma{}L)} \\
&= \En_{\sigma\in\pmo}\burk(\zeta_{t-1} + \suff((x_t,\yh_t) , \sigma{}L)) \\ &\leq{} \burk(\zeta_{t-1}).
\end{align*}
\end{proof}


\section{Algebra of \Bfun Functions}
\label{app:algebra}
% !TEX root = paper.tex

This appendix contains some additional structural results about Burkholder functions which may be useful for algorithm designers.

\begin{proposition}
\label{prop:algebra}
The following statements are true:
\begin{enumerate}
\item Given a \Bfun function $\burk$, if we define the $X_t = \burk(\sum_{j=1}^t \suff(z_j,\delta_j))$, then for any real-valued martingale difference sequence $\delta_t$s and predictable $z_t$s, $(X_t)_{t \ge 0}$ is a supermartingale with $\mathbb{E}[X_0] \le 0$.
\item Any convex combination of \Bfun functions is a \Bfun function.
\item The minimum of a family of \Bfun functions is a \Bfun function.
\item Suppose we have a finite set $A$ that indexes a family of functions $V_a:\cT\to\bbR$, each of which belongs to a sufficient statistic pair $(\suff, V_a)$ for some regret inequality of interest, and suppose each $V_a$ has a corresponding \Bfun function $\burk_a$.  Then the following probabilistic inequality is true:
$$
\mathbb{E}\left[\max_{a \in A} \left\{V_a\prn*{\sum_{t=1}^n \suff(z_t,\delta_t)} - \eta n C[a] \right\}\right]  \le \frac{1}{\eta}\log |A|,
$$
where $C[a] = \sup_{\tau, z ,\alpha} (\burk_a(\tau + \suff(z,\alpha)) - \burk_a(\tau))^2$. Note that $C \in \mathbb{R}^A$ may be thought as a sufficient statistic, though it is fixed and does not depend on instances.
Furthermore, a \Bfun function $\burk: \T \times \mathbb{R}^A \to \reals$ that certifies this inequality is:
\begin{equation}
\label{eq:burkholder_meta}
\burk(\tau, \gamma) = \frac{1}{\eta} \log\left(\sum_{a \in A} \exp\left(\eta \burk_a(\tau) - \eta^2 \gamma[a] \right)\right)  - \frac{\log|A|}{\eta}
\end{equation}
\end{enumerate}
\end{proposition}
\begin{proof}[\pfref{prop:algebra}]
The first statement follows from property \propthree{} of the Burkholder function $\burk$, which immediately implies that it is a supermartingale. The second statement is trivial. To prove the third statement it suffices to verify property \propthree, which holds due to concavity of the minimum. 

We now prove the fourth statement. Given a family of \Bfun functions $\crl*{\burk_a}_{a\in{}A}$, define a new \Bfun function $\burk: \T \times \mathbb{R}^A \to\reals$ as:
$$
\burk(\tau, \gamma) = \frac{1}{\eta} \log\left(\sum_{a \in A} \exp\left(\eta \burk_a(\tau) - \eta^2 \gamma[a] \right)\right)  - \frac{\log|A|}{\eta}.
$$
whose sufficient statistics are the original sufficient statistic of the family of $V_a$s along with an additional $|A|$-dimensional real vector, for which one coordinate per $a \in A$ will be used to represent $C[a] = \sup_{\tau, z ,\alpha} (\burk_a(\tau + \suff(z,\alpha)) - \burk_a(\tau))^2$ (note that this is a vacuous statistic as it is constant for each instance). Property \propthree{} for $\burk$ holds as follows:
\begin{align*}
 \En_\alpha &\burk\left((\tau,\gamma) + (\suff(z,\alpha) , C) \right) \\
 & = \frac{1}{\eta} \En_\alpha \log\left(\sum_{a \in A} \exp\left(\eta \burk_a(\tau + \suff(z,\alpha)) - \eta^2 \gamma[a] - \eta^2 C[a]\right)\right) - \frac{\log|A|}{\eta} \\
 & \le  \frac{1}{\eta}  \log\left(\sum_{a \in A} \En_\alpha \exp\left(\eta \burk_a(\tau + \suff(z,\alpha))  - \eta^2 \gamma[a] - \eta^2 C[a] \right)\right) - \frac{\log|A|}{\eta} \\
 & =  \frac{1}{\eta}  \log\left(\sum_{a \in A} \En_\alpha \exp\left(\eta \left(\burk_a(\tau + \suff(z,\alpha)) - \burk_a(\tau) \right) + \eta\burk_a(\tau)  - \eta^2 \gamma[a] - \eta^2 C[a]\right)\right) - \frac{\log|A|}{\eta}.
\end{align*}
Now note that by property \propthree{} of the \Bfun functions $\crl*{\burk_a}_{a\in{}A}$, the random variable $X_a = \left(\burk_a(\tau + \suff(z,\alpha)) - \burk_a(\tau) \right)$ is such that $\En_{\alpha}[X_a] \le 0$. Further from our assumption we have that $|X_a|^2 \le C[a]$. Hence, the standard mgf bound implies $\mathbb{E}_{\alpha}[\exp(\eta X_a)] \le \exp(\eta^2 C[a]/2)$.
\begin{align*}
 & \le  \frac{1}{\eta}  \log\left(\sum_{a \in A} \exp\left(\eta\burk_a(\tau)  + \frac{\eta^2}{2} C[a]  - \eta^2 \gamma[a] - \eta^2 C[a] \right)\right) - \frac{\log|A|}{\eta} \\
 & \le \frac{1}{\eta}  \log\left(\sum_{a \in A} \exp\left(\eta\burk_a(\tau)   - \eta^2 \gamma[a] \right)\right) - \frac{\log|A|}{\eta}.
\end{align*}
For property \propone{} it can be seen immediately that $\burk(0) \le 0$. Property \proptwo{} holds via
\begin{align*}
 \burk(\tau,\gamma) &= \frac{1}{\eta} \log\left(\sum_{a \in A} \exp\left(\eta \burk_a(\tau) - \eta^2 \gamma[a] \right)\right)  - \frac{\log|A|}{\eta}\\
& \ge \max_{a \in A}\left\{ \burk_a(\tau) - \eta \gamma[a]\right\} - \frac{\log|A|}{\eta} ~~~~~~ \textrm{(softmax upper bounds max)}\\
& \ge \max_{a \in A}\left\{ V_a(\tau) - \eta \gamma[a]\right\} - \frac{\log|A|}{\eta}.
\end{align*}
\end{proof}
We remark that one uses non-additive sufficient statistics as discussed in \pref{app:discussion}, then one can make the bound implied by the Burkholder function $\burk$ above more data-dependent by replacing $C[a]$ with $\sup_{\delta} \left(\burk_a(\tau + \suff(z,\delta)) - \burk_a(\tau) \right)^2$ for each $a$. 



  
\end{document}
