% !TEX root = paper.tex

The core techniques developed in this paper suggest a number of promising future directions and natural extensions.

\paragraph{Finding sufficient statistics} This paper gives multiple examples of Burkholder function constructions and sufficient statistics. If one wishes to find sufficient statistics for an adaptive bound $\cA$ of interest, a basic rule of thumb is to consider a single input instance (instead of all $n$ data points) and determine---say---a polynomial expansion or expansion in another basis for the terms in $\mathrm{Reg}_{n}-\cA$ involving the instance. This gives a coarse sketch of which statistics are necessary. 

As an example, take the standard square loss with linear predictors as the benchmark class and suppose we are interested in a non-adaptive bound. Following the heuristic above, we need to find an expansion for terms of the form ``$(\hat{y} - y)^2 - (\tri*{w,x} - y)^2 - ~\mathsf{constant}$''. Expanding this expression out, we find that $\hat{y}^2$, $y \cdot x$ and $x x^\top$ are all required to write the expression explicitly. In fact, for this square loss example, the weighted sum of the $x_t$s and the sum of the outer products $\sum_t x_t x_t^\top$ turn out to be sufficient statistics as well.  

For the examples in this paper, we exclusively considered benchmark classes $\F$ that were linear, which appears to have made the search for sufficient statistics easier. However, even when one considers a class $\F$ of non-linear functions, the approach of trying to expand the desired regret inequality (which now involves nonlinear $f \in \F$) around a given instance $x$ in terms of some basis may still help to obtain an adequate sufficient statistics. Furthermore, one may enlarge the class $\F$ to make the sufficient statistic search easier. For instance, if we want to learn the class of boolean decision trees of depth $d$, we can exploit that the class can be represented by polynomials of degree $d$ by using the discrete Fourier coefficients of the input instances up to degree $d$ as a sufficient statistic. In summary, for non-linear classes one may still search for sufficient statistics and Burkholder functions by expressing nonlinearities (approximately) via linear combinations of higher-order terms. 

\paragraph{Toward plug-and-play online learning}
A natural next step is to automatize the search for sufficient statistics and \Bfun functions. Suppose that the sufficient statistic pair $(\suff, V)$ is fixed and all that remains is to find a Burkholder function $\burk$. If $V$ can be written as a polynomial of degree over sufficient statistic space $\cT$, a natural approach is to restrict the search to Burkholder functions $\burk$ that are themselves polynomials and relax the inequalities \propone/\proptwo/\propthree{} to sum-of-squares constraints \citep{barak2014sum}. We can then jointly search for a function $\burk$ and a degree-$d$ sum-of-squares proof that this function satisfies the three properties in polynomial time once the degree of $\burk$ is fixed. As a specific example, the problem of finding the zig-zag concave Burkholder function for $\ell_p$ norms explored in \cite{foster2017zigzag} has a sufficient statistic $V$ that is a polynomial of degree $p$ when $p\geq{}2$ is an integer. 

This approach is sound in that it will never incorrectly return a function $\burk$ that does not satisfy the three properties, but may not be complete a-priori. An interesting direction is therefore to explore whether there are conditions under which this system can indeed be made complete.

\paragraph{Generalized/non-additive sufficient statistics} The restriction in \pref{def:sufficiency} that sufficient statistics combine additively can be relaxed. A more general form is as follows. First, define a \emph{representation space} $\cT$. The function $\suff$ now takes the form:
\[
  \suff: \cX\times{}\cY\times{}\brk*{-L, L}\times{}\cT \to \cT.
\]
The restricted concavity condition for $\burk$ under this definition becomes
\[
\forall{}z, \tau:\quad\sup_{\En\brk*{\alpha}=0}\En_{\alpha}\burk\prn*{\suff\prn*{z, \alpha, \tau}} \leq{} \burk(\tau).
\]
Properties \propone{} and \proptwo{} of \pref{lem:equivalence_burkholder} remain the same. This generalized notion of a sufficient statistic allows us to move beyond additive updates---$\suff$ can multiply $z$ with elements of $\cT$, for example---but still restricts storage to the space $\cT$ and is fully compatible with the Burkholder method and general algorithm framework. The generalizations of the equivalence theorem (\pref{lem:equivalence_burkholder}) and the Burkholder algorithm (\pref{lem:universal_algo}) for this notion of sufficient statistic hold as well.