\subsection{Completeness Analysis}
Suppose there is a labeling $\sigma :V \to [k]$ which satisfies all the edges of $\mc{L}$. Define $L^*(\mc{Y})
= \sum_{v \in V} Y^v_{\sigma(v)}$ to be a linear form. Note that $L^*({\bf y}) := \langle {\bf r}^*, {\bf y}\rangle$ for some ${\bf r}^* \in \mc{F}$, and so $L^*$ can be represented in an orthogonal basis for $\mc{F}$. Thus, for any point ${\bf y} \in \mathbb{R}^{\mc{Y}}$, $L^*(y) = L^*(\ol{\bf y})$ where $\ol{\bf y}$ is the projection of ${\bf y}$ on to $\mc{F}$ as defined in Appendix \ref{sec-folding}.

Now consider $({\bf y}, b)$ generated by the Basic PCP Test.
By a union bound over the randomness of the test,  with probability at least $(1 - \eps T)$:  $(\sigma(v_j), j) \not\in \mc{I}$ for each $j \in [T]$. Given this, it is easy to see that $L^*({\bf y}) = b$, and by the above reasoning $L^*(\ol{\bf y}) = b$.  
Thus, $L^*$ satisfies the Final PCP Test with probability at least $(1 - \eps T)$. Our choice of $\eps$ yields the desired accuracy.
