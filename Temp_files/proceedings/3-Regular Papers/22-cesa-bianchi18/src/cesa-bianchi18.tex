%\documentclass[anon,12pt]{colt2018} % Anonymized submission
\documentclass[final,12pt]{colt2018}% Include author names

% The following packages will be automatically loaded:
% amsmath, amssymb, natbib, graphicx, url, algorithm2e

\usepackage{times}
\usepackage{graphicx}
%\usepackage{amsmath,amsthm,amssymb}
%\usepackage{algorithm}
%\usepackage{algorithmic}
\usepackage{ifthen}
%\usepackage[algo2e,ruled]{algorithm2e}
\usepackage{wrapfig}
\usepackage{enumitem}

%\newtheorem{theorem}{Theorem}
%\newtheorem{lemma}[theorem]{Lemma}
%\newtheorem{cor}[theorem]{Corollary}
%\newtheorem{corollary}[theorem]{Corollary}
%\newtheorem{proposition}[theorem]{Proposition}
\newtheorem{fact}[theorem]{Fact}
%\newtheorem{definition}[theorem]{Definition}
%\newtheorem{remark}[theorem]{Remark}

\newcommand{\ncb}[1]{\textcolor{red}{NCB: #1}}

\newcommand{\scU}{\mathcal{U}}
\newcommand{\field}[1]{\mathbb{#1}}
\newcommand{\R}{\field{R}}
\newcommand{\C}{\field{C}}
\newcommand{\E}{\field{E}}
\renewcommand{\Pr}{\field{P}}
\newcommand{\Ind}[1]{\field{I}{\left\{#1\right\}}}
\newcommand{\Var}{\field{V}}
\newcommand{\dt}{\displaystyle}
\newcommand{\ve}{\varepsilon}
\renewcommand{\ss}{\subseteq}
\newcommand{\theset}[2]{ \left\{ {#1} \,:\, {#2} \right\} }
\newcommand{\inner}[1]{ \left\langle {#1} \right\rangle }
\newcommand{\norm}[1]{\|{#1}\|}
\newcommand{\argmin}{\mathop{\rm argmin}}
\newcommand{\argmax}{\mathop{\rm argmax}}
\newcommand{\defeq}{\stackrel{\rm def}{=}}
\newcommand{\sgn}{\mbox{\sc sgn}}
\newcommand{\trace}{\mathrm{Tr}}
\newcommand{\diag}{\mathrm{Diag}}
\newcommand{\scE}{\mathcal{E}}
\newcommand{\scO}{\mathcal{O}}
\newcommand{\scS}{\mathcal{S}}
\newcommand{\scF}{\mathcal{F}}
\newcommand{\scK}{\mathcal{K}}
\newcommand{\be}{\boldsymbol{e}}
\newcommand{\bg}{\boldsymbol{g}}
\newcommand{\bs}{\boldsymbol{s}}
\newcommand{\bx}{\boldsymbol{x}}
\newcommand{\by}{\boldsymbol{y}}
\newcommand{\bu}{\boldsymbol{u}}
\newcommand{\bv}{\boldsymbol{v}}
\newcommand{\bw}{\boldsymbol{w}}
\newcommand{\tbw}{\boldsymbol{{\tilde w}}}
\newcommand{\bp}{\boldsymbol{p}}
\newcommand{\bV}{\boldsymbol{V}}
\newcommand{\bX}{\boldsymbol{X}}
\newcommand{\bZ}{\boldsymbol{Z}}
\newcommand{\bzero}{\boldsymbol{0}}
\newcommand{\bool}{\{0,1\}}
\newcommand{\loss}{\ell}
\newcommand{\bloss}{\boldsymbol{\loss}}
\newcommand{\avgloss}{{\overline{\ell}}}
\newcommand{\avglossf}{{\overline{f}}}
\newcommand{\shat}{\widehat{\nu}}
\newcommand{\Lhat}{\widehat{L}}
\newcommand{\ellhat}{\widehat{\ell}}
\newcommand{\khat}{\widehat{k}}


\newlength{\minipagewidth}
\setlength{\minipagewidth}{\textwidth}
\setlength{\fboxsep}{3mm}
\addtolength{\minipagewidth}{-\fboxrule}
\addtolength{\minipagewidth}{-\fboxrule}
\addtolength{\minipagewidth}{-\fboxsep}
\addtolength{\minipagewidth}{-\fboxsep}
\newcommand{\bookbox}[1]{
\par\medskip\noindent
\framebox[\textwidth]{
\begin{minipage}{\minipagewidth}
{#1}
\end{minipage} } \par\medskip }


\title[Composite Anonymous Feedback]{Nonstochastic Bandits with Composite Anonymous Feedback}

 % Use \Name{Author Name} to specify the name.
 % If the surname contains spaces, enclose the surname
 % in braces, e.g. \Name{John {Smith Jones}} similarly
 % if the name has a "von" part, e.g \Name{Jane {de Winter}}.
 % If the first letter in the forenames is a diacritic
 % enclose the diacritic in braces, e.g. \Name{{\'E}louise Smith}

 % Two authors with the same address
  % \coltauthor{\Name{Author Name1} \Email{abc@sample.com}\and
  %  \Name{Author Name2} \Email{xyz@sample.com}\\
  %  \addr Address}

 % Three or more authors with the same address:
 % \coltauthor{\Name{Author Name1} \Email{an1@sample.com}\\
 %  \Name{Author Name2} \Email{an2@sample.com}\\
 %  \Name{Author Name3} \Email{an3@sample.com}\\
 %  \addr Address}


 % Authors with different addresses:
 \coltauthor{\Name{Nicol\`{o} Cesa-Bianchi} \Email{nicolo.cesa-bianchi@unimi.it} \\
 \addr Dipartimento di Informatica, Universit\`{a} degli Studi di Milano, Italy
 \AND
 \Name{Claudio Gentile} \Email{cla.gentile@gmail.com} \\
 \addr INRIA Lille Nord Europe (France) and Google LLC (USA)
 \AND
 \Name{Yishay Mansour} \Email{mansour.yishay@gmail.com} \\
 \addr Blavatnik School of Computer Science, Tel Aviv University and Google
 }


\begin{document}

\maketitle

\begin{abstract}
We investigate a nonstochastic bandit setting in which the loss of an action is not immediately charged to the player, but rather spread over at most $d$ consecutive steps in an adversarial way. This implies that the instantaneous loss observed by the player at the end of each round is a sum of as many as $d$ loss components of previously played actions. Hence, unlike the standard bandit setting with delayed feedback, here the player cannot observe the individual delayed losses, but only their sum. Our main contribution is a general reduction transforming a standard bandit algorithm into one that can operate in this harder setting. We also show how the regret of the transformed algorithm can be bounded in terms of the regret of the original algorithm. Our reduction cannot be improved in general: we prove a lower bound on the regret of any bandit algorithm in this setting that matches (up to log factors) the upper bound obtained via our reduction. Finally, we show how our reduction can be extended to more complex bandit settings, such as combinatorial linear bandits and online bandit convex optimization.
\end{abstract}

\begin{keywords}
Nonstochastic bandits, composite losses, delayed feedback, bandit convex optimization
\end{keywords}



Online learning algorithms are a key tool in web search and content optimization, adaptively learning what users want to see. In a typical application, each time a user arrives, the algorithm chooses among various content presentation options (\eg news articles to display), the chosen content is presented to the user, and an outcome (\eg a click) is observed. Such algorithms must balance \emph{exploration} (making potentially suboptimal decisions now for the sake of acquiring information that will improve decisions in the future) and \emph{exploitation} (using information collected in the past to make better decisions now). Exploration could degrade the experience of a current user, but improves user experience in the long run. This exploration-exploitation tradeoff is commonly studied in the online learning framework of \emph{multi-armed bandits}~\citep{Bubeck-survey12}.

Concerns have been raised about whether exploration in such scenarios could be unfair, in the sense that some individuals or groups may experience too much of the downside of exploration without sufficient upside \citep{bird2016exploring}. We formally study these concerns in the \emph{linear contextual bandits} model~\citep{Langford-www10,chu2011contextual}, a standard variant of multi-armed bandits appropriate for content personalization scenarios.  We focus on \emph{externalities} arising due to exploration, that is, undesirable side effects that the presence of one party may impose on another.


We first examine the effects of exploration at a group level.  We introduce the notion of a \emph{group externality} in an online learning system, quantifying how much the presence of one population (which we dub the majority) impacts the rewards of another (the minority). We show that this impact can be negative, and that, in a particular precise sense, no algorithm can avoid it. This cannot be explained by the absence of suitably good policies since our adoption of the linear contextual bandits framework implies the existence of a feasible policy that is simultaneously optimal for everyone. Instead, the problem is inherent to the process of exploration. We come to a surprising conclusion that more data can sometimes lead to worse outcomes for the users of an explore-exploit-based system. \looseness=-1

We next turn to the effect of exploration at an individual level. We interpret exploration as a potential externality imposed on the current user by future users of the system. Indeed, it is only for the sake of the future users that the algorithm would forego the action that currently looks optimal. To avoid this externality, one may use the greedy algorithm that always chooses the action that appears optimal according to current estimates of the problem parameters. While this greedy algorithm performs poorly in the worst case,
it tends to work well in many applications and experiments.\footnote{Both positive and negative findings are folklore. One way to precisely state the negative result is that the greedy algorithm incurs constant per-round regret with constant probability; while results of this form have likely been known for decades,
\citet[Corollary A.2(b)]{competingBandits-itcs16}
proved this for a wide variety of scenarios. Very recently, the good empirical performance has been confirmed by state-of-art experiments in \citet{practicalCB-arxiv18}.}

In a new line of work, \citet{bastani2017exploiting} and \citet{kannan2018smoothed}
analyzed conditions under which inherent diversity in the data makes explicit exploration unnecessary.
\citet{kannan2018smoothed} proved that the greedy algorithm achieves a regret rate of
$\tilde{O}(\sqrt{T})$ in expectation over small perturbations of the context vectors (which ensure sufficient data diversity). This is the best rate that can be achieved in the worst case (\ie for all problem instances, without data diversity assumptions), but it leaves open the possibilities that (i) another algorithm may perform much better than the greedy algorithm on some problem instances, or (ii) the greedy algorithm may perform much better than worst case under the diversity conditions. We expand on this line of work. We prove that under the same diversity conditions, the greedy algorithm almost matches the best possible Bayesian regret rate of \emph{any} algorithm \emph{on the same problem instance}. This could be as low as $\polylog(T)$ for some instances, and, as we prove, at most $\tilde{O}(T^{1/3})$ whenever the diversity conditions hold.


Returning to group-level effects, we show that under the same diversity conditions, the negative group externalities imposed by the majority essentially vanish if one runs the greedy algorithm. Together, our results illustrate a sharp contrast between the high individual and group externalities that exist in the worst case, and the ability to remove all externalities if the data is sufficiently diverse.   \looseness=-1

\xhdr{Additional motivation.} Whether and when explicit exploration is necessary is an important concern in the study of the exploration-exploitation tradeoff. Fairness considerations aside, explicit exploration is expensive. It is wasteful and risky in the short term, it adds a layer of complexity to algorithm design \citep{Langford-nips07,monster-icml14}, and its adoption at scale tends to require substantial systems support and buy-in from management \citep{MWT-WhitePaper-2016,DS-arxiv}. A system based on the greedy algorithm would typically be cheaper to design and deploy.

Further, explicit exploration can run into incentive issues in applications such as recommender systems. Essentially, when it is up to the users which products or experiences to choose and the algorithm can only issue recommendations and ratings, an explore-exploit algorithm needs to provide incentives to explore for the sake of the future users \citep{Kremer-JPE14,Frazier-ec14,Che-13,ICexploration-ec15,Bimpikis-exploration-ms17}. Such incentive guarantees tend to come at the cost of decreased performance, and rely on assumptions about human behavior. The greedy algorithm avoids this problem as it is inherently consistent with the users' incentives.



\xhdr{Additional related work.}
Our research draws inspiration from the growing body of work on fairness in machine learning~\cite[e.g.,][]{dwork2012fairness,hardt2016equality,kleinberg2017inherent,chouldechova2017fair}.  Several other authors have studied fairness in the context of the contextual bandits framework.  Our work differs from the line of research on meritocratic fairness in online learning \citep{kearns2017meritocratic,liu2017calibrated,joseph2016fairness}, which considers the allocation of limited resources such as bank loans and requires that nobody should be passed over in favor of a less qualified applicant. We study a fundamentally different scenario in which there are no allocation constraints and we would like to serve each user the best content possible.  Our work also differs from that of \citet{celis2017fair}, who studied an alternative notion of fairness in the context of news recommendations. According to this notion, all users should have approximately the same probability of seeing a particular type of content (e.g., Republican-leaning articles), regardless of their individual preferences, in order to mitigate the possibility of discriminatory personalization.

The data diversity conditions in \citet{kannan2018smoothed} and this paper are inspired by the smoothed analysis framework of \citet{SmoothedAnalysis-jacm04}, who proved that the expected running time of the simplex algorithm is polynomial for perturbations of any initial problem instance (whereas the worst-case running time has long been known to be exponential). Such disparity implies that very bad problem instances are brittle. 
We find a similar disparity for the greedy algorithm in our setting.



\xhdr{Our results on group externalities.}  A typical goal in online learning is to minimize \emph{regret}, the (expected) difference between the cumulative reward that would have been obtained had the optimal policy been followed at every round and the cumulative reward obtained by the algorithm.  We define a corresponding notion of \emph{minority regret}, the portion of the regret experienced by the minority.  Since online learning algorithms update their behavior based on the history of their observations, minority regret is influenced by the entire population on which an algorithm is run.  If the minority regret is much higher when a particular algorithm is run on the full population than it is when the same algorithm is run on the minority alone, we can view the majority as imposing a negative externality on the minority; the minority population would achieve a higher cumulative reward if the majority were not present. Asking whether this can ever happen
amounts to asking whether access to more data points can ever lead an explore-exploit algorithm to make inferior decisions. One might think that more data should always lead to better decisions and therefore better outcomes for the users.
Surprisingly, we show that this is not the case, even with a standard algorithm.

Consider LinUCB~\citep{Langford-www10,chu2011contextual,abbasi2011improved}, a standard algorithm for linear contextual bandits that is based on the principle of ``optimism under uncertainty.''  We provide a specific problem instance on which, after observing $T$ users, LinUCB would have a minority regret of $\Omega(\sqrt T)$ if run on the full population, but only constant minority regret if run on the minority alone. While stylized, this example is motivated by the problem of providing driving directions to different populations of users, about which fairness concerns have been raised~\citep{bird2016exploring}. Further, the situation is reversed on a slight variation of this example: LinUCB obtains constant minority regret when run on the full population and $\Omega(\sqrt T)$ on the minority alone.  That is, group externalities can be large and positive in some cases, and large and negative in others.

Although these regret rates are specific to LinUCB, we show that this phenomenon is, in some sense, unavoidable. Consider the minority regret of LinUCB when run on the full population and the minority regret that LinUCB would incur if run on the minority alone. We know that one may be much smaller or larger than the other. We ask whether any algorithm could  achieve the minimum of the two on every problem instance. Using a variation of the same problem instance, we prove that this is impossible; in fact, no algorithm could simultaneously approximate both up to any $o(\sqrt{T})$ factor. In other words, an externality-free algorithm would sometimes ``leave money on the table."


In terms of techniques, we rely on the special structure of our example, which can be viewed as an instance of the sleeping bandits problem~\citep{SleepingBandits-ml10}. This simplifies the behavior and analysis of LinUCB, allowing us to obtain the $O(1)$ upper bounds.  The lower bounds are obtained using KL-divergence techniques to show that the two variants of our example are essentially indistinguishable, and an algorithm that performs well on one must obtain $\Omega(\sqrt{T})$ regret on the other. \looseness=-1


\xhdr{Our results on the greedy algorithm.} We consider a version of linear contextual bandits in which the latent weight vector $\theta$ is drawn from a known prior. In each round, an algorithm is presented several actions to choose from, each represented by a \emph{context vector}. The expected reward of an action is a linear product of $\theta$ and the corresponding context vector. The tuple of context vectors is drawn independently from a fixed distribution. In the spirit of smoothed analysis, we assume that this distribution has a small amount of jitter. Formally, the tuple of context vectors is drawn from some fixed distribution, and then a small \emph{perturbation}---small-variance Gaussian noise---is added independently to each coordinate of each context vector. This ensures arriving contexts are diverse. We are interested in Bayesian regret, i.e., regret in expectation over the Bayesian prior. Following the literature, we are primarily interested in the dependence on the time horizon $T$. \looseness=-1

We focus on a batched version of the greedy algorithm, in which new data arrives to the algorithm's optimization routine in small batches, rather than every round. This is well-motivated from a practical perspective---in high-volume applications data usually arrives to the ``learner" only after a substantial delay \citep{MWT-WhitePaper-2016,DS-arxiv}---and is essential for our analysis.

Our main result is that the greedy algorithm matches the Bayesian regret of any algorithm up to polylogarithmic factors, for each problem instance, fixing the Bayesian prior and the context distribution. We also prove that LinUCB achieves regret $\tilde{O}(T^{1/3})$ for each realization of $\theta$. This implies a worst-case Bayesian regret of $\tilde{O}(T^{1/3})$ for the greedy algorithm under the perturbation assumption. \looseness=-1

Our results hold for both natural versions of the batched greedy algorithm, Bayesian and frequentist, henceforth called \BayesGreedy and \FreqGreedy. In \BayesGreedy, the chosen action maximizes expected reward according to the Bayesian posterior. \FreqGreedy estimates $\theta$ using ordinary least squares regression and chooses the best action according to this estimate. The results for \FreqGreedy come with additive polylogarithmic factors, but are stronger in that the algorithm does not need to know the prior. Further, the $\tilde{O}(T^{1/3})$ regret bound for \FreqGreedy is approximately prior-independent, in the sense that it applies even to very concentrated priors such as independent Gaussians with standard deviation on the order of $T^{-2/3}$.

The key insight in our analysis of \BayesGreedy is that any (perturbed) data can be used to simulate any other data, with some discount factor. The analysis of \FreqGreedy requires an additional layer of complexity. We consider a hypothetical algorithm that receives the same data as \FreqGreedy, but chooses actions based on the Bayesian-greedy selection rule. We analyze this hypothetical algorithm using the same technique as \BayesGreedy, and then upper bound the difference in Bayesian regret between the hypothetical algorithm and \FreqGreedy.

Our analyses extend to group externalities and (Bayesian) minority regret. In particular, we circumvent the impossibility result mentioned above. We prove that both \BayesGreedy and \FreqGreedy match the Bayesian minority regret of any algorithm run on either the full population or the minority alone, up to polylogarithmic factors

\xhdr{Detailed comparison with prior work.} We substantially improve over the $\tilde{O}(\sqrt{T})$ worst-case regret bound from \citet{kannan2018smoothed}, at the cost of some additional assumptions. First, we consider Bayesian regret, whereas their regret bound is for each realization of $\theta$.%
\footnote{Equivalently, they allow point priors, whereas our priors must have variance $T^{-O(1)}$.} Second, they allow the context vectors to be chosen by an adversary before the perturbation is applied. Third, they extend their analysis to a somewhat more general model, in which there is a separate latent weight vector for every action (which amounts to a more restrictive model of perturbations). However, this extension relies on the greedy algorithm being initialized with a substantial amount of data. The results of \citet{kannan2018smoothed} do not appear to have implications on group externalities.

\citet{bastani2017exploiting} show that the greedy algorithm achieves logarithmic regret in an alternative linear contextual bandits setting that is incomparable to ours in several important ways.
They consider two-action instances where the actions share a common context vector in each round, but are parameterized by different latent vectors. They ensure data diversity via a strong assumption on the context distribution. This assumption does not follow from our perturbation conditions; among other things, it implies that each action is the best action in a constant fraction of rounds. Further, they assume a version of Tsybakov's \emph{margin condition}, which is known to substantially reduce regret rates in bandit problems \citep[\eg see][]{Zeevi-colt10}.



%%!TEX root = LWM.tex
\subsection{Related Work}

Most directly related to our work is a recent series of papers by 
\cite{faradonbeh17a,faradonbeh17b}, who study the
linear system identification problem by proving a non-asymptotic rate
on the convergence of the OLS estimator to the true system matrices.
%Faradonbeh et al.\ distinguish between two cases: (a) when the $\Ast$ matrix
%is stable (i.e. spectral radius $\rho(\Ast)$ is bounded by one), and
%(b) when it is not.
In the regime where $\Ast$ is stable, Faradonbeh et al.\ recover 
a similar rate as our result. The major
difference is that the dependence of their analysis on the spectral properties of
$\Ast$ are qualitatively suboptimal, and difficult to interpret precisely.
%
Their analysis is based on separately establishing concentration of the
sample covariance matrix $\sum_{t=1}^{T} X_t X_t^\top$ to the stationary
covariance matrix and bounding the martingale difference term $\sum_{t=1}^{T} X_t \noise_t$.
This decoupled analysis inevitably picks up a dependence on the condition
number of the stationary covariance matrix, which means that
as the system becomes more unstable, their bound deteriorates. 
Indeed, such a strategy is unable to provide any insight into the behavior of OLS when,
for example, $\Ast$ is a scaled orthogonal matrix.
%\maxs{What does their bound say about marginally stable? Does it even hold} 
On the other hand, our analysis does not decouple the two terms, and as
a result our bounds only degrade in the \emph{logarithm} of the condition
number of the finite-time controllability Gramian $\Gamma_T$.
\cite{faradonbeh17a} also provide a bound in the \emph{unstable regime}, which
we believe can be sharpened using our analysis techniques which couple the
covariate- and noise-processes. We leave this to future work. Moreover,
our analysis of one-dimensional, unstable systems corroborates the linear
convergence behavior that \cite{faradonbeh17a} obtain for ``explosive'' systems,
which are systems where
\emph{all} eigenvalues of $\Ast$ lie outside the complex unit disk.


Another closely related work is the scalar analysis by \cite{rantzer18}.
In fact, our proof technique for scalar systems can be seen as an extension of
his technique. The main difference is that by more carefully tracking the terms
that appear in the moment generating function of the noise and covariate processes, we are able to discriminate behaviors that
arise when $\Ast$ is stable versus unstable, and uncover a linear rate of convergence
in the unstable regime.

Our result qualitatively matches the behavior of the
rate given in \cite{dean17},
in that the key spectral quantity governing the rate of convergence is the
minimum eigenvalue of the finite-time controllability Gramian.
The major difference is that the analysis
in Dean et al.\ uses multiple independent trajectories, and discards all but the last
state-transition in each trajectory. This decouples the covariates, and reduces
the analysis to that of random design linear regression with independent covariates.
We note, however, that the analysis in Dean et al.\ applies even when $\Ast$ is
unstable.

More broadly, there has been recent interest in non-asymptotic analysis of linear system
identification problems. Some of the earlier non-asymptotic literature in system identification
include \cite{campi2002finite} and \cite{vidyasagar2008learning}.
The results provided in this line of work are often quite conservative,
featuring quantities which are exponential in the degree of the system.
Furthermore, the rates given are often difficult to interpret.
More recently, \cite{shah12} pose the problem of recovering
a single-input, single-output (SISO) LTI system from linear measurements in the frequency domain as a sparse recovery
problem, proving polynomial sample complexity for recovery in the $\calH_2$-norm.
\cite{hardt16} show that under fairly restrictive assumptions on the
$\Ast$ matrix, projected gradient descent recovers the state-space representation 
of an LTI system with only a polynomial number of samples. 
The analysis from both Shah et al. and Hardt et al. both degrade polynomially
in $\frac{1}{1-\rho(\Ast)}$, where $\rho(\Ast)$ is the spectral radius of underlying $\Ast$.
%
On the other hand, \cite{hazan17} propose a new spectral filtering algorithm 
for online prediction of linear systems where the rates do not degenerate as $\rho(\Ast) \to 1$,
with the caveat that the analysis only applies to symmetric $\Ast$ matrices. \cite{hazan18} extends the analysis to diagonalizable matrices, but the obtained error rates are polynomial in problem parameters. Both works also consider the more general setting where $X_t$ is observed indirectly via $Y_t = CX_t$ for an unknown observation matrix $C$.
%
Moreover, the main metric of interest in both \cite{hardt16} and \cite{hazan17,hazan18}
is the prediction error.  It is not clear how
prediction error guarantees can be used in downstream robust control synthesis, 
whereas the operator norm bounds we provide can be used as direct
inputs into robust synthesis for optimal control problems~\citep{dean17}.


The most well-established technique in the statistics literature for dealing
with non-independent, time-series data is the use of mixing-time arguments~\citep{yu94}.
In the machine learning literature, mixing arguments have been used to develop
generalization bounds~\citep{mohri07,mohri08,kuznetsov17,mcdonald17b} which
are analogous to the classical generalization bounds for i.i.d.\ data.
As mentioned previously, a fundamental limitation of mixing-time arguments is that
the bounds all degrade as the mixing-time increases. This has two implications for
linear system identification: (a) none of these existing results can correctly capture
the qualitative behavior as the $\Ast$ matrix reaches instability, and (b)
these techniques cannot be applied to the regime where $\Ast$ is unstable, for which
estimation is not only well-posed, but should be quite easy.
It is for these reasons we do not pursue such arguments in this work.



%\maxs{what about Cyril's paper?}
%\stephen{TODO: maybe make a short blurb about Mendelson's small ball method.}

%\stephen{---------------------------------------------------------}
%
%\paragraph{Estimation papers.}
%
%\stephen{michigan paper~\cite{faradonbeh17a,faradonbeh17b}}
%\\
%\stephen{dean et al~\cite{dean17}}
%\\
%\stephen{cyril zhang paper~\cite{hazan17}}
%\\
%\stephen{tengyu paper~\cite{hardt16}}
%\\
%\stephen{rantzer paper}
%\\
%\stephen{campi papers}
%
%\paragraph{Mixing papers.}
%
%\stephen{yu94~\cite{yu94}, mohri~\cite{mohri07,mohri08}, kuznetsov, mcdonald and shalizi~\cite{mcdonald17b}, my recent LQR+LSTD paper}

%%% Local Variables:
%%% mode: latex
%%% TeX-master: t
%%% End:


\newcommand{\lcomp}{\loss^{\circ}}
\section{Preliminaries}\label{s:prel}
%
We start by considering a nonstochastic multiarmed bandit problem on
$K$ actions with oblivious losses in which the loss $\loss_t(i) \in
[0,1]$ at time $t$ of an action $i \in \{1,\ldots, K\}$ is defined
by the sum
\[
  \loss_t(i) = \sum_{s=0}^{d-1} \loss_t^{(s)}(i)
\]
of $d$-many components $\loss_t^{(s)}(i) \ge 0$ for $s=0,\dots,d-1$.
Let $I_t$ denote the action chosen by the player at the
beginning of round $t$. If $I_t = i$, then the player incurs loss
$\loss_t^{(0)}(i)$ at time $t$, loss $\loss_t^{(1)}(i)$ at time $t+1$,
and so on until time $t+d-1$.
Yet, what the player observes at time $t$ is only the combined loss incurred at time $t$, which is the sum
$
\loss_{t}^{(0)}(I_{t}) + \loss_{t-1}^{(1)}(I_{t-1}) + \cdots + \loss_{t-d+1}^{(d-1)}(I_{t-d+1})
$
of the past $d$ loss contributions, where $\loss_t^{(s)}(i) = 0$ for all $i$ and $s$ when $t \le 0$. Since the setting $d=1$ recovers the standard nonstochastic oblivious bandit model, in the following we assume $d \ge 2$. For all sequences of actions $i_1, \ldots, i_d \in \{1,\ldots,K\}$, define the $d$-delayed {\em composite} loss function
%
\begin{equation}\label{e:mixedloss}
    \lcomp_t(i_1,i_2,\dots,i_d) = \sum_{s=0}^{d-1} \loss_{t-s}^{(s)}(i_{d-s})~,
\end{equation}
%
with $\loss_t^{(s)}(i) = 0$ for all $i$ and $s$ when $t \le 0$. With this notation, the $d$-delayed composite anonymous feedback assumption states that what the player observes at the end of each round $t$ is only the composite loss
$
\lcomp_t(I_{t-d+1},I_{t-d+2},\dots,I_t)
$.
Note that, whereas the losses $\loss_t(i)$ are in $[0,1]$, the composite loss can take values as large as $d$. On the other hand, the cumulative composite loss of any action $i$ over $d$ consecutive steps is at most $2d-1$:
\begin{equation}
\label{eq:compsum}
    \sum_{\tau=t-d+1}^t \lcomp_{\tau}(i,\dots,i)
=\!\!\!
    \sum_{\tau=t-d+1}^t \sum_{s=0}^{d-1} \loss_{\tau-s}^{(s)}(i)
\leq\!\!\!
    \sum_{\tau=t-2d+2}^t \sum_{s=0}^{d-1}\loss_{\tau}^{(s)}(i)
=\!\!\!
    \sum_{\tau=t-2d+2}^t \loss_{\tau}(i)
\le
    2d-1~.
\end{equation}
The goal of the algorithm is to bound its regret $R_T$ against the best fixed action in hindsight,
\[
R_T =\E\left[\sum_{t=1}^T \lcomp_t(I_{t-d+1},\dots,I_t)\right] -
\min_k \sum_{t=1}^T \lcomp_t(k,\dots,k)~.
\]
We define the regret in terms of the composite losses $\lcomp_t$ rather than the true losses $\loss_t$ because in our model $\lcomp_t$ is what the algorithm pays overall in round $t$. It is easy to see that a bound on $R_T$ implies a bound on the more standard notion of regret $\E\left[\sum_{t=1}^T \loss_t(I_t)\right] - \min_{k}\sum_{t=1}^T \loss_t(k)$ up to an additive term of at most $d-1$.
%\begin{align*}
%   \E\left[\sum_{t=1}^T \loss_t(I_t)\right] - \min_{k = 1,\ldots,K}\sum_{t=1}^T \loss_t(k)
%&=
%   \E\left[\sum_{t=1}^T \lcomp_t(I_{t-d+1},\dots,I_t)\right] - \sum_{t=1}^T \lcomp_t(k,\dots,k)
%   + \sum_{t=T-d+2}^T \sum_{s=T-t+1}^{d-1} \Big(\E\big[\loss_t^{(s)}(I_t)\big] - \loss_t^{(s)}(k)\Big)
%\\&\le
%   \E\left[\sum_{t=1}^T \lcomp_t(I_{t-d+1},\dots,I_t)\right] - \sum_{t=1}^T \lcomp_t(k,\dots,k) + (d-1)~.
%\end{align*}

Our setting generalizes the composite loss function setting of \citet{ddkp14}.
Specifically, the linear composite loss function therein can be seen as a
special case of the composite loss~(\ref{e:mixedloss}) once we remove
the superscripts $s$ from the loss function components. In fact, in the linear case,
the feedback in \citep{ddkp14} allows one to easily reconstruct each individual
loss component in a recursive manner. This is clearly impossible in our more
involved scenario, where the new loss components that are observed in round $t$ need
not have occurred in past rounds.


\newcommand{\blhat}{\widehat{\bloss}}
\newcommand{\bmu}{\boldsymbol{\mu}}
\newcommand{\bq}{\boldsymbol{q}}
\section{Wrapper Algorithm for Composite Losses}
\label{s:wrapper}
%
Our ``Composite Loss Wrapper'' algorithm (Algorithm~\ref{a:delayed-app}) wraps a standard bandit algorithm called here Base MAB (Base Multi-Armed Bandit). Base MAB operates on standard (noncomposite) losses with values in $[0,1]$,
producing probability distributions $\bp_t$ over the action set $\{1,\ldots,K\}$. The wrapper, which has access to a sequence
$B_0,B_1,\dots$ of i.i.d.\ Bernoulli random variables of parameter $q$ (to be chosen later), experiences three kinds of online rounds: a {\em draw}, an {\em update}, and a {\em stay}
round. If round $t$ is a draw round, the algorithm draws action $I_t$ according to the current distribution $\bp_t$
maintained by Base MAB, but without having Base MAB update $\bp_t$. If $t$ is an update round, then the algorithm's
action $I_t$ is the same as $I_{t-1}$ (in particular, the algorithm does not draw $I_t$ from $\bp_t$), but then a
distribution update $\bp_t \rightarrow \bp_{t+1}$ takes place by invoking the update rule of Base MAB over an
average of the observed losses.
Finally, if $t$ is a stay round, then both $I_t = I_{t-1}$ and $\bp_{t+1} = \bp_{t}$. The way these three
kinds of rounds are interleaved is illustrated in Figure~\ref{f:1}.
%
%% ----------------------------------------------------------------------------
\begin{algorithm2e}[t]
\SetKwSty{textrm} \SetKwFor{For}{{\bf For}}{}{}
\SetKwIF{If}{ElseIf}{Else}{if}{}{else if}{else}{}
\SetKwFor{While}{while}{}{}
\textbf{Input:}  Base MAB algorithm $A$ with parameter $\eta \in (0,1]$.\\%, and meta-parameters for it $\xi$.\\
\textbf{Initialize:}
\begin{itemize}[topsep=0pt,parsep=0pt,itemsep=0pt]
\item Draw $I_0$ from the uniform distribution $\bp_1$ over $\{1,\ldots,K\}$;
\item If $B_0 = 1$ then $t=0$ is an update round.
\end{itemize}
%
\For{$t=1,2,\dots$:} { {
\begin{enumerate}[topsep=0pt,parsep=0pt,itemsep=0pt]
\item If $t-1$ was an update round, then draw $I_t \sim \bp_t$ and play it without updating $\bp_t$ (draw round, $\bp_{t+1} = \bp_t$);
\item Else if an update round was in the interval $\{t-2d+1, \dots,t-2\}$ then play $I_t = I_{t-1}$ without updating $\bp_t$
(stay round, $\bp_{t+1} = \bp_t$);
\item Else play $I_t = I_{t-1}$ (stay round), and if $B_t=1$ then the stay round becomes an update round. In such a case:
%
\begin{minipage}{\textwidth-50pt}
\begin{itemize}
\vspace{0.05in}
\item Feed Base MAB $A(\eta)$ with average composite loss\footnote
{
Recall that when $t \leq 0$, we defined $\ell^{(s)}_{t} =0$, so the initial stretch of $2d-2$ actions $I_1,\ldots,I_{2d-2}$ can be disregarded here at the price of an extra additive $\scO(d)$ regret in the analysis.
}
%
\[
\avgloss_t = \frac{1}{2d}\sum_{\tau=t-d+1}^t \lcomp_\tau(I_{\tau-d+1},\dots,I_{\tau})
\]
\item Use the update rule $\bp_t\rightarrow \bp_{t+1}$ of Base MAB to obtain the new distribution $\bp_{t+1}$.
%(the stay round becomes an update round).
\end{itemize}
\end{minipage}
\end{enumerate}
%\vspace{-0.2in}
} } \caption{The Composite Loss Wrapper.}
\label{a:delayed-app}
\end{algorithm2e}
%% ---------------------------------------------------------------------------
%
\begin{figure}[t!]
\begin{picture}(-40,290)(-40,290)
\scalebox{0.7}{\includegraphics{interleave.pdf}}
\end{picture}
\vspace{-3.0in}
\caption{\label{f:1}
Sequence of rounds the algorithm is undergoing. Each ({\em u})pdate round is always followed by a d({\em r})aw round,
and then by a stretch of ({\em s})tay rounds whose length is random, but is at least $2d-2$.
The actual length of the stay stretch is ruled by the realizations of the Bernoulli random variables $B_t$.
}
\end{figure}


%%% ----------------------------------------------------------------------------
%\begin{algorithm2e}[t]
%\SetKwSty{textrm} \SetKwFor{For}{{\bf For}}{}{}
%\SetKwIF{If}{ElseIf}{Else}{if}{}{else if}{else}{}
%\SetKwFor{While}{while}{}{}
%\textbf{Parameter:} $\eta \in [0,1]$.\\
%\textbf{Initialize:}
%\begin{itemize}
%\item Draw $I_0$ from the uniform distribution $p_1$;
%\item If $B_0 = 1$ then $t=0$ is an update round.
%\end{itemize}
%%
%\For{$t=1,2,\dots$:}
%{ {
%\begin{enumerate}
%\item If $t-1$ was an update round, then draw $I_t \sim p_t$ and play it without updating $p_t$ (draw round);
%\item Else if $d > 2$ and an update round was in the interval $\{t-d+1, \dots,t-2\}$ then play $I_t = I_{t-1}$ without updating $p_t$
%(stay round);
%\item Else play $I_t = I_{t-1}$ (stay round), and if $B_t=1$ then perform an Exp3$(\eta)$ update of $p_t$
%using $\loss_t(I_t,\dots,I_{t-d+1})$ (the stay round becomes an update
%round).
%\end{enumerate}
%
%\vspace{-0.2in}
%} }
%\caption{Delayed Loss Algorithm}
%\label{a:delayed}
%\end{algorithm2e}
%%% ----------------------------------------------------------------------------
%%
%
Note that the algorithm's pseudocode corresponds to the description in Figure~\ref{f:1} in that update and draw rounds are interleaved,
and an update round is immediately followed by a draw round. If an update round occurs at time $t \ge 1$, then no update round
can occur during the next $2d-1$ rounds; the next update takes place at time $t+2d+G$ where $G \ge 0$ is a Geometric random variable
with parameter $q$. Hence a stretch of stay rounds is $2d-2+G$ round long.
Moreover,
\begin{itemize}[topsep=0pt,parsep=0pt,itemsep=0pt]
\item
If $t$ is not a draw round (i.e., it is either an update or a stay round), then the last action is played again.
\item
If $t$ is an update round, then we are guaranteed that $I_t=\cdots=I_{t-2d+1}$, since the last draw could have only occurred
at time $t-2d+1$ or earlier.
\end{itemize}
%
In order for our analysis to go through, we make mild assumptions on Base MAB. The first assumption (fulfilled by many
standard $K$-armed bandit algorithms ---see below) is a stability condition described in the following definition.
%
\begin{definition}\label{d:stabilityexp3}
Let $A(\eta)$ be a Base MAB with learning rate $\eta$,
and $\{\bp_t\}_{t=1}^T$ be the sequence of probability distributions over actions $\{1,\ldots,K\}$ produced by $A(\eta)$ during a run over $T$ rounds. We say that $A(\eta)$ is $\xi$-{\em stable} if for any round $t$ we have that
\[
    \E\left[\sum_{i\,:\,p_{t+1}(i) > p_t(i)} p_{t+1}(i) - p_t(i)\right] \leq \xi
\]
% deterministically
holds, where $\xi = \xi(K,\eta,\ldots)$ is a function of $K$, $\eta$, and possibly other relevant parameters of the Base MAB.
\end{definition}
%
The second assumption is that $A(\eta)$ is {\em nontrivial} for any $\eta > 0$:
when operating in the standard (non-delayed) bandit setting, $A(\eta)$ enjoys a concave (possibly linear) regret bound as a function of the time
horizon $T$.  Specifically, if we let $R_A(T,K,\eta)$ be a regret bound for $A$ when the time horizon is $T$, the number
of actions is $K$ and the learning rate is $\eta$, we have for any $K \geq 1$ and $\eta > 0$ that
%$R_A(T,K,\eta) = o(T)$ as $T \rightarrow \infty$, being
$R_A(T,K,\eta)$ is a concave function of $T$. For example,
$R_A(T,K,\eta) = \scO((\ln K)/\eta + \eta K T)$, which is linear in
$T$.
%
\iffalse
*********************************
Our algorithm receives a general Multi-Arm Bandit algorithm $A$,
which has a learning rate parameter $\eta$ and a set of hyper
parameters $\xi$. We will assume that MAB is {\em stable}, namely,
it does not change the action distribution by more than $2\eta$
between successive rounds. Formally, the definition of stable is the
following.
%
\begin{definition}
A MAB algorithm $A$ with parameter $\eta$ is {\em stable} if for any
history $h$ and loss $\ell$: it outputs the distribution $p$ after
$h$ and the distribution $p'$ after $h$ followed by $\ell$, then
$\|p-p'\|_1\leq 2\eta$
\end{definition}
%
The regret bound of a MAB algorithm $A$ with parameters $\eta$ and
$\xi$ is denoted by $R_A(T,K,\eta,\xi)$. Namely, for any sequence of
losses $\ell_t$ and any fixed action $j$ we have,
\[
\E[\sum_{t=1}^T \ell_t(a_t)]\leq \E[\sum_{t=1}^T \ell_t(j)] + R_A(T,K,\eta,\xi)
\]
where the expectation is over the probabilities that $A$ selects
action $a_t$ at time $t$.

We also assume that for any random variable ${\cal T}$ we have
\[
\E[R_A({\cal T},K,\eta,\xi)]\leq R_A(\E{\cal T},K,\eta,\xi)
\]
*********************************
\fi
We have the following theorem, whose proof is in the appendix.
%
\begin{theorem}\label{thm:delay-generic}
Let $A(\eta)$ be a $\xi$-stable and nontrivial Base MAB algorithm with learning rate $\eta$ and regret bound
$R_A(T,K,\eta)$ for standard $K$-armed bandits. Then
Algorithm~\ref{a:delayed-app} with input $A(\eta)$ achieves regret
\[
R_T \le T\,\xi + 8(2d-1)R_A(T/2d,K,\eta) + \scO(d)
\]
for $K$-armed bandits with $d$-delayed composite anonymous feedback.
%In particular, for $\eta = \sqrt{\frac{d\ln K}{TK}}$ we have a
%cumulative regret of $O(\sqrt{dTK\ln K} )$
\end{theorem}
%
%
%%%%%%%%%%%%%%%%%%%%%%%%%%%%%%%
%
We can now derive corollaries for various algorithms using Theorem~\ref{thm:delay-generic}. Consider for instance, the well-known
Exp3 algorithm of \citet{AuerCeFrSc02}. When operating with losses, the algorithm maintains a probability distribution
$\bp_t = (p_t(1),\ldots,p_t(K))$ over $\{1,\ldots,K\}$ of the form $p_t(i) = w_t(i)/\sum_{j=1}^K w_t(j)$, while the update rule
$\bp_t\rightarrow \bp_{t+1} $ can be described as follows:
\begin{equation}
\label{eq:expW}
w_{t+1}(i) = p_t(i)\,e^{-\eta\ellhat_t(i)},\quad \ellhat_t(i) = \frac{\ell_t(i)\Ind{I_t = i}}{p_t(i)},\quad i = 1, \ldots, K~.
\end{equation}
When the losses $\ell_t(i)$ are in $[0,1]$ we have the regret bound
\(
R_{\mathrm{Exp3}}(T,K,\eta) \leq \frac{\ln K}{\eta}+\frac{\eta}{2}\,K\,T~.
\)
Moreover, the following simple stability property holds (proof in the appendix).
%
\begin{lemma}\label{l:stabilityexp3}
Exp3 with learning rate $\eta$ is $\xi$-stable with $\xi = \eta$.
\end{lemma}
%
Combined with Theorem~\ref{thm:delay-generic}, this implies the following regret bound with composite losses.
%
\sloppypar{
\begin{corollary}\label{c:delay-generic}
%With the notation introduced so far,
If Algorithm~\ref{a:delayed-app} is run with Exp3($\eta$) with $\eta = 4\sqrt{\frac{d\,\ln K}{(4K+1)\,T}}$ as Base MAB, then its regret for $K$-armed bandits with $d$-delayed composite anonymous feedback satisfies
\[
R_T \leq 8\sqrt{d\,(4K+1)\,T\,\ln K} + \scO(d) = \scO(\sqrt{dKT\ln K})~.
\]
\end{corollary}
}
%
$K$-armed bandits are a special case of combinatorial linear bandits
\citep{cesa2012combinatorial}, a setting where actions are incidence
vectors $\bv\in\scK\subset\bool^n$ describing elements in some
combinatorial space (e.g., spanning trees of a given graph) and loss
vectors $\bloss_t \in [0,1]^n$ satisfy $\bloss_t^{\top}\bv \in
[0,1]$ for all $\bv\in\scK$ (in $K$-armed bandits, $\scK$ is simply
the canonical basis of $\bool^n$). Let $\bv_t\in\scK$ be the action
played at time $t$. The generalization of Exp3 for the combinatorial
bandit setting uses the Exp2 algorithm with loss estimates of the
form $\blhat_t = P_t^+ \bv_t \bv_t^{\top} \bloss_t$, where $P_t =
\E_{\bV \sim \bp_t}\bigl[\bV \bV^{\top}\bigr]$ and $P_t^+$ is the
pseudo inverse of $P_t$ ---see \citep{dani2008price}. Note that
these estimates are unbiased: $\E_t\big[\blhat_t^{\top}\bv] =
\bloss_t^{\top}\bv$ for all $\bv\in\scK$. The distribution $\bp_t$
is a mixture $\bp_t = (1-\gamma)\bq_t + \bmu$, where $0 < \gamma <
1$, $\bq_t$ has the exponential form~(\ref{eq:expW})
\[
    q_t(\bv) = \frac{w_t(\bv)}{\sum_{\bv'\in\scK} w_t(\bv')}~,
\quad
    w_{t+1}(\bv)
=
    q_t(\bv)\,e^{-\eta\,\blhat_t^{\top}\bv}~,
\quad
    \bv\in\scK~,
\]
and $\bmu$ is a fixed exploration distribution on $\scK$. When run with an appropriate exploration distribution $\bmu$ and $\gamma = \eta n < 1$, Exp2$(\eta)$ has the following regret bound ---see, e.g., \citep[Theorem~4]{bubeck2012towards},
$
    R_{\mathrm{Exp2}}(T,\scK,\eta) \le \big(\ln|\scK|\big)\big/{\eta} + 3\eta nT
$.
Now, similarly to Lemma~\ref{l:stabilityexp3}, we can prove the following (the proof is provided in the appendix):
%
\begin{lemma}
\label{l:stabExp2}
Exp2 with learning rate $\eta$ and mixing coefficient $\gamma$ is $\xi$-stable with $\xi = (1-\gamma)\eta$.
\end{lemma}
%
Combining again with Theorem~\ref{thm:delay-generic}, the above implies the following regret bound with composite losses.
%
\begin{corollary}
\label{c:delayExp2}
If Algorithm~\ref{a:delayed-app} is run with Exp2($\eta$) with $\eta = 4\sqrt{\frac{d\,\ln|\scK|}{(24n+1)\,T}}$ as Base MAB, then its regret for $\scK$-combinatorial bandits, $\scK \subseteq \{0,1\}^n$, with $d$-delayed composite anonymous feedback satisfies
\[
    R_T
\le
    8\sqrt{d\,(24n+1)\,T\,\ln|\scK|} + \scO(d) = \scO(\sqrt{dnT\ln|\scK|})~.
\]
\end{corollary}
%
\begin{remark}\label{r:firstorderbound}
The proof of Lemma~\ref{l:stabilityexp3} in the appendix shows pointwise stability, a stronger notion than the expected stability of Definition~\ref{d:stabilityexp3}. In fact, an outer expectation over the random variable $I_t$ in the proof of Lemma~\ref{l:stabilityexp3} makes the stability parameter $\xi$ be upper bounded by $\eta\sum_{i=1}^K p_{t}(i)\ell_{t}(i)$ in round $t$, so that the term $T\xi$ in Theorem~\ref{thm:delay-generic} can be replaced by $\eta\,L_A(T)$, where $L_A(T)$ is the cumulative (average) loss of the Base MAB. Coupled with a ``first order" regret analysis of Exp3 where $T$ is indeed replaced by $L_A(T)$ ---see~\citep[Theorem~2]{allenberg2006hannan}, this gives a regret bound in the composite anonymous feedback setting where $T$ is likewise replaced by $L_A(T)$.
\end{remark}


\newcommand{\Rlin}{R^{\mathrm{lin}}}
\section{Lower bound}
\label{s:lower}
In this section we derive a lower bound for bandits with composite anonymous feedback. We do that through a reduction from the setting of linear bandits (in the probability simplex) to our setting. This reduction allows us to upper bound the regret of a linear bandit algorithm in terms of (a suitably scaled version of) the regret of an algorithm in our setting. Since the reduction applies to any instance of a linear bandit problem, we can use a known lower bound for the linear bandit setting to derive a corresponding lower bound for our composite setting.

Let $\Delta_K$ be the probability simplex in $\R^K$. At each round $t$, an algorithm $A$ for linear bandit optimization chooses an action $\bp_t\in\Delta_K$ and suffers loss $\bloss_t^{\top}\bp_t$, where $\bloss_t \in [0,1]^K$ is some unknown loss vector. The feedback observed by the algorithm at the end of round $t$ is the scalar $\bloss_t^{\top}\bp_t$. The regret suffered by algorithm $A$ playing actions $\bp_1,\dots,\bp_T$ is
\begin{equation}
\label{eq:lin-regret}
	\Rlin_T = \sum_{t=1}^T \bloss_t^{\top}\bp_t - \min_{\bp\in\Delta_K} \sum_{t=1}^T \bloss_t^{\top}\bp = \sum_{t=1}^T \bloss_t^{\top}\bp_t - \min_{i=1,\dots,K} \sum_{t=1}^T \loss_t(i)
\end{equation}
where we used the fact that a linear function on the simplex is minimized at one of the corners.
Let $\Rlin_T(A,\Delta_K)$ denote the worst case regret (over the oblivious choice of $\bloss_1,\dots,\bloss_T$) of algorithm $A$. Similarly, let $R_T(A_d,K,d)$ be the worst case regret (over the oblivious choice of loss components $\loss_t^{(s)}(i)$ for all $t$, $s$, and $i$) of algorithm $A_d$ for nonstochastic $K$-armed bandits with $d$-delayed composite anonymous feedback. Our reduction shows the following.
%
\begin{lemma}\label{lem:lower}
For any algorithm $A_d$ for $K$-armed bandits with $d$-delayed composite anonymous feedback, there exists an algorithm $A$ for linear bandits in $\Delta_K$ such that
$
	R_T(A_d,K,d) \ge d\,\Rlin_{T/d}(A,\Delta_K)
$.
\end{lemma}
%
%
Our reduction, described in detail in the proof of the above lemma (see the appendix), essentially builds the probability vectors $\bp_t$ played by $A$ based on the empirical distribution of actions played by $A_d$ during blocks of size $d$. Now, an additional lemma is needed (whose proof is given in the appendix).
\begin{lemma}
\label{l:shamir}
The regret of any algorithm $A$ for linear bandits in the simplex satisfies $\Rlin_T(A,\Delta_K) = \widetilde{\Omega}\big(\sqrt{KT}\big)$.
\end{lemma}
%
Using the above two lemmas we can prove the following theorem.
\begin{theorem}
For any algorithm $A_d$ for $K$-armed bandits with $d$-delayed composite anonymous feedback,
$
R_T(A_d,K,d)=\widetilde{\Omega}\big(\sqrt{dKT}\big)
$.
\end{theorem}
%
\begin{proof}
Fix an algorithm $A_d$. Using the reduction of~Lemma~\ref{lem:lower} gives an algorithm $A$ such that
$
	R_T(A_d,K,d) \ge d\,\Rlin_{T/d}(A,\Delta_K) = \widetilde{\Omega}\big(\sqrt{dKT}\big)
$,
where we used Lemma~\ref{l:shamir} with horizon $T/d$ to prove the $\widetilde{\Omega}$-equality.
\end{proof}
%
Although the loss sequence used to prove the lower bound for linear bandits in the simplex is stochastic i.i.d., the loss sequence achieving the lower bound in our delayed setting is not independent due to the deterministic loss transformation in the proof of Lemma~\ref{lem:lower} (which is defined independent of the algorithm, thus preserving the oblivious nature of the adversary).



%\section{Discounted loss}

Assume that the losses are discounted by some parameter $\gamma
\in(0,1)$ rather than bounded by $d$. Namely,
\[
    \loss_t(I_{1},\dots,I_t) = \sum_{s=0}^{t-1} \loss_{t-s}^{(s)}(I_{t-s})\gamma^{t-s} (1-\gamma)
\]
Note that if for any $\tau$, $s<\tau$ and action $i$ we have
$\loss_{\tau}(i)^{(s)}\in [0,1]$ then we have that
$\loss_t(I_{1},\dots,I_t)\in[0,1]$.

We can run the same algorithm and assume that we have
$d=2\log_\gamma T$, i.e., $\gamma^d=T^{-2}$. We have that
\[
\sum_{s=0}^{t-1} \loss_{t-s}^{(s)}(I_{t-s})\gamma^{t-s} (1-\gamma) -
\sum_{s=0}^{d-1} \loss_{t-s}^{(s)}(I_{t-s})\gamma^{t-s} (1-\gamma)
\leq \frac{1}{T^2}
\]
the two loss sequences should behave very similar. [[YM: I recall we
had problems with the discounted setting, but I cannot reconstruct
them now. Is the issue only the $\log T$ factor in the regret?]]


%% !TeX root = main.tex
\section{Extensions}
\label{sec:extensions}

% !TeX root = main.tex
\subsection{Partial validity}
\label{sec:partial-validity}
In this section, we consider a generalization of our main setting, where we allow some slack in the validity constraint.
More precisely, given some parameter $\alpha > 0$, we now have the requirement that $\Loss(\hat q) \leq \Loss(q^*) + \eps_1$ and $\Inv(\hat q) \leq \alpha + \eps_2$, where $q^*$ is the optimal distribution which minimizes $\Loss(q^*)$ such that $\Inv(q^*) \leq \alpha$.

%In this result, we allow points to be ``partially valid'' -- specifically, we let $\Inv: X \rightarrow [0,1]$ take fractional values.

\subsubsection{Algorithm}
We provide an algorithm for solving the partial validity problem in Algorithm~\ref{alg:partial-validity}.
This method is sample-efficient, requiring a number of samples which is $\poly\left(M, \eps_1^{-1}, \eps_2^{-1}, \log |Q|\right)$.

\begin{algorithm}[ht]
   \caption{Learning a distribution with partial validity}
   \label{alg:partial-validity}
\begin{algorithmic}[1]
   \STATE {\bfseries Input:} Sample and invalidity access to a distribution $p$, parameters $\ve_1, \ve_2, \alpha > 0$, a family of distributions $Q$.
   \STATE Using $n_1$  samples from $p$, empirically estimate $\overline \Loss(q) \in \Loss(q) \pm \frac {\eps_1} 3$ for all $q \in Q.$
   \FOR{$\ell \in \left\{0, \frac {\eps_1} 3,..., M\right\}$} \label{ln:partial-validity-outer-loop}
   \STATE Let $D = \{q \in Q\ |\ \overline \Loss(q) \le \ell \}$.
   \STATE Let $x^*$ be any point with $\Inv(x^*) = 0$.
   \STATE Let $\mu_D$ be the distribution which samples a distribution $q$ uniformly from $D$, and then draws a sample from $q$.
   \WHILE{$D \neq \emptyset$} \label{ln:partial-validity-inner-loop}
   \STATE Draw $n_2$ samples $x_1, ..., x_{n_2}$ from $\mu_D$.
   \IF {$\frac{1}{n_2} \sum_{i=1}^{n_2} \Inv(x_i) \Pr_{q \sim \Unif(D)}[q(x_i) {\eps_1} < 3 \mu_D(x_i)  {M} ] \le \alpha + \frac {4 \eps_2}{5}$}
   \RETURN $\mu'_D$, which samples $x$ from $\mu_D$ with probability $$\Pr_{q \sim \Unif(D)}[q(x) {\eps_1} < 3 \mu_D(x) {M} ],$$ and samples $x^*$ otherwise.
   \ELSE \STATE Remove all distributions $q$ from $D$ for which $$\frac{1}{n_2} \sum_{i=1}^{n_2} \Inv(x_i) \frac {q(x_i)} {\mu_D(x_i)} \Ind[q(x_i) {\eps_1} < 3 \mu_D(x_i) {M} ] > \alpha + \frac {\eps_2} {5}.$$  
   \ENDIF
   \ENDWHILE
   \ENDFOR
\end{algorithmic}
\end{algorithm}

\subsubsection{Analysis}
We will show that, with high probability, Algorithm~\ref{alg:partial-validity} outputs a distribution $\hat q$ that has $\Loss(\hat q) \leq \Loss(q^*) + \eps_1$ and $\Inv(\hat q) \leq \alpha + \eps_2$.

\begin{theorem}\label{thm:partial-validity}
  Suppose that the loss function $L$ is convex.
  The choice of parameters
  \begin{equation}
  n_1 = \Theta\left(\frac{M^2}{\ve_1^2} \log |Q| \right), n_2 = \Theta\left(\frac {M^2} {\eps_1^2 \eps_2^2} \log |Q| \log \left(\frac{M \log |Q|}{\eps_1 \eps_2}\right)\right)
  \end{equation}
  guarantees that Algorithm~\ref{alg:partial-validity} outputs w.p. $3/4$ a distribution with $\Loss(\hat q) \le \Loss(q^*) + \eps_1$ and $\Inv(\hat q) \le \alpha + \eps_2$ using $\Theta\left(\frac{M^2}{\ve_1^2} \log |Q| \right)$ samples from $p$ and $\Theta\left(\frac{M^3}{ \ve_1^3 \ve_2^3} \log^2 |Q| \log \left(\frac{M \log |Q|}{\eps_1\eps_2}\right)\right)$ invalidity queries.
\end{theorem}

\begin{remark}
We note that this algorithm still works in the case where points may be ``partially valid'' -- specifically, we let $\Inv: X \rightarrow [0,1]$ take fractional values.
This requires that we have access to some point $x^*$ where $\Inv(x^*) = 0$, which we assume is given to us by some oracle.
For instance, the distribution may choose to output a dummy symbol $\bot$, rather than output something which may not be valid. 
\end{remark}

We prove Theorem~\ref{thm:partial-validity} through three lemmas.
The sample complexity bound follows from the values of $n_1$, $n_2$, the fact that we have at most $O\left(\frac{M}{\eps_1}\right)$ iterations of the loop at Line~\ref{ln:partial-validity-outer-loop}, and Lemma~\ref{lem:partial-validity-inner-loop} which bounds the number of iterations of the loop at Line~\ref{ln:partial-validity-inner-loop} as $O\left(\frac{\log |Q|}{\eps_2}\right)$ for any $\ell$.
To argue correctness, Lemmas~\ref{lem:partial-validity-invalid} and~\ref{lem:partial-validity-loss} bound the invalidity and loss of any output distribution, respectively.
The proofs of these lemmata appear in Section~\ref{sec:partial-proofs}.

\begin{lemma}
\label{lem:partial-validity-inner-loop}
With probability at least $14/15$, the loop at Line~\ref{ln:partial-validity-inner-loop} requires at most $O\left(\frac{\log |Q|}{\eps_2}\right)$ iterations for each $\ell$.
\end{lemma}

\begin{lemma}
\label{lem:partial-validity-invalid}
With probability at least $14/15$, if at any step a distribution $\mu'_D$ is output, $\Inv(\mu'_D) \le \alpha + \eps_2$.
\end{lemma}

\begin{lemma}
\label{lem:partial-validity-loss}
With probability at least $14/15$, if at any step a distribution $\mu'_D$ is output, $\Loss(\mu'_D) \le \ell + 2\eps_1/3$, where $\ell$ is the step at which the distribution was output.
\end{lemma}

The proof of Theorem~\ref{thm:partial-validity} concludes by observing that the optimal distribution $q^*$ is never eliminated (assuming all estimates involving its loss and validity are accurate, which happens with probability at least $19/20$), and that the loop in line~\ref{ln:partial-validity-outer-loop} steps by increments of $\eps_1/3$. 
Combining this with Lemma~\ref{lem:partial-validity-loss}, if we output $\hat q$, then $\Loss(\hat q) \leq \Loss(q^*) + \eps_1$.

\newcommand{\fat}{s}
\newcommand{\vc}{d}

\subsection{General Densities}
\label{sec:densities}

For simplicity of presentation, we have formulated the above results in terms of probability mass functions $q$ on a discrete domain $X$.
However, we note that all of the above results easily extend to general density functions on an abstract measurable space $X$, which 
may be either discrete or uncountable.  Specifically, if we let $\mu_{0}$ denote an arbitrary reference measure on $X$, 
then we may consider the family $Q$ to be a set of \emph{probability density functions} $q$ with respect to $\mu_{0}$: that is, 
non-negative measurable functions such that $\int q {\rm d}\mu_{0} = 1$.  
For the results above, we require that we have a way to (efficiently) generate iid samples having the distribution whose density is $q$.
For the full-validity results, the only additional requirements are that we are able to (efficiently) test whether a given $x$ is in the support of $q$,
and that we have access to $\text{Oracle}(\cdot,\cdot)$ defined with respect to the set $Q$.
For the results on partial-validity, we require the ability to explicitly evaluate the function $q$ at any $x \in X$.
The results then hold as stated, and the proofs remain unchanged (overloading notation to let $q_{x}$ denote the 
value of the density $q$ at $x$, and $q(A) = \int_{A} q {\rm d}\mu_{0}$ the measure of $A$ 
under the probability measure whose density is $q$).

\subsection{Infinite Families of Distributions}
\label{sec:infinite-families}

%%% maybe no space to define VC-dim and fat-shattering dim, but we can at least give references for both.

It is also possible to extend all of the above results to \emph{infinite} families $Q$, 
expressing the sample complexity requirements in terms of the \emph{VC dimension} (\cite{VapnikC74}) of the supports $\vc = {\rm VCdim}(\{\Supp(q) : q \in Q\})$, 
and the \emph{fat-shattering dimension} (\cite{AlonBCH97}) of the family of loss-composed densities $\fat(\epsilon) = {\rm fat}_{\epsilon}(\{ x \mapsto L(q_{x}) : q \in Q \})$.
In this case, in the context of the full-validity results, 
for simplicity we assume that in the evaluations of $\text{Oracle}(X_{P},X_{N})$ defined above, 
there always \emph{exists} at least one minimizer $q \in Q$ of the empirical loss with respect to $X_{P}$ 
such that $\Supp(q) \cap X_{N} = \emptyset$.\footnote{It is straightforward to remove this assumption by supposing 
$\text{Oracle}(X_{P},X_{N})$ returns a $q$ that \emph{very-nearly} minimizes the empirical loss, and handling 
this case requires only superficial modifications to the arguments.}
We then have the following result.  For completeness, we include a full proof in the appendix.

\begin{theorem}
\label{thm:vc-full-validity}
For a numerical constant $c \in (0,1]$, 
the choice of parameters 
\begin{eqnarray*}
P=\Theta\left(\frac{\fat(c \ve_{1}/M) M^{2}}{\ve_1^2} \log \frac{M}{\ve_{1}} \right), & R = \Theta \left( \frac { M } {\eps_1} \right), &  T = \Theta\left(\frac{R \vc}{\ve_2} \log \frac{1}{\ve_{2}} \right)
\end{eqnarray*}
guarantees that Algorithm~\ref{alg:full-validity} outputs w.p. $3/4$ a distribution $\hat q$ with $\Loss(\hat q) \le \Loss(q^*) + \eps_1$ and $\Inv(\hat q) \le \eps_2$ 
using $P$ samples from $p$ and $R T$ invalidity queries.
  
The algorithm runs in time polynomial in $M$, $\ve_1^{-1}$, $\ve_2^{-1}$, $\vc$, and $\fat_{\ve_{1}/256}$ assuming that queries to the optimization oracle 
can be computed in polynomial time. Moreover, sampling from the resulting distribution $\hat q$ can also be performed in polynomial time.
\end{theorem}

For partial-validity, we can also extend to infinite $Q$, 
though in this case via a more-cumbersome technique.
Specifically, let us suppose the densities $q \in Q$ are bounded by $1$ (this can be replaced by any value by varying the sample size $n_2$).
Then we consider running Algorithm~\ref{alg:partial-validity} as usual, 
except replacing Step 4 with the step 
\begin{equation*}
D = {\rm Cover}_{\ve_{2}}( \{ q \in Q | \overline\Loss(q) \leq \ell \} ),
\end{equation*} 
where for any $R \subseteq Q$, ${\rm Cover}_{\ve_{2}}(R)$ denotes a minimal subset of $R$ such that 
$\forall q \in R$, $\exists q^{\ve_{2}} \in {\rm Cover}_{\ve_{2}}(R)$ with $\int | q_{x} - q^{\ve_{2}}_{x} | \mu_{0}({\rm d}x) \leq \ve_{2}$:
that is, an $\ve_{2}$-cover of $R$ under $L_{1}(\mu_{0})$.
Let us refer to this modified algorithm as Algorithm~\ref{alg:partial-validity}$^{\prime}$.
%Now denote by $\fat_{Q}(\epsilon) = {\rm fat}_{\epsilon}(Q)$.
We have the following result. 

\begin{theorem}
\label{thm:vc-partial-validity}
Suppose that the loss function $L$ is convex.
For a numerical constant $c \in (0,1]$, 
the choice of parameters
\begin{eqnarray*}
n_1 = \Theta\left(\frac{\fat(c\ve_{1}/M) M^{2}}{\ve_1^2} \log \!\left(\frac{M}{\ve_{1}}\right)  \right), & n_2 = \Theta\left(\frac {M^2 {\rm fat}_{c\ve_{2}}(Q)} {\ve_1^2 \ve_2^2} \log^{2} \!\left(\frac{M {\rm fat}_{c\ve_{2}}(Q)}{\ve_1 \ve_2}\right)\right)
\end{eqnarray*}
guarantees that Algorithm~\ref{alg:partial-validity}$^{\prime}$ (with parameters $\eps_{1}$, $\eps_{2}$, and $\alpha+\eps_{2}$) 
outputs w.p. $3/4$ a distribution with 
$\Loss(\hat q) \le \Loss(q^*) + \ve_1$ and $\Inv(\hat q) \le \alpha + 2\ve_2$ using $n_{1}$ samples from $p$ and 
$\Theta\left(\frac {M^3 {\rm fat}_{c\ve_{2}}(Q)^{2}} {\ve_1^3 \ve_2^3} \log^{3} \!\left(\frac{M {\rm fat}_{c\ve_{2}}(Q)}{\ve_1 \ve_2}\right)\right)$ 
invalidity queries.
\end{theorem}

% we'll assume, for simplicity, that there always is an empirical loss minimizer.  it just simplifies the arguments, but is pretty easy to extend to the general case by putting more \epsilon's all over the place.



\newcommand{\fcomp}{f^{\circ}}
\newcommand{\scD}{\mathcal{D}}
\newcommand{\bb}{\boldsymbol{b}}
\newcommand{\B}{\field{B}}

\section{Extensions: Bandit Convex Optimization}\label{s:bco}
%
We now show that a similar reduction as the one in Section~\ref{s:wrapper} can be made to work in the more general Bandit Convex Optimization (BCO) framework. This learning setting is defined by a convex and compact domain $\Omega \subseteq \R^n$ and a sequence of loss functions $f_1, f_2, \ldots, f_T$, where each $f_t\,:\,\Omega \to [0,1]$ is convex over $\Omega$. We assume each function $f_t$ is the cumulated effect of $d$-many convex loss components $f_t^{(0)},\ldots, f_t^{(d-1)}$, with $f_t^{(s)}\,:\,\Omega \to [0,1]$ so that, for any $\bw \in \Omega$,
%
\[
f_t(\bw) = \sum_{s=0}^{d-1} f_t^{(s)}(\bw) \in [0,1]~.
\]
To be concrete, we shall view $f_t$'s components $f_t^{(s)}$ as constant fractions of $f_t$, specifically,
\[
f_t^{(s)}(\bw) = \alpha^{(s)}_{t} f_t(\bw)~, \qquad s = 0,\ldots, d-1~, \qquad t = 1,\ldots, T\,,
\]
for nonnegative constant coefficients $\alpha^{(s)}_{t}$ such that $\sum_{s=0}^{d-1}\alpha^{(s)}_{t} =1$, for $t = 1, \ldots, T$.

Since we are working with oblivious adversaries, we assume that all losses $\{f_{t}\}_{t=1\dots T}$ and coefficients $\{\alpha^{(s)}_{t}\}_{t=1\dots T,s=0\dots d-1}$ are generated before the game starts. At each round $t= 1, 2, \ldots, T$, the learner picks $\tbw_t \in \Omega$ and suffers loss $f^{(0)}_t(\tbw_t) = \alpha^{(0)}_{t}f_t(\tbw_t)$ at time $t$, loss $f^{(1)}_{t}(\tbw_t) =  \alpha^{(1)}_{t} f_{t}(\tbw_t)$ at time $t+1,\ldots,$ loss $f^{(d-1)}_t(\tbw_t) =  \alpha^{(d-1)}_{t}f_t(\tbw_t)$ at time $t+d-1$. However, what the algorithm really observes at time $t$ is the cumulated effect of present and past actions quantified by the composite loss
\(
\fcomp_t(\tbw_{t-d+1},\tbw_{t-d+2},\ldots, \tbw_t)
\)
with
\[
\fcomp_t(\bw_{1},\bw_{2},\ldots, \bw_d)= \sum_{s=0}^{d-1} f_{t-s}^{(s)}(\bw_{d-s}) = \sum_{s=0}^{d-1} \alpha_{t-s}^{(s)}\,f_{t-s}(\bw_{d-s})~,
\]
where in the above $\alpha_t^{(s)} = 0$ for all $s$ if $t \leq 0$.
The aim of the algorithm is to minimize its regret
\[
% R_T = \E\left[\sum_{t=1}^T f_t(\tbw_{t})\right] - \min_{\bw \in \Omega} \sum_{t=1}^T f_t(\bw)~.
    R_T =\E\left[\sum_{t=1}^T \fcomp_t(\tbw_{t-d+1},\dots,\tbw_t)\right] - \min_{\bw} \sum_{t=1}^T \fcomp_t(\bw,\dots,\bw)~.
\]
As in previous sections, we build a wrapper around a base Bandit Convex Optimization algorithm (Base BCO) which operates in the standard BCO framework with standard losses with range in $[0,1]$. Base BCO maintains at each round $t$ a state variable $\bw_t$ which is randomly perturbed to obtain the actual play $\tbw_t \in \Omega$. The wrapper algorithm is described as Algorithm~\ref{a:delayed-bco}.
%
%% ----------------------------------------------------------------------------
\begin{algorithm2e}[t]
\SetKwSty{textrm} \SetKwFor{For}{{\bf For}}{}{}
\SetKwIF{If}{ElseIf}{Else}{if}{}{else if}{else}{}
\SetKwFor{While}{while}{}{}
\textbf{Input:} Base BCO algorithm $A$ with parameter $\eta \in (0,1]$.\\%, and meta-parameters for it $\xi$.\\
\textbf{Initialize:}
\begin{itemize}[topsep=0pt,parsep=0pt,itemsep=0pt]
\item Play any $\bw_1 \in \Omega$;
\item If $B_0 = 1$ then $t=0$ is an update round.
\end{itemize}
%
\For{$t=1,2,\dots$:} { {
\begin{enumerate}[topsep=0pt,parsep=0pt,itemsep=0pt]
\item If $t-1$ was an update round, then play $\tbw_t$ by randomly perturbing state variable $\bw_t$ without updating $\bw_t$ (draw round, $\bw_{t+1} = \bw_t$);
\item Else if an update round was in the interval $\{t-2d+1, \dots,t-2\}$ then play $\tbw_t = \tbw_{t-1}$ without updating $\bw_t$
(stay round, $\bw_{t+1} = \bw_t$);
\item Else play $\tbw_t = \tbw_{t-1}$ (stay round), and if $B_t=1$ then the stay round becomes an update round. In such a case:
%
\begin{minipage}{\textwidth-50pt}
\begin{itemize}[topsep=0pt,parsep=0pt,itemsep=0pt]
\vspace{0.05in}
\item Feed Base BCO $A(\eta)$ with average composite loss\footnote
{
Recall that when $t \leq 0$, we defined $\alpha^{(s)}_{t} =0$, for all $s$, so the initial stretch of $2d-2$ actions $\tbw_1,\ldots,\tbw_{2d-2}$ can be disregarded here at the price of an extra additive $\scO(d)$ regret in the analysis.
}
%
\[
\avglossf_t = \frac{1}{2d}\sum_{\tau=t-d+1}^t \fcomp_\tau(\tbw_{\tau-d+1},\dots,\tbw_{\tau})
\]
\item Use the update rule $\bw_t\rightarrow \bw_{t+1}$ of Base BCO to obtain the new state variable $\bw_{t+1}$.
%(the stay round becomes an update round).
\end{itemize}
\end{minipage}
\end{enumerate}
%\vspace{-0.2in}
} } \caption{The Composite Loss Wrapper for BCO.}
\label{a:delayed-bco}
\end{algorithm2e}
%% ----------------------------------------------------------------------------
%
The notion of stability of the Base BCO has now to refer also to the sequence of loss functions the algorithm is operating with. Notice that, unlike the standard notion of stability in Online Convex Optimization, the kind of stability we need here is a {\em backward} stability, for it involves the backward differences $f_{t+1}(\tbw_{t+1}) - f_{t+1}(\tbw_t)$, rather than the forward differences $f_{t}(\tbw_{t}) - f_{t}(\tbw_{t+1})$. Moreover, we have to consider only the positive part of the backward difference.
%
\begin{definition}\label{d:stabilitybco}
Let $A(\eta)$ be a Base BCO with learning rate $\eta$, and $\{\tbw_t\}_{t=1}^T$ be the sequence of plays produced by $A(\eta)$ during a run over $T$ rounds on the sequence of convex losses $\{f_t\}_{t=1}^T$. We say that $A(\eta)$ is $\xi$-{\em stable} w.r.t.\ $\{f_t\}_{t=1}^T$ if for any round $t$ we have that\footnote
{
Here and throughout, $[x]_+ = \max\{x,0\}$. The outer $[\cdot]_+$ in Definition \ref{d:stabilitybco} forces $\xi$ to be nonnegative. 
}
\[
    \Bigl[\E\Big[f_{t+1}(\tbw_{t+1}) -f_{t+1}(\tbw_t) \Big]\Bigl]_+ \leq \xi
\]
% deterministically
holds, where $\xi$ may depend on the input dimension $n$, the learning rate $\eta$, as well as on relevant properties of the loss functions $\{f_t\}_{t=1}^T$ and parameters of the algorithm.
\end{definition}
%
We call a Base BCO algorithm $A$ {\em nontrivial} w.r.t.\ the sequence of losses $\{f_t\}_{t=1}^T$ if, when applied to the standard setting on $\{f_t\}_{t=1}^T$, $A$ has a regret bound $R_A(T,n,\eta)$ which is concave (possibly linear) in $T$ for any $n \geq 1$, $\eta > 0$, and the other relevant parameters of the algorithm.
%
Theorem~\ref{thm:convex} below rests on the assumption that the properties of the loss functions $\{f_t\}_{t=1...T}$ that make the Base BCO algorithm $A$ work are inherited by the average composite loss functions
%
\begin{align*}
\avglossf_t(\bw) &= \frac{1}{2d}\sum_{\tau=t-d+1}^t \fcomp_\tau(\bw,\dots,\bw)~, \qquad {\mbox{for $t \geq 2d-2$}},
\end{align*}
%
the wrapper feeds to $A$. For the sake of concreteness, let us simply focus on boundedness, Lipschitzness, and $\beta$-smoothness w.r.t. the Euclidean norm $||\cdot||$. Recall that a convex function $f\,:\,\Omega \to [0,1]$ is said to be $\beta$-smooth (or, equivalently, to have $\beta$-Lispchitz continuous gradient) w.r.t.\ $||\cdot||$ if for all $\bw, \bw' \in \Omega$ we have $||\nabla f(\bw) - \nabla f(\bw')|| \leq \beta ||\bw-\bw'||$, where $\beta \geq 0$. Moreover, given constants $\beta_1,\beta_2, b_1, b_2 \geq 0$, if $f_1$ is $\beta_1$-smooth w.r.t.\ $||\cdot||$ and $f_2$ is $\beta_2$-smooth w.r.t.\ $||\cdot||$, then it is easy to see that $b_1 f_1 + b_2 f_2$ is ($b_1\beta_1+b_2\beta_2$)-smooth w.r.t.\ $||\cdot||$. The following proposition lists the relevant properties of the functions $\avglossf_t$ as immediate consequences of the properties of the functions $f_t$ (proven in the appendix).
%
\begin{proposition}\label{l:propcomposite}
Let $f_1,\ldots, f_T\,:\,\Omega \subseteq \R^n \to [0,1]$ be a sequence of convex loss functions, and $\avglossf_{2d-2}, \ldots, \avglossf_T \,:\,\Omega \subseteq \R^n \to \R^+$ be the corresponding sequence of average composite losses. Then the following holds.
\begin{enumerate}[topsep=0pt,parsep=0pt,itemsep=0pt]
\item
$
\avglossf_{t}(\bw) \in [0,1]
$ for all $\bw \in \Omega$~.
\item If, for some constant $L \geq 0$, the loss functions $f_1,\ldots, f_T$ are $L$-Lipschitz on $\Omega$ w.r.t.\ $||\cdot||$, then so are $\avglossf_{2d-2},\ldots, \avglossf_T$.
\item If, for some constant $\beta \geq 0$, the loss functions $f_1,\ldots, f_T$ are $\beta$-smooth w.r.t.\ $||\cdot||$, then so are $\avglossf_{2d-2},\ldots, \avglossf_T$.
\end{enumerate}
\end{proposition}
%
The following theorem, whose proof sketch is in the appendix, is the BCO counterpart to Theorem~\ref{thm:delay-generic}.
%
%\ncb{Define properties P.}
\begin{theorem}\label{thm:convex}
Let $A(\eta)$ be a $\xi$-stable and nontrivial Base BCO algorithm with learning rate $\eta$ and regret bound
$R_A(T,n,\eta)$ in the standard BCO setting on a sequence of convex losses $\{f_t\}_{t=1}^T$ enjoying Properties $P$ (e.g., a subset of those listed in Proposition \ref{l:propcomposite}). If Properties $P$ are inherited by $\{\avglossf_t\}_{t=2d-2}^T$~, then Algorithm~\ref{a:delayed-bco} with input $A(\eta)$ achieves regret $R_T$ satisfying
\[
R_T
\le
T\,\xi + 8(2d-1)R_A(T/2d,n,\eta)+\scO(d)~.
  %  2T\eta + 4(2d-1)\left(\frac{\ln K}{\eta} + \frac{\eta}{4d}KT\right)~.
\]
\end{theorem}
%
%
As an example, consider the Base BCO algorithm by \citet{st11} that works under the assumption of $\beta$-smoothness w.r.t.\ $||\cdot||$. This algorithm is a BCO variant of the SCRIBLe algorithm by \citet{ahr12}. The algorithm takes in input a learning rate $\eta$, a scaling parameter $\delta \in (0,1]$ (which will be set as a function of $\eta$), and a $\nu$-self-concordant (barrier) function $\Psi$ which we assume to be strongly convex w.r.t. $||\cdot||$. For instance, if $\Omega$ is defined by a set of $m$ linear constraints $\Omega = \{\bw \in \R^n\,:\, A\bw \leq b \}$, a standard choice of $\Psi$ is the sum of negative log distances to each boundary, i.e., $\Psi(\bw) = -\sum_{i=1}^m \log(b_i - \be_i^\top A\bw)$, where $b = (b_1,\ldots, b_m)^\top$, and $\be_i$ is the $i$-th unit vector in the canonical basis of $\R^m$. Then $\Psi$ is strongly convex w.r.t.\ $||\cdot||$, up to a strong convexity constant. The algorithm maintains at each round $t$ the state variable $\bw_t \in \Omega$, of the form
%
\begin{equation}\label{e:st11}
\bw_t = \argmin_{\bw\in \Omega}\ \eta\,\sum_{\tau = 1}^{t-1} \bw^\top{\hat g_{\tau}} + \Psi(\bw)~.
\end{equation}
%
Then, it computes a perturbed version $\tbw_t$ of $\bw_t$ as
\(
\tbw_t = \bw_t + \delta H_t^{-1/2}\,s_t~,
\)
where $H_t$ is the Hessian matrix $\nabla^2 \Psi(\bw_t)$, $s_t$ is drawn uniformly at random from the surface of the Euclidean $n$-dimensional unit ball $\B^n$, and $\delta = \delta(\eta) \in (0,1]$ is a scaling parameter. Finally, the update $\bw_t \rightarrow \bw_{t+1}$ amounts to computing the next vector ${\hat g_{t}}$ in~(\ref{e:st11}) as ${\hat g_{t}} = \frac{n}{\delta}\,f_t(\tbw_t)H_t^{1/2}s_t$,
% \citet{st11} show that the vector ${\hat g_{t}}$ is
an unbiased estimate of the gradient at $\bw_t$ of a smoothed version
%${\hat f_t}$
of $f_t$.
%, defined as
%\(
%{\hat f_t}(\bw) = \E_{b\sim \B^n} [f(\bw+\delta H_t^{-1/2} b)]~,
%\)
%where $b$ is drawn uniformly at random from $\B^n$.
From \citep{st11} one can bound $R_A(T,n,\eta) = R_A(T,n,\eta,\delta(\eta))$ as follows:
%
\begin{equation}\label{e:st11bound}
R_A(T,n,\eta) \leq \beta T\delta^2\scD^2 + \eta T\left(\frac{n}{\delta}\right)^2 + \frac{2\nu\log T}{\eta} + \left(\frac{2}{\scD} + \scD\beta\right)\sqrt{T}~,
\end{equation}
%
where $\scD = \max_{\bw,\bw'\in \Omega} ||\bw-\bw'||$ is the diameter of $\Omega$. Moreover, the following stability lemma can be shown (proven in the appendix).
%
\begin{lemma}\label{l:stabilityconvexscrible}
Let $f_1,\ldots,f_T\,:\,\Omega \subseteq \R^n \rightarrow [0, 1]$ be a sequence of $\beta$-smooth convex losses w.r.t. $||\cdot||$, and $\scD$ be the diameter of $\Omega$. Then the Base BCO algorithm by \cite{st11} is $\xi$-stable, with
$
\xi = \scO\left(\left(1/\scD + \scD\beta\right)\frac{\eta\,n}{\delta}  + \beta\delta^2\scD^2\right)
$.
\end{lemma}
%
Combining~(\ref{e:st11bound}) with Theorem~\ref{thm:convex} and Lemma~\ref{l:stabilityconvexscrible} implies the following regret bound for composite losses.
%
\begin{corollary}\label{c:st11}
If Algorithm~\ref{a:delayed-bco} is run with the abovementioned algorithm by \citet{st11} as Base BCO algorithm, with $\eta = \scO\left(\left(\frac{d\log (T/d)}{n\,T}\right)^{2/3}\right)$ and $\delta = \scO(\eta^{1/4}\,n^{1/2})$, then its regret for BCO with $d$-delayed composite anonymous feedback satisfies
$
R_T = \scO\left(\bigl(d\log (T/d)\bigl)^{1/3}\,(n\,T)^{2/3} + \sqrt{d\,T}\right)
$,
where the $\scO$ notation in the tuning of $\eta,\delta$ and in the bound on $R_T$ hides the constants $\beta$, $\scD$ and $\nu$.
\end{corollary}
%
\begin{remark}
A similar statement can be made in the special case of bandit linear optimization, where the losses $f_t$ are $\beta$-smooth with $\beta =0$. In this case, Corollary \ref{c:st11} with $\delta = 1$ and $\eta$ set appropriately gives a bound of the form $\scO\left(\sqrt{d\,n^2\,T\,\log(T/d)}\right)$. The rate $T^{2/3}$ shown in Corollary~\ref{c:st11} is the same as the one achieved by the Base BCO algorithm of \citet{st11}. Likewise, the rate $T^{1/2}$ achieved by Corollary~\ref{c:st11} for the linear case is the same as the one obtained by the analyses in \citep{ahr12,st11}. In both cases (and in line with the results in Sections \ref{s:wrapper} and \ref{s:lower}) we have an extra factor $\sqrt{d}$ introduced by the composite anonymous feedback.
\end{remark}


%\section{Conclusions and future work}

We give a computationally efficient PAC active halfspace learning algorithm that enjoys sharp attribute efficient label complexity bounds.
It combines the margin-based framework of~\cite{BBZ07,BL13} with iterative hard thresholding~\citep{BD09, GK09}.
The main novel technical component in our analysis is a uniform concentration bound of hinge losses over shrinking $\ell_1$ balls in the sampling regions.
We outline several promising directions of future research:
\begin{itemize}
\item Can we extend our algorithm to work under $\eta$-bounded noise, when $\eta$ is arbitrarily close to $\frac 1 2$? Recall that the results of \cite{ZC14}
imply a computationally inefficient algorithm with a label complexity of $O(\frac{t \ln d}{(1-2\eta)^2} \ln \frac 1 \epsilon)$ in this setting,
which state of the art computationally efficient algorithms~\citep[e.g.][]{ABHZ16} cannot achieve.
%A promising direction is to ``activize'' existing computationally and attribute efficient online halfspace learning algorithms, e.g.~\cite{L87, GLS01, G03}, as is done in the full-dimensional setting~\citep{DKM05, YZ17}.

\item Can we design attribute and computationally efficient active learning algorithms that work under broader distributions? Existing results in the active learning and one-bit compressed sensing literature have made substantial progress on settings when the unlabeled distribution is $\alpha$-stable~\citep{L16}, subgaussian~\citep{ALPV14, CB15}, or $s$-concave~\citep{BZ17}; an attribute and computationally efficient, statistically consistent recovery algorithm under any of the above settings would be a step forward.

%Recent work~\citep{L16} shows that when
%the unlabeled distribution is $\alpha$-stable for small $\alpha$, then the support of the target vector can be provably recovered. Can one approximately recover the target halfspace under this distribution?
%Can the result of \cite{L16} be extended to approximately
%recover $u$?

\item In one-bit compressed sensing, under the symmetric noise condition~\citep{PV13b}, algorithms with sample complexity polynomial in $\frac 1 \epsilon$ have been proposed~\citep{PV13b, ZYJ14, ZG15}.
Can we develop adaptive one-bit compressed sensing algorithms with $O(t \polylog(d,\frac 1 \epsilon))$ measurement complexity in this setting?
\end{itemize}


\section{Conclusions}
%
We have investigated the setting of $d$-delayed composite anonymous feedback as applied to nonstochastic bandits.
% Composite anonymous feedback lends itself to formalize scenarios where the actions perfomed by the online decision maker produce long-lasting effects that combine additively over time.
A general reduction technique was introduced that enables the conversion of a (backward stable) algorithm working in a standard bandit framework into one working in the composite feedback framework. In the case of $K$-armed bandits, we relied on a lower bound for bandit linear optimization in the probability simplex to show that no algorithm in the composite feedback framework can do better than $\scO(\sqrt{dKT})$. In turn, up to log factors, this is what we obtain as an upper bound by applying our reduction to the standard Exp3 algorithm. We showed the generality and flexibility of our conversion technique by further applying it to Combinatorial Bandits (the Exp2 algorithm) and to Bandit Convex Optimization (the self-concordant barrier-based algorithms by \cite{ahr12} and \cite{st11}) with smooth/linear loss functions.

Three main directions for extending our work are:
\begin{itemize} 
\item Proving an upper bound for the case of nonoblivious adversaries;
\item Investigating the setting where the delay parameter $d$ is not perfectly known;
\item Extending our results to the nonstochastic contextual case.
\end{itemize}

\iffalse
 ****** CG: still working on it *****

In the latter case, the resulting Bandit Convex Optimization algorithm for composite anonymous feedback turned out to achieve regret bounds with higher rates than those of the corresponding base bandit algorithms, and we suspect that there is margin for improvement in our stability analysis of Lemma \ref{l:stabilityconvexscrible}.

Say something about possible directions
\fi


% Acknowledgments---Will not appear in anonymized version
\acks{YM was supported in part by a grant from the Israel Science
Foundation (ISF).}
\newpage

\bibliography{ncb}


\appendix
%





\section{Proof of Theorem \ref{thm:delay-generic}}

\begin{proof}
Let $\scU \subseteq \{1, \ldots, T\}$ be the (random) subset of update
rounds. Let us call for brevity an update round a $u$-round, and
similarly for the other two kinds. First, observe that if $t$ is a $u$-round, 
we have $\avgloss_t \in [0,1]$. This is because, due to~(\ref{eq:compsum}) and the fact that $I_t = I_{t-1} = \ldots = I_{t-2d+1}$,
%
\begin{align*}
	\avgloss_t 
=
	\frac{1}{2d}\sum_{\tau=t-d+1}^t \lcomp_\tau(I_{\tau-d+1},\dots,I_{\tau})
=
	\frac{1}{2d}\sum_{\tau=t-d+1}^t \lcomp_\tau(I_{t},\dots,I_{t})
\le
   \frac{1}{2d}(2d-1) < 1~.
\end{align*}
%
% First, observe that for any $t \in \scU$ we update using $loss_t$,
% where
%\[
% loss_t = \frac{1}{2d}\sum_{\tau=t-d}^t
% \loss_\tau(I_\tau,\dots,I_{\tau-d+1})\in[0,1]
% \]
% For $t\in \scU$ we have $I_t=\cdots=I_{t-2d+1}$, which implies that
% for any $\tau \in [t-2d+1,t]$ we have
% $\loss_\tau(I_{\tau-d+1},\dots,I_{\tau}) =
% \loss_\tau(I_{t-2d+1},\dots,I_{t-2d+1})$. So that
% \[
%     loss_t = \frac{1}{2d}\sum_{\tau=t-d}^t \sum_{s=0}^{d-1} \loss_{\tau-s}^{(s)}(I_{t-2d+1})\in[0,1]~.
% \]
%
Since a $u$-round is followed by an $r$-round, and during the stretch of $s$-rounds between an $r$-round 
and the next $u$-round the action played by Algorithm \ref{a:delayed-app} does not change, the algorithm
behaves exactly as $A(\eta)$ on the steps in $\scU$. Therefore, if we set for brevity
\[
\Delta_t^k = \frac{1}{2d}\sum_{\tau=t-d+1}^t \Big(\lcomp_\tau(I_{\tau-d+1},\dots,I_{\tau-d+1}) - \lcomp_\tau(k,\dots,k)\Big)~,
\qquad\text{for $t \geq 2d-2$},
\]
we have, for any action $k$,
%
\begin{equation}
\label{e:upperupdate}
	\E\left[\sum_{t \in \scU,\, t\geq 2d-2} \Delta_t^k\right]
\le 
	\E\big[R_A(|\scU|,K,\eta)\big] 
\le
	R_A\big(\E[|\scU|],K,\eta\big)
\le
	R_A(T/2d,K,\eta)~,
%    \le \frac{\ln K}{\eta} + \frac{\eta}{2}K\,\E\bigl[|\scU|\bigr] \le \frac{\ln K}{\eta} + \frac{\eta}{4d}KT
\end{equation}
%
where the second-last inequality is due to the concavity of $R_A(\cdot, K,\eta)$, and
the last inequality derives from $|\scU| \leq T/2d$, for there
can be at most one $u$-round every $2d$ rounds.
%({\bf CG:} I believe one can easily show $\E\bigl[|\scU|\bigr] \leq T/(3d)$ when $q=1/d$.)
Now, notice that by definition of a $u$-round we have, for all $t$,
\[
\Ind{t \in \scU} = \Ind{B_t = 1}\,\Ind{\bigwedge_{s=1}^{2d-1}(t-s\not\in \scU)}~.
\]
Moreover,
% if we set for brevity
% \[
% \Delta_t^k=\avgloss_t - \frac{1}{2d}\sum_{\tau=t-d+1}^t \loss_\tau(k,\dots,k)~,
% \]
% for $t \geq 2d-2$, we can write
%
\begin{align}
	\sum_{t=2d-2}^T \Delta_t^k 
&=
	\frac{1}{2d}\sum_{t=2d-2}^T\sum_{\tau=t-d+1}^t \Big( \lcomp_\tau(I_{\tau-d+1},\dots,I_{\tau-d+1}) - \lcomp_\tau(k,\dots,k) \Big)
\nonumber
\\&= 
	\frac{1}{2d}\sum_{t=d-1}^{2d-3} (t-d+2)\Big( \lcomp_t(I_{t-d+1},\dots,I_{t-d+1}) - \lcomp_t(k,\dots,k) \Big)
\nonumber
\\&
	\quad + \frac{d}{2d}\sum_{t=2d-2}^{T-d+1} \Big( \lcomp_t(I_{t-d+1},\dots,I_{t-d+1}) - \lcomp_t(k,\dots,k) \Big)
\nonumber
\\&
	\quad +\frac{1}{2d}\sum_{t=T-d+2}^{T} (T+1-t)\Big( \lcomp_t(I_{t-d+1},\dots,I_{t-d+1}) - \lcomp_t(k,\dots,k) \Big)
\nonumber
\\&\geq 
	\frac{1}{2}\sum_{t=2d-2}^{T-d+1} \Big( \lcomp_t(I_{t-d+1},\dots,I_{t-d+1}) - \lcomp_t(k,\dots,k) \Big)
\nonumber
\\&
	\quad - \frac{1}{2d}\sum_{t=d-1}^{2d-3} (t-d+2)\lcomp_t(k,\dots,k) - \frac{1}{2d}\sum_{t=T-d+2}^{T} (T+1-t)\lcomp_t(k,\dots,k)
\nonumber
\\&\ge
	\frac{1}{2}\sum_{t=2d-2}^{T-d+1} \Big( \lcomp_t(I_{t-d+1},\dots,I_{t-d+1}) - \lcomp_t(k,\dots,k) \Big) - 2(d-1)~,
\label{e:lowerdelta}
\end{align}
where the last inequality holds because, due to~(\ref{eq:compsum}),
\[
	\sum_{t=d-1}^{2d-3} (t-d+2)\lcomp_t(k,\dots,k) \le (d-1)\sum_{t=d-1}^{2d-3} \lcomp_t(k,\dots,k) \le (d-1)(2d-1)
\]
and
\[
	\sum_{t=T-d+2}^{T} (T+1-t)\lcomp_t(k,\dots,k) \le (d-1)\sum_{t=T-d+2}^{T} \lcomp_t(k,\dots,k) \le (d-1)(2d-1)~.
\]
Now, for any action $k$ we have,
\begin{align}
    \E&\left[\sum_{t \in \scU,\, t\geq 2d-2} \Delta_t^k \right]
=
    \E\left[\sum_{t=2d-2}^T \Ind{t \in \scU} \Delta_t^k \right]
\nonumber\\ &=
    \E\left[\sum_{t=2d-2}^T \Ind{B_t = 1}\,\Ind{\bigwedge_{s=1}^{2d-1}(t-s \not\in \scU)} \Delta_t^k \right]
\nonumber\\ &=
    q\,\E\left[\sum_{t=2d-2}^T \E\left[\Ind{\bigwedge_{s=1}^{2d-1}(t-s \not\in \scU)}\Delta_t^k \,\bigg|\, B_0,\dots,B_{t-2d},I_0,\ldots,I_{t-2d+1}
    \right]\right]
\nonumber\\ &=
    q\,\E\left[\sum_{t=2d-2}^T \Delta_t^k\,\E\left[\Ind{\bigwedge_{s=1}^{2d-1}(t-s \not\in \scU)} \,\bigg|\, B_0,\dots,B_{t-2d},I_0,\ldots,I_{t-2d+1}  \right]
    \right]
\nonumber\\ &=
    q\,\E\left[\sum_{t=2d-2}^T \Delta_t^k\,\E\left[\Ind{\bigwedge_{s=1}^{2d-1}(t-s \not\in \scU)} \,\bigg|\, B_0,\dots,B_{t-2d} \right]
     \right]
\nonumber\\ &=
    q\,\E\left[\sum_{t=2d-2}^T \Delta_t^k\,\Pr'\left(\bigwedge_{s=1}^{2d-1}(t-s \not\in \scU) \right)
     \right]\,,\label{e:expecteddelta}
\end{align}
where we set for brevity $\Pr'(\cdot) = \Pr\bigl(\,\cdot \mid
B_0,B_1\dots,B_{t-2d}\bigr)$.
%
We can thus write
%
\begin{align*}
    1 - \Pr'\left(\bigwedge_{s=1}^{2d-1}(t-s \not\in \scU) \right)
&=
    \Pr'\left(\bigvee_{s=1}^{2d-1}(t-s \in \scU) \right)
\\ &\le
    \sum_{s=1}^{2d-1}\Pr'\bigl(t-s \in \scU\bigr)
\\ & \le
    \sum_{s=1}^{2d-1}\Pr'\bigl(B_{t-s} = 1, t-s-1 \notin \scU, \ldots, t-s-2d+1 \notin \scU\bigr)
\\ & \le
    q\,(2d-1)\,.
\end{align*}
%
Hence, substituting into~(\ref{e:expecteddelta}) and combining with~(\ref{e:lowerdelta}), we conclude that
%
\begin{align}
	\E&\left[\sum_{t \in \scU,\,t\geq 2d-2} \Delta_t^k \right] 
\ge
	q(1-q(2d-1))\sum_{t=2d-2}^T \E\big[\Delta_t^k\big]
\nonumber
\\&\ge
    \frac{q}{2}(1-q(2d-1))\left( \sum_{t=2d-2}^{T-d+1} \Big(\E\big[\lcomp_t(I_{t-d+1},\dots,I_{t-d+1})\big] - \lcomp_t(k,\dots,k)\Big) - 2(d-1) \right)~.
     \label{e:updatenoupdate}
\end{align}
%
Next, using $\E_t[\cdot]$ to denote expectation conditioned on all random events at time steps $1,\dots,t-1$, we observe that
%
\begin{align}
\nonumber
    \E&\Bigl[ \lcomp_t(I_{t-d+1},\dots,I_t) - \lcomp_t(I_{t-d+1},\dots,I_{t-d+1}) \Bigr]
\\&=
\nonumber
    \E\left[ \sum_{s=0}^{d-1} \Bigl( \E_{t-s}\big[\loss_{t-s}^{(s)}(I_{t-s})\big] - \E_{t-d+1}\big[\loss_{t-s}^{(s)}(I_{t-d+1})\big] \Bigr)\right]
\\&=
\nonumber
    \E\left[ \sum_{s=0}^{d-1} \sum_{i=1}^K \loss_{t-s}^{(s)}(i) \bigl(p_{t-s}(i) - p_{t-d+1}(i)\bigr) \right]
\\&\le
\label{eq:stability} 
    \E\left[ \sum_{s=0}^{d-1} \sum_{i\,:\,p_{t-s}(i) > p_{t-d+1}(i)} \bigl(p_{t-s}(i) - p_{t-d+1}(i)\bigr) \right]
%\\ &\le
%    \sum_{s=0}^{d-1} \norm{\bp_{t-s} - \bp_{t-d+1}}_1
\le
    \xi~,
\end{align}
%
where the last inequality is because in any block of $2d$ rounds there is
at most one update of distribution $\bp_t$, each loss component $\loss_{t-s}^{(s)}(i)$ is in $[0,1]$, and because $A(\eta)$ is a $\xi$-stable 
Base MAB.
%update of $A$ changes the distribution by at most $2\eta$ in $1$-norm.
Hence we can write
\begin{align*}
	R_T
&\le
	\E\left[\sum_{t=1}^T \lcomp_t(I_{t-d+1},\dots,I_t)\right] - \sum_{t=1}^T \lcomp_t(k,\dots,k)
\\&=
	\E\left[\sum_{t=1}^T \lcomp_t(I_{t-d+1},\dots,I_t) - \sum_{t=1}^T \lcomp_t(I_{t-d+1},\dots,I_{t-d+1})\right]
\\&
	\quad + \E\left[\sum_{t=1}^T \lcomp_t(I_{t-d+1},\dots,I_{t-d+1})\right] - \sum_{t=1}^T \lcomp_t(k,\dots,k)
\\&\le
	T\,\xi + \underbrace{\E\left[\sum_{t=1}^T \lcomp_t(I_{t-d+1},\dots,I_{t-d+1})\right] - \sum_{t=1}^T \lcomp_t(k,\dots,k)}_{(\star)}
	\tag{from~(\ref{eq:stability})}
\end{align*}
%
Furthermore,
%
\begin{align*}
	(\star)
&\leq
	\E\left[\sum_{t=2d-2}^{T-d+1} \lcomp_t(I_{t-d+1},\dots,I_{t-d+1})\right] - \sum_{t=2d-2}^{T-d+1} \lcomp_t(k,\dots,k) + 3d
\\&\leq
	q(d-1) + \frac{2}{q(1-q(2d-1))}\,\E\left[\sum_{t\in \scU,\,t\geq 2d-2} \Delta_t^k \right] + 3d
\tag{from~(\ref{e:updatenoupdate})}
\\&\leq
	q(d-1) + \frac{2}{q(1-q(2d-1))}\,R_A(T/2d,K,\eta) + 3d~.
\tag{from~(\ref{e:upperupdate})}
\end{align*}
%
By picking $q = \tfrac{1}{2(2d-1)}$ so as to maximize the denominator in the second term of the right-most side yields
$
(*) \leq 8(2d-1)\,R_A(T/2d,K,\eta) + \scO(d)
$, 
so that
$
R_T \leq T\,\xi + 8(2d-1)\,R_A(T/2d,K,\eta) + \scO(d),
$
as claimed.
\end{proof}




\section{Proof of Lemma \ref{l:stabilityexp3}}
\begin{proof}
In this case, stability holds pointwise (for all realizations of $I_1,\dots,I_T$) rather that in expectation. From~\cite[Lemma 1]{cgmm16} we have, for any round $t$,
\[
	p_{t+1}(i) - p_t(i) \leq \eta\,p_{t+1}(i) \sum_{j=1}^K p_t(j) \ellhat_t(j)
\]
%
Hence we can write
%
\begin{align*}
\sum_{i\,:\,p_{t+1}(i) > p_t(i)} p_{t+1}(i)-p_t(i)
&\leq
\sum_{i\,:\,p_{t+1}(i) > p_t(i)} \eta\,p_{t+1}(i) \sum_{j=1}^K p_t(j) \ellhat_t(j)\\
&=
\sum_{i\,:\,p_{t+1}(i) > p_t(i)} \eta\,p_{t+1}(i) \ell_t(I_t)\\
&\leq
\eta\,\sum_{i\,:\,p_{t+1}(i) > p_t(i)} p_{t+1}(i)\\
&\leq
\eta
\end{align*}
concluding the proof.
\end{proof}



\section{Proof of Lemma \ref{l:stabExp2}}
%
\begin{proof}
Since $\bq_t$ has exponential form, we can apply again~\cite[Lemma 1]{cgmm16} and obtain
\[
	p_{t+1}(\bv) - p_t(\bv)
=
	(1-\gamma)\big(q_{t+1}(\bv) - q_t(\bv)\big)
\le
	(1-\gamma)\eta\,q_{t+1}(\bv) \sum_{\bv'\in\scK} q_t(\bv')\,\blhat_t^{\top}\bv'~.
\]
%
Hence we can write
%
\begin{align*}
	\E&\left[\sum_{\bv \,:\, p_{t+1}(\bv) > p_t(\bv)} p_{t+1}(\bv) - p_t(\bv) \right]
\\&\le
	(1-\gamma)\eta\,\E\left[\sum_{\bv \,:\, p_{t+1}(\bv) > p_t(\bv)} q_{t+1}(\bv) \sum_{\bv'\in\scK} q_t(\bv') \E_t\Big[\blhat_t^{\top}\bv'\Big] \right]
\\&=
	(1-\gamma)\eta\,\E\left[\sum_{\bv \,:\, p_{t+1}(\bv) > p_t(\bv)} q_{t+1}(\bv) \sum_{\bv'\in\scK} q_t(\bv') \bloss_t^{\top}\bv' \right]
	\tag{because estimates are unbiased}
\\&\le
	(1-\gamma)\eta\,\E\left[\sum_{\bv \,:\, p_{t+1}(\bv) > p_t(\bv)} q_{t+1}(\bv) \right]
	\tag{because $\bloss_t^{\top}\bv \in [0,1]$ for all $t$ and $\bv$}
\\&\le
	(1-\gamma)\eta
\end{align*}
concluding the proof.
\end{proof}



\section{Proof of Lemma \ref{lem:lower}}

\begin{proof}
Fix an algorithm $A_d$ for the $d$-delayed bandit setting. Given an assignment $\bloss_1,\dots,\bloss_{T/d}$ of loss vectors for the linear bandit setting, define, for each $i=1,\dots,K$, the loss components of the $d$-delayed bandit setting:
\[
	\loss_t^{(s)}(i) = \left\{ \begin{array}{cl}
		\loss_{\lceil t/d\rceil}(i) & \text{if $t+s\!\!\!\pmod d=0$,}
	\\[1mm]
		0 & \text{otherwise.}
	\end{array} \right.
\]
These components define the following composite loss for $A_d$ playing actions $I_t$,
\[
	\lcomp_t(I_{t-d+1},\dots,I_t)
=
	\sum_{s=0}^{d-1} \loss_{t-s}^{(s)}(I_{t-s})
=	
	\left\{ \begin{array}{cl}
		d\,\bp_t^{\top}\bloss_{\lceil t/d\rceil} & \text{if $t\!\!\!\pmod d=0$,}
	\\[1mm]
		0 & \text{otherwise}
	\end{array} \right.
\]
where $\bp_t$ is defined from $I_{t-d+1},\dots,I_t \in \{1,\dots,K\}$ as follows
\begin{equation}
\label{eq:p-reduction}
	p_t(j) = \frac{1}{d}\sum_{s=t-d+1}^t \Ind{I_s = j} \qquad j=1,\dots,K.
\end{equation}
Essentially, $A_d$ observes a nonzero composite loss only every $d$ time steps, when $t\pmod d=0$. When this happens, the composite loss is $d$ times the linear loss $\bp_t^{\top}\bloss_{\lceil t/d\rceil}$, where $p_t(i)$ is the fraction of times action $i$ was played by $A_d$ in the last $d$ rounds.
%
Given the algorithm $A_d$, we define the algorithm $A$ as follows. If $t\pmod d\neq 0$, then $A$ skips the round. On the other hand, when $t\pmod d=0$, $A$ performs action $\bp_t$ defined in~(\ref{eq:p-reduction}), observes the loss $\bp_t^{\top}\bloss_{\lceil t/d\rceil}$, and returns to $A_d$ the composite loss $\lcomp_t(I_{t-d+1},\dots,I_t)$.

Note that the loss of $A_d$ is $d$ times the loss of $A$. Also, pick any $\bp\in\Delta_K$ such that each component $p(i)$ has the form $\frac{k}{d}$ for some $k\in\{0,\dots,d\}$. Then, for any $t$ that divides $d$, for any $\bloss_t$, and for any $i_{t-d+1},\dots,i_t\in\{1,\dots,K\}$ such that
\[
	p(j) = \frac{1}{d}\sum_{s=t-d+1}^t \Ind{i_s = j} \qquad j=1,\dots,K
\]
the composite loss $\lcomp_t(i_{t-d+1},\dots,i_t)$ is exactly $d$ times the linear loss $\bp^{\top}\bloss_t$. Finally, noting that $A$ experiences $T/d$ rounds concludes the proof.
\end{proof}





\section{Proof of Proposition \ref{l:propcomposite}}

\begin{proof}
Simply observe that 
%
\begin{align*}
\avglossf_t(\bw) 
&= \frac{1}{2d}\sum_{\tau=t-d+1}^t \sum_{s=0}^{d-1} \alpha_{\tau-s}^{(s)}f_{\tau-s}(\bw)\\
&= \frac{1}{2d}
    \left(
    \sum_{\tau=t-2d+2}^{t-d} \sum_{s=t-d+1}^{\tau+d-1} \alpha_{\tau}^{(s-\tau)}f_{\tau}(\bw)
    + 
    \sum_{\tau=t-d+1}^{t} \sum_{s=\tau}^{\tau+d-1} \alpha_{\tau}^{(s-\tau)}f_{\tau}(\bw)
    \right)\\
&= \frac{1}{2d}
    \left(
    \sum_{\tau=t-2d+2}^{t-d} f_{\tau}(\bw) \left(\sum_{s=t-d+1}^{\tau+d-1} \alpha_{\tau}^{(s-\tau)}\right)
    + 
    \sum_{\tau=t-d+1}^{t} f_{\tau}(\bw)
    \right)~. 
\end{align*}
%
Now, since the first inner sum $\sum_{s=t-d+1}^{\tau+d-1} \alpha_{\tau}^{(s-\tau)}$ is upper bounded by $\sum_{s=0}^{d-1} \alpha_{\tau}^{(s)} = 1$, we
see that $\avglossf_t(\bw)$ is indeed a linear combination of the form
%
%\begin{equation}\label{e:fcomposite}
\[
\avglossf_t(\bw) = \sum_{\tau=t-2d+2}^{t} b_{\tau} f_{\tau}(\bw)~,
\]
%\end{equation}
%
whose coefficients $b_{\tau}$ are nonnegative and sum to a quantity which is less than one. All the three claimed properties then immediately follow.
%
\end{proof}




\section{Proof of Lemma \ref{thm:convex}}

\begin{proof}
The proof is almost the same as the one of Theorem~\ref{thm:delay-generic} (up to a change of notation), with the additional care that has to be taken when dealing with the inheritance of Properties $P$ from $\{f_t\}_{t=1}^T$ to $\{\avglossf_t\}_{t=2d-2}^T$. 
%
In particular, if we define
\[
\Delta_t^{\bw} = \frac{1}{2d}\sum_{\tau=t-d+1}^t \bigl(\fcomp_\tau(\tbw_{\tau-d+1},\dots,\tbw_{\tau-d+1}) - \fcomp_\tau(\bw,\dots,\bw)\bigl),\qquad {\mbox{for $t \geq 2d-2$,}}
\]
we have, for any $\bw \in \Omega$,
\[
\E\left[\sum_{t \in \scU,\, t\geq 2d-2} \Delta_t^{\bw}  \right] \leq R_A(T/2d, n ,\eta)~,
\]
since the average loss function $\avglossf_t(\bw)$ enjoys the same properties as those that allow us to prove the regret bound $R_A(T, n ,\eta)$ for the Base BCO algorithm $A$.
%
Next,
%
\begin{align*}
\E&\left[\sum_{t \in \scU,\,t\geq 2d-2} \Delta_t^{\bw} \right] \\
  &\geq
    \frac{q}{2}(1-q(2d-1))\left(\sum_{t=2d-2}^{T-d+1} \Bigl(\E\left[\fcomp_t(\tbw_{t-d+1},\dots,\tbw_{t-d+1})\right] - \fcomp_t(\bw,\dots,\bw)\Bigr) -2(d-1)\right)
\end{align*}
%
is the counterpart to~(\ref{e:updatenoupdate}), and is proved in exactly the same manner. Then, from the notion of stability given in Definition~\ref{d:stabilitybco}, we can write
%
\begin{align*}
    \E&\Bigl[ \fcomp_t(\tbw_{t-d+1},\dots,\tbw_t) - \fcomp_t(\tbw_{t-d+1},\dots,\tbw_{t-d+1}) \Bigr]
\\&=
    \E\left[ \sum_{s=0}^{d-1} \Bigl( f_{t-s}^{(s)}(\tbw_{t-s}) - f_{t-s}^{(s)}(\tbw_{t-d+1})   \Bigr)\right]\\
&=
    \sum_{s=0}^{d-1}\alpha_{t-s}^{(s)}\,\E\left[ f_{t-s}(\tbw_{t-s}) - f_{t-s}(\tbw_{t-d+1})\right]\\
&\leq
    \sum_{s=0}^{d-1}\alpha_{t-s}^{(s)}\,\Bigl[\E\left[ f_{t-s}(\tbw_{t-s}) - f_{t-s}(\tbw_{t-d+1})\right]\Bigl]_+
\leq
	\xi~,
\end{align*}
since there is at most one update of the underlying state variable $\bw_t$ (which in turn determines the distribution of the corresponding $\tbw_t$)
during the rounds from $t-d+1$ to $t$, the coefficients $\alpha_{t-s}^{(s)}$ are in $[0,1]$ for all $s$ and $t$, and Base BCO is assumed to be $\xi$-stable in 
the sense of Definition~\ref{d:stabilitybco}.
%
Piecing together as in the proof of Theorem~\ref{thm:delay-generic} proves the claim.
\end{proof}





\section{Proof of Lemma \ref{l:stabilityconvexscrible}}

\begin{proof}
First recall the standard fact that if $f\,:\,\Omega \rightarrow [0,1]$ is $\beta$-smooth w.r.t. $||\cdot||$, then
\[
f(\bw') \leq f(\bw) + \nabla f(\bw)^\top(\bw'-\bw) + \frac{\beta}{2}\,||\bw'-\bw||^2~.
\]
Let $\E_t[\cdot]$ denote expectation conditioned on all random events up to time $t-1$. Then, by the convexity of $f_{t+1}$, we have
%
\begin{align*}
\E[f_{t+1}(\tbw_{t})] 
= \E\big[\E_{t}[f_{t+1}(\tbw_{t})]\big]
\geq \E\big[f_{t+1}(\E_{t}[\tbw_{t}])\big]
= \E\big[f_{t+1}(\E_{t}[\bw_{t}])\big]
= \E[f_{t+1}(\bw_{t})]~.
\end{align*}
%
Moreover, by the $\beta$-smoothness of $f_{t+1}$, we can write
%
\begin{align*}
\E[f_{t+1}(\tbw_{t+1})] 
&= \E\big[\E_{t+1}[f_{t+1}(\tbw_{t+1})]\big]\\
&= \E\big[\E_{t+1}[f_{t+1}(\bw_{t+1} + \delta\,H_{t+1}^{-1/2}s_{t+1})]\big]\\
&\leq \E\left[\E_{t+1}\left[f_{t+1}(\bw_{t+1}) + \delta\,\nabla f_{t+1}(\bw_{t+1})^\top H_{t+1}^{-1/2}s_{t+1}  + \frac{\beta\delta^2}{2}s_{t+1}^\top H_{t+1}^{-1}s_{t+1}\right]\right]\\
&= \E\left[f_{t+1}(\bw_{t+1}) + \E_{t+1}\left[\frac{\beta\delta^2}{2}s_{t+1}^\top H_{t+1}^{-1}s_{t+1}\right]\right]\\
&= \E\left[f_{t+1}(\bw_{t+1})\right] + \frac{\beta\delta^2}{2}\,\E\left[||H_{t+1}^{-1/2}s_{t+1}||^2\right]\\
&\leq \E\left[f_{t+1}(\bw_{t+1})\right] + \frac{\beta\delta^2\scD^2}{2}~,
\end{align*}
%
the last inequality following from the properties of the Dikin ellipsoid associated with the self-concordant barrier $\Psi$, ensuring that $\bw_{t+1} + H_{t+1}^{-1/2}s_{t+1}$ belongs to $\Omega$, hence bounding $||H_{t+1}^{-1/2}s_{t+1}||$ by the diameter $\scD$.
%
Putting together, we have so far obtained
%
\begin{align}\label{e:1}
\Bigl[\E\left[f_{t+1}(\tbw_{t+1}) - f_{t+1}(\tbw_{t})\right]\Bigl]_+ 
&\leq 
\left[\E\left[f_{t+1}(\bw_{t+1}) - f_{t+1}(\bw_{t})\right]  + \frac{\beta\delta^2\scD^2}{2}\right]_+ \notag\\
&\leq 
\Bigl[\E\left[f_{t+1}(\bw_{t+1}) - f_{t+1}(\bw_{t})\right]\Bigl]_+ + \frac{\beta\delta^2\scD^2}{2}~,
\end{align}
%
where we have further used the fact that $[a]_+$ is nondecreasing in $a \in \R$, and that $[a+b]_+ \leq [a]_+ + [b]_+$ for all $a,b \in \R$.

\iffalse
***************************************

For similar reasons, recalling the definition of ${\hat f_t}$, the smoothed version of $f_t$, for any fixed $\bw \in \Omega$ we have
%
\begin{align*}
{\hat f_t}(\bw) - f_t(\bw) 
&= \E_{b\sim \B^n} [f_t(\bw+\delta H_t^{-1/2} b) - f_t(\bw)] \\
&\leq \E_{b\sim \B^n} \left[\delta\,\nabla f_{t}(\bw)^\top H_{t}^{-1/2}v  + \frac{\beta\delta^2}{2}v^\top H_{t}^{-1}v\right]\\
&= \frac{\beta\delta^2}{2}\E[v^\top H_{t}^{-1}v]\\
&\leq \frac{\beta\delta^2\scD^2}{2}~.
\end{align*}
%
Hence, we can bound the absolute value of the difference of losses in~(\ref{e:1}) as
%
\begin{align*}
|f_{t+1}(\bw_{t+1}) - f_{t+1}(\bw_{t})| 
&\leq |f_{t+1}(\bw_{t+1}) - {\hat f_{t+1}}(\bw_{t+1})|\\ 
&\ \qquad \ + |{\hat f_{t+1}}(\bw_{t+1}) - {\hat f_{t+1}}(\bw_{t})|\\ 
&\ \qquad \ +  |{\hat f_{t+1}}(\bw_{t}) - f_{t+1}(\bw_{t})|\\ 
&\leq |{\hat f_{t+1}}(\bw_{t+1}) - {\hat f_{t+1}}(\bw_{t})| + \beta\delta^2\scD^2~,
\end{align*}
%
so that, combining with`(\ref{e:1}) this reduces to
%
\begin{equation}
\E\left[f_{t+1}(\tbw_{t+1}) - f_{t+1}(\tbw_{t})\right] \leq \E\left[\bigl|{\hat f_{t+1}}(\bw_{t+1}) - {\hat f_{t+1}}(\bw_{t})\bigl| \right] + \frac{3\beta\delta^2\scD^2}{2}~.
\end{equation}
%
Now, since ${\hat g_t}$ is an unbiased estimator of $\nabla {\hat f_t}(\bw_t)$, . . .
****************************************
\fi

Now, consider the Bregman divergence associated with the (strongly convex) barrier function $\Psi$:
\[
B_{\Psi}(\bw,\bw') = \Psi(\bw)-\Psi(\bw') -\nabla \Psi(\bw')^\top(\bw-\bw')~.
\]
Since the sequence ${\bw_t}_{t=1...T}$ are generated by a Follow The Regularized Leader (FTRL) algorithm, we have ---see, e.g., \citep[Equation~(5.2)]{hazan16}
\[
B_{\Psi}(\bw_t,\bw_{t+1}) \leq \eta{\hat g_t}^\top(\bw_t-\bw_{t+1}) \leq \eta ||{\hat g_t}||^*_t\,||\bw_t-\bw_{t+1}||_t~,
\]
where $||\cdot||_t$ is the local norm induced by the Hessian of $\Psi$ at $\bw_t$, i.e., $||\bw||_t = \big(\bw^\top \nabla^2 \Psi (\bw_t) \bw\big)^{1/2}$, and $||\cdot||^*_t$ is its dual, $||\bw||^*_t = \big(\bw^\top (\nabla^2 \Psi (\bw_t))^{-1} \bw\big)^{1/2}$. By the strong convexity of $\Psi$ w.r.t.\ $||\cdot||$ we have
\[
B_{\Psi}(\bw_t,\bw_{t+1}) \geq \frac{\alpha}{2}\,||\bw_t-\bw_{t+1}||^2~,
\]
for some constant $\alpha > 0$. Moreover, one can show that $||{\hat g_t}||^*_t \leq n/\delta$ \citep{st11} and, provided $\eta \leq \frac{\delta}{16n}$, also that $||\bw_t-\bw_{t+1}||_t \leq 1/2$ \citep{ahr12}, so that
%
\begin{equation}\label{e:2}
||\bw_t-\bw_{t+1}|| = \scO\left(\sqrt{\frac{\eta\,n}{\delta}}\right)~,
\end{equation}
%
where the $\scO$ notation hides here the inverse dependence on $\alpha$.
Finally, since $f_t$ is [0,1]-bounded and $\beta$-smooth on a set of diameter $\scD$, it must be that $f_t$ is also Lipschitz with constant $L \leq \frac{2}{\scD} + \scD\beta$, so that combining with~(\ref{e:1}) and~(\ref{e:2}) yields
%
\begin{align*}
\Bigl[\E\left[f_{t+1}(\tbw_{t+1}) - f_{t+1}(\tbw_{t})\right] \Bigl]_+
&\leq \left(\frac{2}{\scD} + \scD\beta\right)\E\left[||\bw_{t+1} -\bw_{t}||\right]  + \frac{\beta\delta^2\scD^2}{2}\\
&= \scO\left(\left(\frac{1}{\scD} + \scD\beta\right)\sqrt{\frac{\eta\,n}{\delta}}  + \beta\delta^2\scD^2\right)~,
\end{align*}
%
as claimed.
\end{proof}



\input{appe2}

\end{document}
