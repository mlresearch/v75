%!TEX root = main.tex

A first line of attack in exploratory data analysis is \emph{data visualization}, i.e., generating a 2-dimensional representation of data that makes \emph{clusters} of similar points visually identifiable. Standard Johnson-Lindenstrauss dimensionality reduction does not produce data visualizations. The \emph{t-SNE} heuristic of van der Maaten and Hinton, which is based on non-convex optimization, has become the \emph{de facto} standard for visualization in a wide range of applications.


This work gives a formal framework for the problem of data visualization -- finding a 2-dimensional embedding of clusterable data that correctly separates individual clusters to make them visually identifiable. We then give a rigorous analysis of the performance of t-SNE under a natural, deterministic condition on the ``ground-truth'' clusters (similar to conditions assumed in earlier analyses of clustering) in the underlying data. These are the first provable guarantees on t-SNE for constructing good data visualizations. 

We show that our deterministic condition is satisfied by considerably general probabilistic generative models for clusterable data such as mixtures of well-separated log-concave distributions. Finally, we give theoretical evidence that t-SNE provably succeeds in \emph{partially} recovering cluster structure even when the above deterministic condition is not met. 

 



 