In this paper, we proposed a new framework for studying property testing questions on Markov chains. 
There seem to be multiple avenues for future research and abundant number of open problems arising from this framework. 
We first list some questions which may be of interest here.
\begin{enumerate}
\item What is the optimal sample complexity for identity testing on symmetric Markov chains? In this paper, we show an upper bound of $\wO{\hitt{Q}\cdot\log\left(\hitt{Q}\right)+\frac{n}{\eps}}$ samples (Theorem \ref{th:symmetric_ub}). We conjecture that $\Theta\left( \frac{n}{\eps}\right)$ (same as our lower bound) is the right sample complexity for this problem and an explicit dependence on the hitting time of chain $Q$ may not be necessary. It is implicitly captured to an extent by the guarantee we get from the parameter $\eps$.
%\item What is the optimal sample complexity for identity testing on the sparse Markov chains defined in Section~\ref{sec:shuffle}? In this paper, we show an upper bound of $\widetilde{O}\left( \frac{n^{3/2}}{\epsilon^2} \right)$ (Theorem \ref{th:sparse_ub}). We conjecture that $\Theta\left( \frac{n}{\eps^2}\right)$ (same as our lower bound) is the right sample complexity for this problem.

\item As there is a natural operation of taking a convex combination of Markov chains, it is natural to ask how our spectral definition of distance 
$1-\specr{\srprod{P}{Q}}$ between two symmetric chains changes if we substitute either $P$ or $Q$ with a convex combination of $P$ and $Q$. How does the distance now relate to the original value?
%\item How is the difference parameter $\eps=1-\specr{\srprod{P}{Q}}$ between two Markov chains $P$ and $Q$ related to the difference between Markov chains $P^k$ and $Q^k$,i.e., states in Markov chains $P$ and $Q$ being observed only at intervals of size $k$? 
\item Given $\eps_2 \ge \eps_1$, and access to words from each of two chains, can we distinguish whether the two chains are $\le \eps_1$-close or $\ge \eps_2$-far? This problem, known as two-sample testing in literature, is another interesting direction using our framework.
\end{enumerate}
%More broadly, we are interested in the following research directions
%\begin{enumerate}
%\item Testing other properties of Markov chains, e.g., testing closeness of two Markov chains.
%\item Testing Hidden Markov Models.
%\end{enumerate}


%\nishanth{
%Reminders:
%\begin{enumerate}
%\item Do we want to change the title of the paper so as to sell the framework? Although we solve testing problems only, what we have is quite general and can be applied to statistical problems other than testing as well for e.g. learning). 
%
%%\item Add remark saying that we considered other seemingly more natural and simpler distance notion and mention why they don't capture the behavior well. For example, for GSR L1 between the parameter vectors isn't a good distance notion as it gives large weight to parameters which are of little significance.
%\end{enumerate}
%}